\section{Negative Result}
\label{app:neg}

\begin{theorem}%\myrm
%	\label{thm:negative}
	Let $\CAL_1, \CAL_2 \in \set{\HOp, \HO, \sessp}$.
	There is no typed, minimal encoding from $\tyl{L}_{\CAL_1}$ into $\tyl{L}_{\CAL_2^{\minussh}}$
%	$\enco{\map{\cdot}, \mapt{\cdot}, \mapa{\cdot}}: \sessp \longrightarrow \HOp^{\minussh}$.
%	that enjoys: (i) homomorphism wrt parallel; (ii) barb preservation; (iii) operational completeness.
\end{theorem}

\begin{proof}
	Assume, towards a contradiction, that such a typed encoding indeed exists. 
	Consider the $\sessp$ process
	%
	\[
		P = \breq{a}{s} \inact \Par \bacc{a}{x} \bsel{n}{l_1} \inact \Par \bacc{a}{x} \bsel{m}{l_2} \inact \qquad \text{(with $n \neq m$)}
	\]
	%
	\noi such that 
	$\Gamma; \es; \Delta \proves P \hastype \Proc$.
	From process $P$ we have: %We then have both
	%
	\begin{eqnarray}
		& & \horel{\Gamma}{\Delta}{P}{\hby{\tau}}{\Delta'}{\bsel{n}{l_1} \inact \Par \bacc{a}{x} \bsel{m}{l_2} \inact = P_1} \label{eq:nn3} \\
		& & \horel{\Gamma}{\Delta}{P}{\hby{\tau}}{\Delta'}{\bsel{m}{l_2} \inact \Par \bacc{a}{x} \bsel{n}{l_1} \inact = P_2} \label{eq:nn4}
	\end{eqnarray}
	%
	Thus, by definition of typed barb we  have:
	%
	\begin{eqnarray}
		\Gamma; \Delta' \proves P_1 \barb{n} & \land & 
		\Gamma; \Delta' \proves P_1 \nbarb{m} \label{eq:nn1} \\
		\Gamma; \Delta' \proves P_2 \barb{m} & \land & 
		\Gamma; \Delta' \proves P_2 \nbarb{n} \label{eq:nn2}
	\end{eqnarray}
	%
	Consider now the $\HOp^{\minussh}$ process $\map{P}$.
	% = 
	% \map{\breq{a}{s} \inact} \Par \map{\bacc{a}{x} \bsel{n}{l_1} \inact} \Par \map{\bacc{a}{x} \bsel{m}{l_2}}$.
	By our assumption of operational completeness 
	(\defref{def:ep}-2(a)), 
	from \eqref{eq:nn3} with \eqref{eq:nn4}
	we infer that
	there exist $\HOp^{\minussh}$ processes $S_1$ and $S_2$ such that:
	%we have both:
	\begin{eqnarray}
		& & \horel{\mapt{\Gamma}}{\mapt{\Delta}}{\map{P}}{\Hby{\stau}}{\mapt{\Delta'}}{S_1 \WB \map{P_1}} \label{eq:n1} \\
		& & \horel{\mapt{\Gamma}}{\mapt{\Delta}}{\map{P}}{\Hby{\stau}}{\mapt{\Delta'}}{S_2 \WB \map{P_2}} \label{eq:n2}
		%\map{P} & \Hby{} &  S_1 \WB \map{P_1} \\
		%s\map{P} & \Hby{} & S_2 \WB \map{P_2}
	\end{eqnarray}
	By our assumption of barb preservation, 
	from \eqref{eq:nn1} with \eqref{eq:nn2}
	we infer: 
	%
	\begin{eqnarray}
		\mapt{\Gamma}; \mapt{\Delta'} \proves \map{P_1} \Barb{n} & \land & 
		\mapt{\Gamma}; \mapt{\Delta'} \proves \map{P_1} \nBarb{m} \label{eq:n3} \\
		\mapt{\Gamma}; \mapt{\Delta'} \proves \map{P_2} \Barb{m} & \land & 
		\mapt{\Gamma}; \mapt{\Delta'} \proves \map{P_2} \nBarb{n} \label{eq:n4}
	\end{eqnarray}
	%
	By definition of $\WB$, 
	by combining~\eqref{eq:n1} with~\eqref{eq:n3}
	and~\eqref{eq:n2} with~\eqref{eq:n4}, we infer barbs for $S_1$ and $S_2$:
	\begin{eqnarray}
		\mapt{\Gamma}; \mapt{\Delta'} \proves S_1 \Barb{n} & \land & 
		\mapt{\Gamma}; \mapt{\Delta'} \proves S_1 \nBarb{m} \label{eq:n5} \\
		\mapt{\Gamma}; \mapt{\Delta'} \proves S_2 \Barb{m} & \land & 
		\mapt{\Gamma}; \mapt{\Delta'} \proves S_2 \nBarb{n} \label{eq:n6}
	\end{eqnarray}
	That is, $S_1$ and $\map{P_1}$ 
	(resp. $S_2$ and $\map{P_2}$)
	have the same barbs.
	Now, by $\tau$-inertness (\propref{lem:tau_inert}), we have both 
	\begin{eqnarray}
		& & \horel{\mapt{\Gamma}}{\mapt{\Delta}}{S_1}{\WB}{\mapt{\Delta'}}{\map{P}} \label{eq:n7} \\
		& & \horel{\mapt{\Gamma}}{\mapt{\Delta}}{S_2}{\WB}{\mapt{\Delta'}}{\map{P}} \label{eq:n8}
	\end{eqnarray}
	Combining~\eqref{eq:n7} with~\eqref{eq:n8}, by transitivity of $\WB$,
	we have 
	\begin{equation}
		\horel{\mapt{\Gamma}}{\mapt{\Delta'}}{S_1}{\WB}{\mapt{\Delta'}}{S_2} \label{eq:n9}
	\end{equation}
	In turn, from~\eqref{eq:n9}
	we infer that 
	it must be the case that:
	\begin{eqnarray*}
		\mapt{\Gamma}; \mapt{\Delta'} \proves \map{P_1} \Barb{n} & \land & 
		\mapt{\Gamma}; \mapt{\Delta'} \proves \map{P_1} \Barb{m} \label{eq:n10} \\
		\mapt{\Gamma}; \mapt{\Delta'} \proves \map{P_2} \Barb{m} & \land & 
		\mapt{\Gamma}; \mapt{\Delta'} \proves \map{P_2} \Barb{n} \label{eq:n11}
	\end{eqnarray*}
	which clearly contradict \eqref{eq:n3} and \eqref{eq:n4} above.
	\qed
\end{proof}


%\begin{theorem}\rm
%	There is no encoding $\enco{\map{\cdot}, \mapt{\cdot}, \mapa{\cdot}}: \HOp \longrightarrow \HOp^{\minussh}$
%	that enjoys operational correspondence and full abstraction.
%\end{theorem}

%\begin{proof}
%	Let $\horel{\Gamma_1}{\Delta_1}{P_1}{\not\wb}{\Delta_2}{P_2}$
%	with $P = \breq{a}{s} \inact \Par \bacc{a}{x} P_1 \Par \bacc{a}{x} P_2$ and
%	let $\Gamma; \emptyset; \Delta \proves P \hastype \Proc$.
%	Assume also a encoding
%	$\enco{\map{\cdot}, \mapt{\cdot}, \mapa{\cdot}}: \HOp \longrightarrow \HOp^{\minussh}$
%	that enjoys
%	operational correspondence and full abstraction.
%
%	From operational correspondence we get that:
%	\begin{eqnarray*}
%		P \red P_1 \Par \bacc{a}{x} P_2 &\textrm{implies}& \map{P} \red \map{P_1 \Par \bacc{a}{x} P_2}\\
%		P \red P_2 \Par \bacc{a}{x} P_1 &\textrm{implies}& \map{P} \red \map{P_2 \Par \bacc{a}{x} P_1}
%	\end{eqnarray*}
%
%	From the fact that
%	$\horel{\Gamma_1}{\Delta_1}{P_1}{\not\wb}{\Delta_2}{P_2}$
%	we can derive that
%%
%	\[
%		\horel{\Gamma_1'}{\Delta_1'}{P_1 \Par \bacc{a}{x} P_2}{\not\wb}{\Delta_2'}{P_2 \Par \bacc{a}{x} P_1}
%	\]
%%
%	From Corollary~\ref{cor:tau_inert} we know that
%%
%	\begin{eqnarray*}
%		\horel{\mapt{\Gamma}}{\mapt{\Delta}}{\map{P}}{\wb}{\mapt{\Delta_1'}}{\map{P_1 \Par \bacc{a}{x} P_2}}\\
%		\horel{\mapt{\Gamma}}{\mapt{\Delta}}{\map{P}}{\wb}{\mapt{\Delta_2'}}{\map{P_2 \Par \bacc{a}{x} P_1}}
%	\end{eqnarray*}
%%
%	\noi thus
%	\[
%		\horel{\mapt{\Gamma}}{\mapt{\Delta_1'}}{\map{P_1 \Par \bacc{a}{x} P_2}}{\wb}{\mapt{\Delta_2'}}{\map{P_2 \Par \bacc{a}{x} P_1}}
%	\]
%%
%	From here we conclude that the full abstraction property does not hold,
%	which is a contradiction.
%	\qed
%%	so there is no mapping $\map{\cdot}: \pHO \longrightarrow \spi$ that enjoys
%%	the operational correspondence and full abstraction properties.
%\end{proof}
