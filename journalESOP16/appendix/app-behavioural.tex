% !TEX root = ../journal16kpy.tex


\section{Behavioural Semantics}
\label{app:behavioural}

We report auxiliary definitions and results from~\cite{characteristic_bis,KouzapasPY17}.

\subsection{Labelled Transition System for Processes}
\label{ss:lts}

We define the interaction of processes with their environment using action labels $\ell$:
%
\begin{center}
	\begin{tabular}{l}
		$\ell
			\bnfis  \tau 
			\bnfbar	\news{\widetilde{m}} \bactout{n}{V}
			\bnfbar	\bactinp{n}{V} 
			\bnfbar	\bactsel{n}{l} 
			\bnfbar	\bactbra{n}{l}$
	\end{tabular}
\end{center}
%
\noi 
Label $\tau$ defines internal actions.
Action
$\news{\widetilde{m}} \bactout{n}{V}$
denotes the sending of value $V$
over channel $n$ with a possible empty set of restricted names
$\widetilde{m}$ 
(we may write $\bactout{n}{V}$ when $\widetilde{m}$ is empty).
Dually, the action for value reception is 
$\bactinp{n}{V}$.
Actions for select and branch on
a label~$l$ are denoted $\bactsel{n}{l}$ and $\bactbra{n}{l}$, respectively.
We write $\fn{\ell}$ and $\bn{\ell}$ to denote the
sets of free/bound names in $\ell$, respectively.
%and set $\mathsf{n}(\ell)=\bn{\ell}\cup \fn{\ell}$. 
Given $\ell \neq \tau$, we 
say $\ell$ is a \emph{visible action}; we
write $\subj{\ell}$
to denote its \emph{subject}.
\newc{This way, we have:  
$\subj{\news{\widetilde{m}} \bactout{n}{V}} = 
\subj{\bactinp{n}{V}} = 
\subj{\bactsel{n}{l}} = 
\subj{\bactbra{n}{l}} = n$}.

%%%%%%%%%%%%%%%%%%%% LTS Figure %%%%%%%%%%%%%%%%%%%%%%%%%%%%%%%%%%%%%
\begin{figure}[h!]
	\begin{mathpar}
		\inferrule[\ltsrule{App}]{
		}{
			\appl{(\abs{x}{P})}{V} \by{\tau} P \subst{V}{x}
		}
		\and
		\inferrule[\ltsrule{Snd}]{
		}{
			\bout{n}{V} P \by{\bactout{n}{V}} P
		}
		\and
		\inferrule[\ltsrule{Rv}]{
		}{
			\binp{n}{x} P \by{\bactinp{n}{V}} P\subst{V}{x}
		}
		\and
		\inferrule[\ltsrule{Sel}]{
		}{
			\bsel{s}{l}{P} \by{\bactsel{s}{l}} P
		}
		\and
		\inferrule[\ltsrule{Bra}]{j\in I
		}{
			\bbra{s}{l_i:P_i}_{i \in I} \by{\bactbra{s}{l_j}} P_j 
		}
		\and
		\inferrule[\ltsrule{Alpha}]{
			P \scong_\alpha Q
			\and
			Q\by{\ell} P'
		}{
			P \by{\ell} P'
		}
		\and
		\inferrule[\ltsrule{Res}]{
			P \by{\ell} P'
			\and
			n \notin \fn{\ell}
		}{
			\news{n} P \by{\ell} \news{n} P'
		}
		\and
		\inferrule[\ltsrule{New}]{
			P \by{\news{\widetilde{m}} \bactout{n}{V}} P'
			\and
			m_1 \in \fn{V}
		}{
			\news{m_1} P \by{\news{m_1\cat\widetilde{m}} \bactout{n}{V}} P'
		}
		\and
		\inferrule[\ltsrule{Par${}_L$}]{
			P \by{\ell} P'
			\and
			\bn{\ell} \cap \fn{Q} = \es
		}{
			P \Par Q \by{\ell} P' \Par Q
		}
		\and
		\inferrule[\ltsrule{Tau}]{
			P \by{\ell_1} P'
			\and
			Q \by{\ell_2} Q'
			\and
			\ell_1 \asymp \ell_2
		}{
			P \Par Q \by{\tau} \newsp{\bn{\ell_1} \cup \bn{\ell_2}}{P' \Par Q'}
		}
		\and
		\inferrule[\ltsrule{Rec}]{
			P\subst{\recp{X}{P}}{\rvar{X}} \by{\ell} P'
		}{
			\recp{X}{P}  \by{\ell} P'
		}
	\end{mathpar}
	%
	\caption{The Untyped LTS for \HOp processes. We omit Rule $\ltsrule{Par${}_R$}$.  \label{fig:untyped_LTS}}
	%
\end{figure}

%%%%%%%%%%%%%%%%%%%% End LTS Figure %%%%%%%%%%%%%%%%%%%%%%%%%%%%%%%%%

\emph{Dual actions}
occur on subjects that are dual between them and carry the same
object; thus, output is dual to input and 
selection is dual to branching.

\begin{definition}[Dual Actions]\label{def:dualact}
We define duality on actions
as the least symmetric relation $\asymp$ on action labels that satisfies:
\[
	\bactsel{n}{l} \asymp \bactbra{\dual{n}}{l}
	\qquad \qquad 
	\news{\widetilde{m}} \bactout{n}{V} \asymp \bactinp{\dual{n}}{V}
\]
\end{definition}

The (early) labelled transition system
(LTS) %LTS
fpr \emph{untyped} processes
is given in
\figref{fig:untyped_LTS}. 
We write $P_1 \by{\ell} P_2$ with the usual meaning.
The rules are standard~\cite{KYHH2015,KY2015}; we comment on some of them.
A process with an output prefix can
interact with the environment with an output action that carries a value
$V$ (Rule~$\ltsrule{Snd}$).  Dually, in Rule~$\ltsrule{Rv}$ a
receiver process can observe an input of an arbitrary value $V$.
Select and branch processes observe the select and branch
actions in Rules~$\ltsrule{Sel}$ and $\ltsrule{Bra}$, respectively.
Rule $\ltsrule{Res}$ 
%closes the LTS under restriction 
\newc{enables an observable action from a process with an outermost restriction, provided that 
 the restricted name does not occur free in the action}. 
If a restricted name occurs free in
the carried value of an output action,
the process performs scope opening (Rule~$\ltsrule{New}$).  
Rule~$\ltsrule{Rec}$ handles recursion unfolding.
Rule~$\ltsrule{Tau}$ 
states that two parallel processes which perform
dual actions can synchronise by an internal transition.
Rules~$\ltsrule{Par${}_L$}$/$\ltsrule{Par${}_R$}$ 
and $\ltsrule{Alpha}$ 
%close the LTS under parallel composition and $\alpha$-renaming. 
\newc{define standard treatments for actions under parallel composition and $\alpha$-renaming.}

\subsection{Environmental Labelled Transition System}
\label{ss:elts}
Our typed LTS is obtained by coupling the untyped LTS given before with a labelled transition relation 
on typing environments, given in \figref{fig:envLTS}. 
Building upon the reduction relation for session environments in \defref{def:ses_red},
such a relation
is defined on triples of environments by 
extending the LTSs
in \cite{KYHH2015,KY2015}; it is 
denoted
%
\[
	(\Gamma_1, \Lambda_1, \Delta_1) \by{\ell} (\Gamma_2, \Lambda_2, \Delta_2)
\]
%
\newc{Recall that  $\Gamma$ admits  weakening. 
Using this principle (not valid for $\Lambda$ and $\Delta$), we have  
$
	(\Gamma', \Lambda_1, \Delta_1) \hby{\ell} (\Gamma', \Lambda_2, \Delta_2)
$
whenever 
$
	(\Gamma, \Lambda_1, \Delta_1) \hby{\ell} (\Gamma', \Lambda_2, \Delta_2)
$.}
%
\paragraph{Input Actions} 
These actions
are defined by 
Rules~$\eltsrule{SRv}$ and $\eltsrule{ShRv}$.
In Rule~$\eltsrule{SRv}$
the type of value $V$
and the type of the object associated to the session type on $s$ 
should coincide. 
The resulting type tuple must contain the environments 
associated to $V$. 
The
dual endpoint $\dual{s}$ cannot be
present in the session environment: if it were present
the only possible communication would be the interaction
between the two endpoints (cf. Rule~$\eltsrule{Tau}$).
Following similar principles, 
Rule~$\eltsrule{ShRv}$ defines input actions for shared names.

\paragraph{Output Actions} 
These actions
are defined by Rules~$\eltsrule{SSnd}$
and $\eltsrule{ShSnd}$.  
Rule~$\eltsrule{SSnd}$ states the conditions for observing action
$\news{\widetilde{m}} \bactout{s}{V}$ on a type tuple 
$(\Gamma, \Lambda, \Delta\cdot \AT{s}{S})$. 
The session environment $\Delta \,\cat\, \AT{s}{S}$ 
should include the session environment of the sent value $V$ \newc{(denoted $\Delta'$ in the rule)}, 
{\em excluding} the session environments of names $m_j$ 
in $\widetilde{m}$ which restrict the scope of value $V$ \newc{(denoted $\Delta_j$ in the rule)}. 
Analogously, the linear variable environment 
$\Lambda'$ of $V$ should be included in $\Lambda$. 
\newc{The rule defines the scope extrusion of session names in $\widetilde{m}$; consequently, 
environments associated to 
their dual endpoints  (denoted $\Delta'_j$ in the rule) appear in
the resulting session environment}. Similarly for shared 
names in $\widetilde{m}$ that are extruded.  
All free values used for typing $V$ \newc{(denoted $\Lambda'$ and $\Delta'$ in the rule)} are subtracted from the
resulting type tuple. The prefix of session $s$ is consumed
by the action.
Rule $\eltsrule{ShSnd}$ follows similar ideas for output actions on shared names:
the name must be typed with $\chtype{U}$; 
conditions on value $V$ are identical to those on Rule~$\eltsrule{SSnd}$.

\paragraph{Other Actions}
Rules $\eltsrule{Sel}$ and $\eltsrule{Bra}$ describe actions for
select and branch.
%Both
%rules require the absence of the dual endpoint from the session
%environment.%, and the presence of the action labels in the type.
Rule~$\eltsrule{Tau}$ defines
internal transitions: 
it reduces the session environment (cf. \defref{def:ses_red}) or keeps it 
unchanged.

\smallskip

\newc{We illustrate Rule~$\eltsrule{SSnd}$ by means of an example:}

\begin{example}
	\newc{Consider environment tuple
	$
		(\Gamma;\, \es;\, s: \btout{\lhot{(\btout{S} \tinact)}} \tinact \cat s': S)
	$
	and typed value $V= \abs{x} \bout{x}{s'} \binp{m}{z} \inact$ with 
	\[
		\Gamma; \es; s': S \cat m: \btinp{\tinact} \tinact \proves V \, \hastype \, \lhot{(\btout{S} \tinact)}
	\]
%
%	\noi Let 
%	$
%		\Delta'_1=\set{\overline{m}: \btout{\tinact} \tinact}
%	$
%	and $U=\btout{\lhot{\btout{S} \tinact}} \tinact$.
	Then, by Rule $\eltsrule{SSnd}$, we can derive:
%
	\[
		(\Gamma; \es; s: \btout{\lhot{(\btout{S} \tinact)}} \tinact \cat s': S) \by{\news{m} \bactout{s}{V}} (\Gamma; \es; s: \tinact \cat \dual{m}: \btout{\tinact} \tinact)
	\]
%	\qed
Observe how the protocol along $s$ is partially consumed; also, the resulting session environment is extended with 
  $\dual{m}$, the dual endpoint of the extruded name $m$.}
\end{example}


%\begin{notation}
%Given a value $V$ of type $U$, 
Recall that 
we sometimes annotate the output action 
$\news{\widetilde{m}} \bactout{n}{V}$
with the type of $V$; this is written
as $\news{\widetilde{m}} \bactout{n}{\AT{V}{U}}$
(cf. Remark~\ref{r:anntypes}).

\smallskip


%\end{notation}


%\smallskip

%%%%%%%%%%%%%%%%%%%% Environment LTS Figure %%%%%%%%%%%%%%%%%%%%%%%%%
\begin{figure}[t!]
\begin{mathpar}
	\inferrule[\eltsrule{SRv}]{
		\dual{s} \notin \dom{\Delta}
		\and
		\Gamma; \Lambda'; \Delta' \proves V \hastype U
	}{
		(\Gamma; \Lambda; \Delta \cat s: \btinp{U} S) \by{\bactinp{s}{V}} (\Gamma; \Lambda\cat\Lambda'; \Delta\cat\Delta' \cat s: S)
	}
	\and
	\inferrule[\eltsrule{ShRv}]{
		\Gamma; \es; \es \proves a \hastype \chtype{U}
		\and
		\Gamma; \Lambda'; \Delta' \proves V \hastype U
	}{
		(\Gamma; \Lambda; \Delta) \by{\bactinp{a}{{V}}} (\Gamma; \Lambda\cat\Lambda'; \Delta\cat\Delta')
	}
	\and
	\inferrule[\eltsrule{SSnd}]{
		\begin{array}{l}
			\Gamma \cat \Gamma'; \Lambda'; \Delta' \proves V \hastype U
			\and
			\Gamma'; \es; \Delta_j \proves m_j  \hastype U_j
			\and
			\dual{s} \notin \dom{\Delta}
			\\
			\Delta'\backslash (\cup_j \Delta_j) \subseteq (\Delta \cat s: S)
			\and
			\Gamma'; \es; \Delta_j' \proves \dual{m}_j  \hastype U_j'
			\and
			\Lambda' \subseteq \Lambda
		\end{array}
	}{
		(\Gamma; \Lambda; \Delta \cat s: \btout{U} S)
		\by{\news{\widetilde{m}} \bactout{s}{V}}
		(\Gamma \cat \Gamma'; \Lambda\backslash\Lambda'; (\Delta \cat s: S \cat \cup_j \Delta_j') \backslash \Delta')
	}
	\and
	\inferrule[\eltsrule{ShSnd}]{
		\begin{array}{l}
			\Gamma \cat \Gamma' ; \Lambda'; \Delta' \proves V \hastype U
			\and
			\Gamma'; \es; \Delta_j \proves m_j \hastype U_j
			\and
			\Gamma ; \es ; \es \proves a \hastype \chtype{U}
			\\
			\Delta'\backslash (\cup_j \Delta_j) \subseteq \Delta
			\and
			\Gamma'; \es; \Delta_j' \proves \dual{m}_j\hastype U_j'
			\and
			\Lambda' \subseteq \Lambda
		\end{array}
	}{
		(\Gamma ; \Lambda; \Delta) \by{\news{\widetilde{m}}
		\bactout{a}{V}}
		(\Gamma \cat \Gamma'; \Lambda\backslash\Lambda'; (\Delta \cat \cup_j \Delta_j') \backslash \Delta')
	}
	\and
	\inferrule[\eltsrule{Sel}]{
		\dual{s} \notin \dom{\Delta}
		\and
		j \in I
	}{
		(\Gamma; \Lambda; \Delta \cat s: \btsel{l_i: S_i}_{i \in I}) \by{\bactsel{s}{l_j}} (\Gamma; \Lambda; \Delta \cat s:S_j)
	}
	\and
	\inferrule[\eltsrule{Bra}]{
		\dual{s} \notin \dom{\Delta} \quad j \in I
	}{
		(\Gamma; \Lambda; \Delta \cat s: \btbra{l_i: T_i}_{i \in I}) \by{\bactbra{s}{l_j}} (\Gamma; \Lambda; \Delta \cat s:S_j)
	}
	\and
	\inferrule[\eltsrule{Tau}]{
		\Delta_1 \red \Delta_2 \vee \Delta_1 = \Delta_2
	}{
		(\Gamma; \Lambda; \Delta_1) \by{\tau} (\Gamma; \Lambda; \Delta_2)
	}
\end{mathpar}
%
\caption{Labelled Transition System for Typed Environments. 
\label{fig:envLTS}}
%
\end{figure}


%%%%%%%%%%%%%%%%%%%% End Environment LTS Figure %%%%%%%%%%%%%%%%%%%%%%



\noi
The typed LTS  combines
the LTSs in \figref{fig:untyped_LTS}
and \figref{fig:envLTS}. 

\begin{definition}[Typed Transition System]
	\label{d:tlts}
	A {\em typed transition relation} is a typed relation
	$\horel{\Gamma}{\Delta_1}{P_1}{\by{\ell}}{\Delta_2}{P_2}$
	where:
%
	\begin{enumerate}
		\item
				$P_1 \by{\ell} P_2$ and 
		\item
				$(\Gamma, \emptyset, \Delta_1) \by{\ell} (\Gamma, \emptyset, \Delta_2)$ 
				with $\Gamma; \emptyset; \Delta_i \proves P_i \hastype \Proc$ ($i=1,2$).
%				\dk{We sometimes annotated the output action with
%				the type of value $V$ as in $\widetilde{m} \bactout{n}{V: U}$.}
	\end{enumerate}
%
	We 
	%extend to $\By{}$ and $\By{\hat{\ell}}$  where we 
	write  $\By{}$ for the reflexive and transitive closure of $\by{}$,
	$\By{\ell}$ for the transitions $\By{}\by{\ell}\By{}$, and $\By{\hat{\ell}}$
	for $\By{\ell}$ if $\ell\not = \tau$ otherwise $\By{}$.
\end{definition}

\newc{A typed transition relation requires type judgements with an empty $\Lambda$, i.e., an empty environment 
for linear higher-order types.
Notice that for open process terms (i.e., with free variables), 
we can always apply Rule~$\trule{EProm}$ (cf. \figref{fig:typerulesmy}) and obtain an empty $\Lambda$. 
We will be working with closed process terms, i.e., processes without free  variables.
}


\subsection{Characteristic Values and the Refined LTS}
\label{ss:reflts}

\noi 
We formalise the ideas given in \secref{sec:overview}, concerning 
characteristic processes/values and the refined LTS.
%the introduction.
We first define characteristic processes/values:

\begin{definition}[Characteristic Process and Values]
	\label{def:char}
	Let $u$ and $U$ be a name and a type, respectively.
	\newc{The {\em characteristic process} of $U$ (along $u$), denoted $\mapchar{U}{u}$, and 
	the {\em characteristic value} of $U$, denoted $\omapchar{U}$, are defined in \figref{fig:char}.}
\end{definition}


%%%%%%%%%%%%%%%%%%%%%%%% End Characteristic Process Figure %%%%%%%%%%%%%%%%%%%%%


\noi We can verify that characteristic processes/values  
do inhabit their associated type.

\begin{proposition}[Characteristic Processes/Values Inhabit Their Types]\label{p:inhabit}
	\begin{enumerate}
		\item	Let $U$ be a channel type.
				Then, \newc{for some $\Gamma, \Delta$,  we have} $\Gamma; \es; \Delta \proves \omapchar{U} \hastype U$.
		\item	
				Let $S$ be a session type.
				Then, \newc{for some $\Gamma, \Delta$, we have} $\Gamma; \es; \Delta \cat s: S \proves \mapchar{S}{s} \hastype \Proc$.
		\item	
				Let $U$ be a channel type.
				Then, \newc{for some $\Gamma, \Delta$, we have} $\Gamma \cat a: U; \es; \Delta \proves \mapchar{U}{a} \hastype \Proc$.

	\end{enumerate}
\end{proposition}

%\begin{proof}[Sketch]
%	The proof is done by induction on the syntax of types.
%	See \propref{app:characteristic_inhabit} (\mypageref{app:characteristic_inhabit}) in the Appendix for details.
%	\qed
%\end{proof}

 


 

\begin{definition}[Trigger Value]\label{d:trigger}
Given a fresh name $t$, the \emph{trigger value} on $t$ is defined as the abstraction
$\abs{{x}}{\binp{t}{y} (\appl{y}{{x}})}$.
\end{definition}




%the introduction,
We define the
\emph{refined} typed LTS
by considering a transition rule for input in which admitted values are
trigger or characteristic values or names:

%\dk{(assume extension of the structural
%congruence to acommodate values: i) $\abs{x}{P} \scong \abs{x}{Q}$ if
%$P \scong Q$) and ii) $n \scong m$ if $n = n$)}: 

%\begin{definition}[Refined Typed Labelled Transition System]
%	\label{def:rlts}
%	We define the environment transition rule for input actions 
%	using the input rules in \figref{fig:envLTS}:
%	\begin{mathpar}
%		\inferrule[\eltsrule{RRcv}]{
%			(\Gamma_1; \Lambda_1; \Delta_1) \by{\bactinp{n}{V}} (\Gamma_2; \Lambda_2; \Delta_2)
%			\and
%			V = m 
%			\vee  V \scong \omapchar{U}
%			\vee V  \scong \abs{{x}}{\binp{t}{y} (\appl{y}{{x}})}
%			\textrm{ {\small with $t$ fresh}} 
%		}{
%			(\Gamma_1; \Lambda_1; \Delta_1) \hby{\bactinp{n}{V}} (\Gamma_2; \Lambda_2; \Delta_2)
%		}
%	\end{mathpar}
%	\noi Rule $\eltsrule{RRcv}$ is defined on top
%	of rules $\eltsrule{SRv}$ and $\eltsrule{ShRv}$
%	in \figref{fig:envLTS}.
%	We  use the non-input rules in \figref{fig:envLTS}
%	together with rule $\eltsrule{RRcv}$
%	to define 
%	$$\horel{\Gamma}{\Delta_1}{P_1}{\hby{\ell}}{\Delta_2}{P_2}$$
%	as in \defref{d:tlts}.
%\end{definition}

\begin{definition}[Refined Typed Labelled Transition System]
	\label{def:rlts}
	\newc{The 
	refined typed labelled transition relation on typing environments 
	$$
	(\Gamma_1; \Lambda_1; \Delta_1) \hby{\ell} (\Gamma_2; \Lambda_2; \Delta_2)
	$$
	is 
	defined on top of the rules
	in \figref{fig:envLTS} using the following rules:
	\begin{mathpar}
		\inferrule[\eltsrule{Tr}] {
			(\Gamma_1; \Lambda_1; \Delta_1) \by{\ell} (\Gamma_2; \Lambda_2; \Delta_2)
			\and
			\ell \not= \bactinp{n}{V}
		}{
			(\Gamma_1; \Lambda_1; \Delta_1) \hby{\ell} (\Gamma_2; \Lambda_2; \Delta_2)
		}
		\\
		\inferrule[\eltsrule{RRcv}]{
			(\Gamma_1; \Lambda_1; \Delta_1) \by{\bactinp{n}{V}} (\Gamma_2; \Lambda_2; \Delta_2)
			\and
			V = m 
			\vee  V \scong \omapchar{U}
			\vee V  \scong \abs{{x}}{\binp{t}{y} (\appl{y}{{x}})}
			\textrm{ {\small $t$ fresh}} 
		}{
			(\Gamma_1; \Lambda_1; \Delta_1) \hby{\bactinp{n}{V}} (\Gamma_2; \Lambda_2; \Delta_2)
		}
	\end{mathpar}
	Then, the refined typed labelled transition system 
			$$\horel{\Gamma}{\Delta_1}{P_1}{\hby{\ell}}{\Delta_2}{P_2}$$
	is given as in \defref{d:tlts}, replacing the requirement 
	$$(\Gamma, \emptyset, \Delta_1) \by{\ell} (\Gamma, \emptyset, \Delta_2)$$
	with 
	$
	(\Gamma_1; \Lambda_1; \Delta_1) \hby{\ell} (\Gamma_2; \Lambda_2; \Delta_2)
	$, as just defined.
	Following \defref{d:tlts}, 
	we 
	write  $\Hby{}$ for the reflexive and transitive closure of $\hby{\tau}$,
	$\Hby{\ell}$ for the transitions $\Hby{}\hby{\ell}\Hby{}$, and $\Hby{\hat{\ell}}$
	for $\Hby{\ell}$ if $\ell\not = \tau$ otherwise $\Hby{}$.}
	\end{definition}


 Notice that
the (refined) transition
$\horel{\Gamma}{\Delta_1}{P_1}{\hby{\ell}}{\Delta_2}{P_2}$  implies  
the (ordinary) transition
$\horel{\Gamma}{\Delta_1}{P_1}{\by{\,\ell\,}}{\Delta_2}{P_2}$.

%\begin{notation}\label{not:outtype}
%We sometimes write  
%$\hby{\news{\widetilde{m}} \bactout{n}{\AT{V}{U}}}$
%when the type of $V$ is~$U$.
%\end{notation}


\subsection{More on Deterministic Transitions and Up-to Techniques}
\label{ss:deter}

As hinted at earlier, internal transitions associated to session interactions or  
$\beta$-reductions are deterministic.  To define an auxiliary proof technique that exploits determinacy we require some auxiliary definitions.
		
\begin{definition}[Deterministic Transitions]
\label{def:dettrans}
	Suppose $\Gamma; \es; \Delta \proves P \hastype \Proc$ \newc{with balanced~$\Delta$}.
	Transition $\horel{\Gamma}{\Delta}{P}{\hby{\tau}}{\Delta'}{P'}$ is called:
%
	\begin{enumerate}[$-$]
		\item 
				a {\em \sesstran} whenever transition $P \by{\tau} P'$ 
				is derived using Rule~$\ltsrule{Tau}$ 
				(where $\subj{\ell_1}$ and $\subj{\ell_2}$ in the premise are dual endpoints), 
				possibly followed by uses of Rules~$\ltsrule{Alpha}$, $\ltsrule{Res}$, $\ltsrule{Rec}$, or $\ltsrule{Par${}_L$}/
				\ltsrule{Par${}_R$}$ (cf. \figref{fig:untyped_LTS}).

	%	We write $\horel{\Gamma}{\Delta}{P}{\hby{\stau}}{\Delta'}{P'}$ to denote a \sesstran.
		
		\item
				a {\em \betatran} whenever transition $P \by{\tau} P'$
				is derived using Rule $\ltsrule{App}$,
				possibly followed by uses of Rules~$\ltsrule{Alpha}$, $\ltsrule{Res}$, $\ltsrule{Rec}$, or $\ltsrule{Par${}_L$}/
				\ltsrule{Par${}_R$}$ (cf. \figref{fig:untyped_LTS}).

		%		We write $\horel{\Gamma}{\Delta}{P}{\hby{\btau}}{\Delta'}{P'}$ to denote a $\beta$-transition.

%		\item	Also, $\horel{\Gamma}{\Delta}{P}{\hby{\dtau}}{\Delta'}{P'}$ denotes				either a \sesstran or a \betatran.
	\end{enumerate}
%
%	We write $\Hby{\dtau}$ to denote a (possibly empty) sequence of deterministic steps $\hby{\dtau}$.
\end{definition}

\begin{notation} We use the following notations:
	\begin{enumerate}[$-$]
		\item 	 $\horel{\Gamma}{\Delta}{P}{\hby{\stau}}{\Delta'}{P'}$  denotes a \sesstran.
		
		\item  $\horel{\Gamma}{\Delta}{P}{\hby{\btau}}{\Delta'}{P'}$   denotes a $\beta$-transition.

		\item	 $\horel{\Gamma}{\Delta}{P}{\hby{\dtau}}{\Delta'}{P'}$   denotes
				either a \sesstran or a \betatran.
\item  We write $\Hby{\dtau}$ to denote a (possibly empty) sequence of deterministic steps $\hby{\dtau}$.
	\end{enumerate}
\end{notation}



Using the above properties, we can state the following up-to technique.
Recall that the higher-order trigger 
$\htrigger{t}{V}$
has been defined in 
\eqref{eqb:4} (Page~\pageref{eqb:4}).
%We write $\Hby{\dtau}$ to denote a (possibly empty) sequence of deterministic steps $\hby{\dtau}$.

\begin{lemma}[Up-to Deterministic Transition]
	\label{lem:up_to_deterministic_transition}
	Let $\horel{\Gamma}{\Delta_1}{P_1}{\ \Re\ }{\Delta_2}{Q_1}$ such
	that if whenever:
%
	\begin{enumerate}[1.]
		\item	$\forall \news{\widetilde{m_1}} \bactout{n}{V_1}$ such that
			$
				\horel{\Gamma}{\Delta_1}{P_1}{\hby{\news{\widetilde{m_1}} \bactout{n}{V_1}}}{\Delta_3}{P_3}
			$
			implies that $\exists Q_2, V_2$ such that
			$
				\horel{\Gamma}{\Delta_2}{Q_1}{\Hby{\news{\widetilde{m_2}} \bactout{n}{V_2}}}{\Delta_2'}{Q_2}
			$
			and
			$
				\horel{\Gamma}{\Delta_3}{P_3}{\Hby{\dtau}}{\Delta_1'}{P_2}
			$
			and for 
			a
			fresh name $t$ and $\Delta_1'', \Delta_2''$:\\
			\[
				\horel{\Gamma}{\Delta_1''}{\newsp{\widetilde{m_1}}{P_2 \Par \htrigger{t}{V_1}}}
				{\ \Re\ }
				{\Delta_2''}{}{\newsp{\widetilde{m_2}}{Q_2 \Par \htrigger{t}{V_2}}}
%				\qquad
%				\text{(for some $\Delta_1'', \Delta_2''$)}
			\]
%
		\item	$\forall \ell \not= \news{\widetilde{m}} \bactout{n}{V}$ such that:
			$
				\horel{\Gamma}{\Delta_1}{P_1}{\hby{\ell}}{\Delta_3}{P_3}
			$
			implies that $\exists Q_2$   such that \\
			$
				\horel{\Gamma}{\Delta_1}{Q_1}{\hat{\Hby{\ell}}}{\Delta_2'}{Q_2}
			$
			and
			$
				\horel{\Gamma}{\Delta_3}{P_3}{\Hby{\dtau}}{\Delta_1'}{P_2}
			$
			and
			$\horel{\Gamma}{\Delta_1'}{P_2}{\ \Re\ }{\Delta_2'}{Q_2}$.

		\item	The symmetric cases of 1 and 2.
	\end{enumerate}
	Then $\Re\ \subseteq\ \hwb$.
\end{lemma}

\begin{proof}[Proof (Sketch)]
	The proof proceeds by showing that the relation
	\[
		\Re^{\Hby{\dtau}} = \set{ (P_2,Q_1) \setbar \horel{\Gamma}{\Delta_1}{P_1}{\ \Re\ }{\Delta_2'}{Q_1},~~
		\horel{\Gamma}{\Delta_1}{P_1}{\Hby{\dtau}}{\Delta_1'}{P_2} }
%		\Re^{\Hby{\dtau}} = \set{ \horel{\Gamma}{\Delta_1'}{P_2}{,}{\Delta_2'}{Q_1} \setbar \horel{\Gamma}{\Delta_1}{P_1}{\ \Re\ }{\Delta_2'}{Q_1},~~
%		\horel{\Gamma}{\Delta_1}{P_1}{\Hby{\dtau}}{\Delta_1'}{P_2} }
	\]
	 is a higher-order bisimulation, which requires the use of \propref{lem:tau_inert}.
%	\qed
\end{proof}

