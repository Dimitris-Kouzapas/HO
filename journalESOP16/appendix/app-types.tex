% !TEX root = ../journal16kpy.tex

In this appendix we formally 
%define 
%\emph{type equivalence} and 
%\emph{duality}. 
%We also 
present and describe our typing rules, given in~\figref{fig:typerulesmy}.

%\smallskip
%
%\subsection{Type Equivalence and Duality}
%\label{app:duality}
%
%\begin{definition}[Type Equivalence]
%\label{def:iso}
%Let $\mathsf{ST}$ a set of closed session types. 
%Two types $S$ and $S'$ are said to be {\em isomorphic} if a pair $(S,S')$ is 
%in the largest fixed point of the monotone function
%$F:\mathcal{P}(\mathsf{ST}\times \mathsf{ST}) \to 
%\mathcal{P}(\mathsf{ST}\times \mathsf{ST})$ defined by:
%
%\hspace{-0.5cm}\begin{tabular}{rcl}
%$F(\Re)$ &$\!\!=\!\!$&	$\set{(\tinact, \tinact)}$\\
%         &$\!\!\cup\!\!$&	$\set{(\btout{U_1} S_1, \btout{U_2} S_2)
%\bnfbar (S_1, S_2),(U_1, U_2)\in \Re}$\\ 
%       &$\!\!\cup\!\!$&	$\set{(\btinp{U_1} S_1, \btinp{U_2} S_2)
%\bnfbar(S_1, S_2),(U_1, U_2)\in \Re}$\\ 
%	&$\!\!\cup\!\!$&	$\set{(\btbra{l_i: S_i}_{i \in I} \,,\, \btbra{l_i: S_i'}_{i \in I}) \bnfbar \forall i\in I. (S_i, S_i')\in \Re}$\\
%	&$\!\!\cup\!\!$&	$\set{(\btsel{l_i: S_i}_{i \in I}\,,\, \btsel{l_i: S_i'}_{i \in I}) \bnfbar \forall i\in I. (S_i, S_i')\in \Re}$\\
%	&$\!\!\cup\!\!$&	$\set{(\trec{t}{S}, S')
%\bnfbar (S\subst{\trec{t}{S}}{\vart{t}},S')\in \Re}$\\
%	&$\!\!\cup\!\!$&	$\set{(S,\trec{t}{S'})
%\bnfbar (S,S'\subst{\trec{t}{S'}}{\vart{t}})\in \Re}$
%\end{tabular}
%	
%\noindent
%Standard arguments ensure that $F$ is monotone, thus the greatest fixed point
%of $F$ exists. We write $S_1 \sim S_2$ if  $(S_1,S_2)\in \Re$. 
%\end{definition}
%
%\smallskip 
%
%\begin{definition}[Duality]
%\label{def:dual}
%Let $\mathsf{ST}$ a set of closed session types. 
%Two types $S$ and $S'$ are said to be {\em dual} if a pair $(S,S')$ is 
%in the largest fixed point of the monotone function
%$F:\mathcal{P}(\mathsf{ST}\times \mathsf{ST}) \to 
%\mathcal{P}(\mathsf{ST}\times \mathsf{ST})$ defined by:\\[1mm]
%\begin{tabular}{rcl}
%$F(\Re)$ &$\!\!=\!\!$&	$\set{(\tinact, \tinact)}$\\
%         &$\!\!\cup\!\!$&	$\set{(\btout{U_1} S_1, \btinp{U_2} S_2)
%\bnfbar(S_1, S_2)\in \Re, \  U_1 \sim U_2 }$\\ 
%       &$\!\!\cup\!\!$&	$\set{(\btinp{U_1} S_1, \btout{U_2} S_2)
%\bnfbar(S_1, S_2)\in \Re, \ U_1 \sim U_2}$\\ 
%	&$\!\!\cup\!\!$&	$\set{(\btsel{l_i: S_i}_{i \in I} \,,\, \btbra{l_i: S_i'}_{i \in I}) \bnfbar \forall i\in I. (S_i, S_i')\in \Re}$\\
%	&$\!\!\cup\!\!$&	$\set{(\btbra{l_i: S_i}_{i \in I}\,,\, \btsel{l_i: S_i'}_{i \in I}) \bnfbar \forall i\in I. (S_i, S_i')\in \Re}$\\
%	&$\!\!\cup\!\!$&	$\set{(\trec{t}{S}, S')
%\bnfbar (S\subst{\trec{t}{S}}{\vart{t}},S')\in \Re}$\\
%	&$\!\!\cup\!\!$&	$\set{(S,\trec{t}{S'})
%\bnfbar (S,S'\subst{\trec{t}{S'}}{\vart{t}})\in \Re}$\\[1mm]
%\end{tabular}
%\noindent
%%where $U_1 \sim U_2$ means $U_1$ is type equivalent to $U_2$ \cite{yoshida.vasconcelos:language-primitives}.
%Standard arguments ensure that $F$ is monotone, thus the greatest fixed point
%of $F$ exists. We write $S_1 \dualof S_2$ if  $(S_1,S_2)\in \Re$. 
%\end{definition}

%\smallskip 

%% !TEX root = ../journal16kpy.tex


\begin{figure}[t]
\[
	\begin{array}{c}
		\trule{Sess}~~\Gamma; \emptyset; \set{u:S} \proves u \hastype S 
		\qquad
		\trule{Sh}~~\Gamma \cat u : U; \emptyset; \emptyset \proves u \hastype U
		\qquad
		\trule{LVar}~~\Gamma; \set{x: \lhot{C}}; \emptyset \proves x \hastype \lhot{C}
		\\[4mm]

		\trule{Prom}~~\tree{
			\Gamma; \emptyset; \emptyset \proves V \hastype 
                         \lhot{C}
		}{
			\Gamma; \emptyset; \emptyset \proves V \hastype 
                         \shot{C}
		} 
		\qquad
		\trule{EProm}~~\tree{
		\Gamma; \Lambda \cat x : \lhot{C}; \Delta \proves P \hastype \Proc
		}{
			\Gamma \cat x:\shot{C}; \Lambda; \Delta \proves P \hastype \Proc
		}
		\\[4mm]

		\trule{Abs}~~\tree{
			\Gamma; \Lambda; \Delta_1 \proves P \hastype \Proc
			\quad
			\Gamma; \es; \Delta_2 \proves x \hastype C
		}{
			\Gamma\backslash x; \Lambda; \Delta_1 \backslash \Delta_2 \proves \abs{{x}}{P} \hastype \lhot{{C}}
		}
		\\[4mm]

		\trule{App}~~\tree{
			\begin{array}{c}
				U = \lhot{C} \lor \shot{C}
				\quad
				\Gamma; \Lambda; \Delta_1 \proves V \hastype U
				\quad
				\Gamma; \es; \Delta_2 \proves u \hastype C
			\end{array}
		}{
			\Gamma; \Lambda; \Delta_1 \cat \Delta_2 \proves \appl{V}{u} \hastype \Proc
		} 
		\\[4mm]

		\trule{Send}~~\tree{
			\Gamma; \Lambda_1; \Delta_1 \proves P \hastype \Proc
			\quad
			\Gamma; \Lambda_2; \Delta_2 \proves V \hastype U
			\quad
			u:S \in \Delta_1 \cat \Delta_2
		}{
			\Gamma; \Lambda_1 \cat \Lambda_2; ((\Delta_1 \cat \Delta_2) \setminus u:S) \cat u:\btout{U} S \proves \bout{u}{V} P \hastype \Proc
		}
		\\[4mm]

		\trule{Rcv}~~\tree{
			\Gamma; \Lambda_1; \Delta_1 \cat u: S \proves P \hastype \Proc
			\quad
			\Gamma; \Lambda_2; \Delta_2 \proves {x} \hastype {U}
		}{
			\Gamma \backslash x; \Lambda_1\cat \Lambda_2; \Delta_1\backslash \Delta_2 \cat u: \btinp{U} S \vdash \binp{u}{{x}} P \hastype \Proc
		}
		\\[4mm]

		\trule{Req}~~\tree{
			\begin{array}{c}
				\Gamma; \es; \es \proves u \hastype U_1
				\quad
				\Gamma; \Lambda; \Delta_1 \proves P \hastype \Proc
				\quad
				\Gamma; \es; \Delta_2 \proves V \hastype U_2
				\\
				(U_1 = \chtype{S} 
                                \land %\Leftrightarrow 
                                U_2 = S)
				\lor
				 (U_1 = \chtype{L} 
                                \land %\Leftrightarrow 
                                %\Leftrightarrow 
                                 U_2 = L)
			\end{array}
		}{
			\Gamma; \Lambda; \Delta_1 \cat \Delta_2 \proves \bout{u}{V} P \hastype \Proc
		}
		\\[4mm]

		\trule{Acc}~~\tree{
			\begin{array}{c}
				\Gamma; \emptyset; \emptyset \proves u \hastype U_1 
				\quad
				\Gamma; \Lambda_1; \Delta_1 \proves P \hastype \Proc
				\quad
				\Gamma; \Lambda_2; \Delta_2 \proves x \hastype U_2\\
				(U_1 = \chtype{S} 
                                \land %\Leftrightarrow 
                                U_2 = S)
				\lor
				 (U_1 = \chtype{L} 
                                \land %\Leftrightarrow 
                                %\Leftrightarrow 
                                 U_2 = L)
	               \end{array}
		}{
			\Gamma\backslash x; \Lambda_1 \backslash \Lambda_2; \Delta_1 \backslash \Delta_2 \proves \binp{u}{x} P \hastype \Proc
		}
		\\[4mm]

		\trule{Bra}~~\tree{
			 \forall i \in I \quad \Gamma; \Lambda; \Delta \cat u:S_i \proves P_i \hastype \Proc
		}{
			\Gamma; \Lambda; \Delta \cat u: \btbra{l_i:S_i}_{i \in I} \proves \bbra{u}{l_i:P_i}_{i \in I}\hastype \Proc
		}
		\qquad
	 	\trule{Sel}~~\tree{
			\Gamma; \Lambda; \Delta \cat u: S_j  \proves P \hastype \Proc \quad j \in I

		}{
			\Gamma; \Lambda; \Delta \cat u:\btsel{l_i:S_i}_{i \in I} \proves \bsel{u}{l_j} P \hastype \Proc
		}
		\\[4mm]

		\trule{ResS}~~\tree{
			\Gamma; \Lambda; \Delta \cat s:S_1 \cat \dual{s}: S_2 \proves P \hastype \Proc \quad S_1 \dualof S_2
		}{
			\Gamma; \Lambda; \Delta \proves \news{s} P \hastype \Proc
		}
		\qquad
		\trule{Res}~~\tree{
			\Gamma\cat a:\chtype{S} ; \Lambda; \Delta \proves P \hastype \Proc
		}{
			\Gamma; \Lambda; \Delta \proves \news{a} P \hastype \Proc
		}
		\\[4mm]
 
		\trule{Par}~~\tree{
			\Gamma; \Lambda_{i}; \Delta_{i} \proves P_{i} \hastype \Proc \quad i=1,2
		}{
			\Gamma; \Lambda_{1} \cat \Lambda_2; \Delta_{1} \cat \Delta_2 \proves P_1 \Par P_2 \hastype \Proc
		}
		\qquad
		\trule{End}~~\tree{
			\Gamma; \Lambda; \Delta  \proves P \hastype T \quad u \not\in \dom{\Gamma, \Lambda,\Delta}
		}{
			\Gamma; \Lambda; \Delta \cat u: \tinact  \proves P \hastype \Proc
		}
		\\[4mm]

	 	\trule{Rec}~~\tree{
			\Gamma \cat \rvar{X}: \Delta; \emptyset; \Delta  \proves P \hastype \Proc
		}{
			\Gamma ; \emptyset; \Delta  \proves \recp{X}{P} \hastype \Proc
		}
		\qquad
		\trule{RVar}~~\Gamma \cat \rvar{X}: \Delta; \emptyset; \Delta  \proves \rvar{X} \hastype \Proc
		\qquad
		\trule{Nil}~~\Gamma; \emptyset; \emptyset \proves \inact \hastype \Proc
	\end{array}
\]
\caption{Complete Typing Rules for $\HOp$.\label{fig:typerulesmy}}
%\Hline
\end{figure}
%\myparagraph{Typing System of \HOp}





\subsection{Typing Rules}\label{app:typrules}
% !TEX root = ../journal16kpy.tex


\begin{figure}[t]
\[
	\begin{array}{c}
		\trule{Sess}~~\Gamma; \emptyset; \set{u:S} \proves u \hastype S 
		\qquad
		\trule{Sh}~~\Gamma \cat u : U; \emptyset; \emptyset \proves u \hastype U
		\qquad
		\trule{LVar}~~\Gamma; \set{x: \lhot{C}}; \emptyset \proves x \hastype \lhot{C}
		\\[4mm]

		\trule{Prom}~~\tree{
			\Gamma; \emptyset; \emptyset \proves V \hastype 
                         \lhot{C}
		}{
			\Gamma; \emptyset; \emptyset \proves V \hastype 
                         \shot{C}
		} 
		\qquad
		\trule{EProm}~~\tree{
		\Gamma; \Lambda \cat x : \lhot{C}; \Delta \proves P \hastype \Proc
		}{
			\Gamma \cat x:\shot{C}; \Lambda; \Delta \proves P \hastype \Proc
		}
		\\[4mm]

		\trule{Abs}~~\tree{
			\Gamma; \Lambda; \Delta_1 \proves P \hastype \Proc
			\quad
			\Gamma; \es; \Delta_2 \proves x \hastype C
		}{
			\Gamma\backslash x; \Lambda; \Delta_1 \backslash \Delta_2 \proves \abs{{x}}{P} \hastype \lhot{{C}}
		}
		\\[4mm]

		\trule{App}~~\tree{
			\begin{array}{c}
				U = \lhot{C} \lor \shot{C}
				\quad
				\Gamma; \Lambda; \Delta_1 \proves V \hastype U
				\quad
				\Gamma; \es; \Delta_2 \proves u \hastype C
			\end{array}
		}{
			\Gamma; \Lambda; \Delta_1 \cat \Delta_2 \proves \appl{V}{u} \hastype \Proc
		} 
		\\[4mm]

		\trule{Send}~~\tree{
			\Gamma; \Lambda_1; \Delta_1 \proves P \hastype \Proc
			\quad
			\Gamma; \Lambda_2; \Delta_2 \proves V \hastype U
			\quad
			u:S \in \Delta_1 \cat \Delta_2
		}{
			\Gamma; \Lambda_1 \cat \Lambda_2; ((\Delta_1 \cat \Delta_2) \setminus u:S) \cat u:\btout{U} S \proves \bout{u}{V} P \hastype \Proc
		}
		\\[4mm]

		\trule{Rcv}~~\tree{
			\Gamma; \Lambda_1; \Delta_1 \cat u: S \proves P \hastype \Proc
			\quad
			\Gamma; \Lambda_2; \Delta_2 \proves {x} \hastype {U}
		}{
			\Gamma \backslash x; \Lambda_1\cat \Lambda_2; \Delta_1\backslash \Delta_2 \cat u: \btinp{U} S \vdash \binp{u}{{x}} P \hastype \Proc
		}
		\\[4mm]

		\trule{Req}~~\tree{
			\begin{array}{c}
				\Gamma; \es; \es \proves u \hastype U_1
				\quad
				\Gamma; \Lambda; \Delta_1 \proves P \hastype \Proc
				\quad
				\Gamma; \es; \Delta_2 \proves V \hastype U_2
				\\
				(U_1 = \chtype{S} 
                                \land %\Leftrightarrow 
                                U_2 = S)
				\lor
				 (U_1 = \chtype{L} 
                                \land %\Leftrightarrow 
                                %\Leftrightarrow 
                                 U_2 = L)
			\end{array}
		}{
			\Gamma; \Lambda; \Delta_1 \cat \Delta_2 \proves \bout{u}{V} P \hastype \Proc
		}
		\\[4mm]

		\trule{Acc}~~\tree{
			\begin{array}{c}
				\Gamma; \emptyset; \emptyset \proves u \hastype U_1 
				\quad
				\Gamma; \Lambda_1; \Delta_1 \proves P \hastype \Proc
				\quad
				\Gamma; \Lambda_2; \Delta_2 \proves x \hastype U_2\\
				(U_1 = \chtype{S} 
                                \land %\Leftrightarrow 
                                U_2 = S)
				\lor
				 (U_1 = \chtype{L} 
                                \land %\Leftrightarrow 
                                %\Leftrightarrow 
                                 U_2 = L)
	               \end{array}
		}{
			\Gamma\backslash x; \Lambda_1 \backslash \Lambda_2; \Delta_1 \backslash \Delta_2 \proves \binp{u}{x} P \hastype \Proc
		}
		\\[4mm]

		\trule{Bra}~~\tree{
			 \forall i \in I \quad \Gamma; \Lambda; \Delta \cat u:S_i \proves P_i \hastype \Proc
		}{
			\Gamma; \Lambda; \Delta \cat u: \btbra{l_i:S_i}_{i \in I} \proves \bbra{u}{l_i:P_i}_{i \in I}\hastype \Proc
		}
		\qquad
	 	\trule{Sel}~~\tree{
			\Gamma; \Lambda; \Delta \cat u: S_j  \proves P \hastype \Proc \quad j \in I

		}{
			\Gamma; \Lambda; \Delta \cat u:\btsel{l_i:S_i}_{i \in I} \proves \bsel{u}{l_j} P \hastype \Proc
		}
		\\[4mm]

		\trule{ResS}~~\tree{
			\Gamma; \Lambda; \Delta \cat s:S_1 \cat \dual{s}: S_2 \proves P \hastype \Proc \quad S_1 \dualof S_2
		}{
			\Gamma; \Lambda; \Delta \proves \news{s} P \hastype \Proc
		}
		\qquad
		\trule{Res}~~\tree{
			\Gamma\cat a:\chtype{S} ; \Lambda; \Delta \proves P \hastype \Proc
		}{
			\Gamma; \Lambda; \Delta \proves \news{a} P \hastype \Proc
		}
		\\[4mm]
 
		\trule{Par}~~\tree{
			\Gamma; \Lambda_{i}; \Delta_{i} \proves P_{i} \hastype \Proc \quad i=1,2
		}{
			\Gamma; \Lambda_{1} \cat \Lambda_2; \Delta_{1} \cat \Delta_2 \proves P_1 \Par P_2 \hastype \Proc
		}
		\qquad
		\trule{End}~~\tree{
			\Gamma; \Lambda; \Delta  \proves P \hastype T \quad u \not\in \dom{\Gamma, \Lambda,\Delta}
		}{
			\Gamma; \Lambda; \Delta \cat u: \tinact  \proves P \hastype \Proc
		}
		\\[4mm]

	 	\trule{Rec}~~\tree{
			\Gamma \cat \rvar{X}: \Delta; \emptyset; \Delta  \proves P \hastype \Proc
		}{
			\Gamma ; \emptyset; \Delta  \proves \recp{X}{P} \hastype \Proc
		}
		\qquad
		\trule{RVar}~~\Gamma \cat \rvar{X}: \Delta; \emptyset; \Delta  \proves \rvar{X} \hastype \Proc
		\qquad
		\trule{Nil}~~\Gamma; \emptyset; \emptyset \proves \inact \hastype \Proc
	\end{array}
\]
\caption{Complete Typing Rules for $\HOp$.\label{fig:typerulesmy}}
%\Hline
\end{figure}
%\myparagraph{Typing System of \HOp}




The typing system is defined in \figref{fig:typerulesmy}.
%Types for session names/variables $u$ and
%directly derived from the linear part of the typing
%environment, i.e.~type maps $\Delta$ and $\Lambda$.
Rules $\trule{Sess, Sh, LVar}$ are name and variable introduction rules. 
The shared type $\shot{C}$ %for shared higher order values $V$
is derived using rule $\trule{Prom}$ only  
if the value has a linear type with an empty linear
environment.
Rule~$\trule{EProm}$ allows us to freely use a linear
type variable as shared.
%
Abstraction values are typed with rule~$\trule{Abs}$.
%The key type for an abstraction is the type for
%the bound variables of the abstraction, i.e.~for
%bound variable type $C$ the abstraction
%has type $\lhot{C}$.
Application typing
is governed by rule $\trule{App}$: we expect
the type $C$ of an application name $u$ 
to match the type $\lhot{C}$ or $\shot{C}$
of the application variable $x$.
%
%A process prefixed with a session send operator $\bout{k}{V} P$
%is typed using rule $\trule{Send}$.

In rule $\trule{Send}$, 
the type $U$ of a send value $V$ should appear as a prefix
on the session type $\btout{U} S$ of $u$.
Rule $\trule{Rcv}$ is its dual.  
%defined the typing for the 
%reception of values $\binp{u}{V} P$.
%the type $U$ of a receive value should 
%appear as a prefix on the session type $\btinp{U} S$ of $u$.
We use a similar approach with session prefixes
to type interaction between shared names as defined 
in rules $\trule{Req}$ and $\trule{Acc}$,
where the type of the sent/received object 
($S$ and $L$, respectively) should
match the type of the sent/received subject
($\chtype{S}$ and $\chtype{L}$, respectively).
Rules for selection and branching, denoted
$\trule{Sel}$ and $\trule{Bra}$, are standard. 
%Both
%rules prefix the session type with the selection
%type $\btsel{l_i: S_i}_{i \in I}$ and
%$\btbra{l_i:S_i}_{i \in I}$.
%

A
shared name creation $a$ creates and restricts
$a$ in environment $\Gamma$ as defined in 
rule~\trule{Res}. 
Creation of a session name $s$
creates and restricts two endpoints with dual types 
%and restricts
%them by removing them from the session environment
%$\Delta$ as defined 
in rule \trule{ResS}. 
Rule \trule{Par} 
combines the environments
$\Lambda$ and $\Delta$ of
the components of a parallel process;
%to create a type for the entire parallel process.
the disjointness of environments $\Lambda$ and $\Delta$
is implied. Rule \trule{End} adds 
the names with type $\tinact$ in $\Delta$.  
The recursion requires that the body process 
matches the type of the recursive
variable as in rule \trule{Rec}.
The recursive variable is typed
directly from the shared environment $\Gamma$ as
in rule \trule{RVar}.
The inactive process $\inact$ is typed with no
linear environments as in rule~\trule{Nil}. 


\begin{definition}[Session Environments:  Reduction]\label{d:wtenvred}%\rm
	Let $\Delta$ be a session environment.
%	\begin{enumerate}[$\bullet$]
%	\item  $\Delta$ is {\em balanced} if whenever
%	$s: S_1, \dual{s}: S_2 \in \Delta$ then $S_1 \dualof S_2$.
%	\item 
	We define the reduction relation $\red$ on session environments as: %\\ %[-2mm]
\begin{eqnarray*}
	\Delta \cat s: \btout{U} S_1 \cat \dual{s}: \btinp{U} S_2  & \red & 
	\Delta \cat s: S_1 \cat \dual{s}: S_2  \\
	\Delta \cat s: \btsel{l_i: S_i}_{i \in I} \cat \dual{s}: \btbra{l_i: S_i'}_{i \in I} &\red& \Delta \cat s: S_k \cat \dual{s}: S_k' \ (k \in I)
\end{eqnarray*}
%\end{enumerate}
\end{definition}


%We are ready to present the typing system for the \HOp,
%which is similar to~\cite{tlca07,MostrousY15}.



%%Since the typing system is similar to~\cite{tlca07,MostrousY15}, 
%%we fully describe it  in \appref{app:types}.  The %rest of the 
%%paper can
%%be read without knowing the details of the typing system. 
%%\jpc{Type soundness relies on the following auxiliary notion}.
%%%We list the key properties.
%
%\smallskip
%
%\begin{definition}[Reduction of Session Environments]%\rm
%	\label{def:ses_red}
%	We define the relation $\red$ on session environments as:
%	\begin{eqnarray*}
%			\Delta \cat s: \btout{U} S_1 \cat \dual{s}: \btinp{U} S_2 & \red &
%			\Delta \cat s: S_1 \cat \dual{s}: S_2\\%[1mm]
%			\Delta \cat s: \btsel{l_i: S_i}_{i \in I} \cat \dual{s}: \btbra{l_i: S_i'}_{i \in I} &\red& 
%			 \Delta \cat s: S_k \cat \dual{s}: S_k' \ (k \in I)
%		\end{eqnarray*}
%	%\end{center}
%%\begin{tabular}{rcl}
%%	\setlength{\tabcolsep}{0pt}
%%	$\Delta \cat s: \btout{U} S_1 \cat \dual{s}: \btinp{U} S_2$ & $\red$ & 
%%	$\Delta \cat s: S_1 \cat \dual{s}: S_2$\\[1mm]
%%	$\Delta \cat s: \btsel{l_i: S_i}_{i \in I} \cat \dual{s}: \btbra{l_i: S_i'}_{i \in I}$ & $\red$ & $\Delta \cat s: S_k \cat \dual{s}: S_k' \ (k \in I)$
%%\end{tabular}
%%\[
%%\begin{array}{rcl}
%%\Delta \cat s: \btout{U} S_1 \cat \dual{s}: \btinp{U} S_2 & \red & 
%%\Delta \cat s: S_1 \cat \dual{s}: S_2\\[1mm]
%%\Delta \cat s: \btsel{l_i: S_i}_{i \in I} \cat \dual{s}: \btbra{l_i: S_i'}_{i \in I} & \red & \Delta \cat s: S_k \cat \dual{s}: S_k' \ (k \in I)
%%\end{array}
%%\]
%\end{definition}
%
%%\smallskip
%
%%The following result %Theorem 7.3 in M\&Y
%\noi We state the type soundness result for \HOp; it implies 
%the type soundness of the sub-calculi \HO, \sessp, and $\CAL^{-\mathsf{sh}}$. 
%
%%\smallskip
%
%\begin{theorem}[Type Soundness]\label{t:sr}\rm
%%	\begin{enumerate}[1.]
%%		\item	(Subject Congruence) Suppose $\Gamma; \es; \Delta \proves P \hastype \Proc$.
%%			Then $P \scong P'$ implies $\Gamma; \es; \Delta \proves P' \hastype \Proc$.
%%
%%		\item
%%			(Subject Reduction)
%			Suppose $\Gamma; \es; \Delta \proves P \hastype \Proc$
%			with
%			$\Delta$ balanced. 
%			Then $P \red P'$ implies $\Gamma; \es; \Delta'  \proves P' \hastype \Proc$
%			and $\Delta = \Delta'$ or $\Delta \red \Delta'$
%			with $\Delta'$ balanced. 
%%	\end{enumerate}
%\end{theorem}
%

