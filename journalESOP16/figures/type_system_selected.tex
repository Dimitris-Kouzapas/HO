% !TEX root = ../journal16kpy.tex
\begin{figure}[h!]
\[
	\begin{array}{c}
	\inferrule[(Sess)]{}{\Gamma; \emptyset; \set{u:S} \proves u \hastype S} 
		\qquad
		\inferrule[(Sh)]{}{\Gamma \cat u : U; \emptyset; \emptyset \proves u \hastype U}
		\\ \\ 
		\inferrule[(LVar)]{}{\Gamma; \set{x: \lhot{C}}; \emptyset \proves x \hastype \lhot{C}}
						\qquad
		\inferrule[(RVar)]{}{\Gamma \cat \rvar{X}: \Delta; \emptyset; \Delta  \proves \rvar{X} \hastype \Proc}
				\\  \\
		\inferrule[(Abs)]{
			\Gamma; \Lambda; \Delta_1 \proves P \hastype \Proc
			\quad
			\Gamma; \es; \Delta_2 \proves x \hastype C
		}{
			\Gamma\backslash x; \Lambda; \Delta_1 \backslash \Delta_2 \proves \abs{{x}}{P} \hastype \lhot{{C}}
		}
		\quad
		\inferrule[(App)]{
			%\begin{array}{c}
				%U = \hot{C} \lor \shot{C}
				%\\
				{\Gamma; \Lambda; \Delta_1 \proves V \hastype \ghot{C} ~~
				\leadsto\, \in \{\lollipop, \sharedop\} \quad
				\Gamma; \es; \Delta_2 \proves u \hastype C}
		%	\end{array}
		}{
			\Gamma; \Lambda; \Delta_1 \cat \Delta_2 \proves \appl{V}{u} \hastype \Proc
		} 

		\\  \\
		\inferrule[(Prom)]{
			\Gamma; \emptyset; \emptyset \proves V \hastype 
                         \lhot{C}
		}{
			\Gamma; \emptyset; \emptyset \proves V \hastype 
                         \shot{C}
		} 
		\quad
		\inferrule[(EProm)]{
		\Gamma; \Lambda \cat x : \lhot{C}; \Delta \proves P \hastype \Proc
		}{
			\Gamma \cat x:\shot{C}; \Lambda; \Delta \proves P \hastype \Proc
		}
				\quad
		\inferrule[(End)]{
			\Gamma; \Lambda; \Delta  \proves P \hastype T \quad u \not\in \dom{\Gamma, \Lambda,\Delta}
		}{
			\Gamma; \Lambda; \Delta \cat u: \tinact  \proves P \hastype \Proc
		}
		\\  \\
		\inferrule[(Rec)]{
			\Gamma \cat \rvar{X}: \Delta; \emptyset; \Delta  \proves P \hastype \Proc
		}{
			\Gamma ; \emptyset; \Delta  \proves \recp{X}{P} \hastype \Proc
		}
		\qquad
			\inferrule[(Par)]{
			\Gamma; \Lambda_{i}; \Delta_{i} \proves P_{i} \hastype \Proc \quad i=1,2
		}{
			\Gamma; \Lambda_{1} \cat \Lambda_2; \Delta_{1} \cat \Delta_2 \proves P_1 \Par P_2 \hastype \Proc
		}
		\qquad
				\inferrule[(Nil)]{ }{\Gamma; \emptyset; \emptyset \proves \inact \hastype \Proc}
		\\  \\
		\inferrule[(Send)]{
					%\begin{array}{c}
					u:S \in \Delta_1 \cat \Delta_2 %\\
					\quad
			\Gamma; \Lambda_1; \Delta_1 \proves P \hastype \Proc
			\quad
			\Gamma; \Lambda_2; \Delta_2 \proves V \hastype U
			%\end{array}
		}{
			\Gamma; \Lambda_1 \cat \Lambda_2; ((\Delta_1 \cat \Delta_2) \setminus u:S) \cat u:\btout{U} S \proves \bout{u}{V} P \hastype \Proc
		}
		\\  \\
		\inferrule[(Req)]{
			%\begin{array}{c}
%				\Gamma; \es; \es \proves u \hastype U_1
%				\qquad
%				\Gamma; \Lambda; \Delta_1 \proves P \hastype \Proc
%				%\\
%				\qquad
%				\Gamma; \es; \Delta_2 \proves V \hastype U_2
%				\\
%				(U_1 = \chtype{S} 
%                                \land  
%                                U_2 = S)
%				\lor
%				 (U_1 = \chtype{L} 
%                                \land  
%                                 U_2 = L)
%                                 \\
                                 				{\Gamma; \Lambda; \Delta_1 \proves P \hastype \Proc
				\quad
                                 \Gamma; \es; \es \proves u \hastype \chtype{\mathcal{U}}
				\quad
				\Gamma; \es; \Delta_2 \proves V \hastype \mathcal{U}
				\quad \mathcal{U} \in \{S, L\}}
			%\end{array}
		}{
			\Gamma; \Lambda; \Delta_1 \cat \Delta_2 \proves \bout{u}{V} P \hastype \Proc
		}
		~~
		\\ \\
				\inferrule[(Rcv)]{
		%\begin{array}{c}
			\Gamma; \Lambda_1; \Delta_1 \cat u: S \proves P \hastype \Proc
			\quad
			\Gamma; \Lambda_2; \Delta_2 \proves {x} \hastype {U}
		%	\end{array}
		}{
			\Gamma \backslash x; \Lambda_1\backslash \Lambda_2; \Delta_1\backslash \Delta_2 \cat u: \btinp{U} S \vdash \binp{u}{{x}} P \hastype \Proc
		}
		~~
		\inferrule[(Acc)]{
			\begin{array}{c}
%				\Gamma; \emptyset; \emptyset \proves u \hastype U_1 
%				\quad
%				\Gamma; \Lambda_1; \Delta_1 \proves P \hastype \Proc
%				\\
%				\Gamma; \Lambda_2; \Delta_2 \proves x \hastype U_2\\
%				(U_1 = \chtype{S} 
%                                \land
%                                U_2 = S)
%				\lor
%				 (U_1 = \chtype{L} 
%                                \land 
%                                 U_2 = L)
%                                 \\
                                 {
                                 				\Gamma; \Lambda_1; \Delta_1 \proves P \hastype \Proc
				\quad
                                 	\Gamma; \emptyset; \emptyset \proves u \hastype \chtype{\mathcal{U}} 
				}
				\\
				{\Gamma; \Lambda_2; \Delta_2 \proves x \hastype \mathcal{U}
				\quad \mathcal{U} \in \{S, T\}}
	               \end{array}
		}{
			\Gamma\backslash x; \Lambda_1 \backslash \Lambda_2; \Delta_1 \backslash \Delta_2 \proves \binp{u}{x} P \hastype \Proc
		}	
		\\  \\
				\inferrule[(Bra)]{
			 \forall i \in I \quad \Gamma; \Lambda; \Delta \cat u:S_i \proves P_i \hastype \Proc
		}{
			\Gamma; \Lambda; \Delta \cat u: \btbra{l_i:S_i}_{i \in I} \proves \bbra{u}{l_i:P_i}_{i \in I}\hastype \Proc
		}
		\qquad
	 	\inferrule[(Sel)]{
			\Gamma; \Lambda; \Delta \cat u: S_j  \proves P \hastype \Proc \quad j \in I
		}{
			\Gamma; \Lambda; \Delta \cat u:\btsel{l_i:S_i}_{i \in I} \proves \bsel{u}{l_j} P \hastype \Proc
		}
		\\  \\
		\inferrule[(ResS)]{
			\Gamma; \Lambda; \Delta \cat s:S_1 \cat \dual{s}: S_2 \proves P \hastype \Proc \quad S_1 \dualof S_2
		}{
			\Gamma; \Lambda; \Delta \proves \news{s} P \hastype \Proc
		}
		\qquad
		\inferrule[(Res)]{
			\Gamma\cat a:\chtype{S} ; \Lambda; \Delta \proves P \hastype \Proc
		}{
			\Gamma; \Lambda; \Delta \proves \news{a} P \hastype \Proc
		}
		\end{array}
\]
%\vspace{-3mm}
\caption{Typing Rules for $\HOp$.
%See \appref{app:types} for a full account.
\label{fig:typerulesmys}}
%\Hline
%\vspace{-1mm}
\end{figure}
%\myparagraph{Typing System of \HOp}



