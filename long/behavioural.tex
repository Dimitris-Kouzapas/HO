\section{Behavioural Semantics}
\label{sec:beh_sem}

In this section we define a theory for observational equivalence over
session typed $\HOp$ processes. The theory follows the principles
laid by the previous work of the authors
\cite{DBLP:conf/forte/KouzapasYH11,KY13,dkphdthesis}.
We require a bisimulation relation over typed processes that
is also characterised by the corresponding typed, reduction-closed,
barbed congruence relation.

\dk{(Jorge, I think you have a paper we can cite over session typed bisimulations)}

\subsection{Labelled Transition Semantics}

We define a relation $(P_1, \ell, P_2) \in R$ over
(untyped) processes, that allows us to follow how a process may
interact with a process in its enviroment. The interaction
is defined on action $\ell$:
%
\[
\begin{array}{rcl}
		\ell	& \bnfis  & \tau \bnfbar \bactout{n}{\tilde{m}} \bnfbar \bactout{n}{\abs{\tilde{x}}{P}} \bnfbar\bactinp{n}{\tilde{m}} \bnfbar \bactinp{n}{\abs{\tilde{x}}{P}} \\
			& \bnfbar & \bactsel{n}{l} \bnfbar \bactbra{n}{l} \bnfbar \news{\tilde{s}} \bactout{n}{\tilde{m}} \bnfbar \news{\tilde{m}} \bactout{n}{\abs{\tilde{x}}{P}}
\end{array}
\]
%
\noi The internal action is defined on label $\tau$.
Action $\bactout{n}{\tilde{m}}$ denotes the sending of names $\tilde{m}$ over channel $n$.
Similarly action $\bactout{n}{\abs{\tilde{x}}{P}}$ is the sending of abstraction $\abs{\tilde{x}}{P}$
over channel $n$. Dually actions for the reception of names and abstractions are
$\bactinp{n}{\tilde{m}}$ and $\bactinp{n}{\abs{\tilde{x}}{P}}$ respectively. We also defined
actions for selecting a label $l$, $\bactsel{n}{l}$ and branching on a label
$n$, $\bactbra{s}{l}$. When output actions carry name restrictions a scope
opening is implied.

We define the notion of dual actions as the symmetric relation $\asymp$, that satisfies the rules:
%
\[
	\bactsel{n}{l} \asymp \bactbra{\dual{n}}{l}
	\qquad
	\news{\tilde{s}} \bactout{n}{\abs{\tilde{x}}{P}} \asymp \bactinp{\dual{n}}{\abs{\tilde{x}}{P}}
	\qquad
	\bactout{n}{m} \asymp \bactinp{\dual{n}}{m}
	\qquad
	\news{\tilde{s}} \bactout{n}{\tilde{m}} \asymp \bactinp{\dual{n}}{\tilde{m}}
\]
%
Dual actions happen on subjects that are dual between them; they carry the same
object; and furthermore output action is dual with input action and 
select action is dual with branch action.

\begin{figure}[t]
	\[
	\begin{array}{c}
		\appl{\abs{x}{P}}{u} \by{\tau} P \subst{u}{x}\ \ltsrule{App}
		\qquad
		\bout{n}{V} P \by{\bactout{n}{V}} P\ \ltsrule{Out}
		\qquad
		\binp{n}{x} P \by{\bactinp{n}{V}} P\subst{V}{x}\ \ltsrule{In}
%		\qquad
%		\bout{n}{\abs{x}{Q}} P \by{\bactout{n}{\abs{x}{Q}}} P\ \ltsrule{OutA}
		\\[4mm]

%		\binp{n}{X} P \by{\bactinp{n}{\abs{x}{Q}}} P\subst{\abs{x}Q}{X}\ \ltsrule{InA}
%		\qquad
		\bsel{s}{l}{P} \by{\bactsel{s}{l}} P\ \ltsrule{Sel}
		\qquad
		\tree{
			j \in I
		}
		{
			\bbra{s}{l_i:P_i}_{i \in I} \by{\bactbra{s}{l_j}} P_j
		}\ \ltsrule{Bra}
		\\[6mm]

		\tree{
			P \by{\ell} P' \quad n \notin \fn{\ell}
		}{
			\news{n} P \by{\ell} \news{n} P' 
		}\ \ltsrule{Res}
		\qquad
		\tree{
			P \scong_\alpha P'' \quad P'' \by{\ell} P'
		}{
			P \by{\ell} P'
		}\ \ltsrule{Alpha}
		\qquad
		\tree{
			P \subst{\recp{X}{P}}{\varp{X}} \by{\ell} P'% \qquad P \scong_\alpha P'' \quad P'' \by{\ell} P'
		}{
			\recp{X}{P} \by{\ell} P'
		}\ \ltsrule{Rec}
		\\[6mm]


		\tree{
			P \by{\news{\tilde{m}} \bactout{n}{V}} P' \quad m \in \fn{V}
		}{
			\news{m} P \by{\news{m\cat\tilde{m}} \bactout{n}{V}} P'
		}\ \ltsrule{Scope}
		\qquad
%		\tree{
%			P \by{\news{\tilde{m}} \bactout{n}{\abs{x} Q}} P' \quad m' \in \fn{\abs{x} Q}
%		}{
%			\news{m'} P \by{\news{m'\cat\tilde{m}} \bactout{n}{\abs{x} Q}} P'
%		}\ \ltsrule{ScopeA}
%		\qquad
		\tree{
			P \by{\ell_1} P' \qquad Q \by{\ell_2} Q' \qquad \ell_1 \asymp \ell_2
		}{
			P \Par Q \by{\tau} \newsp{\bn{\ell_1} \cup \bn{\ell_2}}{P' \Par Q'}
		}\ \ltsrule{Tau}
		\\[6mm]

		\tree{

			P \by{\ell} P' \quad \bn{\ell} \cap \fn{Q} = \es
		}{
			P \Par Q \by{\ell} P' \Par Q
		}\ \ltsrule{LPar}
		\qquad
		\tree{
			Q \by{\ell} Q' \quad \bn{\ell} \cap \fn{P} = \es
		}{
			P \Par Q \by{\ell} P \Par Q'
		}\ \ltsrule{RPar}
%		\\[6mm]
	\end{array}
	\]
	\caption{The Untyped (Early) Labelled Transition System \label{fig:untyped_LTS}}
\end{figure}


{\bf Untyped Labelled Transition System:}
The labelled transition system, LTS, is defined in Figure~\ref{fig:untyped_LTS}.
A process with a send prefix can interact with the environment with a send
action that carries a name $m$ or an abstraction $\abs{\tilde{x}}{Q}$. Dually
on a received prefixed process we can observe a receive action of a name or
an abstraction. Select and branch prefixed processes can trigger select
and branch actions respectively. The LTS is closed under the name creation
operator provided that the restricted name does not occur free in the LTS action.
If the restricted name occurs free in the LTS action then we observe a bound name
action and the continuation process performs scope opening. Similarly the LTS 
is closed on the parallel operator provided that the LTS action does not shared
any bound names with parallel processes. If two parallale processes can perform
dual actions then the two actions can synchronise to observe an internal transition.
Finally the LTS is closed under structural congruence.

{\bf Simple Process: }
We can use a session type to define the simplest process that is typed
under the given session type.
%
\begin{definition}[Simple Process]\rm
	Let name $k$ and type $\tilde{C}$; then we define a {\em simple process}
	$\map{\tilde{C}}^{k}$:
	\[
		\map{C_1 \cdots C_n}^{k_1 \cdots k_n} = \map{C_1}^{k_1} \Par \dots \Par \map{C_n}^{k_n}		
	\]
	with $\map{C}^k$ defined as:
	\[
	\begin{array}{rclcrcll}
		\map{\btinp{\tilde{C}} S}^{k} &=& \binp{k}{\tilde{x}} (\map{S}^{k} \Par \map{\tilde{C}}^{\tilde{x}})
		&&
		\map{\btout{\tilde{C}} S}^{k} &=& \bout{k}{\tilde{n}} \map{S}^{k} & \tilde{n}\textrm{ fresh}
		\\

		\map{\btsel{l : S}}^{k} &=& \bsel{k}{l} \map{S}^{k}
		& &
		\map{\btbra{l_i: S_i}_{i \in I}}^{k} &=& \bbra{k}{l_i: \map{S_i}^{k}}_{i \in I}
		\\

		\map{\tvar{t}}^{k} &=& \rvar{X}_t
		& &
		\map{\trec{t}{S}}^{k} &=& \mu \rvar{X}_t.\map{S}^{k}
		\\

		\map{\btout{\lhot{\tilde{C}}} S}^{k} &=& \bout{k}{\abs{\tilde{x}}{\map{\tilde{C}}^{\tilde{x}}}} \map{S}^k
		& &
		\map{\btinp{\lhot{\tilde{C}}} S}^{k} &=& \binp{k}{X} (\map{S}^k \Par \appl{X}{\tilde{n}}) & \tilde{n}\textrm{ fresh}
		\\

		\map{\btout{\shot{\tilde{C}}} S}^{k} &=& \bout{k}{\abs{\tilde{x}}{\map{\tilde{C}}^{\tilde{x}}}} \map{S}^k
		& &
		\map{\btinp{\shot{\tilde{C}}} S}^{k} &=& \binp{k}{X} (\map{S}^k \Par \appl{X}{\tilde{n}}) & \tilde{n}\textrm{ fresh}
		\\

%		\map{\btinp{\chtype{S}} S}^{k} &=& \binp{k}{x} (\map{S}^k \Par \map{\chtype{S}}^y)
%		&&
%		\map{\btout{\chtype{S}} S}^{k} &=& \bout{k}{a} \map{S}^k  & a\textrm{ fresh}
%		\\

		\map{\tinact}^{k} &=& \inact
		&&
		\map{\chtype{S}}^{k} &=& \bout{k}{s} \inact &s\textrm{ fresh}%\binp{k}{x} \map{S}^x 
	\end{array}
	\]
%	We extend the {\em simple process} to the polyadic case:
%\[
%	\begin{array}{c}
%	\end{array}
%\]
\end{definition}

\begin{proposition}\rm
$ $
	\begin{itemize}
		\item	$\Gamma; \emptyset; \Delta \cat k:S \proves \map{S}^k \hastype \Proc$
		\item	$\Gamma \cat k:\chtype{S}; \emptyset; \Delta \proves \map{\chtype{S}}^k \hastype \Proc$
	\end{itemize}
\end{proposition}

\begin{proof}
	By induction on the definition $\map{S}^k$ and $\map{\chtype{S}}^k$.
\end{proof}

\begin{corollary}\rm
	If $\Gamma; \emptyset; \Delta \proves \map{\tilde{C}}^{\tilde{k}} \hastype \Proc$
	then
	$\Gamma; \es; \Delta \proves \tilde{k} \hastype \tilde{C}$.
\end{corollary}


{\bf Labeled Transition System for Typed Environments: }
We define two relations of the form:
%
\[
	((\Gamma, \ell_1, \Delta_1), \ell, (\Gamma, \ell_2, \Delta_2)) \in R \textrm{ and }
\]
%
\noi over type tuples, that allows us to follow the progress of types over actions $\ell$.

\begin{figure}
	\[
	\begin{array}{c}
		\tree{
			\dual{s} \notin \dom{\Sigma} \quad \Gamma; \Lambda_1; \Sigma_1 \proves V \hastype U \quad \Sigma_1 \subseteq \Sigma \quad \Lambda_1 \subseteq \Lambda
		}{
			(\Gamma; \Lambda; \Sigma \cat s: \btout{U} S) \by{\bactout{s}{V}} (\Gamma; \Lambda\backslash\Lambda_1; \Sigma\backslash\Sigma_1 \cat s: S)
		}
		\\[6mm]
		\tree{
			\dual{s} \notin \dom{\Sigma} \quad  \Gamma; \Lambda_1; \Sigma_1 \proves V \hastype U
		}{
			(\Gamma; \Lambda; \Sigma \cat s: \btinp{U} S) \by{\bactinp{s}{U}} (\Gamma; \Lambda \cup \Lambda_1; \Sigma \cup \Sigma_1 \cat s: S)
		}
		\\[6mm]
		\tree{
			\dual{s} \notin \dom{\Sigma} \quad k \in I
		}{
			(\Gamma; \Lambda; \Sigma \cat s: \btsel{l_i: S_i}_{i \in I}) \by{\bactsel{s}{l_k}} (\Gamma; \Lambda; \Sigma \cat s:S_k)
		}
		\qquad
		\tree{
			\dual{s} \notin \dom{\Sigma} \quad k \in I
		}{
			(\Gamma; \Lambda; \Sigma \cat s: \btbra{l_i: T_i}_{i \in I}) \by{\bactbra{s}{l_k}} (\Gamma; \Lambda; \Sigma \cat s:S_k)
		}
		\\[6mm]

		\tree{
			(\Gamma; \Lambda_1; \Sigma_1) \by{\news{\tilde{s}} \bactout{s}{V}} (\Gamma; \Lambda_2; \Sigma_2)
	%		\quad S_1 \dualof S_2
		}{
			(\Gamma; \Lambda_1; \Sigma_1) \by{\news{s' \cat \tilde{s'}} \bactout{s}{V}} (\Gamma; \Lambda_2; \Sigma_2 \cat \dual{s'}: S)
		}
		\qquad
		\tree{
			\Sigma_1 \red \Sigma_2
		}{
			(\Gamma; \Lambda; \Sigma_1) \by{\tau} (\Gamma; \Lambda; \Sigma_2)
		}
	\end{array}
	\]
	\caption{Labelled Transition Semantics for Typed Enviroments \label{fig:envLTS}}
\end{figure}


The upper part of Figure~\ref{fig:envLTS} describes relation $\by{\ell}$.
\dk{describe env LTS}.

A second envrionment LTS, denoted $\hby{\ell}$, is defined in the lower part of
Figure~\ref{fig:envLTS}, by substituting the corresponding input cases
of $\by{\ell}$ with the definitions of $\hby{ell}$. All other definiting
cases remain the same as relation $\by{\ell}. $Relation $\hby{\ell}$ 
restricts higher order input; only simple processes and trigger processes
are allowed to be receive on a higher order input.
\dk{describe env LTS}. 

\paragraph{Typed Transition System}

We define two transition systems over processes, as a combination
of the untyped LTS and the LTS for typed environments:

\begin{definition}[Typed Transition Systems]\rm
	We write
	$\Gamma; \emptyset; \Delta_1 \proves P_1 \hastype \Proc \by{\ell} \Gamma; \emptyset; \Delta_1 \proves P_2 \hastype \Proc$
	whenever:
%
	\begin{itemize}
		\item	$P_1 \by{\ell} P_2$.
		\item	$(\Gamma, \emptyset, \Delta_1) \by{\ell} (\Gamma, \emptyset, \Delta_2)$.
	\end{itemize}
%

	Furthermore we write
	$\Gamma; \emptyset; \Delta_1 \proves P_1 \hastype \Proc \hby{\ell} \Gamma; \emptyset; \Delta_1 \proves P_2 \hastype \Proc$
	whenever:
%
	\begin{itemize}
		\item	$P_1 \by{\ell} P_2$.
		\item	$(\Gamma, \emptyset, \Delta_1) \hby{\ell} (\Gamma, \emptyset, \Delta_2)$.
	\end{itemize}
%
\end{definition}

For notational convenience we can write
$\Gamma; \emptyset; \Delta_1 \by{\ell} \Delta_2 \proves P_1 \by{\ell} P_2$,
instead of $\Gamma; \emptyset; \Delta_1 \proves P_1 \hastype \Proc \by{\ell} \Gamma; \emptyset; \Delta_1 \proves P_2 \hastype \Proc$.
We extend to $\By{}$ (resp.\ $\Hby{}$) and $\By{\hat{\ell}}$ (resp.\ $\Hby{\hat{\ell}}$) in the \dk{standard way}.

The next invariant clarifies the soundness of the
typed transition system.

\begin{lemma}[Invariant]
	\begin{itemize}
		\item	If $\Gamma; \emptyset; \Delta_1 \proves P_1 \hastype \Proc$ and
			$P_1 \by{\ell} P_2$ and $(\Gamma; \emptyset; \Delta_1) \by{\ell} (\Gamma; \emptyset; \Delta_2)$
			then $\Gamma; \emptyset; \Delta_2 \proves P_2 \hastype \Proc$.
		\item	If $\Gamma; \emptyset; \Delta_1 \hby{\ell} \Delta_2 \proves P_1 \hby{\ell} P_2$
			then $\Gamma; \emptyset; \Delta_1 \by{\ell} \Delta_2 \proves P_1 \by{\ell} P_2$

	\end{itemize}
\end{lemma}

\begin{proof}
	\dk{TODO}
\end{proof}

\subsection{Behavioural Semantics}

We use the typed labelled transition semantics to define
a set of relations over typed processes that allow us to compare
typed processes over a notion of observational equivalence.


We begin with a definition of a notion of confluence
over session environments $\Delta$:
\begin{definition}[Session Environment Confluence]\rm
	We denote $\Delta_1 \bistyp \Delta_2$ whenever $\exists \Delta$ such that
	$\Delta_1 \red^* \Delta$ and $\Delta_2 \red^* \Delta$.
\end{definition}

%\jp{The following definition is a bit too "loose". Need to add conditions on $\Delta_1,\Delta_2$, and a better notation not involving the empty $\ell$.}

A typed relation is a relation over typed programs:

\begin{definition}[Typed Relation]\rm
	We say that
	$\Gamma; \emptyset; \Delta_1 \proves P_1 \hastype \Proc\ R\ \Gamma; \emptyset; \Delta_2 \proves P_2 \hastype \Proc$
	is a typed relation whenever
	\begin{itemize}
		\item	$P_1$ and $P_2$ are programs.
		\item	$\Delta_1$ and $\Delta_2$ are well typed.
		\item	$\Delta_1 \bistyp \Delta_2$.
	\end{itemize}
\end{definition}

We require that relate only programs (i.e.\ processes with no free variables) with
well typed session environments and furthermore we require that two related processes
have confluent session environments.

For notational convenience we write $\Gamma; \emptyset; \Delta_1\ R\ \Delta_2 \proves P_1\ R\ P_2$
for expressing the typed relation $\Gamma; \emptyset; \Delta_1 \proves P_1 \hastype \Proc\ R\ \Gamma; \emptyset; \Delta_2 \proves P_2 \hastype \Proc$.

We define the notions of barb and typed barb.

\begin{definition}[Barbs]\rm
	Let program $P$.
	\begin{enumerate}
		\item	We write $P \barb{n}$ if $P \scong \newsp{\tilde{m}}{\bout{n}{V} P_2 \Par P_3}, n \notin \tilde{m}$.
			We write $P \Barb{n}$ if $P \red^* \barb{n}$.

		\item	We write $\Gamma; \emptyset; \Delta \proves P \barb{n}$ if
			$\Gamma; \emptyset; \Delta \proves P \hastype \Proc$ with $P \barb{n}$ and $\dual{n} \notin \Delta$.
			We write $\Gamma; \emptyset; \Delta \proves P \Barb{n}$ if $P \red^* P'$ and
			$\Gamma; \emptyset; \Delta' \proves P' \barb{n}$.			
	\end{enumerate}
\end{definition}

A barb $\barb{n}$ is an observable on an output prefix with subject $n$.
Similarly a weak barb $\Barb{n}$ is a barb after a number of reduction steps.
Typed barbs $\barb{n}$ (resp.\ $\Barb{n}$)
happen on typed processes $\Gamma; \emptyset; \Delta \proves P \hastype \Proc$
where we require that the corresponding dual endpoint $\dual{n}$ is not present
in the session type $\Delta$.

We define the notion of the context:

\begin{definition}[Context]\rm
	$C$ is a context defined on the grammar:

	\begin{tabular}{rcl}
		$C$ &$=$& %$\news{s} C \bnfbar \hole \Par P$
		$\hole \bnfbar P \bnfbar \bout{k}{\abs{x}{C}} C \bnfbar \binp{k}{X} C \bnfbar \binp{k}{x} C$ \\
		& $\bnfbar$ & $\news{s} C \bnfbar C \Par C \bnfbar \bsel{k}{l} C \bnfbar \bbra{k}{l_i:C_i}_{i \in I}$
	\end{tabular}

	Notation $\context{C}{P}$ replaces every $\hole$ in $C$ with $P$.
\end{definition}

A context is a function that takes a process and returns a new process
according to the above syntax.

%We extend the notion of context to the notion of typed context:
%\begin{definition}[Typed Context]
%	Let program $\Gamma; \emptyset; \Delta \proves P \hastype \Proc$ then	
%\end{definition}

The first behavioural relation we define is reduction-closed, barbed congruence:
%
\begin{definition}[Reduction-closed, Barbed Congruence]\rm
	Typed relation $\Gamma; \emptyset; \Delta_1\ \mathcal{R}\ \Delta \proves P_1 \ \mathcal{R}\ P_2$ is a barbed congruence
	whenever:
	\begin{enumerate}
		\item
		\begin{itemize}
			\item	If $P_1 \red P_1'$ then $\exists P_2', P_2 \red^* P_1'$ and $\Gamma; \emptyset; \Delta_1'\ \mathcal{R}\ \Delta_2' \proves P_1'\ \mathcal{R}\ P_2'$.
			\item	If $P_2 \red P_2'$ then $\exists P_1', P_1 \red^* P_1'$ and $\Gamma; \emptyset; \Delta_1'\ \mathcal{R}\ \Delta_2' \proves P_1'\ \mathcal{R}\ P_2'$.
		\end{itemize}
		\item
		\begin{itemize}
			\item	If $\Gamma;\emptyset;\Delta \proves P_1 \barb{s}$ then $\Gamma;\emptyset;\Delta \proves P_2 \Barb{s}$.
			\item	If $\Gamma;\emptyset;\Delta \proves P_2 \barb{s}$ then $\Gamma;\emptyset;\Delta \proves P_1 \Barb{s}$.
		\end{itemize}
		\item	$\forall C, \Gamma; \emptyset; \Delta_1'\ \mathcal{R}\ \Delta_2' \proves \context{C}{P_1}\ \mathcal{R}\ \context{C}{P_2}$.
	\end{enumerate}
	The largest such congruence is denoted with $\cong$.
\end{definition}
%
Reduction-closed, barbed congruence is closed under reduction semantics and 
preserves barbs under any context. In a sense no barb observer can distinguish
between two related processes.

The second behavioural relation is contextual bisimulation:
%
\begin{definition}[Contextual Bisimulation]\rm
	Let typed relation $\mathcal{R}$ such that $\Gamma; \emptyset; \Delta_1\ \mathcal{R}\ \Delta_2 \proves P_1\ \mathcal{R}\ Q_1$.
	$\mathcal{R}$ is a {\em contextual bisimulation} whenever:
	\begin{enumerate}
		\item	$\forall \news{\tilde{m_1}} \bactout{n}{\abs{\tilde{x}}{P}}$ such that
			\[
				\Gamma; \emptyset; \Delta_1 \by{\news{\tilde{m_1}} \bactout{n}{\abs{\tilde{x}}{P}}} \Delta_1' \proves P_1 \by{\news{\tilde{m_1}} \bactout{n}{\abs{\tilde{x}}{P}}} P_2
			\]
			implies that $\exists Q_2, \abs{x}{Q}$ such that
			\[
				\Gamma; \emptyset; \Delta_2 \By{\news{\tilde{m_2}} \bactout{n}{\abs{\tilde{x}}{Q}}} \Delta_2' \proves Q_1 \By{\news{\tilde{m_2}} \bactout{n}{\abs{\tilde{x}}{Q}}} Q_2
			\]
			and $\forall R$ with $\set{X} = \fpv{R}$, %such that
%			\begin{eqnarray*}
%				\Gamma; \emptyset; \Delta_1'' \proves \newsp{\tilde{s}}{P_2 \Par \context{C}{P \subst{s'}{x}}} \hastype \Proc \\
%				\Gamma; \emptyset; \Delta_2'' \proves \newsp{\tilde{s}}{Q_2 \Par \context{C}{Q \subst{s'}{x}}} \hastype \Proc
%			\end{eqnarray*}
			then
			\[
				\Gamma; \emptyset; \Delta_1''\ \mathcal{R}\ \Delta_2'' \proves \newsp{\tilde{m_1}}{P_2 \Par R\subst{(\tilde{x}) P}{X}}
				\ \mathcal{R}\ 
				\newsp{\tilde{m_2}}{Q_2 \Par R\subst{(\tilde{x}) Q}{X}}
			\]
		\item	$\forall \news{\tilde{s_1}} \bactout{n}{\tilde{m_1}}$ such that
			\[
				\Gamma; \emptyset; \Delta_1 \by{\news{\tilde{s_1}} \bactout{n}{\tilde{m_1}}} \Delta_1' \proves P_1 \by{\news{\tilde{s_1}} \bactout{n}{\tilde{m_1}}} P_2
			\]
			implies that $\exists Q_2, \tilde{m_2}$ such that
			\[
				\Gamma; \emptyset; \Delta_2 \By{\news{\tilde{s_2}} \bactout{n}{\tilde{m_2}}} \Delta_2' \proves Q_1 \By{\news{\tilde{s_2}} \bactout{n}{\tilde{m_2}}} Q_2
			\]
			and $\forall R$ with $\tilde{x} = \fn{R}$, %such that
%			\begin{eqnarray*}
%				\Gamma; \emptyset; \Delta_1'' \proves \newsp{\tilde{s}}{P_2 \Par \context{C}{P \subst{s'}{x}}} \hastype \Proc \\
%				\Gamma; \emptyset; \Delta_2'' \proves \newsp{\tilde{s}}{Q_2 \Par \context{C}{Q \subst{s'}{x}}} \hastype \Proc
%			\end{eqnarray*}
			then
			\[
				\Gamma; \emptyset; \Delta_1''\ \mathcal{R}\ \Delta_2'' \proves \newsp{\tilde{s_1}}{P_2 \Par R \subst{\tilde{m_1}}{\tilde{x}}}
				\ \mathcal{R}\ 
				\newsp{\tilde{s_2}}{Q_2 \Par R \subst{\tilde{m_2}}{\tilde{x}}}
			\]

		\item	$\forall \ell \notin \set{\news{\tilde{s}} \bactout{n}{m}, \news{\tilde{m}} \bactout{n}{\abs{x}{P}}}s$ such that
			\[
				\Gamma; \emptyset; \Delta_1 \by{\ell} \Delta_1' \proves P_1 \by{\ell} P_2
			\]
			implies that $\exists Q_2$ such that 
			\[
				\Gamma; \emptyset; \Delta_2 \By{\hat{\ell}} \Delta_2' \proves Q_1 \By{\hat{\ell}} Q_2
			\]
			and
			$\Gamma; \emptyset; \Delta_1\ \mathcal{R}\ \Delta_2 \proves P_2\ \mathcal{R}\ Q_2$.

		\item	The symmetric cases of 1, 2 and 3.
	\end{enumerate}
	The Knaster Tarski theorem ensures that the largest contextual bisimulation exists, it is called contextual bisimilarity and is denoted by $\wb^c$.
\end{definition}

The contextual bisimulation in the general case is a hard relation to be computed
since it is universaly quantified over substituting processes. The next definition
of a bisimulation relation avoids the universal quantifier over processes.

%\newcommand{\outtrigger}[2]{\binp{#1}{X} \newsp{s''}{\appl{X}{s''} \Par \bout{\dual{s''}}{#2} \inact} }
%\newcommand{\nametrigger}[2]{\binp{#1}{X} \appl{X}{#2}}

\begin{definition}[Bisimulation]\rm
	Let typed relation $\mathcal{R}$ such that $\Gamma; \emptyset; \Delta_1\ \mathcal{R}\ \Delta_2 \proves P_1\ \mathcal{R}\ Q_1$.
	$\mathcal{R}$ is a {\em bisimulation} whenever:
	\begin{enumerate}
		\item	$\forall \news{\tilde{m_1}} \bactout{n}{\abs{\tilde{x}}{P}}$ such that
			\[
				\Gamma; \emptyset; \Delta_1 \hby{\news{\tilde{m_1}} \bactout{n}{\abs{\tilde{x}}{P}}} \Delta_1' \proves P_1 \hby{\news{\tilde{m_1}} \bactout{n}{\abs{\tilde{x}}{P}}} P_2
			\]
			implies that $\exists Q_2, \abs{\tilde{x}}{Q}$ such that
			\[
				\Gamma; \emptyset; \Delta_2 \Hby{\news{\tilde{m_2}} \bactout{n}{\abs{\tilde{x}}{Q}}} \Delta_2' \proves Q_1 \Hby{\news{\tilde{m_2}} \bactout{n}{\abs{\tilde{x}}{Q}}} Q_2
			\]
			and for fresh $t$
%			and for $C = \binp{t}{X} \newsp{s}{\appl{X}{s} \Par \bout{s}{\abs{\tilde{x}}{\hole}} \inact}$ with $t$ fresh
%			such that
%			\begin{eqnarray*}
%				\Gamma; \emptyset; \Delta_1'' \proves \newsp{\tilde{s}}{P_2 \Par \context{C}{P \subst{s'}{x}}} \hastype \Proc \\
%				\Gamma; \emptyset; \Delta_2'' \proves \newsp{\tilde{s}}{Q_2 \Par \context{C}{Q \subst{s'}{x}}} \hastype \Proc
%			\end{eqnarray*}
%			then
			\[
				\Gamma; \emptyset; \Delta_1''\ \mathcal{R}\ \Delta_2'' \proves
				\newsp{\tilde{m_1}}{P_2 \Par \hotrigger{t}{\abs{x}{P}}}%\context{C}{P}}
				\ \mathcal{R}\ 
				\newsp{\tilde{m_2}}{Q_2 \Par \hotrigger{t}{\abs{x}{Q}}}%\context{C}{Q}}
			\]

%		\item	$\exists \bactinp{s}{\abs{x} \map{S}^x}$ such that
%			\[
%				\Gamma; \emptyset; \Delta_1 \cat s: S_1 \by{\bactinp{s}{\abs{x} \map{S}^x}} \Delta_1 \cat s: S_2 \proves P_1 \by{\bactinp{s}{\abs{x} \map{S}^x}} P_2 \subst{\abs{x}{\map{S}^x}}{X}
%			\]
%			with $S_1 = \btinp{\lhot{S}} S_2 \vee S_1 = \btinp{\shot{S}} S_1$ implies that $\exists Q_2\subst{\abs{x}{\map{S}^x}}{X}$ such that
%			\[
%				\Gamma; \emptyset; \Delta_2 \By{\bactinp{s}{\abs{x} \map{S}^x}} \Delta_2' \proves Q_1 \By{\bactinp{s}{\abs{x} \map{S}^x}} Q_2\subst{\abs{x}{\map{S}^x}}{X}
%			\]
%			and $\forall s'$, %such that
%			\begin{eqnarray*}
%				\Gamma; \emptyset; \Delta_1'' \proves \newsp{\tilde{s}}{P_2 \Par \context{C}{P \subst{s'}{x}}} \hastype \Proc \\
%				\Gamma; \emptyset; \Delta_2'' \proves \newsp{\tilde{s}}{Q_2 \Par \context{C}{Q \subst{s'}{x}}} \hastype \Proc
%			\end{eqnarray*}
%			and
%			\[
%				\Gamma; \emptyset; \Delta_1 \cat s: S_2\ \mathcal{R}\ \Delta_2'' \proves P_2  \subst{\abs{x}{\map{S}^x}}{X}\ \mathcal{R}\ 
%				Q_2 \subst{\abs{x}{\map{S}^x}}{X}
%			\]

		\item	$\forall \news{\tilde{s_1}} \bactout{n}{\tilde{m_1}}$ such that
			\[
				\Gamma; \emptyset; \Delta_1 \hby{\news{\tilde{s_1}} \bactout{n}{\tilde{m_1}}} \Delta_1' \proves P_1 \hby{\news{\tilde{s_1}} \bactout{n}{\tilde{m_1}}} P_2
			\]
			%with $s_1: S \in \Delta_1 \vee (\dual{s_2}: S' \in \Delta_2 \wedge S \dualof S')$
			implies that $\exists Q_2, \tilde{m_2}$ such that
			\[
				\Gamma; \emptyset; \Delta_2 \Hby{\news{\tilde{s_2}} \bactout{n}{\tilde{m_2}}} \Delta_2' \proves Q_1 \Hby{\news{\tilde{s_2}} \bactout{n}{\tilde{m_2}}} Q_2
			\]
			%such that
%			\begin{eqnarray*}
%				\Gamma; \emptyset; \Delta_1'' \proves \newsp{\tilde{s}}{P_2 \Par \context{C}{P \subst{s'}{x}}} \hastype \Proc \\
%				\Gamma; \emptyset; \Delta_2'' \proves \newsp{\tilde{s}}{Q_2 \Par \context{C}{Q \subst{s'}{x}}} \hastype \Proc
%			\end{eqnarray*}
			and for fresh $t$
			\[
				\Gamma; \emptyset; \Delta_1''\ \mathcal{R}\ \Delta_2'' \proves
				\newsp{\tilde{s_1}}{P_2 \Par \fotrigger{t}{\tilde{m_1}}}  % \map{S}^{s_1}}
				\ \mathcal{R}\ 
				\newsp{\tilde{s_2}}{Q_2 \Par \fotrigger{t}{\tilde{m_2}}} %\map{S}^{s_2}}
			\]

		\item	$\forall \ell \notin \set{\news{\tilde{s}} \bactout{n}{m}, \news{\tilde{m}} \bactout{n}{\abs{x}{P}}}$ such that
			\[
				\Gamma; \emptyset; \Delta_1 \hby{\ell} \Delta_1' \proves P_1 \hby{\ell} P_2
			\]
			implies that $\exists Q_2$ such that 
			\[
				\Gamma; \emptyset; \Delta_2 \Hby{\hat{\ell}} \Delta_2' \proves Q_1 \Hby{\hat{\ell}} Q_2
			\]
			and
			$\Gamma; \emptyset; \Delta_1\ \mathcal{R}\ \Delta_2 \proves P_2\ \mathcal{R}\ Q_2$.

		\item	The symmetric cases of 1, 2 and 3.
	\end{enumerate}
	The Knaster Tarski theorem ensures that the largest bisimulation exists, it is called bisimilarity and is denoted by $\wb$.
\end{definition}

The next theorem justifies our definition choices
for both of the bisimulation relations, since
they coincide between them and they also
coincide with reduction closed, barbed congruence.

\begin{theorem}[Coincidence]\rm
	\label{the:coincidence}
	Relations $\wb, \wbc$ and $\cong$ coincide.
\end{theorem}

\begin{proof}
	The full details of the proof are in Appendix~\ref{app:sub_coinc}.
	The theorem is split into a hierarchy of Lemmas. Specifically
	Lemma~\ref{lem:wb_is_wbc} exploits a process substitution result
	(Corollary~\ref{cor:subst_equiv}) to prove that $\wb \subseteq \wbc$.
	Lemma~\ref{lem:wbc_is_cong} shows that $\wbc$ is a congruence
	which implies $\wbc \subseteq \cong$.
	The final result comes from Lemma~\ref{lem:cong_is_wb} where
	we use label testing to show that $\cong \subseteq \wb$ using
	the technique in developed in~\cite{Hennessy07}. The formulation of input
	triggers in the bisimulation relation allows us to prove
	the latter result without using a matching operator.
	\qed
\end{proof}

Session types are inherintly $\tau$-inert.

\begin{lemma}[$\tau$-inertness]\rm
	\label{lem:tau_inert}
	Let $P$ an $\HOp$ process
	and $\Gamma; \emptyset; \Delta \proves P \hastype \Proc$
	\begin{enumerate}
		\item	If $P \red P'$ then $\Gamma; \es; \Delta \wb \Delta' \proves P \wb P'$.
		\item	If $P \red^* P'$ then $\Gamma; \es; \Delta \wb \Delta' \proves P \wb P'$.
	\end{enumerate}
\end{lemma}

\begin{proof}
	The proof is relied on the fact that processes of the
	form $\Gamma; \es; \Delta \bout{s}{V} P_1 \Par \binp{k}{x} P_2$
	cannot have any typed transition observables and the fact
	that bisimulation is a congruence.
	See details in Appendix~\ref{app:sub_tau_inert}.
	\qed
\end{proof}


