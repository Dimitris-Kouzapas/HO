\section{Type Soundness}
\label{app:ts}

We state type soundness of our system.
As our typed process framework is a sub-calculus of that considered
by Mostrous and Yoshida, the proof of type soundness requires notions
and properties which are specific instances of those already shown in~\cite{tlca07}.
We begin by stating weakening and strengthening lemmas,
which have standard proofs.

%%% Weakening
\begin{lemma}[Weakening - Lemma C.2 in M\&Y]\rm
	\label{l:weak}
	\begin{enumerate}[$-$]
		\item	If $\Gamma; \Lambda; \Delta \proves P \hastype \Proc$
			and
			$X \not\in \dom{\Gamma,\Lambda,\Delta}$
			then
			$\Gamma\cat X: \shot{S}; \Lambda; \Delta \proves P \hastype \Proc$ 
	\end{enumerate}
\end{lemma}

\begin{lemma}[Strengthening - Lemma C.3 and C.4 in M\&Y]\rm
	\label{l:stren}
	\begin{enumerate}[$-$]
		\item	If $\Gamma \cat X: \shot{S}; \Lambda; \Delta \proves P \hastype \Proc$
			and
			$X \not\in \fpv{P}$ then
			$\Gamma; \Lambda; \Delta \proves P \hastype \Proc$

		\item	If $\Gamma; \Lambda; \Delta \cat k: \tinact \proves P \hastype \Proc$
			and
			$k \not\in \fn{P}$
			then
			$\Gamma; \Lambda; \Delta \proves P \hastype \Proc$
	\end{enumerate}
\end{lemma}

\begin{lemma}[Substitution Lemma - Lemma C.10 in M\&Y]\rm
	\label{l:subst}
	\begin{enumerate}[1.]
		\item	Suppose
			$\Gamma; \Lambda; \Delta \cat x:S \proves P \hastype \Proc$
			and
			$k \not\in \dom{\Gamma, \Lambda, \Delta}$.
			Then
			$\Gamma; \Lambda; \Delta \cat k:S  \vdash P\subst{k}{x} \hastype \Proc$.

		\item	Suppose $\Gamma \cat x:\chtype{S}; \Lambda; \Delta \proves P \hastype \Proc$
			and
			$a \not\in \dom{\Gamma, \Lambda, \Delta}$. 
			Then
			$\Gamma \cat a:\chtype{S}; \Lambda; \Delta \vdash P\subst{a}{x} \hastype \Proc$.

		\item	Suppose $\Gamma; \Lambda_1 \cat X:\lhot{S}; \Delta_1  \proves P \hastype \Proc$ 
			and
			$\Gamma; \Lambda_2; \Delta_2  \proves V \hastype \lhot{S}$
			with 
			$\Lambda_1, \Lambda_2$ and $\Delta_1, \Delta_2$
			defined.  
			Then
			$\Gamma; \Lambda_1 \cat \Lambda_2; \Delta_1 \cat \Delta_2  \proves P\subst{V}{X} \hastype \Proc$.

		\item	Suppose
			$\Gamma \cat X:\shot{S}; \Lambda; \Delta  \proves P \hastype \Proc$
			and
			$\Gamma; \emptyset ; \emptyset  \proves V \hastype \shot{S}$.
			Then $\Gamma; \Lambda; \Delta  \proves P\subst{V}{X} \hastype \Proc$.
	\end{enumerate}
\end{lemma}

\begin{proof}
	In all four parts, we proceed by induction on the typing for $P$,
	with a case analysis on the last applied rule. 
	Parts (1) and (2) are standard and omitted. 

	In Part (3), we content ourselves by detailing only the case in
	which the last applied rule is \trule{App}. 
	Then we have $P = \appl{X}{k}$ and. By typing inversion on the first assumption we infer that
	$\Lambda_1 = \emptyset$, $\Delta_1 = \{k : S\}$, and also
%
	\begin{eqnarray}
		\Gamma; \{X : \lhot{S} \} ; \emptyset & \proves &  X \hastype \shot{S} \nonumber \\
		\Gamma; \emptyset; \{k : S\} & \proves & k \hastype S  \label{eq:subseq1}
	\end{eqnarray}
%
	By inversion on the second assumption we infer that either
	(i)\,$V = Y$ (for some process variable $Y$) or 
	(ii)\,$V = (z)Q$, for some $Q$ such that
%
	\begin{equation}
		\Gamma; \Lambda_2 ; \Delta_2 \cat z:S  \proves Q \hastype \Proc \label{eq:subseq2}\\
	\end{equation}
%
	In possibility\,(i), we have a simple substitution on process variables and the thesis follows easily. 
	In possibility\,(ii), we observe that $P\subst{V}{X} = \appl{X}{k}\subst{(z)Q}{X} = Q\subst{k}{z}$.
	The thesis then follows by using Lemma~\ref{l:subst}\,(1) with \eqref{eq:subseq1} and \eqref{eq:subseq2} above to infer 
%
	\begin{equation*}
		\Gamma; \Lambda_2 ; \Delta_2 \cat k:S  \proves Q \subst{k}{z} \hastype \Proc .
	\end{equation*}
%
	The proof of Part (4) follows similar lines as that of Part (3).
	\qed
\end{proof}

\begin{definition}[Well-typed Session Environment]\rm
	Let $\Delta$ a session environment.
	We say that $\Delta$ is {\em well-typed} if whenever
	$s: S_1, \dual{s}: S_2 \in \Delta$ then $S_1 \dualof S_2$.
\end{definition}

\begin{definition}[Session Environment Reduction]\rm
	We define relation $\red$ on session environments as:
	\begin{itemize}
		\item	$\Delta \cat s: \btout{U} S_1 \cat \dual{s}: \btinp{U} S_2 \red \Delta \cat s: S_1 \cat \dual{s}: S_2$
		\item	$\Delta \cat s: \btsel{l_i: S_i}_{i \in I} \cat \dual{s}: \btbra{l_i: S_i'}_{i \in I} \red \Delta \cat s: S_j \cat \dual{s}: S_j', \quad (j \in I)$.
	\end{itemize}
\end{definition}

We now state the instance of type soundness that we
can derived from the Mostrous and Yoshida system.
It is worth noticing that M\&Y have a slightly richer
definition of structural congruence.
Also, their statement for subject reduction relies on an 
ordering on typings associated to queues and other 
runtime elements (such extended typings are denoted $\Delta$ by M\&Y).
Since we are synchronous we can omit such an ordering.

We now repeat the statement of
Theorem~\ref{t:sr} in Page~\pageref{t:sr}:

\begin{theorem}[Type Soundness - Theorem~\ref{t:sr}]\rm
	\begin{enumerate}[1.]
		\item	(Subject Congruence) Suppose $\Gamma; \Lambda; \Delta \proves P \hastype \Proc$.
			Then $P \scong P'$ implies $\Gamma; \Lambda; \Delta \proves P' \hastype \Proc$.

		\item	(Subject Reduction) Suppose $\Gamma; \emptyset; \Delta \proves P \hastype \Proc$
			with
%			$\mathsf{balanced}(\Delta)$. 
			well-typed $\Delta$.
			Then $P \red P'$ implies $\Gamma; \emptyset; \Delta_2  \proves P' \hastype \Proc$
			and $\Delta_1 \red \Delta_2$ or $\Delta_1 = \Delta_2$.
	\end{enumerate}
\end{theorem}

\begin{proof}
	Part (1) is standard, using weakening and strengthening lemmas. Part (2) proceeds by induction on the last reduction rule used:
	\begin{enumerate}[$-$]
		\item	Case \orule{NPass}:
			Then there are two sub-cases, depending on whether the
			communicated name is a shared name or a channel, that is, i.e., $U = S$ or $U = \chtype{S}$.
			The proof is standard, using appropriate rules for typing reception of a name: when $U = S$ we use typing 
			rule~\trule{RecvS} and Lemma~\ref{l:subst}(1); when $U = \chtype{S}$ we use typing 
			rule~\trule{RecvShN} and Lemma~\ref{l:subst}(2). The session environment reduces.

		\item	Case \orule{APass}:
			Then there are two sub-cases, depending on whether the
			communicated process abstraction is a shared or linear, that is, i.e., $U = \shot{S}$ or $U = \lhot{S}$.
			The proof is standard, using appropriate rules for typing reception of a process abstraction: 
			when $U = \lhot{S}$ we use typing rule~\trule{RecvL} and Lemma~\ref{l:subst}(3); 
			when $U = \shot{S}$ we use typing rule~\trule{RecvSh} and Lemma~\ref{l:subst}(4). The session environment reduces.

		\item	Case \orule{Sel}:
			The proof is standard, the session environment reduces.

		\item	Case \orule{Sess}:
			The proof is standard, exploiting induction hypothesis.
			The session environment may remain invariant (channel restriction)  or reduce (name restriction).

		\item	Case \orule{Par}:
			The proof is standard, exploiting induction hypothesis. 

		\item	Case \orule{Cong}:
			follows from Theorem~\ref{t:sr}\,(1).
	\end{enumerate}
	\qed
\end{proof}
