% !TEX root = ../main.tex
\section{Type Soundness}
\label{app:ts}

We state type soundness of our system.
As our typed process framework is a sub-calculus of that considered
by Mostrous and Yoshida, the proof of type soundness requires notions
and properties which are specific instances of those already shown in~\cite{tlca07}.
We begin by stating weakening and strengthening lemmas,
which have standard proofs.

%%% Weakening
\begin{lemma}[Weakening - Lemma C.2 in M\&Y]\rm
	\label{l:weak}
	\begin{enumerate}[$-$]
		\item	If $\Gamma; \Lambda; \Delta \proves P \hastype \Proc$
			and
			$X \not\in \dom{\Gamma,\Lambda,\Delta}$
			then
			$\Gamma\cat X: \shot{S}; \Lambda; \Delta \proves P \hastype \Proc$ 
	\end{enumerate}
\end{lemma}

\begin{lemma}[Strengthening - Lemma C.3 and C.4 in M\&Y]\rm
	\label{l:stren}
	\begin{enumerate}[$-$]
		\item	If $\Gamma \cat X: \shot{S}; \Lambda; \Delta \proves P \hastype \Proc$
			and
			$X \not\in \fpv{P}$ then
			$\Gamma; \Lambda; \Delta \proves P \hastype \Proc$

		\item	If $\Gamma; \Lambda; \Delta \cat k: \tinact \proves P \hastype \Proc$
			and
			$k \not\in \fn{P}$
			then
			$\Gamma; \Lambda; \Delta \proves P \hastype \Proc$
	\end{enumerate}
\end{lemma}

\begin{lemma}[Substitution Lemma - Lemma C.10 in M\&Y]\rm
	\label{l:subst}
	\begin{enumerate}[1.]
		\item	Suppose $\Gamma; \Lambda; \Delta \cat \mytilde{x}:\mytilde{S}  \proves P \hastype \Proc$ and
			$\mytilde{k} \not\in \dom{\Gamma, \Lambda, \Delta}$. 
			Then $\Gamma; \Lambda; \Delta \cat \mytilde{k}:\mytilde{S}  \vdash P\subst{\mytilde{k}}{\mytilde{x}} \hastype \Proc$.

		\item	Suppose $\Gamma \cat x:\chtype{U}; \Lambda; \Delta \proves P \hastype \Proc$ and
			$a \notin \dom{\Gamma, \Lambda, \Delta}$. 
			Then $\Gamma \cat a:\chtype{U}; \Lambda; \Delta   \vdash P\subst{a}{x} \hastype \Proc$.

		\item	Suppose $\Gamma; \Lambda_1 \cat X:\lhot{\mytilde{C}}; \Delta_1  \proves P \hastype \Proc$ 
			and $\Gamma; \Lambda_2; \Delta_2  \proves V \hastype \lhot{\mytilde{C}}$ with 
			$\Lambda_1, \Lambda_2$ and $\Delta_1, \Delta_2$ defined.  
			Then $\Gamma; \Lambda_1 \cat \Lambda_2; \Delta_1 \cat \Delta_2  \proves P\subst{V}{X} \hastype \Proc$.

		\item	Suppose $\Gamma \cat X:\shot{\mytilde{C}}; \Lambda; \Delta  \proves P \hastype \Proc$ and
			$\Gamma; \emptyset ; \emptyset  \proves V \hastype \shot{\mytilde{C}}$.
			Then $\Gamma; \Lambda; \Delta  \proves P\subst{V}{X} \hastype \Proc$.
		\end{enumerate}
\end{lemma}

\begin{proof}
	In all four parts, we proceed by induction on the typing for $P$,
	with a case analysis on the last applied rule. 
	Parts (1) and (2) are standard and therefore omitted. 

	In Part (3), we content ourselves by detailing only the case in
	which the last applied rule is \trule{App}. 
	Then we have $P = \appl{X}{k}$ and. By typing inversion on the first assumption we infer that
	$\Lambda_1 = \emptyset$, $\Delta_1 = \{k : S\}$, and also
%
	\begin{eqnarray}
		\Gamma; \{X : \lhot{S} \} ; \emptyset & \proves &  X \hastype \shot{S} \nonumber \\
		\Gamma; \emptyset; \{k : S\} & \proves & k \hastype S  \label{eq:subseq1}
	\end{eqnarray}
%
	By inversion on the second assumption we infer that either
	(i)\,$V = Y$ (for some process variable $Y$) or 
	(ii)\,$V = (z)Q$, for some $Q$ such that
%
	\begin{equation}
		\Gamma; \Lambda_2 ; \Delta_2 \cat z:S  \proves Q \hastype \Proc \label{eq:subseq2}\\
	\end{equation}
%
	In possibility\,(i), we have a simple substitution on process variables and the thesis follows easily. 
	In possibility\,(ii), we observe that $P\subst{V}{X} = \appl{X}{k}\subst{(z)Q}{X} = Q\subst{k}{z}$.
	The thesis then follows by using Lemma~\ref{l:subst}\,(1) with \eqref{eq:subseq1} and \eqref{eq:subseq2} above to infer 
%
	\begin{equation*}
		\Gamma; \Lambda_2 ; \Delta_2 \cat k:S  \proves Q \subst{k}{z} \hastype \Proc .
	\end{equation*}
%
	The proof of Part (4) follows similar lines as that of Part (3).
	\qed
\end{proof}

%\begin{definition}[Well-typed Session Environment]%\rm
%	Let $\Delta$ be a session environment.
%	We say that $\Delta$ is {\em well-typed} if whenever
%	$s: S_1, \dual{s}: S_2 \in \Delta$ then $S_1 \dualof S_2$.
%\end{definition}
%
%\begin{definition}[Session Environment Reduction]%\rm
%	We define the relation $\red$ on session environments as:
%	\begin{enumerate}[$-$]
%		\item	$\Delta \cat s: \btout{U} S_1 \cat \dual{s}: \btinp{U} S_2 \red \Delta \cat s: S_1 \cat \dual{s}: S_2$
%		\item	$\Delta \cat s: \btsel{l_i: S_i}_{i \in I} \cat \dual{s}: \btbra{l_i: S_i'}_{i \in I} \red \Delta \cat s: S_k \cat \dual{s}: S_k', \quad k \in I$.
%	\end{enumerate}
%\end{definition}

We now state the instance of type soundness that we
can derive from the Mostrous and Yoshida system.
It is worth noticing that M\&Y have a slightly richer
definition of structural congruence.
Also, their statement for subject reduction relies on an 
ordering on typings associated to queues and other 
runtime elements (such extended typings are denoted $\Delta$ by M\&Y).
Since we are working with synchronous communication we can omit such an ordering.

We now repeat the statement of
Theorem~\ref{t:sr} in Page~\pageref{t:sr}:

\begin{theorem}[Type Soundness - Theorem~\ref{t:sr}]%\rm
	\begin{enumerate}[1.]
		\item	(Subject Congruence) Suppose $\Gamma; \Lambda; \Delta \proves P \hastype \Proc$.
			Then $P \scong P'$ implies $\Gamma; \Lambda; \Delta \proves P' \hastype \Proc$.

		\item	(Subject Reduction) Suppose $\Gamma; \es; \Delta \proves P \hastype \Proc$
			with
			well-typed $\Delta$. \\
			Then $P \red P'$ implies $\Gamma; \es; \Delta'  \proves P' \hastype \Proc$
			and $\Delta = \Delta'$ or $\Delta \red \Delta'$.
	\end{enumerate}
\end{theorem}

\begin{proof}
	Part (1) is standard, using weakening and strengthening lemmas. Part (2) proceeds by induction on the last reduction rule used. Below, we give some details:
	\begin{enumerate}[1.]
		\item	Case \orule{NPass}:
			Then there are two sub-cases, depending on whether the
			communication subject is a shared name or a channel. 
			In the first case, we have 
			$$P = \bout{k}{n} P_1 \Par \binp{k}{x} P_2 \red P_1 \Par P_2\subst{n}{x} = P'$$ 
			Suppose $\Gamma; \es; \Delta  \proves \bout{k}{n} P_1 \Par \binp{k}{x} P_2 \hastype \Proc$. This assumption is derived first from rules~\trule{Req} and \trule{AccS}:
			\[
								\tree{
					\Gamma; \emptyset; \emptyset  \proves  k \hastype \chtype{S} ~~~
					\Gamma ; \emptyset ; \Delta_1 \proves   P_1 \hastype \Proc ~~~
					\Gamma ; \emptyset ; \{n:T\} \proves   n \hastype T ~~~
					 S \dualof T
					}{
					\Gamma; \emptyset; \Delta_1 \cat \{n:T\}    \proves  
 					\bout{k}{n} P_1 \hastype \Proc} 
			\]		
			and
			\[		~~ 
				\tree{
					\Gamma; \emptyset; \emptyset  \proves  k \hastype \chtype{S} \quad 
					\Gamma ; \emptyset ; \Delta_3, x:S  \proves  P_2 \hastype \Proc 
					}{
					\Gamma; \emptyset; \Delta_3 \proves  
 					\binp{k}{x} P_2 \hastype \Proc} 
			\]
			and then rule~\trule{Par}, letting $\Delta = \Delta_1 \cat \{n:T\}  \cat \Delta_3$.
			
			Now, by applying Lemma~\ref{lem:subst}(1) on $\Gamma ; \emptyset ; \Delta_3, x:S  \proves  P_2 \hastype \Proc$
			we obtain 
			$$\Gamma ; \emptyset ; \Delta_3, n:S  \proves  P_2\subst{n}{x} \hastype \Proc$$
			and the case is completed by using rule~\trule{Par} with this judgment and
			$\Gamma ; \emptyset ; \Delta_1 \proves   P_1 \hastype \Proc$. NOT QUITE. \\
			
			In the second case we have the following reduction, with   $|\mytilde{h}| = |\mytilde{x}|$:
			$$P = \bout{k}{\mytilde{h}} P_1 \Par \binp{k}{\mytilde{x}} P_2 \red P_1 \Par P_2\subst{\mytilde{h}}{\mytilde{x}} = P'$$ 
			Also in this case the proof is standard, using rules~\trule{RcvS}, \trule{Send}, and \trule{Par} 
			to type $P$, and using Lemma~\ref{lem:subst}(1) and rule~\trule{Par} to type $P'$. 
			The session environment $\Delta$ reduces.

		\item	Case \orule{APass}:
		Then we have
		$$
		P = \bout{n}{\abs{\mytilde{x}}{Q}} P_1 \Par \binp{\dual{n}}{\X} P_2  \red  P_1 \Par P_2 \subst{\abs{\mytilde{x}}{Q}}{\X} = P'
		$$
		and we distinguish two cases, associated to the type of the higher-order value $\abs{\tilde{x}}{Q}$.
		We describe the proof for the case in which the type is $\lhot{\mytilde{C}}$; the proof when 
		the type is $\shot{\mytilde{C}}$ is analogous.
		The typing of $P$ proceeds first by using rule~\trule{Send} on the left-hand side:
		\[
								\tree{
					\Gamma;\, \emptyset;\, \Delta_1 \proves  P_1 \hastype \Proc \quad
					\Gamma ;\, \emptyset ;\, \Delta_2 \proves   \abs{\mytilde{x}}{Q} \hastype \lhot{\mytilde{C}}					}{
					\Gamma;\, \emptyset;\, \big((\Delta_1 \cat \Delta_2) \setminus \{n:S\}\big) \cat n:\btout{\lhot{\mytilde{C}}} S     \proves  
 					\bout{n}{\abs{\mytilde{x}}{Q}} P_1 \hastype \Proc} 
			\]	
			Then, by rule~\trule{RcvH} on the right-hand side we have (assuming $S \dualof T$):
					\[
					\tree{
					\Gamma;\, X:\lhot{\mytilde{C}} ;\, \Delta_3 \cat n:T \proves  P_2 \hastype \Proc \quad
					\Gamma ;\, \{X:\lhot{\mytilde{C}} \} ;\, \Delta_4 \proves   X \hastype \lhot{\mytilde{C}}					}{
					\Gamma;\, \emptyset;\, \Delta_3 \setminus \Delta_4 \cat n:\btinp{\lhot{\mytilde{C}}} T     \proves  
 					\binp{n}{X} P_2 \hastype \Proc} 
			\]	
			Finally, we use rule~\trule{Par} to obtain the typing for $P$.
			The typing of $P'$ is obtained by using the appropriate substitution lemma (Lemma~\ref{lem:subst}(3)) on the typing for $P_2$.
			
			When the type of the higher-order value is $\shot{\mytilde{C}}$,
			the use of rules~\trule{Send} and~\trule{RcvH} for typing $P$ is similar; 
			 one would use Lemma~\ref{lem:subst}(4) to type $P'$. The session environment reduces.

		\item	Case \orule{Sel}:
			The proof is standard, the session environment reduces.

		\item	Case \orule{Sess}:
			The proof is standard, exploiting induction hypothesis.
			The session environment may remain invariant (channel restriction)  or reduce (name restriction).

		\item	Case \orule{Par}:
			The proof is standard, exploiting induction hypothesis. 

		\item	Case \orule{Cong}:
			follows from Theorem~\ref{t:sr}\,(1).
	\end{enumerate}
	\qed
\end{proof}
