% !TEX root = main.tex
\newpage
\section{Session Types for $\HOp$}
\label{s:types}

In this section we define a session type system for
$\HOp$ and establish its main properties. We use as
a reference the type system for higher-order session processes 
developed by Mostrous and Yoshida~\cite{tlca07}.
Our system is simpler than that in~\cite{tlca07}, in order to distill the key
features of higher-order communication in a session-typed setting.

%%%%%%%%%%%%%%%%%%%%%%%%%%%%%%%%%%%%%%%%%%%%%%%%%%
%  SYNTAX FOR TYPES
%%%%%%%%%%%%%%%%%%%%%%%%%%%%%%%%%%%%%%%%%%%%%%%%%%

\subsection{Syntax}
%We write $\Proc$ to denote the type of processes.
The type syntax for values and sessions
(denoted $U$ and $S$, respectively) is as follows:
%
\[
\begin{array}{cl}
	C \bnfis &	S \bnfbar \chtype{S}
	\\

	L \bnfis &	\shot{\tilde{C}} \bnfbar \lhot{\tilde{C}}
	\\

	U \bnfis &	\tilde{C} \bnfbar L \bnfbar \chtype{L} % \bnfbar \Proc
	\\

	S,T \bnfis & 	\btout{U} S \bnfbar \btinp{U} S
			\bnfbar \btsel{l_i:S_i}_{i \in I} \bnfbar \btbra{l_i:S_i}_{i \in I}
			\bnfbar \trec{t}{S} \bnfbar \vart{t}  \bnfbar \tinact
\end{array}
\]
%
\noi
%We give the polyadic definition of our type system since
%polyadic semantics subsumes monadic semantics.
We write $C$ to denote first-order channel types; session types $S$
for session channels and shared types $\chtype{S}$ for shared channels.
We write $L$ to denote types for abstractions (higher-order values): % In particular, 
$\shot{\tilde{C}}$ and $\lhot{\tilde{C}}$ denote
shared and linear abstraction types, respectively.
%Types for values are denoted with $U$ that denotes
%channel types, abstraction types and process types.
We write $U$ to denote types for values (polyadic first-order values and 
higher-order values).
The shared channel type in $U$ is extended
to include shared names that carry abstractions.
Note that our type syntax disallows types of
the form $\chtype{\chtype{U}}$, $\chtype{\tilde{U}}$
and $\shot{(\shot{U})}$,
meaning that shared names cannot carry shared names or
polyadic values and
we that do not apply abstractions over abstractions.
This is a main difference with respect to the type structure in~\cite{tlca07}, which 
supports arrow types 
$U \sharedop T$ and 
$U \lollipop T$, where $T$ denotes an arbitrary term (i.e., a value or a process).


We write $S$ to denote usual (binary) session types with recursion, here including 
polyadic communication. 
Session type $\btout{U} S$ is assigned to a channel that first sends a value of type $U$ and then follows
the protocol described by $S$.
Dually, type $\btinp{U} S$ is assigned to a channel  
that first receives a value of type $U$ and then continues as $S$. 
Session types for labeled choice and selection are standard: they are 
written $\btbra{l_i:S_i}_{i \in I}$ and $\btsel{l_i:S_i}_{i \in I}$ , respectively. 
Recursive types are also as customary, considering usual 
well-formedness conditions.

%Our type structure is a subset of that defined by Mostrous and Yoshida in~\cite{tlca07}.
%In particular, we focus on having higher-order values with types
%$\shot{\tilde{C}}$ and $\lhot{\tilde{C}}$, in contrast with the the type structure
%in~\cite{tlca07} that supports general functions
%$U \sharedop T$ and 
%$U \lollipop T$, where $T$ denotes a term, i.e., a value or a process.

%%%%%%%%%%%%%%%%%%%%%%%%%%%%%%%%%%%%%%%%%%%%%%%%%%
%  DUALITY
%%%%%%%%%%%%%%%%%%%%%%%%%%%%%%%%%%%%%%%%%%%%%%%%%%

\subsection{Duality}
The following definition establishes the key notion of duality on session types.
%
\begin{definition}[Duality]\rm
Let $\mathsf{ST}$ denote the set of well-formed session types.
The function $F(R): \mathsf{ST} \longrightarrow \mathsf{ST}$ is defined as:

	\begin{tabular}{rcl}
		$F(R)$ &$=$&		$\set{(\tinact, \tinact), (\vart{t}, \vart{t})}$\\
			&$\cup$&	$\set{(\btout{U} S, \btinp{U} T)\,,\, (\btinp{U} S, \btout{U} T) \bnfbar S\ R\ T}$\\
			&$\cup$&	$\set{(\btsel{l_i: S_i}_{i \in I} \,,\, \btbra{l_j: T_j}_{j \in J}) \bnfbar I \subseteq J, S_i\ R\ T_j}$\\
			&$\cup$&	$\set{(\btbra{l_i: S_i}_{i \in I}\,,\, \btsel{l_j: T_j}_{j \in J}) \bnfbar J \subseteq I, S_j\ R\ T_i}$\\
			&$\cup$&	$\set{(\trec{t}{S}\,,\, \trec{t}{T}) \cup (S \subst{\trec{t}{S}}{\vart{t}}\,,\, T) \cup (S\,,\, T\subst{\trec{t}{T}}{\vart{t}}) \bnfbar S\ R\ T)}$
	\end{tabular}
	
\noindent
	Standard arguments ensure that $F$ is monotone, thus the greatest fixed point
	of $F$ exists. Let duality be defined as $\dualof = \nu X. F(X)$.
\end{definition}
%

%%%%%%%%%%%%%%%%%%%%%%%%%%%%%%%%%%%%%%%%%%%%%%%%%%
%  TYPING SYSTEM
%%%%%%%%%%%%%%%%%%%%%%%%%%%%%%%%%%%%%%%%%%%%%%%%%%


\subsection{Environments and Judgments}
%Following our decision of focusing on functions
%$\shot{U}$ and $\lhot{U}$, our environments are
%simpler than those in~\cite{tlca07}:
%
Following~\cite{tlca07}, 
our typing discipline relies on the following typing environments: %We consider the
\[
\begin{array}{llcl}
	\text{Linear} & \Lambda & \bnfis & \emptyset \bnfbar \Lambda \cat \varp{X}: \lhot{\tilde{C}}\\
	\text{Session} & \Delta & \bnfis & \emptyset \bnfbar \Delta \cat k:S \\
	\text{Shared} & \Gamma & \bnfis & \emptyset \bnfbar \Gamma \cat \varp{X}: \shot{\tilde{C}} \bnfbar \Gamma \cat k: \chtype{S} \bnfbar \Gamma \cat k: \chtype{L} \bnfbar \Gamma \cat \rvar{X}: \Delta
\end{array}
\]
%
\noi We define typing judgements for values $V$
and processes $P$: % following~\cite{tlca07}:
%
\[	\begin{array}{c}
		\Gamma; \Lambda; \Delta \proves V \hastype U \qquad \qquad \qquad \Gamma; \Lambda; \Delta \proves P \hastype \Proc
	\end{array}
\]
%
\noi As expected, weakening, contraction, and exchange principles apply to $\Gamma$;
environments $\Lambda$ and $\Delta$ behave linearly, and are only subject to exchange.
We require that the domains of $\Gamma, \Lambda$ and $\Delta$ are pairwise distinct.
The first judgement states that under environment $\Gamma; \Lambda; \Delta$
values $V$ have type $U$,
whereas the second judgement states that under environment $\Gamma; \Lambda; \Delta$
process $P$ has the process type $\Proc$.

%%%%%%%%%%%
%	Type system description
%%%%%%%%%%%


\subsection{Typing Rules}
The typing system is defined in Figure~\ref{fig:typerulesmy}.
Types for session names/variables $k$ and
for linear process variables $X$ are
directly derived from the linear part of the typing
environment, i.e.~type maps $\Delta$ and $\Lambda$.
Rules $\trule{Session}$ and $\trule{LVar}$ require
a minimal linear environment in order to type
sessions $k$ with type $S$ and variables $X$ with
type $\shot{\tilde{C}}$. Rule $\trule{Shared}$
types shared names or shared variables $k$ with
type $U$ if the map $k:U$ exists in environment
$\Gamma$. Again the linear environment should
be minimal, i.e.,~empty.
The type $\shot{\tilde{C}}$ for shared values $V$
is derived using rule $\trule{Prom}$, where we require
a value with linear type to be typed without a linear
environment in order to promote shared type.
Rule $\trule{Derelic}$ allows to freely use a linear
type variable as shared type variable. 
A value consisting of a tuple of names/variables is typed using the $\trule{Pol}$ rule,
which requires theto type and combine each value in the tuple.
Abstraction values are typed with rule $\trule{Abs}$.
The key type for an abstraction is the type for
the bound variables of the abstraction, i.e.~for
bound variable type $\tilde{C}$ the abstraction
has type $\lhot{\tilde{C}}$.
The dual of abstraction typing is application typing
governed by rule $\trule{App}$, where we expect
the type $\tilde{C}$ of an application name $k$ 
to match the type $\lhot{\tilde{C}}$ or $\shot{\tilde{C}}$
of the application variable $X$.

A process prefixed with a session send operator $\bout{k}{V} P$
is typed using rule $\trule{Send}$.
The type $U$ of a send value $V$ should appear as a prefix
on the session type $\btout{U} S$ of $s$.
Rules $\trule{RecvS}$ and $\trule{RecvH}$ 
are for typing the 
reception of  session names $\binp{k}{\tilde{x}} P$
and abstractions $\binp{k}{X} P$, respectively.
In both rules type $U$ of a receive value should 
appear as a prefix on the session type $\btinp{U} S$ of $k$.
We use a similar approach with session prefixes
to type interaction between shared channels as defined 
in rules $\trule{Req}, \trule{AccS}$ and $\trule{AccH}$,
where the type of the sent/received object 
($U$, $\tilde{C}$ and $L$, respectively) should
match the type of the sent/received subject
($\chtype{U}$, $\chtype{\tilde{C}}$, and $\chtype{L}$, respectively).
In the case of rule $\trule{Req}$ we require
a duality condition for the communication of session names.
Select and branch prefixes are typed using the rules
$\trule{Sel}$ and $\trule{Bra}$ respectively. Both
rules prefix the session type with the selection
type $\btsel{l_i: S_i}_{i \in I}$ and
$\btbra{l_i:S_i}_{i \in I}$.

Creating and restricting a
 shared name $a$ requires to remove
its type from environment $\Gamma$ as defined in 
rule \trule{NewSh}. 
Creation of a session name $s$
creates two endpoints with dual types and restricts
them by removing them from the session environment
$\Delta$ as defined in rule \trule{NewS}. Rule
\trule{Par} combines the linear environment of
the parallel components of a parallel process
to create a type for the entire parallel process.
Disjointness of environmnts $\Lambda$ and $\Delta$
is implied. Rule \trule{Close} allows a form of weakening 
for the session environment $\Delta$, provided that
the names we add in $\Delta$ have the inactive
type $\tinact$. The inactive process $\inact$ has no
linear environment. The recursive variable is typed
directly from the shared environment $\Gamma$ as
in rule \trule{RVar}.
The recursive operator requires that the body of
a recursive process matches the type of the recursive
variable as in rule \trule{Rec}. 

\begin{figure}[!t]
\[
	\begin{array}{c}
%		\jrule{ }{\Gamma ; \emptyset; \emptyset \vdash \UnitV \hastype \Unit}{Unit} 
%		\qquad\quad  
		\trule{Session}~~\Gamma; \emptyset; \set{k:S} \proves k \hastype S 
		\\[2mm]
		\trule{Shared}~~\Gamma \cat a : \chtype{S}; \emptyset; \emptyset \proves a \hastype \chtype{S}
		\qquad
		\trule{LVar}~~\Gamma; \set{X: \lhot{S}}; \emptyset \proves X \hastype \lhot{S} 
		\\[2mm]
		\trule{Prom}~~\tree{
			\Gamma; \emptyset; \emptyset \proves V \hastype \lhot{S}
		}{
			\Gamma; \emptyset; \emptyset \proves V \hastype \shot{S}
		} 
		\qquad\quad  
		\trule{Derelic}~~\tree{
			\Gamma; \Lambda \cat X{:}\lhot{S}; \Sigma \proves P \hastype \Proc
		}{
			\Gamma \cat X:\shot{S}; \Lambda; \Sigma \proves P \hastype \Proc
		} 
		\\[4mm]
%		\trule{Subt}~~\tree{
%			\Gamma; \Lambda; \Sigma \proves P \hastype T \quad \Sigma \subt \Sigma' \quad T \subt T'
%		}{
%			\Gamma ; \Lambda; \Sigma' \vdash P \hastype T'
%		} 
%		\qquad\quad

		\trule{Abs}~~\tree{
			\Gamma; \Lambda; \Sigma \cat x: S \proves P \hastype \Proc
		}{
			\Gamma; \Lambda; \Sigma \proves \abs{x}{P} \hastype \lhot{S}
		}
		\\[4mm]

		\trule{App}~~\tree{(U = \lhot{S}) \lor (U = \shot{S}) \quad \Gamma; \Lambda_1; \Sigma_1 \proves X \hastype U  \quad \Gamma; \Lambda_2; \Sigma_2 \proves k \hastype S
		}{
			\Gamma; \Lambda_1 \cup \Lambda_2; \Sigma_1 \cup \Sigma_2 \proves \appl{X}{k} \hastype \Proc
		} 
		\\[4mm]

		\trule{Send}~~\tree{
			\Gamma; \Lambda_1; \Sigma_1 \proves P \hastype \Proc  \quad \Gamma; \Lambda_2; \Sigma_2 \vdash V \hastype U  \quad (k:S \in \Sigma_1 \cup \Sigma_2)
		}{
			\Gamma; \Lambda_1 \cup \Lambda_2; (\Sigma_1 \cup \Sigma_2)\backslash\set{k:S} \cat k:\btout{U} S \proves \bout{k}{V} P \hastype \Proc
		}

		\\[4mm]
		\trule{Conn}~~\tree{
			\Gamma; \Lambda; \Sigma \cat x:S \proves P \hastype \Proc  \quad \Gamma; \emptyset; \emptyset \proves a \hastype \chtype{S}
		}{
			\Gamma; \Lambda; \Sigma \proves \binp{a}{x} P \hastype \Proc
		}
		\\[4mm]
%		\trule{ConnDual}~~\tree{
%			\Gamma; \Lambda; \Sigma \cat x: S_1 \proves P \hastype \Proc  \quad \Gamma; \emptyset; \emptyset \proves k \hastype \chtype{S_2} \quad S_1 \dualof S_2
%		}{
%			\Gamma; \Lambda; \Sigma \proves \bout{k}{x} P \hastype \Proc
%		}
%		\\[4mm]

		\trule{ConnDual}~~\tree{
			\Gamma; \Lambda; \Sigma \cat \dual{s}: S_1 \proves P \hastype \Proc  \quad \Gamma; \emptyset; \emptyset \proves a \hastype \chtype{S_2} \quad S_1 \dualof S_2
		}{
			\Gamma; \Lambda; \Sigma  \proves \bout{a}{\dual{s}} P \hastype \Proc
		}

		\\[4mm]

		\trule{NewSh}~~\tree{
			\Gamma\cat a:\chtype{S} ; \Lambda; \Sigma \proves P \hastype \Proc
		}{
			\Gamma; \Lambda; \Sigma \proves \news{a} P \hastype \Proc}
		\qquad\quad
		\trule{NewSes}~~\tree{
			\Gamma; \Lambda; \Sigma \cat s:S_1 \cat \dual{s}: S_2 \proves P \hastype \Proc \quad S_1 \dualof S_2
		}{
			\Gamma; \Lambda; \Sigma \proves \news{s} P \hastype \Proc
		}
		\\[4mm]

		\trule{RecvS}~~\tree{
			\Gamma; \Lambda; \Sigma \cat k: S_1 \cat x: S_2 \proves P \hastype \Proc
		}{
			\Gamma; \Lambda; \Sigma, k: \btinp{S_2} S_1  \vdash \binp{k}{x}P \hastype \Proc
		}
		\quad\quad 
		\trule{RecvL}~~\tree{
			\Gamma; \Lambda \cat X: \lhot{S}; \Sigma \cat k: S_1  \proves P \hastype \Proc
		}{
			\Gamma; \Lambda; \Sigma \cat k:\btinp{\lhot{S}}S_1  \proves \binp{k}{X}P \hastype \Proc
		}
		\\[4mm]

		\trule{RecvSh}~~\tree{
			\Gamma \cat X: \shot{S}; \Lambda; \Sigma \cat k: S_1  \proves P \hastype \Proc
		}{
			\Gamma; \Lambda; \Sigma \cat k:\btinp{\shot{S}}S_1  \proves \binp{k}{X}P \hastype \Proc
		}
		\quad ~~
		\trule{RecvShN}~~\tree{
			\Gamma \cat x: \chtype{S}; \Lambda; \Sigma \cat k: S_1  \proves P \hastype \Proc
		}{
			\Gamma; \Lambda; \Sigma \cat k:\btinp{\chtype{S}}S_1  \proves \binp{k}{x}P \hastype \Proc
		}
		\\[4mm]
		\trule{Par}~~\tree{
			\Gamma; \Lambda_{1}; \Sigma_{1} \proves P_{1} \hastype \Proc \quad \Gamma; \Lambda_{2}; \Sigma_{2} \proves P_{2} \hastype \Proc
		}{
			\Gamma; \Lambda_{1} \cup \Lambda_2; \Sigma_{1} \cup \Sigma_2 \proves P_1 \Par P_2 \hastype \Proc
		}
		\qquad\quad
		\trule{Close}~~\tree{
			\Gamma; \Lambda; \Sigma  \proves P \hastype T \quad k \not\in \dom{\Gamma, \Lambda,\Sigma}
		}{
			\Gamma; \Lambda; \Sigma \cat k: \tinact  \proves P \hastype \Proc
		}
		\\[4mm]
		\trule{Bra}~~\tree{
			 \forall i \in I \quad \Gamma; \Lambda; \Sigma \cat k:S_i \proves P_i \hastype \Proc
		}{
			\Gamma; \Lambda; \Sigma \cat k: \btbra{l_i:S_i}_{i \in I} \proves \bbra{k}{l_i:P_i}_{i \in I}\hastype \Proc
		}
		\qquad\quad 
	 	\trule{Sel}~~\tree{
			\Gamma; \Lambda; \Sigma \cat k: S_j  \proves P \hastype \Proc \quad j \in I
		}{
			\Gamma; \Lambda; \Sigma \cat k:\btsel{l_i:S_i}_{i \in I} \proves \bsel{s}{l_j} P \hastype \Proc
		}
		\\[4mm]

		\trule{Nil}~~\Gamma; \emptyset; \emptyset \proves \inact \hastype \Proc
\qquad \quad
		\trule{Var}~~\tree{
	
		}{
			\Gamma \cat \rvar{X}: \Sigma; \emptyset; \emptyset  \proves \rvar{X} \hastype \Proc
		}
		\qquad\quad 
%	 	\trule{Rec}~~\tree{
%			\Gamma \cat \rvar{X}: \Sigma; \emptyset; \emptyset  \proves P \hastype \Proc
%		}{
%			\Gamma ; \emptyset; \emptyset  \proves \recp{X}{P} \hastype \Proc
%		}
%		\\[4mm]

	 	\trule{Rec}~~\tree{
			\Gamma \cat \rvar{X}: \Sigma; \emptyset; \Sigma  \proves P \hastype \Proc
		}{
			\Gamma ; \emptyset; \Sigma  \proves \recp{X}{P} \hastype \Proc
		}


	\end{array}
\]
\caption{Typing Rules for $\HOp$\label{fig:typerulesmy}}
\end{figure}

\subsection{Type Soundness}
We state results for type safety:
we report instances of more general statements already proved by
Mostrous and Yoshida in the asynchronous case.

\begin{lemma}[Substitution Lemma - Lemma C.10 in M\&Y]\rm
	\label{lem:subst}
	\begin{enumerate}[1.]
		\item	Suppose $\Gamma; \Lambda; \Delta \cat \widetilde{x}:\widetilde{S}  \proves P \hastype \Proc$ and
			$\widetilde{k} \not\in \dom{\Gamma, \Lambda, \Delta}$. 
			Then $\Gamma; \Lambda; \Delta \cat \widetilde{k}:\widetilde{S}  \vdash P\subst{\widetilde{k}}{\widetilde{x}} \hastype \Proc$.

		\item	Suppose $\Gamma \cat x:\chtype{U}; \Lambda; \Delta \proves P \hastype \Proc$ and
			$a \notin \dom{\Gamma, \Lambda, \Delta}$. 
			Then $\Gamma \cat a:\chtype{U}; \Lambda; \Delta   \vdash P\subst{a}{x} \hastype \Proc$.

		\item	Suppose $\Gamma; \Lambda_1 \cat X:\lhot{\widetilde{C}}; \Delta_1  \proves P \hastype \Proc$ 
			and $\Gamma; \Lambda_2; \Delta_2  \proves V \hastype \lhot{\widetilde{C}}$ with 
			$\Lambda_1, \Lambda_2$ and $\Delta_1, \Delta_2$ defined.  
			Then $\Gamma; \Lambda_1 \cat \Lambda_2; \Delta_1 \cat \Delta_2  \proves P\subst{V}{X} \hastype \Proc$.

		\item	Suppose $\Gamma \cat X:\shot{\widetilde{C}}; \Lambda; \Delta  \proves P \hastype \Proc$ and
			$\Gamma; \emptyset ; \emptyset  \proves V \hastype \shot{\widetilde{C}}$.
			Then $\Gamma; \Lambda; \Delta  \proves P\subst{V}{X} \hastype \Proc$.
		\end{enumerate}
\end{lemma}

\begin{proof}
By induction on the typing for $P$, with a case analysis on the last used rule. 
\qed
\end{proof}

\begin{definition}[Well-typed Session Environment]%\rm
	Let $\Delta$ be a session environment.
	We say that $\Delta$ is {\em well-typed} if whenever
	$s: S_1, \dual{s}: S_2 \in \Delta$ then $S_1 \dualof S_2$.
\end{definition}

\begin{definition}[Session Environment Reduction]%\rm
	We define the relation $\red$ on session environments as:
	\begin{enumerate}[$-$]
		\item	$\Delta \cat s: \btout{U} S_1 \cat \dual{s}: \btinp{U} S_2 \red \Delta \cat s: S_1 \cat \dual{s}: S_2$
		\item	$\Delta \cat s: \btsel{l_i: S_i}_{i \in I} \cat \dual{s}: \btbra{l_i: S_i'}_{i \in I} \red \Delta \cat s: S_k \cat \dual{s}: S_k', \quad k \in I$.
	\end{enumerate}
\end{definition}

We now state the instance of type soundness that we can derived from the Mostrous and Yoshida system.
It is worth noticing that M\&Y have a slightly richer definition of structural congruence.
Also, their statement for subject reduction relies on an ordering on typings associated to queues and other 
runtime elements (such extended typings are denoted $\Delta$ by M\&Y).
Since we are synchronous we can omit such an ordering.

\begin{theorem}[Type Soundness - Theorem 7.3 in M\&Y]\label{t:sr}%\rm
	\begin{enumerate}[1.]
		\item	(Subject Congruence) Suppose $\Gamma; \es; \Delta \proves P \hastype \Proc$.
			Then $P \scong P'$ implies $\Gamma; \es; \Delta \proves P' \hastype \Proc$.

		\item	(Subject Reduction) Suppose $\Gamma; \es; \Delta \proves P \hastype \Proc$
			with
			well-typed $\Delta$. \\
			Then $P \red P'$ implies $\Gamma; \es; \Delta'  \proves P' \hastype \Proc$
			and $\Delta = \Delta'$ or $\Delta \red \Delta'$.
	\end{enumerate}
\end{theorem}

\begin{proof}
	See Appendix \ref{app:ts}.
	\qed
\end{proof}
