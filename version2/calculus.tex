% !TEX root = main.tex
\section*{Introduction}
Expressiveness studies for process calculi not only shed light on the intrinsic properties of the models at hand; they
also offer formal justification to the reuse and transference of reasoning techniques between different communication disciplines.
This work presents an expressiveness study of \emph{higher-order} (or \emph{process passing}) concurrency in the setting of \emph{session types}, 
a behavioral type discipline for communicating processes. 


\section{A Higher Order Session Calculus}

We define a session calculus augmented with higher order semantics.

\subsection{Syntax}

We assume countable sets:

\begin{tabular}{lcllclcll}
	$S$ &$=$& $\set{s, s_1, \dots}$ & Sessions
	&$\qquad$&
	$\dual{S}$ &$=$& $\set{\dual{s} \setbar s \in S}$ & Dual Sessions
	\\

	Var &$=$& $\set{x,y,z, \dots}$ & Variables
	&$\qquad$&
	PVar &$=$& $\set{\varp{X}, \varp{Y}, \varp{Z}, \dots}$ & Process Variables\\

	RVar &$=$& $\set{\rvar{X}, \rvar{Y}, \dots}$ & Recursive Variables
	&$\qquad$&
	Abs &$=$& $\set{\abs{x}{P} \setbar P \textrm{ is a process}}$
\end{tabular}

\noi and let

\begin{tabular}{c}
	$\mathcal{N} = S \cup \dual{S} \quad \qquad k \in \mathcal{N} \cup Var \quad \qquad V \in \mathcal{V} \cup PVar \cup Abs$
\end{tabular}

%with the set of names $\mathcal{N} = S \cup \dual{S}$ and let $k \in \mathcal{N} \cup Var$,
%$V \in \mathcal{V} \cup PVar \cup Abs$.
\noi Also for convenience we sometimes denote shared names with $a, b, \dots$ although $a,b,\dots \in N$.

\paragraph{Processes.}

The syntax of processes follows:

\begin{tabular}{rcl}
	$P$	&$\bnfis$&	$\bout{k}{k'} P \bnfbar \binp{k}{x} P \bnfbar \rvar{X} \bnfbar \recp{X}{P}$\\
		&$\bnfbar$&	$\bout{k}{\abs{x} Q} P \bnfbar \binp{k}{X} P \bnfbar \appl{X}{k}$\\ 
		&$\bnfbar$&	$\bsel{s}{l} P \bnfbar \bbra{s}{l_i:P_i}_{i \in I} \bnfbar 
				P_1 \Par P_2 \bnfbar \news{s} P \bnfbar \inact$
		\\
		&\dk{$\bnfbar$}&	\dk{$\bout{k}{\tilde{k'}} P \bnfbar \binp{k}{\tilde{x}} P$}\\
		&\dk{$\bnfbar$}&	\dk{$\bout{k}{\abs{X} Q} P \bnfbar \appl{X}{\abs{x} Q}$}

				
\end{tabular}

We rely on usual assumptions for guarded recursive processes.
We say that a process is a \emph{program} if it contains
no free variables or free process/recursive variables.

\subsection{Reduction Relation}

\subsubsection{Structural Congruence}
\[
	\begin{array}{c}
		P \Par \inact \scong P \qquad P_1 \Par P_2 \scong P_2 \Par P_1 \qquad P_1 \Par (P_2 \Par P_3)
		\qquad (P_1 \Par P_2) \Par P_3 \qquad \news{s} \inact \scong \inact\\
		s \notin \fn{P_1} \Rightarrow P_1 \Par \news{s} P_2 \scong \news{s}(P_1 \Par P_2)
		\qquad \rec{X}{P} \scong P\subst{\rec{X}{P}}{\rvar{X}}
	\end{array}
\]

\subsubsection{Process Variable Substitution}
\[
	\begin{array}{rclcrcl}
			\appl{X}{k} \subst{\abs{x}{Q}}{\X} & = & Q \subst{k}{x} \\
		(\bout{s}{\abs{y}{P_1}} P_2) \subst{\abs{x}{Q}}{\X} &=& \bout{s}{\abs{y}{P_1 \subst{\abs{x}{Q}}{\X}}} (P_2 \subst{\abs{x}{Q}}{\X}) \\
		(\binp{s}{\Y} P) \subst{\abs{x}{Q}}{\X} &=& \binp{s}{\Y} (P \subst{\abs{x}{Q}}{\X}) \\
		(\bsel{s}{l} P) \subst{\abs{x}{Q}}{\X} &=& \bsel{s}{l} (P \subst{\abs{x}{Q}}{\X})\\
		(\bbra{s}{l_i : P_i}_{i \in I}) \subst{\abs{x}{Q}}{\X} &=& \bbra{s}{l_i : P_i \subst{\abs{x}{Q}}{\X}}_{i \in I}\\
		(P_1 \Par P_2) \subst{\abs{x}{Q}}{\X} & = & P_1 \subst{\abs{x}{Q}}{\X} \Par P_2 \subst{\abs{x}{Q}}{\X}\\
		(\news{s} P) \subst{\abs{x}{Q}}{\X} & = & \news{s} (P \subst{\abs{x}{Q}}{\X})\\
		\inact \subst{\abs{x}{Q}}{\X} & = & \inact
	\end{array}
\]

\subsubsection{Operational Semantics}
\[
	\begin{array}{rcl}
%		\spi & \breq{a}{s} P_1 \Par \bacc{a}{x} P_2 &\red& P_1 \Par P_2 \subst{s}{x}
	\end{array}
\]
\[
	\begin{array}{rclcrcl}
		&& && \bout{s}{\abs{x}{P}} P_1 \Par \binp{s}{\X} P_2 &\red& P_1 \Par P_2 \subst{\abs{x}{P}}{\X} \\
		&& && \bout{s}{s'} P_1 \Par \binp{s}{x} P_2 &\red& P_1 \Par P_2 \subst{s'}{x}\\
		&& && \bsel{s}{l_k} P \Par \bbra{s}{l_i : P_i}_{i \in I} &\red& P \Par P_k \qquad k \in I \\
		P_1 &\red& P_1' &\Rightarrow& P_1 \Par P_2 &\red& P_1' \Par P_2\\
		P &\red& P' &\Rightarrow& \news{s} P &\red& \news{s} P'\\
		P &\scong \red \scong& P'&\Rightarrow& P &\red& P' 		
	\end{array}
\]

\subsection{Subcalculi}

We identify two subcalculi of the Higher Order Session Calculus:
\begin{enumerate}
	\item	$\HO$ uses only the semantics that allow abstraction passing
		(i.e.\ using the syntax in the second and third lines of the definition of $P$).
	\item	$\spi$ uses only the semantics that allow name passing
		(i.e.\ using the syntax in the first and second lines of the definition of $P$).
\end{enumerate}

Later in this paper we will identify a third typed subcalculi derived
from $\spi$ that is defined on the non-usage of shared sessions.

\begin{proposition}[Normalisation]
	\label{prop:normal_form}
	Let $P$ a Higher Orser Session Calculus process, then
	$P \scong \newsp{\tilde{s}}{P_1 \Par \dots \Par P_n}$ with
	$P_1, \dots, P_n$ session prefixed processes
	%\dk{recursion $\recp{r}{P}$}
	or application process $\appl{X}{k}$.
\end{proposition}

\begin{proof}
	The proof is a simple induction on the syntax of $P$.
\end{proof}
