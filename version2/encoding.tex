\section{Encoding}

Before we proceed with encodings we define some properties
that encodings may respect:
\begin{definition}
	Given a mapping $\map{\cdot}: L_1 \longrightarrow L_2$ we
	define the following:
	\begin{enumerate}
		\item	Operational Correspondence.
			\begin{itemize}
				\item	$P \red Q$ implies $\map{P} \red^* \map{Q}$.
				\item	$\map{P} \red R$ implies $\exists Q$ such that $P \red Q$ and $R \cong \map{Q}$.	
			\end{itemize}
		\item	Typability. If $\Gamma \proves P \hastype \Delta$ then $\Gamma \proves \map{P} \hastype \Delta'$.

		\item	$\Par$-preservation. $\map{P \Par Q} = \map{P} \Par \map{Q}$.

		\item	Full Abstraction. $P \cong Q$ if and only if $\map{P} \cong \map{Q}$.
	\end{enumerate}
\end{definition}

\subsection{Encode the non-recursive $\ppi$ into $\pHO$}
In this section provide an encoding of the
$\ppi$ with no recursion into the $\pHO$.
%
\[
	\begin{array}{rcl}
		\map{\bout{k}{k'} P}	&\defeq&	\bout{k}{ \abs{z}{\binp{z}{X} \appl{X}{k'}} } \map{P} \\
		\map{\binp{k}{x} P}	&\defeq&	\binp{k}{X} \newsp{s}{\appl{X}{s} \Par \bout{\dual{s}}{\abs{x} \map{P}} \inact}\\
	\end{array}
\]
%
The rest of the operators, except the recursive constructs, are encoded in an isomorphic way:
\[
	\begin{array}{c}
		\map{\inact} \defeq \inact \qquad \map{P\Par Q} \defeq \map{P} \Par \map{Q} \qquad \map{\news{s} P} \defeq \news{s} \map{P}\\
		\map{\bsel{k}{l} P} \defeq \bsel{k}{l} \map{P} \qquad \quad \map{\bbra{k}{l_i:P_i}_{i \in I}} \defeq \bbra{k}{l_i: \map{P_i}}_{i \in I}
	\end{array}
\]
%s
We can also encode the polyadic version of the send and receive primitives.
%
\[
	\begin{array}{rcl}
		\map{\bout{k}{k' \cat \tilde{k}} P}	&\defeq&	\map{\bout{k}{k'} \bout{k}{\tilde{k}} P}\\
		\map{\binp{k}{x \cat \tilde{x}} P}	&\defeq&	\map{\binp{k}{x} \binp{k}{\tilde{x}} P}\\
	\end{array}
\]
%
Unlike the classic $\pi$ calculus we do not need to create a new channel because typable terms
quarranty no race conditions on the two session endpoints.

\subsection{Extend the $\pHO$}

We extend the $\pHO$ with process variable abstraction and process variable application,
as well as polyadic abstractions and polyadic applications to define the
$\pHOp$ (pure Higher Order plus) calculus. We show that all of the
constructs are encodable in the $\pHO$.
%
\[
	\begin{array}{rclcl}
		P &\bnfis&	\bout{k}{\abs{\X} P_1} P_2 & & \textrm{Process Abstraction}\\
		&\bnfbar&	\appl{X}{\abs{x}{P}} & & \textrm{Process Application} \\
		&\bnfbar&	\bout{k}{\abs{\tilde{x}}{P_1}} P_2 & & \textrm{Polyadic Abstraction}\\
		&\bnfbar&	\appl{X}{\tilde{k}} & & \textrm{Polyadic Application}\\
	\end{array}
\]
\subsubsection{Operational Semantics}
In order to define the operational semantics of the $\pHOp$,
we extend the operational semantics of $\pHO$ with the rules:
%
\[
	\begin{array}{rcl}
		\bout{s}{\abs{\Y} P} P_1 \Par \binp{s}{\X} \appl{\X}{\abs{x}{P_2}} &\red& P_1 \Par P \subst{\abs{x}{P_2}}{\Y}\\
		\bout{s}{\abs{\tilde{x}} P_1} P_2 \Par \binp{s}{\X} \appl{\X}{\tilde{k}} &\red& P_2 \Par P_1\subst{\tilde{k}}{\tilde{x}}
	\end{array}
\]
%
\subsubsection{Encoding of $\pHOp$ to $\pHO$}
%
\[
	\begin{array}{rcl}
		\map{\bout{k}{\abs{\X} Q} P}	&\defeq&	\bout{k}{\abs{z} \binp{z}{\X} \map{Q}} \map{P}\\
		\map{\appl{X}{\abs{x} P}}	&\defeq&	\newsp{s}{\appl{X}{s} \Par \bout{\dual{s}}{\abs{x}{\map{P}}} \inact}\\

		\map{\bout{k}{\abs{\tilde{x}}{P_1}} P_2}	&\defeq&	\bout{k}{ \abs{z}{ \map{ \binp{z}{\tilde{x}} P_1} } } \map{P_2}\\
		\map{\appl{\X}{\tilde{k}}}			&\defeq&	\newsp{s}{\appl{\X}{s} \Par \map{\bout{\dual{s}}{\tilde{k}} \inact}}\\
	\end{array}
\]

We are not ready yet to encode recursion. In an iterative process we require
subject abstractions to be non-linear due to the fact that the receiver should
apply an abstraction more than once to achieve iteration,
i.e.~as we have seen in Example~\ref{ex:linear_abstraction} a process:
\[
	\bout{s}{\abs{}{P}} P_1 \Par \binp{s}{\X} (\appl{\X}{} \Par \appl{\X}{})
\]

with $\fs{P} \not= \es$ is not typable, since abstraction $\abs{}{P}$
can only be applied in a linear way.

\subsubsection{Encode linear $\pHO$ processes into non-linear $\pHO$ abstractions.}

Therefore it is convenient to have an encoding from a process to an abstraction
with no free names, that can be used a shared value:

\[
	\amap{\cdot} : \mathcal{P} \longrightarrow \mathcal{V}
\]
\[
	\amap{P} \bnfis \abs{\vmap{\fn{P}}}{\absmap{P}{\es}}
\]

where

Function $\smap{\cdot}: 2^{\mathcal{N}} \longrightarrow \mathcal{N}^\omega$
orders lexicographically a set of names, function 
$\vmap{\cdot}: 2^{\mathcal{N}} \longrightarrow \mathcal{V}^\omega$
maps a set of names to variables:
\[
	\begin{array}{rcl}
		\vmap{\set{s_i}_{i \in I}} &=& \svmap{\smap{\set{s_i}_{i \in I}}}\\
		\svmap{s \cat \tilde{s}} &=& x_s \cat \svmap{\tilde{s}}\\
		\svmap{s} & = & x_s
	\end{array}
\]

\[
	\begin{array}{rcll}
		\absmap{\bout{s}{\abs{x} P'} P}{\sigma} &\bnfis&
		\left\{
		\begin{array}{rl}
			\bout{x_s}{\abs{\vmap{x}{P}} \absmap{P'}{\es}} \absmap{P}{\sigma} & s \notin \sigma\\
			\bout{s}{\abs{\vmap{x}{P}} \absmap{P'}{\es}} \absmap{P}{\sigma} & s \in \sigma
		\end{array}
		\right.
		\\
		\absmap{\binp{s}{X} P}{\sigma} &\bnfis&
		\left\{
		\begin{array}{rl}
			\binp{x_s}{X} \absmap{P}{\sigma} & s \notin \sigma\\
			\binp{s}{X} \absmap{P}{\sigma} & s \in \sigma
		\end{array}
		\right.
		\\
		\absmap{\bsel{s}{l} P}{\sigma} &\bnfis&
		\left\{
		\begin{array}{rl}
			\bsel{x_s}{l} \absmap{P}{\sigma} & s \notin \sigma\\
			\bsel{s}{l} \absmap{P}{\sigma} & s \in \sigma
		\end{array}
		\right.
		\\

		\absmap{\bbra{s}{l_i: P_i}_{i \in I}}{\sigma} &\bnfis&
		\left\{
		\begin{array}{rl}
			\bbra{x_s}{l_i: \absmap{P_i}{\sigma}}_{i \in I} & s \notin \sigma\\
			\bbra{s}{l_i: \absmap{P_i}{\sigma}}_{i \in I} & s \in \sigma\\
		\end{array}
		\right.
		\\

		\absmap{P_1 \Par P_2}{\sigma} &\bnfis& \absmap{P_1}{\sigma} \Par \absmap{P_2}{\sigma} & s \notin \sigma\\
		\absmap{\news{s} P}{\sigma} &\bnfis& \news{s} \absmap{P}{\sigma\cat s}\\
		\absmap{\inact}{\sigma} &\bnfis& \inact\\
		\absmap{\appl{\X}{s}}{\sigma} &\bnfis&
		\left\{
		\begin{array}{rl}
			\appl{\X}{x_s} & s \notin \sigma\\
			\appl{\X}{s} & s \in \sigma\\
		\end{array}
		\right.
	\end{array}
\]

A basic property of the $\amap{\cdot}$ function
is the restoration of the original process when we
apply its free names to the resulting abstraction.

\begin{proposition}
	Let $P$ be a $\pHO$ process, then
	\[
		\newsp{s}{\binp{s}{X} \appl{\X}{\smap{P}} \Par \bout{\dual{s}}{\amap{P}}\inact} \red P
	\]
\end{proposition}

\dk{\proof doit}

\subsubsection{Encode Recursion}
We are ready now to encode Recursion.
%
\[
\begin{array}{rcl}
	\map{\rec{r}{P}} &=& \newsp{s}{\binp{s}{\X} \map{P} \Par \bout{\dual{s}}{\abs{z \cat \vmap{\fn{P}}}{\binp{z}{\X} \absmap{P}{\es}}} \inact}\\
	\map{\varp{r}} &=& \newsp{s}{\appl{\X}{s \cat \smap{\fn{P}}} \Par \bout{\dual{s}}{ \abs{z \cat \vmap{\fn{P}}}{\appl{X}{z \cat \vmap{\fn{P}}}}} \inact}
\end{array}
\]
%

A different process constructor for recursion is the constructor of replication:
\[
	\repl{} P
\]
with
\[
	\repl{} P \scong P \Par \repl{} P
\]

We show that process constructors $\recp{r}{P}$ can encode process constructor $\repl{} P$.
\[
\begin{array}{rcl}
	\map{\repl{} P} \defeq \recp{r}{\map{P} \Par r}.
\end{array}
\]

The other direction is encodable when $P$ is guarded on a shared input:
\[
\begin{array}{rcl}
	\map{\recp{r} \binp{a}{x} P} \defeq \repl{} \binp{a}{x} \map{\context{C}{\bout{a}{x} \inact}}\
\end{array}
\]
where $C$ being the context that results by replacing the recursive variable $r$
with a $\hole$ in $P$.

\subsection{Properties of the Encodings}

\begin{proposition}[Operational Correspondence]
	Let $P$ $\ppi$ or a $\pHOp$ process.
	\begin{enumerate}
		\item	If $P \red Q$ then $\map{P} \red^* \map{Q}$.
		\item	If $\map{P} \red R$ then $\exists Q$ such that $P \red Q$ and $R \red^* \map{Q}$.
	\end{enumerate}
\end{proposition}

\begin{proof}
Part 1 is proved by induction on the reduction rules. The basic step consider 
all leaf reductions.
\[
	\begin{array}{rcl}
		\map{\bout{s}{k'} P_1 \Par \binp{s}{x} P_2} &\bnfis& \bout{s}{ \abs{z}{\binp{z}{X} \appl{X}{k'}} } \map{P_1} \Par \binp{s}{X} \newsp{s'}{\appl{X}{s'} \Par \bout{\dual{s'}}{\abs{x} \map{P_2}} \inact}\\
		&\red& \map{P_1} \Par \newsp{s'}{\binp{s}{X} \appl{X}{k'} \Par \bout{\dual{s'}}{\abs{x} \map{P_2}} \inact}\\
		&\red& \map{P_1} \Par \map{P_2}\subst{k'}{x}
		\\
		\\

		\map{\bout{s}{\abs{\Y} P} P_1 \Par \binp{s}{X} \appl{X}{\abs{x}{P_2}}} &\bnfis& \bout{s}{\abs{z} \binp{z}{\Y} \map{P}} \map{P_1} \Par \binp{s}{X} \newsp{s'}{\appl{X}{s'} \Par \bout{\dual{s'}}{\abs{x}{\map{P_2}}} \inact}\\
		&\red& \map{P_1} \Par \newsp{s'}{\binp{s'}{\Y} \map{P} \Par \bout{\dual{s'}}{\abs{x}{\map{P_2}}} \inact}\\
		&\red& \map{P_1} \Par \map{P} \subst{\abs{x} \map{P_2}}{\Y}
		\\
		\\

		\map{\bout{s}{\abs{\tilde{x}}{P_1}} P_2 \Par \binp{s}{\X} \appl{\X}{\tilde{k}}} &\bnfis& \bout{s}{\abs{z}{\map{\binp{z}{\tilde{x}} P_1}}} \map{P_2} \Par \binp{s}{\X} \newsp{s'}{\appl{\X}{s'} \Par \map{\bout{\dual{s}}{\tilde{k}} \inact}} \\
		&\red& \map{P_2} \Par \newsp{s'}{\map{\binp{s'}{\tilde{x}} P_1} \Par \map{\bout{\dual{s}}{\tilde{k}} \inact}}\\
		&\red^*& \map{P_2} \Par \map{P_1} \subst{\tilde{k}}{\tilde{x}}
	\end{array}
\]
\dk{Operational Correspondence for Recursion TODO}

The inductive step is trivial since the rest of the reduction cases make
use of the isomorphic encoding rules.

\dk{Part 2 TODO}
\end{proof}

An important result is that of the typability of the encodings.

\begin{proposition}[Typable Encodings]
	Let $P$ be a $\ppi$ or $\pHOp$ process and $\Gamma \proves P \hastype \Delta$, then $\Gamma \proves \map{P} \hastype \Delta$
	for some environments $\Gamma$ and $\Delta$.
\end{proposition}

\begin{proof}
\begin{enumerate}
	\item	$\bout{s}{k} P$

	\[
		\tree{
			\Gamma \proves \map{P} \hastype \Delta \cat s: T
			\qquad
			\tree{
				\tree{
					\Gamma \cat \X: \lhot{T'} \proves \appl{\X}{k} \hastype k:T' \cat X
				}{
					\Gamma \cat \X: \lhot{T'} \proves \appl{\X}{k} \hastype k:T' \cat X \cat z: \tinact
				}
			}{
				\Gamma \proves \binp{z}{\X} \appl{\X}{k} \hastype z: \btinp{\lhot{T'}} \tinact
			}
		}{
			\Gamma \proves \bout{s}{\abs{z}{\binp{z}{\X} \appl{\X}{k}}} \map{P} \hastype s: \Delta \cat s: \btout{\lhot{\btinp{\lhot{T'}} \tinact}} T
		}
	\]

	\item	$\binp{s}{x} P$ with $\Gamma' = \Gamma \cat X : \btinp{\lhot{T'}} \tinact$


		\[
			\tree{
				\tree{
					\tree{
						\Gamma' \proves \appl{\X}{s'} \hastype s': \btinp{\lhot{T'}} \tinact \cat X
						\quad
						\tree{
							\tree{
								\Gamma' \proves \inact \hastype \es
							}{
								\Gamma' \proves \inact \hastype \dual{s'}: \tinact
							}
							\quad
							\Gamma' \proves \map{P} \hastype \Delta \cat x:T' \cat s: T
						}{
							\Gamma' \proves \bout{\dual{s'}}{\abs{x} \map{P}} \inact \hastype \Delta \cat \dual{s'}: \btout{\lhot{T'}} \tinact \cat s: T
						}
					}{
							\Gamma' \proves \appl{\X}{s'} \Par \bout{\dual{s'}}{\abs{x} \map{P}} \inact \hastype \Delta \cat s': \btinp{\lhot{T'}} \tinact \cat \dual{s'}: \btout{\lhot{T'}} \tinact \cat s: T \cat X
					}
				}{
					\Gamma' \proves \newsp{s'}{\appl{\X}{s'} \Par \bout{\dual{s'}}{\abs{x} \map{P}} \inact} \hastype  \Delta \cat s: T \cat X
				}
			}{
				\Gamma \proves \binp{s}{X} \newsp{s'}{\appl{\X}{s'} \Par \bout{\dual{s'}}{\abs{x} \map{P}} \inact} \hastype  \Delta \cat s: \btinp{\lhot{\btinp{\lhot{T'}} \tinact}} T
			}
		\]

	\item	$\bout{s}{\abs{\Y} P_2} P_1$

	\[
		\tree{
			\Gamma \proves \map{P_1} \hastype \Delta_1 \cat s:T
			\qquad
			\tree{
				\tree{
					\Gamma \cat \Y: \hot{T'} \proves \map{P_2} \hastype \Delta_2
				}{
					\Gamma \cat \Y: \hot{T'} \proves \map{P_2} \hastype \Delta_2 \cat z : \tinact
				}
			}{
				\Gamma \proves \binp{z}{\Y} \map{P_2} \hastype \Delta_2\backslash\Y \cat z: \btinp{\hot{T'}} \tinact
			}
		}{
			\Gamma \proves \bout{s}{ \abs{z}{ \binp{z}{\Y} \map{P_2} } } \map{P_1} \hastype \Delta_1 \cat \Delta_2\backslash\Y \cat z: \btout{\hot{\btinp{\hot{T'}} \tinact }} T
		}
	\]

	\item	$\appl{\X}{\abs{x} P}$

	\[
		\tree{
			\tree{
				\Gamma \cat \X: \hot{\btinp{\hot{T'}} \inact} \proves \appl{\X}{s} \hastype \Delta_1 \cat s: \btinp{\hot{T'}} \inact
				\quad
				\tree{
					\Gamma' \proves \map{P} \hastype \Delta_2 \cat x: T'
					\quad
					\tree{
						\Gamma' \proves \inact \hastype \es
					}{
						\Gamma' \proves \inact \hastype \dual{s'}: \tinact
					}
				}{
					\Gamma' \proves \bout{\dual{s}}{\abs{x} P} \inact \hastype \Delta_2 \cat \dual{s}: \btout{\hot{T'}} \tinact
				}
			}{
				\Gamma \cat \X: \hot{\btinp{\hot{T'}} \inact} \proves \appl{X}{s} \Par \bout{\dual{s}}{\abs{x} P} \inact \hastype \Delta_1 \cat \Delta_2 \cat s: \btinp{\hot{T'}} \inact \cat \dual{s}: \btout{\hot{T'}} \tinact
			}
		}{
			\Gamma \cat \X: \hot{\btinp{\hot{T'}} \inact} \proves \newsp{s}{\appl{X}{s} \Par \bout{\dual{s}}{\abs{x} P} \inact} \hastype \Delta_1 \cat \Delta_2
		}
	\]

	\item	$\rec{\varp{r}}{P}$


	\[
		\tree{
			\tree{
				\begin{array}{c}
					\tree{
						\Gamma \cat \X: \shot{\btinp{\shot{T'}} \tinact} \proves \map{P} \hastype \Delta \cat s: T
					}{
						\Gamma \proves \binp{s}{\X} \map{P} \hastype \Delta \cat s: \btinp{\shot{\btinp{\shot{T'}} \tinact}} T
					}
					\\
					\\
					\tree{
						\tree{
							\Gamma \cat \X: \shot{T'} \proves \absmap{P}{\es} \hastype z: \tinact \cat \tilde{y}: \tilde{T}
						}{
							\Gamma \proves \binp{z}{\X} \absmap{P}{\es} \hastype z: \btinp{\shot{T'}} \tinact \cat \tilde{y}: \tilde{T}
						}
						\qquad
						\tree{
							\Gamma \proves \inact \hastype \es
						}{
							\Gamma \proves \inact \hastype \dual{s} : \tinact
						}
					}{
						\Gamma \proves \bout{\dual{s}}{\abs{z\tilde{y}}{ \binp{z}{\X} \absmap{P}{s}}} \inact \hastype \dual{s}: \btout{\shot{\btinp{\shot{T'}} \tinact}} \tinact
					}
				\end{array}
			}{
				\Gamma \proves \binp{s}{\X} \map{P} \Par \bout{\dual{s}}{\abs{z\tilde{y}}{ \binp{z}{\X} \absmap{P}{s}}} \inact \hastype \Delta \cat s: \btinp{\shot{\btinp{\shot{T'}} \tinact}} T \cat \dual{s}: \btout{\shot{\btinp{\shot{T'}} \tinact}} \tinact
			}
		}{
			\Gamma \proves \newsp{s}{\binp{s}{\X} \map{P} \Par \bout{\dual{s}}{\abs{z\tilde{y}}{ \binp{z}{\X} \absmap{P}{s}}} \inact} \hastype \Delta
		}
	\]

	\item	$\map{\varp{r}}$

\end{enumerate}
\end{proof}

\subsection{Encode $\pHO$ processes into $\ppi$.}

\subsubsection{First Approach}

\[
\begin{array}{rcl}
	\map{\binp{k}{\X} P} &\defeq& \binp{k}{x} \map{P}\\
	\map{\appl{\X}{k}} &\defeq& \bout{x}{k} \inact\\
	\map{\bout{k}{\abs{x} Q} P} &\defeq& \newsp{s}{\bout{k}{s} \map{P} \Par \binp{\dual{s}}{x} \map{Q}}
\end{array}
\]

\begin{proposition}
	Let $P$ be a $\pHO$ process with $\Gamma \proves P \hastype \Delta$ and
	with the typing derivation to contain only linear abstractions. $\map{P}$
	\begin{itemize}
		\item	is typable.
		\item	enjoys operational correspondence.
	\end{itemize}

\end{proposition}

\begin{proof}
	\dk{TODO}
\end{proof}

Nevertheless the above encoding is not typable and does not respect
operational correspondence for processes that require
shared abstractions.

\begin{proposition}
	Let $P$ be a $\pHO$ process with $\Gamma \proves P \hastype \Delta$ and
	with the typing derivation to contain shared abstractions. $\map{P}$
	\begin{itemize}
		\item	is not typable.
		\item	does not enjoy operational correspondence.
	\end{itemize}
\end{proposition}

\begin{proof}
	Let process $P = \bout{\dual{s}}{\abs{} \inact} \inact \Par \binp{s}{\X} (\appl{\X}{} \Par \appl{\X}{})$.
	The typing of such process requires in its derivation to check process variable $\X$ against a
	shared type.
%
	We get
	\[
		\map{P} \defeq \newsp{s'}{\bout{\dual{s}}{s'} \inact \Par \binp{\dual{s'}{}} \inact} \Par \binp{s}{x} (\bout{x}{} \inact \Par \bout{x}{} \inact)
	\]

	The derivation on the subprocess $\bout{x}{} \inact \Par \bout{x}{} \inact$
	uses the $\trule{Par}$ rule which in turn checks for the disjointness of
	the two linear environemnts. But both environments contain variable $x$
	making the mapping untypable.

	Furthermore in the untyped setting
%
	\[
		\begin{array}{rcl}
			\map{P} \red^* \inact \Par \bout{x}{} \inact
		\end{array}
	\]
	when
	\[
		P \red \inact
	\]
	providing evidence for no operational correspondence.
\end{proof}

As a consequence of the last two proposition the provided encoding
allows only for a limited set of processes (namely purely linear processes)
to be encoded in a sound way.

Nevertheless we claim that there is a sound encoding from
$\pHO$ to $\ppi$, although its definition should be complicated.
We give the basic intuition through an example.

\begin{example}
	Let process 
	\[
		P = \bout{\dual{s}}{\abs{} \inact} \inact \Par \binp{s}{\X} (\recp{r}{\appl{\X}{} \Par r} \Par \appl{\X}{})
	\]

	A sound mapping for this process should be

	\[
		\map{P} \defeq \newsp{s_1, s_2}{\bout{\dual{s}}{s_1, s_2} \inact \Par \recp{r}{(\binp{\dual{s_1}}{} r)} \Par \binp{\dual{s_2}}{} \inact} \Par \binp{s}{x_1, x_2} (\recp{r}{(\bout{x_1}{} r)} \Par \bout{x_2}{} \inact)
	\]
\end{example}

To formalise the above intuition we should use a mapping with
complex side conditions that tracks the entire structure of the
process.

\subsection{Encode $\pHO$ processes into $\spi$.}

However we can easily provide a sound encoding for $\pHO$
process into $\spi$ processes, by exploiting shared channels
to represent shared abstractions.

\[
\begin{array}{rcl}
	\map{\binp{k}{\X} P} &\defeq& \binp{k}{x} \map{P}\\
	\map{\appl{\X}{k}} &\defeq& \newsp{s}{\bout{x}{s} \bout{\dual{s}}{k} \inact}\\
	\map{\bout{k}{\abs{x} Q} P} &\defeq& \newsp{a}{\bout{k}{a} \map{P} \Par \repl{} \binp{a}{y} \binp{\dual{y}}{x} \map{Q}}
\end{array}
\]

\subsubsection{Operational Correspondence}

\dk{TODO}

\subsection{Non Encodability}

We prove an auxiliary result:

\begin{lemma}[\dk{$\tau$-inert}]
	\label{lem:tau_inert}
	Let $P$ a pure HO process
	and $\Gamma; \Lambda; \Sigma \proves P \hastype \Proc$
	If $P \red^* P'$ then
	$\Gamma; \Lambda; \Sigma \proves P \wbc \Gamma'; \Lambda'; \Sigma' \proves P' \hastype \Proc$.
\end{lemma}

\begin{proof}
	The proof is a double induction first on the length and then on the structure of $\red$.
	Induction on the length of $\red$.

	The basic step is trivial
	Induction hypothesis:
	If $P \red^* P''$ then
	$\Gamma; \Lambda; \Sigma \proves P \wbc \Gamma''; \Lambda''; \Sigma'' \proves P'' \hastype \Proc$.

	Induction step:
	Let $P'' \red P'$. We do an induction on the structure of $\red$
	to show that $\Gamma'; \Lambda'; \Sigma' \proves P'' \wbc \Gamma'; \Lambda'; \Sigma' \proves P' \hastype \Proc$.

	Basic step:

	\Case{$P'' = \bout{s}{V} P_1 \Par \binp{\dual{s}}{X} P_2$}
	\[
		\Gamma''; \Lambda''; \Sigma'' \proves \bout{s}{V} P_1 \Par \binp{\dual{s}}{X} P_2 \red \Gamma'; \Lambda'; \Sigma' \proves P_1 \Par P_2 \subst{V}{X} \hastype \Proc
	\]
	The proof follows from the fact that we can only observe a $\tau$
	action on typed process
	$\Gamma'; \Lambda'; \Sigma \proves P''$.
	Actions $\bactout{s}{V}$ and $\bactinp{\dual{s}}{V}$
	are forbiden by the environment LTS.

	We can conclude then that $P'' \wbc P'$.

	\Case{$P'' = \bsel{s}{l} P_1 \Par \bbra{\dual{s}}{l_i: P_i}_{i \in I}$}

	Similar arguments as the previous case.

	Induction hypothesis:
	If $P_1 \red P_2$ then
	$\Gamma_1; \Lambda_1; \Sigma_1 \proves P_1 \wbc \Gamma_2; \Lambda_2; \Sigma_2 \proves P_2 \hastype \Proc$.

	Induction Step:

	\Case{$P'' = \news{s} P_1$}
	\[
		\Gamma''; \Lambda''; \Sigma'' \proves \news{s}{P_1} \red \Gamma'; \Lambda'; \Sigma' \proves \news{s} P_2 \hastype \Proc
	\]
	From the induction hypothesis and the fact that bisimulation is a congruence
	we get that $P'' \wbc P'$.

	\Case{$P'' = P_1 \Par P_3$}

	\[
		\Gamma''; \Lambda''; \Sigma'' \proves P_1 \Par P_3 \red \Gamma'; \Lambda'; \Sigma' \proves P_2 \Par P_3 \hastype \Proc
	\]
	From the induction hypothesis and the fact that bisimulation is a congruence
	we get that $P'' \wbc P'$.

	\Case{$P'' \scong P_1$}

	From the induction hypothesis and the facts that bisimulation is a congruence
	and structural congruence preserves $\wbc$
	we get that $P'' \wbc P'$.

	We can now conclude that
	$P \wbc P'$ because $P \wbc P''$ and $P'' \wbc P'$.
\end{proof}

\begin{theorem}
	Mapping $\map{\cdot}: \pHO \longrightarrow \spi$ that enjoys
	operational correspondence and full abstraction does not exist. 
\end{theorem}

\begin{proof}
	Let $\Gamma_1; \Lambda_1; \Sigma_1 \proves P_1 \not\cong \Gamma_2; \Lambda_2; \Sigma_2 \proves P_2 \hastype \Proc$
	with $P = \breq{a}{s} \inact \Par \bacc{a}{x} P_1 \Par \bacc{a}{x} P_2$ and
	let $\Gamma; \Lambda; \Sigma \proves P \hastype \Proc$. Assume also a mapping
	$\map{\cdot}: \pHO \longrightarrow \spi$ that enjoys
	operational correspondence and full abstraction.

	From operational correspondence we get that:
	\begin{eqnarray*}
		R \red P_1 \Par \bacc{a}{x} P_2 &\textrm{implies}& \map{R} \red \map{P_1 \Par \bacc{a}{x} P_2}\\
		R \red P_2 \Par \bacc{a}{x} P_1 &\textrm{implies}& \map{R} \red \map{P_2 \Par \bacc{a}{x} P_1}
	\end{eqnarray*}

	From the fact that
	$\Gamma_1; \Lambda_1; \Sigma_1 \proves P_1 \not\cong \Gamma_2; \Lambda_2; \Sigma_2 \proves P_2 \hastype \Proc$
	we can derive that
	\[
		\Gamma_1'; \Lambda_1'; \Sigma_1' \proves P_1 \Par \bacc{a}{x} P_2 \not\cong \Gamma_2'; \Lambda_2'; \Sigma_2' \proves P_2 \Par \bacc{a}{x} P_1 \hastype \Proc
	\]

	From Lemma~\ref{lem:tau_inert} we know that
	\begin{eqnarray*}
		\map{R} &\cong& \map{P_1 \Par \bacc{a}{x} P_2}\\
		\map{R} &\cong& \map{P_2 \Par \bacc{a}{x} P_1}
	\end{eqnarray*}
	\noi thus
	\[
		\map{P_1 \Par \bacc{a}{x} P_2} \cong \map{P_2 \Par \bacc{a}{x} P_1}
	\]
	From here we conclude that the full abstraction property does not hold,
	so there is no mapping $\map{\cdot}: \pHO \longrightarrow \spi$ that enjoys
	the operational correspondence and full abstraction properties.
\end{proof}


\begin{comment}
\subsection{Negative Result}

A good encoding of the $\spi$ calculus to the $\pHO$ calculus
should respect the represenation of race conditions over shared
channels. This represantation can be captured by the $\Par$-preservation
property.

In this section we prove that the $\pHO$ calculus cannot represent
$\spi$ processes that model race conditions.

First we prove an auxiliary result:

\begin{lemma}
	\label{lem:unique_red}
	Let $P \Par Q$ a $\pHO$ process with
	$\Gamma \proves P \Par Q \hastype \Delta$ and
	$\Delta$ well typed.
	If $P \Par P' \red P' \Par Q'$ and
	$P \Par Q \red P' \Par Q''$ then
	$Q' \scong Q''$.
\end{lemma}

\begin{proof}
	We write $P \Par Q$ using the normal form (Proposition~\ref{prop:normal_form}).
	\[
		P \Par Q \scong \newsp{\tilde{s}}{P_1 \Par \dots \Par P_n} \Par \newsp{\tilde{s'}}{Q_1 \Par \dots \Par Q_m} 
	\]
	We do a case analysis on the possible reductions:

	\Case{}
%
	\begin{eqnarray*}
		\newsp{\tilde{s}}{P_1 \Par \dots P_i \Par \dots \Par P_j \Par \dots \Par P_n} \Par Q
		&\red&
		\newsp{\tilde{s}}{P_1 \Par \dots P_i' \Par \dots \Par P_j' \Par \dots \Par P_n} \Par Q\\
		\newsp{\tilde{s}}{P_1 \Par \dots P_i \Par \dots \Par P_j \Par \dots \Par P_n} \Par Q
		&\red&
		\newsp{\tilde{s}}{P_1 \Par \dots P_i' \Par \dots \Par P_j' \Par \dots \Par P_n} \Par Q
	\end{eqnarray*}
	The proof is trivial.

	\Case{}
%
	\begin{eqnarray*}
		\newsp{\tilde{s}}{P_1 \Par \dots P_i \Par \dots \Par P_n} \Par \newsp{\tilde{s'}}{Q_1 \Par \dots \Par Q_j \Par \dots \Par Q_m} 
		&\red&
		\newsp{\tilde{s}}{P_1 \Par \dots P_i' \Par \dots \Par P_n} \Par \newsp{\tilde{s'}}{Q_1 \Par \dots \Par Q_j' \Par \dots \Par Q_m} \\
		\newsp{\tilde{s}}{P_1 \Par \dots P_i \Par \dots \Par P_n} \Par \newsp{\tilde{s'}}{Q_1 \Par \dots \Par Q_k \Par \dots \Par Q_m} 
		&\red&
		\newsp{\tilde{s}}{P_1 \Par \dots P_i' \Par \dots \Par P_n} \Par \newsp{\tilde{s'}}{Q_1 \Par \dots \Par Q_k' \Par \dots \Par Q_m} \\
	\end{eqnarray*}

	By normalisation (Lemma~\ref{prop:normal_form}) we get that $P_i$ and $Q_j$ are session prefixed, so we can assume that
	they are prefixed on session $s$. By the well typeness condition of $\Delta$ we get that $s, \dual{s} \in \dom{\Delta}$,
	with $\Gamma \proves P_i \hastype \Delta \cat \dual{s}: T_i$
	If we assume that $k \not= j$ then $\Gamma \proves Q_j \hastype \Delta_j \cat s: T_j$ and $\Gamma \proves Q_k \hastype \Delta_k \cat s: T_k$,
	because the two processes should interact with endpoint $\dual{s}$ in process $P_i$.
	Furthermore typing rule $\trule{Par}$ cannot be applied to type $Q_j \Par Q_k$ because $\Delta_j$ and $\Delta_k$ are not disjoint.
	So it has to be that $k = j$ that results to:
	\[
		\newsp{\tilde{s'}}{Q_1 \Par \dots \Par Q_j' \Par \dots \Par Q_m} \scong \newsp{\tilde{s'}}{Q_1 \Par \dots \Par Q_k' \Par \dots \Par Q_m}
	\]
	as required.
\end{proof}

\begin{theorem}
	Mapping $\map{\cdot}: \pHO \longrightarrow \spi$ that enjoys
	operational correspondence and $\Par$-preservation does not
	exist.
\end{theorem}

\begin{proof}
	Let $\map{\cdot}: \pHO \longrightarrow \spi$ that respects
	operational correspondence and $\Par$-preservation and
	$\pHO$ process:
	$P = \bout{a}{s} P_1 \Par \binp{a}{x} P_2 \Par \binp{a}{x} P_3$
	with $P_1 \not\scong P_2$ and $\Gamma \proves P \hastype \Delta$.

	$\Par$-preservation implies
	\[
		\map{P} \defeq \map{\bout{a}{s} P_1} \Par \map{\binp{a}{x} P_2} \Par \map{\binp{a}{x} P_3}
	\]
%
	and operational correspondence implies
%
	\begin{eqnarray}
		P &\red& P_1 \Par P_2 \subst{s}{x} \Par \binp{a}{x} P_3 \Rightarrow \map{P} \red \map{P_1 \Par P_2 \subst{s}{x} \Par \binp{a}{x} P_3} \\
		P &\red& P_1 \Par \binp{a}{x} P_1 \Par P_3 \subst{s}{x} \Rightarrow \map{P} \red \map{P_1 \Par \binp{a}{x} P_2 \Par P_3 \subst{s}{x}}
	\end{eqnarray}
%
	By the $\Par$-preservation property we get that
%
	\begin{eqnarray}
		\map{\bout{a}{s} P_1} \Par \map{\binp{a}{x} P_2} \Par \map{\binp{a}{x} P_3} &\red& \map{P_1} \Par \map{P_2 \subst{s}{x}} \Par \map{\binp{a}{x} P_3} \\
		\map{\bout{a}{s} P_1} \Par \map{\binp{a}{x} P_2} \Par \map{\binp{a}{x} P_3} &\red& \map{P_1} \Par \map{\binp{a}{x} P_2} \Par \map{P_3 \subst{s}{x}}
	\end{eqnarray}
%
	By Lemma~\ref{lem:unique_red} we get that
	$\map{P_1} \Par \map{P_2 \subst{s}{x}} \Par \map{\binp{a}{x} P_3} \scong \map{P_1} \Par \map{\binp{a}{x} P_2} \Par \map{P_3 \subst{s}{x}}$
	which implies contradiction since $P_1 \subst{s}{x} \not\scong P_2 \subst{s}{x}$.
\end{proof}
\end{comment}

\begin{comment}
\[
	\begin{array}{rcl}
		\vmap{\tilde{x}}{\inact} &\bnfis& \tilde{x}\\
		\vmap{\tilde{x}}{\bout{s}{\abs{x} P'} P} &\bnfis&
		\left\{
		\begin{array}{ll}
			\vmap{\tilde{x}\cat x_s}{P} & x_s \notin \tilde{x}\\
			\vmap{\tilde{x}}{P} & x_s \in \tilde{x}
		\end{array}
		\right.
		\\
		\vmap{\tilde{x}}{\binp{s}{X} P} &\bnfis&
		\left\{
		\begin{array}{ll}
			\vmap{\tilde{x}\cat x_s}{P} & x_s \notin \tilde{x}\\
			\vmap{\tilde{x}}{P} & x_s \in \tilde{x}
		\end{array}
		\right.
		\\
		\vmap{\tilde{x}}{\bsel{s}{l} P} &\bnfis&
		\left\{
		\begin{array}{ll}
			\vmap{\tilde{x}\cat x_s}{P} & x_s \notin \tilde{x}\\
			\vmap{\tilde{x}}{P} & x_s \in \tilde{x}
		\end{array}
		\right.
		\\


	\end{array}
\]
\end{comment}

%Specifically if:
%\[
%	\absmap{P} \bnfis \abs{\tilde{x}} P'
%\]
%we require the property:
%\[
%	P = P' \subst{\fn{P}}{\tilde{x}}
%\]


