% !TEX root = main.tex

\newcommand{\Proc}{\ensuremath{\diamond}}
\newcommand{\Unit}{\ensuremath{\mathsf{unit}}}
\newcommand{\UnitV}{\ensuremath{()}}

\section{Higher-Order Session Types (Mostrous and Yoshida)}

We recall the typed framework for higher-order session communication (as proposed by Mostrous and Yoshida), adapted to our setting.

\subsection{Types}
The type structure that we consider here is a subset of the type syntax considered by Mostrous and Yoshida.
The only (but fundamental) differences are in the types for values: we focus on having 
$\shot{S}$ and $\lhot{S}$, whereas Mostrous and Yoshida handle the more general functions $U \sharedop T$ and 
$U \lollipop T$.
\[
\begin{array}{lcl}
\text{Terms} & T ::= & U  \bnfbar  \Proc\\
\text{Values} & U ::= & \Unit  \bnfbar  \lhot{S} \bnfbar \shot{S} \bnfbar \chtype{S} \bnfbar S \\
\text{Sessions} \quad & S ::= &  \btout{U} S \bnfbar \btinp{U} S
		\bnfbar		\btsel{l_i:S_i}_{i \in I} \bnfbar \btbra{l_i:S_i}_{i \in I} \bnfbar \trec{t}{S} \bnfbar \vart{t}  \bnfbar \tinact 
\end{array}
\]

\subsection{Environments and Judgment}
Following our design decision of focusing on functions $\shot{S}$ and $\lhot{S}$, our environments are also simpler (the only difference wrt M\&Y is the in the `Shared' group):
\[
\begin{array}{lcl}
\text{Shared} & \Gamma :: = & \emptyset \bnfbar \Gamma, \, u{:}\Unit \bnfbar \Gamma,\, u{:}\shot{S} \bnfbar \Gamma, \,u{:}\chtype{S} \\
\text{Linear} & \Lambda :: = & \emptyset \bnfbar \Lambda,\, X{:}\lhot{S}  \\
\text{Session} \quad & \Sigma :: = & \emptyset \bnfbar \Sigma,\, k{:}S  
\end{array}
\]
With these environments the shape of judgments is exactly the same as in Mostrous and Yoshida's system:
\[
\Gamma; \Lambda ; \Sigma \vdash P \hastype T
\]
As expected, weakening, contraction, and exchange principles apply to $\Gamma$; environments $\Lambda$ and $\Sigma$ behave linearly, and are only subject to exchange.
We focus on \emph{well-formed} judgments, which do not share elements in their domains.


\subsection{Typing Rules}

%\newcommand{\jrule}[3]{\displaystyle \frac{#1 }{#2} & \trule{#3}}
\newcommand{\jrule}[3]{\displaystyle \trule{#3}~~\frac{#1 }{#2}}
\newcommand{\addenv}{\circ}

The typing rules for our system are in Fig.~\ref{fig:typerulesmy}. 
They are essentially the rules proposed in Mostrous and Yoshida's system, but adapted to our simpler type structure.
We have two rules for reception, one for receiving sessions, another for receiving abstractions. 
We retain structural rules (notably, the promotion rule which enables us to get a `linear copy' of an unrestricted function $\shot{S}$) and the rule for output, which in Mostrous and Yoshida's system admits a rather compact formulation.

\begin{figure}[!t]
\[
	\begin{array}{c}
	\jrule{ }{\Gamma ; \emptyset; \emptyset \vdash \UnitV \hastype \Unit}{Unit} 
	 \qquad\quad  
	\jrule{ }{\Gamma, \, u{:}U ; \emptyset; \emptyset \vdash u \hastype U}{Shared} 
	\vspace{2mm}
	\\
	\jrule{ }{\Gamma ; \{X{:}\lhot{S}\}; \emptyset \vdash X \hastype \lhot{S}}{LVar} 
	 \qquad\quad 
	\jrule{ }{\Gamma ; \emptyset ; \{k{:}S\} \vdash k \hastype S}{Session} 
	\vspace{2mm}
	\\
	\jrule{\Gamma ; \emptyset; \emptyset \vdash P \hastype \lhot{S} }{\Gamma ; \emptyset; \emptyset \vdash P \hastype \shot{S}}{Prom} 
	 \qquad\quad  
	\jrule{ \Gamma ; \Lambda ,\, X{:}\lhot{S}; \Sigma \vdash P \hastype T }{ \Gamma,\, X{:}\lhot{S}; \Lambda ; \Sigma \vdash P \hastype T}{Derelic} 
	\vspace{2mm}
	\\
	\jrule{\Gamma ; \Lambda; \Sigma \vdash P \hastype T  \quad \Sigma \subt \Sigma' \quad T \subt T'}{\Gamma ; \Lambda; \Sigma' \vdash P \hastype T'}{Subt} 
	\qquad\quad  
	\jrule{ \Gamma ; \Lambda; \Sigma,\, x{:} S \vdash P \hastype \Proc }{ \Gamma ; \Lambda; \Sigma \vdash (x)P \hastype \lhot{S}}{Abs} 
	\vspace{2mm}
	\\
	\jrule{(T = \lhot{S}) \lor (T = \shot{S}) \quad \Gamma ; \Lambda_1; \Sigma_1 \vdash X \hastype T  \quad \Gamma ; \Lambda_2; \Sigma_2 \vdash k \hastype S}{\Gamma ; \Lambda_1, \Lambda_2; \Sigma_1, \Sigma_2 \vdash \appl{X}{k} \hastype \Proc}{App} 
	\vspace{2mm}
	\\
	\jrule{\Gamma ; \Lambda_1; \Sigma_1 \vdash P \hastype \Proc  \quad \Gamma ; \Lambda_2; \Sigma_2 \vdash V \hastype U  \quad (k{:}S \in \Sigma_1) \lor (k{:}S \in \Sigma_2) }{\Gamma ; \Lambda_1, \Lambda_2; (\Sigma_1, \Sigma_2)-\{k{:}S\}, k{:}\btout{U} S \vdash \bout{k}{V} P \hastype \Proc}{Send} 
	\qquad\quad 
	\vspace{2mm}
	\\
	\jrule{\Gamma ; \Lambda; \Sigma, \, x{:}S \vdash P \hastype \Proc  \quad \Gamma ; \emptyset; \emptyset \vdash u \hastype \chtype{S} }{\Gamma ; \Lambda; \Sigma \vdash \binp{u}{x} P \hastype \Proc}{Conn} 
	\qquad\quad  
	\vspace{2mm}
	\\
	\jrule{\Gamma ; \Lambda; \Sigma, \, x{:}S_1 \vdash P \hastype \Proc  \quad \Gamma ; \emptyset; \emptyset \vdash u \hastype \chtype{S_2} \quad dual(S_1, S_2)}{\Gamma ; \Lambda; \Sigma \vdash \bout{u}{x} P \hastype \Proc}{ConnDual}
	\qquad\quad 
	\jrule{\Gamma, \, a{:}\chtype{S} ; \Lambda; \Sigma \vdash P \hastype \Proc  }{\Gamma ; \Lambda; \Sigma \vdash \news{a{:}\chtype{S}} P \hastype \Proc}{New} 
	\vspace{2mm}
	\\
	\jrule{\Gamma ; \Lambda; \Sigma, k{:} S_1, x{:} S_2 \vdash P \hastype \Proc }{ \Gamma ; \Lambda; \Sigma, k{:}\btinp{S_1}S_2  \vdash \binp{k}{x}P \hastype \Proc}{RecvS}
	\qquad\quad 
	\jrule{\Gamma ; \Lambda, X{:} \lhot{S}; \Sigma, k{:} S_1  \vdash P \hastype \Proc }{ \Gamma ; \Lambda; \Sigma, k{:}\btinp{\lhot{S}}S_1  \vdash \binp{k}{X}P \hastype \Proc}{RecvA}
     \vspace{2mm}
	\\
	\jrule{}{\Gamma ; \emptyset; \emptyset \vdash \inact \hastype \Proc}{Nil}
     \vspace{2mm}
	\\
	\jrule{\Gamma ; \Lambda_{1,2}; \Sigma_{1,2} \vdash P_{1,2} \hastype \Proc}{\Gamma ; \Lambda_{1,2}; \Sigma_{1,2} \vdash P_1 \Par P_2 \hastype \Proc}{Par}
	 \qquad\quad  
	 \jrule{\Gamma ; \Lambda ; \Sigma  \vdash P \hastype T \quad k \not\in dom(\Gamma, \Lambda,\Sigma )}{\Gamma ; \Lambda ; \Sigma, k{:}\tinact  \vdash P \hastype T}{Close}
     \vspace{2mm}
	\\
	\jrule{\Gamma ; \Lambda ; \Sigma, k{:}S_i \vdash P_{i} \hastype \Proc \quad (\forall i \in I)}{\Gamma ; \Lambda ; \Sigma, k{:}\btbra{l_i:S_i}_{i \in I} \vdash \bbra{k}{l_i:P_i}_{i \in I}\hastype \Proc}{Bra}
	 \qquad\quad 
	 \jrule{\Gamma ; \Lambda ; \Sigma, k{:}S_j  \vdash P \hastype \Proc \quad j \in I}{\Gamma ; \Lambda ; \Sigma, k{:}\btsel{l_i:S_i}_{i \in I} \vdash \bsel{s}{l_j} P \hastype T}{Sel}
	\end{array}
\]
\caption{Typing Rules (Mostrous and Yoshida)\label{fig:typerulesmy}}
\end{figure}

\subsection{Main Results}
We state results for type safety: we report instances of more general statements already proved by Mostrous and Yoshida in the asynchronous case.

\begin{lemma}[Substitution Lemma - Lemma C.10 in M\&Y]
\begin{enumerate}[1.]
\item Suppose $\Gamma, x{:}U ; \Lambda ; \Sigma  \vdash P \hastype T$ and $\Gamma; \emptyset ; \emptyset  \vdash V \hastype U$.
		Then $\Gamma ; \Lambda ; \Sigma  \vdash P\subst{V}{x} \hastype T$. 
\item Assume $\Gamma ; \Lambda_1, X{:}\lhot{S} ; \Sigma_1  \vdash P \hastype T$ 
		and $\Gamma ; \Lambda_2  ; \Sigma_2  \vdash V \hastype \lhot{S}$ with $ \Lambda_1, \Lambda_2$ and $\Sigma_1, \Sigma_2$ defined.  
		Then $\Gamma ; \Lambda_1, \Lambda_2 ; \Sigma_1, \Sigma_2  \vdash P\subst{V}{X} \hastype T$.
\item Suppose $\Gamma ; \Lambda ; \Sigma, x{:}S  \vdash P \hastype T$ and $k \not\in dom(\Gamma, \Lambda, \Sigma)$. 
		Then $\Gamma ; \Lambda ; \Sigma, k{:}S  \vdash P\subst{k}{x} \hastype T$.
\end{enumerate}
\end{lemma}

We now state the instance of type soundness that we can derived from the Mostrous and Yoshida system. It is worth noticing that M\&Y have a slightly richer definition of structural congruence. Also, their statement for subject reduction relies on an ordering on typings associated to queues and other runtime elements (such extended typings are denoted $\Delta$ by M\&Y). Since we are synchronous we can omit such an ordering. 

\begin{theorem}[Type Soundness - Theorem 7.3 in M\&Y]
\begin{enumerate}[1.]
\item (Subject Congruence) Suppose $\Gamma; \Lambda ; \Sigma  \vdash P \hastype \Proc$. Then $ P \scong P'$ implies $\Gamma; \Lambda ; \Sigma  \vdash P' \hastype \Proc$
\item (Subject Reduction) Suppose $\Gamma; \emptyset ; \Sigma  \vdash P \hastype T$ with $\mathsf{balanced}(\Sigma)$. 
		Then $P \red P'$ implies $\Gamma; \emptyset ; \Sigma  \vdash P' \hastype T$.
\end{enumerate}
\end{theorem}

	