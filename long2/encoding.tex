% !TEX root = main.tex
%%%%%%%%%%%%%%%%%%%%%%%%%%%%%%%%%%%%%%%%%%%%%%%%%%%%%%%%%%%%%%%%%%%%%%%%%
% ENCODING DEFINITION AND PROPERTIESs
%%%%%%%%%%%%%%%%%%%%%%%%%%%%%%%%%%%%%%%%%%%%%%%%%%%%%%%%%%%%%%%%%%%%%%%%%

\section{Typed Encodings}
\label{s:expr}
\label{sec:enc}

This section defines the formal notion of \emph{encoding}, 
extending to a typed setting existing criteria for untyped processes (as in,
e.g.~\cite{Nestmann00,Palamidessi03,DBLP:conf/lics/PalamidessiSVV06,DBLP:journals/iandc/Gorla10,DBLP:conf/icalp/LanesePSS10}). 
We first define a typed calculus parameterised by a syntax, operational semantics, and typing.

\begin{definition}[Typed Calculus]\rm
	\label{d:tcalculus}
	A \emph{typed calculus} $\tyl{L}$ is a tuple:
	\begin{center}
		\begin{tabular}{c}
			$\calc{\CAL}{\cal{T}}{\hby{\ell}}{\wb}{\proves}$
		\end{tabular}
	\end{center}
	\noi where $\CAL$ and $\cal{T}$ are sets of processes and types, 
	respectively; and $\hby{\ell}$, $\wb$, and $\proves$ 
	denote a transition system, a typed equivalence, and a typing system for $\CAL$, respectively. 
\end{definition}

\noi Our notion of encoding considers a mapping on processes, 
types, and transition labels.  

\begin{definition}[Typed Encoding]\rm
        Let  $\tyl{L}_i=\calc{\CAL_i}{{\cal{T}}_i}{{\hby{\ell}}_i}{\wb_i}{\proves_i}$
        ($i=1,2$) be typed calculi, and let $\mathcal{A}_i$ be the
	set of labels used in relation ${\hby{\ell}}_i$.
	Given mappings $\map{\cdot}: \CAL_1 \to \CAL_2$, 
	$\mapt{\cdot}: {\cal{T}}_1 \to {\cal{T}}_2$, and 
	$\mapa{\cdot}: \mathcal{A}_1 \to \mathcal{A}_2$, 
	we write 
	$\enco{\map{\cdot}, \mapt{\cdot}, \mapa{\cdot}} : \tyl{L}_1 \to \tyl{L}_2$ to denote the \emph{typed encoding} of $\tyl{L}_1$ into $\tyl{L}_2$.
\end{definition}

\noi We will often assume that  $\mapt{\cdot}$ extends to typing
environments as expected. This way, e.g., $\mapt{\Delta \cat u:S} = \mapt{\Delta} \cat u:\mapt{S}$.

We introduce two classes of typed encodings, which 
serve different purposes.
Both  consist of syntactic and semantic criteria 
proposed for untyped processes~\cite{Palamidessi03,DBLP:journals/iandc/Gorla10,DBLP:conf/icalp/LanesePSS10},
here extended to account for (higher-order) session types.
First, for stating stronger positive encodability results, 
we define the notion of {\em precise} encodings.
Then, 
with the aim of proving strong non-encodability results, 
precise encodings are relaxed into the weaker {\em minimal} encodings. 

We first state the syntactic criteria. 
Let $\sigma$ denote a substitution of names for names 
(a renaming, in the usual sense). Given environments $\Delta$ and $\Gamma$,
we write $\sigma(\Delta)$ and $\sigma(\Gamma)$ to denote 
the effect of applying $\sigma$ on the 
domains of $\Delta$ and $\Gamma$
(clearly, $\sigma(\Gamma)$ concerns only shared names in $\Gamma$: process and recursion variables in $\Gamma$ are not affected by $\sigma$). 

\begin{definition}[Syntax Preserving Encoding]\rm
	\label{def:sep}
	We say that 
	the typed encoding 
	$\enco{\map{\cdot}, \mapt{\cdot}, \mapa{\cdot}}: \tyl{L}_1 \to \tyl{L}_2$ is \emph{syntax preserving}
	if it is:
	
	\begin{enumerate}[1.]
		\item	\emph{Homomorphic wrt parallel},   if 
		$\mapt{\Gamma}; \emptyset; \mapt{\Delta_1 \cat \Delta_2} \proves_1 \map{P_1 \Par P_2} \hastype \Proc$
		then \\
		$\mapt{\Gamma}; \emptyset; \mapt{\Delta_1} \cat \mapt{\Delta_2} \proves_2 \map{P_1} \Par \map{P_2} \hastype \Proc$.

		\item	\emph{Compositional wrt restriction},  if 
		$\mapt{\Gamma}; \emptyset; \mapt{\Delta} \proves_1 \map{\news{n}P} \hastype \Proc$
		then \\
		$\mapt{\Gamma}; \emptyset; \mapt{\Delta} \proves_2 \news{n}\map{P} \hastype \Proc$.
		
		\item \emph{Name invariant},   if
		$\mapt{\sigma(\Gamma)}; \emptyset; \mapt{\sigma(\Delta)} \proves_1 \map{\sigma(P)} \hastype \Proc$
		then \\
		$\sigma(\mapt{\Gamma}); \emptyset; \sigma(\mapt{\Delta}) \proves_2 \sigma(\map{P}) \hastype \Proc$, 
		for any injective renaming  of names $\sigma$.
	\end{enumerate}
\end{definition}
%
Homomorphism wrt parallel composition (used in, e.g.,~\cite{Palamidessi03,DBLP:conf/lics/PalamidessiSVV06})
expresses that encodings should preserve the distributed topology of source processes. This criteria 
is appropriate for both encodability and non encodability results; in our setting, 
it admits an elegant formulation, also induced by rules for typed composition.
Compositionality wrt restriction is 
also naturally supported by typing and turns out to be 
useful in our encodability results (see the following section).
Our name invariance criteria follows the one given in~\cite{DBLP:journals/iandc/Gorla10,DBLP:conf/icalp/LanesePSS10}. 
Next we define semantic criteria for typed encodings.

\begin{definition}[Semantic Preserving Encoding]\rm
\label{def:ep}
	Let
	$\tyl{L}_i=\calc{\CAL_i}{{\cal{T}}_i}{\hby{\ell}}{\wb_i}{\proves_i}$
	($i=1,2$) be typed calculi. 
	We say that
	$\enco{\map{\cdot}, \mapt{\cdot}, \mapa{\cdot}}: \tyl{L}_1 \to \tyl{L}_2$
	is a \emph{semantic preserving encoding}
	if it satisfies the properties below.
	Given a label $\ell \neq \tau$, we write 
	$\subj{\ell}$
	to denote the \emph{subject} of the action.
	
	\begin{enumerate}[1.]
		\item	\emph{Type Preservation}:
			if
			$\Gamma; \emptyset; \Delta \proves_1 P \hastype \Proc$ then 
			$\mapt{\Gamma}; \emptyset; \mapt{\Delta} \proves_2 \map{P} \hastype \Proc$,  
			for any   $P$ in $\CAL_1$.
		\item	\emph{Subject preserving}: if $\subj{\ell} = u$ then $\mathsf{sub}(\mapa{\ell}) =u$.


		\item	\emph{Operational Correspondence}:
			If $\Gamma; \emptyset; \Delta \proves_1 P \hastype \Proc$ then
			\begin{enumerate}
				\item	Completeness: 
					If  
					$\stytraarg{\Gamma}{\ell_1}{\Delta}{P}{\Delta'}{P'}{1}$
					then $\exists \ell_2, Q, \Delta''$ s.t. \\
					(i)~$\wtytraarg{\mapt{\Gamma}}{\ell_2}{\mapt{\Delta}}{\map{P}}{\mapt{\Delta''}}{Q}{2}$,
					(ii)~$\ell_2 = \mapa{\ell_1}$, 
					and \\
					(iii)~${\mapt{\Gamma}};{\mapt{\Delta''}}\proves_2 {Q}{\wb_2}{\mapt{\Delta'}}\proves_2 {\map{P'}}$.
				
				\item	Soundness:   
					If  $\wtytraarg{\mapt{\Gamma}}{\ell_2}{\mapt{\Delta}}{\map{P}}{\mapt{\Delta''}}{Q}{2}$
					then $\exists \ell_1, P', \Delta'$ s.t.  \\
					(i)~$\stytraarg{\Gamma}{\ell_1}{\Delta}{P}{\Delta'}{P'}{1}$,
					(ii)~$\ell_2 = \mapa{\ell_1}$, and
					(iii)~
					${\mapt{\Gamma}};{\mapt{\Delta'}}\proves_2 {\map{P'}}{\wb_2}
					{\mapt{\Delta''}}\proves_2 {Q}$.
		\end{enumerate}
		
		\item	\emph{Full Abstraction:} 
			\wbbarg{\Gamma}{}{\Delta_1}{P}{\Delta_2}{Q}{1}
			if and only if
			\wbbarg{\mapt{\Gamma}}{}{\mapt{\Delta_1}}{\map{P}}{\mapt{\Delta_2}}{\map{Q}}{2}.
	\end{enumerate}
\end{definition}

\noi Type preservation is a distinguishing criteria in our notion of encoding: 
it enables us to focus on encodings which retain the communication structures denoted by (session) types.
The other semantic
criteria build upon analogous definitions in the untyped setting, as we explain now. 
Operational correspondence, standardly divided into completeness and soundness criteria, is based
in the formulation given in~\cite{DBLP:journals/iandc/Gorla10,DBLP:conf/icalp/LanesePSS10}. 
Soundness ensures that the source process is mimicked 
by its associated encoding; completeness concerns the opposite direction.
Rather than reductions, completeness and soundness rely on 
the typed LTS of \defref{def:restricted_typed_transition}; labels are considered up to  mapping
$\mapa{\cdot}$, which offers flexibility when comparing different subcalculi of \HOp.
We require that $\mapa{\cdot}$ preserves communication subjects, in accordance with the
criteria in~\cite{DBLP:conf/icalp/LanesePSS10}.
It is worth stressing that 
the operational correspondence statements given in
the next section for our  encodings 
are tailored to the specifics of each encoding, and so they
are actually stronger than the criteria given above.
Finally, following~\cite{SangiorgiD:expmpa,DBLP:conf/lics/PalamidessiSVV06,Yoshida96},
we consider full abstraction as an encodability criteria: this results into 
stronger encodability results. 
%The completeness direction of full abstraction is dropped when we prove the negative result. 
From the criteria in \defref{def:sep} and \defref{def:ep}
we have the following derived criteria: 

\begin{proposition}[Derived Criteria]\rm
	\label{p:barbpres}
	Let
	$\enco{\map{\cdot}, \mapt{\cdot}, \mapa{\cdot}}: \tyl{L}_1 \to \tyl{L}_2$
	be a typed encoding.
	Suppose the encoding is both
	operational complete (cf.~\defref{def:ep}-2(a)) 
	and subject preserving (cf.~\defref{def:ep}-3).
	Then, it is also \emph{barb preserving}, i.e., 
	$\Gamma; \Delta \proves_1 P \barb{n}$
	implies
	$\mapt{\Gamma}; \mapt{\Delta} \proves_2 \map{P} \Barb{n}$.
\end{proposition}

\begin{proof}
	The proof follows from the definition of barbs,
	operational completeness, and subject preservation.
	\qed
\end{proof}

We may now define \emph{precise} and \emph{minimal} typed criteria: 

\begin{definition}[Typed Encodings: Precise and Minimal]\rm
	\label{def:goodenc}
	We say that  the typed encoding 
	$\enco{\map{\cdot}, \mapt{\cdot}, \mapa{\cdot}}: \tyl{L}_1 \to \tyl{L}_2$ is 
%
	\begin{enumerate}[(i)]
		\item	\emph{precise}, if it is both syntax and semantic preserving
			(cf.~\defref{def:sep} and \defref{def:ep}).

		\item	\emph{minimal}, if it is syntax preserving 
			(cf.~\defref{def:sep}),
			%barb preserving, 
			and operational complete (cf.~\defref{def:ep}-2(a)).
	\end{enumerate}
\end{definition}

Precise encodings offer more detailed criteria and used for positive 
encodability results (\secref{sec:positive}).
In contrast, minimal encodings contains only 
some of the criteria of precise encodings:    
this reduced notion will be used 
for the negative result in \secref{s:negative}. 

Further we have:
\begin{proposition}[Composability of Precise Encodings]\rm
	\label{prop:enc_composability}
	Let %encodings 
	$\enco{\pmap{\cdot}{1}, \tmap{\cdot}{1}, \mapa{\cdot}^{1}}: \tyl{L}_1 \to \tyl{L}_2$
	and 
	$\enco{\pmap{\cdot}{2}, \tmap{\cdot}{2}, \mapa{\cdot}^{2}}: \tyl{L}_2 \to \tyl{L}_3$
	be two precise typed encodings.
	Then their composition, denoted 
	$\enco{\pmap{\cdot}{2} \circ \pmap{\cdot}{1}, \tmap{\cdot}{2} \circ \tmap{\cdot}{1}, \mapa{\cdot}^{2}\circ \mapa{\cdot}^{1}}: \tyl{L}_1 \to \tyl{L}_3$
	is also a precise encoding.
\end{proposition}

\begin{proof}
	Straightforward application of the definition of each property, with the left-to-right direction of
	full abstraction being crucial.\qed
\end{proof}

In \secref{sec:positive} %and \secref{sec:negative}, 
we consider the following concrete instances of typed calculi
(cf.~\defref{d:tcalculus}):


\begin{definition}[Concrete Typed Calculi]\rm
	We instantiate the following typed calculi:
%
	\begin{enumerate}[1.]
		\item	$\tyl{L}_{\HOp}=\calc{\HOp}{{\cal{T}}_1}{\hby{\ell}}{\wb}{\proves}$,
		\item	$\tyl{L}_{\HO}=\calc{\HO}{{\cal{T}}_2}{\hby{\ell}}{\wb}{\proves}$, and
		\item	$\tyl{L}_{\sessp}=\calc{\sessp}{{\cal{T}}_3}{\hby{\ell}}{\wbf}{\proves}$ 
	\end{enumerate}
%
	where: 
	${\cal{T}}_1$, ${\cal{T}}_2$, 
	and ${\cal{T}}_3$
	are sets of types of $\HOp$, $\HO$, and $\sessp$, respectively;
	the typing $\proves$ is defined in 
	\figref{fig:typerulesmy};  
	$\wb$ is as in \defref{def:bisim}; 
	$\wbf$ is as in \defref{def:cbisim}.
\end{definition}


