% !TEX root = main.tex
\section{Related Work}

\paragraph{Expressiveness in Concurrency.}
There is a vast literature on expressiveness studies for process calculi;
the interested reader is referred to~\cite{DBLP:journals/entcs/Parrow08} for a survey
(see also~\cite[\S\,2.3]{PerezPhD10}). 
In particular, the expressive power of the $\pi$-calculus has received much attention.
Studies cover, e.g., 
relationships between first-order and higher-order concurrency~(see, e.g.,~\cite{San923,San96int})
comparisons between 
synchronous and asynchronous communication~(see, e.g.,~\cite{Boudol92,Palamidessi03,BeauxisPV08}),
and
(non)encodability results of different choice operators~(see, e.g.,~\cite{Nestmann00,DBLP:conf/esop/PetersNG13}).
To substantiate claims related to (relative) expressive power,
early works appealed to different definitions of encoding.
%For instance, Palamidessi~\cite{Palamidessi03} defines \emph{uniform encodings} as those encodings which are homomorphic wrt parallel composition, respect renamings, and respect a ``reasonable semantics.''
Later on, frameworks which formalize the notion of encoding 
in abstract terms
have been proposed, 
also stating desirable syntactic and semantic criteria underlying relative expressiveness; 
two proposals are~\cite{DBLP:journals/iandc/Gorla10,DBLP:journals/tcs/FuL10}. 
As such, these frameworks are applicable to different calculi.
They have shown useful to clarify known results and to derive new ones.
Our formulation of (precise) typed encoding (Definition~\ref{def:goodenc}) 
builds upon existing proposals (including~\cite{Palamidessi03,DBLP:journals/iandc/Gorla10,DBLP:conf/icalp/LanesePSS10})
in order to account for the session type systems
associated to the process languages under comparison.




\paragraph{Expressiveness of Higher-Order Process Calculi.}
Thomsen (second order, no types), 
Works by Sangiorgi (untyped and typed encodings, see his book, also hierarchy of process passing)
Perez et al (ICALP 2010),
Xian Xu (WS-FM 2013).

\paragraph{Expressiveness of (Session) Typed Mobile Processes.}
Nobuko on full abstraction for polyadic.
Quaglia and Walker.
encoding PCF into linear pi fully abstractly
Works by Dardha, Demangeon, and others.

\paragraph{Typed Behavioral Equivalences.}
Boreale and Sangiorgi, 
Deng and Hennessy, 
Jeffrey and Rathke, Hennessy and Kouzapas, Schmitt and Lenglet, Pi�rard and Sumii.
Perez et al (bisimilarities for binary sessions), Kouzapas and Yoshida (bisimilarities for binary and multiparty sessions).

\paragraph{Higher-Order Session Types.}
Works by Mostrous, Demangeon, and others.