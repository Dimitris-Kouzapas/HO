% !TEX root = main.tex
\section{Related Work}

\paragraph{Expressiveness in Concurrency.}
There is a vast literature on expressiveness studies for process calculi;
the interested reader is referred to~\cite{DBLP:journals/entcs/Parrow08} for a survey
(see also~\cite[Chp. 2]{PerezPhD10}). 
In particular, the expressive power of the $\pi$-calculus has received much attention.
Notable studies cover, e.g., 
relationships between first-order and higher-order concurrency~(see, e.g.,~\cite{San923,San96int})
comparisons between 
synchronous and asynchronous communication~(see, e.g.,~\cite{Boudol92,Palamidessi03,BeauxisPV08}),
and
(non)encodability results of different choice operators~(see, e.g.,~\cite{Nestmann00,DBLP:conf/esop/PetersNG13}).
To substantiate claims related to (relative) expressive power,
early works appealed to different definitions of encoding.
For instance, Palamidessi~\cite{Palamidessi03} defines \emph{uniform encodings} 
as those encodings which are homomorphic wrt parallel composition,
respect renamings, and respect a ``reasonable semantics.''
The diversity in formulations of encoding made it difficult to establish useful relationships among the various results.
As a remedy to this, frameworks which formalize the notion of encoding 
(and its associated syntactic and semantic criteria) in abstract terms
have been proposed; notable proposals are~\cite{DBLP:journals/iandc/Gorla10,DBLP:journals/tcs/FuL10}. 
These frameworks 
unify criteria underlying relative expressiveness and
are applicable to different calculi.
They have shown useful to clarify known results and to derive new ones.
Unfortunately, 
to our knowledge, existing proposals 
of abstract frameworks for relative expressiveness do not account for type systems
associated to the process languages under comparison.




\paragraph{Expressiveness of Higher-Order Process Calculi.}
Thomsen (second order, no types), 
Works by Sangiorgi (untyped and typed encodings, see his book, also hierarchy of process passing)
Perez et al (ICALP 2010),
Xian Xu (WS-FM 2013).

\paragraph{Expressiveness of (Session) Typed Mobile Processes.}
Nobuko on full abstraction for polyadic.
Quaglia and Walker.
encoding PCF into linear pi fully abstractly
Works by Dardha, Demangeon, and others.

\paragraph{Typed Behavioral Equivalences.}
Boreale and Sangiorgi, 
Deng and Hennessy, 
Jeffrey and Rathke, Hennessy and Kouzapas, Schmitt and Lenglet, Pi�rard and Sumii.
Perez et al (bisimilarities for binary sessions), Kouzapas and Yoshida (bisimilarities for binary and multiparty sessions).

\paragraph{Higher-Order Session Types.}
Works by Mostrous, Demangeon, and others.