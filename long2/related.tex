% !TEX root = main.tex
\section{Related Work}

\paragraph{Expressiveness in Concurrency.}
There is a vast literature on expressiveness studies for process calculi;
we refer to~\cite{DBLP:journals/entcs/Parrow08} for a survey
(see also~\cite[\S\,2.3]{PerezPhD10}). 
In particular, the expressive power of the $\pi$-calculus has received much attention.
Studies cover, e.g., 
relationships between first-order and higher-order concurrency~(see, e.g.,~\cite{San923,San96int}),
comparisons between 
synchronous and asynchronous communication~(see, e.g.,~\cite{Boudol92,Palamidessi03,BeauxisPV08}),
and
(non)encodability issues for different choice operators~(see, e.g.,~\cite{Nestmann00,DBLP:conf/esop/PetersNG13}).
To substantiate claims related to (relative) expressive power,
early works appealed to different definitions of encoding.
%For instance, Palamidessi~\cite{Palamidessi03} defines \emph{uniform encodings} as those encodings which are homomorphic wrt parallel composition, respect renamings, and respect a ``reasonable semantics.''
Later on, 
proposals of abstract 
frameworks which formalize the notion of encoding 
and state associated syntactic and semantic criteria 
were put forward; 
two proposals are~\cite{DBLP:journals/iandc/Gorla10,DBLP:journals/tcs/FuL10}. 
These frameworks are applicable to different calculi, and 
have shown useful to clarify known results and to derive new ones.
Our formulation of (precise) typed encoding (Definition~\ref{def:goodenc}) 
builds upon existing proposals (including~\cite{Palamidessi03,DBLP:journals/iandc/Gorla10,DBLP:conf/icalp/LanesePSS10})
in order to account for the session type systems
associated to the process languages under comparison.




\paragraph{Expressiveness of Higher-Order Process Calculi.}
Early expressiveness studies for higher-order calculi are~\cite{Tho90,San923}; 
more recent works include~\cite{BundgaardHG06,DBLP:conf/icalp/LanesePSS10,DBLP:journals/iandc/LanesePSS11,XuActa2012,DBLP:conf/wsfm/XuYL13}.
Due to the close relationship between higher-order process calculi and functional calculi, 
works devoted to encoding (variants of) the $\lambda$-calculus into (variants of) the $\pi$-calculus (see, e.g.,~\cite{San92,DBLP:journals/tcs/Fu99,DBLP:journals/iandc/YoshidaBH04,BHY,DBLP:conf/concur/SangiorgiX14}) are also worth mentioning.
The work~\cite{San923} gives an encoding of the higher-order $\pi$-calculus
into the first-order $\pi$-calculus which is fully abstract with respect to barbed congruence. 
A basic form of input/output types is used in~\cite{DBLP:journals/tcs/Sangiorgi01}, where the encoding in~\cite{San923} is casted in the asynchronous setting, with output and applications coalesced in a single construct. Building upon~\cite{DBLP:journals/tcs/Sangiorgi01}, 
a simply typed encoding for synchronous processes is given in~\cite{SaWabook}; the reverse encoding (i.e.,  first-order communication into higher-order processes) is also studied there for an asynchronous, localized $\pi$-calculus (only the output capability of names can be sent around).
The work~\cite{San96int} studies hierarchies for calculi with \emph{internal} first-order mobility and 
with higher-order mobility without name-passing (similarly as the subcalculus \HO). 
The hierarchies are based on expressivity: 
formally defined according to the order of types needed in typing, 
they describe different ``degrees of mobility''.
Via fully abstract encodings, it is shown that that name- and process-passing calculi with equal order of types have the same expressiveness.
With respect to these previous results, our approach based on session types 
has several important consequences and allows us to derive new results.  Our study reinforces the intuitive view of ``encodings as protocols'', namely session protocols which enforce precise linear and shared disciplines for names, a distinction not investigated in~\cite{San923,DBLP:journals/tcs/Sangiorgi01}. 
In turn, the linear/shared distinction is central in proper definitions of trigger processes, which are essential to encodings (cf. Definition~\ref{def:enc:HOp_to_FO}) and behavioral equivalences (cf. Definition~\ref{def:XX}).
More interestingly, we showed that $\HO$, a  minimal higher-order session calculus (no name passing, only first-order application) suffices to encode $\sessp$ (the session calculus with name passing) but also 
$\HOp$  and 
its extension  with higher-order applications (denoted $\HOpp$). 
Thus, using session types all these calculi are shown to be equally expressive with fully abstract encodings.
To our knowledge, these are the first expressiveness results of this kind.

Other related works are~\cite{BundgaardHG06,XuActa2012,DBLP:journals/iandc/LanesePSS11}.
The paper~\cite{BundgaardHG06} proposes a fully abstract, continuation-passing style encoding of the 
$\pi$-calculus into Homer, a rich higher-order process calculus with explicit locations, local names, and nested locations.
The work~\cite{XuActa2012} studies the encodability of the higher-order $\pi$-calculus (extended with a relabeling operator) into the first-order $\pi$-calculus; encodings in the reverse direction are also proposed, following \cite{Tho90}.
A minimal calculus of higher-order concurrency is studied in~\cite{DBLP:journals/iandc/LanesePSS11}: it lacks restriction,  name passing, output prefix (so  communication is asynchronous), and constructs for infinite behavior. 
Nevertheless, this calculus (a sublanguage of \HO) is shown to be Turing complete. Moreover, 
strong bisimilarity is decidable and coincides with barbed congruence. 


Our work is closely related in spirit to the expressiveness studies in~\cite{DBLP:conf/icalp/LanesePSS10,DBLP:conf/wsfm/XuYL13}.
In~\cite{DBLP:conf/icalp/LanesePSS10}
the core calculus in~\cite{DBLP:journals/iandc/LanesePSS11} is extended with restriction, output prefix (thus enabling synchronous communication), 
and polyadic communication. It is shown that 
synchronous communication can encode asynchronous communication (as in the first-order setting),
and that process passing polyadicity induces a hierarchy in expressive power (unlike the first-order setting).
A further extension with process abstractions of order one
(functions from processes to processes)
 is shown to strictly add expressive power with respect to passing of processes only.
The paper~\cite{DBLP:conf/wsfm/XuYL13} complements the study in~\cite{DBLP:conf/icalp/LanesePSS10} by deepening on the expressive power of second-order abstractions (similar to \HO). 
In that setting, name and process abstractions are distinguished and contrasted, also considering polyadicity of abstraction parameters (the same kind of polyadicity present in \pHOp). It is shown that polyadicity of process abstraction induces an expressiveness hierarchy. Moreover, it is shown that name abstraction can encode process abstraction, and therefore it may be considered as a more basic mechanism. 
The works~\cite{DBLP:conf/icalp/LanesePSS10,DBLP:conf/wsfm/XuYL13} focus on untyped processes;
therefore, our work complements such previous results by clarifying the status of typeful, resource-aware structured communications in
trigger-based representations of process passing, both in encodings and (typed) behavioral equivalences.

\paragraph{Expressiveness of (Session) Typed Mobile Processes.}
Since types can limit
contexts (environments) where processes can interact, typed equivalences
usually offer {\em coarse} semantics than untyped semantics. 
This work demonstrated the IO-subtyping can equate 
the optimal encoding of the $\lambda$-caclulus by Milner which was not 
in the untyped polyadic $\pi$-calculus \cite{MilnerR:funp}. 
After \cite{PiSa96b}, many works on typed $\pi$-calculi 
have investigated correctness of encodings of known concurrent and
sequential calculi in order to examine semantic
effects of proposed typing systems. 

The type discipline closely related
to session types is a family of linear typing systems. The
work \cite{LinearPi} first proposed a linearly typed barbed congruence and 
reasoned a tail-call optimisation of higher-order functions which are
encoded 
as processes. 
The work \cite{Yoshida96} had
used a bisimulation of graph-based types to prove the full abstraction
of encodings of the polyadic synchronous $\pi$-calculus into the
monadic synchronous $\pi$-calculus. 
Later typed equivalences of a
family of linear and affine calculi \cite{BHY,YBH04,BergerHY05} 
were used to encode 
PCF \cite{Plotkin1977223,Milner19771}, the simply typed $\lambda$-calculi with sums and products, and system F \cite{GirardJY:protyp}
fully abstractly (a fully abstract encoding of the $\lambda$-calculi 
was an open problem in \cite{MilnerR:funp}).  
The work \cite{YHB02} proposed a new bisimilarity
method associated with linear type structure and strong
normalisation. It presented applications to reason secrecy in
programming languages. A subsequent work \cite{HY02} adapted these results
to a practical direction. It proposes new typing
systems for secure higher-order and multi-threaded programming 
languages. 
In these works, typed properties, linearity and liveness, 
play a fundamental role in the analysis. In general, linear types 
are suitable to encode ``sequentiality'' in the sense of 
\cite{HylandJME:fulapi,AbramskyS:fulap}.

Another recent work \cite{DemangeonH11} gives a fully abstract encoding of a 
binary synchronous
session typed calculus into a linearly typed $\pi$-calculus \cite{BHY}.
The work \cite{Dardha:2012:STR:2370776.2370794} also 
uses linear types to 
encode binary session types for the first and higher-order 
$\pi$-calculi \cite{tlca07}, but it does not 
study full abstraction results with respect to 
a behavioural equivalence or bisimulation.
 


\paragraph{Typed Behavioural Equivalences.}
The current work follows the principles for
session type behavioural semantics that were laid
by the previous works of the
authors~\cite{dkphdthesis,DBLP:conf/forte/KouzapasYH11,KY13,DBLP:journals/iandc/PerezCPT14}.
A bisimulation relation is defined on a labelled
transition system that assumes a session typed
observer.
The bisimilarity is characterised by the corresponding
reduction-closed, barb-preserving congruence using a
proof technique that is derived from~\cite{Hennessy07}.
The theory for higher-order session types developed here
differentiates from 
the work in~\cite{dkphdthesis,DBLP:conf/forte/KouzapasYH11,KY13}, which 
considers the behavioural semantics for first-order
binary and multiparty session types.
Also the work \cite{DBLP:journals/iandc/PerezCPT14} studies typed behavioral equivalencies for binary session types. The underlying process languages does not have shared names which, as we have shown, strictly add expressive power. 
Moreover, for this deterministic language, confluence and $\tau$-inertness properties are established.




%The theory for higher-order session type quivalences is more challenging than
%their corresponding first-order bisimulation theory.
To cope with the challenges presented by the higher-order
session theory, 
our approach continues the line of research 
originally drawn by Sangiorgi~\cite{San96H,SangiorgiD:expmpa}
and later improved by Jeffrey and Rathke~\cite{DBLP:journals/lmcs/JeffreyR05}.
The works %Sangiorgi as part of his Ph.D.~research
\cite{San96H,SangiorgiD:expmpa}
introduced the first fully abstract encoding from the higher-order 
$\pi$-calculus to the $\pi$-calculus. 
The replicated triggered process 
is also used in this work for the encoding of \HOp into \sessp (Definition~\ref{def:enc:HOp_to_FO}).
Sangiorgi's encoding is based on the idea of a replicated input guarded process 
(called a trigger process). Operational correspondence for
the triggered encoding is shown using the contextual bisimulation
with first-order labels.
Although contextual bisimilarity has a satisfactory discriminative power,
its use is hindered by the universal quantification on output clauses.
Sangiorgi then proposed \emph{normal bisimilarity}, a tractable  equivalence 
on processes without universal quantification. 
To show the coincidence between contextual and normal bisimilarities, 
the use of triggered processes and bisimilarity is developed in \cite{San96H}.
%The encoding also motivates the definition of a form of
Triggered bisimulation is also defined on first-order labels
where the contextual bisimulation is restricted to arbitrary
trigger substitution rather than arbitrary process substitutions.
The triggered bisimulation was further refined by Jeffrey and
Rathke, who study calculi with recursive types, not addressed in~\cite{San96H,SangiorgiD:expmpa} and
relevant in our work.
They introduce their own version of triggered
bisimulation~\cite{DBLP:journals/lmcs/JeffreyR05}.
%for a full higher-order $\pi$-calculus that allows
%higher-order applications.
Like Sangiorgi's approach, the labelled transition semantics
in~\cite{DBLP:journals/lmcs/JeffreyR05}
observes first-order triggered values instead of
higher-order values, offering a more direct characterization of contextual equivalence
and lifting the restriction to finite types.


There are similarities of the characteristic bisimulation
the triggered bisimulation as defined by Jeffrey and Rathke:
i) the output of a higher-order higher-order value $Q$ on name
$n$ in the triggered bisimulation requires the output of
a fresh trigger name $t$ on channel $n$ 
and then the introduction of a replicated triggered process
in the context of the acting process:
%
\[
	P \by{\news{t} \bactout{n}{t}} P' \Par \repl{} \binp{t}{x} Q
\]
%
The approached used in the characteristic bisimulation is similar
but instead of a replicated triggered process we require a
triggered process that can input any process that can substitute
the higher-order output.
In essence we are transforming a replicated trigger into a process
that is input-prefixed on a fresh name that receives a higher-order
value;
ii) the input of a higher-order the Jeffrey and Rathke approach requires 
the input of a fresh trigger name, which is substituted on the application
variable, thus having a special syntax for triggered application instead
of higher-order applications.  
\[
	P \by{\bactinp{n}{t}} P \subst{t}{x} 
\]
The approach in the characteristic bisimulation observes the triggered value
$\abs{z}\binp{t}{x} \appl{x}{z}$ as an input instead of the
trigger name, thus avoiding the special syntax for triggered applications;
iii) Lastly, in triggered bisimulation a triggered applications
leads into an output observation of the application value over
the fresh trigger name.
\[
	\appl{t}{V} \by{\bactout{t}{V}} \inact
\]
In the characteristic bisimulation instead of observing an 
application and its value as an action we observe the application
value by inhabiting it in a characteristic process.

%The main differences of the triggered
%bisimulation approach comparing to our approach are:
%i) We use observe higher-order values on the LTS in contrast to first-order 
%values in~\cite{DBLP:journals/lmcs/JeffreyR05}.
%ii) In our approach we avoid the replicated triggered process,
%by transforming the output process into a higher-order guarded input.
%iii) The triggered bisimulation gives semantics for higher-order application,
%whereas in our approach we give semantics for first-order applications
%and show that higher-order applications are fully encodable.

%Boreale and Sangiorgi, 
%Deng and Hennessy, 
%Jeffrey and Rathke, Hennessy and Koutavas, Schmitt and Lenglet, Pi\E9rard and Sumii.
%Perez et al (bisimilarities for binary sessions), Kouzapas and Yoshida (bisimilarities for binary and multiparty sessions).
%Bisimilarities for HO processes: \cite{Xu07}.

\paragraph{Higher-Order Session Types.}
Works by Mostrous, Demangeon, and others.

