% !TEX root = main.tex
\section{Related Work}

\paragraph{Expressiveness in Concurrency.}
There is a vast literature on expressiveness studies for process calculi;
the interested reader is referred to~\cite{DBLP:journals/entcs/Parrow08,PerezPhD10} for surveys. 
In particular, the expressive power of the $\pi$-calculus has received much attention.
Notable studies cover, e.g., 
relationships between first-order and higher-order concurrency~(see, e.g.,~\cite{San923,San96int})
comparisons of asynchronous and synchronous communication~(see, e.g.,~\cite{Boudol92,Palamidessi03,BeauxisPV08}),
and
(non)encodability results of different forms of choice operators~(see, e.g.,~\cite{Nestmann00,DBLP:conf/esop/PetersNG13}).
Early works appealed to diverse definitions of encoding to substantiate claims related to (relative) expressive power;
this made it difficult to assess the significance of the various results.
As a remedy to this, abstract approaches to the notion of encoding 
(and its associated syntactic and semantic criteria)
have been proposed~(see, e.g.,~\cite{DBLP:journals/iandc/Gorla10,DBLP:journals/tcs/FuL10}). 
By unifying criteria underlying relative expressiveness, 
these abstract approaches have shown useful to simplify/generalize known results and derive new ones.
Unfortunately, 
to our knowledge, existing proposals 
of abstract frameworks for relative expressiveness do not account for type systems
associated to the process languages under comparison.




\paragraph{Expressiveness of Higher-Order Process Calculi.}
Thomsen (second order, no types), 
Works by Sangiorgi (untyped and typed encodings, see his book, also hierarchy of process passing)
Perez et al (ICALP 2010),
Xian Xu (WS-FM 2013).

\paragraph{Expressiveness of (Session) Typed Mobile Processes.}
Works by Dardha, Demangeon, and others.

\paragraph{Typed Behavioral Equivalences.}
Boreale and Sangiorgi, 
Deng and Hennessy, 
Jeffrey and Rathke, Hennessy and Kouzapas, Schmitt and Lenglet, Pi�rard and Sumii.
Perez et al (bisimilarities for binary sessions), Kouzapas and Yoshida (bisimilarities for binary and multiparty sessions).

\paragraph{Higher-Order Session Types.}
Works by Mostrous, Demangeon, and others.