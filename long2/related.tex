% !TEX root = main.tex
\section{Related Work}

\paragraph{Expressiveness in Concurrency.}
There is a vast literature on expressiveness studies for process calculi;
we refer to~\cite{DBLP:journals/entcs/Parrow08} for a survey
(see also~\cite[\S\,2.3]{PerezPhD10}). 
In particular, the expressive power of the $\pi$-calculus has received much attention.
Studies cover, e.g., 
relationships between first-order and higher-order concurrency~(see, e.g.,~\cite{San923,San96int})
comparisons between 
synchronous and asynchronous communication~(see, e.g.,~\cite{Boudol92,Palamidessi03,BeauxisPV08}),
and
(non)encodability issues for different choice operators~(see, e.g.,~\cite{Nestmann00,DBLP:conf/esop/PetersNG13}).
To substantiate claims related to (relative) expressive power,
early works appealed to different definitions of encoding.
%For instance, Palamidessi~\cite{Palamidessi03} defines \emph{uniform encodings} as those encodings which are homomorphic wrt parallel composition, respect renamings, and respect a ``reasonable semantics.''
Later on, frameworks which formalize the notion of encoding 
in abstract terms
have been proposed, 
also stating desirable syntactic and semantic criteria underlying relative expressiveness; 
two proposals are~\cite{DBLP:journals/iandc/Gorla10,DBLP:journals/tcs/FuL10}. 
These frameworks are applicable to different calculi, and 
have shown useful to clarify known results and to derive new ones.
Our formulation of (precise) typed encoding (Definition~\ref{def:goodenc}) 
builds upon existing proposals (including~\cite{Palamidessi03,DBLP:journals/iandc/Gorla10,DBLP:conf/icalp/LanesePSS10})
in order to account for the session type systems
associated to the process languages under comparison.




\paragraph{Expressiveness of Higher-Order Process Calculi.}
Early expressiveness studies for higher-order calculi are~\cite{Tho90,San923}; 
more recent works include~\cite{BundgaardHG06,DBLP:conf/icalp/LanesePSS10,DBLP:journals/iandc/LanesePSS11,XuActa2012,DBLP:conf/wsfm/XuYL13}.
Due to the close relationship between higher-order process calculi and functional calculi, 
works devoted to encoding (variants of) the $\lambda$-calculus into (variants of) the $\pi$-calculus (see, e.g.,~\cite{San92,DBLP:journals/tcs/Fu99,DBLP:journals/iandc/YoshidaBH04,BHY,DBLP:conf/concur/SangiorgiX14}) are also worth mentioning.
The work~\cite{San923} gives an encoding of the higher-order $\pi$-calculus
into the first-order $\pi$-calculus which is fully abstract with respect to barbed congruence. 
A basic form of input/output types is used in~\cite{DBLP:journals/tcs/Sangiorgi01}, where the encoding in~\cite{San923} is casted in the asynchronous setting, where output and applications are merged in a single construct. Building upon~\cite{DBLP:journals/tcs/Sangiorgi01}, 
a simply typed encoding for synchronous processes is given in~\cite{SaWabook}; the reverse encoding (i.e.,  first-order communication into higher-order processes) is also studied there, although only for an asynchronous, localized $\pi$-calculus (only the output capability of names can be sent around).
The work~\cite{San96int} studies hierarchies for calculi with \emph{internal} first-order mobility and 
with higher-order mobility without name-passing (similarly as the subcalculus \HO). 
The hierarchies are based on expressivity: they describe different ``degrees of mobility'', and are 
formally defined according to the order of types needed in typing.
Via fully abstract encodings, it is shown that that name- and process-passing calculi with equal order of types have the same expressiveness.
With respect to these previous results, our approach based on session types 
has several important consequences and allows to derive new results.  Our study reinforces the intuitive view of ``encodings as protocols'', namely session protocols which enforce precise linear and shared disciplines for names, a distinction not investigated in~\cite{San923,DBLP:journals/tcs/Sangiorgi01}. 
In turn, the linear/shared distinction is central in proper definitions of trigger processes, which are essential to encodings and behavioral equivalences.
More interestingly, we showed that $\HO$, a  minimal higher-order session calculus (no name passing, only first-order application) suffices to encode $\sessp$ (the session calculus with name passing) but also 
$\HOp$  and 
its extension  with higher-order applications (denoted $\HOpp$). 
Thus, using session types all these calculi are shown to be equally expressive with fully abstract encodings.
To our knowledge, these are the first expressiveness results of this kind.

Other related works are~\cite{BundgaardHG06,XuActa2012,DBLP:journals/iandc/LanesePSS11}.
The paper~\cite{BundgaardHG06} proposes a fully abstract, continuation-passing style encoding of the 
$\pi$-calculus into Homer, a higher-order process calculus with explicit locations, local names, and nested locations.
The work~\cite{XuActa2012} studies the encodability of the higher-order $\pi$-calculus (extended with a relabeling operator) into the first-order $\pi$-calculus; encodings in the reverse direction are also proposed.
A minimal calculus of higher-order concurrency is studied in~\cite{DBLP:journals/iandc/LanesePSS11}: it lacks restriction,  name passing, output prefix (asynchronous communication), and constructs for infinite behavior. 
Nevertheless, this calculus (a sublanguage of \HO) is shown to be Turing complete. Moreover, 
strong bisimilarity is decidable and coincides with barbed congruence. 


Our work is closely related in spirit to the expressiveness studies in~\cite{DBLP:conf/icalp/LanesePSS10,DBLP:conf/wsfm/XuYL13}.
In~\cite{DBLP:conf/icalp/LanesePSS10}
the core calculus in~\cite{DBLP:journals/iandc/LanesePSS11} is extended with restriction, output prefix (thus enabling synchronous communication), 
and polyadic communication. It is shown that 
synchronous communication can encode asynchronous communication (as in the first-order setting),
and that process passing polyadicity induces a hierarchy in expressive power (unlike the first-order setting).
A further extension with process abstractions of order one
(functions from processes to processes)
 is shown to strictly add expressive power with respect to passing of processes only.
The paper~\cite{DBLP:conf/wsfm/XuYL13} complements the study in~\cite{DBLP:conf/icalp/LanesePSS10} by deepening on the expressive power of second-order abstractions (similar to \HO). 
In that setting, name and process abstractions are distinguished and contrasted, also considering polyadicity of abstraction parameters (as we have in \pHOp). It is shown that polyadicity of process abstraction induces an expressiveness hierarchy. Moreover, it is shown that name abstraction can encode process abstraction, and therefore it may be considered as a more basic mechanism. 
The works~\cite{DBLP:conf/icalp/LanesePSS10,DBLP:conf/wsfm/XuYL13} focus on untyped processes;
therefore, our work complements such previous results by clarifying the status of structured communications (as delineated by session types) in representability strategies of process passing (such as those based on triggers) both in encodings and (typed) behavioral equivalences; this allows us to clearly establish the use and role of linear/unrestricted names in such strategies.


\paragraph{Expressiveness of (Session) Typed Mobile Processes.}
Since types can limit
contexts (environments) where processes can interact, typed equivalences
usually offer {\em coarse} semantics than untyped semantics. 
This work demonstrated the IO-subtyping can equate 
the optimal encoding of the $\lambda$-caclulus by Milner which was not 
in the untyped polyadic $\pi$-calculus \cite{MilnerR:funp}. 
After \cite{PiSa96b}, many works on typed $\pi$-calculi 
have investigated correctness of encodings of known concurrent and
sequential calculi in order to examine semantic
effects of proposed typing systems. 

The type discipline closely related
to session types is a family of linear typing systems. The
work \cite{LinearPi} first proposed a linearly typed barbed congruence and 
reasoned a tail-call optimisation of higher-order functions which are
encoded 
as processes. 
The work \cite{Yoshida96} had
used a bisimulation of graph-based types to prove the full abstraction
of encodings of the polyadic synchronous $\pi$-calculus into the
monadic synchronous $\pi$-calculus. 
Later typed equivalences of a
family of linear and affine calculi \cite{BHY,YBH04,BergerHY05} 
were used to encode 
PCF \cite{Plotkin1977223,Milner19771}, the simply typed $\lambda$-calculi with sums and products, and system F \cite{GirardJY:protyp}
fully abstractly (a fully abstract encoding of the $\lambda$-calculi 
was an open problem in \cite{MilnerR:funp}).  
The work \cite{YHB02} proposed a new bisimilarity
method associated with linear type structure and strong
normalisation. It presented applications to reason secrecy in
programming languages. A subsequent work \cite{HY02} adapted these results
to a practical direction. It proposes new typing
systems for secure higher-order and multi-threaded programming 
languages. 
In these works, typed properties, linearity and liveness, 
play a fundamental role in the analysis. In general, linear types 
are suitable to encode ``sequentiality'' in the sense of 
\cite{HylandJME:fulapi,AbramskyS:fulap}.

Another recent work \cite{DemangeonH11} gives a fully abstract encoding of a 
binary synchronous
session typed calculus into a linearly typed $\pi$-calculus \cite{BHY}.
The work \cite{Dardha:2012:STR:2370776.2370794} also 
uses linear types to 
encode binary session types for the first and higher-order 
$\pi$-calculi \cite{tlca07}, but it does not 
study full abstraction results with respect to 
a behavioural equivalence or bisimulation.
 


\paragraph{Typed Behavioural Equivalences.}
This work builds on the session type bisimulations principles
that were laid by previous works of the
authors~\cite{dkphdthesis,DBLP:conf/forte/KouzapasYH11,KY13}.
These principles, that are still followed in the current work,
require a typed labelled transition system
that assumes a session typed observer and develops
tractable bisimulation relations that coincide with
the reduction closed, barb preserving congruence. The
proof tecnhiques for soundness and completeness of the bisimulation
are derived from~\cite{Hennessy07}.

The theory for higher-order equivalences is more challenging than
their corresponding first-order bisimulation theory.
In this work we continue the line drawn originally by
Sangiorgi~\cite{San96H,SangiorgiD:expmpa},
and Jeffrey and Rathke~\cite{DBLP:journals/lmcs/JeffreyR05}.
The first fully abstract encoding from the higher order $\pi$-calculus to
the $\pi$-calculus was given by Sangiorgi in his Ph.D.~thesis~\cite{SangiorgiD:expmpa}.
Sangiorgi introduced the idea of a replicated trigger in order to create
arbitrary copies of a higher order value. The core of the triggered encoding 
is used also used in this work to encode the \HOp to \sessp (Definition~\ref{def:enc:HOp_to_FO}).
The trigger encoding gave the motivation to Jeffrey and
Rathke to introduce the trigger bisimulation~\cite{DBLP:journals/lmcs/JeffreyR05}.
The main idea for higher-order output observation in the
triggered bisimulation requires from
a process $P$ to output a fresh
trigger name and then introduce of a replicated triggered process
in its context:
\[
	P \by{\news{t} \bactout{n}{t}} P' \Par \repl{} \binp{t}{x} Q
\]
Our approach for output observation is similar but instead of 
a replicated triggered process we require a triggered process that 
can input any process that can substitute the higher-order output.
The approach for higher-order input observation by Jeffrey and Rathke requires 
to the input of a triggered process annotated with the application value.
\[
	P \by{\bactinp{n}{t}} P \subst{t}{x} 
\]
whereas in this approach we can observe the triggered value
$\abs{z}\btinp{t}{x} \appl{x}{z}$.
Finally an application in triggger bisimulation semantics
observes a trigger over
\dk{continue}

Boreale and Sangiorgi, 
Deng and Hennessy, 
Jeffrey and Rathke, Hennessy and Koutavas, Schmitt and Lenglet, Pi�rard and Sumii.
Perez et al (bisimilarities for binary sessions), Kouzapas and Yoshida (bisimilarities for binary and multiparty sessions).
Bisimilarities for HO processes: \cite{Xu07}.

\paragraph{Higher-Order Session Types.}
Works by Mostrous, Demangeon, and others.
<<<<<<< HEAD

=======
>>>>>>> bf32e0b2aa3007cc242db0b7846b1c6d0d2e36f0
