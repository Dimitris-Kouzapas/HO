% !TEX root = main.tex
\section{Related Work}
\label{sec:related}

\paragraph{Expressiveness in Concurrency.}
There is a vast literature on expressiveness studies for process calculi;
we refer to~\cite{DBLP:journals/entcs/Parrow08} for a survey
(see also~\cite[\S\,2.3]{PerezPhD10}). 
In particular, the expressive power of the $\pi$-calculus has received much attention.
Studies cover, e.g., 
relationships between first-order and higher-order concurrency~(see, e.g.,~\cite{San923,San96int}),
comparisons between 
synchronous and asynchronous communication~(see, e.g.,~\cite{Boudol92,Palamidessi03,BeauxisPV08}),
and
(non)encodability issues for different choice operators~(see, e.g.,~\cite{Nestmann00,DBLP:conf/esop/PetersNG13}).
To substantiate claims related to (relative) expressive power,
early works appealed to different definitions of encoding.
%For instance, Palamidessi~\cite{Palamidessi03} defines \emph{uniform encodings} as those encodings which are homomorphic wrt parallel composition, respect renamings, and respect a ``reasonable semantics.''
Later on, 
proposals of abstract 
frameworks which formalize the notion of encoding 
and state associated syntactic and semantic criteria 
were put forward; 
two proposals are~\cite{DBLP:journals/iandc/Gorla10,DBLP:journals/tcs/FuL10}. 
These frameworks are applicable to different calculi, and 
have shown useful to clarify known results and to derive new ones.
Our formulation of (precise) typed encoding (Definition~\ref{def:goodenc}) 
builds upon existing proposals (including~\cite{Palamidessi03,DBLP:journals/iandc/Gorla10,DBLP:conf/icalp/LanesePSS10})
in order to account for the session type systems
associated to the process languages under comparison.


\paragraph{Expressiveness of Higher-Order Process Calculi.}
Early expressiveness studies for higher-order calculi are~\cite{Tho90,San923}; 
more recent works include~\cite{BundgaardHG06,DBLP:conf/icalp/LanesePSS10,DBLP:journals/iandc/LanesePSS11,XuActa2012,DBLP:conf/wsfm/XuYL13}.
Due to the close relationship between higher-order process calculi and functional calculi, 
works devoted to encoding (variants of) the $\lambda$-calculus into (variants of) the $\pi$-calculus (see, e.g.,~\cite{San92,DBLP:journals/tcs/Fu99,DBLP:journals/iandc/YoshidaBH04,BHY,DBLP:conf/concur/SangiorgiX14}) are also worth mentioning.
The work~\cite{San923} gives an encoding of the higher-order $\pi$-calculus
into the first-order $\pi$-calculus which is fully abstract with respect to barbed congruence. 
A basic form of input/output types is used in~\cite{DBLP:journals/tcs/Sangiorgi01}, where the encoding in~\cite{San923} is casted in the asynchronous setting, with output and applications coalesced in a single construct. Building upon~\cite{DBLP:journals/tcs/Sangiorgi01}, 
a simply typed encoding for synchronous processes is given in~\cite{SaWabook}; the reverse encoding (i.e.,  first-order communication into higher-order processes) is also studied there for an asynchronous, localized $\pi$-calculus (only the output capability of names can be sent around).
The work~\cite{San96int} studies hierarchies for calculi with \emph{internal} first-order mobility and 
with higher-order mobility without name-passing (similarly as the subcalculus \HO). 
The hierarchies are based on expressivity: 
formally defined according to the order of types needed in typing, 
they describe different ``degrees of mobility''.
Via fully abstract encodings, it is shown that that name- and process-passing calculi with equal order of types have the same expressiveness.
With respect to these previous results, our approach based on session types 
has several important consequences and allows us to derive new results.  Our study reinforces the intuitive view of ``encodings as protocols'', namely session protocols which enforce precise linear and shared disciplines for names, a distinction not investigated in~\cite{San923,DBLP:journals/tcs/Sangiorgi01}. 
In turn, the linear/shared distinction is central in proper definitions of trigger processes, which are essential to encodings (cf. Definition~\ref{def:enc:HOp_to_FO}) and behavioral equivalences (cf. Definition~\ref{def:XX}).
More interestingly, we showed that $\HO$, a  minimal higher-order session calculus (no name passing, only first-order application) suffices to encode $\sessp$ (the session calculus with name passing) but also 
$\HOp$  and 
its extension  with higher-order applications (denoted $\HOpp$). 
Thus, using session types all these calculi are shown to be equally expressive with fully abstract encodings.
To our knowledge, these are the first expressiveness results of this kind.

Other related works are~\cite{BundgaardHG06,XuActa2012,DBLP:journals/iandc/LanesePSS11}.
The paper~\cite{BundgaardHG06} proposes a fully abstract, continuation-passing style encoding of the 
$\pi$-calculus into Homer, a rich higher-order process calculus with explicit locations, local names, and nested locations.
The work~\cite{XuActa2012} studies the encodability of the higher-order $\pi$-calculus (extended with a relabeling operator) into the first-order $\pi$-calculus; encodings in the reverse direction are also proposed, following \cite{Tho90}.
A minimal calculus of higher-order concurrency is studied in~\cite{DBLP:journals/iandc/LanesePSS11}: it lacks restriction,  name passing, output prefix (so  communication is asynchronous), and constructs for infinite behavior. 
Nevertheless, this calculus (a sublanguage of \HO) is shown to be Turing complete. Moreover, 
strong bisimilarity is decidable and coincides with barbed congruence. 


Our work is closely related in spirit to the expressiveness studies in~\cite{DBLP:conf/icalp/LanesePSS10,DBLP:conf/wsfm/XuYL13}.
In~\cite{DBLP:conf/icalp/LanesePSS10}
the core calculus in~\cite{DBLP:journals/iandc/LanesePSS11} is extended with restriction, output prefix (thus enabling synchronous communication), 
and polyadic communication. It is shown that 
synchronous communication can encode asynchronous communication (as in the first-order setting),
and that process passing polyadicity induces a hierarchy in expressive power (unlike the first-order setting).
A further extension with process abstractions of order one
(functions from processes to processes)
 is shown to strictly add expressive power with respect to passing of processes only.
The paper~\cite{DBLP:conf/wsfm/XuYL13} complements the study in~\cite{DBLP:conf/icalp/LanesePSS10} by deepening on the expressive power of second-order abstractions (similar to \HO). 
In that setting, name and process abstractions are distinguished and contrasted, also considering polyadicity of abstraction parameters (the same kind of polyadicity present in \pHOp). It is shown that polyadicity of process abstraction induces an expressiveness hierarchy. Moreover, it is shown that name abstraction can encode process abstraction, and therefore it may be considered as a more basic mechanism. 
The works~\cite{DBLP:conf/icalp/LanesePSS10,DBLP:conf/wsfm/XuYL13} focus on untyped processes;
therefore, our work complements such previous results by clarifying the status of typeful, resource-aware structured communications in
trigger-based representations of process passing, both in encodings and (typed) behavioral equivalences.

\paragraph{Expressiveness of (Session) Typed Mobile Processes.}
Since types can limit
contexts (environments) where processes can interact, typed equivalences
usually offer {\em coarse} semantics than untyped semantics. 
This work demonstrated the IO-subtyping can equate 
the optimal encoding of the $\lambda$-caclulus by Milner which was not 
in the untyped polyadic $\pi$-calculus \cite{MilnerR:funp}. 
After \cite{PiSa96b}, many works on typed $\pi$-calculi 
have investigated correctness of encodings of known concurrent and
sequential calculi in order to examine semantic
effects of proposed typing systems. 

The type discipline closely related
to session types is a family of linear typing systems. The
work \cite{LinearPi} first proposed a linearly typed barbed congruence and 
reasoned a tail-call optimisation of higher-order functions which are
encoded 
as processes. 
The work \cite{Yoshida96} had
used a bisimulation of graph-based types to prove the full abstraction
of encodings of the polyadic synchronous $\pi$-calculus into the
monadic synchronous $\pi$-calculus. 
Later typed equivalences of a
family of linear and affine calculi \cite{BHY,YBH04,BergerHY05} 
were used to encode 
PCF \cite{Plotkin1977223,Milner19771}, the simply typed $\lambda$-calculi with sums and products, and system F \cite{GirardJY:protyp}
fully abstractly (a fully abstract encoding of the $\lambda$-calculi 
was an open problem in \cite{MilnerR:funp}).  
The work \cite{YHB02} proposed a new bisimilarity
method associated with linear type structure and strong
normalisation. It presented applications to reason secrecy in
programming languages. A subsequent work \cite{HY02} adapted these results
to a practical direction. It proposes new typing
systems for secure higher-order and multi-threaded programming 
languages. 
In these works, typed properties, linearity and liveness, 
play a fundamental role in the analysis. In general, linear types 
are suitable to encode ``sequentiality'' in the sense of 
\cite{HylandJME:fulapi,AbramskyS:fulap}.

Another recent work \cite{DemangeonH11} gives a fully abstract encoding of a 
binary synchronous
session typed calculus into a linearly typed $\pi$-calculus \cite{BHY}.
The work \cite{Dardha:2012:STR:2370776.2370794} also 
uses linear types to 
encode binary session types for the first and higher-order 
$\pi$-calculi \cite{tlca07}, but it does not 
study full abstraction results with respect to 
a behavioural equivalence or bisimulation.
 


\paragraph{Typed Behavioural Equivalences.}
The current work follows the principles for
session type behavioural semantics that were laid
by the previous works of the
authors~\cite{dkphdthesis,DBLP:conf/forte/KouzapasYH11,KY13,DBLP:journals/iandc/PerezCPT14}.
A bisimulation relation is defined on a labelled
transition system that assumes a session typed
observer.
The bisimilarity is characterised by the corresponding
reduction-closed, barb-preserving congruence using a
proof technique that is derived from~\cite{Hennessy07}.
The theory for higher-order session types developed here
differentiates from 
the work in~\cite{dkphdthesis,DBLP:conf/forte/KouzapasYH11,KY13}, which 
considers the behavioural semantics for first-order
binary and multiparty session types.
Also the work \cite{DBLP:journals/iandc/PerezCPT14} studies typed behavioral equivalencies for binary session types.
The underlying process languages does not have shared names which, as we have shown, strictly add expressive power. 
Moreover, for this deterministic language, confluence and $\tau$-inertness properties are established.

%The theory for higher-order session type quivalences is more challenging than
%their corresponding first-order bisimulation theory.
To cope with the challenges presented by the higher-order
session theory, 
our approach continues the line of research 
originally drawn by Sangiorgi~\cite{San96H,SangiorgiD:expmpa}
and later improved by Jeffrey and Rathke~\cite{DBLP:journals/lmcs/JeffreyR05}.
The works %Sangiorgi as part of his Ph.D.~research
\cite{San96H,SangiorgiD:expmpa}
introduced the first fully abstract encoding from the higher-order 
$\pi$-calculus to the $\pi$-calculus. 
The replicated triggered process 
is also used in this work for the encoding of \HOp into \sessp (Definition~\ref{def:enc:HOp_to_FO}).
Sangiorgi's encoding is based on the idea of a replicated input guarded process 
(called a trigger process). Operational correspondence for
the triggered encoding is shown using the contextual bisimulation
with first-order labels.
Although contextual bisimilarity has a satisfactory discriminative power,
its use is hindered by the universal quantification on output clauses.
Sangiorgi then proposed \emph{normal bisimilarity}, a tractable  equivalence 
on processes without universal quantification. 
To show the coincidence between contextual and normal bisimilarities, 
the use of triggered processes and bisimilarity is developed in \cite{San96H}.
%The encoding also motivates the definition of a form of
Triggered bisimulation is also defined on first-order labels
where the contextual bisimulation is restricted to arbitrary
trigger substitution rather than arbitrary process substitutions.
The triggered bisimulation was further refined by Jeffrey and
Rathke, who study calculi with recursive types, not addressed in~\cite{San96H,SangiorgiD:expmpa} and
relevant in our work.
They introduce their own version of a
bisimulation~\cite{DBLP:journals/lmcs/JeffreyR05}
based on a LTS which is extended with trigger meta-notation.
%for a full higher-order $\pi$-calculus that allows
%higher-order applications.
Like Sangiorgi's approach, the labelled transition semantics
in~\cite{DBLP:journals/lmcs/JeffreyR05}
observes first-order triggered values instead of
higher-order values, offering a more direct characterization of contextual equivalence
and lifting the restriction to finite types.


There are similarities and differences between of the characteristic bisimulation
and the bisimulation as defined by Jeffrey and Rathke
(below we use the meta-notation adopted in~\cite{DBLP:journals/lmcs/JeffreyR05}):
%
\begin{enumerate}[i)]
	\item	The output of a higher-order value $\abs{x}{Q}$ on name
		$n$ in Jeffrey\&Rathke approach requires the output of
		a fresh trigger name $t$ (notation $\tau_t$)
		on channel $n$ 
		and then the introduction of a replicated triggered process
		(notation $(t \Leftarrow (x) Q)$)
		in the context of the acting process:
		%
		\[
			P \by{\news{t} \bactout{n}{\tau_{t}}} P' \Par (t \Leftarrow (x) Q) \by{\bactinp{t}{v}} P' \Par \appl{(x) Q}{v} \Par (t \Leftarrow (x) Q) 
		\]
		%
		In the characteristic bisimulation approach we only observe
		an output of a value that can be either first- or higher-order:
		%
		\[
			P \hby{\bactout{n}{V}} P' 
		\]
		%
		with $V = \abs{x}{Q}$ or $V = m$.
		A non-replicated triggered process appears in
		the parallel context of the acting process when
		we compare two processes for behavioural equality
		(cf.~characteristic bisimulation Definition~\ref{def:FO_bisim}),
		$P' \Par \htrigger{t}{\abs{x}{Q}}$.
		In fact using the LTS in
		Definition~\ref{def:typed_transition} we can get:
		%
		\begin{eqnarray*}
			P' \Par \htrigger{t}{\abs{x}{Q}}
			&\by{\abs{z}{\binp{z}{y} \repl{} \binp{t}{x} \appl{y}{x}}}&
			P' \Par \newsp{s}{\binp{s}{y} \repl{} \binp{t}{x} \appl{y}{x} \Par \bout{s}{\abs{x}{Q}} \inact}\\
			&\by{\tau}&
			P' \Par \repl{}\binp{t}{y} \appl{\abs{x}{Q}}{y}
		\end{eqnarray*}
		%
		that simulates the Jeffrey\&Rathke approach.

		The characteristic bisimulation differentiates from
		the Jeffrey\&Rathke approach:
		\begin{itemize}
			\item	The typed LTS predicts the case of linear
				output values and will never allow replication
				of such a value;
				if $V$ is linear the input action would have no replication
				operator, as
				$\abs{z}{\binp{z}{y} \binp{t}{x} \appl{y}{x}}$.

			\item	The characteristic bisimulation introduces a uniform approach
				not only for
				higher-order values but for first-order values
				as well. A triggered process can accept any
				process that can substitute a first-order value as well.
				This is derived from the fact that the $\HOp$
				calculus makes no use of a matching operator, in contrast
				to the calculus defined in \cite{DBLP:journals/lmcs/JeffreyR05})
				where name matching is crucial to prove completness
				of the bisimilarity relation.
				We chose not to include the matching operator
				because of the requirement of a minimal calculus.
				In the lack of matching we use types to inhabit
				a value so we can observe its simplest interaction
				with the process environment.

			\item	The \HOp calculus requires only first-order
				applications. Higher-order applications,
				as in the Jeffrey\&Rathke work,
				are presented as an extension in the \HOpp
				calculus.

			\item	The trigger process is non-replicated. In fact
				the trigger process transforms guards the output
				value with a higher-order input prefix. The
				functionality of the input is used to
				simulate the contextual bisimilarity that subsumes
				the replicated trigger approach.
				The transformation of an output action as an input
				action allows for treating an output
				using the restricted LTS~\ref{def:restricted_typed_transition}:
				%
				\[
					P' \Par \htrigger{t}{\abs{x}{Q}} \hby{\bactinp{t}{\abs{x}{\mapchar{U}{x}}}}
					P' \Par \news{s}{ \appl{\mapchar{U}{x}}{s} \Par \bout{\dual{s}}{\abs{x}{Q}} \inact}
				\]
		\end{itemize}
		%
		%In essence we are transforming a replicated trigger into a process
		%that is input-prefixed on a fresh name that receives a higher-order
		%value;

	\item	The input of a higher-order value in the Jeffrey\&Rathke approach requires 
		the input of a fresh trigger name, which is substituted on the application
		variable, thus having a meta-suntax for triggered application instead
		of higher-order applications:
		%
		\[
			\binp{n}{x} P \by{\bactinp{n}{\tau_k}} \appl{\abs{x}{P}}{\tau_k} \by{\tau} P \subst{x}{\tau_k} 
		\]
		%
		with every instance of process variable $x$ in $P$ being substituted
		with trigger value $\tau_k$ to give a process of the form $\appl{\tau_k}{x}$.
		The approach in the characteristic bisimulation observes the
		triggered value
		$\abs{z}\binp{t}{x} \appl{x}{z}$ as an input instead of the
		trigger name:
		%
		\[
			P \hby{\bactinp{n}{\abs{z}\binp{t}{x} \appl{x}{z}}} P \subst{\abs{z}\binp{t}{x} \appl{x}{z}}{x}
		\]
		%
		with applications being transformed to
		$\abs{z}{\binp{t}{x} \appl{x}{z}}{v}$
		Note that in the characteristic bisimulation semantics
		we can also observe a characteristic process as an input.
		
	\item 	Triggered application in the Jeffrey\&Rahtke
		are observe using an output
		lead into an output observation of the
		application value over
		the fresh trigger name.
		%
		\[
			\appl{\tau_k}{v} \by{\bactout{k}{v}} \inact
		\]
		%
		In the characteristic bisimulation instead of observing an 
		application and its value as an action we observe:
		i) the name of the trigger through the trigger value
		application; and ii) the application
		value by inhabiting it in the characteristic value
		and observing the interaction of the corresponding
		characteristic process with its environment.
		%
		\begin{eqnarray*}
			\appl{\abs{z}{\binp{t}{x} \appl{x}{z}}}{v} &\by{\tau}& \binp{t}{x} \appl{x}{v}
			\by{\bactinp{t}{\abs{x}{\mapchar{U}{x}}}}
			\appl{\abs{x}{\mapchar{U}{x}}}{v}
			\by{\tau} \mapchar{U}{x} \subst{n}{x}
		\end{eqnarray*}
		%
\end{enumerate}

%The main differences of the triggered
%bisimulation approach comparing to our approach are:
%i) We use observe higher-order values on the LTS in contrast to first-order 
%values in~\cite{DBLP:journals/lmcs/JeffreyR05}.
%ii) In our approach we avoid the replicated triggered process,
%by transforming the output process into a higher-order guarded input.
%iii) The triggered bisimulation gives semantics for higher-order application,
%whereas in our approach we give semantics for first-order applications
%and show that higher-order applications are fully encodable.

%Boreale and Sangiorgi, 
%Deng and Hennessy, 
%Jeffrey and Rathke, Hennessy and Koutavas, Schmitt and Lenglet, Pi\E9rard and Sumii.
%Perez et al (bisimilarities for binary sessions), Kouzapas and Yoshida (bisimilarities for binary and multiparty sessions).
%Bisimilarities for HO processes: \cite{Xu07}.

Moving on to other works, Sangiorgi et al.~\cite{DBLP:conf/lics/SangiorgiKS07}, use a higher-order LTS 
to define an arguably complex bisimulation relation that stores the knowledge known to
the observer, thus the observation actions are based on the observer's knowledge
at any given time. 
The environmental bisimulation approach is simplified by Koutavas and
Hennessy in~\cite{DBLP:journals/cl/KoutavasH12,DBLP:conf/esop/KoutavasH11}
with the introduction
of a mapping from constants to higher-order values. This
technique allows for the observation of first-order values instead
of higher-order values. It differs from the approaches
in~\cite{San96H,DBLP:journals/lmcs/JeffreyR05} because the
mapping between higher and first order values is no longer implicit.

\paragraph{Higher-Order Session Types.}
The session type discipline developed for the $\HOp$ is a subset
of the session type discipline
for higher-order session processes developed by Mostrous and Yoshida
in~\cite{tlca07,mostrous09sessionbased}.
Mostrous and Yoshida in~\cite{tlca07} develop a full higher order session calculus
with process abstractions and process applications and extend it
in~\cite{mostrous09sessionbased} in the asynchronous setting.
The session type
system considered is influenced by the type systems for $\lambda$-calculi and
uses type syntax of the form $U_1 \rightarrow U_2 \dots U_n \rightarrow \Proc$
for shared values and $(U_1 \rightarrow U_2 \dots U_n \rightarrow \Proc)^{1}$
for linear values.
In the current setting such a typed is expressed only in $\HOpp$
terms using the type $\shot{U}$ (respectively, $\lhot{U}$)
with $U$ being a nested higher-order type.
In the $\HOp$ session type system we only use types of the form
$\shot{C}$ and $\lhot{C}$ with $C$ being a first-order type.
Nevertheless, we show that
the calulus presented in~\cite{tlca07} can be exrpessed in terms of a the
core $\HO$ sub-calculus.

