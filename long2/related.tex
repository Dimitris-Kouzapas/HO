% !TEX root = main.tex
\section{Related Work}

\paragraph{Expressiveness in Concurrency.}
There is a vast literature on expressiveness studies for process calculi;
the interested reader is referred to~\cite{DBLP:journals/entcs/Parrow08} for a survey
(see also~\cite[\S\,2.3]{PerezPhD10}). 
In particular, the expressive power of the $\pi$-calculus has received much attention.
Studies cover, e.g., 
relationships between first-order and higher-order concurrency~(see, e.g.,~\cite{San923,San96int})
comparisons between 
synchronous and asynchronous communication~(see, e.g.,~\cite{Boudol92,Palamidessi03,BeauxisPV08}),
and
(non)encodability results of different choice operators~(see, e.g.,~\cite{Nestmann00,DBLP:conf/esop/PetersNG13}).
To substantiate claims related to (relative) expressive power,
early works appealed to different definitions of encoding.
%For instance, Palamidessi~\cite{Palamidessi03} defines \emph{uniform encodings} as those encodings which are homomorphic wrt parallel composition, respect renamings, and respect a ``reasonable semantics.''
Later on, frameworks which formalize the notion of encoding 
in abstract terms
have been proposed, 
also stating desirable syntactic and semantic criteria underlying relative expressiveness; 
two proposals are~\cite{DBLP:journals/iandc/Gorla10,DBLP:journals/tcs/FuL10}. 
These frameworks are applicable to different calculi, and 
have shown useful to clarify known results and to derive new ones.
Our formulation of (precise) typed encoding (Definition~\ref{def:goodenc}) 
builds upon existing proposals (including~\cite{Palamidessi03,DBLP:journals/iandc/Gorla10,DBLP:conf/icalp/LanesePSS10})
in order to account for the session type systems
associated to the process languages under comparison.




\paragraph{Expressiveness of Higher-Order Process Calculi.}
Early expressiveness studies for higher-order process calculi are~\cite{San923,San96int}; 
more recent works include~\cite{BundgaardHG06,DBLP:conf/icalp/LanesePSS10,DBLP:journals/iandc/LanesePSS11,XuActa2012,DBLP:conf/wsfm/XuYL13}.
Due to the close relationship between higher-order process calculi and functional calculi, 
works devoted to encoding (variants of) the $\lambda$-calculus into (variants of) the $\pi$-calculus (see, e.g.,~\cite{San92,DBLP:journals/tcs/Fu99,DBLP:journals/iandc/YoshidaBH04,BHY,DBLP:conf/concur/SangiorgiX14}) are also worth mentioning.
The work~\cite{San923} defines the representability  of the higher-order $\pi$-calculus
into the first-order $\pi$-calculus, including also full abstraction with respect to barbed congruence. 
In~\cite{San923} a basic sorting mechanism is included; a representability result using input/output types is described in~\cite{SaWabook}. 
Our work thus complements this representability result for a  type structure based on sessions: this not only reinforces the view of ``encodings as protocols'', 
it also allows to explicitly describe, using types, the representability result in~\cite{San923} in terms of clean linear and shared disciplines for names, an aspect not present in~\cite{San923}. In turn, the linear/shared distinction is central in proper definitions of trigger processes (which are essential to encodings and behavioral equivalences).

A hierarchy of process-passing calculi (without name passing) is studied in~\cite{San96int} and is shown to be fully abstract with respect to a hierarchy with internal mobility in the first-order setting; a typed system enforces different degrees of mobility for names in the first-order setting.
The paper~\cite{BundgaardHG06} proposes a fully abstract, continuation-passing style encoding of the 
$\pi$-calculus into Homer, a higher-order process calculus with explicit locations, local names, and nested locations.
The work~\cite{XuActa2012} studies the encodability of the higher-order $\pi$-calculus (extended with a relabeling operator) into the first-order $\pi$-calculus; encodings in the reverse direction are also proposed.
A minimal calculus of higher-order concurrency is studied in~\cite{DBLP:journals/iandc/LanesePSS11}: it lacks restriction,  name passing, output prefix (asynchronous communication), and constructs for infinite behavior. Still, the calculus is shown to be Turing complete. Moreover, 
strong bisimilarity is decidable and coincides with barbed congruence. 


Our work is closely related in spirit to the expressiveness studies in~\cite{DBLP:conf/icalp/LanesePSS10,DBLP:conf/wsfm/XuYL13}.
In~\cite{DBLP:conf/icalp/LanesePSS10}
the core calculus in~\cite{DBLP:journals/iandc/LanesePSS11} is extended with restriction, output prefix (thus enabling synchronous communication), 
and polyadic communication. It is shown that 
synchronous communication can encode asynchronous communication (as in the first-order setting),
and that in the absence of name-passing process passing polyadicity induces a hierarchy in expressive power (unlike the first-order setting).
A further extension with process abstractions of order one
(functions from processes to processes)
 is shown to strictly add expressive power with respect to passing of processes only.
The paper~\cite{DBLP:conf/wsfm/XuYL13} complements the study in~\cite{DBLP:conf/icalp/LanesePSS10} by deepening on the expressive power of abstraction. Name and process abstractions are distinguished and contrasted, also considering polyadicity of abstraction parameters. It is shown that polyadicity of process abstraction induces an expressiveness hierarchy. Moreover, it is shown that name abstractions can encode process abstractions, and therefore may be considered as more basic mechanisms. 
The works~\cite{DBLP:conf/icalp/LanesePSS10,DBLP:conf/wsfm/XuYL13} focus on untyped processes;
therefore, our work complements such previous results by clarifying the status of structured communications (as delineated by session types) in representability strategies of process passing (such as those based on triggers) both in encodings and (typed) behavioral equivalences; this allows us to clearly establish the use and role of linear/unrestricted names in such strategies.


\paragraph{Expressiveness of (Session) Typed Mobile Processes.}
Nobuko on full abstraction for polyadic.
Quaglia and Walker.
encoding PCF into linear pi fully abstractly
Works by Dardha, Demangeon, and others.

\paragraph{Typed Behavioral Equivalences.}
Boreale and Sangiorgi, 
Deng and Hennessy, 
Jeffrey and Rathke, Hennessy and Kouzapas, Schmitt and Lenglet, Pi�rard and Sumii.
Perez et al (bisimilarities for binary sessions), Kouzapas and Yoshida (bisimilarities for binary and multiparty sessions).
Bisimilarities for HO processes: \cite{Xu07}.

\paragraph{Higher-Order Session Types.}
Works by Mostrous, Demangeon, and others.