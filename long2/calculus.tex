% !TEX root = main.tex

\section{The Full Higher-Order Session $\pi$-Calculus (\HOp)}
\label{sec:calculus}

We introduce the 
\emph{Higher-Order Session $\pi$-Calculus} (\HOp).
\HOp includes both name- and abstraction-passing operations
as well as recursions operations; it corresponds to a subcalculus 
of the language
studied by Mostrous and Yoshida
in~\cite{tlca07}. 
Following the literature~\cite{DBLP:journals/lmcs/JeffreyR05},
for simplicity of the presentation
we concentrate on the second-order call-by-value \HOp.  
(In \secref{sec:extension} we consider the extension of 
\HOp with general higher-order abstractions 
and polyadicity in name-passing/abstractions.)
We also introduce two subcalculi of \HOp.
In particular, we define the 
core higher-order session
calculus (\HO), which 
%. The \HO calculus is  minimal: it 
includes constructs for shared name synchronisation and 
constructs for session establish\-ment/communication and 
(monadic) name-abstraction, but lacks name-passing and recursion.

Although minimal, in \secref{s:expr}
the abstraction-passing capabilities of \HOp will prove 
expressive enough to capture key features of session communication, 
such as delegation and recursion.

\subsection{Syntax} 

The syntax for $\HOp$ processes is given in \figref{fig:syntax}.

\myparagraph{Identifiers}
We use $a,b,c, \dots$ to range over shared names, and
$s, \dual{s}, \dots$ to range over session names
whereas $m, n, t, \dots$ range over shared or session names.
We define dual session endpoints $\dual{s}$,
with the dual operator defined as
$\dual{\dual{s}} = s$ and $\dual{a} = a$.
Intuitively, names $s$ and $\dual{s}$ are dual (two) \emph{endpoints}.
Name and abstraction variables are uniformly denoted with $x, y, z, \dots$;
we reserve $k$ for name variables and we sometimes write $\underline{x}$ for abstraction variables.
Recursive variables are denoted with $\varp{X}, \varp{Y} \dots$.
An abstraction $\abs{x}{P}$ is a process $P$ with bound variable $x$.
Symbols $u, v, \dots$ range over names or variables. Furthermore
we use $V, W, \dots$ to denote transmittable values; either channels $u, v$ or
abstractions.

\myparagraph{Terms} 
The name-passing constructs of \HOp include the
$\pi$-calculus prefixes for sending and receiving values $V$.
Process $\bout{u}{V} P$ denotes the output of value $V$
over channel $u$, with continuation $P$;
process $\binp{u}{x} P$ denotes the input prefix on channel $u$ of a value
that it is going to be substituted on variable $x$ in continuation $P$. 
Recursion is expressed by the primitive recursor $\recp{X}{P}$,
which binds the recursive variable $\varp{X}$ in process $P$.
Process $\appl{V}{u}$ is the application
process; it binds channel $u$ on the abstraction $V$.
Prefix $\bsel{u}{l} P$ selects label $l$ on channel $u$
and then behaves as $P$.
Given $i \in I$ process $\bbra{u}{l_i: P_i}_{i \in I}$ offers a choice
on labels $l_i$ with continuation $P_i$.
The calculus also includes standard constructs for 
the inactive process $\inact$, 
parallel composition $P_1 \Par P_2$, and 
name restriction $\news{n} P$.
Session name restriction $\news{s} P$ simultaneously 
binds endpoints $s$ and $\dual{s}$ in $P$.
We use $\fv{P}$ and $\fn{P}$ to denote a set of free 
variables and names, respectively; 
and assume $V$ in $\bout{u}{V}{P}$ does not include free recursive 
variables $\rvar{X}$. 
Furthremore, a well-formed process relies on assumptions for
guarded recursive processes.
If $\fv{P} = \emptyset$, we call $P$ {\em closed}.
We write $\mathcal{P}$ for the set of all well-formed
processes.


\begin{figure}[t!]
%	\begin{tabular}{l|l}
		\begin{tabular}{lrcl}
			1. &	$P,Q$
			 	&$\bnfis$&	$\bout{k}{k'} P \bnfbar \binp{k}{x} P \bnfbar \rvar{X} \bnfbar \recp{X}{P}$ \\
			2. & 	&$\bnfbar$&	$\bbout{k}{\abs{x} Q} P \bnfbar \binp{k}{X} P \bnfbar \appl{X}{k}$ \\ 
			3. & 	&$\bnfbar$&	$\bsel{k}{l} P \bnfbar \bbra{k}{l_i:P_i}_{i \in I} \bnfbar 
						P_1 \Par P_2 \bnfbar \news{s} P \bnfbar \inact$ \\
			4. & 	&$\bnfbar$&	$\bout{k}{\tilde{k'}} P \bnfbar \binp{k}{\tilde{x}} P$\\
			5. & 	&$\bnfbar$&	$\bout{k}{\abs{\tilde{x}} Q} P \bnfbar \appl{X}{\tilde{k}}$\\
		&	Names & : & 	$S = \set{a, b, c, m, n, s, t, \dots} \qquad \dual{S} = \set{\dual{n} \setbar n \in S} \qquad N = S \cup \dual{S}$\\
		&	Variables & : &	$\mathsf{Var} = \set{x,y,z, \dots} \qquad \mathsf{PVar} = \set{\varp{X}, \varp{Y}, \varp{Z}, \dots}
					\qquad \mathsf{RVar} = \set{r, \dots}$\\
		&		& &	$\mathsf{Vars} = \mathsf{Var} \cup \mathsf{PVar} \cup \mathsf{RVar}$\\
		&	Abstractions & : & $\mathsf{Abs} = \set{\abs{x}{P} \setbar P \textrm{ is a process}}$\\
		&		& &	$k \in N \cup \mathsf{Var} \quad \qquad V \in N^n \cup \mathsf{Vars}^n \cup \mathsf{Abs}\ \forall n$
		\end{tabular}
	\caption{Syntax for $\HOp$ \label{fig:syntax}}
\end{figure}



\subsection{Sub-calculi}

\label{subsec:subcalculi}

We identify two main sub-calculi of $\HOp$
that will form the basis of our study:
%
\begin{definition}[Sub-calculi of \HOp]\rm
	We let $\CAL \in \set{\HOp, \HO, \sessp}$ with:
%
	\begin{enumerate}[-]
		\item	{\em Core higher-order session calculus (\HO)}:
			The sub-calculus \HO uses only abstraction passing, i.e.~values
			in \figref{fig:syntax}
			are defined as in the non-gray syntax;
			$V \bnfis \abs{x} P$ and does not use the primitive
			recursion constructs, $\varp{X}$ and $\recp{X}{P}$.

		\item	{\em Session $\pi$-calculus (\sessp)}:
			The sub-calculus \sessp uses only name-passing constructs, i.e.~values
			in \figref{fig:syntax}
			are defined as $V \bnfis u$, and does not use the application process,
			$\appl{x}{u}$.
	\end{enumerate}
%
	We write $\CAL^\minussh$ to denote a sub-calculus without shared names,
	i.e.~indentifiers in \figref{fig:syntax} are defined as $u,v \bnfis s, \dual{s}$.
\end{definition}
%
Thus, while \sessp is essentially the standard session $\pi$-calculus
as defined in the literature~\cite{honda.vasconcelos.kubo:language-primitives,GH05},
\HO can be related to a subcalculus of higher-order process calculi as studied
in the untyped~\cite{San923,SangiorgiD:picatomp,DBLP:journals/lmcs/JeffreyR05}
and typed settings~\cite{tlca07,mostrous09sessionbased}.
In \secref{sec:positive} we show that 
$\HOp$, $\HO$, and $\sessp$ have the same expressivity.

\subsection{Operational Semantics}

\label{subsec:reduction_semantics}

The operational semantics for \HOp is given as a standard reduction relation,
supported by a \emph{structural congruence} relation, denoted $\scong$. 
Structural congruence is 
the least congruence that satisfies the commutative monoid $(\mathcal{P}, \Par, \inact)$:
%
\[
\begin{array}{c}
	P \Par \inact \scong P
	\qquad
	P_1 \Par P_2 \scong P_2 \Par P_1
	\qquad
	P_1 \Par (P_2 \Par P_3) \scong (P_1 \Par P_2) \Par P_3
\end{array}
\]
%
\noi satisfies $\alpha$-conversion:
\[
	P_1 \scong_\alpha P_2\ \textrm{  implies  }\ P_1 \scong P_2
\]
\noi and furthermore, satisfies the rules:
%
\[
\begin{array}{c}
	n \notin \fn{P_1}\ \textrm{  implies  }\ P_1 \Par \news{n} P_2 \scong \news{n}(P_1 \Par P_2)
	\\[2mm]
	\news{n} \inact \scong \inact
	\qquad \quad
	\news{n} \news{m} P \scong \news{m} \news{n} P
	\qquad \quad
	\recp{X}{P} \scong P\subst{\recp{X}{P}}{\rvar{X}}
%	\qquad
%	\appl{(\abs{x} P)}{u} \scong P \subst{x}{u}
\end{array}
\]
\noi The first rule is describes scope opening for names.
Restricting of a name in an inactive process has no effect.
Furthermore, we can permute name restrictions.
Recursion is defined in structural congruence terms;
a recursive term $\recp{X}{P}$ is structurally
equivalent to its unfolding.

Structural congruence is extended to support values,
i.e~is the least congruence over processes and values
that satisfies $\cong$ for processes and, furthermore:
%
\[
	\abs{x}{P_1} \scong_\alpha \abs{y}{P_2}\ \textrm{ implies }\ \abs{x}{P_1} \scong \abs{y}{P_2}
	\qquad \quad
	P_1 \scong P_2 \textrm{ implies } \abs{x}{P_1} \scong \abs{x}{P_2}
\]
%
\noi Abstraction values are congruent up-to $\alpha$-conversion.
Furthermore, abstractions are congurent if their constructed on
congruent processes.

%Name application is also
%defined in structural congruence terms where structural
%congruence defines that application $\appl{(\abs{x} P)}{u}$ 
%substitutes variable $x$ with name $u$ over abstraction $\abs{x} P$.


\begin{figure}
\[
	\begin{array}{c}
		\begin{array}{c}
			\bout{s}{s'} P_1 \Par \binp{\dual{s}}{x} P_2 \red P_1 \Par P_2 \subst{s'}{x}\ \orule{ΝPass} \\
			\bout{s}{\abs{x}{P}} P_1 \Par \binp{\dual{s}}{\X} P_2 \red P_1 \Par P_2 \subst{\abs{x}{P}}{\X}\ \orule{APass}
			\\
			\tree{
				k \in I
			}{
				\bsel{s}{l_k} P \Par \bbra{\dual{s}}{l_i : P_i}_{i \in I} \red P \Par P_k 
			}\ \orule{Sel}
			\quad 
			\tree{
				P \red P'
			}{
				\news{s} P \red \news{s} P'
			}\ \orule{Sess}
			\\[4mm]
			\tree{
				P_1 \red P_1'
			}{
				P_1 \Par P_2 \red P_1' \Par P_2 
			}\ \orule{Par}
			\quad
			\tree{
				P \scong \red \scong P'
			}{
				P \red P'
			}\ \orule{Cong}
		\end{array}
		\\
		\begin{array}{c}
			\appl{X}{k} \subst{\abs{x}{Q}}{\X} = Q \subst{k}{x}
			\qquad
			\inact \subst{\abs{x}{Q}}{\X} = \inact
			\\
			\binp{s}{\Y} P \subst{\abs{x}{Q}}{\X} = \binp{s}{\Y} (P \subst{\abs{x}{Q}}{\X})
			\\
			(\bout{s}{\abs{y}{P_1}} P_2) \subst{\abs{x}{Q}}{\X} = \bout{s}{\abs{y}{P_1 \subst{\abs{x}{Q}}{\X}}} (P_2 \subst{\abs{x}{Q}}{\X})
			\\
			\bsel{s}{l} P \subst{\abs{x}{Q}}{\X} = \bsel{s}{l} (P \subst{\abs{x}{Q}}{\X})
			\\
			\bbra{s}{l_i : P_i}_{i \in I} \subst{\abs{x}{Q}}{\X} = \bbra{s}{l_i : P_i \subst{\abs{x}{Q}}{\X}}_{i \in I}
			\\
			(P_1 \Par P_2) \subst{\abs{x}{Q}}{\X} = P_1 \subst{\abs{x}{Q}}{\X} \Par P_2 \subst{\abs{x}{Q}}{\X}
			\qquad
			\news{s} P \subst{\abs{x}{Q}}{\X} = \news{s} (P \subst{\abs{x}{Q}}{\X})
		\end{array}
	\end{array}
\]
	\caption{The reduction semantics for $\HOp$ \label{fig:reduction}}
\end{figure}


\figref{fig:reduction} defines
the operational semantics for the \HOp.
$\orule{App}$ is a name application.
Rule~$\orule{Pass}$ defines value passing where
value $V$ is being send on channel $n$ to its dual endpoint $\dual{n}$
(for shared interactions $\dual{n} = n$).
As a result of the value passing reduction the continuation of the 
receiving process substitutes the receiving variable $x$ with $V$.
Rule~$\orule{Sel}$ is the standard rule for labelled choice/selection;
given an index set $I$,
a process selects label $l_j, j \in I$ on channel $n$ over a set of
labels $\set{l_i}_{i \in I}$ that are offered by a parallel process
on the dual session endpoint $\dual{n}$.
Remaining rules define congruence 
with respect to parallel composition (rule $\orule{Par}$)
and name restriction (rule $\orule{Ses}$).
Rule $\orule{Cong}$ defines closure under structural congruence.
We write $\red^\ast$ for a multi-step reduction. 

