% !TEX root = main.tex
\section{Positive Expressiveness Results}

\label{sec:positive}
In this section we present a study of the expressiveness
of $\HOp$ and its subcalculi. 
We present two encodability results:
%
\begin{enumerate}[1.]
	\item	The higher-order name passing communications with recursions (\HOp) into
		the higher-order communication without name-passing nor 
		recursions (\HO) (\secref{subsec:HOp_to_HO}).

	\item	\HOp into the first-order name-passing communication
		with recursions (\sessp) (\secref{subsec:HOp_to_p}). 
\end{enumerate}
%
In each case we show that the encoding is precise.

We often omit $H$ and $C$ from $\wb$ and $\fwb$ for simplicity of the notations. 


%In each case, encoding correctness is supported by
%type preservation and operational correspondence statements.
%Full abstraction results are conjectured, for the moment.

\begin{remark}[Polyadic \HOp]
	We can assume a semantic preserving encoding from the polyadic
	\HOp to the monadic \HOp. Polyadic \HOp assumes a polyadic
	extension of the \HOp semantics that defines values as
	$V \bnfis \tilde{u} \bnfbar \abs{\tilde{x}}{P}$
	and input prefix as $\binp{n}{\tilde{x}} P$.
	See \secref{subsec:pol_HOp} for the full definition of
	polyadic \HOp.
\end{remark}

\begin{comment}
\subsection*{Preliminaries}

The following result may be related to Lemma~\ref{l:invariant}: \jp{TO BE ADJUSTED with new typed LTS!}

\begin{lemma}[Inversion for (Untyped) Transitions]
\begin{enumerate}[1.]
% output
\item If $\Gamma;\, \emptyset;\, \Delta_1 \cat \Delta_2 \cat k:\btout{U}S \proves \bout{k}{V}P' \hastype \Proc$
and $\Gamma;\, \emptyset;\, \Delta_2 \cat k:S \proves V \hastype U$ then \\
$\bout{k}{V}P' \by{\bactout{k}{V}} P'$
and $\Gamma;\, \emptyset;\, \Delta_1 \proves P' \hastype \Proc$.

\item If $\Gamma;\, \emptyset;\, \Delta_1 \cat \Delta_2 \cat k:\btout{U}S \proves \bout{k}{V}P' \hastype \Proc$
and $\Gamma;\, \emptyset;\, \Delta_2 \proves V \hastype U$ (with $k:S \not\in \Delta_2$) then \\
$\bout{k}{V}P' \by{\bactout{k}{V}} P'$
and $\Gamma;\, \emptyset;\, \Delta_1 \cat k:S \proves P' \hastype \Proc$.

% input
\item If $\Gamma;\, \emptyset;\, \Delta \cat k:\btinp{\mytilde{C}}S \proves \binp{k}{\mytilde{x}}P' \hastype \Proc$
 then \\
$\binp{k}{\mytilde{x}}P' \by{\bactinp{k}{\mytilde{m}}} P'\subst{\mytilde{m}}{\mytilde{x}}$
and $\Gamma;\, \emptyset;\, \Delta \cat k:S \cat \mytilde{m}:\mytilde{C} \proves P'\subst{\mytilde{m}}{\mytilde{x}} \hastype \Proc$


\item If $\Gamma;\, \emptyset;\, \Delta_1 \setminus \Delta_2 \cat k:\btinp{L}S \proves \binp{k}{X}P' \hastype \Proc$
and
$\Gamma;\, \emptyset;\, \Delta_2   \proves X \hastype L$
 then \\
$\binp{k}{X}P' \by{\bactinp{k}{V}} P'\subst{V}{X}$
and $\Gamma;\, \emptyset;\, \Delta_1 \cat k:S \proves P'\subst{V}{X} \hastype \Proc$

% branch
\item If $\Gamma;\, \emptyset;\, \Delta \cat k:\btbra{l_i:S_i}_{i \in I} \proves \bbra{k}{l_i : P_i}_{i \in I}  \hastype \Proc$
 then, for any $j \in I$, we have \\
$\bbra{k}{l_i : P_i}_{i \in I} \by{\bactbra{k}{l_j}} P_j $
and $\Gamma;\, \emptyset;\, \Delta \cat k:S_j   \proves P_j \hastype \Proc$

% selecy
\item If $\Gamma;\, \emptyset;\, \Delta \cat k:\btsel{l_i:S_i}_{i \in I} \proves \bsel{k}{l_j} P_j  \hastype \Proc$
 then \\
$\bsel{k}{l_j} P_j \by{\bactsel{k}{l_j}} P_j $
and $\Gamma;\, \emptyset;\, \Delta \cat k:S_j   \proves P_j \hastype \Proc$

% bound output
\item Suppose (i)~$\Gamma;\, \emptyset;\, (\Delta_1 \cat \Delta_2) \setminus \{k:S_0, s:T\} \cat k:\btout{U}S_0 
\proves \news{s}\bout{k}{V}P' \hastype \Proc$, \\
(ii)~$\Gamma;\, \emptyset;\, \Delta_2 \cat \dual{s}:S \proves V \hastype U$ (with $S \dualof T$),
and (iii)~$\{k:S_0, s:T\} \in \Delta_1 \cat \Delta_2$. \\
Then 
$\news{s}\bout{k}{V}P' \by{\news{s}\bactout{k}{V}} P'$
and $\Gamma;\, \emptyset;\, \Delta^* \proves P' \hastype \Proc$, with \\
$(\Delta^* = \Delta_1 \cat k:S_0 \cat s:T) \lor (\Delta^* = \Delta_1 \cat k:S_0) \lor
(\Delta^* = \Delta_1 \cat  s:T) \lor (\Delta^* = \Delta_1)$.


% bound output
\item Suppose (i)~$\Gamma;\, \emptyset;\, \Delta \cat k:\btout{U}S_0 
\proves \news{s}\bout{k}{V}P' \hastype \Proc$
(with $\{k:S_0, s:T\} \not\subseteq \Delta$), 
(ii)~$\Gamma;\, \emptyset;\, \Delta_2 \cat \dual{s}:S \proves V \hastype U$ (with $S \dualof T$),
and (iii)~$\Delta_2 \subseteq \Delta$. 
Then $\news{s}\bout{k}{V}P' \by{\news{s}\bactout{k}{V}} P'$ and \\
$\Gamma;\, \emptyset;\, \Delta^* \proves P' \hastype \Proc$, with \\
$(\Delta^* = \Delta \setminus \Delta_2 \cat k:S_0 \cat s:T) \lor (\Delta^* = \Delta \setminus \Delta_2 \cat k:S_0) \lor
(\Delta^* = \Delta \setminus \Delta_2 \cat  s:T) \lor (\Delta^* = \Delta \setminus \Delta_2)$.


\end{enumerate}
\end{lemma}



\subsection{Encoding Polyadic Semantics (\HOp) to Monadic Semantics ($\HOp^{-\mathsf{p}}$)}\label{ss:polmon}


In the untyped $\pi$-calculus, polyadic communication
can be encoded into monadic name passing by first generating a fresh channel and then 
performing $n$ monadic synchronizations on that channel. 
In session-typed $\pi$-calculi this encoding is even more direct, 
thanks to the linearity and non-interference  of session endpoints~\cite{VascoFun}.
Below we  define an encoding of polyadic semantics to monadic semantics.
Using this encoding, %Because of the polyadic to monadic encoding %, denoted  $\auxmap{\cdot}{\mathsf{p}}$,
we may focus on monadic session processes,
and rely on polyadic constructs simply as convenient syntactic sugar.
In fact, we shall rely on polyadicity to encode recursive behaviors.
%
\begin{definition}[Polyadic Into Monadic]\myrm
	\label{d:enc:poltomon}
	Let 
	$\enco{\map{\cdot}^{\mathsf{p}}, \mapt{\cdot}^{\mathsf{p}}, \mapa{\cdot}^{\mathsf{p}}}: \HOp \to \HOp^{-\mathsf{p}}$
	be a typed encoding where
	%$\auxmap{\cdot}{\mathsf{p}}:\pHOp \to \HOp$ as
\begin{figure}[t]
\[
	\begin{array}{rcl}
		\map{\bout{k}{k_1, \cdots, k_m} P}^{\mathsf{p}}
		&\defeq&
		\bout{k}{k_1} \cdots ;  \bout{k}{k_m} \map{P}^{\mathsf{p}}
		\\
			\map{\binp{k}{x_1, \cdots, x_m} P}^{\mathsf{p}}
		&\defeq&
		\binp{k}{x_1} \cdots ; \binp{k}{x_m}  \map{P}^{\mathsf{p}}
		\\
		\map{\bbout{k}{\abs{x_1, \cdots, x_m} Q} P}^{\mathsf{p}}
		&\defeq&
		\bbout{k}{\abs{z}\binp{z}{x_1} \cdots ; \binp{z}{x_m} \map{Q}^{\mathsf{p}}} \map{P}^{\mathsf{p}}
		\\ 
		\map{\appl{X}{k_1, \cdots, k_m}}^{\mathsf{p}}
		&\defeq&
		\newsp{s}{\appl{X}{s} \Par \bout{\dual{s}}{k_1} \cdots ; \bout{\dual{s}}{k_m} \inact} 
        \\ % typed mapping starts here
		\tmap{\btout{S_1, \cdots, S_m}S}{\mathsf{p}}
		&\defeq&
		\btout{\tmap{S_1}{\mathsf{p}}} \cdots ; \btout{\tmap{S_m}{\mathsf{p}}}\tmap{S}{\mathsf{p}}
		\\
		\tmap{\btinp{S_1, \cdots, S_m}S}{\mathsf{p}}
		&\defeq&
		\btinp{\tmap{S_1}{\mathsf{p}}} \cdots ; \btinp{\tmap{S_m}{\mathsf{p}}}\tmap{S}{\mathsf{p}}
		\\
		\tmap{\bbtout{L} S}{\mathsf{p}}
		&\defeq&
		\bbtout{\mapt{L}^{\mathsf{p}}}\mapt{S}^{\mathsf{p}}
		\\
		\tmap{\bbtinp{L} S}{\mathsf{p}}
		&\defeq&
		\bbtinp{\mapt{L}^{\mathsf{p}}}\mapt{S}^{\mathsf{p}}
		\\
%		\tmap{\bbtout{\shot{(C_1, \cdots, C_m)}} S}{\mathsf{p}}
%		&\defeq&
%		\bbtout{
%		\shot{\big(\btinp{\tmap{C_1}{\mathsf{p}}} \cdots; \btinp{\tmap{C_m}{\mathsf{p}}}\tinact\big)}}\mapt{S}^{\mathsf{p}}
%		\\
%		\tmap{\bbtinp{\shot{(C_1, \cdots, C_m)}} S}{\mathsf{p}}
%		&\defeq&
%		\bbtinp{
%		\shot{\big(\btinp{\tmap{C_1}{\mathsf{p}}} \cdots; \btinp{\tmap{C_m}{\mathsf{p}}}\tinact\big)}}\mapt{S}^{\mathsf{p}}
%		\\
		\tmap{\shot{(C_1, \cdots, C_m)}}{\mathsf{p}}
		&\defeq&
		\shot{\big(\btinp{\tmap{C_1}{\mathsf{p}}} \cdots; \btinp{\tmap{C_m}{\mathsf{p}}}\tinact\big)}
		\\
		\tmap{\lhot{(C_1, \cdots, C_m)}}{\mathsf{p}}
		&\defeq&
		\lhot{\big(\btinp{\tmap{C_1}{\mathsf{p}}} \cdots; \btinp{\tmap{C_m}{\mathsf{p}}}\tinact\big)}
		\\
%		\tmap{\lhot{(C_1, \cdots, C_m)}}{\mathsf{p}}
%		&\defeq&
%		\lhot{\big(\btinp{\tmap{C_1}{\mathsf{p}}} \cdots \btinp{\tmap{C_m}{\mathsf{p}}}\tinact\big)}
%		\\
%		\tmap{\shot{(C_1, \cdots, C_m)}}{\mathsf{p}}
%		&\defeq&
%		\shot{\big(\btinp{\tmap{C_1}{\mathsf{p}}} \cdots \btinp{\tmap{C_m}{\mathsf{p}}}\tinact\big)}
%		\\ % action mapping starts here
		\mapa{\bactout{k}{k_1, \ldots, k_m}}^\mathsf{p} &\defeq&   \big\{\bactout{k}{k_1}, \cdots, \bactout{k}{k_m}\big\} \\
		\mapa{\bactinp{k}{k_1, \ldots, k_m}}^\mathsf{p} &\defeq&   \big\{\bactinp{k}{k_1}, \cdots, \bactinp{k}{k_m} \big\}\\
		\mapa{\bactout{k}{\abs{x_1, \ldots, x_m}{P}} }^\mathsf{p} &\defeq&  \bactout{k}{\abs{z}\binp{z}{x_1} \cdots ; \binp{z}{x_m} \map{P}^{\mathsf{p}}} \\
		\mapa{\bactinp{k}{\abs{x_1, \ldots, x_m}{P}} }^\mathsf{p} &\defeq&  \bactinp{k}{\abs{z}\binp{z}{x_1} \cdots ; \binp{z}{x_m} \map{P}^{\mathsf{p}}} 
	\end{array}
\]
\caption{\label{f:enc:poltomon}
Typed encoding of polyadic into monadic communication (cf.~Defintion \ref{d:enc:poltomon}). 
Mappings 
$\map{\cdot}^\mathsf{p}$,
$\mapt{\cdot}^\mathsf{p}$, 
and 
$\mapa{\cdot}^\mathsf{p}$
are homomorphisms for the other processes/types/labels. 
}
\end{figure}
mappings $\map{\cdot}^{\mathsf{p}}$, $\mapt{\cdot}^{\mathsf{p}}$, $\mapa{\cdot}^{\mathsf{p}}$
are 
as in Fig.~\ref{f:enc:poltomon}.
	%\jp{I prefer to be explicit in the encoding of polyadic abstraction/applications. Previous version is commented.}
\end{definition}
%
The encoding is simple:
passing an $m$-tuple of names over a session channel $k$ is represented by 
a $m$ exchanges along channel $k$.
The output of an abstraction with $m$ bound variables $x_1, \ldots, x_m$ is represented by
outputting an abstraction with a single bound variable $z$,
which is used as the subject for receiving $x_1, \ldots, x_m$ individually. 
Accordingly, 
%When we are dealing with an abstraction over a list of bound variables,
%then we create a new abstraction name and we use it to receive in a polyadic
%way the list of names on the abstraction. Similarly 
the encoding of a polyadic application  instantiates
the abstraction subject with a freshly generated session name $s$, which will be used  
to the names 
$k_1, \ldots, k_m$
that are going to be applied on the abstraction.
Observe how $\mapa{\cdot}^{\mathsf{p}}$ maps polyadic labels for input and output into (ordered) sets of
monadic labels. Also, note that we do not allow polyadic mapping on shared names.
The polyadic mapping, as presented here, is sound only on session names.
%The semantics might break if we apply this mapping on shared names.

%\begin{proposition}
%	$\Gamma; \emptyset; \Sigma \proves \map{P}^{p} \hastype \Proc$
%\end{proposition}

\begin{proposition}[Type Preservation, Polyadic to Monadic]\label{prop:typepresp}
Let $P$ be an  $\HOp$ process.
If			$\Gamma; \emptyset; \Delta \proves P \hastype \Proc$ then 
			$\mapt{\Gamma}^{\mathsf{p}}; \emptyset; \mapt{\Delta}^{\mathsf{p}} \proves \map{P}^{\mathsf{p}} \hastype \Proc$. 
\end{proposition}

\begin{proof}
By induction on the inference $\Gamma; \emptyset; \Delta \proves P \hastype \Proc$.
Details in \S\,\ref{app:polmon}.
	\qed
\end{proof}

\begin{proposition}\label{p:poltomo}
Let $P$ be a well-typed process.
\begin{enumerate}[$-$]
\item If $\map{P\subst{k_1,\cdots, k_m}{x_1, \cdots, x_m}}^{\mathsf{p}} = \map{P}^{\mathsf{p}}\subst{k_1}{x_1}\cdots\subst{k_m}{x_m}$.


\item If $\map{P\subst{\abs{x_1, \cdots, x_m} Q}{X}}^{\mathsf{p}} = \map{P}^{\mathsf{p}}\subst{\abs{z}\binp{z}{x_1} \cdots ; \binp{z}{x_m} \map{Q}^{\mathsf{p}}}{X}$.

\end{enumerate}
\end{proposition}
\begin{proof}
Immediate from the definition of $\map{\cdot}^{\mathsf{p}}$ (cf. Def.~\ref{d:enc:poltomon}).
	\qed
\end{proof}


\begin{proposition}[Operational Correspondence, Polyadic to Monadic]\label{p:ocpotomo}
Let $P$ be an  $\HOp$ process.
If $\Gamma; \emptyset; \Delta \proves P \hastype \Proc$ then
		\begin{enumerate}[1.]
			\item	 
			   If  $\stytra{\Gamma}{\ell_1}{\Delta}{P}{\Delta'}{P'}$
			   then either
			   \begin{enumerate}[a)]
					\item $\exists \ell'$ s.t. 
					$\mapa{\ell_1} = \ell'$ and 
			    $\wtytra{\mapt{\Gamma}^{\mathsf{p}}}{\ell'}{\mapt{\Delta}^{\mathsf{p}}}{\map{P}}{\mapt{\Delta'}^{\mathsf{p}}}{\map{P'}}$.
			    
			    	\item $\exists \ell'_1, \cdots, \ell'_n, R_1, \cdots, R_n$ s.t.
				    $\map{P} \Hby{\ell'_1} R_1, R_1 \Hby{\ell'_2} R_2, \ldots, R_{n-1} \Hby{\ell'_n} R_n = \map{P'}^{\mathsf{p}}$.
				    %and \\ $\wbb{\mapt{\Gamma}^{\mathsf{p}}}{\ell}{\mapt{\Delta'}^{\mathsf{p}}}{\map{P'}^{\mathsf{p}}}{\mapt{\Delta'}^{\mathsf{p}}}{R_n}$.
					\end{enumerate}
			   
			   
			    
			\item   
			If  $\stytra{\mapt{\Gamma}^{\mathsf{p}}}{\ell'_1}{\mapt{\Delta}^{\mathsf{p}}}{\map{P}}{\mapt{\Delta'}^{\mathsf{p}}}{R}$
			   then 
			   $\exists \ell_1, P'$ s.t.  
			   $\stytra{\Gamma}{\ell_1}{\Delta}{P}{\Delta'}{P'}$
			   and 
			   either
			   		\begin{enumerate}[a)]
					\item   
			      $\ell'_1 = \mapa{\ell_1}^{\mathsf{p}}$ ~~and~~ 
			    $\wbb{\mapt{\Gamma}^{\mathsf{p}}}{\ell}{\mapt{\Delta'}^{\mathsf{p}}}{\map{P'}^{\mathsf{p}}}{\mapt{\Delta'}^{\mathsf{p}}}{R}$.
			   		
					\item $\exists \ell'_2, \ldots, \ell'_n, R_2, \ldots, R_n$ s.t. 
					(i) $R \hby{\ell'_2} R_2   \cdots  R_{n-1} \hby{\ell'_n} R_n$,  
					(ii)~$\{\ell'_1, \ell'_2, \cdots, \ell'_n\} = \mapa{\ell_1}^{\mathsf{p}}$, \\ and 
					(iii)~$\wbb{\mapt{\Gamma}^{\mathsf{p}}}{\ell}{\mapt{\Delta'}^{\mathsf{p}}}{\map{P'}^{\mathsf{p}}}{\mapt{\Delta'}^{\mathsf{p}}}{R_n}$.
					\end{enumerate}


%			\item   
%			If  $\wtytra{\mapt{\Gamma}^{\mathsf{p}}}{\ell_2}{\mapt{\Delta_1}^{\mathsf{p}}}{\map{P}}{\mapt{\Delta'_1}^{\mathsf{p}}}{Q}$
%			   then $\exists \ell_1, P$ s.t.  
%			    (i)~$\stytra{\Gamma}{\ell_1}{\Delta_1}{P}{\Delta'_1}{P'}$, \\
%			    (ii)~$\ell_2 = \mapa{\ell_1}^{\mathsf{p}}$, 
%			    (iii)~$\wbb{\mapt{\Gamma}^{\mathsf{p}}}{\ell}{\mapt{\Delta'_1}^{\mathsf{p}}}{\map{P'}^{\mathsf{p}}}{\mapt{\Delta'_1}^{\mathsf{p}}}{Q}$.


			    \end{enumerate}
\end{proposition}

\begin{proof}
By transition induction.
%, considering Remark~\ref{r:multilabels} for weak transitions. All cases are easy;
%we only remark that the additional $\tau$-transitions induced by the encoding 
%are directly associated to the polyadicity involved. This is particularly relevant
%when $\ell_1 = \bactinp{n}{\abs{\tilde{x}}{P}}$, for the encoding of a polyadic application involves
%as many $\tau$-transitions (i.e., synchronizations on the restricted name $s$) 
%as polyadic parameters are involved.	\\
We consider parts (1) and (2) separately: \\
\noi \textbf{Part (1)}. We consider two non-trivial cases, using biadic communication:
\begin{enumerate}[1.]

%% Biadic Output 
\item Case  $P =\bout{k}{k_1, k_2} P'$ and $\ell_1 = \bactout{k}{k_1, k_2}$. 
We show that this case falls under part (b) of the thesis.
By assumption, $P$ is well-typed. 
As one particular possibility, we may have:
			\[
				\tree{
					\Gamma; \emptyset; \Delta_0 \cat k:S  \proves  P' \hastype \Proc \quad 
					\Gamma ; \emptyset ; k_1{:} S_1 \cat k_2{:}S_2 \proves  k_1,k_2 \hastype S_1,S_2}{
					\Gamma; \emptyset; \Delta_0 \cat k_1{:}S_1 \cat k_2{:}S_2 \cat k:\btout{S_1,S_2}S \proves  
					\bout{k}{k_1,k_2} P' \hastype \Proc}
			\]
for some $\Gamma, S, S_1, S_2, \Delta_0$, 
such that $\Delta = \Delta_0 \cat k_1{:}S_1 \cat k_2{:}S_2 \cat k:\btout{S_1,S_2}S$.
We may then have the following typed transition
$$
\stytra{\Gamma}{\ell_1}{\Delta_0 \cat k_1{:}S_1 \cat k_2{:}S_2 \cat k:\btout{S_1,S_2}S}{\bout{k}{k_1, k_2} P'}{\Delta_0 \cat k{:}S}{P'}
$$
The encoding of the source judgment for $P$ is as follows:
$$
\mapt{\Gamma}^{\mathsf{p}}; \emptyset; \mapt{\Delta_0 \cat k_1{:}S_1 \cat k_2{:}S_2 \cat k:\btout{S_1,S_2}S}^{\mathsf{p}} \proves \map{\bout{k}{k_1, k_2} P'}^{\mathsf{p}} \hastype \Proc
$$
which, using Def.~\ref{d:enc:poltomon}, can be expressed as 
$$
\mapt{\Gamma}^{\mathsf{p}}; \emptyset; \mapt{\Delta_0} 
\cat k_1{:}\mapt{S_1}^{\mathsf{p}} \cat k_2{:}\mapt{S_2}^{\mathsf{p}} 
\cat k:\btout{\mapt{S_1}^{\mathsf{p}}}\btout{\mapt{S_2}^{\mathsf{p}}}\mapt{S}^{\mathsf{p}}
\proves 
\bout{k}{k_1}\bout{k}{k_2} \map{P'}^{\mathsf{p}} 
\hastype \Proc
$$
Now, $\mapa{\ell_1}^{\mathsf{p}} = \{ \bactout{k}{k_1 }, \bactout{k}{ k_2}\}$. 
It is immediate to infer the following typed transitions for $\map{P}^{\mathsf{p}}  = \bout{k}{k_1}\bout{k}{k_2} \map{P'}^{\mathsf{p}} $:
\begin{eqnarray*}
& & \mapt{\Gamma}^{\mathsf{p}}; 
\mapt{\Delta_0} \cat  k_1{:}\mapt{S_1}^{\mathsf{p}} \cat k_2{:}\mapt{S_2}^{\mathsf{p}} \cat
k:\btout{\mapt{S_1}^{\mathsf{p}}}\btout{\mapt{S_2}^{\mathsf{p}}}\mapt{S}^{\mathsf{p}}
\proves 
\bout{k}{k_1}\bout{k}{k_2} \map{P'}^{\mathsf{p}}  \\
& \hby{\bactout{k}{k_1}} & 
\mapt{\Gamma}^{\mathsf{p}}; \mapt{\Delta_0} \cat  k_2{:}\mapt{S_2}^{\mathsf{p}} \cat
k:\btout{\mapt{S_2}^{\mathsf{p}}}\mapt{S}^{\mathsf{p}}
\proves 
\bout{k}{k_2} \map{P'}^{\mathsf{p}} \\
& \hby{\bactout{k}{k_2}} & 
\mapt{\Gamma}^{\mathsf{p}}; \mapt{\Delta_0}  \cat k{:}\mapt{S}^{\mathsf{p}}
\proves 
 \map{P'}^{\mathsf{p}} \\
 & = & 
 \mapt{\Gamma}^{\mathsf{p}}; \mapt{\Delta_0 \cat
k:S }^{\mathsf{p}}
\proves 
 \map{P'}^{\mathsf{p}}
\end{eqnarray*}
which concludes the proof for this case.

%% Biadic Abstraction Output 
\item Case  $P = \bbout{k}{\abs{x_1, x_2} Q} P' $ and $\ell_1 = \bactout{k}{\abs{x_1, x_2}{Q}}$. 
We show that this case falls under part (a) of the thesis.
By assumption, $P$ is well-typed. 
We may have:
			\[
				\tree{
					\Gamma; \emptyset; \Delta_0 \cat k:S  \proves  P' \hastype \Proc \quad 
					\Gamma ; \emptyset ; \Delta_1 \proves  \abs{x_1,x_2}Q \hastype \lhot{(C_1,C_2)}}{
					\Gamma; \emptyset; \Delta_0 \cat \Delta_1 \cat k:\btout{\lhot{(C_1,C_2)}}S \proves  
					\bout{k}{\abs{x_1,x_2}Q} P' \hastype \Proc}
			\]
for some $\Gamma, S, C_1, C_2, \Delta_0, \Delta_1$, 
such that $\Delta = \Delta_0 \cat \Delta_1 \cat  k:\btout{\lhot{(C_1,C_2)}}S$.
(For simplicity, we consider only the case of a linear function.)
We may have the following typed transition:
$$
\stytra{\Gamma}{\ell_1}{\Delta_0 \cat \Delta_1 \cat k:\bbtout{\lhot{(C_1, C_2)}}S}{\bbout{k}{\abs{x_1, x_2} Q} P' }{\Delta_0 \cat k{:}S}{P'}
$$
The encoding of the source judgment is
$$
\mapt{\Gamma}^{\mathsf{p}}; \emptyset; \mapt{\Delta_0 \cat \Delta_1 \cat k:\bbtout{\lhot{(C_1, C_2)}}S}^{\mathsf{p}} \proves \map{\bbout{k}{\abs{x_1, x_2} Q} P' }^{\mathsf{p}} \hastype \Proc
$$
which, using Def.~\ref{d:enc:poltomon}, can be equivalently expressed as 
$$
\mapt{\Gamma}^{\mathsf{p}}; \emptyset; \mapt{\Delta_0 \cat \Delta_1} \cat
%k:\btout{\mapt{S_1}^{\mathsf{p}}}\btout{\mapt{S_2}^{\mathsf{p}}}\mapt{S}^{\mathsf{p}}
k:\bbtout{
		\lhot{\big(\btinp{\tmap{C_1}{\mathsf{p}}}\btinp{\tmap{C_2}{\mathsf{p}}}\tinact\big)}}\mapt{S}^{\mathsf{p}}
\proves 
\bbout{k}{\abs{z}\binp{z}{x_1} \binp{z}{x_2} \map{Q}^{\mathsf{p}}} \map{P'}^{\mathsf{p}}
\hastype \Proc
$$

Now, $\mapa{\ell_1}^{\mathsf{p}} = \bactout{k}{\abs{z}\binp{z}{x_1}\binp{z}{x_2} \map{Q}^{\mathsf{p}}}$. 
It is immediate to infer the following typed transition for $\map{P}^{\mathsf{p}}  = \bbout{k}{\abs{z}\binp{z}{x_1} \binp{z}{x_2} \map{Q}^{\mathsf{p}}} \map{P'}^{\mathsf{p}}$:
\begin{eqnarray*}
& & \mapt{\Gamma}^{\mathsf{p}}; \mapt{\Delta_0 \cat \Delta_1} \cat
%k:\btout{\mapt{S_1}^{\mathsf{p}}}\btout{\mapt{S_2}^{\mathsf{p}}}\mapt{S}^{\mathsf{p}}
k:\bbtout{
		\lhot{\big(\btinp{\tmap{C_1}{\mathsf{p}}}\btinp{\tmap{C_2}{\mathsf{p}}}\tinact\big)}}\mapt{S}^{\mathsf{p}}
\proves 
\bbout{k}{\abs{z}\binp{z}{x_1} \binp{z}{x_2} \map{Q}^{\mathsf{p}}} \map{P'}^{\mathsf{p}} \\
& \hby{\mapa{\ell_1}^{\mathsf{p}}} & 
\mapt{\Gamma}^{\mathsf{p}}; \mapt{\Delta_0} \cat
k:\mapt{S}^{\mathsf{p}}, \,
\proves 
\map{P'}^{\mathsf{p}} \\
 & = & 
 \mapt{\Gamma}^{\mathsf{p}}; 
 \mapt{\Delta_0 \cat k:S}^{\mathsf{p}}
\proves 
 \map{P'}^{\mathsf{p}}
\end{eqnarray*}
which concludes the proof for this case.
\end{enumerate}

\noi \textbf{Part (2)}. We consider some non-trivial cases, using biadic communication:
\begin{enumerate}[1.]

%% Biadic Input 
\item Case $P =  \binp{k}{x_1, x_2} P' $, 
$\map{P}^{\mathsf{p}} = 
		\binp{k}{x_1}  \binp{k}{x_2}  \map{P'}^{\mathsf{p}}$.
		We show that this case falls under part~(b) of the thesis (cf. Prop.~\ref{p:ocpotomo}). 		
%		and $\ell_2 = \bactinp{k}{k_1}, \bactinp{k}{k_2}$. Then w
		We have  the following typed transitions for $\map{P}^{\mathsf{p}}$, for some $S$, $S_1$, $S_2$, and $\Delta$:
\begin{eqnarray*}
& & \mapt{\Gamma}^{\mathsf{p}}; 
\mapt{\Delta}^{\mathsf{p}} \cat 
k:\btinp{\tmap{S_1}{\mathsf{p}}}\btinp{\tmap{S_2}{\mathsf{p}}}\tmap{S}{\mathsf{p}} \cat
\proves 
\binp{k}{x_1} \binp{k}{x_2}\map{P'}^{\mathsf{p}} \\
& \hby{\bactinp{k}{k_1}} & 
\mapt{\Gamma}^{\mathsf{p}}; 
\mapt{\Delta}^{\mathsf{p}} \cat 
k:\btinp{\tmap{S_2}{\mathsf{p}}}\tmap{S}{\mathsf{p}} \cat
k_1:\mapt{S_1}^{\mathsf{p}}
\proves 
\binp{k}{x_2}\map{P'}^{\mathsf{p}} \subst{k_1}{x_1} \\
& \hby{\bactinp{k}{k_2}} & 
\mapt{\Gamma}^{\mathsf{p}}; 
\mapt{\Delta}^{\mathsf{p}} \cat k:\tmap{S}{\mathsf{p}} \cat
k_1:  \mapt{S_1}^{\mathsf{p}} \cat
k_2: \mapt{S_2}^{\mathsf{p}}
\proves 
\map{P'}^{\mathsf{p}} \subst{k_1}{x_1}\subst{k_2}{x_2} = Q
\end{eqnarray*}
Observe that the substitution lemma (Lemma~\ref{lem:subst}(1)) has been used twice.
%Considering Remark~\ref{r:multilabels} 
It is then immediate to infer the label for the source transition:
$\ell_1 = \bactinp{k}{k_1,k_2}$. Indeed, $\mapa{\ell_1}^{\mathsf{p}} = \{\bactinp{k}{k_1}, \bactinp{k}{k_2}\}$.
Now, in the source term $P$ we can infer the following transition:
$$
\stytra{\Gamma}{\ell_1}{\Delta \cat k:\btinp{S_1, S_2}S}{\binp{k}{x_1, x_2} P' }{\Delta\cat k{:}S \cat k_1:S_1 \cat k_2:S_2}{P'\subst{k_1,k_2}{x_1, x_2}}
$$
We now observe that, by
letting
 $\Delta^* = \mapt{\Delta}^{\mathsf{p}}\cat k:\tmap{S}{\mathsf{p}}, \,
k_1:  \mapt{S_1}^{\mathsf{p}} \cat
k_2: \mapt{S_2}^{\mathsf{p}}$, we have the desired conclusion:
$$\wbb{\mapt{\Gamma}^{\mathsf{p}}}{\ell}{\Delta^*}{\map{P'\subst{k_1,k_2}{x_1, x_2}}^{\mathsf{p}}}{\Delta^*}{Q}$$

%% Biadic Abstraction Output 
\item Case $P =  \bbout{k}{\abs{x_1,x_2} Q} P' $, 
$\map{P}^{\mathsf{p}} = 
		\bbout{k}{\abs{z}\binp{z}{x_1}\binp{z}{x_2} \map{Q}^{\mathsf{p}}} \map{P'}^{\mathsf{p}}$.
		We show that this case falls under part~(a) of the thesis (cf. Prop.~\ref{p:ocpotomo}). 
		We have the following  typed transition, for some $S$, $C_1$, $C_2$, and $\Delta$:
\begin{eqnarray*}
& & \mapt{\Gamma}^{\mathsf{p}}; 
\mapt{\Delta}^{\mathsf{p}}\cat k:\tmap{\bbtout{\lhot{(C_1,  C_2)}} S}{\mathsf{p}}
\proves 
\bbout{k}{\abs{z}\binp{z}{x_1}\binp{z}{x_2} \map{Q}^{\mathsf{p}}} \map{P'}^{\mathsf{p}} \\
& \hby{\ell'_1} & 
\mapt{\Gamma}^{\mathsf{p}}; 
\mapt{\Delta}^{\mathsf{p}}\cat k:\tmap{ S}{\mathsf{p}} 
\proves 
\map{P'}^{\mathsf{p}} = Q
\end{eqnarray*}
where
$\ell'_1 = \bactout{k}{\abs{z}\binp{z}{x_1} \binp{z}{x_2} \map{Q}^{\mathsf{p}}}$.
For simplicity, we consider only the case of linear functions.
It is then immediate to infer the label for the source transition:
$\ell_1 = \bactout{k}{\abs{x_1,  x_2}{Q}} $. 
Now, in the source term $P$ we can infer the following transition:
$$
\stytra{\Gamma}{\ell_1}{\Delta\cat k:\bbtout{\lhot{(C_1,  C_2)}} S}{ \bbout{k}{\abs{x_1,x_2} Q} P'}{\Delta\cat k{:}S}{P'}
$$
Then we have the desired conclusion:
$$\wbb{\mapt{\Gamma}^{\mathsf{p}}}{\ell}{\mapt{\Delta\cat k:S}^{\mathsf{p}}}{\map{P'}^{\mathsf{p}}}{\mapt{\Delta\cat k:S}^{\mathsf{p}}}{Q}$$


%% Biadic Abstraction Input 
\item Case $P =  \binp{k}{X} P' $, 
$\map{P}^{\mathsf{p}} = 
		\binp{k}{X} \map{P'}^{\mathsf{p}}$.
		We show that this case also falls under part~(a) of the thesis (cf. Prop.~\ref{p:ocpotomo}). 
We have  the following typed transition, for some $S$, $C_1$, $C_2$, and $\Delta$:
\begin{eqnarray*}
& & \mapt{\Gamma}^{\mathsf{p}}; 
\mapt{\Delta}^{\mathsf{p}}\cat k:\tmap{\bbtinp{\shot{(C_1,  C_2)}} S}{\mathsf{p}}
\proves 
\binp{k}{X} \map{P'}^{\mathsf{p}} \\
& \hby{\ell'_1} & 
\mapt{\Gamma}^{\mathsf{p}}; 
\mapt{\Delta}^{\mathsf{p}}\cat k:\tmap{ S}{\mathsf{p}} 
\proves 
\map{P'}^{\mathsf{p}}\subst{\abs{z}\binp{z}{x_1} \binp{z}{x_2} \map{Q}^{\mathsf{p}}}{X} = Q
\end{eqnarray*}
where 
 $\ell'_1 = \bactinp{k}{\abs{z}\binp{z}{x_1} \binp{z}{x_2} \map{Q}^{\mathsf{p}}} $. 
For simplicity, we consider only the case of shared functions.
It is then immediate to infer the label for the source transition:
$\ell_1 = \bactinp{k}{\abs{x_1,  x_2}{Q}} $. 
Now, in the source term $P$ we can infer the following transition:
$$
\stytra{\Gamma}{\ell_1}{\Delta\cat k:\bbtinp{\shot{(C_1, C_2)}} S}{ \binp{k}{X} P'}{\Delta\cat k{:}S}{P'\subst{\abs{x_1,  x_2}{Q}}{X}}
$$
Then we have the desired conclusion:
$$\wbb{\mapt{\Gamma}^{\mathsf{p}}}{\ell}{\mapt{\Delta\cat k:S}^{\mathsf{p}}}{\map{P'\subst{\abs{x_1,  x_2}{Q}}{X}}^{\mathsf{p}}}{\mapt{\Delta\cat k:S}^{\mathsf{p}}}{Q}$$
We omit the (easy) conductive argument supporting the last claim,
which uses Prop.~\ref{p:poltomo}.
We content ourselves with noticing that the key difference between 
${\map{P'\subst{\abs{x_1,  x_2}{Q}}{X}}^{\mathsf{p}}}$
and $Q$ consists in the $\tau$-transitions present in $Q$  and absent in ${\map{P'\subst{\abs{x_1,  x_2}{Q}}{X}}^{\mathsf{p}}}$, which are induced   by the monadic representation of polyadic communication.
\end{enumerate}
\qed
\end{proof}

\begin{conjecture}[Full Abstraction: Polyadic / Monadic Communication]
\begin{enumerate}[a)]
\item
If
$\wbb{ \Gamma}{\ell}{\Delta_1}{ P }{ \Delta_2}{Q}$
then
$\wbb{\mapt{\Gamma}^{\mathsf{p}}}{\ell}{\mapt{\Delta_1}^{\mathsf{p}}}{\map{P}^{\mathsf{p}}}{\mapt{\Delta_2}^{\mathsf{p}}}
{\map{Q}^{\mathsf{p}}}$.
\item  
If 
$\wbb{\mapt{\Gamma}^{\mathsf{p}}}{\ell}{\mapt{\Delta_1}^{\mathsf{p}}}{\map{P}^{\mathsf{p}}}{\mapt{\Delta_2}^{\mathsf{p}}}
{\map{Q}^{\mathsf{p}}}$
then 
$\wbb{ \Gamma}{\ell}{\Delta_1}{ P }{ \Delta_2}{Q}$.
\end{enumerate}
While (a) is  completeness (non trivial), (b) is soundness  (easy in principle).
\end{conjecture}

In the light of the tight operational correspondence for the polyadic/monadic encoding,
in the following we restrict to consider monadic communications.

\end{comment}

%%%%%%%%%%%%%%%%%%%%%%%%%%%%%%%%%%%%%%%%%%%%
% HOp ---> HO
%%%%%%%%%%%%%%%%%%%%%%%%%%%%%%%%%%%%%%%%%%%%

\subsection{Encoding \HOp into \HO}
\label{subsec:HOp_to_HO}

We show that the subcalculus $\HO$ is expressive enough to
represent the the full \HOp calculus.

The main challenge is to encode (1) name passing 
and (2) recursions.
Name passing involves {\em packing} a name 
value as an abstraction send it and it and then
substitute on the receiving using a name appication.
The encoding on the recursion semantics are more complex;
A process is encoded as an abstraction with no free names
(i.e~a shared abstraction). We then use higher-order
passing to pass the process and duplicate the process.
One copy of the process is used to reconstitute the
original process and the other is used to enable another
duplicator procedure.
%we only use name abstraction passing.
%For (1), we pass  
%an % simple 
%abstraction which enables to use the name upon application. 
%For (2), we 
%copy a process upon reception; the case of linear abstraction passing
%presents a limitation 
%is \NY{delicate} 
%because 
%linear abstractions cannot be copied.
We handle the transformation of a process into a linear abstraction
with the definition of an
%a preliminary tool which is a mapping from
auxiliary mapping
from processes with free names to processes without free names
(but with free variables) (\defref{def:auxmap}). 
We first require an auxiliary definition:
%
\begin{definition}\myrm 
	Let $\vmap{\cdot}: 2^{\mathcal{N}} \longrightarrow \mathcal{V}^\omega$
	be a map of sequences lexicographically ordered 
	names to sequences of variables, defined
	inductively as:
%
	\[
		\vmap{\epsilon} = \epsilon \qquad \qquad \qquad \vmap{n \cat \tilde{m}} = x_n \cat \vmap{\tilde{m}}
	\]
\end{definition}

Given a process $P$, we write $\ofn{P}$ to denote the
\emph{sequence} of free names of $P$, lexicographically ordered.

The following auxiliary mapping transforms processes
with free names into abstractions and it is
used in \defref{def:enc:HOp_to_HO}.
%
\begin{definition}\myrm
	\label{def:auxmap}
	Let $\sigma$ be a set of session names.
	Define $\auxmap{\cdot}{\sigma}: \HOp \to \HOp$  as in \figref{fig:auxmap}.
\end{definition}

\begin{figure}[t]
\[
	\begin{array}{rcl}
		\auxmap{\news{n} P}{\sigma} &\bnfis& \news{n} \auxmap{P}{{\sigma \cat n}}
		\vspace{1mm} \\

		\auxmap{\bout{n}{\abs{x}{Q}} P}{\sigma} &\bnfis&
		\left\{
		\begin{array}{rl}
			\bout{x_n}{\abs{x}{\auxmap{Q}{\sigma}}} \auxmap{P}{\sigma} & n \notin \sigma\\
			\bout{n}{\abs{x}{\auxmap{Q}{\sigma}}} \auxmap{P}{\sigma} & n \in \sigma
		\end{array}
		\right.
		\vspace{1mm}	\\ 

		\auxmap{\binp{n}{X} P}{\sigma} &\bnfis&
		\left\{
		\begin{array}{rl}
			\binp{x_n}{X} \auxmap{P}{\sigma} & n \notin \sigma\\
			\binp{n}{X} \auxmap{P}{\sigma} & n \in \sigma
		\end{array}
		\right.

		\vspace{1mm}	\\ 
		\auxmap{\bsel{n}{l} P}{\sigma} &\bnfis&
		\left\{
		\begin{array}{rl}
			\bsel{x_n}{l} \auxmap{P}{\sigma} & n \notin \sigma\\
			\bsel{n}{l} \auxmap{P}{\sigma} & n \in \sigma
		\end{array}
		\right.
		\vspace{1mm} \\
		\auxmap{\bbra{n}{l_i:P_i}_{i \in I}}{\sigma} &\bnfis&

		\left\{
		\begin{array}{rl}
			\bbra{x_n}{l_i:\auxmap{P_i}{\sigma}}_{i \in I}  & n \notin \sigma\\
			\bbra{n}{l_i:\auxmap{P_i}{\sigma}}_{i \in I}  & n \in \sigma
		\end{array}
		\right.
		\vspace{1mm} \\
		\auxmap{\appl{x}{n}}{\sigma} &\bnfis&
		\left\{
		\begin{array}{rl}
			\appl{x}{x_n} & n \notin \sigma\\
			\appl{x}{n} & n \in \sigma\\
		\end{array}
		\right.
		\vspace{1mm} \\

		\auxmap{\inact}{\sigma} &\bnfis& \inact
		\vspace{1mm} \\

		\auxmap{P \Par Q}{\sigma} &\bnfis& \auxmap{P}{\sigma} \Par \auxmap{Q}{\sigma}
	\end{array}
\]
\caption{\label{fig:auxmap} The auxiliary map (cf. Def.~\ref{def:auxmap}) 
used in the encoding of the recursive primitives of \HOp into \HO (Def.~\ref{def:enc:HOp_to_HO}).}
\end{figure}



Given a process $P$ with $\fn{P} = m_1, \cdots, m_n$,
we are interested in its associated (polyadic) abstraction,
which is defined as $\abs{x_1, \cdots, x_n}{\auxmap{P}{\es} }$,
where $\vmap{m_j} = x_j$, for all $j \in \{1, \ldots, n\}$.
This transformation from processes into abstractions can be reverted by
using abstraction and application with an appropriate sequence of session names:
%
\begin{proposition}\myrm
	Let $P$ be a \HOp process with $\tilde{n} = \ofn{P}$.
	Also, suppose $\tilde{x} = \vmap{\tilde{n}}$.
	Then $P \scong \appl{x}{\tilde{n}}\subst{\abs{\tilde{x}}\auxmap{P}{\emptyset}}{x}$.
\end{proposition}

\begin{proof}
	\noi The proof is an easy induction on the map $\auxmap{P}{\es}$.
	We show a case since other cases are similar.

	\noi - Case: $\auxmap{\bout{n}{m} P}{\es} = \bout{x_n}{x_m} \auxmap{P}{\es}$

	\noi We rewrite substitution as:
	$\appl{x}{\tilde{n}} \subst{\abs{\tilde{x}}{\bout{x_n}{y_m} \auxmap{P}{\es}}}{x} \scong (\bout{x_n}{y_m} P) \subst{\tilde{x}}{\tilde{n}}$

	\noi If consider that $x_n, y_m \in \vmap{\tilde{n}}$ then from the definition of $\vmap{\cdot}$ we
	get that $n, m \in \tilde{n}$. Furthermore by the fact that $\tilde{n}$ and $\vmap{\tilde{n}}$ are
	ordered, substitution becomes:
	$\bout{n}{m} \auxmap{P}{\es} \subst{\tilde{x}}{\tilde{n}}$.

	\noi The rest of the cases are similar.
	\qed
\end{proof}


We are now ready to define the encoding of \HOp into strict process-passing.
Note that we assume polyadicity in abstraction and application.
Given a session environment $\Delta = \{n_1:S_1, \ldots, n_m:S_m\}$, 
in the following definition we write
$\tilde{S}_{\Delta}$ to stand for $S_1, \ldots, S_m$.
%
\begin{definition}[Encoding \HOp into \HO]\myrm
	\label{def:enc:HOp_to_HO}
	Let $f$ be a function from recursion variables to sequences of name variables.
	Define the typed encoding $\enco{\pmapp{\cdot}{1}{f}, \tmap{\cdot}{1}, \mapa{\cdot}^{1}}: \tyl{L}_{\HOp} \to \tyl{L}_{\HO}$,
	where mappings $\map{\cdot}^{1}$, $\mapt{\cdot}^{1}$, $\mapa{\cdot}^{1}$
	are as in \figref{fig:enc:HOp_to_HO}.
	We assume that the mapping $\tmap{\cdot}{1}$ on types is extended to 
	session environments $\Delta$
	and
	shared environments $\Gamma$ 
	as follows:
%
	\[
	\begin{array}{rcll}
	    \mapt{\Delta \cat s: S}^{1} & =  & \mapt{\Delta}^{1} \cat s:\mapt{S}^{1} & \\
		\tmap{\Gamma \cat u: \chtype{S}}{1} & =  & \tmap{\Gamma}{1} \cat u:\chtype{\tmap{S}{1}} & \\
		\tmap{\Gamma \cat u: \chtype{L}}{1} & = &  \tmap{\Gamma}{1} \cat u:\chtype{\tmap{L}{1}} & \\
		\tmap{\Gamma \cat \varp{X}:\Delta}{1} & = & \tmap{\Gamma}{1} \cat x:\shot{(\tilde{S}_{\Delta}\,,\,S^*)} & 
		\quad\text{(where $ 
%		S^* = \trec{t}{\big((\tilde{S}_{\Delta}\,,\, \btinp{\vart{t}}\tinact)\big)}
		S^* = \trec{t}{\btinp{\shot{(\tilde{S}_{\Delta}\,,\,\vart{t})}} \tinact}$)}
	\end{array}
	\]
%\end{remark}
\end{definition}

\begin{figure}[h!]
\[
	\begin{array}{rclcrcl}
		\multicolumn{7}{l}{\textrm{\bf Terms}}
		\\
		\pmapp{\bout{u}{v} P}{1}{f}	&\defeq&	\bout{u}{ \abs{z}{\,\binp{z}{x} \appl{x}{v}} } \pmapp{P}{1}{f}
		& &
		\pmapp{\binp{u}{k} Q}{1}{f}	&\defeq&	\binp{u}{x} \newsp{s}{\appl{x}{s} \Par \bout{\dual{s}}{\abs{x}{\pmapp{Q}{1}{f}}} \inact}
		\\
		\pmapp{\bout{u}{\abs{x}{Q}} P}{1}{f} &\defeq& \bout{u}{\abs{x}{\pmapp{Q}{1}{f}}} \pmapp{P}{1}{f}
		& &
		\pmapp{\binp{u}{\underline{x}} P}{1}{f}	&\defeq&	\binp{u}{\underline{x}} \pmapp{P}{1}{f}
		\\
		\pmapp{\recp{X}{P}}{1}{f} &\defeq&
		\multicolumn{5}{l}{
			\newsp{s}{\binp{s}{x} \pmapp{P}{1}{{f,\{\varp{X}\to \tilde{n}\}}} \Par
			\bout{\dual{s}}{\abs{\vmap{\tilde{n}}, z } \,{\binp{z}{x} \auxmap{\pmapp{P}{1}{{f,\{\varp{X}\to \tilde{n}\}}}}{\es}}} \inact}
			\quad \tilde{n} = \ofn{P}
		}
		\\
		\pmapp{\varp{X}}{1}{f} &\defeq&
		\multicolumn{5}{l}{
			\newsp{s}{\appl{x}{\tilde{n}, s} \Par \bbout{\dual{s}}{ \abs{\vmap{\tilde{n}},z}\,\,{\appl{x}{ \vmap{\tilde{n}}, z}}} \inact}
			\qquad \qquad \qquad \qquad \qquad \quad \tilde{n} = f(\varp{X})
		}
		\\
		\pmapp{\bsel{s}{l} P}{1}{f}	&\defeq&	\bsel{s}{l} \pmapp{P}{1}{f}
		& & 
		\pmapp{\bbra{s}{l_i: P_i}_{i \in I}}{1}{f} &\defeq& \bbra{s}{l_i: \pmapp{P_i}{1}{f}}_{i \in I}
		\\
		\pmapp{\appl{x}{u}}{1}{f}	&\defeq&	\appl{x}{u}
		& &
		\pmapp{(\appl{\abs{x}{P})}{u}}{1}{f}	&\defeq&	\appl{(\abs{x}{\pmapp{P}{1}{f}})}{u}
		\\
		\pmapp{P \Par Q}{1}{f}		&\defeq&	\pmapp{P}{1}{f} \Par \pmapp{Q}{1}{f}
		& &
		\pmapp{\news{n} P}{1}{f}	&\defeq&	\news{n} \pmapp{P}{1}{f}
		\\
		\pmapp{\inact}{1}{f}		&\defeq&	\inact
		\\
		\multicolumn{7}{l}{\textrm{\bf Types}}
		\\
		\tmap{C}{1}_\mathsf{v}		&\defeq&
		\multicolumn{5}{l}{
			\left\{
			\begin{array}{rcl}
				\lhot{(\btinp{\lhot{\tmap{C}{1}}} \tinact)} && \textrm{if } C = S\\
				\lhot{(\btinp{\shot{\tmap{C}{1}}} \tinact)} && \textrm{otherwise}\\
			\end{array}
			\right.
		}
		\\
		\tmap{\lhot{C}}{1}_\mathsf{v}	&\defeq& \lhot{\tmap{C}{1}}
		& & 
		\tmap{\shot{C}}{1}_\mathsf{v}	&\defeq& \shot{\tmap{C}{1}}
		\\
		\tmap{\chtype{S}}{1}		&\defeq& \chtype{\tmap{S}{1}}
		& &
		\tmap{\chtype{L}}{1}		&\defeq& \chtype{\tmap{L}{1}_\mathsf{v}}
		\\
		\tmap{\btout{U} S}{1}		&\defeq& \btout{\tmap{U}{\mathsf{v}}} \tmap{S}{1}
		& &
		\tmap{\btinp{U} S}{1}		&\defeq& \btinp{\tmap{U}{\mathsf{v}}} \tmap{S}{1}
		\\
		\tmap{\btsel{l_i: S_i}_{i \in I}}{1} &\defeq& \btsel{l_i: \tmap{S_i}{1}}_{i \in I}
		& &
		\tmap{\btbra{l_i: S_i}_{i \in I}}{1} &\defeq& \btbra{l_i: \tmap{S_i}{1}}_{i \in I}
		\\
		\tmap{\vart{t}}{1} &\defeq& \vart{t}
		& &
		\tmap{\trec{t}{S}}{1} &\defeq& \trec{t}{\tmap{S}{1}}
		\\
		\tmap{\tinact}{1} &\defeq& \tinact
		\\
		\multicolumn{7}{l}{\textrm{\bf Labels}}
		\\
		\mapa{\news{\tilde{m_1}}\bactout{n}{m}}^{1}	&\defeq&	\news{\tilde{m_1}}\bactout{n}{\abs{z}{\,\binp{z}{x} \appl{x}{m}} }
		& &
		\mapa{\bactinp{n}{m}}^{1}			&\defeq&	\bactinp{n}{\abs{z}{\,\binp{z}{x} \appl{x}{m}} }
		\\
		\mapa{\news{\tilde{m}}\bactout{n}{\abs{x}{P}}}^{1} &\defeq& \news{\tilde{m}}\bactout{n}{\abs{x}{\pmapp{P}{1}{\es}}}
		& &
		\mapa{\bactinp{n}{\abs{x}{P}}}^{1} &\defeq& \bactinp{n}{\abs{x}{\pmapp{P}{1}{\es}}}
		\\
		\mapa{\bactsel{n}{l} }^{1} &\defeq& \bactsel{n}{l} 
		& &
		\mapa{\bactbra{n}{l} }^{1} &\defeq& \bactbra{n}{l} 
		\\
		\mapa{\tau}^{1} &\defeq& \tau

	\end{array}
\]
	\caption{
		\label{fig:enc:HOp_to_HO}
		Typed encoding of \HOp into \HO (cf.~Defintion~\ref{def:enc:HOp_to_HO}).
%		Mappings 
%		$\map{\cdot}^2$,
%		$\mapt{\cdot}^2$, 
%		and 
%		$\mapa{\cdot}^2$
%		are homomorphisms for the other processes/types/labels. 
	}
\end{figure}



\noi Note that $\Delta$ in $\varp{X}:\Delta$ is mapped to a non-tail
recursive session type.
Non-tail
recursive session types have been studied in
\cite{DBLP:journals/corr/abs-1202-2086,TGC14};
to our knowledge,
this is the first application in the
context of higher-order session types.
%which carries type variable as the last argument.  
For a simplicity of the presentation, we use the polyadic name abstraction and passing.
Polyadic semantics will be formally encoded into \HO in \secref{subsec:pol_HOp}.

\noi We explain the mapping in \figref{def:enc:HOp_to_HO}, focusing 
on {\em name passing} ($\pmapp{\bout{u}{w} P}{1}{f}$ and $\pmapp{\binp{u}{x} P}{1}{f}$), and  
{\em recursion} ($\pmapp{\recp{X}{P}}{1}{f}$ and $\pmapp{\varp{X}}{1}{f}$). 

\myparagraph{Name passing}
A name $w$ is being passed as an input guarded abstraction;
the abstraction receives a higher-order
value and continues with the application of $w$ over
the received higher-order value.
%A name $m$ is being passed as an input
%guarded abstraction. 
%The input prefix receives an abstraction and
%continues with the application of $n$ over the received abstraction.
On the receiver side $\binp{u}{x} P$ 
the encoding realises a mechanism that i) receives
the input guarded abstraction, then ii) applies it on a fresh session endpoint $s$, 
and iii) uses
the dual endpoint $\dual{s}$ to send the continuation $P$ as the abstraction
$\abs{x}{P}$. 
\NY{Then} name substitution is achieved via name application.

\myparagraph{Recursion}
The encoding of a recursive process $\recp{X}{P}$  is delicate, for it 
must preserve the linearity of session endpoints. To this end, we:
\NY{i) record a mapping from recursive variable $X$ to process variables $z_X$;
ii)~encode the recursion body $P$ as a name abstraction
in which free names of $P$ are converted into name variables;
iii)~this higher-order value is embedded in an input-guarded 
``duplicator'' process; and 
iv)~make the encoding of process variable $x$ to 
simulate recursion unfolding by 
invoking the duplicator in a by-need fashion,
i.e.,~upon reception, abstraction $\auxmap{P}{\sigma}$ is duplicated
with one copy used to reconstitute the encoded recursion body $P$ through
the application of $\fn{P}$ and another copy used to re-invoke
the duplicator when needed. % to simulate recursion unfolding.
}
%The idea follows 
%a classical recursion encoding \cite{ThomsenB:plachoasgcfhop}.  
%A mapping of process $P$ is parallel composed, 
%and also being passed as an input
%guarded abstraction, parameterised also by a sequence of trigger names $\tilde{n}$. 
%We record a mapping from $z_X$ (which is a fresh variable of $X$) 
%to $\tilde{n}$, so that 
%when the abstraction is substituted to $z_\rvar{X}$ 
%(which occurs in the mapping of $P\subst{z_X}{X}$), 
%the correct $\tilde{n}$ is applied. In this way, we can 
%send and receive an abstraction which holds $P$, repeatedly. 


%In the higher-order setting, a name $v$ is being passed as an input
%guarded abstraction. The input prefix receives an abstraction and
%continues with the application of $v$ over the received abstraction.
%On the receiver side $\binp{u}{x} P$ 
%the encoding realizes a mechanism that (i) receives
%the input guarded abstraction, then (ii) applies it on a fresh session endpoint $s$, 
%and (iii) uses
%the dual endpoint $\dual{s}$ to send the continuation $P$ as the abstraction
%$\abs{x}{P}$. 
%As a result, name substitution is achieved via name application.


\begin{proposition}[Type Preservation, \HOp into \HO]\myrm
	\label{prop:typepres_HOp_to_HO}
	Let $P$ be a \HOp process.
	If $\Gamma; \emptyset; \Delta \proves P \hastype \Proc$ then 
	$\tmap{\Gamma}{1}; \emptyset; \tmap{\Delta}{1} \proves \pmapp{P}{1}{f} \hastype \Proc$. 
\end{proposition}

\begin{proof}
	By induction on the inference $\Gamma; \emptyset; \Delta \proves P \hastype \Proc$.
	Details in \propref{app:prop:typepres_HOp_to_HO} (Page~\pageref{app:prop:typepres_HOp_to_HO}).
	\qed
\end{proof}

The following proposition formalizes our strategy  for encoding
recursive definitions as passing of polyadic abstractions:
%
\begin{proposition}[Operational Correspondence for Recursive Processes]\myrm
	\label{prop:op_corr_HOprec_to_HO}
	Let $P$ and $P_1$ be \HOp processes s.t. 
	$P =\recp{X}{P'}$ and
	$P_1 = P'\subst{\recp{X}{P'}}{\varp{X}} \scong P$.

	\noi If %$P_1 \hby{\ell} P_2$ 
	$\stytra{\Gamma}{\ell}{\Delta}{P}{\Delta'}{P'}$
	then,  there exist
	processes $R_1$, $R_2$,  $R_3$, action $\ell'$,
	and mappings $f, f_1$, such that: 
	\begin{enumerate}[(i)]
		\item 
		%$\pmapp{P}{1}{f} \hby{\tau} \map{P'}^{1}_{f_{1}} \subst{R_3}{X} = R_1$;
		$\stytra{\mapt{\Gamma}^{1}}{\tau}{\mapt{\Delta}^{1}}{P}{\mapt{\Delta}^{1}}{\map{P'}^{1}\subst{R_3}{X}} = R_1$;
		\item 
		%$R_1 \Hby{\ell'} R_2$, with $\ell' = \mapa{\ell}^{1}$;
		$\wtytra{\mapt{\Gamma}^{1}}{\ell'}{\mapt{\Delta}^{1}}{R_1}{\mapt{\Delta}^{1}}{R_2} $,  with $\ell' = \mapa{\ell}^{1}$;
	
		\item $R_3 = \abs{\tilde{m}}\binp{z}{x}\auxmap{\map{P'}^{1}_{f_{1}}}{\sigma}$, with $\tilde{m} = \ofn{P'},z$)
		and
		$f_1 = f, \set{\varp{X} \to \ofn{P'}}$.
	\end{enumerate}
\end{proposition}

\begin{proof}[Sketch]
	Part~(1) follow directly from the definition of typed encoding for processes $\pmapp{\cdot}{1}{f}$ (\defref{def:enc:HOp_to_HO}),
	observing that the reduction occurs along a restricted name, and so the session environment remains unchanged.
	Part~(2) relies on  \propref{prop:op_corr_HOp_to_HO}.
	Part~(3) is immediate from \defref{def:enc:HOp_to_HO}.
	\qed
\end{proof}

The following proposition formalises completeness and
soundness results for the encoding of \HOp into \HO.
Recall that deterministic transitions $\stau$ and 
$\btau$ have been defined in \defref{def:dettrans}.
%We write $\by{\tau}_k$ to denote a sequence of $k$ $\tau$-transitions.

\begin{proposition}[Operational Correspondence, \HOp into \HO]\myrm
	\label{prop:op_corr_HOp_to_HO}
	Let $P$ be a \HOp process.
	If $\Gamma; \emptyset; \Delta \proves P \hastype \Proc$ then:
%
	\begin{enumerate}[1.]
		\item
			Suppose $\horel{\Gamma}{\Delta}{P}{\hby{\ell_1}}{\Delta'}{P'}$. Then we have:
%
			\begin{enumerate}[a)]
				\item
					If $\ell_1 \in \set{\news{\tilde{m}}\bactout{n}{m}, \,\news{\tilde{m}}\bactout{n}{\abs{x}Q}, \,\bactsel{s}{l}, \,\bactbra{s}{l}}$
					then $\exists \ell_2$ s.t. \\
					$\horel{\tmap{\Gamma}{1}}{\tmap{\Delta}{1}}{\pmapp{P}{1}{f}}{\hby{\ell_2}}{\tmap{\Delta'}{1}}{\pmapp{P'}{1}{f}}$
					and $\ell_2 = \mapa{\ell_1}^{1}$.
			
				\item
					If $\ell_1 = \bactinp{n}{\abs{y}Q}$ and
					$P' = P_0 \subst{\abs{y}Q}{x}$
					then $\exists \ell_2$ s.t. \\
					$\horel{\tmap{\Gamma}{1}}{\tmap{\Delta}{1}}{\pmapp{P}{1}{f}}{\hby{\ell_2}}{\tmap{\Delta'}{1}}{\pmapp{P_0}{1}{f}\subst{\abs{y}\pmapp{Q}{1}{\emptyset}}{x}}$
					and $\ell_2 = \mapa{\ell_1}^{1}$.
			
				\item
					If $\ell_1 = \bactinp{n}{m}$
					and 
					$P' = P_0 \subst{m}{x}$
					then $\exists \ell_2$, $R$ s.t. \\
					$\horel{\tmap{\Gamma}{1}}{\tmap{\Delta}{1}}{\pmapp{P}{1}{f}}{\hby{\ell_2}}{\tmap{\Delta'}{1}}{R}$,
					with $\ell_2 = \mapa{\ell_1}^{1}$, \\
					and
					$\horel{\tmap{\Gamma}{1}}{\tmap{\Delta'}{1}}{R}{\hby{\btau} \hby{\stau} \hby{\btau}}
					{\tmap{\Delta'}{1}}{\pmapp{P_0}{1}{f}\subst{m}{x}}$.
						
				\item
					If $\ell_1 = \tau$
					and $P' \scong \newsp{\tilde{m}}{P_1 \Par P_2\subst{m}{x}}$
					then $\exists R$ s.t. \\
					$\horel{\tmap{\Gamma}{1}}{\tmap{\Delta}{1}}{\pmapp{P}{1}{f}}{\hby{\tau}}{\mapt{\Delta}^{1}}{\newsp{\tilde{m}}{\pmapp{P_1}{1}{f} \Par R}}$,
					and\\ 
					$\horel{\tmap{\Gamma}{1}}{\tmap{\Delta}{1}}{\newsp{\tilde{m}}{\pmapp{P_1}{1}{f} \Par R}}{\hby{\btau} \hby{\stau} \hby{\btau}}
					{\mapt{\Delta}^{1}}{\newsp{\tilde{m}}{\pmapp{P_1}{1}{f} \Par \pmapp{P_2}{1}{f}\subst{m}{x}}}$.
			
				\item
					If $\ell_1 = \tau$
					and $P' \scong \newsp{\tilde{m}}{P_1 \Par P_2 \subst{\abs{y}Q}{x}}$
					then \\
					$\horel{\tmap{\Gamma}{1}}{\tmap{\Delta}{1}}{\pmapp{P}{1}{f}}{\hby{\tau}}
					{\tmap{\Delta_1}{1}}{\newsp{\tilde{m}}{\pmapp{P_1}{1}{f}\Par \pmapp{P_2}{1}{f}\subst{\abs{y}\pmapp{Q}{1}{\emptyset}}{x}}}$.
			
				\item
					If $\ell_1 = \tau$
					and $P' \not\scong \newsp{\tilde{m}}{P_1 \Par P_2 \subst{m}{x}} \land P' \not\scong \newsp{\tilde{m}}{P_1 \Par P_2\subst{\abs{y}Q}{x}}$
					then \\
					$\horel{\tmap{\Gamma}{1}}{\tmap{\Delta}{1}}{\pmapp{P}{1}{f}}{\hby{\tau}}{\tmap{\Delta'_1}{1}}{ \pmapp{P'}{1}{f}}$.
			\end{enumerate}
			
		\item	Suppose $\horel{\tmap{\Gamma}{1}}{\tmap{\Delta}{1}}{\pmapp{P}{1}{f}}{\hby{\ell_2}}{\tmap{\Delta'}{1}}{Q}$.
			Then we have:
%
			\begin{enumerate}[a)]
				\item 
					If $\ell_2 \in
					\set{\news{\tilde{m}}\bactout{n}{\abs{z}{\,\binp{z}{x} (\appl{x}{m})}}, \,\news{\tilde{m}} \bactout{n}{\abs{x}{R}}, \,\bactsel{s}{l}, \,\bactbra{s}{l}}$
					then $\exists \ell_1, P'$ s.t. \\
					$\horel{\Gamma}{\Delta}{P}{\hby{\ell_1}}{\Delta'}{P'}$, 
					$\ell_1 = \mapa{\ell_2}^{1}$, 
					and
					$Q = \pmapp{P'}{1}{f}$.
			
				\item 
					If $\ell_2 = \bactinp{n}{\abs{y} R}$ %(with $R \neq \binp{y}{x} \appl{x}{m}$)
					then either:
%
					\begin{enumerate}[(i)]
						\item	$\exists \ell_1, x, P', P''$ s.t. \\
							$\horel{\Gamma}{\Delta}{P}{\hby{\ell_1}}{\Delta'}{P' \subst{\abs{y}P''}{x}}$, 
							$\ell_1 = \mapa{\ell_2}^{1}$, $\pmapp{P''}{1}{\es} = R$, and $Q = \pmapp{P'}{1}{f}$.

						\item	$R \scong \binp{y}{x} (\appl{x}{m})$ and 
							$\exists \ell_1, z, P'$ s.t. \\
							$\horel{\Gamma}{\Delta}{P}{\hby{\ell_1}}{\Delta'}{P' \subst{m}{z}}$, 
							$\ell_1 = \mapa{\ell_2}^{1}$,
							and\\
							$\horel{\tmap{\Gamma}{1}}{\tmap{\Delta'}{1}}{Q}{\hby{\btau} \hby{\stau} \hby{\btau}}{\tmap{\Delta''}{1}}{\pmapp{P'\subst{m}{z}}{1}{f}}$
					\end{enumerate}
			
				\item 
					If $\ell_2 = \tau$ 
					then $\Delta' = \Delta$ and 
					either
%
					\begin{enumerate}[(i)]
						\item	$\exists P'$ s.t. 
							$\horel{\Gamma}{\Delta}{P}{\hby{\tau}}{\Delta}{P'}$,
							and $Q = \map{P'}^{1}_f$.	

						\item
							$\exists P_1, P_2, x, m, Q'$ s.t. 
							$\horel{\Gamma}{\Delta}{P}{\hby{\tau}}{\Delta}{\newsp{\tilde{m}}{P_1 \Par P_2\subst{m}{x}} }$, and\\
							$\horel{\tmap{\Gamma}{1}}{\tmap{\Delta}{1}}{Q}{\hby{\btau} \hby{\stau} \hby{\btau}}{\tmap{\Delta}{1}}{\pmapp{P_1}{1}{f} \Par \pmapp{P_2\subst{m}{x}}{1}{f}}$ 
%							$Q = \map{P_1}^{1}_f \Par Q'$, where $Q'  \Hby{} $.

%						\item $\exists P_1, P_2, x, R$ s.t. 
%						$\stytra{ \Gamma }{\tau}{ \Delta }{ P}{ \Delta}{ \news{\tilde{m}}(P_1 \Par P_2\subst{\abs{y}R}{x}) }$, and 
%						$Q = \map{\news{\tilde{m}}(P_1 \Par P_2\subst{\abs{y}R}{x})}^{1}_f$.
			\end{enumerate}
		    \end{enumerate}
		    
%		\item   
%			If  $\wtytra{\mapt{\Gamma}^{1}}{\ell_2}{\mapt{\Delta}^{1}}{\pmapp{P}{1}{f}}{\mapt{\Delta'}^{1}}{Q}$
%			then $\exists \ell_1, P'$ s.t.  \\
%			(i)~$\stytra{\Gamma}{\ell_1}{\Delta}{P}{\Delta'}{P'}$,
%			(ii)~$\ell_2 = \mapa{\ell_1}^{1}$, 
%			(iii)~$\wbb{\mapt{\Gamma}^{1}}{\ell}{\mapt{\Delta'}^{1}}{\pmapp{P'}{1}{f}}{\mapt{\Delta'}^{1}}{Q}$.
	\end{enumerate}
\end{proposition}


\begin{comment}
\begin{proposition}[Operational Correspondence, \HOp into \HO]\myrm
	\label{prop:op_corr_HOp_to_HO}
	Let $P$ be a \HOp process.
	If $\Gamma; \emptyset; \Delta \proves P \hastype \Proc$ then
	\begin{enumerate}[1.]
		\item
			Suppose $\stytra{\Gamma}{\ell_1}{\Delta}{P}{\Delta'}{P'}$. Then we have:
			\begin{enumerate}[a)]
		    \item 
			If $\ell_1 \in \{\news{m}\bactout{n}{m}, \,\news{m}\bactout{n}{\abs{x}Q}, \,\bactsel{s}{l}, \,\bactbra{s}{l}   \}$
			then $\exists \ell_2$ s.t. \\
			$\stytra{\mapt{\Gamma}^{1}}{\ell_2}{\mapt{\Delta}^{1}}{\pmapp{P}{1}{f}}{\mapt{\Delta'}^{1}}{\pmapp{P'}{1}{f}}$
			and $\ell_2 = \mapa{\ell_1}^{1}$.
			
			\item If $\ell_1 = \bactinp{n}{\abs{y}Q}$ and
			$P' = P_0\subst{\abs{y}Q}{x}$ and			
			then $\exists \ell_2$ s.t. \\
			$\stytra{\mapt{\Gamma}^{1}}{\ell_2}{\mapt{\Delta}^{1}}{\pmapp{P}{1}{f}}{\mapt{\Delta'}^{1}}{\pmapp{P_0}{1}{f}\subst{\abs{y}\pmapp{Q}{1}{\emptyset}}{x}}$
			and $\ell_2 = \mapa{\ell_1}^{1}$.
			
			\item If $\ell_1 = \bactinp{n}{m}$
			and 
			$P' = P_0\subst{m}{x}$
			then $\exists \ell_2$, $R$ s.t. \\
			$\stytra{\mapt{\Gamma}^{1}}{\ell_2}{\mapt{\Delta}^{1}}{\pmapp{P}{1}{f}}{\mapt{\Delta'}^{1}}{R}$,
			with $\ell_2 = \mapa{\ell_1}^{1}$, \\
			and
			%$\wtytra{\mapt{\Gamma}^{1}}{}{\mapt{\Delta'}^{1}}{R}{\mapt{\Delta'}^{1}}{\pmapp{P_0}{1}{f}\subst{m}{x}}$. \\	
			$\mapt{\Gamma}^{1}; \mapt{\Delta'}^{1} \proves R \hby{\tau}_3
				\mapt{\Gamma}^{1}; \mapt{\Delta'}^{1} \proves \pmapp{P_0}{1}{f}\subst{m}{x}$.
						
			\item If $\ell_1 = \tau$
			and $P' \scong \news{\tilde{m}}(P_1 \Par P_2\subst{m}{x})$
			then $\exists R$ s.t. \\
			$\stytra{\mapt{\Gamma}^{1}}{\tau}{\mapt{\Delta}^{1}}{\pmapp{P}{1}{f}}{\mapt{\Delta}^{1}}{
			\news{\tilde{m}}(\pmapp{P_1}{1}{f}
			\Par R)
			}$, where $R \by{\tau}_3  \pmapp{P_2}{1}{f}\subst{m}{x}$.

			
			
			
			\item If $\ell_1 = \tau$
			and $P' \scong \news{\tilde{m}}(P_1 \Par P_2\subst{\abs{y}Q}{x})$
			then \\
			$\stytra{\mapt{\Gamma}^{1}}{\tau}{\mapt{\Delta}^{1}}{\pmapp{P}{1}{f}}{\mapt{\Delta_1}^{1}}{
			\news{\tilde{m}}(\pmapp{P_1}{1}{f}
			\Par \pmapp{P_2}{1}{f}\subst{\abs{y}\pmapp{Q}{1}{\emptyset}}{x})
			}$.
			
			\item If $\ell_1 = \tau$
			and $P' \not\scong \news{m}(P_1 \Par P_2\subst{m}{x}) \land P' \not\scong \news{\tilde{m}}(P_1 \Par P_2\subst{\abs{y}Q}{x})$
			then \\
			$\stytra{\mapt{\Gamma}^{1}}{\tau}{\mapt{\Delta}^{1}}{\pmapp{P}{1}{f}}{\mapt{\Delta'_1}^{1}}{ \pmapp{P'}{1}{f}
			}$.
			\end{enumerate}
			
		\item Suppose $\stytra{\mapt{\Gamma}^{1}}{\ell_2}{\mapt{\Delta}^{1}}{\pmapp{P}{1}{f}}{\mapt{\Delta'}^{1}}{Q}$.
		Then we have:
			\begin{enumerate}[a)]
		    \item 
		    If $\ell_2 \in \{\news{m}\bactout{n}{\abs{z}{\,\binp{z}{x} \appl{x}{m}}}, \,\news{\tilde{m}}\bactout{n}{\abs{x}R}, \,\bactsel{s}{l}, \,\bactbra{s}{l}   \}$
			then $\exists \ell_1, P'$ s.t. \\
			$\stytra{ \Gamma }{\ell_1}{ \Delta }{ P}{ \Delta' }{ P'}$, 
			$\ell_1 = \mapa{\ell_2}^{1}$, 
			and
			$\map{P'}^{1}_f = Q$.
			
			\item 
		    If $\ell_2 = \bactinp{n}{\abs{y}R}$ (with $R \neq \binp{y}{x} \appl{x}{m}$)
			then $\exists \ell_1, x, P', P''$ s.t. \\
			$\stytra{ \Gamma }{\ell_1}{ \Delta }{ P}{ \Delta' }{ P'\subst{\abs{y}P''}{x}}$, 
			$\ell_1 = \mapa{\ell_2}^{1}$, 
						$\map{P''}^{1}_\es = R$, and 			$\map{P'}^{1}_f = Q$.		
						
			\item 
		    If $\ell_2 = \bactinp{n}{\abs{y}\binp{y}{x} \appl{x}{m}}$ 
			then $\exists \ell_1, z, P'$ s.t. \\
			$\stytra{ \Gamma }{\ell_1}{ \Delta }{ P}{ \Delta' }{ P'\subst{m}{z}}$, 
			$\ell_1 = \mapa{\ell_2}^{1}$,
			and 
			$Q \by{\tau}_3  \map{P'\subst{m}{z}}^{1}_f$.	
		
			%$\horel{\mapt{\Gamma}^{1}}{\mapt{\Delta'}^{1}}{\map{P'}^{1}}{\wbf}{\mapt{\Delta'}^{1}}{Q}$.
			
			\item 
		    If $\ell_2 = \tau$ 
			then $\Delta' = \Delta$ and 
			either
			\begin{enumerate}[(i)]
			\item $\exists P'$ s.t. 
			$\stytra{ \Gamma }{\tau}{ \Delta }{ P}{ \Delta}{ P'}$, and $Q = \map{P'}^{1}_f$.	

			\item $\exists P_1, P_2, x, m, Q'$ s.t. 
			$\stytra{ \Gamma }{\tau}{ \Delta }{ P}{ \Delta}{\news{\tilde{m}}( P_1 \Par P_2\subst{m}{x}) }$, and 
			$Q = \map{P_1}^{1}_f \Par Q'$, where $Q'  \Hby{} \map{P_2\subst{m}{x}}^{1}_f$.

			\item $\exists P_1, P_2, x, R$ s.t. 
			$\stytra{ \Gamma }{\tau}{ \Delta }{ P}{ \Delta}{ \news{\tilde{m}}(P_1 \Par P_2\subst{\abs{y}R}{x}) }$, and 
			$Q = \map{\news{\tilde{m}}(P_1 \Par P_2\subst{\abs{y}R}{x})}^{1}_f$.

			\end{enumerate}
		    \end{enumerate}
		    
%		\item   
%			If  $\wtytra{\mapt{\Gamma}^{1}}{\ell_2}{\mapt{\Delta}^{1}}{\pmapp{P}{1}{f}}{\mapt{\Delta'}^{1}}{Q}$
%			then $\exists \ell_1, P'$ s.t.  \\
%			(i)~$\stytra{\Gamma}{\ell_1}{\Delta}{P}{\Delta'}{P'}$,
%			(ii)~$\ell_2 = \mapa{\ell_1}^{1}$, 
%			(iii)~$\wbb{\mapt{\Gamma}^{1}}{\ell}{\mapt{\Delta'}^{1}}{\pmapp{P'}{1}{f}}{\mapt{\Delta'}^{1}}{Q}$.
	\end{enumerate}
\end{proposition}
\end{comment}

\begin{proof}
	The proof is a mechanical induction on the labelled Transition System.
	Parts (1) and (2) are proved separetely.
	The most demanding cases for the proof can be found in~\propref{app:prop:op_corr_HOp_to_HO}
	(page~\pageref{app:prop:op_corr_HOp_to_HO}).
	\qed
\end{proof}

\begin{proposition}[Full Abstraction, \HOp into \HO]\myrm
	\label{prop:fulla_HOp_to_HO}
	Let $P_1, Q_1$ be \HOp processes.
	$\horel{\Gamma}{\Delta_1}{P_1}{\hwb}{\Delta_2}{Q_1}$
	if and only if
	$\horel{\tmap{\Gamma}{1}}{\tmap{\Delta_1}{1}}{\pmapp{P_1}{1}{f}}{\hwb}{\tmap{\Delta_2}{1}}{\pmapp{Q_1}{1}{f}}$.
\end{proposition}

\begin{proof}
	The proof for the soundness direction considers
	closure that can be shown to be a bisimulation
	following the soundness direction of Operational Correspondence
	(\propref{prop:op_corr_HOp_to_HO}). Whenever needed
	the proof makes use of the $\tau$-inertness result
	(\propref{lem:tau_inert}).

	The proof for the completness direction also considers
	a closure shown to be a bisimulation
	up-to deterministic transition (\propref{lem:up_to_deterministic_transition})
	following the completeness direction of Operational Correspondence
	(\propref{prop:op_corr_HOp_to_HO}).

	Details of the proof can be found in~\propref{app:prop:fulla_HOp_to_HO}
	(page~\pageref{app:prop:fulla_HOp_to_HO}).
	\qed
\end{proof}

\begin{proposition}[Precise encoding of \HOp into \HO]\myrm
	\label{prop:prec:HOp_to_HO}
	The encoding from $\tyl{L}_{\HOp}$ to $\tyl{L}_{\HO}$
	is precise.
\end{proposition}

\begin{proof}
	Syntactic requirements are easily derivable from the
	definition of the mappings in \figref{fig:enc:HOp_to_HO}.
	Semantic requirements are a consequence of
	\propref{prop:typepres_HOp_to_HO}, \propref{prop:op_corr_HOp_to_HO}, and \propref{prop:fulla_HOp_to_HO}.
	\qed
\end{proof}

%\begin{proposition}[Order Preservation]\myrm
%	Let type $U$ of order $k$ and \HOp value $\abs{x}{P}$ such that
%	$\Gamma; \es; \Delta \proves \abs{x}{P} \hastype U$;
%	then $\tmap{\Gamma}{1}; \es; \tmap{\Delta}{1} \proves \abs{x}{\pmapp{P}{f}{1}} \hastype \tmap{U}{1}$
%	with $\tmap{U}{1}$ being of order $k$.
%\end{proposition}
%
%\begin{proof}
%	The proof is a standard induction on the type mapping $\tmap{U}{1}$.
%	\qed
%\end{proof}

%\begin{conjecture}[Full Abstraction]
%\begin{enumerate}[a)]
%\item
%If
%$\wbb{ \Gamma}{\ell}{\Delta_1}{ P }{ \Delta_2}{Q}$
%then
%$\wbb{\mapt{\Gamma}^{1}}{\ell}{\mapt{\Delta_1}^{1}}{\map{P}^{1}}{\mapt{\Delta_2}^{1}}
%{\map{Q}^{1}}$.
%\item  
%If 
%$\wbb{\mapt{\Gamma}^{1}}{\ell}{\mapt{\Delta_1}^{1}}{\map{P}^{1}}{\mapt{\Delta_2}^{1}}
%{\map{Q}^{1}}$
%then 
%$\wbb{ \Gamma}{\ell}{\Delta_1}{ P }{ \Delta_2}{Q}$.
%\end{enumerate}
%While (a) is  completeness, (b) is soundness.
%
%\end{conjecture}

\begin{comment}
\begin{proof}[Sketch]
	We must show completeness and soundness properties. 
	For completeness, it suffices to consider source process
	$P_0 = \bout{k}{k'} P \Par \binp{k}{x} Q$. We have that
%
	\[
		P_0 \red P \Par Q\subst{k'}{x}.
	\]
%
	By the definition of encoding we have:
	\begin{eqnarray*}
		\pmap{P_0}{1} & = & \bbout{k}{ \abs{z}{\,\binp{z}{X} \appl{X}{k'}} } \pmap{P}{1} \Par \binp{k}{X} \newsp{s}{\appl{X}{s} \Par \bbout{\dual{s}}{\abs{x} \pmap{Q}{1}} \inact}  \\
		& \red & \pmap{P}{1} \Par \newsp{s}{\appl{X}{s} \subst{\abs{z}{\,\binp{z}{X} \appl{X}{k'}}}{X} \Par \bbout{\dual{s}}{\abs{x} \pmap{Q}{1}} \inact} \\
		& = & \pmap{P}{1} \Par \newsp{s}{\,\binp{s}{X} \appl{X}{k'} \Par \bbout{\dual{s}}{\abs{x} \pmap{Q}{1}} \inact} \\
		& \red & \pmap{P}{1} \Par \appl{X}{k'} \subst{\abs{x} \pmap{Q}{1}}{X} \Par \inact \\
		& \scong & \pmap{P}{1} \Par \pmap{Q}{1}\subst{k'}{x}  
	\end{eqnarray*}
	For soundness, it suffices to notice that the encoding does not add new visible actions:
	the additional synchronizations induced by the encoding always occur on private (fresh) names.
	We assume weak bisimilarities, which abstract from internal actions used by the encoding,
	and so  constructing a relation witnessing behavioral equivalence is easy.
	\qed
\end{proof}
\end{comment}

%\subsection{Polyadic Into Monadic}
%The encoding from $\psesp$ to $\sesp$ is easier than the
%encoding of polyadic $\pi$-calculus in the $\pi$-calculus because
%we have linear session endpoints.
%
%\begin{definition}[$\psesp$ to $\sesp$]
%	We write $\encod{\cdot}{\cdot}{2}:\psesp \to \sesp$ whenever
%
%	\begin{tabular}{c}
%			$\map{\bout{k}{k'_1, \cdots, k'_n} P}^{2} \defeq \bout{k}{k'_1} \cdots ;  \bout{k}{k'_n}
%			\pmap{P}{2}$\\
%			$\map{\binp{k}{x_1, \cdots, x_n} P}^{2} \defeq \binp{k}{x_1} \cdots ; \binp{k}{x_n}  \pmap{P}{2}$ \\
%			$\tmap{\btout{S_1, \cdots, S_n} S}{2} \defeq \bbtout{\tmap{S_1}{2}} \cdots; \bbtout{\tmap{S_n}{2}} \tmap{S}{2}$\\
%			$\tmap{\btinp{S_1, \cdots, S_n} S}{2} \defeq \bbtinp{\tmap{S_1}{2}} \cdots; \bbtinp{\tmap{S_n}{2}} \tmap{S}{2}$
%%		\end{tabular}
%%		& \quad &
%%		\begin{tabular}{l}
%%			$\tmap{\btout{S_1 \cat \tilde{S}} S}{2} \defeq \btout{S_1} \tmap{\btout{\tilde{S}} S}{2}$\\
%%			$\tmap{\btinp{S_1 \cat \tilde{S}} S}{2} \defeq \btinp{S_1} \tmap{\btinp{\tilde{S}} S}{2}$
%%		\end{tabular}
%	\end{tabular}
%\end{definition}
%
%Polyadic name sending (resp.\ receive) is encoded as sequence of
%send (resp.\ receive) operations. Linearity of session endpoints
%ensures no race conditions, thus the encoding is sound.
%
%The encoding of the polyadic $\sesp$ semantics is as simple as the
%composition of the two former encodings.
%
%\begin{definition}[Encoding from $\psespnr$ to $\HO$]
%	We define $\encod{\cdot}{\cdot}{3}: \psespnr \longrightarrow \HO$
%	as $\encod{\cdot}{\cdot}{3} = \encod{\cdot}{\cdot}{1} \cat \encod{\cdot}{\cdot}{2}$.	
%\end{definition}

%So far we have consider name abstractions and applications which are \emph{monadic}.
%We now consider the \emph{polyadic} extension of these constructs, %name abstractions and applications.
%written $\abs{x_1, \ldots, x_n} P$ and $\appl{X}{k_1, \ldots, k_n}$, respectively.
%Next we give the encoding from $\HOp$ with polyadic name abstraction to $\HOp^{p}$.
%
%\begin{definition}[Encoding from $\pHOpnr$ to $\pHOp$]
%
%	\begin{tabular}{lcl}
%		$\map{\bout{k}{\abs{\tilde{x}} P_1} P_2}^4$ &$\defeq$& $\bout{k}{\abs{z} \binp{z}{\tilde{x}} \map{P_1}^4} \map{P_2}^4$\\
%		$\map{\appl{X}{\tilde{k}}}$ &$\defeq$& $\newsp{s}{\appl{X}{s} \Par \bout{\dual{s}}{\tilde{k}} \inact}$
%	\end{tabular}
%\end{definition}

%We compose the latter encoding with the generalisation $\map{\cdot}^3 : \HOp^{p-\mu} \longrightarrow \HO$
%of the encoding $\map{\cdot}^3 : \sesp^{p-\mu} \longrightarrow \HO$ to get a translation
%of $\HOp^{pa-\mu}$ to $\HO$.
%
%\begin{definition}[Encoding from $\HOp^{pa-\mu}$ to $\HO$]
%	We define $\encod{\cdot}{\cdot}{5}: \HOp^{pa-\mu} \longrightarrow \HO$
%	as $\encod{\cdot}{\cdot}{5} = \encod{\cdot}{\cdot}{4} \cat \encod{\cdot}{\cdot}{3}$.	
%\end{definition}

% !TEX root = main.tex
\begin{example}[The Encoding 
$\pmapp{\cdot}{1}{f}$ At Work]
Let $P = \recp{X}{\bout{a}{m} \varp{X}}$ be an \HOp process.
Its associated encoding into \HO is as follows---we note that initially $f = \emptyset$.
\begin{eqnarray*}
	\pmapp{P}{1}{f} &=&
	\newsp{s_1}{ \binp{s_1}{x} \pmapp{\bout{a}{m} \varp{X}}{1}{{f'}} \Par \bout{\dual{s_1}}{ \abs{(x_a, x_m, z)} \binp{z}{x} \auxmapp{\pmapp{\bout{a}{m} \varp{X}}{1}{{f'}}}{{}}{\es} } \inact} \\
%	&&\bout{\dual{s_1}}{ \abs{(x_a, x_m, z)} \binp{z}{x} \auxmapp{\pmapp{\bout{a}{m} \varp{X}}{1}{{\varp{X} \rightarrow x_ax_m}}}{{}}{\es} } \inact}
%\end{eqnarray*}
%\begin{eqnarray*}	
\pmapp{\bout{a}{m} \varp{X}}{1}{{ f'}} &=&
	\bout{a}{\abs{z}{\binp{z}{x} (\appl{x}{m})}} \pmapp{\varp{X}}{1}{{f'}}
	\\
	&=& \bout{a}{\abs{z}{\binp{z}{x} (\appl{x}{m})}} \newsp{s_2}{\appl{x}{(a,m, s_2)}  \Par \bout{\dual{s_2}}{\abs{(x_a, x_m, z)}{\appl{x}{(x_a, x_m, z)}}} \inact} \\
	\auxmapp{\pmapp{\bout{a}{m} \varp{X}}{1}{{f'}}}{{}}{\es}
	  & = & \auxmapp{\bout{a}{\abs{z}{\binp{z}{x} (\appl{x}{m})}} \newsp{s_2}{\appl{x}{(a,m, s_2)}  \Par \bout{\dual{s_2}}{\abs{(x_a, x_m, z)}{\appl{x}{(x_a, x_m, z)}}} \inact}}{{}}{\es}
	\\
	 & = & \bout{x_a}{\abs{z}{\binp{z}{x} (\appl{x}{x_m})}} \auxmapp{\newsp{s_2}{\appl{x}{(a,m, s_2)}  \Par \bout{\dual{s_2}}{\abs{(x_a, x_m, z)}{\appl{x}{(x_a, x_m, z)}}} \inact}}{{}}{\es}
	\\
	& = & \bout{x_a}{\abs{z}{\binp{z}{x} (\appl{x}{x_m})}} \newsp{s_2}{\appl{x}{(x_a,x_m, s_2)}  \Par \bout{\dual{s_2}}{\abs{(x_a, x_m, z)}{\appl{x}{(x_a, x_m, z)}}} \inact}
\end{eqnarray*}
where $f' = \varp{X} \rightarrow x_ax_m$.
That is, by writing $V$ to denote the process
$$
\abs{(x_a, x_m, z)} \binp{z}{x} \bout{x_a}{\abs{z}{\binp{z}{x} (\appl{x}{x_m})}} \newsp{s_2}{\appl{x}{(x_a,x_m, s_2)}  \Par \bout{\dual{s_2}}{\abs{(x_a, x_m, z)}{\appl{x}{(x_a, x_m, z)}}} \inact}
$$
we would have that $P = \recp{X}{\bout{a}{m} \varp{X}}$ is mapped into the \HO process
$$
\newsp{s_1}{\binp{s_1}{x}  \bout{a}{\abs{z}{\binp{z}{x} (\appl{x}{m})}} \newsp{s_2}{\appl{x}{(a,m, s_2)}  \Par \bout{\dual{s_2}}{\abs{(x_a, x_m, z)}{\appl{x}{(x_a, x_m, z)}}} \inact}\Par \bout{\dual{s_1}}{V} \inact}
$$
We illustrate the behavior of the encoded process (below, we let $\lambda = \bactout{a}{\abs{z}{\binp{z}{x} (\appl{x}{m})}}$):
\begin{eqnarray*}
\pmapp{P}{1}{f} & \scong & \newsp{s_1}{\bout{\dual{s_1}}{V} \inact \Par \binp{s_1}{x} \bout{a}{\abs{z}{\binp{z}{x} (\appl{x}{m})}} \newsp{s_2}{\bout{\dual{s_2}}{\abs{(x_a, x_m, z)}{\appl{x}{(x_a, x_m, z)}}} \inact}  \\
& & \qquad \Par \appl{x}{(a,m, s_2)}} \\
& \by{\tau} & \bout{a}{\abs{z}{\binp{z}{x} (\appl{x}{m})}} \newsp{s_2}{\bout{\dual{s_2}}{V} \inact \Par \binp{s_2}{x} \bout{a}{\abs{z}{\binp{z}{x} (\appl{x}{m})}} \\
& & \qquad \qquad \quad \qquad \qquad \qquad \newsp{s_3}{\bout{\dual{s_3}}{\abs{(x_a, x_m, z)}{\appl{x}{(x_a, x_m, z)}}} \inact} \Par \appl{x}{(a,m, s_3)}} \\
& \scong_{\alpha} & \bout{a}{\abs{z}{\binp{z}{x} (\appl{x}{m})}} \newsp{s_1}{\bout{\dual{s_1}}{V} \inact \Par \binp{s_1}{x} \bout{a}{\abs{z}{\binp{z}{x} (\appl{x}{m})}} \\
& & \qquad \qquad \qquad \qquad \quad \qquad \newsp{s_2}{\bout{\dual{s_2}}{\abs{(x_a, x_m, z)}{\appl{x}{(x_a, x_m, z)}}} \inact} \Par \appl{x}{(a,m, s_2)}} \\
& \scong & 
		\bout{a}{\abs{z}{\binp{z}{x} (\appl{x}{m})}} \pmapp{\recp{X}{\bout{a}{m} \varp{X}}}{1}{f} \by{\lambda} 
		\pmapp{\recp{X}{\bout{a}{m} \varp{X}}}{1}{f}
%& \by{\lambda} & 
%		\pmapp{\recp{X}{\bout{a}{m} \varp{X}}}{1}{f}
\end{eqnarray*}
The encoding preserves also typing; associated derivations are given in \cite{KouzapasPY15}.
\qed
\end{example}



\begin{comment}
\subsection{Encoding Recursion into Abstraction Passing}\label{ss:fullfotoho}

Encoding the constructs for recursion present in $\sessp$ as process-passing
communication requires to follow the fundamental
principle of copying the process that needs to exhibit recursive behaviour.
The primitive recursor operation creates copies of a process and uses them
as continuations.

We use an example to demostrate our basic intuitions:
%
\begin{example}
	Assume process $P = \recp{X}{\bout{n}{m} \rvar{X}}$. We have
%
	\begin{eqnarray}
		\label{ex:rec1}
		P \scong \bout{n}{m} \recp{X}{\bout{n}{m} \rvar{X}} 
	\end{eqnarray}
%
	\noi The above process emits to its environment infinitely many send actions of channel $m$ along channel $n$.
	Name $n$ includes the recursive
	variable $\rvar{X}$, so the type for $n$ should be recursive.
%
	\[
		\recp{X}{\bout{n}{m} \rvar{X}} \by{\bactout{n}{m}} \recp{X}{\bout{n}{m} \rvar{X}}
	\]
%
	To get a better understanding of how name $n$ is handled
	on such scenarios, consider the process:
	\[
		P \scong \newsp{a}{\bout{a}{n} \inact \Par \recp{X}{\binp{a}{x} \bout{x}{m} (\bout{a}{x} \inact \Par \rvar{X})}}
		%\red \newsp{a}{\bout{n}{m} (\bout{a}{n} \inact \Par \recp{X}{\binp{a}{x} \bout{x}{m} (\bout{a}{x} \inact \Par \rvar{X}))}}
	\]
%
	\noi The above process exhibits the same behaviour as
	process~\ref{ex:rec1}.
	Endpoint $n$ is being passed sequentially on copies of the 
	same process to achieve the effect of infinite sending of value $m$.
%
	\begin{eqnarray*}
		P	&\scong&	\newsp{a}{\bout{a}{n} \inact \Par \recp{X}{\binp{a}{x} \bout{x}{m} (\bout{a}{x} \inact \Par \rvar{X})}}\\
			&\red&		\newsp{a}{\bout{n}{m} (\bout{a}{n} \inact \Par \recp{X}{\binp{a}{x} \bout{x}{m} (\bout{a}{x} \inact \Par \rvar{X}))}}\\
			&\by{\bactout{n}{m}}& \newsp{a}{\bout{a}{n} \inact \Par \recp{X}{\binp{a}{x} \bout{x}{m} (\bout{a}{x} \inact \Par \rvar{X})}}\\
			&\scong&	P
	\end{eqnarray*}
%
	\noi If we want to apply the same principles on higher order semantics we should first
	abstract the recursive process:
%
	\[
		\recp{X}{\binp{a}{x} \bout{x}{m} ( \rvar{X} \Par \bout{a}{x} \inact)}
	\]
%
	\noi as
%
	\[
		V \scong (z) \bout{n}{m} \binp{z}{X} \newsp{s}{\appl{X}{s} \Par \bout{\dual{s}}{\abs{z}{\appl{X}{z}}} \inact}
	\]
%
	So the entire process can be written as:
	\[
		P \scong \newsp{s_1}{\bout{s_1}{V} \inact \Par \binp{\dual{s_1}}{X} \newsp{s_2}{\appl{X}{s_2} \Par \bout{\dual{s_2}}{\abs{z}{\appl{X}{z}}} \inact}}	
	\]
%
	\noi where abstraction $V$ is copied and passed to itself
	infinitely many times:
	\[
		\begin{array}{rcl}
			P &\scong& \newsp{s_1}{\bout{s_1}{V} \inact \Par \binp{\dual{s_1}}{X} \newsp{s_2}{\appl{X}{s_2} \Par \bout{\dual{s_2}}{\abs{z}{\appl{X}{z}}} \inact}} \\
			&\red&
			\newsp{s_2}{\bout{\dual{s_2}}{V} \inact \Par \bout{n}{m} \binp{s_2}{X} \newsp{s}{\appl{X}{s} \Par \bout{\dual{s}}{{z}{\appl{X}{z}} \inact}}}\\
			&\by{\bactout{n}{m}}&
			\newsp{s_2}{\bout{\dual{s_2}}{V} \inact \Par \binp{s_2}{X} \newsp{s}{\appl{X}{s} \Par \bout{\dual{s}}{\abs{z}{\appl{X}{z}} \inact}}}\\
			&\scong_\alpha&
			P
		\end{array}
	\]
%

	\noi In the typing setting, abstraction $V$ has a linear type:
	\[
		m: U; \es; n: \btout{U} S_1 \tinact \proves
		(z) \bout{n}{m} \binp{z}{X} \newsp{s_2}{\appl{X}{s_2} \Par \bout{\dual{s_2}}{\abs{z{\appl{X}{z}} } \inact} \hastype
		\lhot{S_2}
	\]
	because of the free occurence of session channel $n$ in $V$,
	i.e.\ we cannot apply typing rule $\trule{Prom}$ to the latter
	judgement.

	\noi But when passed, abstraction $V$ is applied in a shared manner, i.e.\ two
	copies of the abstraction are instantiated, thus the whole
	encoding is untypable: 
	\[
		\Gamma; X: \lhot{S_2}; \es \not\proves \newsp{s_2}{\appl{X}{s_2} \Par \bout{\dual{s_2}}{\abs{z} \appl{X}{z}}} \inact}
	\]
%
	\noi The untypability problem would not exist
	provided that the abstraction being passed were not linear.

	\noi A typable solution of the above example would be first to
	define a shared abstraction by replacing the free
	occurence of session name $n$ with an abstraction variable:
%
	\[
		V' = (z, x) \bout{x}{m} \binp{z}{X} \newsp{s}{\appl{X}{s, x} \Par \bout{\dual{s}}{\abs{z, x}{\appl{X}{z, x}} } \inact}
	\]
%
	Abstraction $V'$ can be typed using a shared type:
	\[
		\tree{
			m: U_1; \es; \es \proves
			(z, x) \bout{x}{m} \binp{z}{X} \newsp{s}{\appl{X}{s, x} \Par \bout{\dual{s}}{\abs{z, x}{\appl{X}{z, x}} } \inact}
			\hastype \lhot{U_2}
		}{
			m: U_1; \es; \es \proves
			(z, x) \bout{x}{m} \binp{z}{X} \newsp{s}{\appl{X}{s, x} \Par \bout{\dual{s}}{\abs{z, x}{\appl{X}{z, x}} } \inact}
			\hastype \shot{U_2}
		}~~\trule{Prom}
	\]
%
	\noi and the definition and behaviour of the recursive process, becomes:
%
	\begin{eqnarray*}
		P' &\scong&	\newsp{s_1}{\bout{s_1}{V} \inact \Par \binp{\dual{s_1}}{X} \newsp{s_2}{\appl{X}{s_2, n} \Par \bout{\dual{s_2}}{\abs{z,x}{\appl{X}{z,x}}} \inact}}\\
		&\red&		\newsp{s_2}{\bout{\dual{s_2}}{V} \inact \Par \bout{n}{m} \binp{s_2}{X} \newsp{s}{\appl{X}{s, n} \Par \bout{\dual{s}}{\abs{z, x}{\appl{X}{z, x}}} \inact}}\\
		&\by{\bactout{n}{m}}& \newsp{s_2}{\bout{\dual{s_2}}{V} \inact \Par \binp{s_2}{X} \newsp{s}{\appl{X}{s, n} \Par \bout{\dual{s}}{\abs{z, x}{\appl{X}{z, x}}} \inact}}\\
		&\scong_\alpha& P'
	\end{eqnarray*}
%
	\noi Session channel $n$ is passed and applied
	together with the recursive process.
\end{example}

A preliminary tool to encode the $\sessp$ recursion primitives would be to
provide a mapping from processes to processes with no free names.
We require some auxiliary definitions.
%
\begin{definition}\myrm 
	Let $\vmap{\cdot}: 2^{\mathcal{N}} \longrightarrow \mathcal{V}^\omega$
	be a map of sequences names to sequences of variables, defined
	inductively as follows:
%
\[
	\vmap{n} = x_n \qquad \qquad \qquad \vmap{n \cat \tilde{m}} = x_n \cat \vmap{\tilde{m}}
\]
\end{definition}

Given a process $P$, we write $\ofn{P}$ to denote the
\emph{sequence} of free names of $P$, lexicographically ordered.
Intuitively, the following mapping transforms processes
with free session names into abstractions:
%
\begin{definition}\label{d:trabs}\label{d:auxmap}
	Let $\sigma$ be a set of session names.
	Define $\auxmapp{\cdot}{\mathsf{v}}{\sigma}: \HOp \to \HOp$  as in Fig.~\ref{f:auxmap}.
%
\begin{figure}[t]
\[
	\begin{array}{rcl}
		\auxmapp{\news{n} P}{\sigma}{\mathsf{v}} &\bnfis& \news{n} \auxmapp{P}{\mathsf{v}}{{\sigma \cat n}}
		\vspace{1mm} \\

%		\auxmapp{\bout{n}{\abs{x}{Q}} P}{\mathsf{v}}{\sigma} &\bnfis&
%		\left\{
%		\begin{array}{rl}
%			\bout{x_n}{\abs{x,\vmap{\ofn{P}}}{\auxmapp{Q}{\mathsf{v}}{\sigma}}} \auxmapp{P}{\mathsf{v}}{\sigma} & n \notin \sigma\\
%			\bout{n}{\abs{x,\vmap{\ofn{P}}}{\auxmapp{Q}{\mathsf{v}}{\sigma}}} \auxmapp{P}{\mathsf{v}}{\sigma} & n \in \sigma
%		\end{array}
%		\right.

		\auxmapp{\bout{n}{\abs{x}{Q}} P}{\mathsf{v}}{\sigma} &\bnfis&
		\left\{
		\begin{array}{rl}
			\bout{x_n}{\abs{x}{\auxmapp{Q}{\mathsf{v}}{\sigma}}} \auxmapp{P}{\mathsf{v}}{\sigma} & n \notin \sigma\\
			\bout{n}{\abs{x}{\auxmapp{Q}{\mathsf{v}}{\sigma}}} \auxmapp{P}{\mathsf{v}}{\sigma} & n \in \sigma
		\end{array}
		\right.
		\vspace{1mm}	\\ 

%		\auxmapp{\bout{n}{m} P}{\mathsf{v}}{\sigma} &\bnfis&
%		\left\{
%		\begin{array}{rl}
%		    \bout{n}{m}\auxmapp{P}{\mathsf{v}}{\sigma} & n, m \in \sigma \\
%		    \bout{x_n}{m}\auxmapp{P}{\mathsf{v}}{\sigma} & n \not\in \sigma, m \in \sigma \\
%		    \bout{n}{x_m}\auxmapp{P}{\mathsf{v}}{\sigma} & n \in \sigma, m \not\in \sigma \\
%		    \bout{x_n}{x_m}\auxmapp{P}{\mathsf{v}}{\sigma} & n, m \not\in \sigma 
%		\end{array}
%		\right.
%		\vspace{1mm} \\ 

		\auxmapp{\binp{n}{X} P}{\mathsf{v}}{\sigma} &\bnfis&
		\left\{
		\begin{array}{rl}
			\binp{x_n}{X} \auxmapp{P}{\mathsf{v}}{\sigma} & n \notin \sigma\\
			\binp{n}{X} \auxmapp{P}{\mathsf{v}}{\sigma} & n \in \sigma
		\end{array}
		\right.
		\vspace{1mm}	\\ 
%		\auxmapp{\binp{n}{x}P}{\mathsf{v}}{\sigma} &\bnfis&
%		\left\{
%		\begin{array}{rl}
%		    \binp{n}{x}\auxmapp{P}{\mathsf{v}}{\sigma} & n \in \sigma \\
%		    \binp{x_n}{x}\auxmapp{P}{\mathsf{v}}{\sigma} & n \not\in \sigma 
%		\end{array}
%		\right.
%		\vspace{1mm} \\ 
		\auxmapp{\bsel{n}{l} P}{\mathsf{v}}{\sigma} &\bnfis&
		\left\{
		\begin{array}{rl}
			\bsel{x_n}{l} \auxmapp{P}{\mathsf{v}}{\sigma} & n \notin \sigma\\
			\bsel{n}{l} \auxmapp{P}{\mathsf{v}}{\sigma} & n \in \sigma
		\end{array}
		\right.
		\vspace{1mm} \\
		\auxmapp{\bbra{n}{l_i:P_i}_{i \in I}}{\mathsf{v}}{\sigma} &\bnfis&
		%\auxmapp{\bsel{n}{l} P}{\mathsf{v}}{\sigma} &\bnfis&
		\left\{
		\begin{array}{rl}
			\bbra{x_n}{l_i:\auxmapp{P_i}{\mathsf{v}}{\sigma}}_{i \in I}  & n \notin \sigma\\
			\bbra{n}{l_i:\auxmapp{P_i}{\mathsf{v}}{\sigma}}_{i \in I}  & n \in \sigma
		\end{array}
		\right.
		\vspace{1mm} \\
		\auxmapp{\appl{\X}{n}}{\mathsf{v}}{\sigma} &\bnfis&
		\left\{
		\begin{array}{rl}
			\appl{\X}{x_n} & n \notin \sigma\\
			\appl{\X}{n} & n \in \sigma\\
		\end{array}
		\right. 
%		\auxmapp{\inact}{\mathsf{v}}{\sigma} &\bnfis& \inact\\
%		\auxmapp{P \Par Q}{\mathsf{v}}{\sigma} &\bnfis& \auxmapp{P}{\mathsf{v}}{\sigma} \Par \auxmapp{Q}{\mathsf{v}}{\sigma} 
	\end{array}
\]
\caption{\label{f:auxmap} The auxiliary map (cf. Def.~\ref{d:auxmap}) 
used in the encoding of first-order communication with recursive definitions into higher-order communication (Def.~\ref{d:enc:fotohorec}).
The mapping is defined homomorphically for inaction and parallel composition.}
\end{figure}
\end{definition}

Given a process $P$ with $\fn{P} = m_1, \cdots, m_n$, we are interested in its associated (polyadic) abstraction, which is defined as
$\abs{x_1, \cdots, x_n}{\auxmapp{P}{\mathsf{v}}{\es} }$, where $\vmap{m_j} = x_j$, for all $j \in \{1, \ldots, n\}$.
This transformation from processes into abstractions can be reverted by
using abstraction and application with an appropriate sequence of session names:
%
\begin{proposition}\myrm
	Let $P$ be a \HOp process with $\tilde{n} = \ofn{P}$.
	Also, suppose $\tilde{x} = \vmap{\tilde{n}}$.
%	Also, let $A_P$ be the polyadic abstraction $\abs{\tilde{x}}\auxmapp{P}{\mathsf{v}}{\emptyset}$ (cf. Def.~\ref{d:trabs}).
	Then $P \scong \appl{X}{\tilde{n}}\subst{\abs{\tilde{x}}\auxmapp{P}{\mathsf{v}}{\emptyset}}{X}$.
%	$\appl{X}{\smap{\fn{P}}} \subst{(\vmap{\fn{P}}) \map{P}^{\emptyset}}{X} \scong P$
\end{proposition}

\begin{proof}
	The proof is an easy induction on the map $\auxmapp{P}{\mathsf{v}}{\es}$.
	We give a case since other cases are similar.

	\noi - Case: $\auxmapp{\bout{n}{m} P}{\mathsf{v}}{\es} = \bout{x_n}{x_m} \auxmapp{P}{\mathsf{v}}{\es}$

	\noi We rewrite process substitution as:
	$\appl{X}{\tilde{n}} \subst{\abs{\tilde{x}}{\bout{x_n}{y_m} \auxmapp{P}{\mathsf{v}}{\es}}}{X} = (\bout{x_n}{y_m} P) \subst{\tilde{x}}{\tilde{n}}$

	\noi If consider that $x_n, y_m \in \vmap{\tilde{n}}$ then from the definition of $\vmap{\cdot}$ we
	get that $n, m \in \tilde{n}$. Furthermore by the fact that $\tilde{n}$ and $\vmap{\tilde{n}}$ are
	ordered, substitution becomes:
	$\bout{n}{m} \auxmapp{P}{\mathsf{v}}{\es} \subst{\tilde{x}}{\tilde{n}}$.

	\noi The rest of the cases are similar.
	\qed
\end{proof}

We are now ready to define the encoding of $\sessp$
(including constructs for recursion) into strict process-passing.
Thanks to the encoding in \S\,\ref{ss:polmon}, we may use polyadicity in abstraction and application only
as syntactic sugar.
For the sake of completeness, we give again the encodings for 
finite processes and types, as
formalized 
in \S\,\ref{ss:ffotoho}.
%by $\encod{\cdot}{\cdot}{1}: \sessp^{-\mu} \to \HO$.

\begin{definition}[Full First-Order into Higher-Order]\label{d:enc:fotohorec}
	Let $f$ be a function from recursion variables to sequences of name variables.
	%Define $\fencod{\cdot}{\cdot}{2}{f}: \sessp \to \HO$ as
%
Define the typed encoding $\enco{\map{\cdot}^{2}_f, \mapt{\cdot}^{2}, \mapa{\cdot}^{2}}: \sessp \to \HO$,
\begin{figure}[t]
\[
	\begin{array}{rcll}
%		\map{\rec{X}{P}}^{2} &=& \newsp{s}{\binp{s}{\X} \map{P}^{2} \Par \bout{\dual{s}}{\abs{z \cat \vmap{\fn{P}}}{\binp{z}{\X} \map{P}^{\es}}} \inact}\\
%		\map{r}^{2} &=& \newsp{s}{\appl{\X}{s \cat \smap{\fn{P}}} \Par \bout{\dual{s}}{ \abs{z \cat \vmap{\fn{P}}}{\appl{X}{z \cat \vmap{\fn{P}}}}} \inact} \\
		\pmapp{\recp{X}{P}}{2}{f} &\defeq&
		\newsp{s}{\binp{s}{\X} \pmapp{P}{2}{{f,\{\rvar{X}\to \tilde{n}\}}} \Par \bout{\dual{s}}{\abs{\vmap{\tilde{n}}, z } \,{\binp{z}{\X} \auxmapp{\pmapp{P}{2}{{f,\{\rvar{X}\to \tilde{n}\}}}}{\mathsf{v}}{\es}}} \inact} & \quad \tilde{n} = \ofn{P} \\ 
		\pmapp{\rvar{X}}{2}{f} &\defeq& \newsp{s}{\appl{\X}{\tilde{n}, s} \Par \bbout{\dual{s}}{ \abs{\vmap{\tilde{n}},z}\,\,{\appl{X}{ \vmap{\tilde{n}}, z}}} \inact} & \quad \tilde{n} = f(\rvar{X}) \\
		\pmapp{\bout{k}{n} P}{2}{f}	&\defeq&	\bout{k}{ \abs{z}{\,\binp{z}{X} \appl{X}{n}} } \pmapp{P}{2}{f} \\
		\pmapp{\binp{k}{x} Q}{2}{f}	&\defeq&	\binp{k}{X} \newsp{s}{\appl{X}{s} \Par \bout{\dual{s}}{\abs{x}{\pmapp{Q}{2}{f}}} \inact} \\


		\pmapp{\bsel{s}{l} P}{2}{f} &\defeq& \bsel{s}{l} \pmapp{P}{2}{f} \qquad
		\pmapp{\bbra{s}{l_i: P_i}_i \in I}{2}{f} \defeq \bbra{s}{l_i: \pmapp{P_i}{2}{f}}_i \in I\\
		\pmapp{\bout{k}{\abs{\tilde{x}}{Q}} P}{2}{f} &\defeq& \bout{k}{\abs{\tilde{x}}{\pmapp{Q}{2}{f}}} \pmapp{P}{2}{f} \qquad
		\pmapp{\binp{k}{X} P}{2}{f} \defeq \binp{k}{X} \pmapp{P}{2}{f}\\

		\pmapp{P \Par Q}{2}{f} &\defeq& \pmapp{P}{2}{f} \Par \pmapp{Q}{2}{f} \qquad
		\pmapp{\news{n} P}{2}{f} \defeq \news{n} \pmapp{P}{2}{f} \qquad

		\pmapp{\inact}{2}{f} \defeq \inact\\


		\tmap{\btout{S_1} {S} }{2}	&\defeq&	\bbtout{\lhot{\btinp{\lhot{\tmap{S_1}{2}}}\tinact}} \tmap{S}{2}  \\
		\tmap{\btinp{S_1} S }{2}	&\defeq&	\bbtinp{\lhot{\btinp{\lhot{\tmap{S_1}{2}}}\tinact}} \tmap{S}{2} \\
		\tmap{\bbtout{\chtype{S_1}} {S} }{2}	&\defeq&	\bbtout{\shot{\btinp{\shot{\chtype{\tmap{S_1}{2}}}}\tinact}} \tmap{S}{2}  \\
		\tmap{\bbtinp{\chtype{S_1}} {S} }{2}	&\defeq&	\bbtinp{\shot{\btinp{\shot{\chtype{\tmap{S_1}{2}}}}\tinact}} \tmap{S}{2} \\

		\tmap{\btout{L} S}{2} &\defeq& \btout{L} \tmap{S}{2}\\
		\tmap{\btinp{L} S}{2} &\defeq& \btinp{L} \tmap{S}{2}\\
		\tmap{\btsel{l_i: S_i}_{i \in I}}{2} &\defeq& \btsel{l_i: \tmap{S_i}{2}}_{i \in I}\\
		\tmap{\btbra{l_i: S_i}_{i \in I}}{2} &\defeq& \btbra{l_i: \tmap{S_i}{2}}_{i \in I}\\

		\tmap{\vart{t}}{2} &\defeq& {t} \qquad
		\tmap{\trec{t}{S}}{2} \defeq \trec{t}{\tmap{S}{2}} \qquad
		\tmap{\tinact}{2} \defeq \tinact\\

		\mapa{\bactout{n}{m}}^{2} &\defeq&   \bactout{n}{\abs{z}{\,\binp{z}{X} \appl{X}{m}} } \\
		\mapa{\bactinp{n}{m}}^{2} &\defeq&   \bactinp{n}{\abs{z}{\,\binp{z}{X} \appl{X}{m}} } \\

		\mapa{\bactout{n}{\abs{\tilde{x}}{P}}}^{2} &\defeq& \bactout{n}{\abs{\tilde{x}}{\pmapp{P}{2}{\es}}} \qquad
		\mapa{\bactinp{n}{\abs{\tilde{x}}{P}}}^{2} \defeq \bactinp{n}{\abs{\tilde{x}}{\pmapp{P}{2}{\es}}}\\
		\mapa{\tau} &\defeq& \tau
	\end{array}
\]
\caption{\label{f:enc:fotohorec}
Typed encoding of first-order communication into higher-order communication (cf.~Defintion~\ref{d:enc:fotohorec}).
Mappings 
$\map{\cdot}^2$,
$\mapt{\cdot}^2$, 
and 
$\mapa{\cdot}^2$
are homomorphisms for the other processes/types/labels. 
}

\end{figure}
where mappings $\map{\cdot}^{2}$, $\mapt{\cdot}^{2}$, $\mapa{\cdot}^{2}$
are as in Fig.~\ref{f:enc:fotohorec}.
\end{definition}

\begin{remark}
Let $\Delta = \{n_1:S_1, \ldots, n_m:S_m\}$ be a session environment.
Write $\tilde{S}_{\Delta} = S_1, \ldots, S_m$.
	We define  mapping $\mapt{\cdot}^{2}$ on (first-order) shared environments $\Gamma$ as follows:
	\begin{eqnarray*}
	\mapt{\Gamma \cat k:\chtype{S}}^{2} & =  & \mapt{\Gamma}^{2} \cat k:\mapt{\chtype{S}}^{2} \\
	\tmap{\Gamma \cat \rvar{X}:\Delta}{2} & = & \tmap{\Gamma}{2} \cat X:\shot{(\tilde{S}_{\Delta}\,,\,S^*)}\qquad 
		\text{where
	$S^* = \trec{t}{\big((\tilde{S}_{\Delta}\,,\, \btinp{\vart{t}}\tinact)\big)}$}
	\end{eqnarray*}
\end{remark}

%\begin{proposition}\myrm
%	Encoding $\fencod{\cdot}{\cdot}{2}{f}: \sessp \to \HO$  
%	is type-preserving (cf. Def.~\ref{def:ep}\,(1)).
%\end{proposition}


\begin{proposition}[Type Preservation, Full First-Order into Higher-Order]\label{prop:typepres2}
Let $P$ be a  $\sessp$ process.
If			$\Gamma; \emptyset; \Delta \proves P \hastype \Proc$ then 
			$\mapt{\Gamma}^{2}; \emptyset; \mapt{\Delta}^{2} \proves \map{P}_f^{2} \hastype \Proc$. 
\end{proposition}

\begin{proof}
By induction on the inference $\Gamma; \emptyset; \Delta \proves P \hastype \Proc$. 
Details in Appendix~\ref{app:enc_sesp_to_HO}.
	\qed
\end{proof}

%\begin{proposition}\myrm
%	Encoding $\fencod{\cdot}{\cdot}{2}{f}: \sessp \to \HO$ 
%	enjoys operational correspondence (cf. Def.~\ref{def:ep}\,(2)).
%\end{proposition}

The following proposition formalizes our strategy  for encoding recursive definitions
as passing of polyadic abstractions:
\begin{proposition}[Operational Correspondence for Recursive Processes]
Let $P$ and $P_1$ be $\sessp$ processes s.t. $P =\recp{X}{P'}$ and $P_1 = P'\subst{\recp{X}{P'}}{\rvar{X}} \scong P$. \\
If $P_1 \hby{\ell} P_2$ then there exist
processes $R_1$, $R_2$,
context $C$, 
action $\ell'$,
and 
mapping $f_1$
such that: 
(1)~$\map{P}_f^{2} \hby{\tau} \map{P'}_{f_1}^{2}\subst{\context{C}{\map{P'}_{f_1}^{2}}}{X} = R_1$; and
(2)~$R_1 \Hby{\ell'} R_2$, with $\ell' = \mapa{\ell}^{2}$.
\end{proposition}

\begin{proof}[Sketch]
Part (1) follows directly from the definition of typed encoding for processes $\map{\cdot}_f^{2}$
(Defintion~\ref{d:enc:fotohorec}).
We have that 
context $\context{C}{\cdot} = \abs{\tilde{m}}\binp{z}{X}\auxmapp{\cdot}{\mathsf{v}}{\sigma}$
(where $\tilde{m} = \ofn{P'},z$)
and
$f_1 = f, \{\rvar{X} \to \ofn{P'}\}$.
Part (2) relies on  Prop.~\ref{p:opcorrfho}.
\qed
\end{proof}

\begin{proposition}[Operational Correspondence, Full First-Order into Higher-Order]
\label{p:auxfullfho}
Let $P$ be a  $\sessp$ process.
If $\Gamma; \emptyset; \Delta \proves P \hastype \Proc$ then
		\begin{enumerate}[a)]
			\item	 
			   If  $\stytra{\Gamma}{\ell_1}{\Delta}{P}{\Delta'}{P'}$
			   then  $\exists \ell_2$ s.t. \\
			    $\wtytra{\mapt{\Gamma}^{2}}{\ell_2}{\mapt{\Delta}^{2}}{\map{P}_f^{2}}{\mapt{\Delta'}^{2}}{\map{P'}_f^{2}}$
			    and $\ell_2 = \mapa{\ell_1}^{2}$.
			\item   
			If  $\wtytra{\mapt{\Gamma}^{2}}{\ell_2}{\mapt{\Delta}^{2}}{\map{P}^{2}_f}{\mapt{\Delta'}^{2}}{Q}$
			   then $\exists \ell_1, P'$ s.t.  \\
			    (i)~$\stytra{\Gamma}{\ell_1}{\Delta}{P}{\Delta'}{P'}$,
			    (ii)~$\ell_2 = \mapa{\ell_1}^{2}$, 
			    (iii)~$\wbb{\mapt{\Gamma}^{2}}{\ell}{\mapt{\Delta'}^{2}}{\map{P'}_f^{2}}{\mapt{\Delta'}^{2}}{Q}$.
			    \end{enumerate}
\end{proposition}

\begin{proof}[Sketch]
The proof follows similar lines as the proof of Prop.~\ref{p:opcorrfho},
using 
Prop.~\ref{p:auxfullfho} in the extra cases $P =\recp{X}{P'}$ and $P = \rvar{X}$.
	\qed
\end{proof}

\end{comment}


%%%%%%%%%%%%%%%%%%%%%%%%%%%%%%%%%%%%%%%%%%%%%%%%%%%%%%%
%  HOp ---> pi
%%%%%%%%%%%%%%%%%%%%%%%%%%%%%%%%%%%%%%%%%%%%%%%%%%%%%%%


\subsection{From \HOp to \sessp}
\label{subsec:HOp_to_p}

We now discuss the encodability of  $\HO$ into $\sessp$ where
we essentially follow the representability result put forward by 
Sangiorgi~\cite{San92,SaWabook}, but casted in the 
setting of session-typed communications. 
Intuitively, the strategy represents the exchange of a process 
with the exchange of a freshly generated \emph{trigger name}. 
Trigger names are used to activate copies of the process, 
which now becomes a persistent 
resource represented by an input-guarded replication.
In our calculi, a session name 
is a linear resource and cannot be replicated.
Consider the following (naive) adaptation of 
Sangiorgi's strategy in which session names are used are triggers and 
exchanged processes would be have to used exactly once:
%
\[
	\begin{array}{lcl}
		\pmap{\bout{u}{\abs{x}{Q}} P}{n} & \defeq &  \newsp{s}{\bout{u}{s} (\pmap{P}{n} \Par \binp{\dual{s}}{x} \pmap{Q}{n})} \\
		\pmap{\binp{u}{x} P}{n} & \defeq& \binp{u}{x} \pmap{P}{n}\\
		\pmap{\appl{x}{u}}{n} & \defeq & \bout{x}{u} \inact
	\end{array}
\]
%
with the remaining \HOp constructs being mapped homomorphically.
Although $\pmap{\cdot}{n}$ captures the correct semantics when
dealing with systems that allow only linear abstractions,
it suffers from non-typability in the presence
of shared abstractions. For instance,
mapping for $P = \bout{n}{\abs{x}{\bout{x}{m}\inact}} \inact \Par \binp{\dual{n}}{x} (\appl{x}{s_1} \Par \appl{x}{s_2})$
would be:
%
\[
	\pmap{P}{n} \defeq
	\newsp{s}{\bout{n}{s} \binp{\dual{s}}{x} \bout{x}{m} \inact \Par \binp{\dual{n}}{x} (\bout{x}{s_1} \inact \Par \bout{x}{s_2} \inact)}
\]
%
The above process is non typable since processes $(\bout{x}{s_1} \inact$ and $\bout{x}{s_2} \inact)$
cannot be put in parallel because they do not have disjoint session environments.

The correct approach would be to use replicated shared names
as triggers instead of session names, when dealing with shared abstractions. 
Below we write $\repl{} P$ as a shorthand notation for $\recp{X}{(P \Par \varp{X})}$.

\begin{definition}[Encoding \HOp to \sessp]\myrm
	\label{def:enc:HOp_to_p}
	Define encoding
	$\enco{\map{\cdot}^2, \mapt{\cdot}^2, \mapa{\cdot}^2}: \tyl{L}_{\HOp} \to \tyl{L}_{\sessp}$
	with mappings 
	$\map{\cdot}^{2}$, $\mapt{\cdot}^{2}$, $\mapa{\cdot}^{2}$ as
	in \figref{fig:enc:HOp_to_p}.
\end{definition}
%

\begin{figure}[t]
	\[
	\begin{array}{rcl}
		\pmap{\bout{u}{\abs{x}{Q}} P}{2} & \defeq &  \left\{
		\begin{array}{ll}
			\newsp{a}{\bout{u}{a} (\pmap{P}{2} \Par \repl{} \binp{a}{y} \binp{y}{x} \pmap{Q}{2})\,} & s \notin \fn{Q} \\
			\newsp{s}{\bout{u}{\dual{s}} (\pmap{P}{2} \Par \binp{s}{y} \binp{y}{x} \pmap{Q}{2})\,} & \textrm{otherwise} %\dk{Q \textrm{ linear}} \\
		\end{array}
		\right.
		\\
		\pmap{\binp{u}{x} P}{2} &\defeq&  \binp{u}{x} \pmap{P}{2}
		\\
		\pmap{\appl{x}{u}}{2} & \defeq & \newsp{s}{\bout{x}{s} \bout{\dual{s}}{u} \inact}
		\\
		\pmap{\appl{(\abs{x} P)}{u}}{2} & \defeq & \newsp{s}{\bout{\dual{n}}{s} \bout{\dual{s}}{u} \inact \Par \binp{n}{y} \binp{y}{x} \pmap{P}{2}}
		\\

%		\qquad
%		\left\{
%		\begin{array}{rcl}
%			v = x && \textrm{ if } V = x\\
%			v = m && \textrm{ if } V = \abs{x} P \wedge m \textrm{ fresh} \\
%		\end{array}
%		\right.
%		\\
		\\
		\tmap{\btout{\shot{S}}S_1}{2} & \defeq & \bbtout{\chtype{\btinp{\tmap{S}{2}}\tinact}}\tmap{S_1}{2} \\
		\tmap{\btinp{\shot{S}}S_1}{2} & \defeq & \bbtinp{\chtype{\btinp{\tmap{S}{2}}\tinact}}\tmap{S_1}{2} \\

		\tmap{\btout{\lhot{S}}S_1}{2} & \defeq & \bbtout{\btinp{\tmap{S}{2}}\tinact}\tmap{S_1}{2} \\
		\tmap{\btinp{\lhot{S}}S_1}{2} & \defeq & \bbtinp{\btinp{\tmap{S}{2}}\tinact}\tmap{S_1}{2} \\
		\mapa{\news{\tilde{m_1}}\bactout{n}{\abs{ x}{P}} }^{2} &  \defeq & \news{m \cdot \tilde{m_1}} \bactout{n}{m} \\
		\mapa{\bactinp{n}{\abs{ x}{P}} }^{2} &  \defeq & \bactinp{n}{m}
	\end{array}
	\]
	\caption{
		Typed encoding of higher-order  into first-order communication (cf.~Defintion~\ref{def:enc:HOp_to_p}).
		\label{fig:enc:HOp_to_p}
		Mappings 
		$\map{\cdot}^3$,
		$\mapt{\cdot}^3$, 
		and 
		$\mapa{\cdot}^3$
		are homomorphisms for the other processes/types/labels. 
	}
\end{figure}


\begin{proposition}[Type Preservation, \HOp into \sessp]\myrm
	\label{prop:typepres_HOp_to_p}
	Let $P$ be a \HOp process. 
	If $\Gamma; \emptyset; \Delta \proves P \hastype \Proc$ then 
	$\mapt{\Gamma}^{2}; \emptyset; \mapt{\Delta}^{2} \proves \map{P}^{2} \hastype \Proc$.
\end{proposition}

\begin{proof}
	By induction on the inference $\Gamma; \emptyset; \Delta \proves P \hastype \Proc$. 
	Details in \propref{app:prop:typepres_HOp_to_p}
	(Page~\pageref{app:prop:typepres_HOp_to_p}).
	\qed
\end{proof}

\begin{remark}
	As stated in  \cite[Lem.\,5.2.2]{SangiorgiD:expmpa}, 
	due to the replicated trigger,  
	operational correspondence in \defref{def:ep} is refined to prove  
	full abstraction: 
	e.g., completeness of the case $\ell_1 \neq \tau$, is changed as follows.
	Suppose:
%
	\[
		\horel{\Gamma}{\Delta}{P}{\hby{\ell_1}}{\Delta'}{P'}{}
	\]
%
	If $\ell_1 = \news{\tilde{m}} \bactout{n}{\abs{ x}{R}}$, 
	then %$\exists \ell_2, Q$ s.t. 
%
	\[
		\horel{\tmap{\Gamma}{2}}{\tmap{\Delta}{2}}{\pmap{P}{2}}{\hby{\ell_2}}{\tmap{\Delta'}{2}}{Q}{}
	\]
%
	where  $\ell_2 = (\nu a)\bactout{n}{a}$ and
	$Q = \pmap{P' \Par  \repl{} \binp{a}{y} \binp{y}{x} R}{2}$.

	\noi Similarly, if  
	%$\stytraarg{\Gamma}{\ell_1}{\Delta}{P}{\Delta'}{P'}{}$
	%with 
	$\ell_1 = \bactinp{n}{\abs{ x}{R}}$, 
	then %$\exists \ell_2, Q$ s.t.
	\[
		\horel{\tmap{\Gamma}{2}}{\tmap{\Delta}{2}}{\pmap{P}{2}}{\hby{\ell_2}}{\tmap{\Delta'}{2}}{Q}{}
	\]
	where $\ell_2 = \bactout{n}{a}$ and
	$\pmap{P'}{2} \wb \newsp{a}{Q \Par  \repl{} \binp{a}{y} \binp{y}{x} \pmap{R}{2}}$.
	Soundness is stated in a symmetric way.
	%Operational correspondence for the encoding in~\defref{d:enc:hopitopi}
	%is different from that in~\defref{def:ep}, due to triggers. 
	%In particular,  completeness differs when $\ell_1 \neq \tau$.
	%This way, e.g., if  
	%$\stytraarg{\Gamma}{\ell_1}{\Delta}{P}{\Delta'}{P'}{}$
	%with $\ell_1 = (\nu \tilde{m})\bactout{n}{\abs{ x}{R}}$, 
	%then %$\exists \ell_2, Q$ s.t. 
	%$\stytraarg{\mapt{\Gamma}^2}{\ell_2}{\mapt{\Delta}^2}{\map{P}^2}{\mapt{\Delta'}^2}{Q}{}$,
	%where 
	%$\ell_2 = (\nu a)\bactout{n}{a}$ and
	%$Q = \pmap{P' \Par  \repl{} \binp{a}{y} \binp{y}{x} R}{2}$.
	%This 
	%statement, essential in proofs of full abstraction,
	%is the same given by Sangiorgi~\cite{SangiorgiD:expmpa}.
	%Completeness is as in~\defref{def:ep} when  $\ell_1 = \tau$.
	%See~\cite{KouzapasPY15} for details.
\end{remark}

This last remark is stated formally in the next proposition:
%
\begin{proposition}[Operational Correspondence, \HOp into \sessp]\myrm
	\label{prop:op_corr_HOp_to_p}
	Let $P$ be an  $\HOp$ process such that  $\Gamma; \emptyset; \Delta \proves P \hastype \Proc$.
	
	\begin{enumerate}[1.]
		\item Suppose $\horel{\Gamma}{\Delta}{P}{\hby{\ell_1}}{\Delta'}{P'}$.
		Then we have:
		\begin{enumerate}[a)]
			\item
				If  $\ell_1 = \news{\tilde{m}}\bactout{n}{\abs{x}Q}$,
				then $\exists \Gamma', \Delta'', R$ where either:
				\begin{enumerate}[-]
					\item 
						$\tmap{\Gamma}{2};\, \tmap{\Delta}{2} \proves  \pmap{P}{2} 
						\hby{\mapa{\ell_1}^{2}}
						\Gamma' \cdot \tmap{\Gamma}{2};\, \tmap{\Delta'}{2} \proves \pmap{P'}{2} \Par \repl{} \binp{a}{y} \binp{y}{x} \pmap{Q}{2}$
					\item 
						$\tmap{\Gamma}{2};\, \tmap{\Delta}{2} \proves \pmap{P}{2} 
						\hby{\mapa{\ell_1}^{2}}
						\tmap{\Gamma}{2};\, \Delta'' \proves \pmap{P'}{2} \Par \binp{s}{y} \binp{y}{x} \pmap{Q}{2}$
				\end{enumerate}

			\item
				If   
				$\ell_1 = \bactinp{n}{\abs{y}Q}$
				then $\exists R$ where
				either
				\begin{enumerate}[-]
					\item 
						$\tmap{\Gamma}{2};\, \tmap{\Delta}{2} \proves \pmap{P}{2} 
						\hby{\mapa{\ell_1}^{2}}
						\Gamma';\, \tmap{\Delta''}{2} \proves  R$, for some $ \Gamma'$
						and \\ 
						$\horel{\tmap{\Gamma}{2}}{\tmap{\Delta'}{2}}{\pmap{P'}{2}}{\wb}{\tmap{\Delta''}{2}}{\newsp{a}{R \Par \repl{} \binp{a}{y} \binp{y}{x} \pmap{Q}{2}}}$
					\item 
						$\tmap{\Gamma}{2};\, \tmap{\Delta}{2} \proves \pmap{P}{2}
						\hby{\mapa{\ell_1}^{2}}
						\tmap{\Gamma}{2};\, \tmap{\Delta''}{2} \proves R$, 
						and \\ 
						$\horel{\tmap{\Gamma}{2}}{\tmap{\Delta'}{2}}{\pmap{P'}{2}}{\wb}{\tmap{\Delta''}{2}}{\newsp{s}{R \Par \binp{s}{y} \binp{y}{x} \pmap{Q}{2}}}$  		
				\end{enumerate}

			\item	If
				$\ell_1 = \tau$ then either:

				\begin{enumerate}[-]
					\item	$\exists R$ such that
						\[
						\mhorel{\tmap{\Gamma}{2}}{\tmap{\Delta}{2}}{\pmap{P}{2}}
						{\hby{\tau}}
						{\tmap{\Delta'}{2}}{}{\newsp{\tilde{m}}{\pmap{P_1}{2} \Par \newsp{a}
						{\pmap{P_2}{2}\subst{a}{x} \Par \repl{} \binp{a}{y} \binp{y}{x} \pmap{Q}{2}}}}
						\]

					\item	$\exists R$ such that
						\[
						\mhorel{\tmap{\Gamma}{2}}{\tmap{\Delta}{2}}{\pmap{P}{2}}
						{\hby{\tau}}
						{\tmap{\Delta'}{2}}{}{\newsp{\tilde{m}}{\pmap{P_1}{2} \Par \newsp{s}
						{\pmap{P_2}{2}\subst{\dual{s}}{x} \Par \binp{s}{y} \binp{y}{x} \pmap{Q}{2}}}}
						\]

					\item	%$\ell_1 = \btau$ and
						$\tmap{\Gamma}{2};\, \tmap{\Delta}{2} \proves \pmap{P}{2}
						\hby{\tau}
						\tmap{\Gamma}{2};\, \tmap{\Delta'}{2} \proves \pmap{P'}{2}$

					\item	$\ell_1 = \btau$ and
						$\tmap{\Gamma}{2};\, \tmap{\Delta}{2} \proves \pmap{P}{2}
						\hby{\stau}
						\tmap{\Gamma}{2};\, \tmap{\Delta'}{2} \proves \pmap{P'}{2}$
				\end{enumerate}

%			\item	 
%				If  
%				%$\stytra{\Gamma}{\ell_1}{\Delta}{P}{\Delta'}{P_1 \Par P_2\subst{\abs{x}Q}{X}}$
%				$\ell_1 = \tau$ and $P' 	\not \scong \news{\tilde{m}}(P_1 \Par P_2\subst{\abs{x}Q}{X})$
%				then \\
%				$\mapt{\Gamma}^{2};\, \mapt{\Delta}^{2} \proves  \map{P}^{2}
%				\hby{\tau}
%				\mapt{\Gamma}^{2};\, \mapt{\Delta'}^{2} \proves  \map{P'}^{2}$.
				   			   
%			   then  $\exists \ell_2$ s.t. 
%			    $\wtytra{\mapt{\Gamma}^{3}}{\ell_2}{\mapt{\Delta}^{3}}{\map{P}^{3}}{\mapt{\Delta'}^{3}}{\map{P'}^{3}}$
%			    and $\ell_2 = \mapa{\ell_1}^{3}$.

			\item	 
				If  
				$\ell_1 \in \set{\bactsel{n}{l}, \bactbra{n}{l}}$
				%\not\in \set{\tau,\, \news{\tilde{m}}\bactout{n}{\abs{x}Q}, \, \bactinp{n}{\abs{x}Q}}$ 
				 then \\
				$\exists \ell_2 = \mapa{\ell_1}^{2}$ such that 
				$\mapt{\Gamma}^{2};\, \mapt{\Delta}^{2} \proves  \map{P}^{2}
				\hby{\ell_2}
				\mapt{\Gamma}^{2};\, \mapt{\Delta'}^{2} \proves  \map{P'}^{2}$.			
		\end{enumerate}
		
		%%%%%%% SOUNDNESSS
		\item Suppose 
		$\stytra{\mapt{\Gamma}^{2}}{\ell_2}{\mapt{\Delta}^{2}}{\map{P}^{2}}{\mapt{\Delta'}^{2}}{R}$.
			\begin{enumerate}[a)]
				\item %% soutput
					%\footnote{$\mapt{\Gamma}^{2}$ in the following three items need adjustments.}
					If  
					$\ell_2 = \news{m}\bactout{n}{m}$
					%$\stytra{\mapt{\Gamma}^{2}}{\news{m}\bactout{n}{m}}{\mapt{\Delta}^{2}}{\map{P}^{2}}{\mapt{\Delta'}^{2}}{R}$
					then 
					either 
					\begin{enumerate}[-]
					\item	$\exists P'$ such that $P \hby{\news{m} \bactout{n}{m}} P'$
						and $R = \pmap{P'}{2}$.

					\item	$\exists Q, P'$ such that $P \hby{\bactout{n}{\abs{x}Q}} P'$
						and $R = \map{P'}^{2} \Par \repl{} \binp{a}{y} \binp{y}{x} \pmap{Q}{2}$

					\item	$\exists Q, P'$ such that $P \hby{\bactout{n}{\abs{x}Q}} P'$
						and $R = \map{P'}^{2} \Par \binp{s}{y} \binp{y}{x} \pmap{Q}{2}$
					\end{enumerate}

				\item   %% sinput
					If  $\ell_2 = \bactinp{n}{m}$ 
					%$\stytra{\mapt{\Gamma}^{2}}{\bactinp{n}{m}}{\mapt{\Delta}^{2}}{\map{P}^{2}}{\mapt{\Delta'}^{2}}{R}$
					then either
					\begin{enumerate}[-]
					\item	$\exists P'$ such that $P \hby{\bactinp{n}{m}} P'$
						and $R = \pmap{P'}{2}$.

					\item	$\exists Q, P'$ such that
						$P \hby{\bactinp{n}{\abs{x}Q}} P'$\\
						and $\horel{\mapt{\Gamma}^{2}}{\mapt{\Delta'}^{2}}{\map{P'}^{2}}{\wb}{\mapt{\Delta'}^{2}}{\news{a}(R \Par \repl{} \binp{a}{y} \binp{y}{x} \pmap{Q}{2})}$
					\item	$\exists Q, P'$ such that
						$P \hby{\bactinp{n}{\abs{x}Q}} P'$\\
						and $\horel{\mapt{\Gamma}^{2}}{\mapt{\Delta'}^{2}}{\map{P'}^{2}}{\wb}{\mapt{\Delta'}^{2}}{\news{s}(R \Par \binp{s}{y} \binp{y}{x} \pmap{Q}{2})}$  
					\end{enumerate}
		
				\item   
					If  %$\stytra{\mapt{\Gamma}^{2}}{\tau}{\mapt{\Delta}^{2}}{\map{P}^{2}}{\mapt{\Delta'}^{2}}{R}$
					$\ell_2 = \tau$ 
					then $\exists P'$ such that
					$P \hby{\tau} P'$
					and $\horel{\mapt{\Gamma}^{2}}{\mapt{\Delta'}^{2}}{\map{P'}^{2}}{\wb}{\mapt{\Delta'}^{2}}{R}$.
				\item	 
					If  
					$\ell_2 \not\in \set{\bactout{n}{m}, \bactsel{n}{l}, \bactbra{n}{l}}$ 
					 then 
					$\exists \ell_1$ such that 
					$\ell_1 = \mapa{\ell_2}^{2}$ and \\
					$ \Gamma ;\, \Delta  \proves   P
					\hby{\ell_1}
					\Gamma ;\, \Delta  \proves   P'$.
		\end{enumerate}
	\end{enumerate}
\end{proposition}

\begin{proof}
	The proof is done by induction on the labelled transition system
	considering \defref{def:enc:HOp_to_p}.
	The most demaning cases are Part 1b and Part 2b where
	we require a further induction to proof bisimulation
	closure.

	Details of the proof of the most demanding cases can be
	found in \propref{app:prop:op_corr_HOp_to_p}
	(page \pageref{def:enc:HOp_to_p}).
	\qed
\end{proof}

\begin{comment}
\begin{proof}
	\noi The proof is done by transition induction.
	We conside the two parts separately.

	\noi - Part 1

	\noi - Basic Step:
 
	\noi - Subcase: $P= \bout{n}{\abs{x}{Q}} P'$ 
	and also from \defref{def:enc:HOp_to_p}
	we have that\\
	$\pmap{P}{2} = \newsp{a}{\bout{n}{a} \pmap{P'}{2} \Par \repl{} \binp{a}{y} \binp{y}{x} \pmap{Q}{2}}$

	\noi Then
%
	\begin{eqnarray*}
		\Gamma; \es; \Delta \proves P &\hby{\bactout{n}{\abs{x}{Q}}} & \Delta' \proves P'\\
		\tmap{\Gamma}{2}; \es; \tmap{\Delta}{2} \proves \pmap{P}{2} &\hby{\news{a} \bactout{n}{a}}& \tmap{\Delta}{2} \proves \pmap{P'}{2} \Par \repl{} \binp{a}{y} \binp{y}{x} \pmap{Q}{2}
	\end{eqnarray*}
%
	\noi and from \defref{def:enc:HOp_to_p}
%
	\begin{eqnarray*}
		\mapa{\bactout{n}{\abs{x}{Q}}} = \news{a} \bactout{n}{a}
	\end{eqnarray*}
%
	\noi as required.

	\noi - Subcase: $P= \bout{n}{\abs{x}{Q}} P'$ 
	and also from \defref{def:enc:HOp_to_p}
	we have that\\
	$\pmap{P}{2} = \newsp{s}{\bout{n}{\dual{s}} \pmap{P'}{2} \Par \binp{s}{y} \binp{y}{x} \pmap{Q}{2}}$
	is similar as above. 

	\noi - Subcase $P = \binp{n}{x} P'$.

	\noi - From \defref{def:enc:HOp_to_p}
	we have that
	$\pmap{P}{2} = \binp{n}{x} \pmap{P'}{2}$

	\noi Then
%
	\begin{eqnarray*}
		\Gamma; \es; \Delta \proves P &\hby{\bactinp{n}{\abs{x}{Q}}}& \Delta' \proves P' \subst{\abs{x}{Q}}{x}\\
		\tmap{\Gamma}{2}; \es; \tmap{\Delta}{2} \proves \pmap{P}{2} &\by{\bactinp{n}{a}}& \tmap{\Delta''}{2} \proves R \subst{a}{x}
	\end{eqnarray*}
%
	\noi with
%
	\begin{eqnarray*}
		\mapa{\bactinp{n}{\abs{x}{Q}}}^{2} &=& \bactinp{n}{a}
	\end{eqnarray*}
%
	It remains to show that
%
	\begin{eqnarray*}
		\tmap{\Gamma}{2}; \es; \tmap{\Delta'}{2} \proves \pmap{P' \subst{\abs{x}{Q}}{x}}{2} \wb
		\tmap{\Delta''}{2} \proves \newsp{a}{R \subst{a}{x} \Par \repl{} \binp{a}{y} \binp{y}{x} \pmap{Q}{2}}
	\end{eqnarray*}
%
	\noi The proof is an induction on the syntax structure of $P'$.
	Suppose $P' = \appl{x}{m}$, then:
%
	\begin{eqnarray*}
		\pmap{\appl{x}{m} \subst{\abs{x}{Q}}{x}}{2} &=& \pmap{Q \subst{m}{x}}{2}\\
		\newsp{a}{R \subst{a}{x} \Par \repl{} \binp{a}{y} \binp{y}{x} \pmap{Q}{2}} &=& \newsp{a}{\newsp{s}{ \bout{x}{s} \bout{\dual{s}}{m} \inact} \subst{a}{x} \Par \repl{} \binp{a}{y} \binp{y}{x} \pmap{Q}{2}}
	\end{eqnarray*}
%
	\noi The second term can be deterministically reduced as:
%
	\begin{eqnarray*}
		\mhorel{\tmap{\Gamma}{2}}{\tmap{\Delta''}{2}}{\newsp{a}{\newsp{s}{ \bout{x}{s} \bout{\dual{s}}{m} \inact} \subst{a}{x} \Par \repl{} \binp{a}{y} \binp{y}{x} \pmap{Q}{2}}}
		{\hby{\tau} \hby{\stau}}
		{\tmap{\Delta''}{2}}{}{\newsp{a}{\pmap{Q \subst{m}{x}}{2} \Par \repl{} \binp{a}{y} \binp{y}{x} \pmap{Q}{2}}}
	\end{eqnarray*}
%
	\noi which is bisimilar with:
%
	\begin{eqnarray*}
		\pmap{Q \subst{m}{x}}{2}
	\end{eqnarray*}
%
	\noi because $a$ is fresh and cannot interact anymore.

	\noi An interesting inductive step case is parallel composition. Suppose $P' = P_1 \Par P_2$. We need to show that:
%
	\begin{eqnarray*}
		&& \tmap{\Gamma}{2}; \es; \tmap{\Delta'}{2} \proves \pmap{(P_1 \Par P_2) \subst{\abs{x}{Q}}{x}}{2} \wb
		\tmap{\Delta''}{2} \proves \newsp{a}{\pmap{P_1 \Par P_2}{2} \subst{a}{x} \Par \repl{} \binp{a}{y} \binp{y}{x} \pmap{Q}{2}}
	\end{eqnarray*}
%
	\noi We know that
%
	\begin{eqnarray*}
		\horel{\tmap{\Gamma}{2}}{\tmap{\Delta_1}{2}}{\pmap{P_1\subst{\abs{x}{Q}}{x}}{2}}{&\wb&}
		{\tmap{\Delta_1''}{2}}{\newsp{a}{\pmap{P_1}{2} \subst{a}{x} \Par \repl{} \binp{a}{y} \binp{y}{x} \pmap{Q}{2}}}\\
		\horel{\tmap{\Gamma}{2}}{\tmap{\Delta_2}{2}}{\pmap{P_2\subst{\abs{x}{Q}}{x}}{2}}{&\wb&}
		{\tmap{\Delta_1''}{2}}{\newsp{a}{\pmap{P_2}{2} \subst{a}{x} \Par \repl{} \binp{a}{y} \binp{y}{x} \pmap{Q}{2}}}
	\end{eqnarray*}
%
	\noi We conclude from the congruence of $\wb$.

	\noi - The rest of the cases for Part 1 are easy to follow using \defref{def:enc:HOp_to_p}.

	\noi - Part 2.

	\noi The proof for Part 2 is straightforward following \defref{def:enc:HOp_to_p}.
	We give some distinctive cases:

	\noi - Case $P = \bout{n}{\abs{x}{Q}} P'$
%
	\begin{eqnarray*}
		\horel{\Gamma}{\Delta}{P}{&\hby{\bactout{n}{\abs{x}{Q}}}&}{\Delta'}{P'}\\
		\horel{\tmap{\Gamma}{2}}{\tmap{\Delta}{2}}{\pmap{P}{2}}{& \hby{\news{a} \bactout{n}{a}}&}{\tmap{\Delta'}{2}}{\pmap{P'}{2} \Par \repl{} \binp{a}{y} \binp{y}{s} \pmap{Q}{2}}
	\end{eqnarray*}
%
	\noi as required.

	\noi - Case $P = \binp{n}{x} P'$
%
	\begin{eqnarray*}
		\horel{\Gamma}{\Delta}{P}{&\hby{\bactinp{n}{\abs{x}{Q}}}&}{\Delta'}{P' \subst{\abs{x}}{Q}}{x}\\
		\horel{\tmap{\Gamma}{2}}{\tmap{\Delta}{2}}{\pmap{P}{2}}{& \hby{\bactinp{n}{a}}&}{\tmap{\Delta''}{2}}{\pmap{P'}{2} \subst{a}{x}}
	\end{eqnarray*}
%
	\noi We now use a similar argumentation as the input case in Part 1 to prove that:
%
	\begin{eqnarray*}
		\horel{\Gamma}{\Delta'}{P' \subst{\abs{x}{Q}}{x}}
		{\wb}
		{\tmap{\Delta''}{2}}{\newsp{a}{\pmap{P'}{2} \subst{a}{x} \Par \repl{} \binp{a}{y} \binp{y}{x} \pmap{Q}{2}}}
	\end{eqnarray*}
%
	\qed
\end{proof}
\end{comment}


%\begin{proposition}\myrm
%	Let $k$-order type $U$ and value $\abs{x}{P}$ such that $\Gamma; \es; \Delta \proves \abs{x}{P} \hastype U$;
%	then $\tmap{\Gamma}{2}; \es; \tmap{\Delta}{2} \proves \pmap{\abs{x}{P}}{2} \hastype \tmap{U}{2}$
%	with 
%\end{proposition}

\begin{comment}
\begin{proof}[Sketch]
For completeness, we 
consider the \HO process $P = {\bbout{k}{\abs{x} Q} P_1} \Par \binp{k}{X} P_2$. We have that
\[
P \red P_1 \Par P_2 \subst{\abs{x}Q}{X}
\]
In the target language, this reduction is mimicked as follows:
\begin{eqnarray*}
\pmap{P}{2} & = & \newsp{a}{\bout{k}{a} (\pmap{P_1}{3} \Par \repl{} \binp{a}{y} \binp{y}{x} \pmap{Q}{3})\,} 
                  \Par \binp{k}{x} \pmap{P_2}{3} \\
            & \red & \newsp{a}{\pmap{P_1}{3} \Par \repl{} \binp{a}{y} \binp{y}{x} \pmap{Q}{3} 
                  \Par  \pmap{P_2}{3}\subst{a}{x}}
\end{eqnarray*}
\qed
\end{proof}
\end{comment}

\begin{proposition}[Full Abstraction, From \HOp to \sessp]\myrm
	\label{prop:fulla_HOp_to_p}
	Let $P_1, Q_1$ be \HOp processes.
	$\horel{\Gamma}{\Delta_1}{P_1}{\hwb}{\Delta_2}{Q_1}$
	if and only if
	$\horel{\tmap{\Gamma}{2}}{\tmap{\Delta_1}{2}}{\pmap{P_1}{2}}{\fwb}{\tmap{\Delta_2}{2}}{\pmap{Q_1}{2}}$.
\end{proposition}

\begin{proof}
%	The proof for the soundness direction considers
%	closure that can be shown to be a bisimulation
%	following the soundness direction of Operational Correspondence
%	(\propref{prop:op_corr_HOp_to_p}). Whenever needed
%	the proof makes use of the $\tau$-inertness result
%	(\propref{lem:tau_inert}).

%	The proof for the completness direction also considers
%	a closure shown to be a bisimulation
%	up-to deterministic transition (\propref{lem:up_to_deterministic_transition})
%	following the completeness direction of Operational Correspondence
%	(\propref{prop:op_corr_HOp_to_p}).
%
	Proof follows directly from \propref{prop:op_corr_HOp_to_p}. The cases
	of \propref{prop:op_corr_HOp_to_p} are used to create a
	bisimulation closure to prove the the soundness direction and
	a bisimulation up to determinate transition (\lemref{lem:up_to_deterministic_transition})
	to prove the
	completeness direction.
	\qed
\end{proof}

\begin{proposition}[Precise encoding of \HOp into \sessp]\myrm
	\label{prop:prec:HOp_to_p}
	The encoding from $\tyl{L}_{\HOp}$ to $\tyl{L}_{\sessp}$
	is precise.
\end{proposition}

\begin{proof}
	Syntactic requirements are easily derivable from the
	definition of the mappings in \figref{fig:enc:HOp_to_p}.
	Semantic requirements are a consequence of
	\propref{prop:typepres_HOp_to_p}, \propref{prop:op_corr_HOp_to_p}, and \propref{prop:fulla_HOp_to_p}.
	\qed
\end{proof}



