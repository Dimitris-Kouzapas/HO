% !TEX root = ../main.tex
\newcommand{\Client}{\mathsf{Client}}
\newcommand{\rtype}{\mathsf{room}}
\newcommand{\Quote}{\mathsf{quote}}
\newcommand{\accept}{\mathsf{accept}}
\newcommand{\reject}{\mathsf{reject}}
\newcommand{\creditc}{\mathsf{credit}}

\section{Examples: Hotel Booking Scenario}
To illustrate \HOp and its expressive power, 
we consider a usecase scenario that adapts the example given by Mostrous and Yoshida~\cite{tlca07,MostrousY15}.
The scenario involves a $\Client$ process that wants to book
a hotel room. % for her holidays. % in a remote island
%The Client 
$\Client$
narrows the choice down to two hotels, and requires 
 a quote from the two in order to
decide. The round-trip time (RTT) required for
taking quotes from the two hotels in not optimal, % (cf.~\cite{MostrousY15}),
so the client sends mobile processes to both hotels
to automatically negotiate and book a room. 

We now present two \HOp implementations of this scenario.
For convenience, we write $\If e\ \Then (P_1\ \Else \ P_2)$ 
to denote a conditional process that executes $P_1$ or $P_2$ depending on boolean expression $e$ (encodable using labelled choice).
The \emph{first implementation} is  as follows:
%
%\[
	\begin{eqnarray*}%{rcl}
		 P_{xy}  \!\!\! & \defeq &  \!\!\! \bout{x}{\rtype} \binp{x}{\Quote} \bout{y}{\Quote}
		y \triangleright \left\{
				\begin{array}{l}
					\accept: \bsel{x}{\accept} \bout{x}{\creditc} \inact,\\
					\reject: \bsel{x}{\reject} \inact
				\end{array}
				\right\}
		\\ %[3mm]
		 \Client_1 \!\!\!\!\! & \defeq  &  \!\!\! \newsp{h_1, h_2}{\bout{s_1}{\abs{x}{P_{xy} \subst{h_1}{y}}} \bout{s_2}{\abs{x}{P_{xy} \subst{h_2}{y}}} \inact \Par  \\
		& & 
		\!\!\! \binp{\dual{h_1}}{x} \binp{\dual{h_2}}{y}  \If x \leq y\   \Then (\bsel{\dual{h_1}}{\accept} \bsel{\dual{h_2}}{\reject} \inact \ \Else \ \bsel{\dual{h_1}}{\reject} \bsel{\dual{h_2}}{\accept} \inact )
		}
	\end{eqnarray*}
%\]
%
Process $\Client_1$ sends two abstractions with body $P_{xy}$, one to each hotel, 
		using sessions $s_1$ and $s_2$.
		That is, $P_{xy}$ is the mobile code:
	while
		name $x$ is meant to be instantiated by the hotel as the negotiating
		endpoint, name $y$ is used to interact with $\Client_1$.	
		Intuitively, process $P_{xy}$ (i)  sends the room requirements to the hotel;
		(ii) receives a quote from the hotel;
		(iii) sends the quote to  $\Client_1$;
		(iv) expects a choice from   $\Client_1$ whether to accept or reject the offer;
		(v) if the choice is accept then it informs the hotel and performs the booking;
		otherwise, if the choice is reject then it informs the hotel and ends the session.
				$\Client_1$ instantiates two copies of  $P_{xy}$ as abstractions
		on session $x$. It uses two
		fresh endpoints $h_1, h_2$ to substitute channel $y$
		in $P_{xy}$. This enables communication with the mobile code(s).
		In fact, 
		$\Client_1$ uses the dual endpoints $\dual{h_1}$ and $\dual{h_2}$
		to receive the negotiation
		result from the two remote instances of $P$ and then inform the two
		processes for the final booking decision.

Notice that	the above implementation does not affect
the time needed for the whole protocol to execute,
since the two remote processes are used
to send/receive data to $\Client_1$.

We present now a \emph{second  implementation}
%of the same scenario, 
in which the two mobile processes are meant 
to interact with each other (rather than with the client) to reach to an agreement:
%
\[
	\begin{array}{rcl}
	    R_x & \defeq & \If\ \Quote_1 \leq \Quote_2 \, \Then  (\bsel{x}{\accept} \bout{x}{\creditc} \inact \  \Else \ \bsel{x}{\reject} \inact) \\
		Q_1 &\defeq&	\bout{x}{\rtype} \binp{x}{\Quote_1} \bout{y}{\Quote_1} \binp{y}{\Quote_2} R_x \\
		Q_2 &\defeq&	\bout{x}{\rtype} \binp{x}{\Quote_1} \binp{y}{\Quote_2} \bout{y}{\Quote_1} R_x \\
%			&&
%				\begin{array}{ll}
%					\If\ \Quote_1 \leq \Quote_2 &\Then  \bsel{x}{\accept} \bout{x}{\creditc} \inact \  \Else \ \bsel{x}{\reject} \inact %\\
%				%	 & \Else \bsel{x}{\reject} \inact
%				\end{array}
%		\\
%		Q_2 &\defeq&	\bout{x}{\rtype} \binp{x}{\Quote_1} \binp{y}{\Quote_2} \bout{y}{\Quote_1}\\
%			&&
%				\begin{array}{ll}
%					\If\ \Quote_1 \leq \Quote_2  & \Then \bsel{x}{\accept} \bout{x}{\creditc} \inact\\
%					 & \Else \bsel{x}{\reject} \inact
%				\end{array}
%		\\
		\Client_2 &\defeq& \newsp{h}{\bout{s_1}{\abs{x}{Q_1 \subst{h}{y}}} \bout{s_2}{\abs{x}{Q_2 \subst{\dual{h}}{y}}} \inact}
	\end{array}
\]
%\end{example}
Processes $Q_1$ and $Q_2$  negotiate a quote from the
		hotel in the same fashion as process $P_{xy}$ in $\Client_1$.
%		Notice that $Q_2$ is defined exactly as $Q_1$ except for the sequence of messages on~$y$:
%		rather than 
%		sending $\Quote_1$ first and receiving $\Quote_2$ later, 
%		process $Q_2$ receives $\Quote_2$ first and sends $\Quote_1$ later.
		The key difference with respect to $P_{xy}$ is that $y$ is used for
		interaction between process $Q_1$ and $Q_2$. Both processes send
		their quotes to each other and then internally follow the same
		logic to reach to a decision.
		Process  $\Client_2$ then uses sessions $s_1$ and $s_2$ to send the two
		instances of $Q_1$ and $Q_2$ to the two hotels, using them 
	 as abstractions
		on name $x$. It further substitutes
		the two endpoints of a fresh channel $h$ to channels $y$ respectively,
		in order for the two instances to communicate with each other.



%\begin{itemize}
%	\item	Processes $P_1$ and $P_2$ are responsible for negotiating a quote from the
%		hotel in the same fashion as process $P$ in the previous implementation.
%
%	\item	The difference with process $P$ is that the channel $y$ is used for
%		interaction between process $P_1$ and $P_2$. Both processes send
%		there quotes to each other and then internally follow the same
%		logic to reach to a decision.
%
%	\item	The role of $\Client_2$ is to instantiate $P_1$ and $P_2$ as abstractions
%		on name $x$. It further substitutes
%		the two endpoints of a fresh channel $h$ to channels $y$ respectively,
%		in order for the two instances to be able to communicate with each other.
%
%	\item	Process $\Client_2$ then uses sessions $s_1$ and $s_2$ to send the two
%		instances of $P_1$ and $P_2$ to the two hotels.
%\end{itemize}

The differences between $\Client_1$ and $\Client_2$ can be  seen in the sequence diagrams of \figref{fig:exam}. 
We will assign session types to these client processes in Example \ref{exam:type}.
Later on, we will show that they are behaviourally equivalent using characteristic bisimilarity;
see \propref{p:examp}.
\begin{figure}[!t]

\newcommand{\Hotel}{\mathsf{Hotel}}
\newcommand{\Code}{\mathsf{Code}}

%\makeatletter
%\newcommand{\gettikzxy}[3]{%
%  \tikz@scan@one@point\pgfutil@firstofone#1\relax
%  \edef#2{\the\pgf@x}%
%  \edef#3{\the\pgf@y}%
%}
%\makeatother
%	\gettikzxy{(Client1.south)}{\ax}{\ay}
%	\draw[dashed]			(Client1.south) -- (\ax, 0);


%\begin{center}

\begin{tikzpicture}
%	\draw[help lines]		(0, 0) grid (13, 10);

	%%%%%%%%%%%%%%%%%%%%% Scenario 1

	%%%% Nodes
	\node	(Client1)	at	(0, 10) {\footnotesize $\Client_1$};
	\node	(Hotel1)	at	(2.5, 10) {\footnotesize $\Hotel_1$};
	\node	(Hotel2)	at	(5, 10) {\footnotesize $\Hotel_2$};

	\node	(Code1)		at	(1.25, 8.8) {\scriptsize $\Code_1$};
	\node	(Code2)		at	(3.75, 8.8) {\scriptsize $\Code_2$};

	%%%% Lines for Nodes
%	\draw[dashed]		(Client1.south west) -- (Client1.south east);
	\draw
		let
			\p1 = (Client1.south),
			\p2 = (Hotel1.south),
			\p3 = (Hotel2.south),
			\p4 = (Code1.south),
			\p5 = (Code2.south)
		in
			(\x1, \y1) -- (\x1, 4.8)
			(\x2, \y2) -- (\x2, 4.8)
			(\x3, \y3) -- (\x3, 4.8)
			(\x4, \y4) -- (\x4, 4.8)
			(\x5, \y5) -- (\x5, 4.8);

	%%%% Arrows
	\draw[->]
		let
			\p1 = (Client1),
			\p2 = (Hotel1)
		in
			(\x1, 9.6) to node[above] {\scriptsize $\abs{x}{P_{xy}}$} (\x2, 9.6);

	\draw[->]
		let
			\p1 = (Client1),
			\p2 = (Hotel2)
		in
			(\x1, 9.2) to node[above] {\qquad \qquad \scriptsize $\abs{x}{P_{xy}}$} (\x2, 9.2);

	\draw[->]
		let
			\p1 = (Hotel1)
		in
			(\x1, 9.6) -- (Code1.north);

	\draw[->]
		let
			\p1 = (Hotel2)
		in
			(\x1, 9.2) -- (Code2.north);


	\draw[->]
		let
			\p1 = (Code1),
			\p2 = (Hotel1)
		in
			(\x1, 8.4) to node[above] {\scriptsize $\rtype$} (\x2, 8.4);

	\draw[->]
		let
			\p1 = (Code1),
			\p2 = (Hotel1)
		in
			(\x2, 8) to node[above] {\scriptsize $\Quote$} (\x1, 8);

	\draw[->]
		let
			\p1 = (Code2),
			\p2 = (Hotel2)
		in
			(\x1, 8.4) to node[above] {\scriptsize $\rtype$} (\x2, 8.4);
	\draw[->]
		let
			\p1 = (Code2),
			\p2 = (Hotel2)
		in
			(\x2, 8) to node[above] {\scriptsize $\Quote$} (\x1, 8);

	\draw[->]
		let
			\p1 = (Code1),
			\p2 = (Client1)
		in
			(\x1, 7.6) to node[above] {\scriptsize $\Quote$} (\x2, 7.6);

	\draw[->]
		let
			\p1 = (Code2),
			\p2 = (Client1)
		in
			(\x1, 7.2) to node[above] {\scriptsize $\Quote$} (\x2, 7.2);


	%%%% Choice 
	%% Client1 --> Code1
	\draw[dashed]
		let
			\p1 = (Client1),
			\p2 = (Code1)
		in
			(-0.4, 6.8) to node {$\oplus$} (-0.4, 5.4);
%			(\x1, 4.8) -- (\x2, 4.8)
%			(\x1, 2.8) -- (\x2, 2.8);

	\draw[dashed, ->]
		let
			\p1 = (Client1),
			\p2 = (Code1)
		in
			(-0.4, 6.8) to node[above] {\scriptsize $\accept$} (\x2, 6.8);

%	\draw[dashed]
%		let
%			\p1 = (Code1),
%			\p2 = (Hotel1)
%		in
%			(\x1, 6.4) -- (\x2, 6.4)
%			(\x1, 5.2) -- (\x2, 5.2)
%			(\x1, 4) -- (\x2, 4)
%			(\x1, 3.2) -- (\x2, 3.2);

	\draw[->,dashed]
		let
			\p1 = (Code1),
			\p2 = (Hotel1)
		in
			(\x1, 6.2) to node[above] {\scriptsize $\accept$} (\x2, 6.2);

	\draw[->]
		let
			\p1 = (Code1),
			\p2 = (Hotel1)
		in
			(\x1, 5.8) to node[above] {\scriptsize $\creditc$} (\x2, 5.8);

	\draw[dashed, ->]
		let
			\p1 = (Client1),
			\p2 = (Code1)
		in
			(-0.4, 5.4) to node[above] {\scriptsize $\reject$} (\x2, 5.4);

	\draw[dashed, ->]
		let
			\p1 = (Code1),
			\p2 = (Hotel1)
		in
			(\x1, 4.8) to node[above] {\scriptsize $\reject$} (\x2, 4.8);


	%%%% Choice 
	%% Client1 --> Code2
	\draw[dashed]
		let
			\p1 = (Client1),
			\p2 = (Code2)
		in
			(-0.2, 6.6) to node {$\oplus$} (-0.2, 5.2);
%			(\x1, 4.8) -- (\x2, 4.8)
%			(\x1, 2.8) -- (\x2, 2.8);

	\draw[dashed, ->]
		let
			\p1 = (Client1),
			\p2 = (Code2)
		in
			(-0.2, 6.6) to node[above] {\scriptsize $\accept$} (\x2, 6.6);

%	\draw[dashed]
%		let
%			\p1 = (Code1),
%			\p2 = (Hotel1)
%		in
%			(\x1, 6.4) -- (\x2, 6.4)
%			(\x1, 5.2) -- (\x2, 5.2)
%			(\x1, 4) -- (\x2, 4)
%			(\x1, 3.2) -- (\x2, 3.2);

	\draw[->,dashed]
		let
			\p1 = (Code2),
			\p2 = (Hotel2)
		in
			(\x1, 6.2) to node[above] {\scriptsize $\accept$} (\x2, 6.2);

	\draw[->]
		let
			\p1 = (Code2),
			\p2 = (Hotel2)
		in
			(\x1, 5.8) to node[above] {\scriptsize $\creditc$} (\x2, 5.8);

	\draw[dashed, ->]
		let
			\p1 = (Client1),
			\p2 = (Code2)
		in
			(-0.2, 5.2) to node[above] {\scriptsize $\reject$} (\x2, 5.2);

	\draw[dashed, ->]
		let
			\p1 = (Code2),
			\p2 = (Hotel2)
		in
			(\x1, 4.8) to node[above] {\scriptsize $\reject$} (\x2, 4.8);


	%%%%%%%%%%%%%%%%%%%%% Scenario 2

	%%%% Nodes
	\node	(Client1)	at	(6.5, 10) {\footnotesize $\Client_2$};
	\node	(Hotel1)	at	(9, 10) {\footnotesize $\Hotel_1$};
	\node	(Hotel2)	at	(11.5, 10) {\footnotesize $\Hotel_2$};

	\node	(Code1)		at	(7.75, 8.8) {\scriptsize $\Code_1$};
	\node	(Code2)		at	(10.25, 8.8) {\scriptsize $\Code_2$};
%	\node	(Client1)	at	(7.5, 10) {\footnotesize $\Client_2$};
%	\node	(Hotel1)	at	(10, 10) {\footnotesize $\Hotel_1$};
%	\node	(Hotel2)	at	(12.5, 10) {\footnotesize $\Hotel_2$};
%
%	\node	(Code1)		at	(8.75, 8.8) {\scriptsize $\Code_1$};
%	\node	(Code2)		at	(11.25, 8.8) {\scriptsize $\Code_2$};


	\draw
		let
			\p1 = (Client1.south),
			\p2 = (Hotel1.south),
			\p3 = (Hotel2.south),
			\p4 = (Code1.south),
			\p5 = (Code2.south)
		in
			(\x1, \y1) -- (\x1, 4.8)
			(\x2, \y2) -- (\x2, 4.8)
			(\x3, \y3) -- (\x3, 4.8)
			(\x4, \y4) -- (\x4, 4.8)
			(\x5, \y5) -- (\x5, 4.8);

	%%%% Arrows
	\draw[->]
		let
			\p1 = (Client1),
			\p2 = (Hotel1)
		in
			(\x1, 9.6) to node[above] {\scriptsize $\abs{x}{Q_1}$} (\x2, 9.6);

	\draw[->]
		let
			\p1 = (Client1),
			\p2 = (Hotel2)
		in
			(\x1, 9.2) to node[above] {\qquad \qquad \scriptsize $\abs{x}{Q_2}$} (\x2, 9.2);

	\draw[->]
		let
			\p1 = (Hotel1)
		in
			(\x1, 9.6) -- (Code1.north);

	\draw[->]
		let
			\p1 = (Hotel2)
		in
			(\x1, 9.2) -- (Code2.north);


	\draw[->]
		let
			\p1 = (Code1),
			\p2 = (Hotel1)
		in
			(\x1, 8.4) to node[above] {\scriptsize $\rtype$} (\x2, 8.4);

	\draw[->]
		let
			\p1 = (Code1),
			\p2 = (Hotel1)
		in
			(\x2, 8) to node[above] {\scriptsize $\Quote$} (\x1, 8);

	\draw[->]
		let
			\p1 = (Code2),
			\p2 = (Hotel2)
		in
			(\x1, 8.4) to node[above] {\scriptsize $\rtype$} (\x2, 8.4);
	\draw[->]
		let
			\p1 = (Code2),
			\p2 = (Hotel2)
		in
			(\x2, 8) to node[above] {\scriptsize $\Quote$} (\x1, 8);

	\draw[->]
		let
			\p1 = (Code1),
			\p2 = (Code2)
		in
			(\x1, 7.6) to node[above] {\qquad \qquad \scriptsize $\Quote$} (\x2, 7.6);

	\draw[->]
		let
			\p1 = (Code2),
			\p2 = (Code1)
		in
			(\x1, 7.2) to node[above] {\scriptsize $\Quote$ \qquad \qquad \qquad} (\x2, 7.2);


	%%%% Choice
	% Client1 --> Hotel1
	\draw[dashed]
		let
			\p1 = (Code1),
			\p2 = (Hotel1)
		in
			(7.35, 6.8) to node {$\oplus$} (7.35, 6);
%			(\x1, 5.6) -- (\x2, 5.6)
%			(\x1, 4.8) -- (\x2, 4.8);

	\draw[dashed, ->]
		let
			\p1 = (Code1),
			\p2 = (Hotel1)
		in
			(7.35, 6.8) to node[above] {\scriptsize  $\accept$} (\x2, 6.8);

	\draw[->]
		let
			\p1 = (Code1),
			\p2 = (Hotel1)
		in
			(\x1, 6.4) to node[above] {\scriptsize $\creditc$} (\x2, 6.4);

	\draw[dashed, ->]
		let
			\p1 = (Code1),
			\p2 = (Hotel1)
		in
			(7.35, 6) to node[above] {\scriptsize $\reject$} (\x2, 6);

	% Client2 --> Hotel2
	\draw[dashed]
		let
			\p1 = (Code2),
			\p2 = (Hotel2)
		in
			(9.85, 6.8) to node {$\oplus$} (9.85, 6);
%			(\x1, 5.6) -- (\x2, 5.6)
%			(\x1, 4.8) -- (\x2, 4.8);

	\draw[dashed, ->]
		let
			\p1 = (Code2),
			\p2 = (Hotel2)
		in
			(9.85, 6.8) to node[above] {\scriptsize $\accept$} (\x2, 6.8);

	\draw[->]
		let
			\p1 = (Code2),
			\p2 = (Hotel2)
		in
			(\x1, 6.4) to node[above] {\scriptsize $\creditc$} (\x2, 6.4);

	\draw[dashed, ->]
		let
			\p1 = (Code2),
			\p2 = (Hotel2)
		in
			(9.85, 6) to node[above] {\scriptsize $\reject$} (\x2, 6);

\end{tikzpicture}
%\end{center}

\caption{Sequence diagrams for $\Client_1$ and $\Client_2$ as in Example~\ref{exam:proc}\label{fig:exam}.}
\vspace{-2mm}
\end{figure}



\section{Comparison with Jeffrey and Rathke's Approach}