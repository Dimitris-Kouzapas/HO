% !TEX root = ../main.tex
\newcommand{\Client}{\mathsf{Client}}
\newcommand{\rtype}{\mathsf{room}}
\newcommand{\Quote}{\mathsf{quote}}
\newcommand{\accept}{\mathsf{accept}}
\newcommand{\reject}{\mathsf{reject}}
\newcommand{\creditc}{\mathsf{credit}}

\section{Examples: Hotel Booking Scenario}\label{app:s:exam}

To illustrate \HOp and its expressive power, 
we consider a usecase scenario that adapts the example given by Mostrous and Yoshida~\cite{tlca07,MostrousY15}.
The scenario involves a $\Client$ process that wants to book
a hotel room. % for her holidays. % in a remote island
%The Client 
$\Client$
narrows the choice down to two hotels, and requires 
 a quote from the two in order to
decide. The round-trip time (RTT) required for
taking quotes from the two hotels in not optimal, % (cf.~\cite{MostrousY15}),
so the client sends mobile processes to both hotels
to automatically negotiate and book a room. 

\subsection{The Scenario, as \HOp Processes}\label{appss:exam}


We now present two \HOp implementations of this scenario.
For convenience, we write $\If e\ \Then (P_1\ \Else \ P_2)$ 
to denote a conditional process that executes $P_1$ or $P_2$ depending on boolean expression $e$ (encodable using labelled choice).
The \emph{first implementation} is  as follows:
%
%\[
	\begin{eqnarray*}%{rcl}
		 P_{xy}  \!\!\! & \defeq &  \!\!\! \bout{x}{\rtype} \binp{x}{\Quote} \bout{y}{\Quote}
		y \triangleright \left\{
				\begin{array}{l}
					\accept: \bsel{x}{\accept} \bout{x}{\creditc} \inact,\\
					\reject: \bsel{x}{\reject} \inact
				\end{array}
				\right\}
		\\ %[3mm]
		 \Client_1 \!\!\!\!\! & \defeq  &  \!\!\! \newsp{h_1, h_2}{\bout{s_1}{\abs{x}{P_{xy} \subst{h_1}{y}}} \bout{s_2}{\abs{x}{P_{xy} \subst{h_2}{y}}} \inact \Par \binp{\dual{h_1}}{x} \binp{\dual{h_2}}{y} \\
		& & 
		\quad  \If x \leq y\   \Then (\bsel{\dual{h_1}}{\accept} \bsel{\dual{h_2}}{\reject} \inact \ \Else \ \bsel{\dual{h_1}}{\reject} \bsel{\dual{h_2}}{\accept} \inact )
		}
	\end{eqnarray*}
%\]
%
Process $\Client_1$ sends two abstractions with body $P_{xy}$, one to each hotel, 
		using sessions $s_1$ and $s_2$.
		That is, $P_{xy}$ is the mobile code:
	while
		name $x$ is meant to be instantiated by the hotel as the negotiating
		endpoint, name $y$ is used to interact with $\Client_1$.	
		Intuitively, process $P_{xy}$ (i)  sends the room requirements to the hotel;
		(ii) receives a quote from the hotel;
		(iii) sends the quote to  $\Client_1$;
		(iv) expects a choice from   $\Client_1$ whether to accept or reject the offer;
		(v) if the choice is accept then it informs the hotel and performs the booking;
		otherwise, if the choice is reject then it informs the hotel and ends the session.
				$\Client_1$ instantiates two copies of  $P_{xy}$ as abstractions
		on session $x$. It uses two
		fresh endpoints $h_1, h_2$ to substitute channel $y$
		in $P_{xy}$. This enables communication with the mobile code(s).
		In fact, 
		$\Client_1$ uses the dual endpoints $\dual{h_1}$ and $\dual{h_2}$
		to receive the negotiation
		result from the two remote instances of $P$ and then inform the two
		processes for the final booking decision.

Notice that	the above implementation does not affect
the time needed for the whole protocol to execute,
since the two remote processes are used
to send/receive data to $\Client_1$.

We present now a \emph{second  implementation}
%of the same scenario, 
in which the two mobile processes are meant 
to interact with each other (rather than with the client) to reach to an agreement:
%
\[
	\begin{array}{rcl}
	    R_x & \defeq & \If\ \Quote_1 \leq \Quote_2 \, \Then  (\bsel{x}{\accept} \bout{x}{\creditc} \inact \  \Else \ \bsel{x}{\reject} \inact) \\
		Q_1 &\defeq&	\bout{x}{\rtype} \binp{x}{\Quote_1} \bout{y}{\Quote_1} \binp{y}{\Quote_2} R_x \\
		Q_2 &\defeq&	\bout{x}{\rtype} \binp{x}{\Quote_1} \binp{y}{\Quote_2} \bout{y}{\Quote_1} R_x \\
		\Client_2 &\defeq& \newsp{h}{\bout{s_1}{\abs{x}{Q_1 \subst{h}{y}}} \bout{s_2}{\abs{x}{Q_2 \subst{\dual{h}}{y}}} \inact}
	\end{array}
\]
%\end{example}
Processes $Q_1$ and $Q_2$  negotiate a quote from the
		hotel in the same fashion as process $P_{xy}$ in $\Client_1$.
%		Notice that $Q_2$ is defined exactly as $Q_1$ except for the sequence of messages on~$y$:
%		rather than 
%		sending $\Quote_1$ first and receiving $\Quote_2$ later, 
%		process $Q_2$ receives $\Quote_2$ first and sends $\Quote_1$ later.
		The key difference with respect to $P_{xy}$ is that $y$ is used for
		interaction between process $Q_1$ and $Q_2$. Both processes send
		their quotes to each other and then internally follow the same
		logic to reach to a decision.
		Process  $\Client_2$ then uses sessions $s_1$ and $s_2$ to send the two
		instances of $Q_1$ and $Q_2$ to the two hotels, using them 
	 as abstractions
		on name $x$. It further substitutes
		the two endpoints of a fresh channel $h$ to channels $y$ respectively,
		in order for the two instances to communicate with each other.



The differences between $\Client_1$ and $\Client_2$ can be  seen in the sequence diagrams of \figref{fig:exam}. 
Next, we will assign session types to these client processes. % in Example \ref{exam:type}.
Later on, we will show that they are behaviourally equivalent using characteristic bisimilarity;
see \propref{p:examp}.
\begin{figure}[!t]

\newcommand{\Hotel}{\mathsf{Hotel}}
\newcommand{\Code}{\mathsf{Code}}

%\makeatletter
%\newcommand{\gettikzxy}[3]{%
%  \tikz@scan@one@point\pgfutil@firstofone#1\relax
%  \edef#2{\the\pgf@x}%
%  \edef#3{\the\pgf@y}%
%}
%\makeatother
%	\gettikzxy{(Client1.south)}{\ax}{\ay}
%	\draw[dashed]			(Client1.south) -- (\ax, 0);


%\begin{center}

\begin{tikzpicture}
%	\draw[help lines]		(0, 0) grid (13, 10);

	%%%%%%%%%%%%%%%%%%%%% Scenario 1

	%%%% Nodes
	\node	(Client1)	at	(0, 10) {\footnotesize $\Client_1$};
	\node	(Hotel1)	at	(2.5, 10) {\footnotesize $\Hotel_1$};
	\node	(Hotel2)	at	(5, 10) {\footnotesize $\Hotel_2$};

	\node	(Code1)		at	(1.25, 8.8) {\scriptsize $\Code_1$};
	\node	(Code2)		at	(3.75, 8.8) {\scriptsize $\Code_2$};

	%%%% Lines for Nodes
%	\draw[dashed]		(Client1.south west) -- (Client1.south east);
	\draw
		let
			\p1 = (Client1.south),
			\p2 = (Hotel1.south),
			\p3 = (Hotel2.south),
			\p4 = (Code1.south),
			\p5 = (Code2.south)
		in
			(\x1, \y1) -- (\x1, 4.8)
			(\x2, \y2) -- (\x2, 4.8)
			(\x3, \y3) -- (\x3, 4.8)
			(\x4, \y4) -- (\x4, 4.8)
			(\x5, \y5) -- (\x5, 4.8);

	%%%% Arrows
	\draw[->]
		let
			\p1 = (Client1),
			\p2 = (Hotel1)
		in
			(\x1, 9.6) to node[above] {\scriptsize $\abs{x}{P_{xy}}$} (\x2, 9.6);

	\draw[->]
		let
			\p1 = (Client1),
			\p2 = (Hotel2)
		in
			(\x1, 9.2) to node[above] {\qquad \qquad \scriptsize $\abs{x}{P_{xy}}$} (\x2, 9.2);

	\draw[->]
		let
			\p1 = (Hotel1)
		in
			(\x1, 9.6) -- (Code1.north);

	\draw[->]
		let
			\p1 = (Hotel2)
		in
			(\x1, 9.2) -- (Code2.north);


	\draw[->]
		let
			\p1 = (Code1),
			\p2 = (Hotel1)
		in
			(\x1, 8.4) to node[above] {\scriptsize $\rtype$} (\x2, 8.4);

	\draw[->]
		let
			\p1 = (Code1),
			\p2 = (Hotel1)
		in
			(\x2, 8) to node[above] {\scriptsize $\Quote$} (\x1, 8);

	\draw[->]
		let
			\p1 = (Code2),
			\p2 = (Hotel2)
		in
			(\x1, 8.4) to node[above] {\scriptsize $\rtype$} (\x2, 8.4);
	\draw[->]
		let
			\p1 = (Code2),
			\p2 = (Hotel2)
		in
			(\x2, 8) to node[above] {\scriptsize $\Quote$} (\x1, 8);

	\draw[->]
		let
			\p1 = (Code1),
			\p2 = (Client1)
		in
			(\x1, 7.6) to node[above] {\scriptsize $\Quote$} (\x2, 7.6);

	\draw[->]
		let
			\p1 = (Code2),
			\p2 = (Client1)
		in
			(\x1, 7.2) to node[above] {\scriptsize $\Quote$} (\x2, 7.2);


	%%%% Choice 
	%% Client1 --> Code1
	\draw[dashed]
		let
			\p1 = (Client1),
			\p2 = (Code1)
		in
			(-0.4, 6.8) to node {$\oplus$} (-0.4, 5.4);
%			(\x1, 4.8) -- (\x2, 4.8)
%			(\x1, 2.8) -- (\x2, 2.8);

	\draw[dashed, ->]
		let
			\p1 = (Client1),
			\p2 = (Code1)
		in
			(-0.4, 6.8) to node[above] {\scriptsize $\accept$} (\x2, 6.8);

%	\draw[dashed]
%		let
%			\p1 = (Code1),
%			\p2 = (Hotel1)
%		in
%			(\x1, 6.4) -- (\x2, 6.4)
%			(\x1, 5.2) -- (\x2, 5.2)
%			(\x1, 4) -- (\x2, 4)
%			(\x1, 3.2) -- (\x2, 3.2);

	\draw[->,dashed]
		let
			\p1 = (Code1),
			\p2 = (Hotel1)
		in
			(\x1, 6.2) to node[above] {\scriptsize $\accept$} (\x2, 6.2);

	\draw[->]
		let
			\p1 = (Code1),
			\p2 = (Hotel1)
		in
			(\x1, 5.8) to node[above] {\scriptsize $\creditc$} (\x2, 5.8);

	\draw[dashed, ->]
		let
			\p1 = (Client1),
			\p2 = (Code1)
		in
			(-0.4, 5.4) to node[above] {\scriptsize $\reject$} (\x2, 5.4);

	\draw[dashed, ->]
		let
			\p1 = (Code1),
			\p2 = (Hotel1)
		in
			(\x1, 4.8) to node[above] {\scriptsize $\reject$} (\x2, 4.8);


	%%%% Choice 
	%% Client1 --> Code2
	\draw[dashed]
		let
			\p1 = (Client1),
			\p2 = (Code2)
		in
			(-0.2, 6.6) to node {$\oplus$} (-0.2, 5.2);
%			(\x1, 4.8) -- (\x2, 4.8)
%			(\x1, 2.8) -- (\x2, 2.8);

	\draw[dashed, ->]
		let
			\p1 = (Client1),
			\p2 = (Code2)
		in
			(-0.2, 6.6) to node[above] {\scriptsize $\accept$} (\x2, 6.6);

%	\draw[dashed]
%		let
%			\p1 = (Code1),
%			\p2 = (Hotel1)
%		in
%			(\x1, 6.4) -- (\x2, 6.4)
%			(\x1, 5.2) -- (\x2, 5.2)
%			(\x1, 4) -- (\x2, 4)
%			(\x1, 3.2) -- (\x2, 3.2);

	\draw[->,dashed]
		let
			\p1 = (Code2),
			\p2 = (Hotel2)
		in
			(\x1, 6.2) to node[above] {\scriptsize $\accept$} (\x2, 6.2);

	\draw[->]
		let
			\p1 = (Code2),
			\p2 = (Hotel2)
		in
			(\x1, 5.8) to node[above] {\scriptsize $\creditc$} (\x2, 5.8);

	\draw[dashed, ->]
		let
			\p1 = (Client1),
			\p2 = (Code2)
		in
			(-0.2, 5.2) to node[above] {\scriptsize $\reject$} (\x2, 5.2);

	\draw[dashed, ->]
		let
			\p1 = (Code2),
			\p2 = (Hotel2)
		in
			(\x1, 4.8) to node[above] {\scriptsize $\reject$} (\x2, 4.8);


	%%%%%%%%%%%%%%%%%%%%% Scenario 2

	%%%% Nodes
	\node	(Client1)	at	(6.5, 10) {\footnotesize $\Client_2$};
	\node	(Hotel1)	at	(9, 10) {\footnotesize $\Hotel_1$};
	\node	(Hotel2)	at	(11.5, 10) {\footnotesize $\Hotel_2$};

	\node	(Code1)		at	(7.75, 8.8) {\scriptsize $\Code_1$};
	\node	(Code2)		at	(10.25, 8.8) {\scriptsize $\Code_2$};
%	\node	(Client1)	at	(7.5, 10) {\footnotesize $\Client_2$};
%	\node	(Hotel1)	at	(10, 10) {\footnotesize $\Hotel_1$};
%	\node	(Hotel2)	at	(12.5, 10) {\footnotesize $\Hotel_2$};
%
%	\node	(Code1)		at	(8.75, 8.8) {\scriptsize $\Code_1$};
%	\node	(Code2)		at	(11.25, 8.8) {\scriptsize $\Code_2$};


	\draw
		let
			\p1 = (Client1.south),
			\p2 = (Hotel1.south),
			\p3 = (Hotel2.south),
			\p4 = (Code1.south),
			\p5 = (Code2.south)
		in
			(\x1, \y1) -- (\x1, 4.8)
			(\x2, \y2) -- (\x2, 4.8)
			(\x3, \y3) -- (\x3, 4.8)
			(\x4, \y4) -- (\x4, 4.8)
			(\x5, \y5) -- (\x5, 4.8);

	%%%% Arrows
	\draw[->]
		let
			\p1 = (Client1),
			\p2 = (Hotel1)
		in
			(\x1, 9.6) to node[above] {\scriptsize $\abs{x}{Q_1}$} (\x2, 9.6);

	\draw[->]
		let
			\p1 = (Client1),
			\p2 = (Hotel2)
		in
			(\x1, 9.2) to node[above] {\qquad \qquad \scriptsize $\abs{x}{Q_2}$} (\x2, 9.2);

	\draw[->]
		let
			\p1 = (Hotel1)
		in
			(\x1, 9.6) -- (Code1.north);

	\draw[->]
		let
			\p1 = (Hotel2)
		in
			(\x1, 9.2) -- (Code2.north);


	\draw[->]
		let
			\p1 = (Code1),
			\p2 = (Hotel1)
		in
			(\x1, 8.4) to node[above] {\scriptsize $\rtype$} (\x2, 8.4);

	\draw[->]
		let
			\p1 = (Code1),
			\p2 = (Hotel1)
		in
			(\x2, 8) to node[above] {\scriptsize $\Quote$} (\x1, 8);

	\draw[->]
		let
			\p1 = (Code2),
			\p2 = (Hotel2)
		in
			(\x1, 8.4) to node[above] {\scriptsize $\rtype$} (\x2, 8.4);
	\draw[->]
		let
			\p1 = (Code2),
			\p2 = (Hotel2)
		in
			(\x2, 8) to node[above] {\scriptsize $\Quote$} (\x1, 8);

	\draw[->]
		let
			\p1 = (Code1),
			\p2 = (Code2)
		in
			(\x1, 7.6) to node[above] {\qquad \qquad \scriptsize $\Quote$} (\x2, 7.6);

	\draw[->]
		let
			\p1 = (Code2),
			\p2 = (Code1)
		in
			(\x1, 7.2) to node[above] {\scriptsize $\Quote$ \qquad \qquad \qquad} (\x2, 7.2);


	%%%% Choice
	% Client1 --> Hotel1
	\draw[dashed]
		let
			\p1 = (Code1),
			\p2 = (Hotel1)
		in
			(7.35, 6.8) to node {$\oplus$} (7.35, 6);
%			(\x1, 5.6) -- (\x2, 5.6)
%			(\x1, 4.8) -- (\x2, 4.8);

	\draw[dashed, ->]
		let
			\p1 = (Code1),
			\p2 = (Hotel1)
		in
			(7.35, 6.8) to node[above] {\scriptsize  $\accept$} (\x2, 6.8);

	\draw[->]
		let
			\p1 = (Code1),
			\p2 = (Hotel1)
		in
			(\x1, 6.4) to node[above] {\scriptsize $\creditc$} (\x2, 6.4);

	\draw[dashed, ->]
		let
			\p1 = (Code1),
			\p2 = (Hotel1)
		in
			(7.35, 6) to node[above] {\scriptsize $\reject$} (\x2, 6);

	% Client2 --> Hotel2
	\draw[dashed]
		let
			\p1 = (Code2),
			\p2 = (Hotel2)
		in
			(9.85, 6.8) to node {$\oplus$} (9.85, 6);
%			(\x1, 5.6) -- (\x2, 5.6)
%			(\x1, 4.8) -- (\x2, 4.8);

	\draw[dashed, ->]
		let
			\p1 = (Code2),
			\p2 = (Hotel2)
		in
			(9.85, 6.8) to node[above] {\scriptsize $\accept$} (\x2, 6.8);

	\draw[->]
		let
			\p1 = (Code2),
			\p2 = (Hotel2)
		in
			(\x1, 6.4) to node[above] {\scriptsize $\creditc$} (\x2, 6.4);

	\draw[dashed, ->]
		let
			\p1 = (Code2),
			\p2 = (Hotel2)
		in
			(9.85, 6) to node[above] {\scriptsize $\reject$} (\x2, 6);

\end{tikzpicture}
%\end{center}

\caption{Sequence diagrams for $\Client_1$ and $\Client_2$ as in \S\,\ref{appss:exam}\label{fig:exam}.}
%\vspace{-2mm}
\end{figure}

\paragraph{Assigning Types to Clients.}
Assume $S = \btout{\Quote} \btbra{\accept: \tinact, \reject: \tinact}$ and
$U = \btout{\rtype} \btinp{\Quote} \btsel{\accept: \btout{\creditc} \tinact, \reject: \tinact }$.
We give types to the client processes of~\S\,\ref{appss:exam}:
\begin{eqnarray*}
\es; \es; y: S & \proves &  \abs{x}{P_{xy}} \hastype \lhot{U} \\
\es; \es; s_1: \btout{\lhot{U}} \tinact \cat s_2: \btout{\lhot{U}} \tinact & \proves &  \Client_1 \hastype \Proc \\
\es; \es; y: \btout{\Quote} \btinp{\Quote} \tinact & \proves &  \abs{x}{Q_i} \hastype \lhot{U} \quad (i=1,2)\\
\es; \es; s_1: \btout{\lhot{U}} \tinact \cat s_2: \btout{\lhot{U}} \tinact & \proves &  \Client_2 \hastype \Proc
\end{eqnarray*}

%\paragraph{Typed LTS.}
%	Consider environment %tuple
%	$
%		(\Gamma; \es; s: \btout{\lhot{\btout{S} \tinact}} \tinact \cat s': S)
%	$
%	and typed value
%	\[
%		\Gamma; \es; s': S \cat m: \btinp{\tinact} \tinact \proves V \, \hastype \, 
%\lhot{\btout{S} \tinact} \quad \mbox{with} \quad 
%V= \abs{x} \bout{x}{s'} \binp{m}{z} \inact
%	\]
%We illustrate	rule~$\eltsrule{SSnd}$ in \figref{fig:envLTS}.
%Let 
%$\Delta'_1=\{\overline{m}: \btout{\tinact} \tinact\}$ and 
%$U= \lhot{\btout{S} \tinact}$.
%	Then we can derive:
%	\[
%		(\Gamma; \es; s: \btout{\lhot{\btout{S} \tinact}} \tinact \cat s': S) \by{\news{m} \bactout{s}{V}} (\Gamma; \es; s: \tinact)
%	\]

\paragraph{Equivalence of the Client Implementations.}
Now we prove 
that  processes 
$\Client_1$ and $\Client_2$ 
in \S\,\ref{appss:exam}
are behaviourally equivalent.

\begin{proposition}\label{p:examp}
	Let
	$S = \btout{\rtype} \btinp{\Quote} \btsel{\accept: \btout{\creditc} \tinact, \reject: \tinact}$ and 
$\Delta = s_1: \btout{\lhot{S}} \tinact \cat s_2: \btout{\lhot{S}} \tinact$. 
Then
	$ \horel
	{\es}{\Delta}{\Client_1}
	{\wbf}
	{\Delta}{\Client_2}$. %and $\Client_1$, $\Client_2$ in Example \ref{exam:proc}. 
%\vspace{-2mm}
\end{proposition}

\begin{proof}
	\noi We show a case where each typed process simulates the other.

	\noi For fresh sessions $s, k$ we get
	$
		\mapchar{\btinp{\lhot{S}} \tinact}{s} = \binp{s}{x} (\mapchar{\tinact}{s} \Par \mapchar{\lhot{S}}{x})
%		= \binp{s}{x} (\inact \Par \appl{x}{\omapchar{S}})
%		= \binp{s}{x} (\inact \Par \appl{x}{k})
		\scong \binp{s}{x} (\appl{x}{k})
	$
	

	\noi To observe $\Client_1$ assume:
%
	\begin{eqnarray*}
		R' &\scong& \If\ x \leq y\ \Then (\bsel{\dual{h_1}}{\accept} \bsel{\dual{h_2}}{\reject} \inact
		\Else \bsel{\dual{h_1}}{\reject} \bsel{\dual{h_2}}{\accept} \inact)\\
		Q &\scong& \bbra{z}{\accept: \bsel{k_2}{\accept} \bout{k_2}{\creditc} \inact, \reject: \bsel{k_2}{\reject} \inact}
%		Q &\scong& z \triangleleft \left\{
%		\begin{array}{l}
%			\accept: \bsel{k_2}{\accept} \bout{k_2}{\creditc} \inact,\\
%			\reject: \bsel{k_2}{\reject} \inact
%		\end{array}
%		\right\}
	\end{eqnarray*}
%
	\noi We can now observe $\Client_1$ as:
\[
	\begin{array}{ll}
		& \es; \es; \Delta \proves \Client_1
		\\[1mm]

		\by{\bactout{s_1}{\abs{x}{P_{xy} \subst{h_1}{y}}}}&
		\es; \es; s_2: \btout{\lhot{S}} \tinact \cat k_1: S \proves \\
		& \newsp{h_1, h_2}{\bout{s_2}{\abs{x}{P_{xy} \subst{h_2}{y}}} \inact
		\Par \binp{\dual{h_1}}{x} \binp{\dual{h_2}}{y} R'\\
		& \Par \ftrigger{t_1}{P_{xy} \subst{h_1}{y}}{\lhot{S}}}
		%\binp{t_1}{x} \newsp{s}{\mapchar{\btinp{\lhot{S}}}{s} \Par \bout{\dual{s}}{\abs{x}{P \subst{h_1}{y}}} \inact }}
		\\[1mm]

		\by{\bactout{s_2}{\abs{x}{P_{xy} \subst{h_2}{y}}}}&
		\es; \es; k_1: S \cat k_2: S \proves \newsp{h_1, h_2}{
		\binp{\dual{h_1}}{x} \binp{\dual{h_2}}{y} R'\\
		& \ftrigger{t_1}{P_{xy} \subst{h_1}{y}}{\lhot{S}} \Par \ftrigger{t_2}{P_{xy} \subst{h_2}{y}}{\lhot{S}}}
%		\Par \binp{t_1}{x} \newsp{s}{\binp{s}{x} \appl{x}{k_1} \Par \bout{\dual{s}}{\abs{x}{P \subst{h_1}{y}}} \inact }\\
%		& \Par \binp{t_2}{x} \newsp{s}{\mapchar{\btinp{\lhot{S}}}{s} \Par \bout{\dual{s}}{\abs{x}{P \subst{h_2}{y}}} \inact }}
		\\[1mm]

		\by{\bactinp{t_1}{b}} \by{\bactinp{t_2}{b}} \by{\dtau}\by{\dtau}&
		\es; \es; k_1: S \cat k_2: S \proves \newsp{h_1, h_2}{
		\binp{\dual{h_1}}{x} \binp{\dual{h_2}}{y} R'\\
		& \Par P_{xy}\subst{h_1}{y} \subst{k_1}{x} \Par P_{xy}\subst{h_1}{y} \subst{k_2}{x}}
		\\[1mm]

		\by{\bactout{k_1}{\rtype}} \by{\bactout{k_2}{\rtype}}\\
		\by{\bactinp{k_1}{\Quote}} \by{\bactinp{k_2}{\Quote}}
		& \es; \es; k_1: S' \cat k_2: S' \proves \newsp{h_1, h_2}{
		\binp{\dual{h_1}}{x} \binp{\dual{h_2}}{y} R'\\
		& \Par \bout{h_1}{\Quote} Q \subst{h_1}{z} \Par \bout{h_2}{\Quote} Q \subst{h_2}{z}}
		\\[1mm]

		\by{\dtau} \by{\dtau} \by{\dtau}&
		\es; \es; k_1: S' \cat k_2: S' \proves \\
		& \newsp{h_1, h_2}{\bsel{\dual{h_1}}{\accept} \bsel{\dual{h_2}}{\reject} \inact
		\Par Q \subst{h_1}{z} \Par Q \subst{h_2}{z}}
		\\[1mm]

		\by{\dtau} \by{\dtau}&
		\es; \es; k_1: S' \cat k_2: S' \proves
		\bsel{k_1}{\accept} \bout{k_1}{\creditc} \inact 
		\Par \bsel{k_2}{\reject} \inact
		\\[1mm]

		\by{\bactsel{k_1}{\accept}} \by{\bactsel{k_2}{\reject}} \by{\bactsel{k_1}{\creditc}}&
		\es; \es; \es \proves \inact
	\end{array}
\]

\noi	We can observe the same sequence of external transitions for $\Client_2$:

\[
	\begin{array}{ll}
		& \es; \es; \Delta \proves \Client_2
\\[1mm]

		\by{\bactout{s_1}{\abs{x}{Q_1 \subst{h}{y}}}}&
		\es; \es; s_2: \btout{\lhot{S}} \tinact \cat k_1: S \proves \newsp{h}{\bout{s_2}{\abs{x}{Q_2 \subst{\dual{h}}{y}}} \inact\\
		& \Par \ftrigger{t_1}{Q_1 \subst{h}{y}}{\lhot{S}}}
		% \binp{t_1}{x} \newsp{s}{\mapchar{\btinp{\lhot{S}}}{s} \Par \bout{\dual{s}}{\abs{x}{Q_1 \subst{h}{y}}} \inact }}
\\[1mm]

		\by{\bactout{s_2}{\abs{x}{Q_2 \subst{\dual{h}}{y}}}}&
		\es; \es; k_1: S \cat k_2: S \proves \newsp{h}{\\
		& \ftrigger{t_1}{Q_1 \subst{h}{y}}{\lhot{S}} \Par \ftrigger{t_2}{Q_2 \subst{\dual{h}}{y}}{\lhot{S}}}
%		\binp{t_1}{x} \newsp{s}{\binp{s}{x} \appl{x}{k_1} \Par \bout{\dual{s}}{\abs{x}{Q_1 \subst{h}{y}}} \inact }\\
%		& \Par \binp{t_2}{x} \newsp{s}{\mapchar{\btinp{\lhot{S}}}{s} \Par \bout{\dual{s}}{\abs{x}{Q_2 \subst{\dual{h}}{y}}} \inact }}
\\[1mm]

		\by{\bactinp{t_1}{b}} \by{\bactinp{t_2}{b}} \by{\dtau}\by{\dtau}&
		\es; \es; k_1: S \cat k_2: S \proves \newsp{h}{
		P\subst{h}{y} \subst{k_1}{x} \Par P_{xy}\subst{\dual{h}}{y} \subst{k_2}{x}}
\\[1mm]

		\by{\bactout{k_1}{\rtype}} \by{\bactout{k_2}{\rtype}}\\
		\by{\bactinp{k_1}{\Quote}} \by{\bactinp{k_2}{\Quote}}
		& \es; \es; k_1: S' \cat k_2: S' \proves \newsp{h}{
		\bout{h}{\Quote_1} \binp{h}{\Quote_2} R \subst{k_1}{x} \\
		& \Par \binp{\dual{h}}{\Quote_2} \bout{\dual{h}}{\Quote_1} R \subst{k_2}{x}}
\\[1mm]
		\by{\dtau} \by{\dtau}&
		\es; \es; k_1: S' \cat k_2: S' \proves R \subst{k_1}{x} \Par R \subst{k_2}{x}
\\[1mm]
		\by{\dtau} \by{\dtau}&
		\es; \es; k_1: S' \cat k_2: S' \proves
		\bsel{k_1}{\accept} \bout{k_1}{\creditc} \inact 
		\Par \bsel{k_2}{\reject} \inact
\\[1mm]
		\by{\bactsel{k_1}{\accept}} \by{\bactsel{k_2}{\reject}} \by{\bactsel{k_1}{\creditc}}&
		\es; \es; \es \proves \inact
	\end{array}
\]
\end{proof}

\section{Comparison with Jeffrey and Rathke, By Example}
\label{app:jandr}

\noi 
Following up the claim made in \S\,\ref{sec:related}
we contrast our approach with that in~\cite{JeffreyR05} in a concrete, representative example.

	Consider process
	$$\Gamma; \es; \Delta \cat n: \btout{U} \tinact \proves \bout{n}{\abs{x}{\appl{x}(\abs{y}{\bout{y}{m}} \inact)}} \inact \hastype \Proc$$
	with $U = \shot{(\shot{(\shot{(\btout{S} \tinact)})})}$. 
	We describe the transitions required to check the bisimilarity
	of this process with itself. 
	In our framework, first we have a typed transition
	$$
	\Gamma; \es; \Delta \cat n: \btout{U} \tinact \proves \bout{n}{\abs{x}{\appl{x}(\abs{y}{\bout{y}{m}} \inact)}} \inact \by{\bactout{n}{\abs{x}{\appl{x}(\abs{y}{\bout{y}{m}} \inact)}}}\Gamma; \es; \Delta \proves \inact
	$$
	In the framework of~\cite{JeffreyR05} a similar (but untyped) output transition takes place.
    In \figref{f:comparison} we compare the closures obtained by the definition of bisimilarity in our approach (lines (1) to (5)) and in~\cite{JeffreyR05} (lines (6) to (10)).
    In the upper part, we let 
    \begin{eqnarray*}
    V & = & \abs{x}{\appl{x}(\abs{y}{\bout{y}{m}} \inact)} \\
	\mapchar{\btinp{U} \inact}{s} & = & \binp{s}{x} \appl{x}{(\abs{y}{(\appl{y}{a}))}}\qquad \text{for some fresh $a$}
	\end{eqnarray*}
	Then we have one visible input transition (line (1)), followed by four deterministic internal transitions; no replicated processes are needed.
	The approach of~\cite{JeffreyR05} uses 
	the same number of transitions (five), but more visible transitions are required
	(three, in lines (6), (8), and (9)) and at the end, two replicated processes remain.
	This is how linearity information in session types allows us to have more economical closures.
	Note that $\tau_l$ and $\tau_k$ in lines (6) and (8) denote triggered processes on names $l$ and $k$.
	
	
%This simple example shows how both approaches feature the same number of (typed) transitions.
%It is interesting to see how our approach based on refined LTS and characteristic bisimilarity requires less observable actions than that in~\cite{JeffreyR05}.
%Also, as we are able to distinguish between linear and shared names, we require less replicated processes than in~\cite{JeffreyR05}.
	
	
%
%	\begin{eqnarray*}
%		\mapchar{\btinp{U} \inact}{s}\\
%		= && \binp{s}{x} \mapchar{\shot{(\shot{(\shot{(\btout{S} \tinact)})})}}{x}\\
%		= && \binp{s}{x} \appl{x}{\omapchar{\shot{(\shot{(\btout{S} \tinact)})}}}\\
%		= && \binp{s}{x} \appl{x}{(\abs{y}{\mapchar{\shot{(\btout{S} \tinact)}}{y}})}\\
%		= && \binp{s}{x} \appl{x}{(\abs{y}{\appl{y}{\omapchar{\btout{S} \tinact}})}}\\
%		= && \binp{s}{x} \appl{x}{(\abs{y}{\appl{y}{a})}}\\
%	\end{eqnarray*}
%
%	The characteristic process of
%	the type $\shot{(\shot{(\shot{(\btout{S} \tinact)})})}$ is

\begin{figure}[!t]
%\begin{tabular}{c}
%\hline 
	\begin{tabular}{rcl}
%		(1)& & $\Gamma; \es; \Delta \cat n: \btout{U} \tinact \proves \bout{n}{\abs{x}{\appl{x}(\abs{y}{\bout{y}{m}} \inact)}} \inact$ \\
%		&$\by{\bactout{n}{\abs{x}{\appl{x}(\abs{y}{\bout{y}{m}} \inact)}}}$& $\Gamma; \es; \Delta \proves \inact$\\
         &    &   $\ftrigger{t}{V}{U}$  \\
		   &  $=$ & $\Gamma; \es; \Delta \proves \binp{t}{z} \newsp{s}{\mapchar{\btinp{U} \inact}{s} \Par \bout{\dual{s}}{\abs{x}{\appl{x}(\abs{y}{\bout{y}{m}} \inact)}} \inact}$\\
		&  $=$& $\Gamma; \es; \Delta \proves \binp{t}{z} \newsp{s}{\binp{s}{x} \appl{x}{(\abs{y}{(\appl{y}{a}))}} \Par \bout{\dual{s}}{\abs{x}{\appl{x}(\abs{y}{\bout{y}{m}} \inact)}} \inact}$\\
		(1)   &$\by{\bactinp{t}{b}}$& $\Gamma; \es; \Delta \proves \newsp{s}{\binp{s}{x} \appl{x}{(\abs{y}{(\appl{y}{a}))}} \Par \bout{\dual{s}}{\abs{x}{\appl{x}(\abs{y}{\bout{y}{m}} \inact)}} \inact}$\\
		(2)  &$\by{\dtau}$& $\Gamma; \es; \Delta \proves \appl{\abs{x}{\appl{x}(\abs{y}{\bout{y}{m} \inact})}}{(\abs{y}{(\appl{y}{a}))}}$\\
		(3)  &$\by{\dtau}$& $\Gamma; \es; \Delta \proves \appl{(\abs{y}{(\appl{y}{a})})}{(\abs{y}{\bout{y}{m} \inact})} $\\
		(4)   &$\by{\dtau}$& $\Gamma; \es; \Delta \proves \appl{(\abs{y}{\bout{y}{m} \inact})}{a}$\\
		(5)   &$\by{\dtau}$& $\Gamma; \es; \Delta \proves \bout{a}{m} \inact$ \vspace{1mm} 
%	\end{tabular} 
	\\ \hline
%		\begin{tabular}{rrl}
%		(1)& & $\Gamma; \es; \Delta \cat n: \btout{U} \tinact \proves \bout{n}{\abs{x}{\appl{x}(\abs{y}{\bout{y}{m}} \inact)}} \inact$ \\
%		&$\by{\bactout{n}{\abs{x}{\appl{x}(\abs{y}{\bout{y}{m}} \inact)}}}$& $\Gamma; \es; \Delta \proves \inact$\\
		 & & $\Gamma; \es; \Delta \proves \repl{} \binp{t}{x} \appl{x}(\abs{y}{\bout{y}{m}} \inact) $\\
		(6) &$\by{\bactinp{t}{\tau_l}}$& $\Gamma; \es; \Delta \proves \repl{} \binp{t}{x} \appl{x}(\abs{y}{\bout{y}{m}} \inact) \Par \appl{(\abs{x}{\appl{x}(\abs{y}{\bout{y}{m}} \inact)})}{\tau_l}$\\
		(7) &$\by{\dtau}$& $\Gamma; \es; \Delta \proves \repl{} \binp{t}{x} \appl{x}(\abs{y}{\bout{y}{m}} \inact) \Par \appl{\tau_l}{(\abs{y}{\bout{y}{m}} \inact)}$\\
		(8) &$\by{\news{k} \bactout{l}{\tau_k}}$& $\Gamma; \es; \Delta \proves \repl{} \binp{t}{x} \appl{x}(\abs{y}{\bout{y}{m}} \inact) \Par \repl{} \binp{k}{y} \bout{y}{m} \inact $\\
		(9) &$\by{\bactinp{k}{a}}$& $\Gamma; \es; \Delta \proves \repl{} \binp{t}{x} \appl{x}(\abs{y}{\bout{y}{m}} \inact) \Par \repl{} \binp{k}{y} \bout{y}{m} \inact \Par \appl{(\abs{y}{\bout{y}{m} \inact})}{a}$\\
		(10) &$\by{\dtau}$& $\Gamma; \es; \Delta \proves \repl{} \binp{t}{x} \appl{x}(\abs{y}{\bout{y}{m}} \inact) \Par \repl{} \binp{k}{y} \bout{y}{m} \inact \Par \bout{a}{m} \inact$
	\end{tabular}
	 %\\ \hline
	%\end{tabular}
\caption{Comparing our approach (upper part) and Jeffrey and Rathke's~\cite{JeffreyR05} (lower part).\label{f:comparison} }
\end{figure}

