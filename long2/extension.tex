% !TEX root = main.tex
\section{Extending \HOp}
\label{sec:extension}

This section studies two extensions of \HOp 
with higher-order applications/abstractions 
and 
with polyadicity.

\subsection{From \HOpp to \HOp}

So far we have studied a higher-order calculus that allows
only name applications. 
In this section we introduce \HOpp which is an 
extension of the \HOp calculus that includes
higher-order applications.
We further show that \HOpp has a precise encoding
into \HOp and thus by the composability of
encoding (\propref{prop:enc_composability})
\HOpp has a precise encoding to \HO and \sessp.
The latter result implies that \HO is powerful
enough to express full higher-order semantics.

\subsubsection{Extending the Semantics}

\myparagraph{Syntax and Operational Semantics:}
The syntax of \figref{fig:syntax} is changed to
include
$$\appl{W}{V}$$
instead of $\appl{V}{u}$.
The reduction relation in \figref{fig:reduction}
replaces rule $\orule{App}$ 
with rule $$\appl{(\abs{x}{P})}{V} \red P \subst{V}{x}$$
%The structural congruence in \secref{subsec:reduction_semantics}
%is changed to include $\appl{(\abs{x} P)}{V} \scong P \subst{x}{V}$
%instead of $\appl{(\abs{x} P)}{u} \scong P \subst{x}{u}$.

\myparagraph{Types}
The syntax of types in \figref{def:types}
is changed to include $$L \bnfis \shot{U} \bnfbar \lhot{U}$$
instead of $L \bnfis \shot{C} \bnfbar \lhot{C}$.
We expect the obvious extension for type equivalence
(\defref{def:type_equiv})
and for the definition of type environments.

We also expect the straightforward extension of the type 
system to accomodate the new type syntax. The two
most characterestic rules $\trule{Abs}$ and $\trule{App}$ 
for application and abstraction, respectively, now become:
\[
	\begin{array}{c}
		\trule{Abs}~~\tree{
			\Gamma; \Lambda; \Delta_1 \proves P \hastype \Proc
			\quad
			\Gamma; \es; \Delta_2 \proves x \hastype U
		}{
			\Gamma; \Lambda; \Delta_1 \backslash \Delta_2 \proves \abs{x}{P} \hastype \lhot{U}
		}
		\\[6mm]

		\trule{App}~~\tree{
			\begin{array}{c}
				U = \lhot{U'} \lor \shot{U'}
				\quad
				\Gamma; \Lambda; \Delta_1 \proves V \hastype U
				\quad
				\Gamma; \es; \Delta_2 \proves W \hastype U'
			\end{array}
		}{
			\Gamma; \Lambda; \Delta_1 \cat \Delta_2 \proves \appl{V}{W} \hastype \Proc
		} 
	\end{array}
\]

\iftodo\dk{prove subject reduction for the extension}\else\fi

\myparagraph{Behavioural Semantics}
The label definition remain the same.
Rule $\ltsrule{App}$ in the untyped LTS
is replaced with rule
$\appl{\abs{x}{P}}{V} \by{\tau} P \subst{V}{x}$.
The definition of the
characteristic process (\defref{def:characteristic_process})
now includes
\begin{eqnarray*}
\mapchar{\shot{U}}{x} & \defeq & \mapchar{\lhot{U}}{x} \defeq \appl{x}{\omapchar{U}} \\
\omapchar{\shot{U}} & \defeq & \omapchar{\lhot{U}} \defeq \abs{x}{\mapchar{U}{x}}
\end{eqnarray*}
instead of $\mapchar{\shot{C}}{x} \defeq \mapchar{\lhot{C}}{x} \defeq \appl{x}{\omapchar{C}}$
and
$\omapchar{\shot{C}} \defeq \omapchar{\lhot{C}} \defeq \abs{x}{\mapchar{C}{x}}$, respectively.
\noi The rest of the definition for the behavioural remains the same.

\subsubsection{Encoding from \HOpp to \HOp}

We now present an encoding from \HOpp to \HOp.
%
\begin{definition}[Encoding from \HOpp to \HOp]\rm
	\label{def:enc:HOpp_to_HOp}
	Let mapping $\enco{\pmap{\cdot}{3}, \tmap{\cdot}{3}, \mapa{\cdot}^{3}}: \HOpp \to \HOp$
	as defined in \figref{fig:enc:HOpp_to_HOp}.
\end{definition}
%
\begin{figure}[t]
	\[
	\begin{array}{rcl}
		\pmap{\appl{x}{\abs{x} P}}{3} &\defeq& \newsp{s}{\appl{x}{s} \Par \bout{\dual{s}}{\abs{x} \pmap{P}{3}} \inact}
		\\
		\pmap{\appl{(\abs{x} P)}{\abs{x} Q}}{3} &\defeq& \newsp{s}{\appl{(\abs{x} \pmap{P}{3})}{s} \Par \bout{\dual{s}}{\abs{x} \pmap{Q}{3}} \inact}
		\\
		\pmap{\bout{u}{\abs{\underline{x}}{Q}} P}{3} &\defeq& \bout{u}{\abs{z}{\binp{z}{\underline{x}} \pmap{Q}{3}}} \pmap{P}{3}
		\\
		\pmap{\bout{u}{\abs{k}{Q}} P}{3} &\defeq& \bout{u}{\abs{k}{\pmap{Q}{3}}} \pmap{P}{3}
		\\
		\\
		\tmap{\shot{L}}{3} &\defeq& \shot{\btinp{\tmap{L}{3}} \tinact}
		\\
		\tmap{\lhot{L}}{3} &\defeq& \lhot{\btinp{\tmap{L}{3}} \tinact}
		\\
		\tmap{\btout{\shot{L}} S}{3} &\defeq& \btout{\tmap{\shot{L}}{3}} \tmap{S}{3}
		\\
		\tmap{\btout{\lhot{L}} S}{3} &\defeq& \btout{\tmap{\lhot{L}}{3}} \tmap{S}{3}
		\\
		\tmap{\btinp{\shot{L}} S}{3} &\defeq& \btinp{\tmap{\shot{L}}{3}} \tmap{S}{3}
		\\
		\tmap{\btinp{\lhot{L}} S}{3} &\defeq& \btinp{\tmap{\lhot{L}}{3}} \tmap{S}{3}
		\\
		\\
		\mapa{\news{\tilde{m}} \bactout{n}{\abs{k}{P}}}^{3} &\defeq& \news{\tilde{m}} \bactout{n}{\abs{x}{\pmap{P}{3}}}
		\\
		\mapa{\bactinp{n}{\abs{k}{P}}}^{3} &\defeq& \bactinp{n}{\abs{x}{\pmap{P}{3}}}
		\\
		\mapa{\news{\tilde{m}} \bactout{n}{\abs{\underline{x}}{P}}}^{3} &\defeq& \news{\tilde{m}} \bactout{n}{\abs{z}{\binp{z}{x} \pmap{P}{3}}}
		\\
		\mapa{\bactinp{n}{\abs{\underline{x}}{P}}}^{3} &\defeq& \bactinp{n}{\abs{z}{\binp{z}{x} \pmap{P}{3}}}

	\end{array}
	\]
%
	\caption{Encoding of \HOpp into \HOp (cf.~\defref{def:enc:HOpp_to_HOp}).
	We assume that the rest of the encoding is homomorphic on the syntax of
	processes, types and labels, respectively. \label{fig:enc:HOpp_to_HOp}}
\end{figure}

\begin{proposition}[Type Preservation. From \HOpp to \HOp]\rm
	\label{prop:typepres_HOpp_to_HOp}
	Let $P$ be a \HOpp process.
	If $\Gamma; \emptyset; \Delta \proves P \hastype \Proc$ then 
	$\tmap{\Gamma}{3}; \emptyset; \tmap{\Delta}{3} \proves \pmap{P}{3} \hastype \Proc$. 
\end{proposition}

\iftodo
\begin{proof}
	\dk{Put some proof.}
	\qed
\end{proof}
\else 
\begin{proof}
Similar with the case of the encoding of $\pmap{P}{1}$. 
	\qed
\end{proof}
\fi

\begin{proposition}[Operational Correspondence. From \HOpp to \HOp]\rm
	\label{prop:op_corr_HOpp_to_HOp}
	\begin{enumerate}
		\item	Let $\Gamma; \es; \Delta \proves P$.
			$\horel{\Gamma}{\Delta}{P}{\hby{\ell}}{\Delta'}{P'}$ implies
%
			\begin{enumerate}[a)]
				\item	If $\ell \in \set{\news{\tilde{m}} \bactout{n}{\abs{x}{Q}}, \bactinp{n}{\abs{x}{Q}}}$ then
%					$\exists l' $ such that
					$\horel{\tmap{\Gamma}{3}}{\tmap{\Delta}{3}}{\pmap{P}{3}}{\hby{\ell'}}
					{\tmap{\Delta'}{3}}{\pmap{P'}{3}}$ with $\mapa{\ell}^{3} = \ell'$.

%				\item	If $\ell = \bactinp{n}{\abs{x: C}{Q}}$ then
%					$\horel{\tmap{\Gamma}{3}}{\tmap{\Delta}{3}}{\pmap{P}{3}}{\hby{\bactinp{n}{\abs{x: C}{\pmap{Q}{3}}}}}
%					{\tmap{\Delta'}{3}}{\pmap{P'}{3}}$.
%
%				\item	If $\ell = \news{\tilde{m}} \bactout{n}{\abs{x: L}{Q}}$ then
%					$\horel{\tmap{\Gamma}{3}}{\tmap{\Delta}{3}}{\pmap{P}{3}}{\hby{\news{\tilde{m}} \bactout{n}{\abs{z}{\binp{z}{x} \pmap{Q}{3}}}}}
%					{\tmap{\Delta'}{3}}{\pmap{P'}{3}}$.
%
%				\item	If $\ell = \bactinp{n}{\abs{x: L}{Q}}$ then
%					$\horel{\tmap{\Gamma}{3}}{\tmap{\Delta}{3}}{\pmap{P}{3}}{\hby{\bactinp{n}{\abs{z}{\binp{z}{x} \pmap{Q}{3}}}}}
%					{\tmap{\Delta'}{3}}{\pmap{P'}{3}}$.

				\item	If $\ell \notin \set{\news{\tilde{m}} \bactout{n}{\abs{x}{Q}}, \bactinp{n}{\abs{x}{Q}}, \tau}$ then
					$\horel{\tmap{\Gamma}{3}}{\tmap{\Delta}{3}}{\pmap{P}{3}}{\hby{\ell}}
					{\tmap{\Delta'}{3}}{\pmap{P'}{3}}$.

				\item	If $\ell = \btau$ then
					$\horel{\tmap{\Gamma}{3}}{\tmap{\Delta}{3}}{\pmap{P}{3}}{\hby{\btau} \hby{\stau}}
					{\tmap{\Delta'}{3}}{\pmap{P'}{3}}$.

				\item	If $\ell = \tau$ and $\ell \not= \btau$ then %and $\hby{\ell}$ is not a \betatran then
					$\horel{\tmap{\Gamma}{3}}{\tmap{\Delta}{3}}{\pmap{P}{3}}{\hby{\tau}}
					{\tmap{\Delta'}{3}}{\pmap{P'}{3}}$.
			\end{enumerate}

		\item	Let $\Gamma; \es; \Delta \proves P$.
			$\horel{\tmap{\Gamma}{3}}{\tmap{\Delta}{3}}{\pmap{P}{3}}{\hby{\ell}}
			{\tmap{\Delta''}{3}}{Q}$ implies
%
			\begin{enumerate}[a)]
				\item	If $\ell \in \set{\news{\tilde{m}} \bactout{n}{\abs{x}{Q}}, \bactinp{n}{\abs{x}{Q}}, \tau}$
					then
					$\horel{\Gamma}{\Delta}{P}{\hby{\ell'}}{\Delta'}{P'}$
%					and $\horel{\tmap{\Gamma}{3}}{\tmap{\Delta''}{3}}{Q}{\hby{\hat{\ell}}}{\tmap{\Delta'}{3}}{\pmap{P'}{3}}$
					with $\mapa{\ell'}^{3} = \ell$ and $Q \scong \pmap{P'}{3}$.

				\item	If $\ell \notin \set{\news{\tilde{m}} \bactout{n}{\abs{x}{R}}, \bactinp{n}{\abs{x}{R}}, \tau}$
					then
					$\horel{\Gamma}{\Delta}{P}{\hby{\ell}}{\Delta'}{P'}$ and $Q \scong \pmap{P'}{3}$.
%					and $\horel{\tmap{\Gamma}{3}}{\tmap{\Delta''}{3}}{Q}{\hby{\hat{\ell}}}{\tmap{\Delta'}{3}}{\pmap{P'}{3}}$.

				\item	If $\ell = \tau$ then
					either
					$\horel{\Gamma}{\Delta}{\Delta}{\hby{\tau}}{\Delta'}{P'}$ with $Q \scong \pmap{P'}{3}$\\
					or
					$\horel{\Gamma}{\Delta}{\Delta}{\hby{\btau}}{\Delta'}{P'}$ and
					$\horel{\tmap{\Gamma}{3}}{\tmap{\Delta''}{3}}{Q}{\hby{\btau}}
					{\tmap{\Delta''}{3}}{\pmap{P'}{3}}$.
			\end{enumerate}
	\end{enumerate}
\end{proposition}

\begin{proof}
	The proof of Part 1 does a transition induction and
	considers the mapping as defined in \defref{def:enc:HOpp_to_HOp}.
	We give the most interesting cases.

	\begin{itemize}
		\item	Case: $P = \appl{(\abs{x}{Q_1})}{\abs{x}{Q_2}}$.

			$\horel{\Gamma}{\Delta}{\appl{(\abs{x}{Q_1})}{\abs{x}{Q_2}}}{\hby{\btau}}{\Delta}{Q_1 \subst{\abs{x}{Q_2}}{x}}$ implies
\[
			\mhorel{\tmap{\Gamma}{3}}{\tmap{\Delta}{3}}{ \newsp{s}{ \appl{(\abs{z} \binp{z}{x} \pmap{Q}{3})}{s} \Par \bout{\dual{s}}{\abs{x}{Q_2}} \inact}}{\hby{\btau} \hby{\stau}}
			{\tmap{\Delta'}{3}}{}{\pmap{Q_1}{3} \subst{\abs{x}{\pmap{Q_2}{3}}}{x}}
\]

		\item	Case: $P = \bout{n}{\abs{\underline{x}} Q} P$

			$\horel{\Gamma}{\Delta}{\bout{n}{\abs{\underline{x}} Q} P}{\hby{ \bactout{n}{\abs{x}{Q}}}}{\Delta}{P}$ implies

			$\horel{\tmap{\Gamma}{3}}{\tmap{\Delta}{3}}{\bout{n}{\abs{z} \binp{z}{x} \pmap{Q}{3}} \pmap{P}{3}}{\hby{ \bactout{n}{\abs{z}{\binp{z}{x} \pmap{Q}{3}} } }}{\Delta}{\pmap{P}{3}}$
	\end{itemize}

	The proof of Part 2 also does a transition induction and
	considers the mapping as defined in \defref{def:enc:HOpp_to_HOp}.
	We give the most interesting cases.

	\begin{itemize}
		\item	Case: $P = \appl{(\abs{x}{Q_1})}{\abs{x}{Q_2}}$.
%
		\[
			\mhorel{\tmap{\Gamma}{3}}{\tmap{\Delta}{3}}{\newsp{s}{ \appl{(\abs{z}\binp{z}{x} \pmap{Q}{3})}{s}  \Par \bout{\dual{s}}{\abs{x}{Q_2}} \inact}}{\hby{\btau}}
			{\tmap{\Delta'}{3}}{}{\newsp{s}{\binp{s}{x} \pmap{Q}{3}  \Par \bout{\dual{s}}{\abs{x}{Q_2}} \inact}}
		\]
%
			\noi implies
			$\horel{\Gamma}{\Delta}{\appl{(\abs{x}{Q_1})}{\abs{x}{Q_2}}}{\hby{\btau}}{\Delta}{Q_1 \subst{\abs{x}{Q_2}}{x}}$ and
%
		\[
			\mhorel{\tmap{\Gamma}{3}}{\tmap{\Delta}{3}}{\newsp{s}{\binp{s}{x} \pmap{Q}{3}  \Par \bout{\dual{s}}{\abs{x}{Q_2}} \inact}}{\hby{\stau}}
			{\tmap{\Delta'}{3}}{}{\pmap{Q_1}{3} \subst{\abs{x}{\pmap{Q_2}{3}}}{x}}
		\]

\iftodo
		\item	Case: $P = \bout{n}{\abs{\underline{x}} Q} P$

			$\horel{\tmap{\Gamma}{3}}{\tmap{\Delta}{3}}{\bout{n}{\abs{z} \binp{z}{x} \pmap{Q}{3}} \pmap{P}{3}}{\hby{ \bactout{n}{\abs{z}{\binp{z}{x} \pmap{Q}{3}} } }}{\Delta}{\pmap{P}{3}}$ implies

			$\horel{\Gamma}{\Delta}{\bout{n}{\abs{\underline{x}} Q} P}{\hby{ \bactout{n}{\abs{x}{Q}}}}{\Delta}{P}$ implies
			\dk{this case is incomplete}
	\end{itemize}
	\dk{Do more cases}
\else 
\item Other cases are similar. 
\end{itemize}
\fi
	\qed
\end{proof}

\begin{proposition}[Full Abstraction. From \HOpp to \HOp]\rm
	\label{prop:fulla_HOpp_to_HOp}
	Let $P, Q$ \HOpp processes with $\Gamma; \es; \Delta_1 \proves P \hastype \Proc$ and 
	$\Gamma; \es; \Delta_2 \proves Q \hastype \Proc$.
	$\horel{\Gamma}{\Delta_1}{P}{\wb}{\Delta_2}{Q}$ if and only if $\horel{\tmap{\Gamma}{3}}{\tmap{\Delta_1}{3}}{\pmap{P}{3}}{\wb}{\tmap{\Delta_2}{3}}{\pmap{Q}{3}}$
\end{proposition}

\begin{proof}
	\noi {\bf Soundness Direction.}

	\noi We create the closure
%
	\[
		\Re = \set{\horel{\Gamma}{\Delta_1}{P}{\ ,\ }{\Delta_2}{Q} \setbar \horel{\tmap{\Gamma}{3}}{\tmap{\Delta_1}{3}}{\pmap{P}{3}}{\wb}{\tmap{\Delta_2}{3}}{\pmap{Q}{3}}}
	\]
%
	\noi	It is straightforward to show that $\Re$ is a bisimulation if we follow Part 2 of
		\propref{prop:op_corr_HOpp_to_HOp} for subcases a and b.
		In subcase c we make use of \propref{lem:tau_inert}.

	\noi {\bf Completeness Direction.}

	\noi We create the closure
%
	\[
		\Re = \set{\horel{\tmap{\Gamma}{3}}{\tmap{\Delta_1}{3}}{\pmap{P}{3}}{\ ,\ }{\tmap{\Delta_2}{3}}{\pmap{Q}{3}} \setbar \horel{\Gamma}{\Delta_1}{P}{\wb}{\Delta_2}{Q}}
	\]
%
	\noi	We show that $\Re$ is a bisimulation up to deterministic transitions
		by following Part 1 of \propref{prop:op_corr_HOpp_to_HOp}.
		The proof is straightforward for subcases a), b) and d).
		In subcase c) we make use of \lemref{lem:up_to_deterministic_transition}.
	\qed
\end{proof}

\begin{proposition}[Precise encoding of \HOpp into \HOp]\rm
	\label{prop:prec:HOpp_to_HOp}
	The encoding from $\tyl{L}_{\HOpp}$ to $\tyl{L}_{\HOp}$
	is precise.
\end{proposition}

\begin{proof}
	Syntactic requirements are easily derivable from the
	definition of the mappings in \figref{fig:enc:HOpp_to_HOp}.
	Semantic requirements are a consequence of
	\propref{prop:typepres_HOpp_to_HOp}, \propref{prop:op_corr_HOpp_to_HOp}, and \propref{prop:fulla_HOpp_to_HOp}.
	\qed
\end{proof}



\subsection{Polyadic \HOp}
\label{subsec:pol_HOp}


\noi Embedding the polyadic name passing 
into the monadic name passing is well-studied in the literature.    
Using the linear typing, 
the preciseness (full abstraction) can be obtained \cite{Yoshida96}.
Here we summarise how $\pHOp$ can be encoded into $\HOp$. 


%The \HOp is presented in monadic terms.
%Nevertheless, we use polyadic \HOp terms
%as syntax sugar to define the encoding from
%$\HOp$ to $\HO$ in \defref{def:enc:HOp_to_HO},
%since polyadic encoding for session types
%has already been studied (cf.~\cite{}).

%For clarity reasons we present the polyadic variants
%\pHOp of \HOp and show a fully abstract
%encoding from \pHOp to \HOp.

\subsubsection{Extending the Semantics}

\myparagraph{Syntax and Operational Semantics}
The syntax of \figref{fig:syntax} is changed to
include 
$$V \bnfis \tilde{u} \bnfbar \abs{\tilde{x}}{P}$$
and
$$\binp{n}{\tilde{x}} P$$
instead of $\binp{n}{x} P$.
The reduction relation in \figref{fig:reduction}
replaces rule $\orule{App}$ 
with rule
%
\[
	\appl{(\abs{\tilde{x}}{P})}{\tilde{u}} \red P \subst{\tilde{u}}{\tilde{x}} \quad |\tilde{x}| = |\tilde{u}|
\]
%
\noi and rule $\orule{Pass}$ with
%
\[
	\bout{n}{V} P_1 \Par \binp{\dual{n}}{\tilde{x}} P_2 \red P_1 \Par P_2 \subst{V}{\tilde{x}} \quad |V| = |\tilde{x}|
\]
%

\myparagraph{Types}
The syntax of types in \figref{def:types}
is changed to include 
$$L \bnfis \shot{\tilde{C}} \bnfbar \lhot{\tilde{C}}$$
instead of $L \bnfis \shot{C} \bnfbar \lhot{C}$ and
to $U \bnfis \tilde{C} \bnfbar L$.

The type system disallows polyadic shared names as in \cite{tlca07,MostrousY15}.

%In the current polyadic extension we do not
%require polyadic shared names since the
%polyadic encoding for shared names is
%less straightforward than the polyadic
%encoding for session names.

We extend the type system in \figref{fig:typerulesmy}
to include the type rule:
%
\[
	\trule{Pol}~~\tree{
		V = a_i \dots a_n \qquad \Gamma; \Lambda_i; \Delta_i \proves u_i \hastype C_i \qquad U = C_1 \dots C_n
	}{
		\Gamma; \bigcup_{i \in I} \Lambda_i; \bigcup_{i \in I} \Delta_i \proves V \hastype U
	}
\]
%
that allows us to type polyadic values. We expect the
extension of the type rules to accomodate polyadic values.
Note that rules $\trule{Req}$ and $\trule{Acc}$
continue to type monadic instances of shared
name prefixes.

\iftodo
\dk{sketch subject reduction (just revise the related case)}
\else\fi

\myparagraph{Behavioural Semantics}
The definition of the labelled transition system
replaces rule $\ltsrule{App}$ with rule:
%
\[
	\appl{(\abs{\tilde{x}} P)}{\tilde{u}} \by{\tau} P\subst{\tilde{u}}{\tilde{x}}
\]
%
The characteristic process and characteristic value 
definition (\defref{def:characteristic_process})
is extended to include the cases:
\[
	\mapchar{C_1 \dots C_n}{u_1 \cdots u_n} \defeq \mapchar{C_1}{x_1} \Par \dots \Par \mapchar{C_n}{x_n}
\]
\noi and
\[
	\omapchar{U_1 \dots U_n} \defeq \omapchar{U_1}, \dots, \omapchar{U_n}
\]
%
A polyadic type is inhabited in a process where its
parallel components inhabit type terms of the polyadic
type. A polyadic value type is inhabited as a list
of values that inhabit the polyadic values.

The rest of the behavioural semantics remains unchanged.

%Other definitions are straightforwardly extended. 

\subsubsection{Encoding from \pHOp to \HOp}

We slightly modify \defref{def:ep} to capture that a 
label $\ell$ may be mapped into a sequence of labels $\tilde{\ell}$.
Also, \defref{def:ep} stays as the same
assuming that if 
$P \hby{\ell} P'$ and $\mapa{\ell} = \{\ell_1, \ell_2,  \cdots, \ell_m\}$ then
$\map{P} \Hby{\mapa{\ell}} \map{P'}$
should be understood as
$\map{P} \Hby{\ell_1} P_1 \Hby{\ell_2} P_2 \cdots \Hby{\ell_m} P_m =  \map{P'}$,
for some
$P_1, P_2, \ldots, P_m$.

Let $\tyl{L}_{\pHOp}=\calc{\pHOp}{{\cal{T}}_5}{\hby{\ell}}{\wb}{\proves}$
where 
${\cal{T}}_5$ is a set of types of $\HOpp$;  
the typing $\proves$ is defined in 
\figref{fig:typerulesmy} with polyadic types.

\begin{definition}[Encoding from \pHOp to \HOp]\rm
	\label{def:enc:pHOp_to_HOp}
	Encoding $\enco{\pmap{\cdot}{4}, \tmap{\cdot}{4}, \mapa{\cdot}^{4}}: \tyl{L}_{\pHOp} \to \tyl{L}_{\HOp}$
	to be defined as in \figref{fig:enc:pHOp_to_HOp}.
\end{definition}

\begin{figure}[t]
	\[
		\begin{array}{rcl}
			\multicolumn{3}{l}{\textrm{\bf Terms}}
			\\
			\pmap{\bout{n}{u_1, \dots, u_n} P}{4} &\defeq& \bout{n}{u_1} \dots ; \bout{n}{u_n} \pmap{P}{4}
			\\
			\pmap{\binp{n}{x_1, \dots, x_n} P}{4} &\defeq& \binp{n}{x_1} \dots ; \binp{n}{x_n} \pmap{P}{4}
			\\
			\pmap{\bout{n}{\abs{x_1, \dots, x_n}{Q}} P}{4} &\defeq& \bout{n}{\abs{z}{\binp{z}{x_1} \dots; \binp{z}{x_n} \pmap{Q}{4}}} \pmap{P}{4}
			\\
			\pmap{\appl{x}{(u_1, \dots, u_n)}}{4} &\defeq& \newsp{s}{\appl{x}{s} \Par \bout{\dual{s}}{u_1} \dots; \bout{\dual{s}}{u_1} \inact}
			\\
			\pmap{\appl{(\abs{x}{P})}{(u_1, \dots, u_n)}}{4} &\defeq& \newsp{s}{\appl{(\abs{x}{\pmap{P}{4}})}{s} \Par \bout{\dual{s}}{u_1} \dots; \bout{\dual{s}}{u_1} \inact}
			\\
			\\
			\multicolumn{3}{l}{\textrm{\bf Types}}
			\\
			\tmap{\lhot{(C_1, \dots, C_n)}}{4} &\defeq& \lhot{(\btinp{C_1} \dots; \btinp{C_n} \tinact)}
			\\
			\tmap{\shot{(C_1, \dots, C_n)}}{4} &\defeq& \shot{(\btinp{C_1} \dots; \btinp{C_n} \tinact)}
			\\
			\tmap{\btout{L} S}{4} &\defeq& \btout{\tmap{L}{4}} \tmap{S}{4}
			\\
			\tmap{\btinp{L} S}{4} &\defeq& \btinp{\tmap{L}{4}} \tmap{S}{4}
			\\
			\tmap{\btout{C_1, \dots, C_n} S}{4} &\defeq& \btout{C_1} \dots; \btout{C_n} \tmap{S}{4}
			\\
			\tmap{\btinp{C_1, \dots, C_n} S}{4} &\defeq& \btinp{C_1} \dots; \btout{C_n} \tmap{S}{4}
			\\
			\\
			\multicolumn{3}{l}{\textrm{\bf Labels}}
			\\
			\mapa{\news{\tilde{m}'} \bactout{n}{m_1, \dots, m_n}}^{4} &\defeq& \news{\tilde{m_1}'} \bactout{n}{m_1} \dots \news{\tilde{m_n}'}\bactout{n}{m_n}
			\quad \left.
			\begin{array}{rcl}
				\tilde{m_i}' &=& m_i \Leftrightarrow m_i \in \tilde{m}' \wedge\\
				\tilde{m_i}' &=& \es \Leftrightarrow m_i \notin \tilde{m}'
			\end{array}
			\right.
			\\
			\mapa{\bactinp{n}{m_1, \dots, m_n}}^{4} &\defeq& \bactinp{n}{m_1} \dots \bactinp{n}{m_n}
			\\
			\mapa{\news{\tilde{m}} \bactout{n}{\abs{x_1, \dots, x_n}{P}}}{4} &\defeq& \news{\tilde{m}} \bactout{n}{\abs{z}{\binp{z}{x_1} \dots; \binp{z}{x_n} \pmap{P}{4}}}
			\\
			\mapa{\bactinp{n}{\abs{x_1, \dots, x_n}{P}}}{4} &\defeq& \bactinp{n}{\abs{z}{\binp{z}{x_1} \dots; \binp{z}{x_n} \pmap{P}{4}}}
			\\
			\mapa{\btau}^{4} &\defeq& \btau, \stau \dots \stau
			\\
			\mapa{\tau}^{4} &\defeq& \tau \dots \tau
		\end{array}
	\]
%
	\caption{Encoding of \pHOp into \HOp (cf.~\defref{def:enc:pHOp_to_HOp}).
	We assume that the rest of the encoding is homomorphic on the syntax of
	processes, types and labels, respectively. \label{fig:enc:pHOp_to_HOp}}
\end{figure}

\begin{proposition}[Type Preservation. From \pHOp to \HOp]\rm
	\label{prop:typepres_pHOp_to_HOp}
	Let $P$ be a \pHOp process.
	If $\Gamma; \emptyset; \Delta \proves P \hastype \Proc$ then 
	$\tmap{\Gamma}{4}; \emptyset; \tmap{\Delta}{4} \proves \pmap{P}{4} \hastype \Proc$. 
\end{proposition}

\begin{proof}
	By induction on the inference $\Gamma; \emptyset; \Delta \proves P \hastype \Proc$.
	Details in \propref{app:prop:typepres_pHOp_to_HOp} (Page~\pageref{app:prop:typepres_pHOp_to_HOp}).
	\qed
\end{proof}


\begin{proposition}[Operational Correspondendce. From \pHOp to \HOp]\rm
	\label{prop:op_cor:pHOp_to_HOp}
%
	\begin{enumerate}
		\item	Let $\Gamma; \es; \Delta \proves P$.
			$\horel{\Gamma}{\Delta}{P}{\hby{\ell}}{\Delta'}{P'}$ implies
%
			\begin{enumerate}[a)]
				\item	If $\ell = \news{\tilde{m}'} \bactout{n}{\tilde{m}}$ then
					$\horel{\tmap{\Gamma}{4}}{\tmap{\Delta}{4}}{\pmap{P}{4}}{\hby{\ell_1} \dots \hby{\ell_n}}{\tmap{\Delta'}{4}}{\pmap{P}{4}}$
					with $\mapa{\ell}^{4} = \ell_1 \dots \ell_n$.

				\item	If $\ell = \bactinp{n}{\tilde{m}}$ then
					$\horel{\tmap{\Gamma}{4}}{\tmap{\Delta}{4}}{\pmap{P}{4}}{\hby{\ell_1} \dots \hby{\ell_n}}{\tmap{\Delta'}{4}}{\pmap{P}{4}}$
					with $\mapa{\ell}^{4} = \ell_1 \dots \ell_n$.

				\item	If $\ell \in \set{\news{\tilde{m}} \bactout{n}{\abs{\tilde{x}}{R}}, \bactinp{n}{\abs{\tilde{x}}{R}}}$ then
%					$\exists l' $ such that
					$\horel{\tmap{\Gamma}{4}}{\tmap{\Delta}{4}}{\pmap{P}{4}}{\hby{\ell'}}
					{\tmap{\Delta'}{4}}{\pmap{P'}{4}}$ with $\mapa{\ell}^{4} = \ell'$.

				\item	If $\ell \in \set{\bactsel{n}{l}, \bactbra{n}{l}}$ then
					$\horel{\tmap{\Gamma}{4}}{\tmap{\Delta}{4}}{\pmap{P}{4}}{\hby{\ell}}
					{\tmap{\Delta'}{4}}{\pmap{P'}{4}}$.

				\item	If $\ell = \btau$ then either
					$\horel{\tmap{\Gamma}{4}}{\tmap{\Delta}{4}}{\pmap{P}{4}}{\hby{\btau} \hby{\stau} \dots \hby{\stau}}
					{\tmap{\Delta'}{4}}{\pmap{P'}{4}}$ with $\mapa{\ell} = \btau, \stau \dots \stau$.

				\item	If $\ell = \tau$ then %and $\hby{\ell}$ is not a \betatran then
					$\horel{\tmap{\Gamma}{4}}{\tmap{\Delta}{4}}{\pmap{P}{4}}{\hby{\tau} \dots \hby{\tau}}
					{\tmap{\Delta'}{4}}{\pmap{P'}{4}}$ with $\mapa{\ell}^{4} = \tau \dots \tau$.
			\end{enumerate}

		\item	Let $\Gamma; \es; \Delta \proves P$.
			$\horel{\tmap{\Gamma}{4}}{\tmap{\Delta}{4}}{\pmap{P}{4}}{\hby{\ell_1}}
			{\tmap{\Delta_1}{4}}{P_1}$ implies
%
			\begin{enumerate}[a)]
				\item	If $\ell \in \set{\bactinp{n}{m}, \bactout{n}{m}, \news{m} \bactout{n}{m}}$ then
					$\horel{\Gamma}{\Delta}{P}{\hby{\ell}}{\Delta'}{P'}$ and\\
					$\horel{\tmap{\Gamma}{4}}{\tmap{\Delta_1}{4}}{P_1}{\hby{\ell_2} \dots \hby{\ell_n}}
					{\tmap{\Delta'}{4}}{\tmap{P'}{4}}$ with $\mapa{\ell}^{4} = \ell_1 \dots \ell_n$.

				\item	If $\ell \in \set{\news{\tilde{m}} \bactout{n}{\abs{x}{R}}, \bactinp{n}{\abs{x}{R}}}$
					then
					$\horel{\Gamma}{\Delta}{P}{\hby{\ell'}}{\Delta'}{P'}$
					with $\mapa{\ell'}^{4} = \ell$ and $P_1 \scong \pmap{P'}{4}$.

				\item	If $\ell \in \set{\bactsel{n}{l}, \bactbra{n}{l}}$
					then
					$\horel{\Gamma}{\Delta}{P}{\hby{\ell}}{\Delta'}{P'}$ and $P_1 \scong \pmap{P'}{4}$.
%					and $\horel{\tmap{\Gamma}{3}}{\tmap{\Delta''}{3}}{Q}{\hby{\hat{\ell}}}{\tmap{\Delta'}{3}}{\pmap{P'}{3}}$.

				\item	If $\ell = \btau$ then
					$\horel{\Gamma}{\Delta}{P}{\hby{\btau}}{\Delta'}{P'}$ and
					$\horel{\tmap{\Gamma}{4}}{\tmap{\Delta_1}{4}}{P_1}{\hby{\stau} \dots \hby{\stau}}
					{\tmap{\Delta'}{4}}{\tmap{P'}{4}}$ with $\mapa{\ell}^{4} = \btau, \stau \dots \stau$.

				\item	If $\ell = \tau$ then
					$\horel{\Gamma}{\Delta}{P}{\hby{\tau}}{\Delta'}{P'}$ and
					$\horel{\tmap{\Gamma}{4}}{\tmap{\Delta_1}{4}}{P_1}{\hby{\tau} \dots \hby{\tau}}
					{\tmap{\Delta'}{4}}{\tmap{P'}{4}}$ with $\mapa{\ell}^{4} = \tau \dots \tau$.
			\end{enumerate}
	\end{enumerate}
\end{proposition}

\begin{proof}
	The proof of both Parts does a transition induction and
	considers the mapping as defined in \defref{def:enc:HOpp_to_HOp}.
\iftodo{
	\dk{do some cases}
}\else\fi
	\qed
\end{proof}


\begin{proposition}[Full Abstraction. From \HOpp to \HOp]\rm
	\label{prop:fulla:pHOp_to_HOp}
	Let $P, Q$ \pHOp process with $\Gamma; \es; \Delta_1 \proves P \hastype \Proc$ and 
	$\Gamma; \es; \Delta_2 \proves Q \hastype \Proc$.
	$\horel{\Gamma}{\Delta_1}{P}{\wb}{\Delta_2}{Q}$ if and only if $\horel{\tmap{\Gamma}{4}}{\tmap{\Delta_1}{4}}{\pmap{P}{4}}{\wb}{\tmap{\Delta_2}{4}}{\pmap{Q}{4}}$
\end{proposition}

\begin{proof}
	The proof for both direction is a consequence of Operational Correspondence,
	\propref{prop:op_cor:pHOp_to_HOp}.

	\noi {\bf Soundness Direction.}

	\noi We create the closure
%
	\[
		\Re = \set{\horel{\Gamma}{\Delta_1}{P}{\ ,\ }{\Delta_2}{Q} \setbar \horel{\tmap{\Gamma}{4}}{\tmap{\Delta_1}{4}}{\pmap{P}{4}}{\wb}{\tmap{\Delta_2}{4}}{\pmap{Q}{4}}}
	\]
%
	\noi	It is straightforward to show that $\Re$ is a bisimulation if we follow Part 2 of
		\propref{prop:op_cor:pHOp_to_HOp}.
%		for subcases a and b.
%		In subcase c we make use of \propref{lem:tau_inert}.

	\noi {\bf Completeness Direction.}

	\noi We create the closure
%
	\[
		\Re = \set{\horel{\tmap{\Gamma}{4}}{\tmap{\Delta_1}{4}}{\pmap{P}{4}}{\ ,\ }{\tmap{\Delta_2}{4}}{\pmap{Q}{4}} \setbar \horel{\Gamma}{\Delta_1}{P}{\wb}{\Delta_2}{Q}}
	\]
%
%	\dk{Is the proof easy? do the proof}
	\noi	We show that $\Re$ is a bisimulation up to deterministic transitions
		by following Part 1 of \propref{prop:op_cor:pHOp_to_HOp}.
%		The proof is straightforward for subcases a), b) and d).
%		In subcase c) we make use of \lemref{lem:up_to_deterministic_transition}.
	\qed
\end{proof}

\begin{proposition}[Precise encoding of \HOpp into \HOp]\rm
	\label{prop:prec:pHOp_to_HOp}
	The encoding from $\tyl{L}_{\pHOp}$ to $\tyl{L}_{\HOp}$
	is precise.
\end{proposition}

\begin{proof}
	Syntactic requirements are easily derivable from the
	definition of the mappings in \figref{fig:enc:pHOp_to_HOp}.
	Semantic requirements are a consequence of
	\propref{prop:typepres_pHOp_to_HOp}, \propref{prop:op_cor:pHOp_to_HOp}, and \propref{prop:fulla:pHOp_to_HOp}.
	\qed
\end{proof}

