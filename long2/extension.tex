\newcommand{\HOpp}{\ensuremath{\mathsf{HO\pi^{+}}}\xspace}
\section{Extension of \HOp to \HOpp}

In this section we extend the \HOp calculus to include higher order applications.

\subsection{Extending the Semantics}

\paragraph{Syntax and Operational Semantics.}

The syntax of Figure~\ref{fig:syntax} is changed to include $\appl{x}{V}$
instead of $\appl{x}{u}$.
The structural congruence in Section~\ref{subsec:reduction_semantics}
is changed to include $\appl{(\abs{x} P)}{V} \scong P \subst{x}{V}$
instead of $\appl{(\abs{x} P)}{u} \scong P \subst{x}{u}$.

\paragraph{Types.}

The syntax of types changes to include $L \bnfis \shot{U} \bnfbar \lhot{U}$
instead of $L \bnfis \shot{C} \bnfbar \lhot{C}$. We expect the obvious
extension for type equivalence (Definition~\ref{def:type_equiv})
and for the definition of type environments.

We also expect the straightforward extension of the type 
system to accomodate the new type syntax. The two
most characterestic rules $\trule{Abs}$ and $\trule{App}$ 
for application and abstraction, respectively, now become:
\[
	\begin{array}{c}
		\trule{Abs}~~\tree{
			\Gamma; \Lambda; \Delta_1 \proves P \hastype \Proc
			\quad
			\Gamma; \es; \Delta_2 \proves x \hastype U
		}{
			\Gamma; \Lambda; \Delta_1 \backslash \Delta_2 \proves \abs{x}{P} \hastype \lhot{U}
		}
		\\[6mm]

		\trule{App}~~\tree{
			\begin{array}{c}
				U = \lhot{U'} \lor \shot{U'}
				\quad
				\Gamma; \Lambda; \Delta_1 \proves x \hastype U
				\quad
				\Gamma; \es; \Delta_2 \proves u \hastype U'
			\end{array}
		}{
			\Gamma; \Lambda; \Delta_1 \cat \Delta_2 \proves \appl{x}{u} \hastype \Proc
		} 
	\end{array}
\]

\dk{prove subject reduction for the extension}

\paragraph{Behavioural Semantics}

Labels and untyped LTS remain the same. The definition of the
characteristic process~\ref{def:characteristic_process} should be:
%
\begin{definition}[Characteristic Process]\rm
	Let name $u$ and type $U$; then we define a {\em characteristic process}:
	$\map{U}^k$ as:
%
	\[
	\begin{array}{cc}
		\begin{array}{rclcl}
			\map{\btinp{U} S}^{u} &=& \binp{u}{x} (\map{S}^{u} \Par \map{U}^{x})
			\\
			\map{\btout{U} S}^{u} &=& \bout{u}{\omap{U}} \map{S}^{u} %& & n \textrm{ fresh}
			\\
			\map{\btsel{l : S}}^{u} &=& \bsel{u}{l} \map{S}^{u}
			\\
			\map{\btbra{l_i: S_i}_{i \in I}}^{u} &=& \bbra{u}{l_i: \map{S_i}^{u}}_{i \in I}
			\\
			\map{\tvar{t}}^{k} &=& \varp{X}_{\vart{t}}
			\\
			\map{\trec{t}{S}}^{u} &=& \recp{X_{\vart{t}}}{\map{S}^{u}}
			\\
			\map{\tinact}^{u} &=& \inact
			\\
		\end{array}
		&
		\begin{array}{rcrclcl}
			&&\map{\chtype{S}}^{u} &=& \bout{u}{\omap{S}} \inact
			\\
			&&\map{\chtype{L}}^{u} &=& \bout{u}{\omap{L}} \inact
			\\
			\map{\shot{U}}^{x} &=& \map{\lhot{U}}^{x} &=& \appl{x}{\omap{U}}
			\\
			\\
			&&\omap{S} &=& s && s \textrm{ fresh}
			\\
			\omap{\chtype{S}} &=& \omap{\chtype{L}} &=& a && a \textrm{ fresh}
			\\
			\omap{\shot{U}} &=& \omap{\lhot{U}} &=& \abs{x}{\map{U}^x}
		\end{array}
	\end{array}
	\]
\end{definition}

The rest of the behavioural definitions remain intact. 
\dk{prove all cases}

\subsection{Encoding from \HOpp to \HOp}

We now present an encoding from \HOpp to \HOp.

\[
\begin{array}{rcl}
	\pmap{\appl{x}{\abs{x} P}}{3} &\defeq& \newsp{s}{\appl{x}{s} \Par \bout{\dual{s}}{\abs{x} \pmap{P}{3}} \inact}
	\\
	\pmap{\bout{u}{\abs{x: L}{Q}} P}{3} &\defeq& \bout{u}{\abs{z}{\binp{z}{x} \pmap{Q}{3}}} \pmap{P}{3}
	\\
	\\
	\tmap{\shot{L}}{3} &\defeq& \shot{\btinp{L} \tinact}
	\\
	\tmap{\lhot{L}}{3} &\defeq& \lhot{\btinp{L} \tinact}
	\\
	\tmap{\btout{\shot{L}} S}{3} &\defeq& \btout{\tmap{\shot{L}}{3}} \tmap{S}{3}
	\\
	\tmap{\btout{\lhot{L}} S}{3} &\defeq& \btout{\tmap{\lhot{L}}{3}} \tmap{S}{3}
	\\
	\tmap{\btinp{\shot{L}} S}{3} &\defeq& \btinp{\tmap{\shot{L}}{3}} \tmap{S}{3}
	\\
	\tmap{\btinp{\lhot{L}} S}{3} &\defeq& \btinp{\tmap{\lhot{L}}{3}} \tmap{S}{3}
	\\
	\\
	\mapa{\bactout{n}{\abs{x:C}{P}}}^{3} &\defeq& \bactout{n}{\abs{x}{\pmap{P}{3}}}
	\\
	\mapa{\bactinp{n}{\abs{x:C}{P}}}^{3} &\defeq& \bactinp{n}{\abs{x}{\pmap{P}{3}}}
	\\
	\mapa{\bactout{n}{\abs{x:L}{P}}}^{3} &\defeq& \bactout{n}{\abs{z}{\binp{z}{x} \pmap{P}{3}}}
	\\
	\mapa{\bactinp{n}{\abs{x:L}{P}}}^{3} &\defeq& \bactinp{n}{\abs{z}{\binp{z}{x} \pmap{P}{3}}}

\end{array}
\]
\noi We assume that the rest of the encoding is homomorphic on the syntax of
processes, types and labels, respectively.

\dk{Show type preservarnce, operatational correspondence and full abstraction}.
