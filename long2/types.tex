% !TEX root = main.tex
%\newpage
\section{Session Types for $\HOp$}
\label{sec:types}

In this section we define a session typing system for
$\HOp$ and establish its main properties. We use as
a reference the type system for higher-order session processes 
developed by Mostrous and Yoshida~\cite{tlca07,mostrous09sessionbased}.
Our system is simpler than that in~\cite{tlca07}, in order to distil the key
features of higher-order communication in a session-typed setting.

%%%%%%%%%%%%%%%%%%%%%%%%%%%%%%%%%%%%%%%%%%%%%%%%%%
%  SYNTAX FOR TYPES
%%%%%%%%%%%%%%%%%%%%%%%%%%%%%%%%%%%%%%%%%%%%%%%%%%

\subsection{Syntax}

We define the syntax of session types for \HOp.

\begin{definition}[Syntax of Types]\rm
	\label{def:types}
	The syntax of types is defined on the types for sessions $S$,
	and the types for values $U$:
	%
	\[
	\begin{array}{lrcl}
		\textrm{(value)} & U & \bnfis &		C \bnfbar L 
		\\

		\textrm{(Names)} & C & \bnfis &	S \bnfbar \chtype{S} \bnfbar \chtype{L}
		\\

		\textrm{(lambda)} & L & \bnfis &	\shot{C} \bnfbar \lhot{C}
		\\

		\textrm{(session)} & S,T & \bnfis & 	\btout{U} S \bnfbar \btinp{U} S
							\bnfbar \btsel{l_i:S_i}_{i \in I} \bnfbar \btbra{l_i:S_i}_{i \in I}\\
					&&\bnfbar&	\trec{t}{S} \bnfbar \vart{t}  \bnfbar \tinact
	\end{array}
	\]
\end{definition}
%
\noi \myparagraph{Types for Values}
Types for values range over symbol $U$ which includes
first-order types $C$ and higher-order types $L$.
First-order types $C$ are used to type names;
session types $S$ type session names and shared types
$\chtype{S}$ or $\chtype{L}$ type shared names that
carry session values and higher-order values, respectively.
Higher-order types $L$ are used to type abstraction values;
$\shot{C}$ and $\lhot{C}$ denote
shared and linear abstraction types, respectively.

\myparagraph{Session Types}
The syntax of session types $S$ follows the usual
(binary) session types with
recursion~\cite{honda.vasconcelos.kubo:language-primitives,GH05}.
{\em Output type} $\btout{U} S$ is assigned to a name that
first sends a value of type $U$ and then follows
the type described by $S$.
Dually, the {\em input type} $\btinp{U} S$ is assigned to a name
that first receives a value of type $U$ and then continues as $S$. 
Session types for labelled choice and selection, 
%are standard: they are 
written $\btbra{l_i:S_i}_{i \in I}$ and $\btsel{l_i:S_i}_{i \in I}$, respectively,
require a set of types $\set{S_i}_{i \in I}$ that correspond to a set of
labels $\set{i \in I}_{i \in I}$. 
{\em Recursive session types} are defined using the primitive recursor.
We that a {\em type variable} is guarded, i.e.~$\trec{t}{\vart{t}}$ is not allowed.
We let $\mathsf{T}$ to be the set of all well-formed types and
Type $\tinact$ is the termination type.
\ST to be the set of all well-formed session types.

\begin{remark}[Restriction on Types for Values]
	The syntax for value types is restricted
	to dissallow types of the form:
	\begin{enumerate}[$\bullet$]
		\item	$\chtype{\chtype{U}}$: shared names
			cannot carry shared names; and

		\item  $\shot{U}$: abstractions do not
			bind higher-order variables.
	\end{enumerate}
\end{remark}

The difference between the syntax of process
in \HOp with the syntax of processes in~\cite{tlca07}
is also reflected on the two corresponding type syntax;
the type structure  in~\cite{tlca07}, 
supports the arrow types of the form $U \sharedop T$ and 
$U \lollipop T$, where $T$ denotes an arbitrary type of a term 
(i.e.~a value or a process).

%%%%%%%%%%%%%%%%%%%%%%%%%%%%%%%%%%%%%%%%%%%%%%%%%%
%  DUALITY
%%%%%%%%%%%%%%%%%%%%%%%%%%%%%%%%%%%%%%%%%%%%%%%%%%

\subsection{Duality}

Duality is defined following the co-inductive
approach following~\cite{GH05,TGC14}.
We first require the notion of type equivalence.
%
\begin{definition}[Type Equivalence]\rm
\label{def:type_equiv}
	Define function $F(\Re): \mathsf{T} \longrightarrow \mathsf{T}$:
	%
	\[
		\begin{array}{rcl}
			F(\Re) 	&=&	\set{(\tinact, \tinact)} \\
				&\cup&	\set{(\chtype{S}, \chtype{T}) \bnfbar S\ \Re\ T} \cup \set{(\chtype{L_1}, \chtype{L_2}) \bnfbar L_1\ \Re\ L_2}\\
				&\cup&	\set{(\shot{C_1}, \shot{C_2}), (\lhot{C_1}, \lhot{C_2}) \bnfbar C_1\ \Re\ C_2}\\
				&\cup&	\set{(\btout{U_1} S, \btout{U_2} T)\,,\, (\btinp{U_1} S, \btinp{U_1} T) \bnfbar U_1\ \Re\ U_2, S\ \Re\ T}\\
				&\cup&	\set{(\btsel{l_i: S_i}_{i \in I} \,,\, \btsel{l_i: T_i}_{i \in I}) \bnfbar  S_i\ \Re\ T_i}\\
				&\cup&	\set{(\btbra{l_i: S_i}_{i \in I}\,,\, \btbra{l_i: T_i}_{i \in I}) \bnfbar S_i\ \Re\ T_i}\\
				&\cup&	\set{(S\,,\, T) \bnfbar S\subst{\trec{t}{S}}{\vart{t}}\ \Re\ T)}
				\cup	\set{(S\,,\, T) \bnfbar S\ \Re\ T\subst{\trec{t}{T}}{\vart{t}})}
		\end{array}
	\]	
	\noi Standard arguments ensure that $F$ is monotone, thus the greatest fixed point
	of $F$ exists. Let type equivalence be defined as $\iso = \nu X. F(X)$.
\end{definition}
%
\noi Type equivalence is in a essence a co-inductive definition that
equates types up-to recursive unfolding.

The duality relation is defined in terms of type equivalence.
%
\begin{definition}[Duality]\rm
	Define function $F(\Re): \mathsf{ST} \longrightarrow \mathsf{ST}$:
	%
	\[
		\begin{array}{rcl}
			F(\Re) 	&=&	\set{(\tinact, \tinact)}\\
				&\cup&	\set{(\btout{U_1} S, \btinp{U_2} T)\,,\, (\btinp{U} S, \btout{U} T) \bnfbar U_1 \iso U_2, S\ \Re\ T}\\
				&\cup&	\set{(\btsel{l_i: S_i}_{i \in I} \,,\, \btbra{l_i: T_i}_{i \in I}) \bnfbar  S_i\ \Re\ T_i}\\
				&\cup&	\set{(\btbra{l_i: S_i}_{i \in I}\,,\, \btsel{l_i: T_i}_{i \in I}) \bnfbar S_i\ \Re\ T_i}\\
				&\cup&	\set{(S\,,\, T) \bnfbar S\subst{\trec{t}{S}}{\vart{t}}\ \Re\ T)}\\
				&\cup&	\set{(S\,,\, T) \bnfbar S\ \Re\ T\subst{\trec{t}{T}}{\vart{t}})}
		\end{array}
	\]	
	\noi Standard arguments ensure that $F$ is monotone, thus the greatest fixed point
	of $F$ exists. Let duality be defined as $\dualof = \nu X. F(X)$.
\end{definition}
%
Duality is applied co-inductively to session types
up-to recursive unfolding.
Dual session types are prefixed
on dual session type constructors
($!$ is dual to $?$ and $\oplus$ is dual to $\&$)
that carry equivalent types.
%The co-inductive definition of duality relates
%session types up-to recursive unfolding.

%%%%%%%%%%%%%%%%%%%%%%%%%%%%%%%%%%%%%%%%%%%%%%%%%%
%  TYPING SYSTEM
%%%%%%%%%%%%%%%%%%%%%%%%%%%%%%%%%%%%%%%%%%%%%%%%%%

\subsection{Type Environments and Judgements}
We follow~\cite{tlca07},
to defined the typing environment.
%
\begin{definition}[Typing environment]\rm
	We define the {\em shared type environment} $\Gamma$,
	the {\em linear type environment} $\Lambda$, and
	the {\em session type environment} $\Delta$ as:
	\[
	\begin{array}{llcl}
		\text{(Shared)}		& \Gamma  & \bnfis &	\emptyset \bnfbar \Gamma \cat x: \shot{C} \bnfbar \Gamma \cat u: \chtype{S} \bnfbar
								\Gamma \cat u: \chtype{L} \bnfbar \Gamma \cat \varp{X}: \Delta
		\\
		\text{(Linear)}		& \Lambda & \bnfis &	\emptyset \bnfbar \Lambda \cat x: \lhot{C}
		\\
		\text{(Session)}	& \Delta  & \bnfis &	\emptyset \bnfbar \Delta \cat u:S
	\end{array}
	\]
	We imply
	\begin{enumerate}[i.]
		\item	Domains of $\Gamma, \Lambda, \Delta$ are pairwise distinct.
		\item	Weakening, contraction and exchange apply to shared environment $\Gamma$.
		\item	Exchange applies to linear environments $\Lambda$ and $\Delta$. 
	\end{enumerate}
\end{definition}
%
\noi We define typing judgements for values $V$
and processes $P$:
%
\[	\begin{array}{c}
		\Gamma; \Lambda; \Delta \proves V \hastype U \qquad \qquad \qquad \qquad \Gamma; \Lambda; \Delta \proves P \hastype \Proc
	\end{array}
\]
%
\noi The first judgement asserts that under environment $\Gamma; \Lambda; \Delta$
values $V$ have type $U$,
whereas the second judgement asserts that under environment $\Gamma; \Lambda; \Delta$
process $P$ has the typed process type $\Proc$.

%%%%%%%%%%%
%	Type system description
%%%%%%%%%%%

\subsection{Typing Rules}

\begin{figure}[!t]
\[
	\begin{array}{c}
%		\jrule{ }{\Gamma ; \emptyset; \emptyset \vdash \UnitV \hastype \Unit}{Unit} 
%		\qquad\quad  
		\trule{Session}~~\Gamma; \emptyset; \set{k:S} \proves k \hastype S 
		\\[2mm]
		\trule{Shared}~~\Gamma \cat a : \chtype{S}; \emptyset; \emptyset \proves a \hastype \chtype{S}
		\qquad
		\trule{LVar}~~\Gamma; \set{X: \lhot{S}}; \emptyset \proves X \hastype \lhot{S} 
		\\[2mm]
		\trule{Prom}~~\tree{
			\Gamma; \emptyset; \emptyset \proves V \hastype \lhot{S}
		}{
			\Gamma; \emptyset; \emptyset \proves V \hastype \shot{S}
		} 
		\qquad\quad  
		\trule{Derelic}~~\tree{
			\Gamma; \Lambda \cat X{:}\lhot{S}; \Sigma \proves P \hastype \Proc
		}{
			\Gamma \cat X:\shot{S}; \Lambda; \Sigma \proves P \hastype \Proc
		} 
		\\[4mm]
%		\trule{Subt}~~\tree{
%			\Gamma; \Lambda; \Sigma \proves P \hastype T \quad \Sigma \subt \Sigma' \quad T \subt T'
%		}{
%			\Gamma ; \Lambda; \Sigma' \vdash P \hastype T'
%		} 
%		\qquad\quad

		\trule{Abs}~~\tree{
			\Gamma; \Lambda; \Sigma \cat x: S \proves P \hastype \Proc
		}{
			\Gamma; \Lambda; \Sigma \proves \abs{x}{P} \hastype \lhot{S}
		}
		\\[4mm]

		\trule{App}~~\tree{(U = \lhot{S}) \lor (U = \shot{S}) \quad \Gamma; \Lambda_1; \Sigma_1 \proves X \hastype U  \quad \Gamma; \Lambda_2; \Sigma_2 \proves k \hastype S
		}{
			\Gamma; \Lambda_1 \cup \Lambda_2; \Sigma_1 \cup \Sigma_2 \proves \appl{X}{k} \hastype \Proc
		} 
		\\[4mm]

		\trule{Send}~~\tree{
			\Gamma; \Lambda_1; \Sigma_1 \proves P \hastype \Proc  \quad \Gamma; \Lambda_2; \Sigma_2 \vdash V \hastype U  \quad (k:S \in \Sigma_1 \cup \Sigma_2)
		}{
			\Gamma; \Lambda_1 \cup \Lambda_2; (\Sigma_1 \cup \Sigma_2)\backslash\set{k:S} \cat k:\btout{U} S \proves \bout{k}{V} P \hastype \Proc
		}

		\\[4mm]
		\trule{Conn}~~\tree{
			\Gamma; \Lambda; \Sigma \cat x:S \proves P \hastype \Proc  \quad \Gamma; \emptyset; \emptyset \proves a \hastype \chtype{S}
		}{
			\Gamma; \Lambda; \Sigma \proves \binp{a}{x} P \hastype \Proc
		}
		\\[4mm]
%		\trule{ConnDual}~~\tree{
%			\Gamma; \Lambda; \Sigma \cat x: S_1 \proves P \hastype \Proc  \quad \Gamma; \emptyset; \emptyset \proves k \hastype \chtype{S_2} \quad S_1 \dualof S_2
%		}{
%			\Gamma; \Lambda; \Sigma \proves \bout{k}{x} P \hastype \Proc
%		}
%		\\[4mm]

		\trule{ConnDual}~~\tree{
			\Gamma; \Lambda; \Sigma \cat \dual{s}: S_1 \proves P \hastype \Proc  \quad \Gamma; \emptyset; \emptyset \proves a \hastype \chtype{S_2} \quad S_1 \dualof S_2
		}{
			\Gamma; \Lambda; \Sigma  \proves \bout{a}{\dual{s}} P \hastype \Proc
		}

		\\[4mm]

		\trule{NewSh}~~\tree{
			\Gamma\cat a:\chtype{S} ; \Lambda; \Sigma \proves P \hastype \Proc
		}{
			\Gamma; \Lambda; \Sigma \proves \news{a} P \hastype \Proc}
		\qquad\quad
		\trule{NewSes}~~\tree{
			\Gamma; \Lambda; \Sigma \cat s:S_1 \cat \dual{s}: S_2 \proves P \hastype \Proc \quad S_1 \dualof S_2
		}{
			\Gamma; \Lambda; \Sigma \proves \news{s} P \hastype \Proc
		}
		\\[4mm]

		\trule{RecvS}~~\tree{
			\Gamma; \Lambda; \Sigma \cat k: S_1 \cat x: S_2 \proves P \hastype \Proc
		}{
			\Gamma; \Lambda; \Sigma, k: \btinp{S_2} S_1  \vdash \binp{k}{x}P \hastype \Proc
		}
		\quad\quad 
		\trule{RecvL}~~\tree{
			\Gamma; \Lambda \cat X: \lhot{S}; \Sigma \cat k: S_1  \proves P \hastype \Proc
		}{
			\Gamma; \Lambda; \Sigma \cat k:\btinp{\lhot{S}}S_1  \proves \binp{k}{X}P \hastype \Proc
		}
		\\[4mm]

		\trule{RecvSh}~~\tree{
			\Gamma \cat X: \shot{S}; \Lambda; \Sigma \cat k: S_1  \proves P \hastype \Proc
		}{
			\Gamma; \Lambda; \Sigma \cat k:\btinp{\shot{S}}S_1  \proves \binp{k}{X}P \hastype \Proc
		}
		\quad ~~
		\trule{RecvShN}~~\tree{
			\Gamma \cat x: \chtype{S}; \Lambda; \Sigma \cat k: S_1  \proves P \hastype \Proc
		}{
			\Gamma; \Lambda; \Sigma \cat k:\btinp{\chtype{S}}S_1  \proves \binp{k}{x}P \hastype \Proc
		}
		\\[4mm]
		\trule{Par}~~\tree{
			\Gamma; \Lambda_{1}; \Sigma_{1} \proves P_{1} \hastype \Proc \quad \Gamma; \Lambda_{2}; \Sigma_{2} \proves P_{2} \hastype \Proc
		}{
			\Gamma; \Lambda_{1} \cup \Lambda_2; \Sigma_{1} \cup \Sigma_2 \proves P_1 \Par P_2 \hastype \Proc
		}
		\qquad\quad
		\trule{Close}~~\tree{
			\Gamma; \Lambda; \Sigma  \proves P \hastype T \quad k \not\in \dom{\Gamma, \Lambda,\Sigma}
		}{
			\Gamma; \Lambda; \Sigma \cat k: \tinact  \proves P \hastype \Proc
		}
		\\[4mm]
		\trule{Bra}~~\tree{
			 \forall i \in I \quad \Gamma; \Lambda; \Sigma \cat k:S_i \proves P_i \hastype \Proc
		}{
			\Gamma; \Lambda; \Sigma \cat k: \btbra{l_i:S_i}_{i \in I} \proves \bbra{k}{l_i:P_i}_{i \in I}\hastype \Proc
		}
		\qquad\quad 
	 	\trule{Sel}~~\tree{
			\Gamma; \Lambda; \Sigma \cat k: S_j  \proves P \hastype \Proc \quad j \in I
		}{
			\Gamma; \Lambda; \Sigma \cat k:\btsel{l_i:S_i}_{i \in I} \proves \bsel{s}{l_j} P \hastype \Proc
		}
		\\[4mm]

		\trule{Nil}~~\Gamma; \emptyset; \emptyset \proves \inact \hastype \Proc
\qquad \quad
		\trule{Var}~~\tree{
	
		}{
			\Gamma \cat \rvar{X}: \Sigma; \emptyset; \emptyset  \proves \rvar{X} \hastype \Proc
		}
		\qquad\quad 
%	 	\trule{Rec}~~\tree{
%			\Gamma \cat \rvar{X}: \Sigma; \emptyset; \emptyset  \proves P \hastype \Proc
%		}{
%			\Gamma ; \emptyset; \emptyset  \proves \recp{X}{P} \hastype \Proc
%		}
%		\\[4mm]

	 	\trule{Rec}~~\tree{
			\Gamma \cat \rvar{X}: \Sigma; \emptyset; \Sigma  \proves P \hastype \Proc
		}{
			\Gamma ; \emptyset; \Sigma  \proves \recp{X}{P} \hastype \Proc
		}


	\end{array}
\]
\caption{Typing Rules for $\HOp$\label{fig:typerulesmy}}
\end{figure}

The type relation is defined in Figure~\ref{fig:typerulesmy}.
%Types for session names/variables $u$ and
%directly derived from the linear part of the typing
%environment, i.e.~type maps $\Delta$ and $\Lambda$.
Rules $\trule{Session}$ and $\trule{LVar}$ require,
respectively, the minimal:
i) session environment $\Delta$ to type session 
$u$ with type $S$; and,
ii) linear environment $\Lambda$ to type 
higher-order variable $x$ with type $\shot{C}$.
Rule $\trule{Shared}$
assigns the value type $U$
to shared names or shared variables $u$ 
if the map $u:U$ exists in environment
$\Gamma$. Rule $\trule{Shared}$ also requires 
that the linear environment is
be minimal, i.e.~empty.
The type $\shot{C}$ for shared higher-order values $V$
is derived using rule $\trule{Prom}$, where we require
a value with linear type to be typed without a linear
environment present in order to be used as a shared type.
Rule $\trule{EProm}$ allows to freely use a linear
type variable as shared type variable. 
%A value consisting of a tuple of names/variables is typed using the $\trule{Pol}$ rule,
%which requires theto type and combine each value in the tuple.
Abstraction values are typed with rule $\trule{Abs}$.
The key type for an abstraction is the type for
the bound variables of the abstraction, i.e.~for
bound variable with type $C$ the abstraction
has type $\lhot{C}$.
The dual of abstraction typing is application typing
governed by rule $\trule{App}$, where we expect
the type $C$ of an application name $u$ 
to match the type $\lhot{C}$ or $\shot{C}$
of the application variable $x$.

A process prefixed with a session send operator $\bout{u}{V} P$
is typed using rule $\trule{Send}$.
The type $U$ of a send value $V$ should appear as a prefix
on the session type $\btout{U} S$ of $s$.
Rule $\trule{Rcv}$
defines the typing for the 
reception of values $\binp{u}{V} P$.
%and abstractions $\binp{k}{X} P$, respectively.
The type $U$ of a receive value should 
appear as a prefix on the session type $\btinp{U} S$ of $u$.
We use a similar approach with session prefixes
to type interaction between shared channels as defined 
in rules $\trule{Req}$ and $\trule{Acc}$,
where the type of the sent/received object 
($S$ and $L$, respectively) should
match the type of the sent/received subject
($\chtype{S}$ and $\chtype{L}$, respectively).
%In the case of rule $\trule{Req}$ we require
%a duality condition for the communication of session names.
Select and branch prefixes are typed using the rules
$\trule{Sel}$ and $\trule{Bra}$ respectively. Both
rules prefix the session type with the selection
type $\btsel{l_i: S_i}_{i \in I}$ and
$\btbra{l_i:S_i}_{i \in I}$.

The creation of a
shared name $a$ requires to add
its type in environment $\Gamma$ as defined in 
rule \trule{Res}. 
Creation of a session name $s$
creates two endpoints with dual types and adds them to
the session environment 
$\Delta$ as defined in rule \trule{ResS}. 
Rule \trule{Par} concatanates the linear environment of
the parallel components of a parallel operator
to create a type for the entire process.
The disjointness of environments $\Lambda$ and $\Delta$
is implied. Rule \trule{End} allows a form of weakening 
for the session environment $\Delta$, provided that
the name added in $\Delta$ has the inactive
type $\tinact$. The inactive process $\inact$ has no
linear environment. The recursive variable is typed
directly from the shared environment $\Gamma$ as
in rule \trule{RVar}.
The recursive operator requires that the body of
a recursive process matches the type of the recursive
variable as in rule \trule{Rec}.

\begin{comment}
\subsection{Order of Types}

In~\cite{tlca07} the type syntax for values includes the definition
$U_1 \sharedop U_2$ and $U_1 \lollipop U_2$, that
allows us to define types of arbitrary order $k$.
An abstraction of $k$-order types requires to extend the syntax
to include higher-order applications:
\[
	\abs{z}{\binp{z}{x} \appl{x}{\abs{y} Q}}
\]
with with the type of $\abs{y}{Q}$ being of order
$k-1$. The type of of such an abstraction in the current setting would
be $\shot{U}$ (or $\lhot{U}$) with the order of the type being defined
as the number of nested higher-order types~\cite{San96int}.

In the type system we develop for the \HOp we only have
types of the form $\shot{C}$.
If we maintain the definition of counting the order
of the type as the nesting of higher-order types we
can still express $k$-order types, e.g:
\[
	\shot{(\btinp{U} \tinact)}
\]
with $U$ being of order $k-1$.
An $k$-order abstraction in \HOp would be:
\[
	\abs{z}{\binp{z}{x} \binp{x}{y} \appl{y}{n}}
\]
with $y$ being of order $k-1$.

\begin{definition}[Order of Value Type]\rm
	\label{def:order_type}
	Let type $U$ and value $V$ such that $\Gamma; \Lambda; \Delta \proves V \hastype U$.
	The order of $U$ is the number of using rule $\trule{Abs}$
	in the typing derivation $\Gamma; \Lambda; \Delta \proves V \hastype U$.
\end{definition}
\end{comment}

\subsection{Type Soundness}

%We state results for type safety:
Type safety result are instances of more general
statements already proved by
Mostrous and Yoshida~\cite{tlca07} in the asynchronous case.
%
\begin{lemma}[Substitution Lemma - Lemma C.10 in M\&Y]\rm
	\label{lem:subst}
	\begin{enumerate}[1.]
		\item	$\Gamma; \Lambda; \Delta \cat x:S  \proves P \hastype \Proc$ and
			$u \not\in \dom{\Gamma, \Lambda, \Delta}$
			implies
			$\Gamma; \Lambda; \Delta \cat u:S  \proves P\subst{u}{x} \hastype \Proc$.

		\item	$\Gamma \cat x:\chtype{U}; \Lambda; \Delta \proves P \hastype \Proc$ and
			$a \notin \dom{\Gamma, \Lambda, \Delta}$
			implies
			$\Gamma \cat a:\chtype{U}; \Lambda; \Delta \proves P\subst{a}{x} \hastype \Proc$.

		\item	If $\Gamma; \Lambda_1 \cat x:\lhot{C}; \Delta_1  \proves P \hastype \Proc$ 
			and $\Gamma; \Lambda_2; \Delta_2  \proves V \hastype \lhot{C}$ with 
			$\Lambda_1 \cat \Lambda_2$ and $\Delta_1 \cat \Delta_2$ defined,
			then $\Gamma; \Lambda_1 \cat \Lambda_2; \Delta_1 \cat \Delta_2  \proves P\subst{V}{x} \hastype \Proc$.

		\item	$\Gamma \cat x:\shot{C}; \Lambda; \Delta  \proves P \hastype \Proc$ and
			$\Gamma; \emptyset ; \emptyset \proves V \hastype \shot{C}$
			implies
			$\Gamma; \Lambda; \Delta \proves P\subst{V}{x} \hastype \Proc$.
		\end{enumerate}
\end{lemma}
%
\begin{proof}
	By induction on the typing for $P$, with a case analysis on the last used rule. 
	\qed
\end{proof}

We are interested in session environments that whenever they contain dual endpoints
their types are dual:
%
\begin{definition}[Balanced Session Environment]\label{d:wtenv}\rm
	We say that session environment $\Delta$ is {\em balanced} if
	$s: S_1, \dual{s}: S_2 \in \Delta$ implies $S_1 \dualof S_2$.
\end{definition}
%
The type soundness relies on the following auxiliary definition:
%
\begin{definition}[Session Environment Reduction]\rm
	\label{def:ses_red}
	The reduction relation $\red$ on session environments is defined as:
%
\[
	\begin{array}{rcl}
		\Delta \cat s: \btout{U} S_1 \cat \dual{s}: \btinp{U} S_2 &\red& \Delta \cat s: S_1 \cat \dual{s}: S_2
		\\
		\Delta \cat s: \btsel{l_i: S_i}_{i \in I} \cat \dual{s}: \btbra{l_i: S_i'}_{i \in I} &\red& \Delta \cat s: S_k \cat \dual{s}: S_k', \quad k \in I
	\end{array}
\]
%
	We write $\red^\ast$ for the multistep environment reduction.
\end{definition}
%
We now state the main soundness result as an instance
of type soundness from the system in~\cite{tlca07}.
It is worth noticing that in~\cite{tlca07} has a slightly richer
definition of structural congruence.
Also, their statement for subject reduction relies on an
ordering on typing associated to queues and other 
runtime elements. %(such extended typing is denoted as $\Delta$ by M\&Y).
Since we are dealing with synchronous semantics we can omit such an ordering.
The type soundness result implies soundness for the sub-calculi
\HO, \sessp, and $\CAL^{\minussh}$

\begin{theorem}[Type Soundness - Theorem 7.3 in M\&Y]\rm
	\label{thm:sr}
%
	\begin{enumerate}[1.]
		\item	(Subject Congruence)
			$\Gamma; \es; \Delta \proves P \hastype \Proc$
			and
			$P \scong P'$
			implies
			$\Gamma; \es; \Delta \proves P' \hastype \Proc$.

		\item	(Subject Reduction)
			$\Gamma; \es; \Delta \proves P \hastype \Proc$
			with
			balanced $\Delta$
			and
			$P \red P'$
			implies $\Gamma; \es; \Delta'  \proves P' \hastype \Proc$
			and either (i)~$\Delta = \Delta'$ or (ii)~$\Delta \red \Delta'$
			with $\Delta'$ balanced.
	\end{enumerate}
\end{theorem}

\begin{proof}
	See Appendix \ref{app:ts}.
	\qed
\end{proof}
