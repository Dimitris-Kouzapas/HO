% !TEX root = main.tex
\section{Negative Results}

In the encoding from $\HOp$ to $\sessp$ we showed that
an easy and straightforward encoding would be to create
a new shared name for every abstraction we want to pass
in order to use it as a trigger that activate copies of
the abstraction.

At this point a reasonable question could be whether we can
encode shared name behaviour to session name behaviour and at
the same time maintain the type, operational and behavioural semantics.
If such result holds then its impact would be much bigger than
the encoding from $\HOp$ to $\sessp$, since it would
allow us to have session type systems without shared names
and still have the modelling convenience of shared names.

In this section we prove the intution among researches 
that a semantic preserving encoding between a calculus
with shared names and a calculus with only session names
does not exist.

\begin{theorem}\rm
	There is no encoding $\enco{\map{\cdot}, \mapt{\cdot}, \mapa{\cdot}}: \HOp \longrightarrow \HOp^{\minussh}$
	that enjoys operational correspondence and full abstraction.
\end{theorem}

\begin{proof}
	The proof is based on the fact that
	transitions on session channels are
	$\tau$-inert in contrast with shared
	channels which do not enjoy
	$\tau$-inertness.

	Details of the proof in Appendix~\ref{app:neg}
	\qed
\end{proof}

\subsection{Alternative Proof}

We say that an encoding is \emph{barb preserving} (or that enjoys barb preservation)
if 
$P \barb{n}$
then 
$\map{P} \Barb{n}$.

%We have the following proposition:
%
%\begin{proposition}\rm
%	\label{lem:tau_inert}
%	Let balanced \HOp process $\Gamma; \es; \Delta \proves P \hastype \Proc$.
%	Suppose that $P \barb{n}$ and 
%	$\horel{\Gamma}{\Delta}{P}{\shby{\tau}}{\Delta'}{P'}$.
%	Then $P' \barb{n}$.
%\end{proposition}
%
%\begin{proof}
%Follows directly from $\tau$-inertness. \qed
%\end{proof}

\begin{theorem}\rm
	There is no typed encoding $\enco{\map{\cdot}, \mapt{\cdot}, \mapa{\cdot}}: \HOp \longrightarrow \HOp^{\minussh}$
	that enjoys: (i) homomorphism wrt parallel; (ii) barb preservation; (iii) operational completeness.
\end{theorem}

\begin{proof}[Sketch]
Suppose, towards a contradiction, that such a typed encoding indeed exists. 
Consider the $\HOp$ process
$$
P = \breq{a}{s} \inact \Par \bacc{a}{x} \bsel{n}{l_1} \inact \Par \bacc{a}{x} \bsel{m}{l_2} \inact \qquad \text{(with $n \neq m$)}
$$
We then have that
$$P \by{\tau} \bsel{n}{l_1} \inact \Par \bacc{a}{x} \bsel{m}{l_2} \inact = P_1$$
and
$$P \by{\tau} \bsel{m}{l_2} \inact \Par \bacc{a}{x} \bsel{n}{l_1} \inact = P_2$$
Thus, we clearly have:\\
(a) $P_1 \barb{n}$, $P_1 \not \barb{m}$, and \\
(b) $P_2 \barb{m}$, $P_2 \not \barb{n}$.

Consider now the $\HOp^{\minussh}$ process $\map{P}$.
By our assumption of operational completeness we have both:\\
$\map{P} \Hby{} S_1 \wb \map{P_1}$
and \\
$\map{P} \Hby{} S_2 \wb \map{P_2}$.

By our assumption of barb preservation, 
we know that: \\
(c) $\map{P_1} \Barb{n}$, $\map{P_1} \not \Barb{m}$, and \\
(d) $\map{P_2} \Barb{m}$, $\map{P_2} \not \Barb{n}$.



Notice that behavioral equivalence implies that $S_1$ and $\map{P_1}$  have the same barbs
(and analogously for $S_2$, $\map{P_2}$). More precisely, from (c) and (d) we infer: \\
(e) $S_1 \Barb{n}$, $S_1 \not \Barb{m}$, and \\
(f) $S_2 \Barb{m}$, $S_2 \not \Barb{n}$.

Now, by $\tau$-inertness, we have both 
$\map{P} \wb S_1$ and 
$\map{P} \wb S_2$.
Therefore, by transitivity of $\wb$, we have that \\
(g) $S_1 \wb S_2$\\
which entails both \\
(h) $S_1 \Barb{n}$, $S_ 1 \Barb{m}$, and \\
(i) $S_2 \Barb{m}$, $S_2 \Barb{n}$ \\
a contradiction with (e) and and (f).
\qed
\end{proof}

