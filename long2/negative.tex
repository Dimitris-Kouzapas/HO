% !TEX root = main.tex
\section{Negative Results}\label{s:negative}

In the encoding from $\HOp$ to $\sessp$ we showed that
an easy and straightforward encoding would be to create
a new shared name for every abstraction we want to pass
in order to use it as a trigger that activate copies of
the abstraction.

At this point a reasonable question could be whether we can
encode shared name behaviour to session name behaviour and at
the same time maintain the type, operational and behavioural semantics.
If such result holds then its impact would be much bigger than
the encoding from $\HOp$ to $\sessp$, since it would
allow us to have session type systems without shared names
and still have the modelling convenience of shared names.

In this section we prove the intution among researches 
that a semantic preserving encoding between a calculus
with shared names and a calculus with only session names
does not exist.

\begin{theorem}\rm
	There is no encoding $\enco{\map{\cdot}, \mapt{\cdot}, \mapa{\cdot}}: \HOp \longrightarrow \HOp^{\minussh}$
	that enjoys operational correspondence and full abstraction.
\end{theorem}

\begin{proof}
	The proof is based on the fact that
	transitions on session channels are
	$\tau$-inert in contrast with shared
	channels which do not enjoy
	$\tau$-inertness.

	Details of the proof in Appendix~\ref{app:neg}
	\qed
\end{proof}

\subsection{Alternative Proof}



%We have the following proposition:
%
%\begin{proposition}\rm
%	\label{lem:tau_inert}
%	Let balanced \HOp process $\Gamma; \es; \Delta \proves P \hastype \Proc$.
%	Suppose that $P \barb{n}$ and 
%	$\horel{\Gamma}{\Delta}{P}{\shby{\tau}}{\Delta'}{P'}$.
%	Then $P' \barb{n}$.
%\end{proposition}
%
%\begin{proof}
%Follows directly from $\tau$-inertness. \qed
%\end{proof}

\begin{definition}\label{d:bpoc}
We say that 
	the typed encoding
		$\enco{\map{\cdot}, \mapt{\cdot}, \mapa{\cdot}}: \tyl{L}_1 \to \tyl{L}_2$ is 
		\begin{enumerate}[(a)]
\item  \emph{Barb preserving}, %(or that it enjoys barb preservation)
if 
$\Gamma; \Delta \proves P \barb{n}$
then 
$\mapt{\Gamma}; \mapt{\Delta} \proves \map{P} \Barb{n}$.

			\item	\emph{Operationally complete},  
				if  $\stytra{\Gamma}{\ell_1}{\Delta}{P}{\Delta'}{P'}$
				then \\ $\exists \ell_2$ s.t. 
				$\wtytra{\mapt{\Gamma}}{\ell_2}{\mapt{\Delta}}{P}{\mapt{\Delta'}}{Q}$,
				$\ell_2 = \mapa{\ell_1}$,
				and
				$\horel{\mapt{\Gamma}}{\mapt{\Delta'}}{Q}{\wb}{\mapt{\Delta'}}{\mapt{P'}}$.

\end{enumerate}
\end{definition}

Our definition of operational completeness is 
less constrained than that 
in 
Definition \ref{def:ep}.

\begin{definition}[Good Typed Encodings]\label{def:basenc}
We say that 
	the typed encoding 
is	\emph{good} 
	if it is syntax preserving 
	(cf. Definition \ref{def:sep}),
	barb preserving, and  operational complete.
\end{definition}



\begin{theorem}%\rm
	There is no good encoding $\enco{\map{\cdot}, \mapt{\cdot}, \mapa{\cdot}}: \HOp \longrightarrow \HOp^{\minussh}$
	that is good.
%	that enjoys: (i) homomorphism wrt parallel; (ii) barb preservation; (iii) operational completeness.
\end{theorem}

\begin{proof}[Sketch]
Assume, towards a contradiction, that such a typed encoding indeed exists. 
Consider the $\HOp$ process
$$
P = \breq{a}{s} \inact \Par \bacc{a}{x} \bsel{n}{l_1} \inact \Par \bacc{a}{x} \bsel{m}{l_2} \inact \qquad \text{(with $n \neq m$)}
$$
such that 
$\Gamma; \es; \Delta \proves P \hastype \Proc$.
From process $P$ we have: %We then have both
\begin{eqnarray}
& & \stytra{\Gamma}{\tau}{\Delta}{P}{\Delta'}{\bsel{n}{l_1} \inact \Par \bacc{a}{x} \bsel{m}{l_2} \inact = P_1} \label{eq:nn3} \\
& & \stytra{\Gamma}{\tau}{\Delta}{P}{\Delta'}{\bsel{m}{l_2} \inact \Par \bacc{a}{x} \bsel{n}{l_1} \inact = P_2} \label{eq:nn4}
\end{eqnarray}
Thus, by definition of typed barb we  have:
\begin{eqnarray}
\Gamma; \Delta' \proves P_1 \barb{n} & \land & 
\Gamma; \Delta' \proves P_1 \nbarb{m} \label{eq:nn1} \\
\Gamma; \Delta' \proves P_2 \barb{m} & \land & 
\Gamma; \Delta' \proves P_2 \nbarb{n} \label{eq:nn2}
\end{eqnarray}

Consider now the $\HOp^{\minussh}$ process $\map{P}$.
By our assumption of operational completeness 
(Def.~\ref{d:bpoc}(b)), 
from \eqref{eq:nn3} with \eqref{eq:nn4}
we infer that
there exist $\HOp^{\minussh}$ processes $S_1$ and $S_2$ such that:
%we have both:
\begin{eqnarray}
& & \shwtytra{\mapt{\Gamma}}{\tau}{\mapt{\Delta}}{\map{P}}{\mapt{\Delta'}}{S_1 \wb \map{P_1}} \label{eq:n1} \\
& & \shwtytra{\mapt{\Gamma}}{\tau}{\mapt{\Delta}}{\map{P}}{\mapt{\Delta'}}{S_2 \wb \map{P_2}} \label{eq:n2}
%\map{P} & \Hby{} &  S_1 \wb \map{P_1} \\
%s\map{P} & \Hby{} & S_2 \wb \map{P_2}
\end{eqnarray}
By our assumption of barb preservation, 
from \eqref{eq:nn1} with \eqref{eq:nn2}
we infer: 
\begin{eqnarray}
\mapt{\Gamma}; \mapt{\Delta'} \proves \map{P_1} \Barb{n} & \land & 
\mapt{\Gamma}; \mapt{\Delta'} \proves \map{P_1} \nBarb{m} \label{eq:n3} \\
\mapt{\Gamma}; \mapt{\Delta'} \proves \map{P_2} \Barb{m} & \land & 
\mapt{\Gamma}; \mapt{\Delta'} \proves \map{P_2} \nBarb{n} \label{eq:n4}
\end{eqnarray}
By definition of $\wb$, 
by combining 
\eqref{eq:n1} with \eqref{eq:n3}
and
\eqref{eq:n2} with \eqref{eq:n4}, we infer barbs for $S_1$ and $S_2$:
\begin{eqnarray}
\mapt{\Gamma}; \mapt{\Delta'} \proves S_1 \Barb{n} & \land & 
\mapt{\Gamma}; \mapt{\Delta'} \proves S_1 \nBarb{m} \label{eq:n5} \\
\mapt{\Gamma}; \mapt{\Delta'} \proves S_2 \Barb{m} & \land & 
\mapt{\Gamma}; \mapt{\Delta'} \proves S_2 \nBarb{n} \label{eq:n6}
\end{eqnarray}
That is, $S_1$ and $\map{P_1}$ 
(resp. $S_2$ and $\map{P_2}$)
 have the same barbs.
Now, by $\tau$-inertness, we have both 
\begin{eqnarray}
& & \horel{\mapt{\Gamma}}{\mapt{\Delta}}{S_1}{\wb}{\mapt{\Delta'}}{\map{P}} \label{eq:n7} \\
& & \horel{\mapt{\Gamma}}{\mapt{\Delta}}{S_2}{\wb}{\mapt{\Delta'}}{\map{P}} \label{eq:n8}
\end{eqnarray}
Combining \eqref{eq:n7} with \eqref{eq:n8}, by transitivity of $\wb$,
we have 
\begin{equation}
\horel{\mapt{\Gamma}}{\mapt{\Delta'}}{S_1}{\wb}{\mapt{\Delta'}}{S_2} \label{eq:n9}
\end{equation}
In turn, from \eqref{eq:n9}
we infer that 
it must be the case that:
\begin{eqnarray*}
\mapt{\Gamma}; \mapt{\Delta'} \proves \map{P_1} \Barb{n} & \land & 
\mapt{\Gamma}; \mapt{\Delta'} \proves \map{P_1} \Barb{m} \label{eq:n10} \\
\mapt{\Gamma}; \mapt{\Delta'} \proves \map{P_2} \Barb{m} & \land & 
\mapt{\Gamma}; \mapt{\Delta'} \proves \map{P_2} \Barb{n} \label{eq:n11}
\end{eqnarray*}
which clearly contradict \eqref{eq:n3} and \eqref{eq:n4} above.
\qed
\end{proof}

