\newcommand{\hask}[1]{\text{\fontfamily{cmtt}\selectfont #1}}

The \PHOpp calculus has all the ingredients for a straightforward
representation of $\lambda$-terms:
\begin{itemize}
	\item	first and higher order abstractions/applications.
	\item	curried abstractions are represented as polyadic abstractions.
\end{itemize}
The \PHOpp calculus has channels that allow first- and higher-order
passing, along with the standard parallel composition operator.
Furthermore, the calculus makes use of a session type system to
type processes.
In a sense the \PHOpp can be used to have a straightforward representation
of session typed parallel $\lambda$-terms that communicated on channels.

On the other hand the Concurrent Haskell programming language uses the
Haskell programming language, where all its operators are encodable in
$\lambda$-terms and typed in System $F_{\omega}$ and a system
library which is typed using the $IO$ monad 
and allows for parallel Haskell processes that communicate on channels.

We can identify the correponding terms of \PHOpp in Concurrent Haskell:
\begin{itemize}
	\item	$\map{x} = x$
	\item	$\map{\abs{\tilde{x}}{P}} = \hask{absL ($\lambda{}x_1$ -> ... absL ($\lambda{}x_n$  -> $\map{P}$))}$
% \backslash x_1 \rightarrow \dots \backslash x_n \rightarrow \map{P}$
	\item	$\map{\appl{V}{V}} = \hask{appL $\map{V}$ $\map{V}$}$
	\item	$\map{\bout{n}{V} P} = \hask{(send $\map{n}$ $\map{V}$) >>= ($\lambda$() -> $\map{P}$)}$
	\item	$\map{\bout{n}{n'} P} = \hask{(chSend $\map{n}$ \map{n'}) >>= ($\lambda$() -> $\map{P}$)}$
	\item	$\map{\binp{n}{x} P} = \hask{(recv $\map{n}$) >>= ($\lambda$x -> $\map{P}$)}$
	\item	$\map{\binp{n}{n'} P} = \hask{chRecv $\map{n}$ ($\lambda$\map{n'} -> $\map{P}$)}$
	\item	$\map{P\Par Q} = \hask{par $\map{P}$ $\map{Q}$}$
	\item	$\map{\inact} = \hask{return ()}$
	\item	$\map{\bsel{s}{P}} = \hask{(select $\map{s}$ $\map{l}$) >>= ($\lambda$() -> $\map{P}$)}$
	\item	$\map{\bbra{s}{l_i: P}_{i \in I}} = \hask{branch $\map{s}$ ($\lambda{}l_0$ -> $\map{P_0}$) 
($\lambda{}l_1$ -> $\map{P_1}$)}$
%\dots \mathsf{case}\ x\ \mathsf{of}\ \set{l_i}: \map{P_i}$
	\item	$\map{\recp{X}{P}} = \mathsf{fixpoint\ operator}$
\end{itemize}

In the above encoding the Concurrent Haskell operators enjoy the full power
of $\lambda$-terms with the basic operators of the Concurrent Haskell library.
Furthermore, the types on the left are typed on session types and the
types on the right are typed on System $F_\omega$.

By implementing an embedding of the $\PHOpp$ session type system into 
System $F_\omega$ we can have a straightforward embedding of $\PHOpp$
in Concurrent Haskell.


The main contribution of this section comes by:
\begin{itemize}
	\item	reversing the above mapping
	\item	relying on the fact that session types embed (i.e.~are isomorphic) with polymorphic logic, \cite{DBLP:conf/esop/CairesPPT13}
\end{itemize}

Now we can claim that $\PHOpp$ is the first complete
theoretical model that can encode the full capabilities of Concurrent
Haskell.

Going beyond the $\PHOpp$ calculus we have developed a series
of encodings that allow us to encode $\PHOpp$ in a number
of its sub-calculus. So, we can further identify fragments of
the above encodings to have a core represantation of Concurrent
Haskell into session typed higher order calculus.
Our results show that Concurrent Haskell is encodable
in the terms of Concurrent Haskell that represent
$\HOp, \HO$ and, $\sessp$ with the latter two being
core representations.

We argue that it is more convenient to use $\HO$ rather
than $\sessp$ as a core theoretical model for Concurrent Haskell,
so \dk{we have implemented an encoding of Concurent Haskell $\lambda$-terms
and communication primitives in the fragment of Concurrent Haskell that
represents \HO}.

\dk{At this point maybe a correspondence between the type derivation
of the session type system and a type derivation of Concurrent Haskell
session typed processes is in order.}

\subsection{Derivability}

Let $\Gamma \vdash M :: t$ be typing derivations for Haskell programs. We write concrete
Haskell syntax in \hask{teletype}, and meta variables (denoting Haskell programs or types)
in standard font. 

The core of the embedding is the graded monad \hask{Session} which
takes two parameters, a type-level representation of session types
environments, of the form \hask{\{n$_1$ :-> $S_1$, ..., n$_k$ :-> $S_k$\}} for channel names
\hask{n$_1$} to \hask{n$_k$} with corresponding session types \hask{S$_1$} to \hask{S$_k$}. 

Type-level session types are formed by the following inductive grammar of 
type constructors for names (with endpoints) $n$, name symbols $m$, session types $S$ and
labels $l$:
%%
\begin{align*}
n & ::= \hask{Ch} \, m \mid \hask{Op} \, m \\
m & ::= \hask{A} \mid \hask{B} \mid \hask{C} \mid \ldots \\
S & ::= \hask{End} \mid \hask{$t$:!$S$} \mid \hask{(Delg $S$):!$S$} \mid \hask{$t$:?$S$} \mid 
\hask{(Delg $S$):?$S$}\\
 & \quad\; \mid \hask{Sel $l$ $S$ $S$} \mid \hask{$S$ :\& $S$} \\
l & ::= \hask{Left} \mid \hask{Right} \mid \hask{Sup}
\end{align*}
%%
where $t$ ranges over Haskell types, 
\hask{Delg} denotes a delegated channel with its session type.

\newcommand{\interp}[1]{\map{#1}}

We define an interpretation function $\interp{-}$, overloaded on terms, types, and contexts,
that maps $HO\pi$ terms, types, and contexts to Haskell.  For session types this is defined:
%%
\begin{align*}
\interp{\btout{L} S} & = \hask{$\interp{L}$ :! $\interp{S}$} \\
\interp{\btout{S} S} & = \hask{(Delg $\interp{S}$) :! $\interp{S}$} \\
\interp{\btinp{L} S} & = \hask{$\interp{L}$ :? $\interp{S}$} \\
\interp{\btinp{S} S} & = \hask{(Delg $\interp{S}$) :? $\interp{S}$} \\
\interp{\btsel{l_0:S}} & = \hask{Sel Left \interp{S} t} \\
\interp{\btsel{l_1:S}} & = \hask{Sel Right t \interp{S}} \\
\interp{\btsel{l_n:S}} & = \hask{Sel Left $\interp{\btsel{l_{n-1}:S}}$ t} \\
\interp{\btsel{l_i:S_i}_{i \in \textit{fin$(n)$}}} & = \hask{Sel Sup ($\interp{\btsel{(l_i : S_i) \backslash (l_n : S_n)}_{i \in \textit{fin$(n-1)$}}}$) $\interp{S_n}$} \\
\interp{\btbra{l_i:S_i}_{i \in \textit{fin$(n)$}}} & = \hask{$\interp{\btbra{(l_i : S_i) \backslash (l_n : S_n)}_{i \in \textit{fin$(n-1)$}}}$ :\& $\interp{S_n}$} \\
\interp{\texttt{end}} & = \hask{End} 
\end{align*}
%%
The interpretation of contexts preserves the structure, applying the session-type interpretation.

We define a sequential composition operator for session-types and contexts, written $\bullet$,
defined:
%%
\begin{align*}
\begin{array}{rl}
S \; \bullet\; \hask{end} & \; = \; S \\
\hask{end} \; \bullet\; S & \; = \; S \\
\hask{$t$ :! $S$} \; \bullet \; T & \; = \; \hask{$t$ :! $(S \; \bullet \; T)$} \\
\hask{$t$ :? $S$} \; \bullet \; T & \; = \; \hask{$t$ :? $(S \; \bullet \; T)$} \\
\oplus[\tilde{l} : \tilde{S}] \bullet \; T & \; = \; \oplus[\tilde{l} : \tilde{S} \; \bullet \; T] \\
\&[\tilde{l} : \tilde{S}] \bullet \; T & \; = \; \&[\tilde{l} : \tilde{S} \; \bullet \; T] 
\end{array}
\end{align*}
%%

\paragraph{Correspondence between rules}
\begin{align*}
&		\trule{Send-val} \;\; \tree{
			\Gamma; \Lambda_1; \Delta_1 \proves P \hastype \Proc
			\quad
			\Gamma; \emptyset; \emptyset \proves V \hastype L
			\quad
			u:S \in \Delta_1 
		}{
			\Gamma; \Lambda_1; (\Delta_1 \setminus u:S) \cat u:\btout{L} S \proves \bout{u}{V} P \hastype \Proc
		} \\[0.6em]
\Rightarrow \quad  & \dfrac{\interp{\Gamma} \vdash \interp{P} : \hask{Session $\interp{\Delta_1}$ ()} \quad \interp{\Gamma} \vdash \interp{V} : \interp{L}}
                    {\Gamma, \; \hask{c} : \hask{Channel $\interp{u}$} \vdash \hask{(send c $\interp{V}$) >>= ($\lambda$() -> $\interp{P}$)}} \\
            & \qquad\quad \hask{:: Session \{$\interp{u}$ :-> $\interp{L}$:! End\}$\,\bullet \; \interp{\Delta_1}$ ()}
\end{align*}

\begin{align*}
&		\trule{Rcv-val } \;\; \tree{
		\Gamma; \Lambda_1; \Delta_1 \cat u: S \proves P \hastype \Proc
			\quad
			\Gamma; \emptyset; \emptyset \proves {x} \hastype {L}
		}{
			\Gamma \backslash x; \Lambda_1; \Delta_1 \cat u: \btinp{L} S \vdash \binp{u}{{x}} P \hastype \Proc
		} \\[0.6em]
\Rightarrow \quad & \dfrac{\interp{\Gamma, x : L} \vdash \interp{P} : \hask{Session $\interp{\Delta_1 \cat u: S}$ ()}}
                          {\interp{\Gamma}, \; \hask{c} : \hask{Channel $\interp{u}$} \vdash \hask{(recv c) >>= ($\lambda$x -> $\interp{P}$)}} \\
 & \qquad\quad \hask{:: Session \{$\interp{u}$ :-> $\interp{L}$:? End\}$\,\bullet \; \interp{\Delta_1 \cat u: S}$ ()}
\end{align*}
