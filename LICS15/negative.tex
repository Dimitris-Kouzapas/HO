% !TEX root = main.tex

\subsection{A Negative Result}
\noi As most session calculi, 
\HOp includes communication on both shared and linear names.
The former enables non determinism and unrestricted behavior; the latter allows to represent
deterministic and linear communication structures.
The expressive power of shared names is also illustrated by our 
encoding from \HOp into \sessp (\thmref{f:enc:hotopi}).
%Shared and linear names are fundamentally different; still, to the best of our knowledge,
%the status of shared communication, in terms of expressiveness, has not been formalized for session calculi.
This result begs the question: are shared names truly indispensable for communication, or could they
be encoded using linear communication?
Here we prove that shared names actually add expressiveness to \HOp,
showing 
the non existence of a minimal encoding 
(cf.~\defref{def:goodenc}(ii))
of shared name communication into linear 
communication. 
%for their behavior cannot be represented using purely deterministic processes.
%To this end, we show the non existence of a minimal encoding 
%(cf.~\defref{def:goodenc}(ii))
%of shared name communication into linear 
%communication. 
As described next, 
$\tau$-inertness (\lemref{lem:tau_inert}) is critical in the proof.
Recall that minimal encodings preserve barbs (\propref{p:barbpres}).

\begin{theorem}\rm
There is no %semantic preserving 
minimal
encoding 
from $\sessp$ to $\HOp^{\minussh}$. Hence, 
for any $\CAL_1,\CAL_2\in \{ \HOp, \HO, \sessp\}$, 
there is no 
minimal
encoding from $\CAL_1$ into $\CAL_2^{-\mathsf{sh}}$.  
\end{theorem}

\dk{
\begin{remark}
	It is straightforward to derive from 
	\defref{d:enc:hopitoho} and~\ref{d:enc:hopitopi} 
	and \propref{f:enc:hopitoho} and~\ref{f:enc:hotopi}
	that
	encoding $\CAL_1^{-\mathsf{sh}}$ to $\CAL_2^{-\mathsf{sh}}$
	is precise.
\end{remark}
}

\begin{comment}
\begin{IEEEproof}[Proof]
	Let $\horel{\Gamma_1}{\Delta_1}{P_1}{\not\wb}{\Delta_2}{P_2}$
	with $P = \breq{a}{s} \inact \Par \bacc{a}{x} P_1 \Par \bacc{a}{x} P_2$ and	let $\Gamma; \emptyset; \Delta \proves P \hastype \Proc$.
	Assume also a encoding
	$\enc{\cdot}{\cdot}: \sessp \longrightarrow \HOp^{\minussh}$
is minimum. 
	From operational correspondence we obtain:
\[
\begin{array}{rcl}
		P \red P_1 \Par \bacc{a}{x} P_2 &\textrm{implies}& \map{P} \red \map{P_1 \Par \bacc{a}{x} P_2}\\
		P \red P_2 \Par \bacc{a}{x} P_1 &\textrm{implies}& \map{P} \red \map{P_2 \Par \bacc{a}{x} P_1}
\end{array}
\]
	From the fact that
	$\horel{\Gamma_1}{\Delta_1}{P_1}{\not\wb}{\Delta_2}{P_2}$
	we can derive that
%
	\[
		\horel{\Gamma_1'}{\Delta_1'}{P_1 \Par \bacc{a}{x} P_2}{\not\wb}{\Delta_2'}{P_2 \Par \bacc{a}{x} P_1}
	\]
%
	From \lemref{lem:tau_inert}(2) we know that
%
\[
\begin{array}{rcl}
		\horel{\mapt{\Gamma}}{\mapt{\Delta}}{\map{P}}{\wb}{\mapt{\Delta_1'}}{\map{P_1 \Par \bacc{a}{x} P_2}}\wb 
{\mapt{\Delta_2'}}\proves {\map{P_2 \Par \bacc{a}{x} P_1}}
\end{array}
\]
%
	\noi thus
$\horel{\mapt{\Gamma}}{\mapt{\Delta_1'}}{\map{P_1 \Par \bacc{a}{x} P_2}}{\wb}{\mapt{\Delta_2'}}{\map{P_2 \Par \bacc{a}{x} P_1}}$, 
%
	which contradicts the assumption. 
%	so there is no mapping $\map{\cdot}: \pHO \longrightarrow \spi$ that enjoys
%	the operational correspondence and full abstraction properties.
\end{IEEEproof}
\end{comment}


