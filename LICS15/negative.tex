% !TEX root = main.tex
\noi As most session calculi, 
\HOp includes communication on both shared and linear channels.
The former enables non determinism and unrestricted behavior; the latter allows to represent
deterministic and linear communication structures.
The expressive power of shared names is also illustrated by our 
encoding from \HOp into \sessp (Definition \ref{def:enc:HOp_to_FO}).
Shared and linear channels are fundamentally different; still, to the best of our knowledge,
the status of shared communication, in terms of expressiveness, has not been formalized for session calculi.

The above begs the question: are shared names truly indispensable for communication, or could they
be encoded using linear communication?
Here we prove that shared names actually add expressiveness to \HOp,
for their behavior cannot be represented using purely deterministic processes.
To this end, we show the non existence of a minimal encoding 
(cf. Definition~\ref{def:goodenc}(ii))
of shared name communication into linear 
communication. Recall that minimal encodings preserve barbs (Proposition~\ref{p:barbpres}).


\begin{theorem}\rm
There is no semantic preserving encoding 
from $\sessp$ to $\HOp^{\minussh}$. Hence 
for any $\CAL_1,\CAL_2\in \{ \HOp, \HO, \sessp\}$, 
there is no encoding from $\CAL_1$ into $\CAL_2^{-\mathsf{sh}}$.  
\end{theorem}
\begin{IEEEproof}[Proof (Sketch)]
	Let $\horel{\Gamma_1}{\Delta_1}{P_1}{\not\wb}{\Delta_2}{P_2}$
	with $P = \breq{a}{s} \inact \Par \bacc{a}{x} P_1 \Par \bacc{a}{x} P_2$ and	let $\Gamma; \emptyset; \Delta \proves P \hastype \Proc$.
	Assume also a encoding
	$\enc{\cdot}{\cdot}: \sessp \longrightarrow \HOp^{\minussh}$
	that enjoys
	operational correspondence and full abstraction.

	From operational correspondence we get that:
	\begin{eqnarray*}
		P \red P_1 \Par \bacc{a}{x} P_2 &\textrm{implies}& \map{P} \red \map{P_1 \Par \bacc{a}{x} P_2}\\
		P \red P_2 \Par \bacc{a}{x} P_1 &\textrm{implies}& \map{P} \red \map{P_2 \Par \bacc{a}{x} P_1}
	\end{eqnarray*}

	From the fact that
	$\horel{\Gamma_1}{\Delta_1}{P_1}{\not\wb}{\Delta_2}{P_2}$
	we can derive that
%
	\[
		\horel{\Gamma_1'}{\Delta_1'}{P_1 \Par \bacc{a}{x} P_2}{\not\wb}{\Delta_2'}{P_2 \Par \bacc{a}{x} P_1}
	\]
%
	From Lemma~\ref{lem:tau_inert} we know that
%
	\begin{eqnarray*}
		\horel{\mapt{\Gamma}}{\mapt{\Delta}}{\map{P}}{\wb}{\mapt{\Delta_1'}}{\map{P_1 \Par \bacc{a}{x} P_2}}\\
		\horel{\mapt{\Gamma}}{\mapt{\Delta}}{\map{P}}{\wb}{\mapt{\Delta_2'}}{\map{P_2 \Par \bacc{a}{x} P_1}}
	\end{eqnarray*}
%
	\noi thus
	\[
		\horel{\mapt{\Gamma}}{\mapt{\Delta_1'}}{\map{P_1 \Par \bacc{a}{x} P_2}}{\wb}{\mapt{\Delta_2'}}{\map{P_2 \Par \bacc{a}{x} P_1}}
	\]
%
	From here we conclude that the full abstraction property does not hold,
	which is a contradiction.
%	so there is no mapping $\map{\cdot}: \pHO \longrightarrow \spi$ that enjoys
%	the operational correspondence and full abstraction properties.
\end{IEEEproof}


