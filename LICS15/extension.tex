% !TEX root = main.tex
\noi This section studies %two extensions of \HOp: 
(i)~the extension of \HOp with higher-order applications/abstractions 
and 
(ii)~the extension of \HOp
with polyadicity.
In both cases, we detail required modifications in the syntax
and types, and
describe further encodability results.
%These extensions are denoted \HOpp and \PHOp, respectively. 
 
\subsection{Encoding from $\HOpp$ to $\HOp$}
\label{subsec:hop}
\noi 
The calculus \HOpp 
extends \HOp with higher-order abstractions and applications.

\myparagraph{Syntax, Operational Semantics and Types.}
\noi First, the syntax of \figref{fig:syntax} extends 
$\appl{V}{u}$ to 
 $\appl{V}{W}$, including higher-order value $W$. 
The rule $\appl{(\abs{x}{P})}{V} \red P \subst{V}{x}$
replaces
rule $\orule{App}$ in \figref{fig:reduction}.
The syntax of types is modified as follows: %changes to include: 
\begin{center}
\begin{tabular}{c}
$L \bnfis \shot{U} \bnfbar \lhot{U}$
\end{tabular}
\end{center}
These types can be easily accommodated in the type system:
 in \figref{fig:typerulesmy}, 
we replace $C$ by $U$ in \trule{Abs} and $C$ by $U'$ in \trule{App}.
\smallskip 

\myparagraph{Behavioural Semantics.}
Labels remain the same. Rule $\ltsrule{App}$ in the untyped LTS
(\figref{fig:untyped_LTS}) 
is replaced with rule $\appl{(\abs{x}{P})}{V} \by{\tau} P \subst{V}{x}$.
\defref{def:char} (characteristic processes) is extended with  
${\mapchar{\shot{U}}{x}} \defeq\! \mapchar{\lhot{U}}{x} \defeq\! {\appl{x}{\omapchar{U}}}$ and 
${\omapchar{\shot{U}}} \defeq {\omapchar{\lhot{U}}} \!\!\defeq\!\! \abs{x}{\mapchar{U}{x}}$. 
We can then use the same definitions for $\cong$, $\wbc$, $\hwb$ and $\fwb$. 

\smallskip 

\myparagraph{Encoding \HOpp into \HOp.} 
Let $\tyl{L}_{\HOpp}=\calc{\HOpp}{{\cal{T}}_4}{\hby{\ell}}{\wb_H}{\proves}$
where 
${\cal{T}}_4$ is a set of types of $\HOpp$;  
the typing $\proves$ is defined in 
\figref{fig:typerulesmy} with extended rules \trule{Abs} and \trule{App}. 
We define 
the typed encoding $\enco{\map{\cdot}^{3}, \mapt{\cdot}^{3}, \mapa{\cdot}^{3}}: \HOpp \to \HOp$ in 
\figref{f:enc:hopip_to_hopi}.
By \propref{pro:composition}, 
we derive the following theorem. 

\smallskip 

\begin{theorem}[Encoding \HOpp into~\HOp]
\label{f:enc:hopiptohopi}
The encoding from $\tyl{L}_{\HOpp}$ into $\tyl{L}_{\HOp}$ (cf. \figref{f:enc:hopip_to_hopi})
is precise. Hence, the encodings 
from $\tyl{L}_{\HOpp}$ to $\tyl{L}_{\HO}$ 
and $\tyl{L}_{\sessp}$ 
are also precise. 
\end{theorem}
\smallskip 

\begin{figure}[t]
\[
{\small
\begin{array}{lrcll}
\noindent{\bf Types:} & 
		\tmap{\shot{L}}{3} &\!\!\!\!\defeq\!\!\!\!& \shot{\btinp{\tmap{L}{3}} \tinact}
		\\
%&		\tmap{\lhot{L}}{3} &\!\!\!\!\defeq\!\!\!\!& \lhot{\btinp{\tmap{L}{3}} \tinact}
%		\\
&		\tmap{\btout{\shot{L}} S}{3} &\!\!\!\!\defeq\!\!\!\!& \btout{\tmap{\shot{L}}{3}} \tmap{S}{3}
		\\
%&		\tmap{\btout{\lhot{L}} S}{3} &\!\!\!\!\defeq\!\!\!\!& \btout{\tmap{\lhot{L}}{3}} \tmap{S}{3}
%		\\
&		\tmap{\btinp{\shot{L}} S}{3} &\!\!\!\!\defeq\!\!\!\!& \btinp{\tmap{\shot{L}}{3}} \tmap{S}{3}
%		\\
%&		\tmap{\btinp{\lhot{L}} S}{3} &\!\!\!\!\defeq\!\!\!\!& \btinp{\tmap{\lhot{L}}{3}} \tmap{S}{3}
\\
\hline
\noindent{\bf Labels:} & 
%		\mapa{\bactout{n}{\abs{x:C}{P}}}^{3} &\!\!\!\!\defeq\!\!\!\!& \bactout{n}{\abs{x}{\pmap{P}{3}}}
%		\\
%		\mapa{\bactinp{n}{\abs{x:C}{P}}}^{3} &\!\!\!\!\defeq\!\!\!\!& \bactinp{n}{\abs{x}{\pmap{P}{3}}}
%		\\
		\mapa{\news{\tilde{m}} \bactout{n}{\abs{\AT{x}{L}}{P}}}^{3} &\!\!\!\!\defeq\!\!\!\!& \news{\tilde{m}} \bactout{n}{\abs{z}{\binp{z}{x} \pmap{P}{3}}}
		\\
&		\mapa{\bactinp{n}{\abs{\AT{x}{L}}{P}}}^{3} &\!\!\!\!\defeq\!\!\!\!& \bactinp{n}{\abs{z}{\binp{z}{x} \pmap{P}{3}}}
\\
\hline
{\bf Terms}: & 
	\pmap{\appl{x}{(\abs{y} P)}}{3} &\!\!\!\!\defeq\!\!\!\!& \newsp{s}{\appl{x}{s} \Par \bout{\dual{s}}{\abs{y} \pmap{P}{3}} \inact}
		\\
&	\pmap{\bout{u}{\abs{\AT{x}{L}}{Q}} P}{3} &\!\!\!\!\defeq\!\!\!\!& \bout{u}{\abs{z}{\binp{z}{x} \pmap{Q}{3}}} \pmap{P}{3} \\
%		\pmap{\bout{u}{\abs{x: C}{Q}} P}{3} &\!\!\!\!\defeq\!\!\!\!& \bout{u}{\abs{x}{\pmap{Q}{3}}} \pmap{P}{3}
&  \pmap{\appl{(\abs{x}{P})}{(\abs{x}Q})}{3} &\!\!\!\!\defeq\!\!\!\!& \newsp{s}{\binp{s}{x} \pmap{P}{3} \Par  \bout{\dual{s}}{\abs{x} \pmap{Q}{3}} \inact}
%[[(?x P_1) (?x P_2)]] = (? s)((?x [[P]])s | \bout{s}{?x P_2} 0 )
	\end{array}
	}
	\]
	{\small $\tmap{\lhot{L}}{3}$ is defined as
	$\tmap{\shot{L}}{3}$
	by replacing $\shot{L}$ with~$\lhot{L}$. 
	Label and term mappings for $\abs{x:C}{P}$  
are %defined 
as in \figref{f:enc:hopi_to_ho}, replacing 
%The mapping of types for $\lhot{L}$ is defined by replacing 
%$\shot{L}$ by $\lhot{L}$. 
%The case of $\abs{x:C}{P}$ in the label and term mappings 
%are %defined 
%as in \figref{f:enc:hopi_to_ho}, replacing 
$\tmap{\cdot}{1}$,
$\mapa{\cdot}^{1}$, and 
$\pmap{\cdot}{1}_f$, by  
$\tmap{\cdot}{3}$,
$\mapa{\cdot}^{3}$, and 
$\pmap{\cdot}{3}$. \\
The other mappings for processes, types and labels are  homomorphic.} \\
\caption{\label{f:enc:hopip_to_hopi} 
Encoding of \HOpp into \HOp.
}
\Hlinefig
\end{figure} 

\subsection{Encoding from Polyadic $\HOp$ to $\HOp$}
\label{subsec:pho}
\noi Embeddings of polyadic name passing into monadic name passing are
well-studied. % in the literature. 
Using a linear typing, precise
encodings (including full abstraction) can be obtained~\cite{Yoshida96}.
Here we summarise how $\PHOp$ can be encoded into $\HOp$. 
The syntax of 
$\HOp$ is extended %from \HOp by including 
with
polyadic name passing $\tilde{n}$ and $\abs{\tilde{x}}{Q}$ in the syntax 
of value $V$. The type syntax is extended to: 
%
\begin{center}
\begin{tabular}{c}
$
L ::= \shot{\tilde{C}} \ | \ \lhot{\tilde{C}}
\quad\quad S \ ::= \  \btout{\tilde{U}} S \bnfbar \btinp{\tilde{U}} S \bnfbar \cdots 
$
\end{tabular}
\end{center}
%
\NY{The type system disallows polyadicity along shared names as in \cite{tlca07,MostrousY15}.}
Other definitions are straightforwardly extended. 
\jpc{We slightly modify \defref{def:tenc} to capture that a 
label $\ell$ may be mapped into a sequence of labels $\tilde{\ell}$.}
%We extend the mapping for labels 
%($\mapa{\cdot}: \ell \to \tilde{\ell}$ in  
%\defref{def:tenc}) to capture 
%a sequence of labels  and 
Also, \defref{def:ep} is kept unchanged, 
assuming that if 
$P \hby{\ell} P'$ and $\mapa{\ell} = \ell_1, \ell_2,  \cdots, \ell_m$ then
$\map{P} \Hby{\mapa{\ell}} \map{P'}$
should be understood as
$\map{P} \Hby{\ell_1} P_1 \Hby{\ell_2} P_2 \cdots \Hby{\ell_m} P_m =  \map{P'}$,
for some
$P_1, P_2, \ldots, P_m$.

Let $\tyl{L}_{\PHOp}=\calc{\PHOp}{{\cal{T}}_5}{\hby{\ell}}{\wb_H}{\proves}$
where 
${\cal{T}}_5$ is a set of types of $\HOpp$;  
the typing $\proves$ is defined in 
\figref{fig:typerulesmy} with polyadic types. 
We define %the typed encoding 
the typed encoding $\enco{\map{\cdot}^{4}, \mapt{\cdot}^{4}, \mapa{\cdot}^{4}}: \PHOp \to \HOp$ 
in \figref{f:enc:poltomon}. We give the dyadic case (tuples of length 2), for simplicity;
the general case is as expected.
Then we have:

\smallskip 

\begin{theorem}[Encoding of \pHOp into \HOp]
\label{f:enc:phopiptohopi}
The encoding from $\tyl{L}_{\PHOp}$ into $\tyl{L}_{\HOp}$ (cf. \figref{f:enc:poltomon})
is precise. 
Hence, the encodings 
from $\tyl{L}_{\PHOp}$ to 
$\tyl{L}_{\HO}$ 
and $\tyl{L}_{\sessp}$ 
are also precise. 
\end{theorem}
\NY{
By combining Thms.~\ref{f:enc:hopiptohopi} and~\ref{f:enc:phopiptohopi},
we can easily extend the preciseness to 
$\PHOpp$, the super-calculus of $\HOpp$ and $\PHOp$. % (denoted by   in Fig.~\ref{fig:express}) 
}

\begin{figure}[t]
\small
\[
\begin{array}{rcl}
% typed mapping starts here
{\bf Types:}\hspace{2.2cm}\\
		\tmap{\btout{S_1, S_2}S}{4}
		&\!\!\!\!\defeq\!\!\!\!&
		\btout{\tmap{S_1}{4}}  \btout{\tmap{S_2}{4}}\tmap{S}{4}
%		\\
%		\tmap{\btinp{S_1, \cdots, S_m}S}{4}
%		&\!\!\!\!\defeq\!\!\!\!&
%		\btinp{\tmap{S_1}{4}} \cdots ; \btinp{\tmap{S_m}{4}}\tmap{S}{4}
		\\
		\tmap{\bbtout{L} S}{4}
		&\!\!\!\!\defeq\!\!\!\!&
		\bbtout{\mapt{L}^{4}}\mapt{S}^{4}
%		\\
%		\tmap{\bbtinp{L} S}{4}
%		&\defeq&
%		\bbtinp{\mapt{L}^{4}}\mapt{S}^{4}
		\\
%		\tmap{\bbtout{\shot{(C_1, \cdots, C_m)}} S}{4}
%		&\defeq&
%		\bbtout{
%		\shot{\big(\btinp{\tmap{C_1}{4}} \cdots; \btinp{\tmap{C_m}{4}}\tinact\big)}}\mapt{S}^{4}
%		\\
%		\tmap{\bbtinp{\shot{(C_1, \cdots, C_m)}} S}{4}
%		&\defeq&
%		\bbtinp{
%		\shot{\big(\btinp{\tmap{C_1}{4}} \cdots; \btinp{\tmap{C_m}{4}}\tinact\big)}}\mapt{S}^{4}
%		\\
		\tmap{\shot{(C_2,  C_2)}}{4}
		&\!\!\!\!\defeq\!\!\!\!&
		\shot{\big(\btinp{\tmap{C_1}{4}}  \btinp{\tmap{C_2}{4}}\tinact\big)}
		\\
		\tmap{\lhot{(C_1,  C_2)}}{4}
		&\!\!\!\!\defeq\!\!\!\!&
		\lhot{\big(\btinp{\tmap{C_1}{4}}  \btinp{\tmap{C_2}{4}}\tinact\big)}
		\\[1mm]
%		\tmap{\lhot{(C_1, \cdots, C_m)}}{4}
%		&\defeq&
%		\lhot{\big(\btinp{\tmap{C_1}{\mathsf{p}}} \cdots \btinp{\tmap{C_m}{\mathsf{p}}}\tinact\big)}
%		\\
%		\tmap{\shot{(C_1, \cdots, C_m)}}{\mathsf{p}}
%		&\defeq&
%		\shot{\big(\btinp{\tmap{C_1}{\mathsf{p}}} \cdots \btinp{\tmap{C_m}{\mathsf{p}}}\tinact\big)}
\hline
{\bf Labels:}\hspace{2.2cm}\\
%%		\\ % action mapping starts here
%		\mapa{\news{\tilde{m}} \bactout{n}{m_1, \ldots, m_k}}^4 
%		&\!\!\!\!\defeq\!\!\!\!&
%		\news{\tilde{m_1}'}\bactout{n}{m_1} \cdots \news{\tilde{m_k}'}\bactout{n}{m_k} \\
%		&& m_i \in \tilde{m} \Leftrightarrow \tilde{m_i}' = m_i \wedge
%		\\
%		&& m_i \notin \tilde{m} \Leftrightarrow \tilde{m_i}' = \es 
%		\\
%%		\\ % proposal for action mapping starts here
		\mapa{\news{\tilde{m}} \bactout{n}{m_1,  m_2}}^4 
		&\!\!\!\!\defeq\!\!\!\!&
		\jpc{\ell_1, \ell_2 }\\
		& &  (m_i \in \tilde{m}  \Leftrightarrow \ell_i = \news{m_i}\bactout{n}{m_i} ) ~ \lor \\
		& & (m_i \not\in \tilde{m}  \Leftrightarrow  \ell_i = \bactout{n}{m_i}) \\
%		\mapa{\bactinp{k}{k_1, \ldots, k_m}}^4 &\defeq&   \big\{\bactinp{k}{k_1}, \cdots, \bactinp{k}{k_m} \big\}\\
		\mapa{\news{\tilde{m}} \bactout{n}{\abs{(x_1, x_2)}{P}} }^4 
		&\!\!\!\!\defeq\!\!\!\!&
		\news{\tilde{m}} \bactout{n}{\abs{z}\binp{z}{x_1} \binp{z}{x_2} \map{P}^{4}}\\[1mm]
%
\dk{		\mapa{\tau}^4 }
		&\!\!\!\!\defeq\!\!\!\!&
\dk{		\tau,  \tau}
		\\[1mm]

%		\mapa{\bactinp{k}{\abs{x_1, \ldots, x_m}{P}} }^4 &\defeq&  \bactinp{k}{\abs{z}\binp{z}{x_1} \cdots . \binp{z}{x_m} \map{P}^{4}} 
\hline
{\bf Terms:}\hspace{2.2cm}\\
		\map{\bout{u}{u_1, u_2} P}^{4}
		&\!\!\!\!\defeq\!\!\!\!&
		\bout{u}{u_1} \bout{u}{u_2} \map{P}^{4}
		\\
%			\map{\binp{k}{x_1, \cdots, x_m} P}^{4}
%		&\defeq&
%		\binp{k}{x_1} \cdots . \binp{k}{x_m}  \map{P}^{4}
%		\\
		\map{\bbout{u}{\abs{(x_1,  x_2)} Q} P}^{4}
		&\!\!\!\!\defeq\!\!\!\!&
		\bbout{u}{\abs{z}\binp{z}{x_1} \binp{z}{x_2} \map{Q}^{4}} \map{P}^{4}
		\\ 
		\map{\appl{x}{(u_1,   u_2)}}^{4}
		&\!\!\!\!\defeq\!\!\!\!&
		\newsp{s}{\appl{x}{s} \Par \bout{\dual{s}}{u_1}   \bout{\dual{s}}{u_2} \inact}
		\\
		\map{\appl{(\abs{(x_1,x_2)}{P})}{(u_1,   u_2)}}^{4}
		&\!\!\!\!\defeq\!\!\!\!& \\
		%\newsp{s}{\binp{s}{x_1} \dots \binp{s}{x_n} \pmap{P}{4} \Par \bout{\dual{s}}{u_1}  \bout{\dual{s}}{u_2} \inact} 
		\multicolumn{3}{l}{
		\qquad \qquad \qquad \qquad \newsp{s}{\binp{s}{x_1}  \binp{s}{x_2} \pmap{P}{4} \Par \bout{\dual{s}}{u_1}  \bout{\dual{s}}{u_2} \inact} 
		}
        \\ 
	\end{array}
\]
We give the dyadic case;
the general polyadic case is as expected.
The input cases are defined as the outputs replacing $!$ by $?$. 
Mappings for the other processes/types/labels are 
homomorphic.
\caption{\label{f:enc:poltomon}
Encoding of \PHOp into \HOp. 
}
\Hlinefig 
\end{figure}

