This section studies two extensions of \HOp, 
polyadic (\PHO)
and arbitrariry higher-order applications (\HOpp). 

\subsection{Encoding from Polyadic $\HOp$ to $\HOp$}
\label{subsec:pho}

\subsection{Encoding from $\HOpp$ to $\HOp$}
\label{subsec:pho}
\myparagraph{Syntax, Operational Semantics and Types}
\noi First, the syntax of Fig.~\ref{fig:syntax} extends 
$\appl{x}{u}$ to 
 $\appl{x}{V}$, including higher-order value $V$. 
We then replace rule $\orule{App}$ in Fig.~\ref{fig:reduction}
with rule $\appl{(\abs{x}{P})}{V} \red P \subst{V}{x}$.
The syntax of types changes to include: 
\[ L \bnfis \shot{U} \bnfbar \lhot{U}\]  
We apply the straightforward extension of the typing  
system to accomodate the extended type syntax 
(we just need to replace $C$ by $U$ in \trule{Abs} and $C$ by $U'$ in \trule{App} in Fig.~\ref{fig:typerulesmy}).
\smallskip 

\myparagraph{Behavioural Semantics}
Labels remain the same. Rule $\ltsrule{App}$ in the untyped LTS
(Fig.~\ref{fig:untyped_LTS}) 
is replaced with rule $\appl{\abs{x}{P}}{V} \by{\tau} P \subst{V}{x}$.
Characteristic processes (Definition~\ref{def:char}) are extended with  
${\mapchar{\shot{U}}{x}} \!\!\defeq\!\! \mapchar{\lhot{U}}{x} \!\!\defeq\!\! {\appl{x}{\omapchar{U}}}$ and ${\omapchar{\shot{U}}} \defeq {\omapchar{\lhot{U}}} \!\!\defeq\!\! \abs{x}{\mapchar{U}{x}}$. 
Then we can use the same definitions for $\cong$, $\wbc$, $\hwb$ and $\fwb$. 

\smallskip 

\myparagraph{Encoding from \HOpp to \HOp} 
Let $\tyl{L}_{\HOpp}=\calc{\HOpp}{{\cal{T}}_5}{\hby{\ell}}{\wb_H}{\proves}$
where 
${\cal{T}}_5$ is a set of types of $\HOpp$;  
the typing $\proves$ is defined in 
Appendix~\ref{app:types}. 
We define the typed encoding 
the typed encoding $\enco{\map{\cdot}^{4}, \mapt{\cdot}^{4}, \mapa{\cdot}^{4}}: \HOpp \to \HOp$ in 
in Fig.~\ref{f:enc:hopip_to_hopi}. 
By Proposition~\ref{pro:composition}, 
we have the following main result. 

\smallskip 

\begin{theorem}[Encoding of $n$-Higher-Order Pi into Pi]
\label{f:enc:hopiptohopi}
The encoding from $\tyl{L}_{\HOpp}$ into $\tyl{L}_{\HOp}$ 
defined in Definition~\ref{d:enc:hopiptohopi}
is precise. Hence the encoding 
from $\tyl{L}_{\HOpp}$ to $\tyl{L}_{\HO}$ 
and from $\tyl{L}_{\HOpp}$ to $\tyl{L}_{\sessp}$ 
are also precise. 
\end{theorem}

\begin{figure}[t]
\[
\begin{array}{lrcll}
\noindent{\bf Types:} & 
		\tmap{\shot{L}}{3} &\defeq& \shot{\btinp{\tmap{L}{3}} \tinact}
		\\
&		\tmap{\lhot{L}}{3} &\defeq& \lhot{\btinp{\tmap{L}{3}} \tinact}
		\\
&		\tmap{\btout{\shot{L}} S}{3} &\defeq& \btout{\tmap{\shot{L}}{3}} \tmap{S}{3}
		\\
&		\tmap{\btout{\lhot{L}} S}{3} &\defeq& \btout{\tmap{\lhot{L}}{3}} \tmap{S}{3}
		\\
&		\tmap{\btinp{\shot{L}} S}{3} &\defeq& \btinp{\tmap{\shot{L}}{3}} \tmap{S}{3}
		\\
&		\tmap{\btinp{\lhot{L}} S}{3} &\defeq& \btinp{\tmap{\lhot{L}}{3}} \tmap{S}{3}
\\[1mm]
\hline
\noindent{\bf Labels:} & 
%		\mapa{\bactout{n}{\abs{x:C}{P}}}^{3} &\defeq& \bactout{n}{\abs{x}{\pmap{P}{3}}}
%		\\
%		\mapa{\bactinp{n}{\abs{x:C}{P}}}^{3} &\defeq& \bactinp{n}{\abs{x}{\pmap{P}{3}}}
%		\\
		\mapa{\bactout{n}{\abs{\AT{x}{L}}{P}}}^{3} &\defeq& \bactout{n}{\abs{z}{\binp{z}{x} \pmap{P}{3}}}
		\\
&		\mapa{\bactinp{n}{\abs{\AT{x}{L}}{P}}}^{3} &\defeq& \bactinp{n}{\abs{z}{\binp{z}{x} \pmap{P}{3}}}
\\[1mm]
\hline
{\bf Terms}: & 
	\pmap{\appl{x}{(\abs{x} P)}}{3} &\defeq& \newsp{s}{\appl{x}{s} \Par \bout{\dual{s}}{\abs{x} \pmap{P}{3}} \inact}
		\\
&	\pmap{\bout{u}{\abs{\AT{x}{L}}{Q}} P}{3} &\defeq& \bout{u}{\abs{z}{\binp{z}{x} \pmap{Q}{3}}} \pmap{P}{3}
%		\pmap{\bout{u}{\abs{x: C}{Q}} P}{3} &\defeq& \bout{u}{\abs{x}{\pmap{Q}{3}}} \pmap{P}{3}
	\end{array}
	\]
The case of $\abs{x:C}{P}$ in the label and term mappings 
are defined as in Fig.~\ref{f:enc:hopi_to_ho}.  
The other processes, types and labels are  homomorphic. 
\caption{\label{f:enc:hopip_to_hopi} 
Encoding of \HOpp into \HOp.
}
\Hline
\end{figure} 







