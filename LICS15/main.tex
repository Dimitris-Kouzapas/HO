
%% bare_conf.tex
%% V1.4a
%% 2014/09/17
%% by Michael Shell
%% See:
%% http://www.michaelshell.org/
%% for current contact information.
%%
%% This is a skeleton file demonstrating the use of IEEEtran.cls
%% (requires IEEEtran.cls version 1.8a or later) with an IEEE
%% conference paper.
%%
%% Support sites:
%% http://www.michaelshell.org/tex/ieeetran/
%% http://www.ctan.org/tex-archive/macros/latex/contrib/IEEEtran/
%% and
%% http://www.ieee.org/

%%*************************************************************************
%% Legal Notice:
%% This code is offered as-is without any warranty either expressed or
%% implied; without even the implied warranty of MERCHANTABILITY or
%% FITNESS FOR A PARTICULAR PURPOSE! 
%% User assumes all risk.
%% In no event shall IEEE or any contributor to this code be liable for
%% any damages or losses, including, but not limited to, incidental,
%% consequential, or any other damages, resulting from the use or misuse
%% of any information contained here.
%%
%% All comments are the opinions of their respective authors and are not
%% necessarily endorsed by the IEEE.
%%
%% This work is distributed under the LaTeX Project Public License (LPPL)
%% ( http://www.latex-project.org/ ) version 1.3, and may be freely used,
%% distributed and modified. A copy of the LPPL, version 1.3, is included
%% in the base LaTeX documentation of all distributions of LaTeX released
%% 2003/12/01 or later.
%% Retain all contribution notices and credits.
%% ** Modified files should be clearly indicated as such, including  **
%% ** renaming them and changing author support contact information. **
%%
%% File list of work: IEEEtran.cls, IEEEtran_HOWTO.pdf, bare_adv.tex,
%%                    bare_conf.tex, bare_jrnl.tex, bare_conf_compsoc.tex,
%%                    bare_jrnl_compsoc.tex, bare_jrnl_transmag.tex
%%*************************************************************************


% *** Authors should verify (and, if needed, correct) their LaTeX system  ***
% *** with the testflow diagnostic prior to trusting their LaTeX platform ***
% *** with production work. IEEE's font choices and paper sizes can       ***
% *** trigger bugs that do not appear when using other class files.       ***                          ***
% The testflow support page is at:
% http://www.michaelshell.org/tex/testflow/



\documentclass[conference]{IEEEtran}
% Some Computer Society conferences also require the compsoc mode option,
% but others use the standard conference format.
%
% If IEEEtran.cls has not been installed into the LaTeX system files,
% manually specify the path to it like:
% \documentclass[conference]{../sty/IEEEtran}

% Some very useful LaTeX packages include:
% (uncomment the ones you want to load)


% *** MISC UTILITY PACKAGES ***
%
%\usepackage{ifpdf}
% Heiko Oberdiek's ifpdf.sty is very useful if you need conditional
% compilation based on whether the output is pdf or dvi.
% usage:
% \ifpdf
%   % pdf code
% \else
%   % dvi code
% \fi
% The latest version of ifpdf.sty can be obtained from:
% http://www.ctan.org/tex-archive/macros/latex/contrib/oberdiek/
% Also, note that IEEEtran.cls V1.7 and later provides a builtin
% \ifCLASSINFOpdf conditional that works the same way.
% When switching from latex to pdflatex and vice-versa, the compiler may
% have to be run twice to clear warning/error messages.

% *** CITATION PACKAGES ***
%
%\usepackage{cite}
% cite.sty was written by Donald Arseneau
% V1.6 and later of IEEEtran pre-defines the format of the cite.sty package
% \cite{} output to follow that of IEEE. Loading the cite package will
% result in citation numbers being automatically sorted and properly
% "compressed/ranged". e.g., [1], [9], [2], [7], [5], [6] without using
% cite.sty will become [1], [2], [5]--[7], [9] using cite.sty. cite.sty's
% \cite will automatically add leading space, if needed. Use cite.sty's
% noadjust option (cite.sty V3.8 and later) if you want to turn this off
% such as if a citation ever needs to be enclosed in parenthesis.
% cite.sty is already installed on most LaTeX systems. Be sure and use
% version 5.0 (2009-03-20) and later if using hyperref.sty.
% The latest version can be obtained at:
% http://www.ctan.org/tex-archive/macros/latex/contrib/cite/
% The documentation is contained in the cite.sty file itself.

% *** GRAPHICS RELATED PACKAGES ***
%
\ifCLASSINFOpdf
  % \usepackage[pdftex]{graphicx}
  % declare the path(s) where your graphic files are
  % \graphicspath{{../pdf/}{../jpeg/}}
  % and their extensions so you won't have to specify these with
  % every instance of \includegraphics
  % \DeclareGraphicsExtensions{.pdf,.jpeg,.png}
\else
  % or other class option (dvipsone, dvipdf, if not using dvips). graphicx
  % will default to the driver specified in the system graphics.cfg if no
  % driver is specified.
  % \usepackage[dvips]{graphicx}
  % declare the path(s) where your graphic files are
  % \graphicspath{{../eps/}}
  % and their extensions so you won't have to specify these with
  % every instance of \includegraphics
  % \DeclareGraphicsExtensions{.eps}
\fi
% graphicx was written by David Carlisle and Sebastian Rahtz. It is
% required if you want graphics, photos, etc. graphicx.sty is already
% installed on most LaTeX systems. The latest version and documentation
% can be obtained at: 
% http://www.ctan.org/tex-archive/macros/latex/required/graphics/
% Another good source of documentation is "Using Imported Graphics in
% LaTeX2e" by Keith Reckdahl which can be found at:
% http://www.ctan.org/tex-archive/info/epslatex/
%
% latex, and pdflatex in dvi mode, support graphics in encapsulated
% postscript (.eps) format. pdflatex in pdf mode supports graphics
% in .pdf, .jpeg, .png and .mps (metapost) formats. Users should ensure
% that all non-photo figures use a vector format (.eps, .pdf, .mps) and
% not a bitmapped formats (.jpeg, .png). IEEE frowns on bitmapped formats
% which can result in "jaggedy"/blurry rendering of lines and letters as
% well as large increases in file sizes.
%
% You can find documentation about the pdfTeX application at:
% http://www.tug.org/applications/pdftex





% *** MATH PACKAGES ***
%
%\usepackage[cmex10]{amsmath}
% A popular package from the American Mathematical Society that provides
% many useful and powerful commands for dealing with mathematics. If using
% it, be sure to load this package with the cmex10 option to ensure that
% only type 1 fonts will utilized at all point sizes. Without this option,
% it is possible that some math symbols, particularly those within
% footnotes, will be rendered in bitmap form which will result in a
% document that can not be IEEE Xplore compliant!
%
% Also, note that the amsmath package sets \interdisplaylinepenalty to 10000
% thus preventing page breaks from occurring within multiline equations. Use:
%\interdisplaylinepenalty=2500
% after loading amsmath to restore such page breaks as IEEEtran.cls normally
% does. amsmath.sty is already installed on most LaTeX systems. The latest
% version and documentation can be obtained at:
% http://www.ctan.org/tex-archive/macros/latex/required/amslatex/math/





% *** SPECIALIZED LIST PACKAGES ***
%
%\usepackage{algorithmic}
% algorithmic.sty was written by Peter Williams and Rogerio Brito.
% This package provides an algorithmic environment fo describing algorithms.
% You can use the algorithmic environment in-text or within a figure
% environment to provide for a floating algorithm. Do NOT use the algorithm
% floating environment provided by algorithm.sty (by the same authors) or
% algorithm2e.sty (by Christophe Fiorio) as IEEE does not use dedicated
% algorithm float types and packages that provide these will not provide
% correct IEEE style captions. The latest version and documentation of
% algorithmic.sty can be obtained at:
% http://www.ctan.org/tex-archive/macros/latex/contrib/algorithms/
% There is also a support site at:
% http://algorithms.berlios.de/index.html
% Also of interest may be the (relatively newer and more customizable)
% algorithmicx.sty package by Szasz Janos:
% http://www.ctan.org/tex-archive/macros/latex/contrib/algorithmicx/




% *** ALIGNMENT PACKAGES ***
%
%\usepackage{array}
% Frank Mittelbach's and David Carlisle's array.sty patches and improves
% the standard LaTeX2e array and tabular environments to provide better
% appearance and additional user controls. As the default LaTeX2e table
% generation code is lacking to the point of almost being broken with
% respect to the quality of the end results, all users are strongly
% advised to use an enhanced (at the very least that provided by array.sty)
% set of table tools. array.sty is already installed on most systems. The
% latest version and documentation can be obtained at:
% http://www.ctan.org/tex-archive/macros/latex/required/tools/


% IEEEtran contains the IEEEeqnarray family of commands that can be used to
% generate multiline equations as well as matrices, tables, etc., of high
% quality.




% *** SUBFIGURE PACKAGES ***
%\ifCLASSOPTIONcompsoc
%  \usepackage[caption=false,font=normalsize,labelfont=sf,textfont=sf]{subfig}
%\else
%  \usepackage[caption=false,font=footnotesize]{subfig}
%\fi
% subfig.sty, written by Steven Douglas Cochran, is the modern replacement
% for subfigure.sty, the latter of which is no longer maintained and is
% incompatible with some LaTeX packages including fixltx2e. However,
% subfig.sty requires and automatically loads Axel Sommerfeldt's caption.sty
% which will override IEEEtran.cls' handling of captions and this will result
% in non-IEEE style figure/table captions. To prevent this problem, be sure
% and invoke subfig.sty's "caption=false" package option (available since
% subfig.sty version 1.3, 2005/06/28) as this is will preserve IEEEtran.cls
% handling of captions.
% Note that the Computer Society format requires a larger sans serif font
% than the serif footnote size font used in traditional IEEE formatting
% and thus the need to invoke different subfig.sty package options depending
% on whether compsoc mode has been enabled.
%
% The latest version and documentation of subfig.sty can be obtained at:
% http://www.ctan.org/tex-archive/macros/latex/contrib/subfig/




% *** FLOAT PACKAGES ***
%
%\usepackage{fixltx2e}
% fixltx2e, the successor to the earlier fix2col.sty, was written by
% Frank Mittelbach and David Carlisle. This package corrects a few problems
% in the LaTeX2e kernel, the most notable of which is that in current
% LaTeX2e releases, the ordering of single and double column floats is not
% guaranteed to be preserved. Thus, an unpatched LaTeX2e can allow a
% single column figure to be placed prior to an earlier double column
% figure. The latest version and documentation can be found at:
% http://www.ctan.org/tex-archive/macros/latex/base/


%\usepackage{stfloats}
% stfloats.sty was written by Sigitas Tolusis. This package gives LaTeX2e
% the ability to do double column floats at the bottom of the page as well
% as the top. (e.g., "\begin{figure*}[!b]" is not normally possible in
% LaTeX2e). It also provides a command:
%\fnbelowfloat
% to enable the placement of footnotes below bottom floats (the standard
% LaTeX2e kernel puts them above bottom floats). This is an invasive package
% which rewrites many portions of the LaTeX2e float routines. It may not work
% with other packages that modify the LaTeX2e float routines. The latest
% version and documentation can be obtained at:
% http://www.ctan.org/tex-archive/macros/latex/contrib/sttools/
% Do not use the stfloats baselinefloat ability as IEEE does not allow
% \baselineskip to stretch. Authors submitting work to the IEEE should note
% that IEEE rarely uses double column equations and that authors should try
% to avoid such use. Do not be tempted to use the cuted.sty or midfloat.sty
% packages (also by Sigitas Tolusis) as IEEE does not format its papers in
% such ways.
% Do not attempt to use stfloats with fixltx2e as they are incompatible.
% Instead, use Morten Hogholm'a dblfloatfix which combines the features
% of both fixltx2e and stfloats:
%
% \usepackage{dblfloatfix}
% The latest version can be found at:
% http://www.ctan.org/tex-archive/macros/latex/contrib/dblfloatfix/




% *** PDF, URL AND HYPERLINK PACKAGES ***
%
%\usepackage{url}
% url.sty was written by Donald Arseneau. It provides better support for
% handling and breaking URLs. url.sty is already installed on most LaTeX
% systems. The latest version and documentation can be obtained at:
% http://www.ctan.org/tex-archive/macros/latex/contrib/url/
% Basically, \url{my_url_here}.




% *** Do not adjust lengths that control margins, column widths, etc. ***
% *** Do not use packages that alter fonts (such as pslatex).         ***
% There should be no need to do such things with IEEEtran.cls V1.6 and later.
% (Unless specifically asked to do so by the journal or conference you plan
% to submit to, of course. )

\usepackage[dvipsnames]{xcolor}
\usepackage{amsmath}
\usepackage{amssymb}
\usepackage{xspace}
\usepackage{graphicx}
\usepackage{latexsym}
\usepackage{listings}
\usepackage{multirow}
\usepackage{suffix}
\usepackage{url}
\usepackage{mathptmx}
\usepackage{mathrsfs}
\usepackage{comment}
\usepackage{enumerate}
\usepackage{txfonts}
\usepackage{hyperref}
%\usepackage{space}
\usepackage{color}      % use if color is used in text

% correct bad hyphenation here
\hyphenation{op-tical net-works semi-conduc-tor}


\begin{document}
%
% paper title
% Titles are generally capitalized except for words such as a, an, and, as,
% at, but, by, for, in, nor, of, on, or, the, to and up, which are usually
% not capitalized unless they are the first or last word of the title.
% Linebreaks \\ can be used within to get better formatting as desired.
% Do not put math or special symbols in the title.
\title{\LARGE Something Higher-Order Session Communications}

% author names and affiliations
% use a multiple column layout for up to three different
% affiliations
\author{
\IEEEauthorblockN{
Dimitrios Kouzapas
}
\IEEEauthorblockA{
Imperial College London	
}
\and
\IEEEauthorblockN{
Jorge A. P\'{e}rez
}
\IEEEauthorblockA{
University of Groningen
}
\and 
\IEEEauthorblockN{
Nobuko Yoshida
}
\IEEEauthorblockA{
Imperial College London	
}
}

\maketitle

%%%%%%%%%%%%%%%%%%%%%%%%%%%%%%%%%%%%%%%%%%%%%%%%%%%%%%%%%%%%%%%%%%%%%%%%%%%%%%%%%%%%%%%%%%%%%%%%%%%%
% Contents
% --------

% 1.  Formating
% 2.  Maths - Theorems
% 3.  The pi Calculus
% 4.  Session Syntax
% 5.  Subject Reduction
% 6.  Global Session Types
% 7.  Global Session Types Equivalence
% 8.  Projection
% 9.  Local Session Types
% 10. Behavioural Theory
% 11. Typed Transitions - Reductions
% 12. Typed Relations
% 13. Confluence Determinacy
% 14. Mapping
% 15. pi Constructs
% 16. LN Transform
% 17. General Types Processes Names Sessions ETC
% 18. newtheorem - newenvironment
% 19. Misc
%%%%%%%%%%%%%%%%%%%%%%%%%%%%%%%%%%%%%%%%%%%%%%%%%%%%%%%%%%%%%%%%%%%%%%%%%%%%%%%%%%%%%%%%%%%%%%%%%%%%


%%%%%%%%%%%%%%%%%%%%%%%%%%%%%%%%%%%%%%%%%%%%%%%%%%%%%%%%%%%%%%%%%%%%%%%%%%%%%%%%%%%%%%%%%%%%%%%%%%%%
%                                       FORMATING
%%%%%%%%%%%%%%%%%%%%%%%%%%%%%%%%%%%%%%%%%%%%%%%%%%%%%%%%%%%%%%%%%%%%%%%%%%%%%%%%%%%%%%%%%%%%%%%%%%%%

% Symbols
\newcommand{\semicolon}{:}
%\newcommand{\colon}{;}
\newcommand{\lrangle}[1]{\langle #1 \rangle}
\newcommand{\blrangle}[1]{\big\langle #1 \big\rangle}

%Tags
\newcommand{\parenthtext}[1]{(\textrm{\small #1})}
\newcommand{\brtext}[1]{[\textrm{\small #1}]}
\newcommand{\textinmath}[1]{\textrm{#1}}
\newcommand{\srule}[1]{\parenthtext{#1}}
\newcommand{\strule}[1]{\textrm{#1}}
\newcommand{\stypes}[1]{{\footnotesize \parenthtext{#1}}}
\newcommand{\ltsrule}[1]{{\footnotesize \lrangle{\textrm{#1}}}}
\newcommand{\eltsrule}[1]{{\footnotesize [\textrm{#1}]}}
\newcommand{\trule}[1]{{\footnotesize\brtext{#1}}}
\newcommand{\orule}[1]{{\scriptsize{\brtext{#1}}}}
\newcommand{\mrule}[1]{{\footnotesize{\parenthtext{#1}}}}

\newcommand{\iftag}{{\textrm{if }}}

% General
\newcommand{\noi}{\noindent}
\newcommand{\Hline}{\rule{\linewidth}{.5pt}}
\newcommand{\Hlinefig}{\rule{\linewidth}{.5pt}\vspace{-4mm}}
\newcommand{\myparagraph}[1]{\noindent{\textbf{#1}\ }}
\newcommand{\jparagraph}[1]{\paragraph{\textbf{#1}}}

%%%%%%%%%%%%%%%%%%%%%%%%%%%%%%%%%%%%%%%%%%%%%%%%%%%%%%%%%%%%%%%%%%%%%%%%%%%%%%%%%%%%%%%%%%%%%%%%%%%%
%                                       MATHS - THEOREMS
%%%%%%%%%%%%%%%%%%%%%%%%%%%%%%%%%%%%%%%%%%%%%%%%%%%%%%%%%%%%%%%%%%%%%%%%%%%%%%%%%%%%%%%%%%%%%%%%%%%%

%\newtheorem{notation}[definition]{Notation}

% BNF form
\newcommand{\bnfis}{\;\;::=\;\;}
\newcommand{\bnfbar}{\;\;\;|\;\;\;}
\newcommand{\sbnfbar}{\;\;|\;\;}

% Proof
\newcommand{\Case}[1]{\noi {\bf Case: }#1\\}
\newcommand{\proofend}{\qed}
%\newcommand{\proofend}{}
\newcommand{\Proof}{\noi {\bf Proof: }}

% Logic
\newcommand{\LogAnd}{\texttt{ and }}
\newcommand{\LogOr}{\texttt{ or }}

% Induction

\newcommand{\basic}{\noi {\bf Basic Step:}\\}
\newcommand{\inductive}{\noi {\bf Inductive Hypothesis:}\\}
\newcommand{\induction}{\noi {\bf Induction Step:}\\}

% Tree

\newcommand{\tree}[2]{
\ensuremath{\displaystyle
		\frac
		{
			%%\raisebox{0.0mm}{$\displaystyle{#1}$}
			#1
			%\vspace{0mm}
		}{
			%\vspace{2mm}
			#2
			%\raisebox{-0.4mm}{$\displaystyle{#2}$}
		}
	}
}


%\newcommand{\tree}[2]{
%\begin{prooftree}
%	#1
%	\justifies
%	#2
%\end{prooftree}
%}

\newcommand{\treeusing}[3]{
\begin{prooftree}
	#1
	\justifies
	#2
	\using
	#3
\end{prooftree}}

% Vectors
\newcommand{\vect}[1]{\tilde{#1}}
\newcommand{\mytilde}[1]{\widetilde{#1}}

% Functions - Set theory
\newcommand{\set}[1]{\{#1\}}
\newcommand{\es}{\emptyset}
\newcommand{\maxset}[1]{\max(#1)}
\newcommand{\setbar}{\ \ |\ \ }
\newcommand{\tuple}[2]{(#1, #2)}
\newcommand{\suchthat}{\cdot}
\newcommand{\powerset}[1]{\mathcal{P}(#1)}
\newcommand{\product}{\times}

\newcommand{\eval}{\downarrow}

\newcommand{\setsubtr}[2]{#1 \backslash #2}

\newcommand{\func}[2]{#1(#2)}
\newcommand{\dom}[1]{\mathtt{dom}(#1)}
\newcommand{\codom}[1]{\mathtt{codom}(#1)}

\newcommand{\funcbr}[2]{#1\lrangle{#2}}

\newcommand{\entails}{\text{implies}}


%%%%%%%%%%%%%%%%%%%%%%%%%%%%%%%%%%%%%%%%%%%%%%%%%%%%%%%%%%%%%%%%%%%%%%%%%%%%%%%%%%%%%%%%%%%%%%%%%%%%
%                                        pi - CALCULUS
%%%%%%%%%%%%%%%%%%%%%%%%%%%%%%%%%%%%%%%%%%%%%%%%%%%%%%%%%%%%%%%%%%%%%%%%%%%%%%%%%%%%%%%%%%%%%%%%%%%%

% Free-Bound notation
\newcommand{\freev}[1]{\lrangle{#1}}
\newcommand{\boundv}[1]{(#1)}

% General pi calculus Syntax
\newcommand{\send}[1]{\overline{#1}}
\newcommand{\ol}[1]{\overline{#1}}
\newcommand{\receive}[1]{#1.}
\newcommand{\inact}{\mathbf{0}}
\newcommand{\If}{\sessionfont{if}\ }
\newcommand{\Then}{\sessionfont{then}\ }
\newcommand{\Else}{\sessionfont{else}\ }
\newcommand{\ifthen}[2]{\If #1\ \Then #2\ }
\newcommand{\ifthenelse}[3]{\ifthen{#1}{#2} \Else #3}
\newcommand{\Par}{\;|\;}
\newcommand{\news}[1]{(\nu\, #1)}
\newcommand{\newsp}[2]{(\nu\, #1)(#2)}
\newcommand{\varp}[1]{#1}
%\newcommand{\rvar}[1]{\mathcal{#1}}
\newcommand{\rvar}[1]{#1}
%\newcommand{\rec}[2]{\mu #1. #2}
\newcommand{\recp}[2]{\mu \rvar{#1}. #2}

\newcommand{\Def}{\sessionfont{def}\ }

\newcommand{\defeq}{\stackrel{\Def}{=}}

\newcommand{\repl}{\ast\,}
\newcommand{\parcomp}[2]{\prod_{#1}{#2}}

% Free-Bound-Names sets
\newcommand{\bn}[1]{\mathtt{bn}(#1)}
\newcommand{\fn}[1]{\mathtt{fn}(#1)}
\newcommand{\ofn}[1]{\mathsf{ofn}(#1)}
\newcommand{\fv}[1]{\mathtt{fv}(#1)}
\newcommand{\bv}[1]{\mathtt{bv}(#1)}
\newcommand{\fs}[1]{\mathtt{fs}(#1)}
\newcommand{\fpv}[1]{\mathtt{fpv}(#1)}
\newcommand{\nam}[1]{\mathtt{n}(#1)}

%Subject - Object
\newcommand{\subj}[1]{\mathtt{subj}(#1)}
\newcommand{\obj}[1]{\mathtt{obj}(#1)}

% Relations
\newcommand{\relfont}[1]{\mathcal{#1}}
\newcommand{\rel}[3]{#1\ \relfont{#2}\ #3}

\newcommand{\scong}{\equiv}
\newcommand{\acong}{\scong_{\alpha}}
\newcommand{\wb}{\approx}
\newcommand{\fwb}{\approx^C}
\newcommand{\hwb}{\approx^H}
\newcommand{\swb}{\approx^{s}}
\newcommand{\wbc}{\approx}
\newcommand{\WB}{\approx}

\newcommand{\red}{\longrightarrow}
\newcommand{\Red}{\rightarrow\!\!\!\!\!\rightarrow}
\newcommand{\Redleft}{\leftarrow\!\!\!\!\!\leftarrow}

%\newcommand{\subst}[2]{\set{#1/#2 }}
\def\subst#1#2{\{\raisebox{.5ex}{\small$#1$}\! / \mbox{\small$#2$}\}}

% Context
\newcommand{\hole}{-}
\newcommand{\context}[2]{#1[#2]}
\newcommand{\Ccontext}[1]{\C[#1]}

% Expression Context
\newcommand{\Econtext}[1]{\E[#1]}

% Barbs
\newcommand{\barb}[1]{\downarrow_{#1}}
\newcommand{\Barb}[1]{\Downarrow_{#1}}
\newcommand{\nbarb}[1]{\not\downarrow_{#1}}
\newcommand{\nBarb}[1]{\not\Downarrow_{#1}}

% General
%\newcommand{\ESP}{\ensuremath{\mathbf{ESP}}}
\newcommand{\ESP}{\text{ESP}}
\newcommand{\ESPsel}{\ESP^+}

%%%%%%%%%%%%%%%%%%%%%%%%%%%%%%%%%%%%%%%%%%%%%%%%%%%%%%%%%%%%%%%%%%%%%%%%%%%%%%%%%%%%%%%%%%%%%%%%%%%%
%                                        SESSION SYNTAX
%%%%%%%%%%%%%%%%%%%%%%%%%%%%%%%%%%%%%%%%%%%%%%%%%%%%%%%%%%%%%%%%%%%%%%%%%%%%%%%%%%%%%%%%%%%%%%%%%%%%

% Session font
\newcommand{\sessionfont}[1]{\mathtt{#1}}
\newcommand{\vart}[1]{\mathsf{#1}}

% General Session symbols
\newcommand{\ssep}{;}
\newcommand{\shsep}{.}
\newcommand{\outses}{!}
\newcommand{\inpses}{?}
\newcommand{\selses}{\triangleleft}
\newcommand{\brases}{\triangleright}
\newcommand{\dual}[1]{\overline{#1}}
\newcommand{\cat}{\cdot}

\newcommand{\allstypes}{\mathcal{S}}

% Binary Session Syntax


\newcommand{\bacc}[2]{#1 \boundv{#2} \shsep}
\newcommand{\breq}[2]{\send{#1} \freev{#2} \shsep}
\newcommand{\bareq}[2]{\send{#1} \freev{#2}}

\newcommand{\breqt}[3]{\send{#1} \boundv{#2:#3} \shsep}
\newcommand{\bacct}[3]{#1 \boundv{#2:#3} \shsep}

\newcommand{\bout}[2]{#1 \outses \freev{#2} \shsep}
\newcommand{\bbout}[2]{#1 \outses \blrangle{#2} \shsep}
\newcommand{\binp}[2]{#1 \inpses \boundv{#2} \shsep}
\newcommand{\bsel}[2]{#1 \selses #2 \shsep}

%\newcommand{\bout}[2]{#1 \outses \freev{#2} \ssep}
%\newcommand{\bbout}[2]{#1 \outses \blrangle{#2} \ssep}
%\newcommand{\binp}[2]{#1 \inpses \boundv{#2} \ssep}
%\newcommand{\bsel}[2]{#1 \selses #2 \ssep}
\newcommand{\bbra}[2]{#1 \brases \set{#2}}
\newcommand{\bbras}[2]{#1 \brases #2}
\newcommand{\bbraP}[1]{#1 \brases \lPi}

% Multiparty Session syntax

\newcommand{\role}[1]{[#1]}

\newcommand{\srole}[2]{#1\role{#2}}
\newcommand{\sqrole}[2]{#1^{[]}\role{#2}}

\newcommand{\fromto}[2]{\role{#1} \role{#2}}
\newcommand{\sfromto}[3]{#1\fromto{#2}{#3}}

\newcommand{\sout}[3]{\srole{#1}{#2} \outses \freev{#3} \ssep}
\newcommand{\sinp}[3]{\srole{#1}{#2} \inpses \boundv{#3} \ssep}
\newcommand{\sdel}[4]{\srole{#1}{#2} \outses \freev{\srole{#3}{#4}} \ssep}
\newcommand{\ssel}[3]{\srole{#1}{#2} \selses #3 \ssep}
\newcommand{\sbra}[3]{\srole{#1}{#2} \brases \set{#3}}
\newcommand{\sbras}[3]{\srole{#1}{#2} \brases #3}
\newcommand{\sbraP}[2]{\srole{#1}{#2} \brases \lPi}

\newcommand{\acc}[3]{#1 \role{#2} \boundv{#3} \shsep}
\newcommand{\req}[3]{\send{#1} \role{#2} \boundv{#3} \shsep}
\newcommand{\areq}[3]{\send{#1} \role{#2} \freev{#3}}

\newcommand{\out}[4]{\sfromto{#1}{#2}{#3} \outses \freev{#4} \ssep}
\newcommand{\inp}[4]{\sfromto{#1}{#2}{#3} \inpses \boundv{#4} \ssep}
\newcommand{\del}[5]{\sfromto{#1}{#2}{#3} \outses \freev{\srole{#4}{#5}} \ssep}
\newcommand{\sel}[4]{\sfromto{#1}{#2}{#3} \selses #4 \ssep}
\newcommand{\bra}[4]{\sfromto{#1}{#2}{#3} \brases \set{#4}}
\newcommand{\bras}[4]{\sfromto{#1}{#2}{#3} \brases #4}
\newcommand{\braP}[3]{\sfromto{#1}{#2}{#3} \brases \lPi}

% Arrive construct
\newcommand{\arrivetext}{\mathtt{arrive}}
\newcommand{\arrive}[1]{\arrivetext\ #1}
\newcommand{\arrivem}[2]{\arrivetext\ #1\ #2}

% Typecase construct
\newcommand{\typecasetext}{\mathtt{typecase}}
\newcommand{\oftext}{\mathtt{of}}
\newcommand{\typecase}[2]{\typecasetext\ #1\ \oftext\ \set{#2}}

% IO symbols
\newcommand{\inputsym}{\mathtt{i}}
\newcommand{\outputsym}{\mathtt{o}}

%%%%%%%%%%%%%%%%%%%%%%%%%%%%%%%%%%%%%%%%%%%%%%%%%%%%%%%%%%%%%%%%%%%%%%%%%%%%%%%%%%%%%%%%%%%%%%%%%%%%
%                                      SUBJECT REDUCTION
%%%%%%%%%%%%%%%%%%%%%%%%%%%%%%%%%%%%%%%%%%%%%%%%%%%%%%%%%%%%%%%%%%%%%%%%%%%%%%%%%%%%%%%%%%%%%%%%%%%%

% typing reduction
\newcommand{\typingred}{\red}
\newcommand{\typingRed}{\Red}

\newcommand{\wellconf}[1]{\mathtt{wc}(#1)}
\newcommand{\cohses}[2]{\mathtt{co}(#1(#2))}
\newcommand{\coherent}[1]{\mathtt{co}(#1)}
\newcommand{\fcoherent}[1]{\mathtt{fco}(#1)}

%%%%%%%%%%%%%%%%%%%%%%%%%%%%%%%%%%%%%%%%%%%%%%%%%%%%%%%%%%%%%%%%%%%%%%%%%%%%%%%%%%%%%%%%%%%%%%%%%%%%
%                                      SESSION ENDPOINTS
%%%%%%%%%%%%%%%%%%%%%%%%%%%%%%%%%%%%%%%%%%%%%%%%%%%%%%%%%%%%%%%%%%%%%%%%%%%%%%%%%%%%%%%%%%%%%%%%%%%%

% Asynchronous syntax
%\newcommand{\mareq}[4]{\newsp{\srole{#2}{#3}, \dots, \srole{#2}{#4}}{\send{#1}[#3] \freev{#2} \Par \dots \Par \send{#1}[#4]\freev{#2}}}

%\newcommand{\areqs}[3]{\send{#1}[\set{#2}] \freev{#3}}

% Queues
\newcommand{\emp}{\epsilon}
\newcommand{\squeue}[3]{\srole{#1}{#2}:#3}
\newcommand{\srqueue}[4]{\srole{#1}{#2}[\inputsym: #3, \outputsym: #4]}
\newcommand{\srqueuei}[3]{\srole{#1}{#2}[\inputsym: #3]}
\newcommand{\srqueueo}[3]{\srole{#1}{#2}[\outputsym: #3]}
\newcommand{\srqueueio}[4]{\srole{#1}{#2}[\inputsym: #3, \outputsym: #4]}
\newcommand{\sgqueue}[2]{\srole{#1}:#2}

% Shared Names Queues
\newcommand{\shqueue}[2]{#1[#2]}
%\newcommand{\shqueuet}[3]{#1[#2, #3]}

% IO Queues
\newcommand{\squeueio}[3]{#1[\inputsym: #2, \outputsym: #3]}
\newcommand{\squeuei}[2]{#1[\inputsym: #2]}
\newcommand{\squeueo}[2]{#1[\outputsym: #2]}

\newcommand{\squeuetio}[4]{#1[#2, \inputsym: #3, \outputsym: #4]}
\newcommand{\squeueto}[3]{#1[#2, \outputsym: #3]}
\newcommand{\squeueti}[3]{#1[#2, \inputsym: #3]}
\newcommand{\squeuet}[2]{#1[#2]}

% Queue Values

\newcommand{\queuev}[2]{\role{#1}(#2)}
\newcommand{\queuel}[2]{\role{#1} #2}
\newcommand{\queues}[3]{\role{#1}(\srole{#2}{#3})}

%%%%%%%%%%%%%%%%%%%%%%%%%%%%%%%%%%%%%%%%%%%%%%%%%%%%%%%%%%%%%%%%%%%%%%%%%%%%%%%%%%%%%%%%%%%%%%%%%%%%
%                                        GLOBAL SESSION TYPES
%%%%%%%%%%%%%%%%%%%%%%%%%%%%%%%%%%%%%%%%%%%%%%%%%%%%%%%%%%%%%%%%%%%%%%%%%%%%%%%%%%%%%%%%%%%%%%%%%%%%

\newcommand{\gtfont}[1]{\mathtt{#1}}
\newcommand{\gsep}{.}

\newcommand{\globaltype}[1]{\lrangle{#1}}
\newcommand{\parties}[1]{\mathtt{\p}(#1)}
\newcommand{\roles}[1]{\mathtt{roles}(#1)}

\newcommand{\fromtogt}[2]{#1 \rightarrow #2 \semicolon}

\newcommand{\valuegt}[3]{\fromtogt{#1}{#2} \lrangle{#3} \gsep}
\newcommand{\selgt}[3]{\fromtogt{#1}{#2} \set{#3}}
\newcommand{\selgtG}[2]{\fromtogt{#1}{#2} \lGi}
\newcommand{\recgt}[2]{\mu \vart{#1}. #2}
\newcommand{\vargt}[1]{\vart{#1}}
\newcommand{\inactgt}{\gtfont{end}}

%%%%%%%%%%%%%%%%%%%%%%%%%%%%%%%%%%%%%%%%%%%%%%%%%%%%%%%%%%%%%%%%%%%%%%%%%%%%%%%%%%%%%%%%%%%%%%%%%%%%
%                              GLOBAL SESSION TYPES EQUIVALENCE
%%%%%%%%%%%%%%%%%%%%%%%%%%%%%%%%%%%%%%%%%%%%%%%%%%%%%%%%%%%%%%%%%%%%%%%%%%%%%%%%%%%%%%%%%%%%%%%%%%%%

\newcommand{\projset}[1]{\mathtt{proj}(#1)}
\newcommand{\aprojset}[1]{\mathtt{aproj}\ #1 }
\newcommand{\gcong}{\equiv}
\newcommand{\govcong}{\cong_g}
\newcommand{\gperm}{\simeq}

%%%%%%%%%%%%%%%%%%%%%%%%%%%%%%%%%%%%%%%%%%%%%%%%%%%%%%%%%%%%%%%%%%%%%%%%%%%%%%%%%%%%%%%%%%%%%%%%%%%%
%                                        PROJECTION
%%%%%%%%%%%%%%%%%%%%%%%%%%%%%%%%%%%%%%%%%%%%%%%%%%%%%%%%%%%%%%%%%%%%%%%%%%%%%%%%%%%%%%%%%%%%%%%%%%%%

\newcommand{\projsymb}{\lceil}
\newcommand{\proj}[2]{#1 \projsymb #2}

%%%%%%%%%%%%%%%%%%%%%%%%%%%%%%%%%%%%%%%%%%%%%%%%%%%%%%%%%%%%%%%%%%%%%%%%%%%%%%%%%%%%%%%%%%%%%%%%%%%%
%                                        LOCAL SESSION TYPES
%%%%%%%%%%%%%%%%%%%%%%%%%%%%%%%%%%%%%%%%%%%%%%%%%%%%%%%%%%%%%%%%%%%%%%%%%%%%%%%%%%%%%%%%%%%%%%%%%%%%

\newcommand{\tfont}[1]{\mathtt{#1}}
\newcommand{\tsep}{;}

\newcommand{\chtype}[1]{\lrangle{#1}}
\newcommand{\chtypei}[1]{\inputsym \lrangle{#1}}
\newcommand{\chtypeo}[1]{\outputsym \lrangle{#1}}
\newcommand{\chtypeio}[1]{\inputsym \outputsym \lrangle{#1}}

\newcommand{\outtype}{\outses}
\newcommand{\inptype}{\inpses}
\newcommand{\seltype}{\selses}
\newcommand{\bratype}{\brases}

\newcommand{\trec}[2]{\mu\vart{#1}.#2}
\newcommand{\tvar}[1]{\vart{#1}}
%\newcommand{\settype}[1]{\set{#1}}
\newcommand{\tset}[1]{\set{#1}}
\newcommand{\tinact}{\tfont{end}}

%\newcommand{\sminus}[1]{#1^-}
\newcommand{\sminus}[1]{#1^{\text{--}}}

\newcommand{\subt}{\leq}
\newcommand{\supt}{\geq}

% Multiparty Local Session Types
\newcommand{\tout}[2]{\role{#1} \outtype \lrangle{#2} \tsep}
\newcommand{\tinp}[2]{\role{#1} \inptype (#2) \tsep}
\newcommand{\tsel}[2]{\role{#1} \seltype \set{#2}}
\newcommand{\tsels}[2]{\role{#1} \seltype #2}
\newcommand{\tselT}[1]{\role{#1} \seltype \lTi}
\newcommand{\tbra}[2]{\role{#1} \bratype \set{#2}}
\newcommand{\tbras}[2]{\role{#1} \bratype #2}
\newcommand{\tbraT}[1]{\role{#1} \bratype \lTi}

% Binary Session Types
\newcommand{\btout}[1]{\outtype \lrangle{#1} \tsep}
\newcommand{\bbtout}[1]{\outtype \big\langle{#1}\big\rangle \tsep}
\newcommand{\btinp}[1]{\inptype (#1) \tsep}
\newcommand{\bbtinp}[1]{\inptype \big({#1}\big) \tsep}
\newcommand{\btsel}[1]{\oplus \set{#1}}
\newcommand{\btselS}{\oplus \lSi}
\newcommand{\btbra}[1]{\& \set{#1}}
\newcommand{\btbraS}{\& \lSi}

% Queue Typing

\newcommand{\mtout}[2]{\role{#1} \outtype \lrangle{#2}}
\newcommand{\mtinp}[2]{\role{#1} \inptype (#2)}
\newcommand{\mtsel}[2]{\role{#1} \seltype #2}
\newcommand{\mtbra}[2]{\role{#1} \bratype #2}


% Binary Queue Typing

\newcommand{\bmtout}[1]{\outtype \lrangle{#1}}
\newcommand{\bmtinp}[1]{\inptype (#1)}
\newcommand{\bmtsel}[1]{\seltype #1}
\newcommand{\bmtbra}[1]{\bratype #1}

% Message concatanation
\newcommand{\mcat}{\;*\;}
\newcommand{\icat}{\;\circ\;}


%%%%%%%%%%%%%%%%%%%%%%%%%%%%%%%%%%%%%%%%%%%%%%%%%%%%%%%%%%%%%%%%%%%%%%%%%%%%%%%%%%%%%%%%%%%%%%%%%%%%
%                                        TYPED PROCESSES
%%%%%%%%%%%%%%%%%%%%%%%%%%%%%%%%%%%%%%%%%%%%%%%%%%%%%%%%%%%%%%%%%%%%%%%%%%%%%%%%%%%%%%%%%%%%%%%%%%%%

\newcommand{\Ga}{\Gamma}
\newcommand{\De}{\Delta}
\newcommand{\proves}{\vdash}
\newcommand{\hastype}{\triangleright}

\newcommand{\Decat}[1]{\De \cat #1}
\newcommand{\Gacat}[1]{\Ga \cat #1}

\newcommand{\tcat}{\circ}

\newcommand{\typed}[1]{#1:}
\newcommand{\typedrole}[2]{\typed{\srole{#1}{#2}}}
%\newcommand{\typedqrole}[2]{\typed{\srole{#1^{[]}}{#2}}}

\newcommand{\typedprocess}[3]{#1 \proves #2 \hastype #3}

\newcommand{\Eproves}[3]{#1 \proves \typed{#2} #3}
\newcommand{\Gproves}[2]{\Eproves{\Ga}{#1}{#2}}

\newcommand{\tprocess}[3]{#1 \proves #2 \hastype #3}
\newcommand{\Gtprocess}[2]{\tprocess{\Ga}{#1}{#2}}
\newcommand{\Gptprocess}[2]{\tprocess{\Ga'}{#1}{#2}}

\newcommand{\noGtprocess}[2]{#1 \hastype #2}

%%%%%%%%%%%%%%%%%%%%%%%%%%%%%%%%%%%%%%%%%%%%%%%%%%%%%%%%%%%%%%%%%%%%%%%%%%%%%%%%%%%%%%%%%%%%%%%%%%%%
%                                    MULTIPARTY TYPED THEORY
%%%%%%%%%%%%%%%%%%%%%%%%%%%%%%%%%%%%%%%%%%%%%%%%%%%%%%%%%%%%%%%%%%%%%%%%%%%%%%%%%%%%%%%%%%%%%%%%%%%%

\newcommand{\globalenv}[1]{\set{#1}}
%\newcommand{\globalenvI}{\set{\typed{s_i} \G_i}_{i \in I}}
\newcommand{\globalenvI}{E}
\newcommand{\globalenvJ}{\set{\typed{s_j} \G_j}_{j \in J}}

\newcommand{\Gltprocess}[4]{\tprocess{#1}{#2}{#3, #4}}

\newcommand{\geI}{\globalenvI}
\newcommand{\geJ}{\globalenvJ}

\newcommand{\Stprocess}[3]{\tprocess{\geI, #1}{#2}{#3}}
\newcommand{\SGtprocess}[2]{\Gtprocess{#1}{#2, \globalenvI}}
\newcommand{\SJGtprocess}[2]{\globalenvJ, \Gtprocess{#1}{#2}}

\newcommand{\Observer}[2]{\mathsf{Observer}(#1, #2)}
\newcommand{\ObserverG}[1]{\mathsf{Observer}(\globalenvI, #1)}

\newcommand{\Obs}{\ensuremath{\mathsf{Obs}}}

%%%%%%%%%%%%%%%%%%%%%%%%%%%%%%%%%%%%%%%%%%%%%%%%%%%%%%%%%%%%%%%%%%%%%%%%%%%%%%%%%%%%%%%%%%%%%%%%%%%%
%                                        BEHAVIOURAL THEORY
%%%%%%%%%%%%%%%%%%%%%%%%%%%%%%%%%%%%%%%%%%%%%%%%%%%%%%%%%%%%%%%%%%%%%%%%%%%%%%%%%%%%%%%%%%%%%%%%%%%%

\newcommand{\fromtolts}[2]{\fromto{#1}{#2}}

\newcommand{\outlts}{\outses}
\newcommand{\inplts}{\inpses}
\newcommand{\sellts}{\oplus}
\newcommand{\bralts}{\&}

% Multiparty Labels
\newcommand{\actreq}[3]{\send{#1} \role{#2} \boundv{#3}}
%\newcommand{\actbreq}[3]{\send{#1} \role{#2} \boundv{#3}}

\newcommand{\actreqs}[3]{\send{#1} \role{\set{#2}} \boundv{#3}}
%\newcommand{\actbreqs}[3]{\send{#1} \role{\set{#2}} \boundv{#3}}

\newcommand{\actacc}[3]{#1 \role{#2} \boundv{#3}}
\newcommand{\actaccs}[3]{#1 \role{\set{#2}} \boundv{#3}}

\newcommand{\actout}[4]{#1 \fromtolts{#2}{#3} \outlts \freev{#4}}
\newcommand{\actqout}[4]{#1^{[]} \fromtolts{#2}{#3} \outlts \freev{#4}}
\newcommand{\actbout}[4]{#1 \fromtolts{#2}{#3} \outlts \boundv{#4}}

\newcommand{\actdel}[5]{#1 \fromtolts{#2}{#3} \outlts \freev{\srole{#4}{#5}}}
\newcommand{\actbdel}[5]{#1 \fromtolts{#2}{#3} \outlts \boundv{\srole{#4}{#5}}}
\newcommand{\actqdel}[5]{#1^{[]} \fromtolts{#2}{#3} \outlts \boundv{\srole{#4}{#5}}}

\newcommand{\actinp}[4]{#1 \fromtolts{#2}{#3} \inplts \freev{#4}}
\newcommand{\actqinp}[4]{#1^{[]} \fromtolts{#2}{#3} \inplts \freev{#4}}

\newcommand{\actsel}[4]{#1 \fromtolts{#2}{#3} \sellts #4}
\newcommand{\actqsel}[4]{#1^{[]} \fromtolts{#2}{#3} \sellts #4}

\newcommand{\actbra}[4]{#1 \fromtolts{#2}{#3} \bralts #4}
\newcommand{\actqbra}[4]{#1^{[]} \fromtolts{#2}{#3} \bralts #4}

\newcommand{\actval}[4]{#1: #2 \rightarrow #3:#4}
\newcommand{\actgsel}[4]{#1: #2 \rightarrow #3:#4}

% Binary Labels
\newcommand{\bactreq}[2]{\send{#1} \freev{#2}}
\newcommand{\bactbreq}[2]{\send{#1} \boundv{#2}}
\newcommand{\bactacc}[2]{#1 \freev{#2}}

\newcommand{\bactout}[2]{#1 \outlts \freev{#2}}
\newcommand{\bactbout}[2]{#1\outlts \boundv{#2}}
\newcommand{\bactinp}[2]{#1 \inplts \freev{#2}}
\newcommand{\bactsel}[2]{#1 \sellts #2}
\newcommand{\bactbra}[2]{#1 \bralts #2}

% Labelled transition relations
\newcommand{\by}[1]{\stackrel{#1}{\longrightarrow}}
\newcommand{\By}[1]{\stackrel{#1}{\Longrightarrow}}

\newcommand{\hby}[1]{\stackrel{#1}{\longmapsto}}
\newcommand{\Hby}[1]{\stackrel{#1}{\Longmapsto}}


% Session barbs
\newcommand{\barbreq}[1]{\barb{#1}}
%\newcommand{\barbacc}[2]{\barb{#1\role{\set{#2}}}}
\newcommand{\barbout}[3]{\barb{\sfromto{#1}{#2}{#3}}}
%\newcommand{\barbinp}[3]{\barb{#1\fromto{#2}{#3}\inpses}}

\newcommand{\Barbreq}[2]{\Barb{\send{#1}\role{#2}}}
%\newcommand{\Barbacc}[2]{\Barb{#1\role{\set{#2}}}}
%\newcommand{\Barbout}[3]{\Barb{\sfromto{#1}{#2}{#3}\outses}}
\newcommand{\Barbout}[3]{\Barb{\sfromto{#1}{#2}{#3}}}
%\newcommand{\Barbinp}[3]{\Barb{#1\fromto{#2}{#3}\inpses}}

% Binary Session barbs
\newcommand{\bbarbreq}[1]{\barb{#1}}
%\newcommand{\bbarbacc}[1]{\barb{#1}}
\newcommand{\bbarbout}[1]{\barb{#1}}
%\newcommand{\bbarbinp}[1]{\barb{#1\inpses}}

\newcommand{\bBarbreq}[1]{\Barb{\send{#1}}}
%\newcommand{\bBarbacc}[2]{\Barb{#1}}
\newcommand{\bBarbout}[1]{\Barb{#1\outses}}
%\newcommand{\bBarbinp}[1]{\Barb{#1\inpses}}

\newcommand{\comp}{\asymp}
\newcommand{\coh}{\asymp}
\newcommand{\bistyp}{\rightleftharpoons}

\newcommand{\typingbeh}{\leftrightarrow}

\newcommand{\bufrel}{\succ}

\newcommand{\ordercup}{\bowtie}

%%%%%%%%%%%%%%%%%%%%%%%%%%%%%%%%%%%%%%%%%%%%%%%%%%%%%%%%%%%%%%%%%%%%%%%%%%%%%%%%%%%%%%%%%%%%%%%%%%%%
%                                TYPED TRANSITIONS - REDUCTIONS
%%%%%%%%%%%%%%%%%%%%%%%%%%%%%%%%%%%%%%%%%%%%%%%%%%%%%%%%%%%%%%%%%%%%%%%%%%%%%%%%%%%%%%%%%%%%%%%%%%%%

% Environment Transitions
\newcommand{\envtrans}[1]{\by{#1}}
\newcommand{\envTrans}[1]{\by{#1}}
\newcommand{\typedtrans}[1]{\by{#1}}
\newcommand{\typedTrans}[1]{\By{#1}}
\newcommand{\typedred}{\red}
\newcommand{\typedRed}{\Red}

% Typed Environment Transitions - Binary case
\newcommand{\benv}[2]{(#1, #2)}
\newcommand{\bGenv}[1]{\benv{\Ga}{#1}}
\newcommand{\bGDenv}{\envtyp{\Ga}{\De}}

\newcommand{\envby}[5]{\benv{#1}{#2} \envtrans{#3} \benv{#4}{#5}}
\newcommand{\Genvby}[3]{\envby{\Ga}{#1}{#2}{\Ga}{#3}}

% Typed Environment Transitions - Multiparty case

\newcommand{\env}[3]{(#1, #2, #3)}
\newcommand{\Genv}[2]{\env{\globalenvI}{#1}{#2}}
\newcommand{\GGenv}[1]{\env{\globalenvI}{\Ga}{#1}}
\newcommand{\GGDenv}{\env{\globalenvI}{\Ga}{\De}}

% Typed Process Transitions - Binary case

\newcommand{\ftby}[7]{\tprocess{#1}{#2}{#3} \typedtrans{#4} \tprocess{#5}{#6}{#7}}
\newcommand{\ftBy}[7]{\tprocess{#1}{#2}{#3} \typedTrans{#4} \tprocess{#5}{#6}{#7}}

\newcommand{\tpby}[6]{\tprocess{#1}{#2}{#3} \typedtrans{#4} \noGtprocess{#5}{#6}}
\newcommand{\tpBy}[6]{\tprocess{#1}{#2}{#3} \typedTrans{#4} \noGtprocess{#5}{#6}}
\newcommand{\Gtpby}[5]{\Gtprocess{#1}{#2} \typedtrans{#3} \noGtprocess{#4}{#5}}
\newcommand{\GtpBy}[5]{\Gtprocess{#1}{#2} \typedTrans{#3} \noGtprocess{#4}{#5}}

\newcommand{\GGtpby}[5]{\tprocess{\globalenvI, \Ga}{#1}{#2} \typedtrans{#3} \noGtprocess{#4}{#5}}
\newcommand{\GGtpBy}[5]{\tprocess{\globalenvI, \Ga}{#1}{#2} \typedTrans{#3} \noGtprocess{#4}{#5}}

% Typed Reductions - Binary case

\newcommand{\ftpred}[6]{\tprocess{#1}{#2}{#3} \typedred \tprocess{#4}{#5}{#6}}
\newcommand{\ftpRed}[6]{\tprocess{#1}{#2}{#3} \typedRed \tprocess{#4}{#5}{#6}}

\newcommand{\tpred}[5]{\tprocess{#1}{#2}{#3} \typedred \noGtprocess{#4}{#5}}
\newcommand{\tpRed}[5]{\tprocess{#1}{#2}{#3} \typedRed \noGtprocess{#4}{#5}}
\newcommand{\Gtpred}[4]{\Gtprocess{#1}{#2} \typedred \noGtprocess{#3}{#4}}
\newcommand{\GtpRed}[4]{\Gtprocess{#1}{#2} \typedRed \noGtprocess{#3}{#4}}

% Observer Reductions

\newcommand{\obsred}{\red_{obs}}
\newcommand{\obsRed}{\Red_{obs}}


%%%%%%%%%%%%%%%%%%%%%%%%%%%%%%%%%%%%%%%%%%%%%%%%%%%%%%%%%%%%%%%%%%%%%%%%%%%%%%%%%%%%%%%%%%%%%%%%%%%%
%                                    TYPED RELATIONS
%%%%%%%%%%%%%%%%%%%%%%%%%%%%%%%%%%%%%%%%%%%%%%%%%%%%%%%%%%%%%%%%%%%%%%%%%%%%%%%%%%%%%%%%%%%%%%%%%%%%

% Typed Relations
\newcommand{\fulltrel}[7]{\rel{\typedprocess{#1}{#2}{#3}}{#4}{\typedprocess{#5}{#6}{#7}}}
\newcommand{\treld}[6]{\rel{\typedprocess{#1}{#2}{#3}}{#4}{\noGtypedprocess{#5}{#6}}}
\newcommand{\trel}[5]{\rel{#1 \proves #2}{#3}{\noGtypedprocess{#4}{#5}}}

\newcommand{\tcong}{\cong}
\newcommand{\twb}{\approx}
\newcommand{\govwb}{\approx_g}
\newcommand{\tequiv}{\approx}

%%%%%%%%%%%%%%%%%%%%%%%%%%%%%%%%%%%%%%%%%%%%%%%%%%%%%%%%%%%%%%%%%%%%%%%%%%%%%%%%%%%%%%%%%%%%%%%%%%%%
%                                    CONFIGURATION THEORY
%%%%%%%%%%%%%%%%%%%%%%%%%%%%%%%%%%%%%%%%%%%%%%%%%%%%%%%%%%%%%%%%%%%%%%%%%%%%%%%%%%%%%%%%%%%%%%%%%%%%

\newcommand{\confpair}[2]{(#1, #2)}
\newcommand{\uptoconfpair}[2]{[#1, #2]}


%%%%%%%%%%%%%%%%%%%%%%%%%%%%%%%%%%%%%%%%%%%%%%%%%%%%%%%%%%%%%%%%%%%%%%%%%%%%%%%%%%%%%%%%%%%%%%%%%%%%
%                                   CONFLUENCE DETERMINACY
%%%%%%%%%%%%%%%%%%%%%%%%%%%%%%%%%%%%%%%%%%%%%%%%%%%%%%%%%%%%%%%%%%%%%%%%%%%%%%%%%%%%%%%%%%%%%%%%%%%%

\newcommand{\sesstrans}[1]{\stackrel{#1}{\longrightarrow_{s}}}
\newcommand{\sessTrans}[1]{\stackrel{#1}{\Longrightarrow_{s}}}

\newcommand{\fulltypedsesstrans}[7]{\typedprocess{#1}{#2}{#3} \sesstrans{#4} \typedprocess{#5}{#6}{#7}}
\newcommand{\fulltypedsessTrans}[7]{\typedprocess{#1}{#2}{#3} \sessTrans{#4} \typedprocess{#5}{#6}{#7}}

\newcommand{\typedsesstrans}[6]{\typedprocess{#1}{#2}{#3} \sesstrans{#4} \noGtypedprocess{#5}{#6}}
\newcommand{\typedsessTrans}[6]{\typedprocess{#1}{#2}{#3} \sessTrans{#4} \noGtypedprocess{#5}{#6}}
\newcommand{\Gtypedsesstrans}[5]{\Gtypedprocess{#1}{#2} \sesstrans{#3} \noGtypedprocess{#4}{#5}}
\newcommand{\GtypedsessTrans}[5]{\Gtypedprocess{#1}{#2} \sessTrans{#3} \noGtypedprocess{#4}{#5}}

% Actions
\newcommand{\confact}[2]{#1 \lfloor #2}

%%%%%%%%%%%%%%%%%%%%%%%%%%%%%%%%%%%%%%%%%%%%%%%%%%%%%%%%%%%%%%%%%%%%%%%%%%%%%%%%%%%%%%%%%%%%%%%%%%%%
%                                    MAPPING AND ENCODINGS
%%%%%%%%%%%%%%%%%%%%%%%%%%%%%%%%%%%%%%%%%%%%%%%%%%%%%%%%%%%%%%%%%%%%%%%%%%%%%%%%%%%%%%%%%%%%%%%%%%%%

\newcommand{\map}[1]{[\!\![#1]\!\!]}
\newcommand{\umap}[1]{[\!\![#1]\!\!]^u}
\newcommand{\pmap}[2]{\ensuremath{[\!\![#1]\!\!]^#2}}
\newcommand{\pmapp}[3]{\ensuremath{[\!\![#1]\!\!]^#2_#3}}
\newcommand{\auxmap}[2]{\ensuremath{\{\!\{#1\}\!\}^#2}}
\newcommand{\tauxmap}[2]{\ensuremath{\{\!|#1|\!\}^#2}}
\newcommand{\auxmapp}[3]{\ensuremath{\big\lfloor\!\!\big\lfloor#1\big\rfloor\!\!\big\rfloor^#2_#3}}
\newcommand{\tmap}[2]{\ensuremath{(\!\!\langle#1\rangle\!\!)^{#2}}}
\newcommand{\vtmap}[2]{{\ensuremath{\big\lfloor #1\big\rfloor^{#2}}}}
\newcommand{\mapt}[1]{\ensuremath{(\!\!\langle#1\rangle\!\!)}}
\newcommand{\mapa}[1]{\ensuremath{\{\!\!\{#1\}\!\!\}}}
\newcommand{\namemap}[2]{#1\map{#2}}

\newcommand{\enc}[2]{\big\langle\map{#1}, \mapt{#2}\big\rangle}
\newcommand{\enco}[1]{\big\langle #1\big\rangle}
\newcommand{\encod}[3]{\lrangle{\map{#1}^{#3}, \mapt{#2}^{#3}}}
\newcommand{\fencod}[4]{\lrangle{\map{#1}^{#3}_{#4} \, , \, \mapt{#2}^{#3}}}

\newcommand{\calc}[5]{\lrangle{#1, #2, #3, #4, #5}}
\newcommand{\tyl}[1]{\ensuremath{\mathcal{#1}}}

%%%%%%%%%%%%%%%%%%%%%%%%%%%%%%%%%%%%%%%%%%%%%%%%%%%%%%%%%%%%%%%%%%%%%%%%%%%%%%%%%%%%%%%%%%%%%%%%%%%%
%                                    PI CONSTRUCTS
%%%%%%%%%%%%%%%%%%%%%%%%%%%%%%%%%%%%%%%%%%%%%%%%%%%%%%%%%%%%%%%%%%%%%%%%%%%%%%%%%%%%%%%%%%%%%%%%%%%%

\newcommand{\constrtype}[1]{\mathtt{#1}}


\newcommand{\Let}{\constrtype{let}\ }
\newcommand{\In}{\constrtype{in}\ }
\newcommand{\To}{\constrtype{to}\ }
\newcommand{\new}{\constrtype{new}\ }
\newcommand{\from}{\constrtype{from}\ }
\newcommand{\select}{\constrtype{select}\ }
\newcommand{\register}{\constrtype{register}\ }
\newcommand{\Update}{\constrtype{update}\ }

\newcommand{\selectfrom}[2]{\select #1\ \from #2\ \In}
\newcommand{\registerto}[2]{\register #1\ \To #2\ \In}

\newcommand{\newselector}[1]{\new \constrtype{sel}\ #1\ \In}
\newcommand{\newselectorT}[2]{\new \constrtype{sel}\lrangle{#2}\ #1\ \In}
\newcommand{\selecttype}[1]{\dual{\constrtype{sel}}\lrangle{#1}}
\newcommand{\sselecttype}[1]{\constrtype{sel}\lrangle{#1}}

\newcommand{\update}[3]{\Update(#1, #2, #3)\ \In}

\newcommand{\newenv}[1]{\new \mathtt{env}\ #1\ \In\ }
\newcommand{\Letin}[2]{\Let #1 = #2\ \In}

\newcommand{\selqueue}[2]{#1\lrangle{#2}}

%%%%%%%%%%%%%%%%%%%%%%%%%%%%%%%%%%%%%%%%%%%%%%%%%%%%%%%%%%%%%%%%%%%%%%%%%%%%%%%%%%%%%%%%%%%%%%%%%%%%
%                                    DUALITY
%%%%%%%%%%%%%%%%%%%%%%%%%%%%%%%%%%%%%%%%%%%%%%%%%%%%%%%%%%%%%%%%%%%%%%%%%%%%%%%%%%%%%%%%%%%%%%%%%%%%
\newcommand{\dualof}{\ \mathsf{dual}\ }


%%%%%%%%%%%%%%%%%%%%%%%%%%%%%%%%%%%%%%%%%%%%%%%%%%%%%%%%%%%%%%%%%%%%%%%%%%%%%%%%%%%%%%%%%%%%%%%%%%%%
%                                        lambda - CALCULUS
%%%%%%%%%%%%%%%%%%%%%%%%%%%%%%%%%%%%%%%%%%%%%%%%%%%%%%%%%%%%%%%%%%%%%%%%%%%%%%%%%%%%%%%%%%%%%%%%%%%%

\newcommand{\labs}[2]{\lambda #1. #2}

%%%%%%%%%%%%%%%%%%%%%%%%%%%%%%%%%%%%%%%%%%%%%%%%%%%%%%%%%%%%%%%%%%%%%%%%%%%%%%%%%%%%%%%%%%%%%%%%%%%%
%                                    HIGHER ORDER SESSION PI
%%%%%%%%%%%%%%%%%%%%%%%%%%%%%%%%%%%%%%%%%%%%%%%%%%%%%%%%%%%%%%%%%%%%%%%%%%%%%%%%%%%%%%%%%%%%%%%%%%%%
%\newcommand{\pHOp}{\ensuremath{\mathsf{HO}\pi_{\mathsf{p}}}\xspace}
%\newcommand{\pHOpnr}{\ensuremath{\mathsf{HO}\pi^{-\mu}_{\mathsf{p}}}\xspace}
\newcommand{\HOp}{\ensuremath{\mathsf{HO}\pi}\xspace}
%\newcommand{\sessp}{\ensuremath{\mathtt{SE}\pi}\xspace}
\newcommand{\sessp}{\ensuremath{\pi}\xspace}
\newcommand{\haskp}{\ensuremath{\pi^{\lambda}}\xspace}
\newcommand{\pHOp}{\ensuremath{\mathsf{HO}\tilde{\pi}}\xspace}
%\newcommand{\psesp}{\ensuremath{\mathtt{sess}\pi_{\mathsf{p}}}\xspace}
%\newcommand{\psespnr}{\ensuremath{\mathtt{sess}\pi^{-\mu}_{\mathsf{p}}}\xspace}
%\newcommand{\sespnr}{\ensuremath{\mathtt{sess}\pi^{-\mu}}\xspace}
\newcommand{\HO}{\ensuremath{\mathsf{HO}}\xspace}
\newcommand{\HOpp}{\ensuremath{\mathsf{HO\pi^{+}}}\xspace}
\newcommand{\PHOp}{\ensuremath{\mathsf{HO}\,{\widetilde{\pi}}}\xspace}
\newcommand{\PHOpp}{\ensuremath{\mathsf{HO}\,{\widetilde{\pi}}^{\,+}}\xspace}
\newcommand{\PHO}{\ensuremath{\vec{\mathsf{HO}}}\xspace}
\newcommand{\Psessp}{\ensuremath{\vec{\pi}}\xspace}


\newcommand{\CAL}{\ensuremath{\mathsf{C}}\xspace}

\newcommand{\pol}{\mathsf{p}}


\newcommand{\ST}{\mathsf{ST}}


%\newcommand{\pHO}{\mathsf{pure\ HO}}

%\newcommand{\HOp}{\HO^+}
%\newcommand{\pHOp}{\pHO^+}
%\newcommand{\ppi}{\mathsf{pure\ session\ }\pi}
%\newcommand{\spi}{\mathsf{session\ }\pi}

\newcommand{\Proc}{\ensuremath{\diamond}}


%\newcommand{\appl}[2]{#1\lrangle{#2}}
\newcommand{\appl}[2]{#1\, {#2}}
%\newcommand{\abs}[2]{(#1)#2}
\newcommand{\abs}[2]{\lambda #1.\,#2}

\newcommand{\lollipop}{\multimap}
\newcommand{\sharedop}{\rightarrow}
\newcommand{\logicop}{\multimapdot}

\newcommand{\lhot}[1]{#1\!\! \lollipop\!\! \diamond}
\newcommand{\shot}[1]{#1\!\! \sharedop\!\! \diamond}
\newcommand{\hot}[1]{#1 \logicop \diamond}

%\newcommand{\absmap}[1]{}
\newcommand{\vmap}[1]{|\!|#1|\!|}
\newcommand{\smap}[1]{(\!|\!|#1|\!|\!)^s}
\newcommand{\svmap}[1]{(\!|\!|#1|\!|\!)^{s\rightarrow v}}
\newcommand{\amap}[1]{\mathcal{A}\map{#1}}
\newcommand{\absmap}[2]{\mathcal{A}\map{#1}^{#2}}



%%%%% triggers

\newcommand{\hotrigger}[2]{\binp{#1}{x} \newsp{s}{\appl{x}{s} \Par \bout{\dual{s}}{#2} \inact}}
\newcommand{\fotrigger}[5]{\binp{#1}{#2} \newsp{#3}{\map{#4}^{#3} \Par \bout{\dual{#3}}{#5} \inact}}
%\newcommand{\fotrigger}[2]{\binp{#1}{X} \appl{X}{#2}}

%%%%%% Typed relations

\newcommand{\horel}[6]{#1; #2 \proves #3 #4 #5 \proves #6}
%\newcommand{\horel}[6]{#1; \es; #2 \proves #3 #4 #5 \proves #6}

\newcommand{\mhorel}[7]{
	\begin{array}{rcll}
		#1; \es; #2 &#4& #5 \proves& #3\\
			&#4& #6 & #7
	\end{array}
}

%%%%%%%%%%%%%%%%%%%%%%%%%%%%%%%%%%%%%%%%%%%%%%%%%%%%%%%%%%%%%%%%%%%%%%%%%%%%%%%%%%%%%%%%%%%%%%%%%%%%
%                                    LN TRANSFORM
%%%%%%%%%%%%%%%%%%%%%%%%%%%%%%%%%%%%%%%%%%%%%%%%%%%%%%%%%%%%%%%%%%%%%%%%%%%%%%%%%%%%%%%%%%%%%%%%%%%%

\newcommand{\Loop}{\mathsf{Loop}}
\newcommand{\CodeBlocks}{\mathsf{CodeBlocks}}

\newcommand{\lnmap}[1]{\namemap{LN}{#1}}
\newcommand{\lnrmap}[1]{\namemap{LNR}{#1}}
\newcommand{\lnblockmap}[1]{\namemap{\mathcal{B}}{#1}}
\newcommand{\lnnonblockmap}[2]{\map{#1, #2}}

\newcommand{\mapenv}[2]{\map{#1}_{#2}}

%%%%%%%%%%%%%%%%%%%%%%%%%%%%%%%%%%%%%%%%%%%%%%%%%%%%%%%%%%%%%%%%%%%%%%%%%%%%%%%%%%%%%%%%%%%%%%%%%%%%
%                                        GENERAL TYPES
%%%%%%%%%%%%%%%%%%%%%%%%%%%%%%%%%%%%%%%%%%%%%%%%%%%%%%%%%%%%%%%%%%%%%%%%%%%%%%%%%%%%%%%%%%%%%%%%%%%%

% Values
\newcommand{\true}{\sessionfont{tt}}
\newcommand{\false}{\sessionfont{ff}}

% Typed
\newcommand{\bool}{\sessionfont{bool}}
\newcommand{\nat}{\sessionfont{nat}}



%%%%%%%%%%%%%%%%%%%%%%%%%%%%%%%%%%%%%%%%%%%%%%%%%%%%%%%%%%%%%%%%%%%%%%%%%%%%%%%%%%%%%%%%%%%%%%%%%%%%
%                                        PROCESSES NAMES SESSIONS ETC
%%%%%%%%%%%%%%%%%%%%%%%%%%%%%%%%%%%%%%%%%%%%%%%%%%%%%%%%%%%%%%%%%%%%%%%%%%%%%%%%%%%%%%%%%%%%%%%%%%%%

% Processes
\newcommand{\PP}{\ensuremath{P}}
\newcommand{\Q}{\ensuremath{Q}}
\newcommand{\R}{\ensuremath{R}}
\newcommand{\OP}{\ensuremath{\mathsf{O}}}

% Global environments
\newcommand{\En}{\ensuremath{En}}

% Session channels
\newcommand{\s}{\ensuremath{s}}
\newcommand{\ds}{\ensuremath{\dual{s}}}
%\newcommand{\Ms}[2]{\ensuremath{s}\role{#1}\role{#2}}

%Dummy channels
\newcommand{\sd}{\mathtt{sd}}
\newcommand{\shd}{\mathtt{shd}}

% Names
\newcommand{\Ia}{\ensuremath{a}}
\newcommand{\Iu}{\ensuremath{u}}

% Variables, values, expressions
\newcommand{\x}{\ensuremath{x}}
\newcommand{\y}{\ensuremath{y}}
\newcommand{\ks}{\ensuremath{k}}
\newcommand{\cc}{\ensuremath{c}}
\newcommand{\va}{\ensuremath{v}}
\newcommand{\e}{\ensuremath{e}}
\newcommand{\n}{\ensuremath{n}}

% Process Variables

\newcommand{\X}{\varp{X}}
\newcommand{\Y}{\varp{Y}}

% Roles
\newcommand{\p}{\ensuremath{\mathtt{p}}}
\newcommand{\q}{\ensuremath{\mathtt{q}}}
\newcommand{\A}{\ensuremath{A}}

% Types
\newcommand{\G}{\ensuremath{G}}
\newcommand{\gG}{\globaltype{\G}}
\newcommand{\U}{\ensuremath{U}}
\newcommand{\So}{\ensuremath{S}}
\newcommand{\T}{\ensuremath{T}}

% Queue Types
\newcommand{\M}{\ensuremath{M}}
\newcommand{\I}{\ensuremath{M_\inputsym}}
\newcommand{\Om}{\ensuremath{M_\outputsym}}
\newcommand{\Typ}{\ensuremath{\mathsf{T}}}

% Queues values
\newcommand{\h}{\ensuremath{h}}

%barbs
\newcommand{\m}{\ensuremath{\mu}}

% Contexts
\newcommand{\C}{\ensuremath{{\Bbb C}}}
\newcommand{\E}{\ensuremath{E}}

% Congruence completness - Definibility
\newcommand{\TT}{\ensuremath{T}}
\newcommand{\suc}{\textrm{succ}}
\newcommand{\fail}{\textrm{fail}}


% Set selection labels
\newcommand{\lPi}{\set{l_i:\PP_i}_{i \in I}}
\newcommand{\lGi}{\set{l_i:\G_i}_{i \in I}}
\newcommand{\lTi}{\set{l_i:\T_i}_{i \in I}}
\newcommand{\lSi}{\set{l_i:\So_i}_{i \in I}}

% Selector proof

\newcommand{\SEL}{P_\mathit{Sel}}
\newcommand{\DSEL}{P_\mathit{DSel}}
\newcommand{\Sel}{\mathsf{Sel}}
\newcommand{\IfSel}{\mathsf{IfSel}}
\newcommand{\DSel}{\mathsf{DSel}}
\newcommand{\PSel}{\mathsf{PermSel}}
\newcommand{\PIfSel}{\mathsf{PermIfSel}}
\newcommand{\PDSel}{\mathsf{PermDSel}}

%%%%%%%%%%%%%%%%%%%%%%%%%%%%%%%%%%%%%%%%%%%%%%%%%%%%%%%%%%%%%%%%%%%%%%%%%%%%%%%%%%%%%%%%%%%%%%%%%%%%
%                                        ENVIRONMENTS
%%%%%%%%%%%%%%%%%%%%%%%%%%%%%%%%%%%%%%%%%%%%%%%%%%%%%%%%%%%%%%%%%%%%%%%%%%%%%%%%%%%%%%%%%%%%%%%%%%%%
\newtheorem{fact}{Fact}[section]
\newtheorem{notation}{Notation}

%\newenvironment{notation}{\paragraph{{\bf Notation}}}{}

%\newtheorem{proposition}[fact]{{\bf\em Proposition}}
%\newtheorem{example}[fact]{{\bf\em Example}}
%\newtheorem{lemma}[fact]{{\bf\em Lemma}}
%\newtheorem{corollary}[fact]{{\bf\em Corollary}}
%\newtheorem{definition}[fact]{{\bf\em Definition}}
%\newtheorem{theorem}[fact]{{\bf\em Theorem}}
%\newtheorem{remark}[fact]{{\bf\em Remark}}

\newcommand{\nonhosyntax}[1]{\colorbox{lightgray}{\ensuremath{#1}}}

\newenvironment{mytheorem}{%\vspace{-3pt}
	\begin{theorem}
}{%\vspace{-4pt}
	\end{theorem}
}

\newenvironment{myproposition}{
	\begin{proposition}%\vspace{-3pt}
}{%\vspace{-4pt}
	\end{proposition}
}

\newenvironment{mycorollary}{
	\begin{corollary}%\vspace{-3pt}
}{%\vspace{-4pt}
	\end{corollary}
}

\newenvironment{mylemma}{
	\begin{lemma}%\vspace{-3pt}
}{%\vspace{-4pt}
	\end{lemma}
}


\newenvironment{mydefinition}{%\vspace{-3pt}
	\begin{definition}
}{%\vspace{-3pt}
	\end{definition}
}

%\newenvironment{proof}{
%	{\em Proof.}
%}{}


%\newcommand{\qed}{\ensuremath{\square}}





%%%%%%%%%%%%%%%%%%%%%%%%%%%%%%%%%%%%%%%%%%%%%%%%%%%%%%%%%%%%%%%%%%%%%%%%%%%%%%%%%%%%%%%%%%%%%%%%%%%%
%                                        MISC
%%%%%%%%%%%%%%%%%%%%%%%%%%%%%%%%%%%%%%%%%%%%%%%%%%%%%%%%%%%%%%%%%%%%%%%%%%%%%%%%%%%%%%%%%%%%%%%%%%%%
\newcommand{\Appendix}[1]{Appendix \ref{#1}}

\newcommand{\dimcom}[1]{{\bf Comment: #1 \\}}

\newcommand{\hintcom}[1]{{\bf Hint: #1 \\}}

\newif\ifny\nyfalse
%\nytrue
\newcommand{\NY}[1]
{\ifny{\color{purple}{#1}}\else{#1}\fi}

\newcommand{\KH}[1]
{\ifny{\color{brown}{#1}}\else{#1}\fi}

\newif\ifdm\dmtrue
%\dmfalse
\newcommand{\dk}[1]
{\ifdm{\color{blue}{#1}}\else{#1}\fi}

\newif\ifrhu\rhutrue
%\rhufalse
\newcommand{\rh}[1]
{\ifdm{\color{red}{#1}}\else{#1}\fi}

\newif\ifjp\jptrue
%\jpfalse
\newcommand{\jp}[1]
{\ifjp{\color{red}{#1}}\else{#1}\fi}

\newif\ifjp\jptrue
%\jpfalse
\newcommand{\jpc}[1]
{\ifjp{\color{red}{#1}}\else{#1}\fi}

\newcommand{\ENCan}[1]{\langle #1 \rangle}
\newcommand{\NI}{\noindent}


\newcommand{\syntaxvspace}{\\[1mm]}

\newcommand{\TO}[2]{#1\to #2}
\newcommand{\GS}[3]{\TO{#1}{#2}\colon \!\ENCan{#3}}

\newcommand{\ASET}[1]{\{#1\}}
\newcommand{\participant}[1]{\mathtt{#1}}
\newcommand{\CODE}[1]{{\tt #1}}

\newcommand{\AT}[2]{#1 \! : \! #2}


\newcommand{\myrm}{}


\newcommand{\secref}[1]{\S\,\ref{#1}}
\newcommand{\defref}[1]{Def.~\ref{#1}}
\newcommand{\notref}[1]{Not.~\ref{#1}}
\newcommand{\defsref}[1]{Defs.~\ref{#1}}
\newcommand{\figref}[1]{Fig.~\ref{#1}}
\newcommand{\thmref}[1]{Thm.~\ref{#1}}
\newcommand{\thmsref}[1]{Thms.~\ref{#1}}
\newcommand{\exref}[1]{Ex.~\ref{#1}}
\newcommand{\propref}[1]{Prop.~\ref{#1}}
\newcommand{\propsref}[1]{Props.~\ref{#1}}
\newcommand{\appref}[1]{App.~\ref{#1}}
\newcommand{\lemref}[1]{Lem.~\ref{#1}}



\newcommand{\stytra}[6]{\ensuremath{#1; #3 \proves #4 \hby{#2} #5 \proves #6 }}
\newcommand{\stytraarg}[7]{\ensuremath{#1; #3 \proves_{#7} #4 \hby{#2} #5 \proves_{#7} #6 }}
\newcommand{\stytraargi}[8]{\ensuremath{#1; #3 \proves_{#7} #4 \hby{#2}_{#8} #5 \proves_{#7} #6 }}
\newcommand{\wtytra}[6]{\ensuremath{#1; #3 \proves #4 \Hby{#2}  #5 \proves #6}}
\newcommand{\wtytraarg}[7]{\ensuremath{#1; #3 \proves_{#7} #4 \Hby{#2}  #5 \proves_{#7} #6 }}
\newcommand{\wtytraargi}[8]{\ensuremath{#1; #3 \proves_{#7} #4 \Hby{#2}_{#8}  #5 \proves_{#7} #6 }}
\newcommand{\wbb}[6]{\ensuremath{#1; #3 \proves #4 \wb #5 \proves #6 }}
\newcommand{\wbbarg}[7]{\ensuremath{#1; #3 \proves_{#7} #4 \wb_{#7} #5 \proves_{#7} #6 }}

\newcommand{\minussh}{\ensuremath{\mathsf{-sh}}\xspace}

\definecolor{lightgray}{gray}{0.75}

\newcommand\greybox[1]{%
  \vskip\baselineskip%
  \par\noindent\colorbox{lightgray}{%
    \begin{minipage}{\textwidth}#1\end{minipage}%
  }%
  \vskip\baselineskip%
}

%\newcommand{\myparagraph}[1]{\paragraph{\bf #1}}

\newcommand{\mapchar}[2]{\ensuremath{[\!\!(#1)\!\!]^{#2}}}
\newcommand{\omapchar}[1]{\ensuremath{[\!\!(#1)\!\!]_{\mathsf{c}}}}

\newcommand{\trigger}[3]{#1 \leftarrow\!\!\!\!\!\!\!\leftarrow #2:#3 }
\newcommand{\htrigger}[2]{#1 \Leftarrow #2}
\newcommand{\ftrigger}[3]{#1 \Leftarrow \AT{#2}{#3}}

\newcommand{\btau}{\tau_{\beta}}
\newcommand{\stau}{\tau_{s}}
\newcommand{\dtau}{\tau_{d}}



%\newcommand{\HOpp}{\ensuremath{\mathsf{HO\pi^{+}}}\xspace}
%\newcommand{\PHOp}{\ensuremath{\mathsf{HO}{\vec{\pi}}}\xspace}
%\newcommand{\PHOpp}{\ensuremath{\mathsf{HO}{\vec{\pi}}^{+}}\xspace}



% As a general rule, do not put math, special symbols or citations
% in the abstract
\begin{abstract}
The abstract goes here.
\end{abstract}

% no keywords

% For peer review papers, you can put extra information on the cover
% page as needed:
% \ifCLASSOPTIONpeerreview
% \begin{center} \bfseries EDICS Category: 3-BBND \end{center}
% \fi
%
% For peerreview papers, this IEEEtran command inserts a page break and
% creates the second title. It will be ignored for other modes.
\IEEEpeerreviewmaketitle

\section{Introduction}
\label{sec:intro}

\section{Higher-Order Session $\pi$-Calculi}
\label{sec:calculus}
% !TEX root = main.tex
\noindent 
We introduce the 
\emph{Higher-Order Session $\pi$-Calculus} (\HOp).
\HOp includes both name- and abstraction-passing 
as well as recursion; it is a subcalculus 
of the language
studied 
in~\cite{tlca07}. 
Following the literature~\cite{JeffreyR05},
for simplicity of the presentation
we concentrate on the second-order call-by-value \HOp.  
(In \secref{sec:extension} we consider extensions of 
\HOp with higher-order abstractions 
and polyadicity in name-passing/abstractions.)
%We also introduce two subcalculi of \HOp. In particular, we define the 
%core higher-order session
%calculus (\HO), which 
%%. The \HO calculus is  minimal: it 
%includes constructs for shared name synchronisation and 
%%constructs for session establish\-ment/communication and 
%(monadic) name-abstraction, but lacks name-passing and recursion.

%Although minimal, in \secref{s:expr}
%the abstraction-passing capabilities of \HOp will prove 
%expressive enough to capture key features of session communication, 
%such as delegation and recursion.

\subsection{Syntax of \HOp}
\label{subsec:syntax}
\noindent\myparagraph{Values} The syntax of \HOp is defined in \figref{fig:syntax}
\begin{figure}[t]
\[ 
\begin{array}{lll}
u,w  ::=  n \ | \ x,y,z
& n ::= a,b  \ | \ s, \dual{s} 
& V,W  ::=   \nonhosyntax{u} \ | \ \abs{x}{P}
\end{array}
\]
\[
\begin{array}{rclllll}
P,Q \!\!\!\!\!\! & ::= & \!\! \bout{u}{V}{P}  \bnfbar  \binp{u}{x}{P} \bnfbar
 \bsel{u}{l} P \bnfbar \bbra{u}{l_i:P_i}_{i \in I}   \\[1mm]
 & \bnfbar & \!\!  \nonhosyntax{\rvar{X} \bnfbar \recp{X}{P}} \bnfbar \appl{V}{u} \bnfbar P\Par Q \bnfbar \news{n} P 
\bnfbar 
%ny
%\appl{x}{u}

 \inact
%\\[1mm]
 %    & \bnfbar & \nonhosyntax{\rvar{X} \bnfbar \recp{X}{P}}
\end{array}
\]
 \caption{Syntax of \HOp (\HO lacks the constructs in \nonhosyntax{\text{grey}}).}
\label{fig:syntax}
\Hlinefig
\end{figure}
We use $a,b,c, \dots$ (resp.~$s, \dual{s}, \dots$) 
to range over shared (resp. session) names. 
We use $m, n, t, \dots$ for session or shared names. 
We define the dual operation over names $n$ as $\dual{n}$ with
$\dual{\dual{s}} = s$ and $\dual{a} = a$.
Intuitively, names $s$ and $\dual{s}$ are dual (two) \emph{endpoints} while 
shared names represent shared (non-deterministic) points. 
Variables are denoted with $x, y, z, \dots$, 
and recursive variables are denoted with $\varp{X}, \varp{Y} \dots$.
An abstraction %(or higher-order value) 
$\abs{x}{P}$ is a process $P$ with name parameter $x$.
%Symbols $u, v, \dots$ range over identifiers; and  $V, W, \dots$ to denote values. 
Values $V,W$ include 
identifiers $u, v, \ldots$ %(first-order values) 
and 
abstractions $\abs{x}{P}$ (first- and higher-order values, resp.). 

\myparagraph{Terms} 
include the
$\pi$-calculus prefixes for sending and receiving values $V$.
Process $\bout{u}{V} P$ denotes the output of value $V$
over name $u$, with continuation $P$;
process $\binp{u}{x} P$ denotes the input prefix on name $u$ of a value
that 
will substitute variable $x$ in continuation $P$. 
Recursion is expressed by $\recp{X}{P}$,
which binds the recursive variable $\varp{X}$ in process $P$.
Process 
%ny
%$\appl{x}{u}$ 
$\appl{V}{u}$ 
is the application
which substitutes name $u$ on the abstraction~$V$. 
\dk{Typing  ensures \jpc{that} $V$ is not a name.}
Prefix $\bsel{u}{l} P$ selects label $l$ on name $u$ and then behaves as $P$.
%Given $i \in I$ 
Process $\bbra{u}{l_i: P_i}_{i \in I}$ offers a choice on labels $l_i$ with
continuation $P_i$, given that $i \in I$.
%Others are standard c
Constructs for 
inaction $\inact$,  parallel composition $P_1 \Par P_2$, and 
name restriction $\news{n} P$ are standard.
Session name restriction $\news{s} P$ simultaneously binds endpoints $s$ and $\dual{s}$ in $P$.
%A well-formed process relies on assumptions for
%guarded recursive processes.
We use $\fv{P}$ and $\fn{P}$ to denote a set of free 
%\jpc{recursion}
variables and names; 
and assume $V$ in $\bout{u}{V}{P}$ does not include free recursive 
variables $\rvar{X}$. 
If $\fv{P} = \emptyset$, we call $P$ {\em closed}.
%; and closed $P$ without 
%free session names a {\em program}. 

\subsection{Subcalculi of \HOp}
\label{subsec:subcalculi}
\noi
We define two subcalculi of \HOp. 
%We now define several sub-calculi of \HOp. 
The first is the 
{\em core higher-order session calculus} (denoted \HO),
which lacks recursion and name passing; its 
formal syntax is obtained from \figref{fig:syntax} by excluding 
constructs in \nonhosyntax{\text{grey}}.
The second subcalculus is 
the {\em session $\pi$-calculus} 
(denoted $\sessp$), which 
lacks  the
higher-order constructs
(i.e., abstraction passing and application), but includes recursion.
Let $\CAL \in \{\HOp, \HO, \sessp\}$. We write 
$\CAL^{-\mathsf{sh}}$ for $\CAL$ without shared names
(we delete $a,b$ from $n$). 
We shall demonstrate that 
$\HOp$, $\HO$, and $\sessp$ have the same expressivity.

\subsection{Operational Semantics}
\label{subsec:semantics}
\begin{figure}
\[
\begin{array}{rclrcrclr}
(\abs{x}{P}) \, u  & \red & P \subst{u}{x} 
& \orule{App}
		\\[1mm]
%\bout{a}{V} P \Par \binp{a}{x} Q & \red & P \Par Q \subst{V}{x} 
%& \orule{Com}
%		\\[1mm]
\bout{n}{V} P \Par \binp{\dual{n}}{x} Q & \red & P \Par Q \subst{V}{x} 
& \orule{Pass}
		\\[1mm]
		\bsel{n}{l_j} Q \Par \bbra{\dual{n}}{l_i : P_i}_{i \in I} & \red & Q \Par P_j ~~(j \in I)~~  & \orule{Sel}\\[1mm]
		P \red P' & \Rightarrow & \news{n} P  \red  \news{n} P'  & \orule{Res}\\[1mm]
			P \red P' & \Rightarrow  &  P \Par Q  \red   P' \Par Q  & \orule{Par}\\[1mm]
			P \scong Q \red Q' \scong P' & \Rightarrow & P  \red  P' & \orule{Cong}
	\end{array}
\]
{\small
\[
	\begin{array}{c}
		P \Par \inact \scong P
		\quad
		P_1 \Par P_2 \scong P_2 \Par P_1
		\quad
		P_1 \Par (P_2 \Par P_3) \scong (P_1 \Par P_2) \Par P_3
		%\quad
		%P \scong Q \textrm{ if } P \scong_\alpha Q
		\\[1mm]
		\news{n} \inact \scong \inact
		\quad 
		P \Par \news{n} Q \scong \news{n}(P \Par Q)
		\	(n \notin \fn{P})\quad 
		\recp{X}{P} \scong P\subst{\recp{X}{P}}{\rvar{X}}
		\\[1mm]
		P \scong Q \textrm{ if } P \scong_\alpha Q
%		\qquad
%		\dk{V \scong W \textrm{ if } V \scong_\alpha W
%\quad \abs{x}{P} \scong \abs{x}{Q} \textrm{ if } P \scong Q}
	\end{array}
\]
}
\caption{Operational Semantics of $\HOp$. 
\label{fig:reduction}}
\Hlinefig
\end{figure}
\noindent \figref{fig:reduction} defines the operational semantics 
of \HOp.
$\orule{App}$ is a name application; 
$\orule{Pass}$ defines a shared interaction at $n$ 
(\jpc{with} $\dual{n}=n$) or a session interaction;  
$\orule{Sel}$ is the standard rule for labelled choice/selection:
given an index set $I$, 
a process selects label $l_j$ on name $n$ over a set of
labels $\set{l_i}_{i \in I}$ offered by a branching 
on the dual endpoint $\dual{n}$; and other rules are standard.
Rules for \emph{structural congruence} are defined in \figref{fig:reduction} (bottom). 
\jpc{We assume the expected extension of $\scong$ to values $V$.}
We write $\red^\ast$ for a multi-step reduction. 


\section{Types and Typing}
\label{sec:types}
\section{Types for $\HOp$}

We define a session type system for $\HOp$, which is based on the type system developed by Mostrous
and Yoshida in~\cite{tlca07}.

\subsection{Session Types}
We consider a minimal type structure, a fragment of that defined in~\cite{tlca07}.
The only (but fundamental) differences are in the types for values: we focus on having 
$\shot{S}$ and $\lhot{S}$, whereas the structure in~\cite{tlca07} supports general functions $U \sharedop T$ and 
$U \lollipop T$.
\[
	\begin{array}{lcl}
		\text{\emph{Values}} & U ::= & S \bnfbar \lhot{S} \bnfbar \shot{S} \bnfbar \chtype{S} \qquad \quad \text{\emph{Terms}} \quad T ::= U  \bnfbar  \Proc\\
		\text{\emph{Sessions}} \ & S ::= &  \btout{U} S \bnfbar \btinp{U} S
		\bnfbar		\btsel{l_i:S_i}_{i \in I} \bnfbar \btbra{l_i:S_i}_{i \in I} \bnfbar \trec{t}{S} \bnfbar \vart{t}  \bnfbar \tinact 
	\end{array}
\]

There are four different value types $U$; session value $S$, linear higher order value $\lhot{S}$, 
shared higher order value $\shot{S}$; shared channel $\chtype{S}$. Terms can either have a
value type $U$ or a process type $\Proc$.

Session types follow the standard binary session types syntax \cite{}. Session send prefix $\btout{U} S$ 
denotes a session type that sends a value of type $U$ and continues as $S$. Dually receive prefix $\btinp{U} S$
denotes a session type that receives a value of type $U$ and continues as $S$. 
Set $\mathsf{ST}$ is the space of all session types.

\begin{definition}[Duality]
	Let function $F(R): \mathsf{ST} \longrightarrow \mathsf{ST}$ to be defined as:

	\begin{tabular}{rcl}
		$F(R)$ &$=$&		$\set{(\tinact, \tinact), (\vart{t}, \vart{t})}$\\
			&$\cup$&	$\set{(\btout{U};S_1, \btinp{U}; S_2), (\btinp{U};S_1, \btout{U}; S_2) \bnfbar S_1\ R\ S_2}$\\
			&$\cup$&	$\set{(\btsel{l_i: S_i}_{i \in I}, \btbra{l_j: S_j'}_{j \in J}) \bnfbar I \subseteq J, S_i\ R\ S_j'}$\\
			&$\cup$&	$\set{(\btbra{l_i: S_i}_{i \in I}, \btsel{l_j: S_j'}_{j \in J}) \bnfbar J \subseteq I, S_j\ R\ S_i'}$\\
			&$\cup$&	$\set{(\trec{t}{S_1}, \trec{t}{S_2}) \cup (S_1 \subst{\trec{t}{S}}{\vart{t}}, S_2), (S_1, S_2\subst{\trec{t}{S}}{\vart{t}}) \bnfbar S_1\ R\ S_2)}$
	\end{tabular}

	Standard arguments ensure that $F$ is monotone, thus the greatest fix point
	of $F$ exists and let duality defined as $\dualof = \nu X. F(X)$.
\end{definition}


Following our decision of focusing on functions $\shot{S}$ and $\lhot{S}$,
our environments are also simpler than those in~\cite{tlca07}:
\[
	\begin{array}{lcl}
		\text{Shared} & \Gamma \bnfis & \emptyset \bnfbar \Gamma \cat \varp{X}: \shot{S} \bnfbar \Gamma \cat k: \chtype{S} \bnfbar \Gamma \cat \rvar{X}: \Sigma \\
		\text{Linear} & \Lambda \bnfis & \emptyset \bnfbar \Lambda \cat \varp{X}: \lhot{S}  \\
		\text{Session} \quad & \Sigma \bnfis & \emptyset \bnfbar \Sigma \cat k:S  
	\end{array}
\]


With these environments the shape of judgments is exactly the same as in Mostrous and Yoshida's system:
\[
	\begin{array}{c}
		\Gamma; \Lambda ; \Sigma \proves P \hastype T
%		\Gamma; \Lambda; \Sigma \proves V \hastype U
	\end{array}
\]
As expected, weakening, contraction, and exchange principles apply to $\Gamma$;
environments $\Lambda$ and $\Sigma$ behave linearly, and are only subject to exchange.
We require that the domains of $\Gamma, \Lambda$ and $\Sigma$ are pairwise distinct.
%We focus on \emph{well-formed} judgments, which do not share elements in their domains.
%\newcommand{\jrule}[3]{\displaystyle \frac{#1 }{#2} & \trule{#3}}
\newcommand{\jrule}[3]{\displaystyle \trule{#3}~~\frac{#1 }{#2}}
\newcommand{\addenv}{\circ}

\begin{figure}[!t]
\[
	\begin{array}{c}
%		\jrule{ }{\Gamma ; \emptyset; \emptyset \vdash \UnitV \hastype \Unit}{Unit} 
%		\qquad\quad  
		\trule{Session}~~\Gamma; \emptyset; \set{k:S} \proves k \hastype S 
		\\[2mm]
		\trule{Shared}~~\Gamma \cat a : \chtype{S}; \emptyset; \emptyset \proves a \hastype \chtype{S}
		\qquad
		\trule{LVar}~~\Gamma; \set{X: \lhot{S}}; \emptyset \proves X \hastype \lhot{S} 
		\\[2mm]
		\trule{Prom}~~\tree{
			\Gamma; \emptyset; \emptyset \proves V \hastype \lhot{S}
		}{
			\Gamma; \emptyset; \emptyset \proves V \hastype \shot{S}
		} 
		\qquad\quad  
		\trule{Derelic}~~\tree{
			\Gamma; \Lambda \cat X{:}\lhot{S}; \Sigma \proves P \hastype \Proc
		}{
			\Gamma \cat X:\shot{S}; \Lambda; \Sigma \proves P \hastype \Proc
		} 
		\\[4mm]
%		\trule{Subt}~~\tree{
%			\Gamma; \Lambda; \Sigma \proves P \hastype T \quad \Sigma \subt \Sigma' \quad T \subt T'
%		}{
%			\Gamma ; \Lambda; \Sigma' \vdash P \hastype T'
%		} 
%		\qquad\quad

		\trule{Abs}~~\tree{
			\Gamma; \Lambda; \Sigma \cat x: S \proves P \hastype \Proc
		}{
			\Gamma; \Lambda; \Sigma \proves \abs{x}{P} \hastype \lhot{S}
		}
		\\[4mm]

		\trule{App}~~\tree{(U = \lhot{S}) \lor (U = \shot{S}) \quad \Gamma; \Lambda_1; \Sigma_1 \proves X \hastype U  \quad \Gamma; \Lambda_2; \Sigma_2 \proves k \hastype S
		}{
			\Gamma; \Lambda_1 \cup \Lambda_2; \Sigma_1 \cup \Sigma_2 \proves \appl{X}{k} \hastype \Proc
		} 
		\\[4mm]

		\trule{Send}~~\tree{
			\Gamma; \Lambda_1; \Sigma_1 \proves P \hastype \Proc  \quad \Gamma; \Lambda_2; \Sigma_2 \vdash V \hastype U  \quad (k:S \in \Sigma_1 \cup \Sigma_2)
		}{
			\Gamma; \Lambda_1 \cup \Lambda_2; (\Sigma_1 \cup \Sigma_2)\backslash\set{k:S} \cat k:\btout{U} S \proves \bout{k}{V} P \hastype \Proc
		}

		\\[4mm]
		\trule{Conn}~~\tree{
			\Gamma; \Lambda; \Sigma \cat x:S \proves P \hastype \Proc  \quad \Gamma; \emptyset; \emptyset \proves a \hastype \chtype{S}
		}{
			\Gamma; \Lambda; \Sigma \proves \binp{a}{x} P \hastype \Proc
		}
		\\[4mm]
%		\trule{ConnDual}~~\tree{
%			\Gamma; \Lambda; \Sigma \cat x: S_1 \proves P \hastype \Proc  \quad \Gamma; \emptyset; \emptyset \proves k \hastype \chtype{S_2} \quad S_1 \dualof S_2
%		}{
%			\Gamma; \Lambda; \Sigma \proves \bout{k}{x} P \hastype \Proc
%		}
%		\\[4mm]

		\trule{ConnDual}~~\tree{
			\Gamma; \Lambda; \Sigma \cat \dual{s}: S_1 \proves P \hastype \Proc  \quad \Gamma; \emptyset; \emptyset \proves a \hastype \chtype{S_2} \quad S_1 \dualof S_2
		}{
			\Gamma; \Lambda; \Sigma  \proves \bout{a}{\dual{s}} P \hastype \Proc
		}

		\\[4mm]

		\trule{NewSh}~~\tree{
			\Gamma\cat a:\chtype{S} ; \Lambda; \Sigma \proves P \hastype \Proc
		}{
			\Gamma; \Lambda; \Sigma \proves \news{a} P \hastype \Proc}
		\qquad\quad
		\trule{NewSes}~~\tree{
			\Gamma; \Lambda; \Sigma \cat s:S_1 \cat \dual{s}: S_2 \proves P \hastype \Proc \quad S_1 \dualof S_2
		}{
			\Gamma; \Lambda; \Sigma \proves \news{s} P \hastype \Proc
		}
		\\[4mm]

		\trule{RecvS}~~\tree{
			\Gamma; \Lambda; \Sigma \cat k: S_1 \cat x: S_2 \proves P \hastype \Proc
		}{
			\Gamma; \Lambda; \Sigma, k: \btinp{S_2} S_1  \vdash \binp{k}{x}P \hastype \Proc
		}
		\quad\quad 
		\trule{RecvL}~~\tree{
			\Gamma; \Lambda \cat X: \lhot{S}; \Sigma \cat k: S_1  \proves P \hastype \Proc
		}{
			\Gamma; \Lambda; \Sigma \cat k:\btinp{\lhot{S}}S_1  \proves \binp{k}{X}P \hastype \Proc
		}
		\\[4mm]

		\trule{RecvSh}~~\tree{
			\Gamma \cat X: \shot{S}; \Lambda; \Sigma \cat k: S_1  \proves P \hastype \Proc
		}{
			\Gamma; \Lambda; \Sigma \cat k:\btinp{\shot{S}}S_1  \proves \binp{k}{X}P \hastype \Proc
		}
		\quad ~~
		\trule{RecvShN}~~\tree{
			\Gamma \cat x: \chtype{S}; \Lambda; \Sigma \cat k: S_1  \proves P \hastype \Proc
		}{
			\Gamma; \Lambda; \Sigma \cat k:\btinp{\chtype{S}}S_1  \proves \binp{k}{x}P \hastype \Proc
		}
		\\[4mm]
		\trule{Par}~~\tree{
			\Gamma; \Lambda_{1}; \Sigma_{1} \proves P_{1} \hastype \Proc \quad \Gamma; \Lambda_{2}; \Sigma_{2} \proves P_{2} \hastype \Proc
		}{
			\Gamma; \Lambda_{1} \cup \Lambda_2; \Sigma_{1} \cup \Sigma_2 \proves P_1 \Par P_2 \hastype \Proc
		}
		\qquad\quad
		\trule{Close}~~\tree{
			\Gamma; \Lambda; \Sigma  \proves P \hastype T \quad k \not\in \dom{\Gamma, \Lambda,\Sigma}
		}{
			\Gamma; \Lambda; \Sigma \cat k: \tinact  \proves P \hastype \Proc
		}
		\\[4mm]
		\trule{Bra}~~\tree{
			 \forall i \in I \quad \Gamma; \Lambda; \Sigma \cat k:S_i \proves P_i \hastype \Proc
		}{
			\Gamma; \Lambda; \Sigma \cat k: \btbra{l_i:S_i}_{i \in I} \proves \bbra{k}{l_i:P_i}_{i \in I}\hastype \Proc
		}
		\qquad\quad 
	 	\trule{Sel}~~\tree{
			\Gamma; \Lambda; \Sigma \cat k: S_j  \proves P \hastype \Proc \quad j \in I
		}{
			\Gamma; \Lambda; \Sigma \cat k:\btsel{l_i:S_i}_{i \in I} \proves \bsel{s}{l_j} P \hastype \Proc
		}
		\\[4mm]

		\trule{Nil}~~\Gamma; \emptyset; \emptyset \proves \inact \hastype \Proc
\qquad \quad
		\trule{Var}~~\tree{
	
		}{
			\Gamma \cat \rvar{X}: \Sigma; \emptyset; \emptyset  \proves \rvar{X} \hastype \Proc
		}
		\qquad\quad 
%	 	\trule{Rec}~~\tree{
%			\Gamma \cat \rvar{X}: \Sigma; \emptyset; \emptyset  \proves P \hastype \Proc
%		}{
%			\Gamma ; \emptyset; \emptyset  \proves \recp{X}{P} \hastype \Proc
%		}
%		\\[4mm]

	 	\trule{Rec}~~\tree{
			\Gamma \cat \rvar{X}: \Sigma; \emptyset; \Sigma  \proves P \hastype \Proc
		}{
			\Gamma ; \emptyset; \Sigma  \proves \recp{X}{P} \hastype \Proc
		}


	\end{array}
\]
\caption{Typing Rules for $\HOp$\label{fig:typerulesmy}}
\end{figure}


The typing rules for our system are in Fig.~\ref{fig:typerulesmy}. 


\section{Higher-Order Session Bisimulation}
\label{sec:behavioural}
%% !TEX root = main.tex
\noi We develop a theory for observational equivalence over
session typed \HOp processes. The theory follows the principles
laid by the previous work of the authors
\cite{KYHH2015,KY2015}.
We introduce three different bisimulations and prove
\jpc{that}
all of them coincide with typed, reduction-closed,
barbed congruence. 

\subsection{Labelled Transition System for Processes}\label{ss:lts}
\myparagraph{Labels}
\noi We define a labelled transition system (LTS) over
untyped processes. 
Later on, using the \emph{environmental} transition semantics, 
we can define a typed transition relation to formalise 
how a process interacts with a process in its environment. The interaction
is defined on action $\ell$:

\begin{tabular}{l}
	$\ell	\bnfis   \tau 
		\bnfbar	\bactinp{n}{V} 
		\bnfbar	\news{\tilde{m}} \bactout{n}{V}
		\bnfbar	\bactsel{n}{l} 
		\bnfbar	\bactbra{n}{l} $
\end{tabular}

\noi The internal action is defined by label $\tau$.
Action $\news{\tilde{m}} \bactout{n}{V}$ denotes the sending of value $V$
over channel $n$ with
a possible empty set of names $\tilde{m}$ being restricted
(we may also write $\bactout{n}{V}$ when $\tilde{m}$ is empty).
Dually, the action for value reception is 
$\bactinp{n}{V}$.
We also have actions for selecting 
and branching on
a label $l$ ($\bactsel{n}{l}$ and $\bactbra{n}{l}$, resp.)
$\fn{\ell}$ and $\bn{\ell}$ denote 
 sets of free/bound names in $\ell$, resp.
%and set $\mathsf{n}(\ell)=\bn{\ell}\cup \fn{\ell}$. 

Dual actions, defined below, occur on subjects that are dual between them and carry the same
object. Thus, output is dual with input and 
selection is dual with branching.
Formally, duality 
\jpc{on actions}
is the symmetric relation $\asymp$ that satisfies:
\jpc{(i) $\bactsel{n}{l} \asymp \bactbra{\dual{n}}{l}$ 
and (ii) $\news{\tilde{m}} \bactout{n}{V} \asymp \bactinp{\dual{n}}{V}$}.
%
%\begin{tabular}{c}
%	$\bactsel{n}{l} \asymp \bactbra{\dual{n}}{l}
%	\qquad \qquad \qquad
%	\news{\tilde{m}} \bactout{n}{V} \asymp \bactinp{\dual{n}}{V}$s
%
%\end{tabular}
\smallskip

\begin{figure}[t]
\[
\ltsrule{App} \quad 
(\abs{x}{P}) \, u   \by{\tau}  P \subst{u}{x} 
\]
	\[
	\begin{array}{ll}
\ltsrule{Out}\	\bout{n}{V} P \by{\bactout{n}{V}} P 
&
\ltsrule{In}\	\binp{n}{x} P \by{\bactinp{n}{V}} P\subst{V}{x} 
\\[3mm]
 \ltsrule{Sel}\ \bsel{s}{l}{P} \by{\bactsel{s}{l}} P
&
\hspace{-1cm}
\ltsrule{Bra}\ \bbra{s}{l_i:P_i}_{i \in I} \by{\bactbra{s}{l_j}} P_j
\quad (j\in I)
\\[3mm]
\ltsrule{Alpha}
		\tree{
			P \scong_\alpha Q \quad Q\by{\ell} P'
		}{
			P \by{\ell} P'
		}
&
 \ltsrule{Res}	\tree{
			P \by{\ell} P' \quad n \notin \fn{\ell}
		}{
			\news{n} P \by{\ell} \news{n} P' 
		}
\end{array}
\]
\[
\begin{array}{ll}
\ltsrule{New}&	\tree{
		P \by{\news{\tilde{m}} \bactout{n}{V}} P' \quad 
               m \in \fn{V}
		}{
			\news{m} P \by{\news{m\cat\tilde{m}'} 
\bactout{n}{V}} P'
		}
		\\[6mm]
\ltsrule{Tau}	& \tree{
			P \by{\ell_1} P' \qquad Q \by{\ell_2} Q' \qquad \ell_1 \asymp \ell_2
		}{
			P \Par Q \by{\tau} \newsp{\bn{\ell_1} \cup \bn{\ell_2}}{P' \Par Q'}
		} 
		\\[6mm]
 \ltsrule{Par${}_L$}	& \tree{

			P \by{\ell} P' \quad \bn{\ell} \cap \fn{Q} = \es
		}{
			P \Par Q \by{\ell} P' \Par Q
		}

%\\[3mm]
%		\tree{
%			Q \by{\ell} Q' \quad \bn{\ell} \cap \fn{P} = \es
%		}{
%			P \Par Q \by{\ell} P \Par Q'
%		}\ \ltsrule{RPar}
	\end{array}
	\]
%We omit $\ltsrule{Par${}_R$}$. 
	\caption{The Untyped (Early) Labelled Transition System. We omit rule $\ltsrule{Par${}_R$}$.  \label{fig:untyped_LTS}}
\Hlinefig
\end{figure}
\myparagraph{LTS over Untyped Processes}
The labelled transition system (LTS) over untyped processes
is given in
\figref{fig:untyped_LTS}. 
We write $P_1 \by{\ell} P_2$ with the usual meaning.
The rules are standard~\cite{KYHH2015,KY2015}.
A process with a send prefix can
interact with the environment with a send action that carries a value
$V$ as in rule~$\ltsrule{Out}$.  Dually, in rule $\ltsrule{In}$ a
received process can observe an input action of a value $V$.
Selection and branching processes observe the select and branch
actions in rules $\ltsrule{Sel}$ and $\ltsrule{Bra}$, respectively.
Rule $\ltsrule{Res}$ closes the LTS under the name creation operator
if the restricted name does not occur free in the
observable action. 
%If a restricted name occurs free,  
Otherwise, 
the process performs scope opening (rule $\ltsrule{New}$).  
Rule $\ltsrule{Tau}$ states that if two parallel processes can perform
dual actions then the two actions can synchronise by 
an internal transition. Rules $\ltsrule{Par${}_L$}$/$\ltsrule{Par${}_R$}$ 
and $\ltsrule{Alpha}$ close the LTS
under parallel  and $\alpha$-renaming. 
%provided that the observable
%action does not share any bound names with the parallel processes.

\subsection{Environmental Labelled Transition System}
\label{ss:elts}
\noi 
\figref{fig:envLTS}
defines a labelled transition relation between 
a triple of environments, 
denoted
$(\Gamma, \Lambda_1, \Delta_1) \by{\ell} (\Gamma, \Lambda_2, \Delta_2)$.
It extends the transition systems
in \cite{KYHH2015,KY2015} 
to higher-order sessions. 

\myparagraph{Input Actions} 
are defined by 
$\eltsrule{SRv}$ and $\eltsrule{ShRv}$
%describe the input action
($n$ session or shared channel respectively $\bactinp{n}{V}$). 
We require the value $V$ has
the same type as name $s$ and $a$, respectively.  Furthermore we
expect the resulting type tuple to contain the values that consist
with value $V$. The condition $\dual{s} \notin \dom{\Delta}$
in $\eltsrule{SRv}$ ensures that 
the dual channel $\dual{s}$ should not be
present in the session environment, since if it were present
the only communication that could take place is the interaction
between the two endpoints (using $\eltsrule{Tau}$ below).

\myparagraph{Output Actions} are defined by $\eltsrule{SSnd}$
and $\eltsrule{ShSnd}$.  
Rule $\eltsrule{SSnd}$ states the conditions for observing action
$\news{\tilde{m}} \bactout{s}{V}$ on a type tuple 
$(\Gamma, \Lambda, \Delta\cdot \AT{s}{S})$. 
The session environment $\Delta$ with $\AT{s}{S}$ 
should include the session environment of sent value $V$, 
{\em excluding} the session environments of the name $n_j$ 
in $\tilde{m}$ which restrict the scope of value $V$. 
Similarly the linear variable environment 
$\Lambda'$ of $V$ should be included in $\Lambda$. 
Scope extrusion of session names in $\tilde{m}$ requires
that the dual endpoints of $\tilde{m}$ appear in
the resulting session environment. Similarly for shared 
names in $\tilde{m}$ that are extruded.  
All free values used for typing $V$ are subtracted from the
resulting type tuple. The prefix of session $s$ is consumed
by the action.
Similarly, an output on a shared name is described
by rule $\eltsrule{ShSnd}$ where we require that the name
is typed with $\chtype{U}$. Conditions for
the output $V$ are identical to those
% the requirements 
for rule~$\eltsrule{SSnd}$.

\myparagraph{Other Actions}
Rules $\eltsrule{Sel}$ and $\eltsrule{Bra}$ describe actions for
select and branch. The only requirements for both
rules is that the dual endpoint is not present in the session
environment and the action labels are present
in the type.
Hidden transitions defined by rule $\eltsrule{Tau}$ 
do not change the session environment or they follow the reduction on session
environments (\defref{def:ses_red}). 


%A second environment LTS, denoted $\hby{\ell}$,
%is defined in the lower part of \figref{fig:envLTS}.
%The definition substitutes rules
%$\eltsrule{SRecv}$ and $\eltsrule{ShRecv}$
%of relation $\by{\ell}$ with rule $\eltsrule{RRcv}$.
%% the corresponding input cases
%%of $\by{\ell}$ with the definitions of $\hby{\ell}$.
%All other cases remain the same as the cases for
%relation $\by{\ell}$.
%Rule $\eltsrule{RRcv}$ restricts the higher-order input
%in relation $\hby{\ell}$;
%only characteristic processes and trigger processes
%are allowed to be received on a higher-order input.
%Names can still be received as in the definition of
%the $\by{\ell}$ relation.
%The conditions for input follow the conditions
%for the $\by{\ell}$ definition.


\begin{figure}[t]
\[
\begin{array}{lc}
	\eltsrule{SRv}&\tree{
			\dual{s} \notin \dom{\Delta} \quad \Gamma; \Lambda'; \Delta' \proves V \hastype U
		}{
			(\Gamma; \Lambda; \Delta \cat s: \btinp{U} S) \by{\bactinp{s}{V}} (\Gamma; \Lambda\cat\Lambda'; \Delta\cat\Delta' \cat s: S)
		}
		\\[8mm]
		\eltsrule{ShRv}&\tree{
			\Gamma; \es; \es \proves a \hastype \chtype{U}
			\quad
			\Gamma; \Lambda'; \Delta' \proves V \hastype U
		}{
			(\Gamma; \Lambda; \Delta) \by{\bactinp{a}{{V}}} (\Gamma; \Lambda\cat\Lambda'; \Delta\cat\Delta')
		}
%		\eltsrule{RRcv}&\tree {
%\begin{array}{c}
%(\Gamma_1; \Lambda_1; \Delta_1) \by{\bactinp{n}{V}} (\Gamma_2; \Lambda_2; \Delta_2)
%\\
%			\begin{array}{lll}
%				 V  =  
%(\abs{{x}}{\binp{t}{y} (\appl{y}{{x}})}
% \vee  \abs{{x}}{\map{U}^{{x}}}  \vee m)  \textrm{ with } t \textrm{ fresh} 
%			\end{array}
%			\end{array}
%		}{
%			(\Gamma_1; \Lambda_1; \Delta_1) \hby{\bactinp{n}{V}} (\Gamma_2; \Lambda_1; \Delta_2)
%		}
	\end{array}
	\]
	\[
	\begin{array}{l}
		\eltsrule{SSnd}\\
\tree{
			\begin{array}{lll}
			\Gamma \cat \Gamma'; \Lambda'; \Delta' \proves V \hastype U
&				
				\Gamma'; \es; \Delta_j \proves m_j  \hastype U_j
& 
				\dual{s} \notin \dom{\Delta}
\\
						\Delta'\backslash \cup_j \Delta_j \subseteq (\Delta \cat s: S)
& 
	\Gamma'; \es; \Delta_j' \proves \dual{m}_j  \hastype U_j'
& 
				\Lambda' \subseteq \Lambda
%				\dual{s} \notin \dom{\Delta}
%				\qquad 
%				\Gamma \cat \Gamma'; \Lambda'; \Delta_1 \cat \Delta_2 \proves V \hastype U
%				\qquad
%				\tilde{m} = m_1 \dots m_n
%				\\
%				\Gamma'; \es; \Delta_2 \proves m_1 \dots m_n \hastype U_1
%				\qquad
%				\Gamma'; \es; \Delta_3 \proves \dual{m}_1 \dots \dual{m}_n \hastype U_2
%				\qquad
%				\Lambda' \subseteq \Lambda
%				\qquad
%				\Delta_1 \subseteq (\Delta \cat s: S)
			\end{array}
		}{
			(\Gamma; \Lambda; \Delta \cat s: \btout{U} S) \by{\news{\tilde{m}} \bactout{s}{V}} (\Gamma \cat \Gamma'; \Lambda\backslash\Lambda';
			(\Delta \cat s: S \cat \cup_j \Delta_j') \backslash \Delta')
		}
\\[6mm]
\eltsrule{ShSnd}\\
\tree{
		\begin{array}{lll}
			\Gamma \cat \Gamma' ; \Lambda'; \Delta' \proves V \hastype U &  
		\Gamma'; \es; \Delta_j \proves m_j \hastype U_j
& \Gamma ; \es ; \es \proves a \hastype \chtype{U}
				\\
			\Delta'\backslash \cup_j \Delta_j 
                         \subseteq \Delta
& 
		\Gamma'; \es; \Delta_j' \proves \dual{m}_j\hastype U_j'
& 
				\Lambda' \subseteq \Lambda
			\end{array}
%			\begin{array}{c}
%				\Gamma \cat \Gamma' \cat a: \chtype{U}; \Lambda'; \Delta_1 \cat \Delta_2 \proves V \hastype U
%				\qquad
%				\tilde{m} = m_1 \dots m_n
%				\\
%				\Gamma'; \es; \Delta_2 \proves m_1 \dots m_n \hastype U_1
%				\qquad
%				\Gamma'; \es; \Delta_3 \proves \dual{m}_1 \dots \dual{m}_n \hastype U_2
%				\qquad
%				\Lambda' \subseteq \Lambda
%				\qquad
%				\Delta_1 \subseteq \Delta
%			\end{array}
		}{
			(\Gamma ; \Lambda; \Delta) \by{\news{\tilde{m}} \bactout{a}{V}} (\Gamma \cat \Gamma' ; \Lambda\backslash\Lambda';
			(\Delta \cat \cup_j \Delta_j') \backslash \Delta')
		}
\end{array}
\]
\[
\begin{array}{lc}
		\eltsrule{Sel}&\tree{
			\dual{s} \notin \dom{\Delta} \quad j \in I
		}{
			(\Gamma; \Lambda; \Delta \cat s: \btsel{l_i: S_i}_{i \in I}) \by{\bactsel{s}{l_j}} (\Gamma; \Lambda; \Delta \cat s:S_j)
		}
\\[8mm]
		\eltsrule{Bra}&\tree{
			\dual{s} \notin \dom{\Delta} \quad j \in I
		}{
			(\Gamma; \Lambda; \Delta \cat s: \btbra{l_i: T_i}_{i \in I}) \by{\bactbra{s}{l_j}} (\Gamma; \Lambda; \Delta \cat s:S_j)
		}
		\\[8mm]
		\eltsrule{Tau}&\tree{
			\Delta_1 \red \Delta_2 \vee \Delta_1 = \Delta_2
		}{
			(\Gamma; \Lambda; \Delta_1) \by{\tau} (\Gamma; \Lambda; \Delta_2)
		}
%\\[6mm]
%		\eltsrule{RRcv}&\tree {
%\begin{array}{c}
%(\Gamma_1; \Lambda_1; \Delta_1) \by{\bactinp{n}{V}} (\Gamma_2; \Lambda_2; \Delta_2)
%\\
%			\begin{array}{lll}
%				 V  =  
%(\abs{{x}}{\binp{t}{y} (\appl{y}{{x}})}
% \vee  \abs{{x}}{\map{U}^{{x}}}  \vee m)  \textrm{ with } t \textrm{ fresh} 
%			\end{array}
%			\end{array}
%		}{
%			(\Gamma_1; \Lambda_1; \Delta_1) \hby{\bactinp{n}{V}} (\Gamma_2; \Lambda_1; \Delta_2)
%		}
	\end{array}
	\]
\caption{Labelled Transition Systen for Typed Environments. 
\label{fig:envLTS}}
\Hlinefig
\end{figure}

Below we define the typed LTS combining 
the LTS of processes and the LTS of environments. 

\smallskip

\begin{definition}[Typed Transition Systems]\label{d:tlts}\rm
A {\em typed transition relation} is a typed relation
%\begin{enumerate}
%\item 
$\horel{\Gamma}{\Delta_1}{P_1}{\by{\ell}}{\Delta_2}{P_2}$
%$\Gamma; \emptyset; \Delta_1 \proves P_1 \hastype \Proc \by{\ell} \Gamma; \emptyset; \Delta_2 \proves P_2 \hastype \Proc$
	where:
%
(1) $P_1 \by{\ell} P_2$ and (2) 
$(\Gamma, \emptyset, \Delta_1) \by{\ell} (\Gamma, \emptyset, \Delta_2)$ 
with $\Gamma; \emptyset; \Delta_i \proves P_i \hastype \Proc$ 
($i=1,2$). 
%
% Efficient 
%\item 
%$\horel{\Gamma}{\Delta_1}{P_1}{\hby{\ell}}{\Delta_2}{P_2}$
%whenever: 
%$P_1 \by{\ell} P_2$, 
%$(\Gamma, \emptyset, \Delta_1) \hby{\ell} (\Gamma, \emptyset, \Delta_2)$, 
%and $\Gamma; \emptyset; \Delta_i \proves P_i \hastype \Proc$ 
%($i=1,2$)
%\end{enumerate}
%
We extend to $\By{}$ 
%(resp.\ $\Hby{}$) and  
and $\By{\hat{\ell}}$ 
%(resp.\ $\Hby{\hat{\ell}}$) 
where we write 
$\By{}$ for the reflexive and
transitive closure of $\by{}$, $\By{\ell}$ for the transitions
$\By{}\by{\ell}\By{}$ and $\By{\hat{\ell}}$ for $\By{\ell}$ if
$\ell\not = \tau$ otherwise $\By{}$. 
%We extend to $\By{}$ 
%(resp.\ $\Hby{}$) and  and 
%$\By{\hat{\ell}}$ 
%(resp.\ $\Hby{\hat{\ell}}$) 
%in the standard way.
\end{definition}

\subsection{Reduction-Closed, Barbed Congruence}
\label{subsec:rc}
\noi We define the typed relation and the contextual equivalence.  
%\begin{definition}[Session Environment Confluence]\rm
We begin with a definition of a notion of confluence
over session environments $\Delta$:
we denote $\Delta_1 \bistyp \Delta_2$ if there exists $\Delta$ such that
	$\Delta_1 \red^\ast \Delta$ and $\Delta_2 \red^\ast \Delta$
	\jpc{(here we write $\red^\ast$ for the multistep environment reduction in \defref{def:ses_red})}.
%\end{definition}

\smallskip 

\begin{definition}\rm %[Typed Relation]\rm
	We say that
	$\Gamma; \emptyset; \Delta_1 \proves P_1 \hastype \Proc\ \Re \ \Gamma; \emptyset; \Delta_2 \proves P_2 \hastype \Proc$
	is a {\em typed relation} whenever $P_1$ and $P_2$ are closed;
		$\Delta_1$ and $\Delta_2$ are balanced; and 
		$\Delta_1 \bistyp \Delta_2$.
We write
$\horel{\Gamma}{\Delta_1}{P_1}{\ \Re \ }{\Delta_2}{P_2}$
for the typed relation $\Gamma; \emptyset; \Delta_1 \proves P_1 \hastype \Proc\ \Re \ \Gamma; \emptyset; \Delta_2 \proves P_2 \hastype \Proc$.
\end{definition}

\smallskip 

\noi Type relations relate only closed terms with
balanced session environments and the two session environments
are confluent.
Next we define the {\em barb} \cite{MiSa92} 
with respect to types. 

\smallskip 

\begin{definition}[Barbs]\rm
Let $P$ closed. We define:
\begin{enumerate}
		\item	$P \barb{n}$ if $P \scong \newsp{\tilde{m}}{\bout{n}{V} P_2 \Par P_3}, n \notin \tilde{m}$. %; $P \Barb{n}$ if $P \red^* \barb{n}$.

		\item	$\Gamma; \Delta \proves P \barb{n}$ if
			$\Gamma; \emptyset; \Delta \proves P \hastype \Proc$ with $P \barb{n}$ and $\dual{n} \notin \dom{\Delta}$.

	$\Gamma; \Delta \proves P \Barb{n}$ if $P \red^* P'$ and
			$\Gamma; \Delta' \proves P' \barb{n}$.			
	\end{enumerate}
\end{definition}

\smallskip 

\noi A barb $\barb{n}$ is an observable on an output prefix with subject $n$.
Similarly a weak barb $\Barb{n}$ is a barb after a number of reduction steps.
Typed barbs $\barb{n}$ (resp.\ $\Barb{n}$)
happen on typed processes $\Gamma; \emptyset; \Delta \proves P \hastype \Proc$
where we require that the corresponding dual endpoint $\dual{n}$ is not present
in the session type $\Delta$.

To define a congruence relation, we introduce the context $\C$:\\  

%\begin{definition}[Context]\rm
%	A context $\C$ is defined as:
\noi 
\begin{tabular}{rl}
	$\C::=$\!\!\!\! & $\hole \bnfbar \bout{u}{V} \C \bnfbar \binp{u}{x} \C
\bnfbar \bout{u}{\lambda x.\C} P
\bnfbar \news{n} \C$\\
             $\bnfbar$\!\!\!\!& $(\lambda x.\C)u \bnfbar \recp{X}{\C} \bnfbar \C \Par P \bnfbar P \Par \C$\\ 
$\bnfbar$\!\!\!\!& $\bsel{u}{l} \C \bnfbar \bbra{u}{l_1:P_1,..,l_i:\C,..l_n:P_n}$\\
	\end{tabular}
\smallskip 

\noi 
Notation $\context{\C}{P}$ replaces 
\jpc{the hole}
$\hole$ in $\C$ with $P$.
%\end{definition}

\smallskip 

\noi The first behavioural relation we define is reduction-closed, barbed congruence \cite{HondaKYoshida95}. 

\smallskip 

\begin{definition}[Reduction-Closed, Barbed Congruence]\rm
\label{def:rc}
	Typed relation
	$\horel{\Gamma}{\Delta_1}{P_1}{\ \Re\ }{\Delta_2}{P_2}$
	is a {\em reduction-closed, barbed congruence} whenever:
	\begin{enumerate}
		\item	If $P_1 \red P_1'$ then there exists $P_2'$ such that $P_2 \red^* P_2'$ and
			$\horel{\Gamma}{\Delta_1'}{P_1'}{\ \Re\ }{\Delta_2'}{P_2'}$; 
			and its symmetric case;
%		\item	If $P_2 \red P_2'$ then $\exists P_1', P_1 \red^* P_1'$ and
%		$\horel{\Gamma}{\Delta_1'}{P_1'}{\ \Re\ }{\Delta_2'}{P_2'}$
%		\end{itemize}

%		\item
%		\begin{itemize}
			\item	If $\Gamma;\Delta \proves P_1 \barb{n}$ then $\Gamma;\Delta \proves P_2 \Barb{n}$; and its symmetric case; 

%			\item	If $\Gamma;\emptyset;\Delta \proves P_2 \barb{s}$ then $\Gamma;\emptyset;\Delta \proves P_1 \Barb{s}$.
%		\end{itemize}

		\item	for all $\C$, $\horel{\Gamma}{\Delta_1'}{\context{\C}{P_1}}{\ \Re\ }{\Delta_2'}{\context{\C}{P_2}}$
	\end{enumerate}
	The largest such congruence is denoted with $\cong$.
\end{definition}


\subsection{Contextual Bisimulation ($\wbc$)}
\label{subsec:bisimulation}
\noi The first bisimulation which we define 
is the standard contextual bisimulation~\cite{San96H}. 
%
\begin{definition}[Contextual Bisimulation]\rm
\label{def:wbc}
A typed relation $\Re$ is {\em a contextual bisimulation} if
for all $\horel{\Gamma}{\Delta_1}{P_1}{\ \Re \ }{\Delta_2}{Q_1}$, 
	\begin{enumerate}[1)] 
	\item	whenever 
$\horel{\Gamma}{\Delta_1}{P_1}
        {\by{\news{\tilde{m_1}} \bactout{n}{V_1}}}{\Delta_1'}{P_2}$,
there exists $\horel{\Gamma}{\Delta_2}{Q_1}{\By{\news{\tilde{m_2}} \bactout{n}{V_2}}}{\Delta_2'}{Q_2}$ such that, 
for all $R$ with $\fv{R}=x$:
\[\horel{\Gamma}{\Delta_1''}{\newsp{\tilde{m_1}}{P_2 \Par R\subst{V_1}{x}}}
				{\ \Re\ }
				{\Delta_2''}{\newsp{\tilde{m_2}}{Q_2 \Par R\subst{V_2}{x}}};\]  
%\item	$\forall \news{\tilde{m_1}'} \bactout{n}{\tilde{m_1}}$ such that
%			\[
%				\horel{\Gamma}{\Delta_1}{P_1}{\by{\news{\tilde{m_1}'} \bactout{n}{\tilde{m_1}}}}{\Delta_1'}{P_2}
%			\]
%			implies that $\exists Q_2, \tilde{m_2}$ such that
%			\[
%				\horel{\Gamma}{\Delta_2}{Q_1}{\By{\news{\tilde{m_2}'} \bactout{n}{\tilde{m_2}}}}{\Delta_2'}{Q_2}
%			\]
%			and $\forall R$ with $\tilde{x} = \fn{R}$, 
%			then
%			\[
%				\horel{\Gamma}{\Delta_1''}{\newsp{\tilde{m_1}'}{P_2 \Par R \subst{\tilde{m_1}}{\tilde{x}}}}
%				{\ \Re \ }
%				{\Delta_2''}{\newsp{\tilde{m_2}'}{Q_2 \Par R \subst{\tilde{m_2}}{\tilde{x}}}}
%			\]
		\item	
for all $\horel{\Gamma}{\Delta_1}{P_1}{\by{\ell}}{\Delta_1'}{P_2}$ such that 
$\ell$ is not an output, 
 there exists $Q_2$ such that 
$\horel{\Gamma}{\Delta_2}{Q_1}{\By{\ell}}{\Delta_2'}{Q_2}$
			and
			$\horel{\Gamma}{\Delta_1'}{P_2}{\ \Re \ }{\Delta_2'}{Q_2}$; and  

                      \item	The symmetric cases of 1 and 2.                
	\end{enumerate}
	The largest such bisimulation is called contextual bisimilarity \jpc{and} denoted by $\wbc$.
\end{definition}

\smallskip 

\noi As explained in \secref{subsec:intro:bisimulation}, 
in the general case,
contextual bisimulation 
is hard to compute. Below we introduce $\hwb$ and $\fwb$.
%due to: (1) the universal
%quantification over contexts in the output case;
%and (2) a higher-order input prefix which can observe
%infinitely many different input actions (since
%infinitely many different processes can match
%the session type of an input prefix).

\subsection{Higher-Order  and  
Characteristic  Bisimulations ($\hwb$/$\fwb$)}\label{ss:hwb}
\noi 
We now formalise the novelties motivated in \secref{subsec:intro:bisimulation}.
Our main result is \thmref{the:coincidence}.
We define characteristic processes/values:

\begin{definition}[Characteristic Process and Values]\rm
\label{def:char}
%	Let names $\tilde{k}$ and type $\tilde{C}$; then we define a {\em characteristic process}:
%	$\map{\tilde{C}}^{\tilde{k}}$:
%	\[
%		\map{C_1, \cdots, C_n}^{k_1 \cdots k_n} = \map{C_1}^{k_1} \Par \dots \Par \map{C_n}^{k_n}		
%	\]
%	with 
	Let name $u$ and type $U$. 
	\figref{fig:char} defines the {\em characteristic process} 
	$\mapchar{U}{u}$ and the {\em characteristic value} $\omapchar{U}$.
\end{definition}

\smallskip

\begin{figure}[t]
	\[
	\begin{array}{c}
		\begin{array}{rclcl}
			\mapchar{\btinp{U} S}{u} &\!\!\defeq\!\!
& \binp{u}{x} (\mapchar{S}{u} \Par \mapchar{U}{x})
			\\
			\mapchar{\btout{U} S}{u} &\!\!\defeq\!\!& \bout{u}{\omapchar{U}} \mapchar{S}{u} %& & n \textrm{ fresh}
			\\
			\mapchar{\btsel{l : S}}{u} &\!\!\defeq\!\!& \bsel{u}{l} \mapchar{S}{u}
			\\
			\mapchar{\btbra{l_i: S_i}_{i \in I}}{u} &\!\!\defeq\!\!& \bbra{u}{l_i: \mapchar{S_i}{u}}_{i \in I}
			\\
		\mapchar{\tvar{t}}{u} \!\defeq\! \varp{X}_{\vart{t}}
& & 
			\mapchar{\trec{t}{S}}{u} \!\defeq\! \recp{X_{\vart{t}}}{\mapchar{S}{u}}
			\\
			\mapchar{\tinact}{u} &\!\!\defeq\!\!& \inact
			\\
\mapchar{\chtype{S}}{u} \!\defeq\! \bout{u}{\omapchar{S}} \inact & & 
\quad\mapchar{\chtype{L}}{u} \!\defeq\! \bout{u}{\omapchar{L}} \inact
			\\
\mapchar{\shot{C}}{u} & \!\!\defeq\!\! & \mapchar{\lhot{C}}{u} \!\defeq\! 
\appl{u}{\omapchar{C}}
\end{array}
\\
\Hline
\\
		\begin{array}{rcll}
\omapchar{S} &\!\!\defeq\!\!& s & (s \textrm{ fresh})
			\\
\omapchar{\chtype{S}} \defeq \omapchar{\chtype{L}} &\!\!\defeq\!\!& a & 
(a \textrm{ fresh})\quad\quad
			\\
			\omapchar{\shot{C}} \defeq \omapchar{\lhot{C}} &\!\!\defeq\!\!& \abs{x}{\mapchar{C}{x}}
		\end{array}
	\end{array}
	\]
\caption{Characteristic Processes \jpc{(top)} and Values \jpc{(bottom)}.\label{fig:char}}
\Hlinefig
\end{figure}

%	\[
%	\begin{array}{rcllrcll}
%		\map{\btinp{C} S}^{u} &=& \binp{u}{x} (\map{S}^{u} \Par 
%\map{{C}}^{x})\\
%		\map{\btout{C} S}^{u} &=& \bout{u}{n} \map{S}^{u} & n\textrm{ fresh}
%		\\
%		\map{\btsel{l : S}}^{u} &=& \bsel{u}{l} \map{S}^{u}
%\\
%		\map{\btbra{l_i: S_i}_{i \in I}}^{u} &=& \bbra{k}{l_i: \map{S_i}^{u}}_{i \in I}
%		\\

%		\map{\tvar{t}}^{u} &=& \rvar{X}_{\tvar{t}}
%\\
%		\map{\trec{t}{S}}^{u} &=& \mu \rvar{X}_{\tvar{t}}.\map{S}^{u}
%		\\

%		\map{\btout{\lhot{C}} S}^{u} &=& \bout{u}{\abs{x}{\map{C}^{x}}} \map{S}^u
%\\
%		\map{\btinp{\lhot{C}} S}^{u} &=& \binp{u}{x} (\map{S}^u \Par \appl{x}{n}) & {n}\textrm{ fresh}
%		\\
%		\map{\btout{\shot{C}} S}^{u} &=& \bout{u}{\abs{x}{\map{C}^{x}}}
%\map{S}^u \\
%		\map{\btinp{\shot{C}} S}^{u} &=& \binp{u}{x} (\map{S}^u \Par \appl{x}{n}) & n\textrm{ fresh}
%		\\

%%		\map{\btinp{\chtype{S}} S}^{k} &=& \binp{k}{x} (\map{S}^k \Par \map{\chtype{S}}^y)
%%		&&
%%		\map{\btout{\chtype{S}} S}^{k} &=& \bout{k}{a} \map{S}^k  & a\textrm{ fresh}
%%		\\
%		\map{\tinact}^{u} &=& \inact
%\\
%		\map{\chtype{S}}^{u} &=& \bout{u}{s} \inact &s\textrm{ fresh}
%\\
%		\map{\chtype{\lhot{C}}}^{u} &=& \bout{u}{\abs{x} \map{C}^{x}} \inact
%		\\
%		\map{\chtype{\shot{C}}}^{u} &=& \bout{u}{\abs{x} \map{C}^{x}} \inact
%	\end{array}
%	\]
%\end{definition}


%\noi Characteristic processes are inhabitants of their associated type:

%\begin{proposition}[Characteristic Processes]\rm
%\label{pro:characteristic}
%\begin{enumerate}
%\item $\Gamma; \emptyset; \Delta \cat u:S \proves \mapchar{S}{u} \hastype \Proc$; and $\Gamma \cat u:\chtype{S}; \emptyset; \Delta \proves \mapchar{\chtype{S}}{u} \hastype \Proc$; and 
%\item  	If $\Gamma; \emptyset; \Delta \proves \mapchar{C}{u} \hastype \Proc$
%	then
%	$\Gamma; \es; \Delta \proves u \hastype C$.
%\end{enumerate}
%\end{proposition}
%%\begin{IEEEproof}
%%	By induction on $\mapchar{S}{u}$, $\mapchar{\chtype{S}}{u}$
%%and $\mapchar{C}{u}$. 
%%\end{IEEEproof}

\noi We can easily verify characteristic processes are
inhabitants of their associated type. 
The example below explains the motivation of the refined 
LTS explained in \secref{subsec:intro:bisimulation}.
\smallskip  

\begin{example}[The Need for Refined Typed LTS]
\label{ex:motivation}
First we demonstrate that observing a characteristic value
input alone is not sufficient
\dk{to define a sound bisimulation closure}.
Consider typed processes $P_1, P_2$:
%
\begin{eqnarray}
	P_1 = \binp{s}{x} (\appl{x}{s_1} \Par \appl{x}{s_2}) 
	& & 
	P_2 = \binp{s}{x} (\appl{x}{s_1} \Par \binp{s_2}{y} \inact) 
	\label{equ:6}
\end{eqnarray}
%
We can show that $\Gamma; \es; \Delta \cat s: \btinp{\shot{(\btinp{C} \tinact)}} \tinact \proves P_i \hastype \Proc$ ($i \in \{1,2\}$).
If the above processes input and substitute over $x$
the characteristic value $\dk{\omapchar{\shot{(\btinp{C} \tinact})} =} \abs{x}{\binp{x}{y} \inact}$, 
then both processes evolve into:%(\ref{eq:5}) and (\ref{eq:6}) in become:

\begin{tabular}{c}
	$\Gamma; \es; \Delta \proves \binp{s_1}{y} \inact \Par \binp{s_2}{y} \inact \hastype \Proc$
\end{tabular}

\noi therefore becoming 
contextually bisimilar.
%after the input of $\abs{x}{\binp{x}{y}} \inact$.
However, the processes in (\ref{equ:6}) 
are clearly {\em not} contextually bisimilar: there exist many input actions
which may be used to distinguish them.
For example, if 
$P_1$ and $P_2$ input 
$\abs{x} \newsp{s_3}{\bout{a}{s_3} \binp{x}{y} \inact}$ with
$\Gamma; \es; \Delta \proves s,\dual{s} \hastype \tinact, \tinact$
then their derivatives are not contextually bisimilar. 

Observing only the characteristic value 
results in an over-discriminating bisimulation.
However, if a trigger value, 
$\abs{{x}}{\binp{t}{y} (\appl{y}{{x}})}$ 
is received on $s$, 
then we can distinguish 
processes in \eqref{equ:6}:  
%
\small
\begin{eqnarray*}
%	\Gamma; \es; \Delta &\proves& 
	P_1 \By{\ell} \binp{t}{x} (\appl{x}{s_1}) \Par 
\binp{t}{x} (\appl{x}{s_2})
%\hastype \Proc
	\mbox{~~and~~}
%	\Gamma; \es; \Delta &\proves& 
	P_2 \By{\ell} \binp{t}{x} (\appl{x}{s_1}) \Par \binp{s_2}{y} \inact 
%\hastype \Proc
\end{eqnarray*}
\normalsize
%\noi resulting two distinct processes.  
%
\noi where 
$\ell = s?\ENCan{\abs{{x}}{\binp{t}{y} (\appl{y}{{x}})}}$.
One question that arises here is whether the trigger value is enough
to distinguish two processes, hence no need of 
characteristic values as the input. 
This is not the case since the trigger value
alone also results in an over-discriminating bisimulation relation.
In fact the input trigger can be observed on any input prefix
of {\em any type}. For example, consider the following processes:
%
\begin{eqnarray}
	\Gamma; \es; \Delta \proves \newsp{s}{\binp{n}{x} \appl{x}{s} \Par \bout{\dual{s}}{\abs{x} P} \inact} \hastype \Proc\label{equ:7}
	\\
	\Gamma; \es; \Delta \proves \newsp{s}{\binp{n}{x} \appl{x}{s} \Par \bout{\dual{s}}{\abs{x} Q} \inact} \hastype \Proc\label{equ:8}
\end{eqnarray}
%
\noi if processes in \eqref{equ:7}/\eqref{equ:8}
input the trigger abstraction, we obtain: % they evolved to 
\begin{eqnarray*}
%\Gamma; \es; \Delta \proves 
	\newsp{s}{\binp{t}{x} \appl{x}{s} \Par \bout{\dual{s}}{\abs{x} P} \inact} 
%\hastype \Proc
	\mbox{ and }
%      \\
%\Gamma; \es; \Delta \proves 
	\newsp{s}{\binp{t}{x} \appl{x}{s} \Par \bout{\dual{s}}{\abs{x} Q} \inact}
%\hastype \Proc
\end{eqnarray*}

\noi thus we can easily derive a bisimulation closure if we 
assume a bisimulation definition that allows only trigger value input.
%
%\noi It is easy to obtain a closure if allow only the
%trigger value as the input value. 
But if processes in \eqref{equ:7}/\eqref{equ:8}
input the characteristic value $\abs{z}{\binp{z}{x} (\appl{x}{m})}$,  
then they would become:
%
\begin{eqnarray*}
	\Gamma; \es; \Delta \proves \newsp{s}{\binp{s}{x} \appl{x}{m} \Par \bout{\dual{s}}{\abs{x} P} \inact} \wbc \Delta \proves P \subst{m}{x}
	\\
	\Gamma; \es; \Delta \proves \newsp{s}{\binp{s}{x} \appl{x}{m} \Par \bout{\dual{s}}{\abs{x} Q} \inact} \wbc \Delta \proves Q \subst{m}{x}
\end{eqnarray*}
\noi which are not bisimilar if $P \subst{m}{x} \not\wb Q \subst{m}{x}$.
%\qed
%In conclusion, these examples explain a need of both 
%trigger and characteristic values 
%as an input observation in the input transition relation (\eltsrule{RRcv})
%which will be defined in Definition~\ref{def:rlts}.  
\end{example}

\smallskip 
\noi We now define the \emph{refined} typed LTS. 
As explained in \secref{sec:intro}, this new LTS is defined 
by considering a transition rule for input in which admitted values are
trigger or characteristic values:
%\dk{(assume extension of the structural
%congruence to acommodate values: i) $\abs{x}{P} \scong \abs{x}{Q}$ if
%$P \scong Q$) and ii) $n \scong m$ if $n = n$)}: 

\smallskip 

\begin{definition}[Refined Typed Labelled Transition Relation]
	\label{def:rlts}
	We define the environment transition rule for input actions in
	%restricted environment transition relation using the
	%following rule %using the environment transition relation defined in 
	using the input rules in \figref{fig:envLTS}:
	\[
	\begin{array}{rl}
			\eltsrule{RRcv}&\tree {
	\begin{array}{c}
	(\Gamma_1; \Lambda_1; \Delta_1) \by{\bactinp{n}{V}} (\Gamma_2; \Lambda_2; \Delta_2)
	\\
				\begin{array}{lll}
					V  \scong
					(\abs{{x}}{\binp{t}{y} (\appl{y}{{x}})}
					\vee  \omapchar{U}%\abs{{x}}{\map{U}^{{x}}}
					\vee m)  \textrm{ with } t \textrm{ fresh} 
				\end{array}
				\end{array}
			}{
				(\Gamma_1; \Lambda_1; \Delta_1) \hby{\bactinp{n}{V}} (\Gamma_2; \Lambda_1; \Delta_2)
			}
	\end{array}
	\]
	\noi $\eltsrule{RRcv}$ is defined on top
	of rules $\eltsrule{SRv}$ and $\eltsrule{ShRv}$
	in \figref{fig:envLTS}.
%	uses the environment transition
%	$(\Gamma, \Lambda_1, \Delta_1) \hby{\ell} (\Gamma, \Lambda_2, \Delta_2)$
%	in \figref{fig:envLTS}. 
\dk{	We then use the non-receiving rules in \figref{fig:envLTS}
	together with rule $\eltsrule{RRcv}$
	to define 
	$\horel{\Gamma}{\Delta_1}{P_1}{\hby{\ell}}{\Delta_2}{P_2}$
	as in \defref{d:tlts}.}
%	by replacing $\by{\ell}$ by $\hby{\ell}$ in \defref{d:tlts}. 
\end{definition}

\smallskip 

\noi Note 
$\horel{\Gamma}{\Delta_1}{P_1}{\hby{\ell}}{\Delta_2}{P_2}$ implies  
$\horel{\Gamma}{\Delta_1}{P_1}{\by{\ell}}{\Delta_2}{P_2}$.
%See \exref{ex:motivation} for the reason why {\em both} 
%the trigger values ($\lambda x.\binp{t}{y} (\appl{y}{{x}})$) 
%and characteristic values ($\lambda x.\map{U}^{{x}}$) are required 
%to define the following two bisimulations. 

\smallskip 

\myparagraph{The Two Bisimulations.} We now define 
higher-order bisimulation, 
a more tractable bisimulation for $\HO$ and $\HOp$.%-calculi. 
\dk{The two bisimulations differ on the fact that
they use the different (but equivalent)
trigger processes: $\htrigger{t}{V}$ and $\ftrigger{t}{V}{U}$.}

\smallskip 

\begin{definition}[Higher-Order Bisimulation]\rm
	\label{d:hbw}
A typed relation $\Re$ is {\em the HO bisimulation} if 
for all $\horel{\Gamma}{\Delta_1}{P_1}{\ \Re \ }{\Delta_2}{Q_1}$ 
\begin{enumerate}[1)]
\item 
whenever 
$\horel{\Gamma}{\Delta_1}{P_1}{\hby{\news{\tilde{m_1}} \bactout{n}{V_1}}}{\Delta_1'}{P_2}$, there exits 
$\horel{\Gamma}{\Delta_2}{Q_1}{\Hby{\news{\tilde{m_2}} \bactout{n}{V_2}}}{\Delta_2'}{Q_2}$ such that, for fresh $t$, 
\[
\begin{array}{lrlll}
\Gamma; \Delta''_1  \proves  {\newsp{\tilde{m_1}}{P_2 \Par 
\htrigger{t}{V_1}}}
\ \Re 
\ \Delta''_2 \proves {\newsp{\tilde{m_2}}{Q_2 \Par \htrigger{t}{V_2}}}
\end{array}
\]
		\item	
for all $\horel{\Gamma}{\Delta_1}{P_1}{\hby{\ell}}{\Delta_1'}{P_2}$ such that 
$\ell$ is not an output, 
 there exists $Q_2$ such that 
$\horel{\Gamma}{\Delta_2}{Q_1}{\Hby{\ell}}{\Delta_2'}{Q_2}$
			and
			$\horel{\Gamma}{\Delta_1'}{P_2}{\ \Re \ }{\Delta_2'}{Q_2}$; and 

                      \item	The symmetric cases of 1 and 2.                
	\end{enumerate}
	The largest such bisimulation
	is called Higher-Order bisimilarity \jpc{and} denoted by $\hwb$.
\end{definition}

\smallskip 

\noi The characteristic bisimulation is given using 
characteristic trigger processes. 

\smallskip 

\begin{definition}[Characteristic Bisimulation]\rm
\label{d:fwb}
The first order bisimilarity, denoted by $\fwb$, is defined \jpc{by} replacing 
Clause 1) in \defref{d:hbw} with the following clause:\\[1mm]
whenever 
$\horel{\Gamma}{\Delta_1}{P_1}{\hby{\news{\tilde{m_1}} \bactout{n}{V_1}}}{\Delta_1'}{P_2}$ with $\Gamma; \es; \Delta \proves V_1 \hastype U$,  
there exits 
$\horel{\Gamma}{\Delta_2}{Q_1}{\Hby{\news{\tilde{m_2}}\bactout{n}{V_2}}}{\Delta_2'}{Q_2}$ with $\Gamma; \es; \Delta' \proves V_2 \hastype U$,  
such that, for fresh $t$, \\[1mm]
$\begin{array}{lrlll}
\Gamma; \Delta''_1  \proves  {\newsp{\tilde{m_1}}{P_2 \Par 
\ftrigger{t}{V_1}{U_1}}}
\ \Re 
\ \Delta''_2 \proves {\newsp{\tilde{m_2}}{Q_2 \Par \ftrigger{t}{V_2}{U_2}}}
\end{array}
$
\end{definition}

\smallskip 

\noi Below we state the main theorem.

\smallskip 

\begin{theorem}[Coincidence]\rm
	\label{the:coincidence}
$\cong$, $\wbc$, $\hwb$ and $\fwb$ coincide in $\CAL\in \{\HOp, \HO\}$
and 
$\cong$, $\wbc$ and $\fwb$ coincide in $\CAL\in \{\HOp, \HO, \sessp\}$. 
\end{theorem}

\smallskip 

\noi The above theorem shows that using $\hwb$ is the most tractable 
in the higher-order setting, but if the calculus is limited into 
%the $\sessp$-calculus, 
\jpc{\sessp}
we can still use $\fwb$. 


\smallskip  

\noi Processes that do not use shared names, are inherently deterministic. 
The following \jpc{determinacy property will be} useful 
\jpc{in formalizing our expressiveness results (\S\,\ref{sec:positive})}:
%for both positive and negative results. 

%\smallskip 

\begin{lemma}[$\tau$-Inertness]\rm
	\label{lem:tau_inert}
	\begin{enumerate}[1)]
		\item (deterministic transitions) 
		Transition $\horel{\Gamma}{\Delta}{P}{\hby{\tau}}{\Delta'}{P'}$ is called
		{\em deterministic} if it is derived by $\ltsrule{App}$ or 
		$\ltsrule{Tau}$ where $\subj{\ell_1}$ and $\subj{\ell_2}$ in the premise 
		are dual session names. Suppose $\Delta$ is balanced. Then 
		$\Gamma; \Delta \proves P \cong \Delta'\proves P'$ 
		with $\Delta \red^\ast \Delta'$ balanced. 
		\item 
		Let $P$ is the $\HOp^{-\mathsf{sh}}$-calculus. 
		Assume $\Gamma; \emptyset; \Delta \proves P \hastype \Proc$. Then 
		$P \red^\ast P'$ implies $\Gamma; \Delta \proves 
		P \cong \Delta'\proves P'$ with $\Delta \red^\ast \Delta'$. 
	\end{enumerate}
\end{lemma}


%\smallskip 


%\begin{IEEEproof}
%	The full details of the proof are in Appendix~\ref{app:sub_coinc}.
%	The theorem is split into a hierarchy of Lemmas. Specifically
%	Lemma~\ref{lem:wb_eq_wbf} proves 
%	$\wb$ coincides with $\fwb$; 
%	Lemma~\ref{lem:wb_is_wbc} exploits the process substitution result
%	(Lemma~\ref{lem:proc_subst}) to prove that $\hwb \subseteq \wbc$.
%	Lemma~\ref{lem:wbc_is_cong} shows that $\wbc$ is a congruence
%	which implies $\wbc \subseteq \cong$.
%	The final result comes from Lemma~\ref{lem:cong_is_wb} where
%	we use label testing to show that $\cong \subseteq \fwb$ using
%	the technique in developed in~\cite{Hennessy07}. The formulation of input
%	triggers in the bisimulation relation allows us to prove
%	the latter result without using a matching operator.
%\end{IEEEproof}

%\smallskip 

%\noi Processes that do not use shared names, are inherently $\tau$-inert.

%\smallskip 

%\begin{lemma}[$\tau$-inertness]\rm
%	\label{lem:tau_inert}
%	Let $P$ is the $\HOp^{-\mathsf{sh}}$-calculus. 
%Assume $\Gamma; \emptyset; \Delta \proves P \hastype \Proc$. Then 
%$P \red^\ast P'$ implies $\Gamma; \Delta \proves 
%P \cong \Delta'\proves P'$ with $\Delta \red^\ast \Delta'$. 
%\end{lemma}


%\begin{IEEEproof}
%	The proof is relied on the fact that processes of the
%	form $\Gamma; \es; \Delta \proves_s \bout{s}{V} P_1 \Par \binp{k}{x} P_2$
%	cannot have any typed transition observables and the fact
%	that bisimulation is a congruence.
%	See details in Appendix~\ref{app:sub_tau_inert}.
%	\qed
%\end{IEEEproof}



%%%%%%%%%%%%%%%%%%%%%%%%%%%%%%%%%%%%%%%%%%%%%%%%%%%%%%%%%%%%%%%%%%%%%%%%%%%%%
% Bibliography.
%%%%%%%%%%%%%%%%%%%%%%%%%%%%%%%%%%%%%%%%%%%%%%%%%%%%%%%%%%%%%%%%%%%%%%%%%%%%%

\cite{ecoop06}
%\bibliographystyle{IEEEtran}
\bibliographystyle{abbrv}
{\bibliography{session}}

\end{document}


