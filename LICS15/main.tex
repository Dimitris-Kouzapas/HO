
%% bare_conf.tex
%% V1.4a
%% 2014/09/17
%% by Michael Shell
%% See:
%% http://www.michaelshell.org/
%% for current contact information.
%%
%% This is a skeleton file demonstrating the use of IEEEtran.cls
%% (requires IEEEtran.cls version 1.8a or later) with an IEEE
%% conference paper.
%%
%% Support sites:
%% http://www.michaelshell.org/tex/ieeetran/
%% http://www.ctan.org/tex-archive/macros/latex/contrib/IEEEtran/
%% and
%% http://www.ieee.org/

%%*************************************************************************
%% Legal Notice:
%% This code is offered as-is without any warranty either expressed or
%% implied; without even the implied warranty of MERCHANTABILITY or
%% FITNESS FOR A PARTICULAR PURPOSE! 
%% User assumes all risk.
%% In no event shall IEEE or any contributor to this code be liable for
%% any damages or losses, including, but not limited to, incidental,
%% consequential, or any other damages, resulting from the use or misuse
%% of any information contained here.
%%
%% All comments are the opinions of their respective authors and are not
%% necessarily endorsed by the IEEE.
%%
%% This work is distributed under the LaTeX Project Public License (LPPL)
%% ( http://www.latex-project.org/ ) version 1.3, and may be freely used,
%% distributed and modified. A copy of the LPPL, version 1.3, is included
%% in the base LaTeX documentation of all distributions of LaTeX released
%% 2003/12/01 or later.
%% Retain all contribution notices and credits.
%% ** Modified files should be clearly indicated as such, including  **
%% ** renaming them and changing author support contact information. **
%%
%% File list of work: IEEEtran.cls, IEEEtran_HOWTO.pdf, bare_adv.tex,
%%                    bare_conf.tex, bare_jrnl.tex, bare_conf_compsoc.tex,
%%                    bare_jrnl_compsoc.tex, bare_jrnl_transmag.tex
%%*************************************************************************


% *** Authors should verify (and, if needed, correct) their LaTeX system  ***
% *** with the testflow diagnostic prior to trusting their LaTeX platform ***
% *** with production work. IEEE's font choices and paper sizes can       ***
% *** trigger bugs that do not appear when using other class files.       ***                          ***
% The testflow support page is at:
% http://www.michaelshell.org/tex/testflow/



\documentclass[conference]{IEEEtran}
% Some Computer Society conferences also require the compsoc mode option,
% but others use the standard conference format.
%
% If IEEEtran.cls has not been installed into the LaTeX system files,
% manually specify the path to it like:
% \documentclass[conference]{../sty/IEEEtran}

% Some very useful LaTeX packages include:
% (uncomment the ones you want to load)


% *** MISC UTILITY PACKAGES ***
%
%\usepackage{ifpdf}
% Heiko Oberdiek's ifpdf.sty is very useful if you need conditional
% compilation based on whether the output is pdf or dvi.
% usage:
% \ifpdf
%   % pdf code
% \else
%   % dvi code
% \fi
% The latest version of ifpdf.sty can be obtained from:
% http://www.ctan.org/tex-archive/macros/latex/contrib/oberdiek/
% Also, note that IEEEtran.cls V1.7 and later provides a builtin
% \ifCLASSINFOpdf conditional that works the same way.
% When switching from latex to pdflatex and vice-versa, the compiler may
% have to be run twice to clear warning/error messages.

% *** CITATION PACKAGES ***
%
%\usepackage{cite}
% cite.sty was written by Donald Arseneau
% V1.6 and later of IEEEtran pre-defines the format of the cite.sty package
% \cite{} output to follow that of IEEE. Loading the cite package will
% result in citation numbers being automatically sorted and properly
% "compressed/ranged". e.g., [1], [9], [2], [7], [5], [6] without using
% cite.sty will become [1], [2], [5]--[7], [9] using cite.sty. cite.sty's
% \cite will automatically add leading space, if needed. Use cite.sty's
% noadjust option (cite.sty V3.8 and later) if you want to turn this off
% such as if a citation ever needs to be enclosed in parenthesis.
% cite.sty is already installed on most LaTeX systems. Be sure and use
% version 5.0 (2009-03-20) and later if using hyperref.sty.
% The latest version can be obtained at:
% http://www.ctan.org/tex-archive/macros/latex/contrib/cite/
% The documentation is contained in the cite.sty file itself.

% *** GRAPHICS RELATED PACKAGES ***
%
\ifCLASSINFOpdf
  % \usepackage[pdftex]{graphicx}
  % declare the path(s) where your graphic files are
  % \graphicspath{{../pdf/}{../jpeg/}}
  % and their extensions so you won't have to specify these with
  % every instance of \includegraphics
  % \DeclareGraphicsExtensions{.pdf,.jpeg,.png}
\else
  % or other class option (dvipsone, dvipdf, if not using dvips). graphicx
  % will default to the driver specified in the system graphics.cfg if no
  % driver is specified.
  % \usepackage[dvips]{graphicx}
  % declare the path(s) where your graphic files are
  % \graphicspath{{../eps/}}
  % and their extensions so you won't have to specify these with
  % every instance of \includegraphics
  % \DeclareGraphicsExtensions{.eps}
\fi
% graphicx was written by David Carlisle and Sebastian Rahtz. It is
% required if you want graphics, photos, etc. graphicx.sty is already
% installed on most LaTeX systems. The latest version and documentation
% can be obtained at: 
% http://www.ctan.org/tex-archive/macros/latex/required/graphics/
% Another good source of documentation is "Using Imported Graphics in
% LaTeX2e" by Keith Reckdahl which can be found at:
% http://www.ctan.org/tex-archive/info/epslatex/
%
% latex, and pdflatex in dvi mode, support graphics in encapsulated
% postscript (.eps) format. pdflatex in pdf mode supports graphics
% in .pdf, .jpeg, .png and .mps (metapost) formats. Users should ensure
% that all non-photo figures use a vector format (.eps, .pdf, .mps) and
% not a bitmapped formats (.jpeg, .png). IEEE frowns on bitmapped formats
% which can result in "jaggedy"/blurry rendering of lines and letters as
% well as large increases in file sizes.
%
% You can find documentation about the pdfTeX application at:
% http://www.tug.org/applications/pdftex





% *** MATH PACKAGES ***
%
%\usepackage[cmex10]{amsmath}
% A popular package from the American Mathematical Society that provides
% many useful and powerful commands for dealing with mathematics. If using
% it, be sure to load this package with the cmex10 option to ensure that
% only type 1 fonts will utilized at all point sizes. Without this option,
% it is possible that some math symbols, particularly those within
% footnotes, will be rendered in bitmap form which will result in a
% document that can not be IEEE Xplore compliant!
%
% Also, note that the amsmath package sets \interdisplaylinepenalty to 10000
% thus preventing page breaks from occurring within multiline equations. Use:
%\interdisplaylinepenalty=2500
% after loading amsmath to restore such page breaks as IEEEtran.cls normally
% does. amsmath.sty is already installed on most LaTeX systems. The latest
% version and documentation can be obtained at:
% http://www.ctan.org/tex-archive/macros/latex/required/amslatex/math/





% *** SPECIALIZED LIST PACKAGES ***
%
%\usepackage{algorithmic}
% algorithmic.sty was written by Peter Williams and Rogerio Brito.
% This package provides an algorithmic environment fo describing algorithms.
% You can use the algorithmic environment in-text or within a figure
% environment to provide for a floating algorithm. Do NOT use the algorithm
% floating environment provided by algorithm.sty (by the same authors) or
% algorithm2e.sty (by Christophe Fiorio) as IEEE does not use dedicated
% algorithm float types and packages that provide these will not provide
% correct IEEE style captions. The latest version and documentation of
% algorithmic.sty can be obtained at:
% http://www.ctan.org/tex-archive/macros/latex/contrib/algorithms/
% There is also a support site at:
% http://algorithms.berlios.de/index.html
% Also of interest may be the (relatively newer and more customizable)
% algorithmicx.sty package by Szasz Janos:
% http://www.ctan.org/tex-archive/macros/latex/contrib/algorithmicx/




% *** ALIGNMENT PACKAGES ***
%
%\usepackage{array}
% Frank Mittelbach's and David Carlisle's array.sty patches and improves
% the standard LaTeX2e array and tabular environments to provide better
% appearance and additional user controls. As the default LaTeX2e table
% generation code is lacking to the point of almost being broken with
% respect to the quality of the end results, all users are strongly
% advised to use an enhanced (at the very least that provided by array.sty)
% set of table tools. array.sty is already installed on most systems. The
% latest version and documentation can be obtained at:
% http://www.ctan.org/tex-archive/macros/latex/required/tools/


% IEEEtran contains the IEEEeqnarray family of commands that can be used to
% generate multiline equations as well as matrices, tables, etc., of high
% quality.




% *** SUBFIGURE PACKAGES ***
%\ifCLASSOPTIONcompsoc
%  \usepackage[caption=false,font=normalsize,labelfont=sf,textfont=sf]{subfig}
%\else
%  \usepackage[caption=false,font=footnotesize]{subfig}
%\fi
% subfig.sty, written by Steven Douglas Cochran, is the modern replacement
% for subfigure.sty, the latter of which is no longer maintained and is
% incompatible with some LaTeX packages including fixltx2e. However,
% subfig.sty requires and automatically loads Axel Sommerfeldt's caption.sty
% which will override IEEEtran.cls' handling of captions and this will result
% in non-IEEE style figure/table captions. To prevent this problem, be sure
% and invoke subfig.sty's "caption=false" package option (available since
% subfig.sty version 1.3, 2005/06/28) as this is will preserve IEEEtran.cls
% handling of captions.
% Note that the Computer Society format requires a larger sans serif font
% than the serif footnote size font used in traditional IEEE formatting
% and thus the need to invoke different subfig.sty package options depending
% on whether compsoc mode has been enabled.
%
% The latest version and documentation of subfig.sty can be obtained at:
% http://www.ctan.org/tex-archive/macros/latex/contrib/subfig/




% *** FLOAT PACKAGES ***
%
%\usepackage{fixltx2e}
% fixltx2e, the successor to the earlier fix2col.sty, was written by
% Frank Mittelbach and David Carlisle. This package corrects a few problems
% in the LaTeX2e kernel, the most notable of which is that in current
% LaTeX2e releases, the ordering of single and double column floats is not
% guaranteed to be preserved. Thus, an unpatched LaTeX2e can allow a
% single column figure to be placed prior to an earlier double column
% figure. The latest version and documentation can be found at:
% http://www.ctan.org/tex-archive/macros/latex/base/


%\usepackage{stfloats}
% stfloats.sty was written by Sigitas Tolusis. This package gives LaTeX2e
% the ability to do double column floats at the bottom of the page as well
% as the top. (e.g., "\begin{figure*}[!b]" is not normally possible in
% LaTeX2e). It also provides a command:
%\fnbelowfloat
% to enable the placement of footnotes below bottom floats (the standard
% LaTeX2e kernel puts them above bottom floats). This is an invasive package
% which rewrites many portions of the LaTeX2e float routines. It may not work
% with other packages that modify the LaTeX2e float routines. The latest
% version and documentation can be obtained at:
% http://www.ctan.org/tex-archive/macros/latex/contrib/sttools/
% Do not use the stfloats baselinefloat ability as IEEE does not allow
% \baselineskip to stretch. Authors submitting work to the IEEE should note
% that IEEE rarely uses double column equations and that authors should try
% to avoid such use. Do not be tempted to use the cuted.sty or midfloat.sty
% packages (also by Sigitas Tolusis) as IEEE does not format its papers in
% such ways.
% Do not attempt to use stfloats with fixltx2e as they are incompatible.
% Instead, use Morten Hogholm'a dblfloatfix which combines the features
% of both fixltx2e and stfloats:
%
% \usepackage{dblfloatfix}
% The latest version can be found at:
% http://www.ctan.org/tex-archive/macros/latex/contrib/dblfloatfix/




% *** PDF, URL AND HYPERLINK PACKAGES ***
%
%\usepackage{url}
% url.sty was written by Donald Arseneau. It provides better support for
% handling and breaking URLs. url.sty is already installed on most LaTeX
% systems. The latest version and documentation can be obtained at:
% http://www.ctan.org/tex-archive/macros/latex/contrib/url/
% Basically, \url{my_url_here}.




% *** Do not adjust lengths that control margins, column widths, etc. ***
% *** Do not use packages that alter fonts (such as pslatex).         ***
% There should be no need to do such things with IEEEtran.cls V1.6 and later.
% (Unless specifically asked to do so by the journal or conference you plan
% to submit to, of course. )

\usepackage[dvipsnames]{xcolor}
\usepackage{amsmath}
\usepackage{amssymb}
\usepackage{xspace}
\usepackage{graphicx}
\usepackage{latexsym}
\usepackage{listings}
\usepackage{multirow}
\usepackage{suffix}
\usepackage{url}
\usepackage{mathptmx}
\usepackage{mathrsfs}
\usepackage{comment}
\usepackage{enumerate}
\usepackage{txfonts}
\usepackage{hyperref}
%\usepackage{space}
\usepackage{color}      % use if color is used in text

% correct bad hyphenation here
\hyphenation{op-tical net-works semi-conduc-tor}


\begin{document}
%
% paper title
% Titles are generally capitalized except for words such as a, an, and, as,
% at, but, by, for, in, nor, of, on, or, the, to and up, which are usually
% not capitalized unless they are the first or last word of the title.
% Linebreaks \\ can be used within to get better formatting as desired.
% Do not put math or special symbols in the title.
\title{\LARGE Something Higher-Order Session Communications}

% author names and affiliations
% use a multiple column layout for up to three different
% affiliations
\author{
\IEEEauthorblockN{
Dimitrios Kouzapas
}
\IEEEauthorblockA{
Imperial College London	
}
\and
\IEEEauthorblockN{
Jorge A. P\'{e}rez
}
\IEEEauthorblockA{
University of Groningen
}
\and 
\IEEEauthorblockN{
Nobuko Yoshida
}
\IEEEauthorblockA{
Imperial College London	
}
}

\maketitle

%%%%%%%%%%%%%%%%%%%%%%%%%%%%%%%%%%%%%%%%%%%%%%%%%%%%%%%%%%%%%%%%%%%%%%%%%%%%%%%%%%%%%%%%%%%%%%%%%%%%
% Contents
% --------

% 1.  Formating
% 2.  Maths - Theorems
% 3.  The pi Calculus
% 4.  Session Syntax
% 5.  Subject Reduction
% 6.  Global Session Types
% 7.  Global Session Types Equivalence
% 8.  Projection
% 9.  Local Session Types
% 10. Behavioural Theory
% 11. Typed Transitions - Reductions
% 12. Typed Relations
% 13. Confluence Determinacy
% 14. Mapping
% 15. pi Constructs
% 16. LN Transform
% 17. General Types Processes Names Sessions ETC
% 18. newtheorem - newenvironment
% 19. Misc
%%%%%%%%%%%%%%%%%%%%%%%%%%%%%%%%%%%%%%%%%%%%%%%%%%%%%%%%%%%%%%%%%%%%%%%%%%%%%%%%%%%%%%%%%%%%%%%%%%%%


%%%%%%%%%%%%%%%%%%%%%%%%%%%%%%%%%%%%%%%%%%%%%%%%%%%%%%%%%%%%%%%%%%%%%%%%%%%%%%%%%%%%%%%%%%%%%%%%%%%%
%                                       FORMATING
%%%%%%%%%%%%%%%%%%%%%%%%%%%%%%%%%%%%%%%%%%%%%%%%%%%%%%%%%%%%%%%%%%%%%%%%%%%%%%%%%%%%%%%%%%%%%%%%%%%%

% Symbols
\newcommand{\semicolon}{:}
%\newcommand{\colon}{;}
\newcommand{\lrangle}[1]{\langle #1 \rangle}
\newcommand{\blrangle}[1]{\big\langle #1 \big\rangle}

%Tags
\newcommand{\parenthtext}[1]{(\textrm{\small #1})}
\newcommand{\brtext}[1]{[\textrm{\small #1}]}
\newcommand{\textinmath}[1]{\textrm{#1}}
\newcommand{\srule}[1]{\parenthtext{#1}}
\newcommand{\strule}[1]{\textrm{#1}}
\newcommand{\stypes}[1]{{\footnotesize \parenthtext{#1}}}
\newcommand{\ltsrule}[1]{{\footnotesize \lrangle{\textrm{#1}}}}
\newcommand{\eltsrule}[1]{{\footnotesize [\textrm{#1}]}}
\newcommand{\trule}[1]{{\footnotesize\brtext{#1}}}
\newcommand{\orule}[1]{{\scriptsize{\brtext{#1}}}}
\newcommand{\mrule}[1]{{\footnotesize{\parenthtext{#1}}}}

\newcommand{\iftag}{{\textrm{if }}}

% General
\newcommand{\noi}{\noindent}
\newcommand{\Hline}{\rule{\linewidth}{.5pt}}
\newcommand{\Hlinefig}{\rule{\linewidth}{.5pt}\vspace{-4mm}}
\newcommand{\myparagraph}[1]{\noindent{\textbf{#1}\ }}
\newcommand{\jparagraph}[1]{\paragraph{\textbf{#1}}}

%%%%%%%%%%%%%%%%%%%%%%%%%%%%%%%%%%%%%%%%%%%%%%%%%%%%%%%%%%%%%%%%%%%%%%%%%%%%%%%%%%%%%%%%%%%%%%%%%%%%
%                                       MATHS - THEOREMS
%%%%%%%%%%%%%%%%%%%%%%%%%%%%%%%%%%%%%%%%%%%%%%%%%%%%%%%%%%%%%%%%%%%%%%%%%%%%%%%%%%%%%%%%%%%%%%%%%%%%

%\newtheorem{notation}[definition]{Notation}

% BNF form
\newcommand{\bnfis}{\;\;::=\;\;}
\newcommand{\bnfbar}{\;\;\;|\;\;\;}
\newcommand{\sbnfbar}{\;\;|\;\;}

% Proof
\newcommand{\Case}[1]{\noi {\bf Case: }#1\\}
\newcommand{\proofend}{\qed}
%\newcommand{\proofend}{}
\newcommand{\Proof}{\noi {\bf Proof: }}

% Logic
\newcommand{\LogAnd}{\texttt{ and }}
\newcommand{\LogOr}{\texttt{ or }}

% Induction

\newcommand{\basic}{\noi {\bf Basic Step:}\\}
\newcommand{\inductive}{\noi {\bf Inductive Hypothesis:}\\}
\newcommand{\induction}{\noi {\bf Induction Step:}\\}

% Tree

\newcommand{\tree}[2]{
\ensuremath{\displaystyle
		\frac
		{
			%%\raisebox{0.0mm}{$\displaystyle{#1}$}
			#1
			%\vspace{0mm}
		}{
			%\vspace{2mm}
			#2
			%\raisebox{-0.4mm}{$\displaystyle{#2}$}
		}
	}
}


%\newcommand{\tree}[2]{
%\begin{prooftree}
%	#1
%	\justifies
%	#2
%\end{prooftree}
%}

\newcommand{\treeusing}[3]{
\begin{prooftree}
	#1
	\justifies
	#2
	\using
	#3
\end{prooftree}}

% Vectors
\newcommand{\vect}[1]{\tilde{#1}}
\newcommand{\mytilde}[1]{\widetilde{#1}}

% Functions - Set theory
\newcommand{\set}[1]{\{#1\}}
\newcommand{\es}{\emptyset}
\newcommand{\maxset}[1]{\max(#1)}
\newcommand{\setbar}{\ \ |\ \ }
\newcommand{\tuple}[2]{(#1, #2)}
\newcommand{\suchthat}{\cdot}
\newcommand{\powerset}[1]{\mathcal{P}(#1)}
\newcommand{\product}{\times}

\newcommand{\eval}{\downarrow}

\newcommand{\setsubtr}[2]{#1 \backslash #2}

\newcommand{\func}[2]{#1(#2)}
\newcommand{\dom}[1]{\mathtt{dom}(#1)}
\newcommand{\codom}[1]{\mathtt{codom}(#1)}

\newcommand{\funcbr}[2]{#1\lrangle{#2}}

\newcommand{\entails}{\text{implies}}


%%%%%%%%%%%%%%%%%%%%%%%%%%%%%%%%%%%%%%%%%%%%%%%%%%%%%%%%%%%%%%%%%%%%%%%%%%%%%%%%%%%%%%%%%%%%%%%%%%%%
%                                        pi - CALCULUS
%%%%%%%%%%%%%%%%%%%%%%%%%%%%%%%%%%%%%%%%%%%%%%%%%%%%%%%%%%%%%%%%%%%%%%%%%%%%%%%%%%%%%%%%%%%%%%%%%%%%

% Free-Bound notation
\newcommand{\freev}[1]{\lrangle{#1}}
\newcommand{\boundv}[1]{(#1)}

% General pi calculus Syntax
\newcommand{\send}[1]{\overline{#1}}
\newcommand{\ol}[1]{\overline{#1}}
\newcommand{\receive}[1]{#1.}
\newcommand{\inact}{\mathbf{0}}
\newcommand{\If}{\sessionfont{if}\ }
\newcommand{\Then}{\sessionfont{then}\ }
\newcommand{\Else}{\sessionfont{else}\ }
\newcommand{\ifthen}[2]{\If #1\ \Then #2\ }
\newcommand{\ifthenelse}[3]{\ifthen{#1}{#2} \Else #3}
\newcommand{\Par}{\;|\;}
\newcommand{\news}[1]{(\nu\, #1)}
\newcommand{\newsp}[2]{(\nu\, #1)(#2)}
\newcommand{\varp}[1]{#1}
%\newcommand{\rvar}[1]{\mathcal{#1}}
\newcommand{\rvar}[1]{#1}
%\newcommand{\rec}[2]{\mu #1. #2}
\newcommand{\recp}[2]{\mu \rvar{#1}. #2}

\newcommand{\Def}{\sessionfont{def}\ }

\newcommand{\defeq}{\stackrel{\Def}{=}}

\newcommand{\repl}{\ast\,}
\newcommand{\parcomp}[2]{\prod_{#1}{#2}}

% Free-Bound-Names sets
\newcommand{\bn}[1]{\mathtt{bn}(#1)}
\newcommand{\fn}[1]{\mathtt{fn}(#1)}
\newcommand{\ofn}[1]{\mathsf{ofn}(#1)}
\newcommand{\fv}[1]{\mathtt{fv}(#1)}
\newcommand{\bv}[1]{\mathtt{bv}(#1)}
\newcommand{\fs}[1]{\mathtt{fs}(#1)}
\newcommand{\fpv}[1]{\mathtt{fpv}(#1)}
\newcommand{\nam}[1]{\mathtt{n}(#1)}

%Subject - Object
\newcommand{\subj}[1]{\mathtt{subj}(#1)}
\newcommand{\obj}[1]{\mathtt{obj}(#1)}

% Relations
\newcommand{\relfont}[1]{\mathcal{#1}}
\newcommand{\rel}[3]{#1\ \relfont{#2}\ #3}

\newcommand{\scong}{\equiv}
\newcommand{\acong}{\scong_{\alpha}}
\newcommand{\wb}{\approx}
\newcommand{\fwb}{\approx^C}
\newcommand{\hwb}{\approx^H}
\newcommand{\swb}{\approx^{s}}
\newcommand{\wbc}{\approx}
\newcommand{\WB}{\approx}

\newcommand{\red}{\longrightarrow}
\newcommand{\Red}{\rightarrow\!\!\!\!\!\rightarrow}
\newcommand{\Redleft}{\leftarrow\!\!\!\!\!\leftarrow}

%\newcommand{\subst}[2]{\set{#1/#2 }}
\def\subst#1#2{\{\raisebox{.5ex}{\small$#1$}\! / \mbox{\small$#2$}\}}

% Context
\newcommand{\hole}{-}
\newcommand{\context}[2]{#1[#2]}
\newcommand{\Ccontext}[1]{\C[#1]}

% Expression Context
\newcommand{\Econtext}[1]{\E[#1]}

% Barbs
\newcommand{\barb}[1]{\downarrow_{#1}}
\newcommand{\Barb}[1]{\Downarrow_{#1}}
\newcommand{\nbarb}[1]{\not\downarrow_{#1}}
\newcommand{\nBarb}[1]{\not\Downarrow_{#1}}

% General
%\newcommand{\ESP}{\ensuremath{\mathbf{ESP}}}
\newcommand{\ESP}{\text{ESP}}
\newcommand{\ESPsel}{\ESP^+}

%%%%%%%%%%%%%%%%%%%%%%%%%%%%%%%%%%%%%%%%%%%%%%%%%%%%%%%%%%%%%%%%%%%%%%%%%%%%%%%%%%%%%%%%%%%%%%%%%%%%
%                                        SESSION SYNTAX
%%%%%%%%%%%%%%%%%%%%%%%%%%%%%%%%%%%%%%%%%%%%%%%%%%%%%%%%%%%%%%%%%%%%%%%%%%%%%%%%%%%%%%%%%%%%%%%%%%%%

% Session font
\newcommand{\sessionfont}[1]{\mathtt{#1}}
\newcommand{\vart}[1]{\mathsf{#1}}

% General Session symbols
\newcommand{\ssep}{;}
\newcommand{\shsep}{.}
\newcommand{\outses}{!}
\newcommand{\inpses}{?}
\newcommand{\selses}{\triangleleft}
\newcommand{\brases}{\triangleright}
\newcommand{\dual}[1]{\overline{#1}}
\newcommand{\cat}{\cdot}

\newcommand{\allstypes}{\mathcal{S}}

% Binary Session Syntax


\newcommand{\bacc}[2]{#1 \boundv{#2} \shsep}
\newcommand{\breq}[2]{\send{#1} \freev{#2} \shsep}
\newcommand{\bareq}[2]{\send{#1} \freev{#2}}

\newcommand{\breqt}[3]{\send{#1} \boundv{#2:#3} \shsep}
\newcommand{\bacct}[3]{#1 \boundv{#2:#3} \shsep}

\newcommand{\bout}[2]{#1 \outses \freev{#2} \shsep}
\newcommand{\bbout}[2]{#1 \outses \blrangle{#2} \shsep}
\newcommand{\binp}[2]{#1 \inpses \boundv{#2} \shsep}
\newcommand{\bsel}[2]{#1 \selses #2 \shsep}

%\newcommand{\bout}[2]{#1 \outses \freev{#2} \ssep}
%\newcommand{\bbout}[2]{#1 \outses \blrangle{#2} \ssep}
%\newcommand{\binp}[2]{#1 \inpses \boundv{#2} \ssep}
%\newcommand{\bsel}[2]{#1 \selses #2 \ssep}
\newcommand{\bbra}[2]{#1 \brases \set{#2}}
\newcommand{\bbras}[2]{#1 \brases #2}
\newcommand{\bbraP}[1]{#1 \brases \lPi}

% Multiparty Session syntax

\newcommand{\role}[1]{[#1]}

\newcommand{\srole}[2]{#1\role{#2}}
\newcommand{\sqrole}[2]{#1^{[]}\role{#2}}

\newcommand{\fromto}[2]{\role{#1} \role{#2}}
\newcommand{\sfromto}[3]{#1\fromto{#2}{#3}}

\newcommand{\sout}[3]{\srole{#1}{#2} \outses \freev{#3} \ssep}
\newcommand{\sinp}[3]{\srole{#1}{#2} \inpses \boundv{#3} \ssep}
\newcommand{\sdel}[4]{\srole{#1}{#2} \outses \freev{\srole{#3}{#4}} \ssep}
\newcommand{\ssel}[3]{\srole{#1}{#2} \selses #3 \ssep}
\newcommand{\sbra}[3]{\srole{#1}{#2} \brases \set{#3}}
\newcommand{\sbras}[3]{\srole{#1}{#2} \brases #3}
\newcommand{\sbraP}[2]{\srole{#1}{#2} \brases \lPi}

\newcommand{\acc}[3]{#1 \role{#2} \boundv{#3} \shsep}
\newcommand{\req}[3]{\send{#1} \role{#2} \boundv{#3} \shsep}
\newcommand{\areq}[3]{\send{#1} \role{#2} \freev{#3}}

\newcommand{\out}[4]{\sfromto{#1}{#2}{#3} \outses \freev{#4} \ssep}
\newcommand{\inp}[4]{\sfromto{#1}{#2}{#3} \inpses \boundv{#4} \ssep}
\newcommand{\del}[5]{\sfromto{#1}{#2}{#3} \outses \freev{\srole{#4}{#5}} \ssep}
\newcommand{\sel}[4]{\sfromto{#1}{#2}{#3} \selses #4 \ssep}
\newcommand{\bra}[4]{\sfromto{#1}{#2}{#3} \brases \set{#4}}
\newcommand{\bras}[4]{\sfromto{#1}{#2}{#3} \brases #4}
\newcommand{\braP}[3]{\sfromto{#1}{#2}{#3} \brases \lPi}

% Arrive construct
\newcommand{\arrivetext}{\mathtt{arrive}}
\newcommand{\arrive}[1]{\arrivetext\ #1}
\newcommand{\arrivem}[2]{\arrivetext\ #1\ #2}

% Typecase construct
\newcommand{\typecasetext}{\mathtt{typecase}}
\newcommand{\oftext}{\mathtt{of}}
\newcommand{\typecase}[2]{\typecasetext\ #1\ \oftext\ \set{#2}}

% IO symbols
\newcommand{\inputsym}{\mathtt{i}}
\newcommand{\outputsym}{\mathtt{o}}

%%%%%%%%%%%%%%%%%%%%%%%%%%%%%%%%%%%%%%%%%%%%%%%%%%%%%%%%%%%%%%%%%%%%%%%%%%%%%%%%%%%%%%%%%%%%%%%%%%%%
%                                      SUBJECT REDUCTION
%%%%%%%%%%%%%%%%%%%%%%%%%%%%%%%%%%%%%%%%%%%%%%%%%%%%%%%%%%%%%%%%%%%%%%%%%%%%%%%%%%%%%%%%%%%%%%%%%%%%

% typing reduction
\newcommand{\typingred}{\red}
\newcommand{\typingRed}{\Red}

\newcommand{\wellconf}[1]{\mathtt{wc}(#1)}
\newcommand{\cohses}[2]{\mathtt{co}(#1(#2))}
\newcommand{\coherent}[1]{\mathtt{co}(#1)}
\newcommand{\fcoherent}[1]{\mathtt{fco}(#1)}

%%%%%%%%%%%%%%%%%%%%%%%%%%%%%%%%%%%%%%%%%%%%%%%%%%%%%%%%%%%%%%%%%%%%%%%%%%%%%%%%%%%%%%%%%%%%%%%%%%%%
%                                      SESSION ENDPOINTS
%%%%%%%%%%%%%%%%%%%%%%%%%%%%%%%%%%%%%%%%%%%%%%%%%%%%%%%%%%%%%%%%%%%%%%%%%%%%%%%%%%%%%%%%%%%%%%%%%%%%

% Asynchronous syntax
%\newcommand{\mareq}[4]{\newsp{\srole{#2}{#3}, \dots, \srole{#2}{#4}}{\send{#1}[#3] \freev{#2} \Par \dots \Par \send{#1}[#4]\freev{#2}}}

%\newcommand{\areqs}[3]{\send{#1}[\set{#2}] \freev{#3}}

% Queues
\newcommand{\emp}{\epsilon}
\newcommand{\squeue}[3]{\srole{#1}{#2}:#3}
\newcommand{\srqueue}[4]{\srole{#1}{#2}[\inputsym: #3, \outputsym: #4]}
\newcommand{\srqueuei}[3]{\srole{#1}{#2}[\inputsym: #3]}
\newcommand{\srqueueo}[3]{\srole{#1}{#2}[\outputsym: #3]}
\newcommand{\srqueueio}[4]{\srole{#1}{#2}[\inputsym: #3, \outputsym: #4]}
\newcommand{\sgqueue}[2]{\srole{#1}:#2}

% Shared Names Queues
\newcommand{\shqueue}[2]{#1[#2]}
%\newcommand{\shqueuet}[3]{#1[#2, #3]}

% IO Queues
\newcommand{\squeueio}[3]{#1[\inputsym: #2, \outputsym: #3]}
\newcommand{\squeuei}[2]{#1[\inputsym: #2]}
\newcommand{\squeueo}[2]{#1[\outputsym: #2]}

\newcommand{\squeuetio}[4]{#1[#2, \inputsym: #3, \outputsym: #4]}
\newcommand{\squeueto}[3]{#1[#2, \outputsym: #3]}
\newcommand{\squeueti}[3]{#1[#2, \inputsym: #3]}
\newcommand{\squeuet}[2]{#1[#2]}

% Queue Values

\newcommand{\queuev}[2]{\role{#1}(#2)}
\newcommand{\queuel}[2]{\role{#1} #2}
\newcommand{\queues}[3]{\role{#1}(\srole{#2}{#3})}

%%%%%%%%%%%%%%%%%%%%%%%%%%%%%%%%%%%%%%%%%%%%%%%%%%%%%%%%%%%%%%%%%%%%%%%%%%%%%%%%%%%%%%%%%%%%%%%%%%%%
%                                        GLOBAL SESSION TYPES
%%%%%%%%%%%%%%%%%%%%%%%%%%%%%%%%%%%%%%%%%%%%%%%%%%%%%%%%%%%%%%%%%%%%%%%%%%%%%%%%%%%%%%%%%%%%%%%%%%%%

\newcommand{\gtfont}[1]{\mathtt{#1}}
\newcommand{\gsep}{.}

\newcommand{\globaltype}[1]{\lrangle{#1}}
\newcommand{\parties}[1]{\mathtt{\p}(#1)}
\newcommand{\roles}[1]{\mathtt{roles}(#1)}

\newcommand{\fromtogt}[2]{#1 \rightarrow #2 \semicolon}

\newcommand{\valuegt}[3]{\fromtogt{#1}{#2} \lrangle{#3} \gsep}
\newcommand{\selgt}[3]{\fromtogt{#1}{#2} \set{#3}}
\newcommand{\selgtG}[2]{\fromtogt{#1}{#2} \lGi}
\newcommand{\recgt}[2]{\mu \vart{#1}. #2}
\newcommand{\vargt}[1]{\vart{#1}}
\newcommand{\inactgt}{\gtfont{end}}

%%%%%%%%%%%%%%%%%%%%%%%%%%%%%%%%%%%%%%%%%%%%%%%%%%%%%%%%%%%%%%%%%%%%%%%%%%%%%%%%%%%%%%%%%%%%%%%%%%%%
%                              GLOBAL SESSION TYPES EQUIVALENCE
%%%%%%%%%%%%%%%%%%%%%%%%%%%%%%%%%%%%%%%%%%%%%%%%%%%%%%%%%%%%%%%%%%%%%%%%%%%%%%%%%%%%%%%%%%%%%%%%%%%%

\newcommand{\projset}[1]{\mathtt{proj}(#1)}
\newcommand{\aprojset}[1]{\mathtt{aproj}\ #1 }
\newcommand{\gcong}{\equiv}
\newcommand{\govcong}{\cong_g}
\newcommand{\gperm}{\simeq}

%%%%%%%%%%%%%%%%%%%%%%%%%%%%%%%%%%%%%%%%%%%%%%%%%%%%%%%%%%%%%%%%%%%%%%%%%%%%%%%%%%%%%%%%%%%%%%%%%%%%
%                                        PROJECTION
%%%%%%%%%%%%%%%%%%%%%%%%%%%%%%%%%%%%%%%%%%%%%%%%%%%%%%%%%%%%%%%%%%%%%%%%%%%%%%%%%%%%%%%%%%%%%%%%%%%%

\newcommand{\projsymb}{\lceil}
\newcommand{\proj}[2]{#1 \projsymb #2}

%%%%%%%%%%%%%%%%%%%%%%%%%%%%%%%%%%%%%%%%%%%%%%%%%%%%%%%%%%%%%%%%%%%%%%%%%%%%%%%%%%%%%%%%%%%%%%%%%%%%
%                                        LOCAL SESSION TYPES
%%%%%%%%%%%%%%%%%%%%%%%%%%%%%%%%%%%%%%%%%%%%%%%%%%%%%%%%%%%%%%%%%%%%%%%%%%%%%%%%%%%%%%%%%%%%%%%%%%%%

\newcommand{\tfont}[1]{\mathtt{#1}}
\newcommand{\tsep}{;}

\newcommand{\chtype}[1]{\lrangle{#1}}
\newcommand{\chtypei}[1]{\inputsym \lrangle{#1}}
\newcommand{\chtypeo}[1]{\outputsym \lrangle{#1}}
\newcommand{\chtypeio}[1]{\inputsym \outputsym \lrangle{#1}}

\newcommand{\outtype}{\outses}
\newcommand{\inptype}{\inpses}
\newcommand{\seltype}{\selses}
\newcommand{\bratype}{\brases}

\newcommand{\trec}[2]{\mu\vart{#1}.#2}
\newcommand{\tvar}[1]{\vart{#1}}
%\newcommand{\settype}[1]{\set{#1}}
\newcommand{\tset}[1]{\set{#1}}
\newcommand{\tinact}{\tfont{end}}

%\newcommand{\sminus}[1]{#1^-}
\newcommand{\sminus}[1]{#1^{\text{--}}}

\newcommand{\subt}{\leq}
\newcommand{\supt}{\geq}

% Multiparty Local Session Types
\newcommand{\tout}[2]{\role{#1} \outtype \lrangle{#2} \tsep}
\newcommand{\tinp}[2]{\role{#1} \inptype (#2) \tsep}
\newcommand{\tsel}[2]{\role{#1} \seltype \set{#2}}
\newcommand{\tsels}[2]{\role{#1} \seltype #2}
\newcommand{\tselT}[1]{\role{#1} \seltype \lTi}
\newcommand{\tbra}[2]{\role{#1} \bratype \set{#2}}
\newcommand{\tbras}[2]{\role{#1} \bratype #2}
\newcommand{\tbraT}[1]{\role{#1} \bratype \lTi}

% Binary Session Types
\newcommand{\btout}[1]{\outtype \lrangle{#1} \tsep}
\newcommand{\bbtout}[1]{\outtype \big\langle{#1}\big\rangle \tsep}
\newcommand{\btinp}[1]{\inptype (#1) \tsep}
\newcommand{\bbtinp}[1]{\inptype \big({#1}\big) \tsep}
\newcommand{\btsel}[1]{\oplus \set{#1}}
\newcommand{\btselS}{\oplus \lSi}
\newcommand{\btbra}[1]{\& \set{#1}}
\newcommand{\btbraS}{\& \lSi}

% Queue Typing

\newcommand{\mtout}[2]{\role{#1} \outtype \lrangle{#2}}
\newcommand{\mtinp}[2]{\role{#1} \inptype (#2)}
\newcommand{\mtsel}[2]{\role{#1} \seltype #2}
\newcommand{\mtbra}[2]{\role{#1} \bratype #2}


% Binary Queue Typing

\newcommand{\bmtout}[1]{\outtype \lrangle{#1}}
\newcommand{\bmtinp}[1]{\inptype (#1)}
\newcommand{\bmtsel}[1]{\seltype #1}
\newcommand{\bmtbra}[1]{\bratype #1}

% Message concatanation
\newcommand{\mcat}{\;*\;}
\newcommand{\icat}{\;\circ\;}


%%%%%%%%%%%%%%%%%%%%%%%%%%%%%%%%%%%%%%%%%%%%%%%%%%%%%%%%%%%%%%%%%%%%%%%%%%%%%%%%%%%%%%%%%%%%%%%%%%%%
%                                        TYPED PROCESSES
%%%%%%%%%%%%%%%%%%%%%%%%%%%%%%%%%%%%%%%%%%%%%%%%%%%%%%%%%%%%%%%%%%%%%%%%%%%%%%%%%%%%%%%%%%%%%%%%%%%%

\newcommand{\Ga}{\Gamma}
\newcommand{\De}{\Delta}
\newcommand{\proves}{\vdash}
\newcommand{\hastype}{\triangleright}

\newcommand{\Decat}[1]{\De \cat #1}
\newcommand{\Gacat}[1]{\Ga \cat #1}

\newcommand{\tcat}{\circ}

\newcommand{\typed}[1]{#1:}
\newcommand{\typedrole}[2]{\typed{\srole{#1}{#2}}}
%\newcommand{\typedqrole}[2]{\typed{\srole{#1^{[]}}{#2}}}

\newcommand{\typedprocess}[3]{#1 \proves #2 \hastype #3}

\newcommand{\Eproves}[3]{#1 \proves \typed{#2} #3}
\newcommand{\Gproves}[2]{\Eproves{\Ga}{#1}{#2}}

\newcommand{\tprocess}[3]{#1 \proves #2 \hastype #3}
\newcommand{\Gtprocess}[2]{\tprocess{\Ga}{#1}{#2}}
\newcommand{\Gptprocess}[2]{\tprocess{\Ga'}{#1}{#2}}

\newcommand{\noGtprocess}[2]{#1 \hastype #2}

%%%%%%%%%%%%%%%%%%%%%%%%%%%%%%%%%%%%%%%%%%%%%%%%%%%%%%%%%%%%%%%%%%%%%%%%%%%%%%%%%%%%%%%%%%%%%%%%%%%%
%                                    MULTIPARTY TYPED THEORY
%%%%%%%%%%%%%%%%%%%%%%%%%%%%%%%%%%%%%%%%%%%%%%%%%%%%%%%%%%%%%%%%%%%%%%%%%%%%%%%%%%%%%%%%%%%%%%%%%%%%

\newcommand{\globalenv}[1]{\set{#1}}
%\newcommand{\globalenvI}{\set{\typed{s_i} \G_i}_{i \in I}}
\newcommand{\globalenvI}{E}
\newcommand{\globalenvJ}{\set{\typed{s_j} \G_j}_{j \in J}}

\newcommand{\Gltprocess}[4]{\tprocess{#1}{#2}{#3, #4}}

\newcommand{\geI}{\globalenvI}
\newcommand{\geJ}{\globalenvJ}

\newcommand{\Stprocess}[3]{\tprocess{\geI, #1}{#2}{#3}}
\newcommand{\SGtprocess}[2]{\Gtprocess{#1}{#2, \globalenvI}}
\newcommand{\SJGtprocess}[2]{\globalenvJ, \Gtprocess{#1}{#2}}

\newcommand{\Observer}[2]{\mathsf{Observer}(#1, #2)}
\newcommand{\ObserverG}[1]{\mathsf{Observer}(\globalenvI, #1)}

\newcommand{\Obs}{\ensuremath{\mathsf{Obs}}}

%%%%%%%%%%%%%%%%%%%%%%%%%%%%%%%%%%%%%%%%%%%%%%%%%%%%%%%%%%%%%%%%%%%%%%%%%%%%%%%%%%%%%%%%%%%%%%%%%%%%
%                                        BEHAVIOURAL THEORY
%%%%%%%%%%%%%%%%%%%%%%%%%%%%%%%%%%%%%%%%%%%%%%%%%%%%%%%%%%%%%%%%%%%%%%%%%%%%%%%%%%%%%%%%%%%%%%%%%%%%

\newcommand{\fromtolts}[2]{\fromto{#1}{#2}}

\newcommand{\outlts}{\outses}
\newcommand{\inplts}{\inpses}
\newcommand{\sellts}{\oplus}
\newcommand{\bralts}{\&}

% Multiparty Labels
\newcommand{\actreq}[3]{\send{#1} \role{#2} \boundv{#3}}
%\newcommand{\actbreq}[3]{\send{#1} \role{#2} \boundv{#3}}

\newcommand{\actreqs}[3]{\send{#1} \role{\set{#2}} \boundv{#3}}
%\newcommand{\actbreqs}[3]{\send{#1} \role{\set{#2}} \boundv{#3}}

\newcommand{\actacc}[3]{#1 \role{#2} \boundv{#3}}
\newcommand{\actaccs}[3]{#1 \role{\set{#2}} \boundv{#3}}

\newcommand{\actout}[4]{#1 \fromtolts{#2}{#3} \outlts \freev{#4}}
\newcommand{\actqout}[4]{#1^{[]} \fromtolts{#2}{#3} \outlts \freev{#4}}
\newcommand{\actbout}[4]{#1 \fromtolts{#2}{#3} \outlts \boundv{#4}}

\newcommand{\actdel}[5]{#1 \fromtolts{#2}{#3} \outlts \freev{\srole{#4}{#5}}}
\newcommand{\actbdel}[5]{#1 \fromtolts{#2}{#3} \outlts \boundv{\srole{#4}{#5}}}
\newcommand{\actqdel}[5]{#1^{[]} \fromtolts{#2}{#3} \outlts \boundv{\srole{#4}{#5}}}

\newcommand{\actinp}[4]{#1 \fromtolts{#2}{#3} \inplts \freev{#4}}
\newcommand{\actqinp}[4]{#1^{[]} \fromtolts{#2}{#3} \inplts \freev{#4}}

\newcommand{\actsel}[4]{#1 \fromtolts{#2}{#3} \sellts #4}
\newcommand{\actqsel}[4]{#1^{[]} \fromtolts{#2}{#3} \sellts #4}

\newcommand{\actbra}[4]{#1 \fromtolts{#2}{#3} \bralts #4}
\newcommand{\actqbra}[4]{#1^{[]} \fromtolts{#2}{#3} \bralts #4}

\newcommand{\actval}[4]{#1: #2 \rightarrow #3:#4}
\newcommand{\actgsel}[4]{#1: #2 \rightarrow #3:#4}

% Binary Labels
\newcommand{\bactreq}[2]{\send{#1} \freev{#2}}
\newcommand{\bactbreq}[2]{\send{#1} \boundv{#2}}
\newcommand{\bactacc}[2]{#1 \freev{#2}}

\newcommand{\bactout}[2]{#1 \outlts \freev{#2}}
\newcommand{\bactbout}[2]{#1\outlts \boundv{#2}}
\newcommand{\bactinp}[2]{#1 \inplts \freev{#2}}
\newcommand{\bactsel}[2]{#1 \sellts #2}
\newcommand{\bactbra}[2]{#1 \bralts #2}

% Labelled transition relations
\newcommand{\by}[1]{\stackrel{#1}{\longrightarrow}}
\newcommand{\By}[1]{\stackrel{#1}{\Longrightarrow}}

\newcommand{\hby}[1]{\stackrel{#1}{\longmapsto}}
\newcommand{\Hby}[1]{\stackrel{#1}{\Longmapsto}}


% Session barbs
\newcommand{\barbreq}[1]{\barb{#1}}
%\newcommand{\barbacc}[2]{\barb{#1\role{\set{#2}}}}
\newcommand{\barbout}[3]{\barb{\sfromto{#1}{#2}{#3}}}
%\newcommand{\barbinp}[3]{\barb{#1\fromto{#2}{#3}\inpses}}

\newcommand{\Barbreq}[2]{\Barb{\send{#1}\role{#2}}}
%\newcommand{\Barbacc}[2]{\Barb{#1\role{\set{#2}}}}
%\newcommand{\Barbout}[3]{\Barb{\sfromto{#1}{#2}{#3}\outses}}
\newcommand{\Barbout}[3]{\Barb{\sfromto{#1}{#2}{#3}}}
%\newcommand{\Barbinp}[3]{\Barb{#1\fromto{#2}{#3}\inpses}}

% Binary Session barbs
\newcommand{\bbarbreq}[1]{\barb{#1}}
%\newcommand{\bbarbacc}[1]{\barb{#1}}
\newcommand{\bbarbout}[1]{\barb{#1}}
%\newcommand{\bbarbinp}[1]{\barb{#1\inpses}}

\newcommand{\bBarbreq}[1]{\Barb{\send{#1}}}
%\newcommand{\bBarbacc}[2]{\Barb{#1}}
\newcommand{\bBarbout}[1]{\Barb{#1\outses}}
%\newcommand{\bBarbinp}[1]{\Barb{#1\inpses}}

\newcommand{\comp}{\asymp}
\newcommand{\coh}{\asymp}
\newcommand{\bistyp}{\rightleftharpoons}

\newcommand{\typingbeh}{\leftrightarrow}

\newcommand{\bufrel}{\succ}

\newcommand{\ordercup}{\bowtie}

%%%%%%%%%%%%%%%%%%%%%%%%%%%%%%%%%%%%%%%%%%%%%%%%%%%%%%%%%%%%%%%%%%%%%%%%%%%%%%%%%%%%%%%%%%%%%%%%%%%%
%                                TYPED TRANSITIONS - REDUCTIONS
%%%%%%%%%%%%%%%%%%%%%%%%%%%%%%%%%%%%%%%%%%%%%%%%%%%%%%%%%%%%%%%%%%%%%%%%%%%%%%%%%%%%%%%%%%%%%%%%%%%%

% Environment Transitions
\newcommand{\envtrans}[1]{\by{#1}}
\newcommand{\envTrans}[1]{\by{#1}}
\newcommand{\typedtrans}[1]{\by{#1}}
\newcommand{\typedTrans}[1]{\By{#1}}
\newcommand{\typedred}{\red}
\newcommand{\typedRed}{\Red}

% Typed Environment Transitions - Binary case
\newcommand{\benv}[2]{(#1, #2)}
\newcommand{\bGenv}[1]{\benv{\Ga}{#1}}
\newcommand{\bGDenv}{\envtyp{\Ga}{\De}}

\newcommand{\envby}[5]{\benv{#1}{#2} \envtrans{#3} \benv{#4}{#5}}
\newcommand{\Genvby}[3]{\envby{\Ga}{#1}{#2}{\Ga}{#3}}

% Typed Environment Transitions - Multiparty case

\newcommand{\env}[3]{(#1, #2, #3)}
\newcommand{\Genv}[2]{\env{\globalenvI}{#1}{#2}}
\newcommand{\GGenv}[1]{\env{\globalenvI}{\Ga}{#1}}
\newcommand{\GGDenv}{\env{\globalenvI}{\Ga}{\De}}

% Typed Process Transitions - Binary case

\newcommand{\ftby}[7]{\tprocess{#1}{#2}{#3} \typedtrans{#4} \tprocess{#5}{#6}{#7}}
\newcommand{\ftBy}[7]{\tprocess{#1}{#2}{#3} \typedTrans{#4} \tprocess{#5}{#6}{#7}}

\newcommand{\tpby}[6]{\tprocess{#1}{#2}{#3} \typedtrans{#4} \noGtprocess{#5}{#6}}
\newcommand{\tpBy}[6]{\tprocess{#1}{#2}{#3} \typedTrans{#4} \noGtprocess{#5}{#6}}
\newcommand{\Gtpby}[5]{\Gtprocess{#1}{#2} \typedtrans{#3} \noGtprocess{#4}{#5}}
\newcommand{\GtpBy}[5]{\Gtprocess{#1}{#2} \typedTrans{#3} \noGtprocess{#4}{#5}}

\newcommand{\GGtpby}[5]{\tprocess{\globalenvI, \Ga}{#1}{#2} \typedtrans{#3} \noGtprocess{#4}{#5}}
\newcommand{\GGtpBy}[5]{\tprocess{\globalenvI, \Ga}{#1}{#2} \typedTrans{#3} \noGtprocess{#4}{#5}}

% Typed Reductions - Binary case

\newcommand{\ftpred}[6]{\tprocess{#1}{#2}{#3} \typedred \tprocess{#4}{#5}{#6}}
\newcommand{\ftpRed}[6]{\tprocess{#1}{#2}{#3} \typedRed \tprocess{#4}{#5}{#6}}

\newcommand{\tpred}[5]{\tprocess{#1}{#2}{#3} \typedred \noGtprocess{#4}{#5}}
\newcommand{\tpRed}[5]{\tprocess{#1}{#2}{#3} \typedRed \noGtprocess{#4}{#5}}
\newcommand{\Gtpred}[4]{\Gtprocess{#1}{#2} \typedred \noGtprocess{#3}{#4}}
\newcommand{\GtpRed}[4]{\Gtprocess{#1}{#2} \typedRed \noGtprocess{#3}{#4}}

% Observer Reductions

\newcommand{\obsred}{\red_{obs}}
\newcommand{\obsRed}{\Red_{obs}}


%%%%%%%%%%%%%%%%%%%%%%%%%%%%%%%%%%%%%%%%%%%%%%%%%%%%%%%%%%%%%%%%%%%%%%%%%%%%%%%%%%%%%%%%%%%%%%%%%%%%
%                                    TYPED RELATIONS
%%%%%%%%%%%%%%%%%%%%%%%%%%%%%%%%%%%%%%%%%%%%%%%%%%%%%%%%%%%%%%%%%%%%%%%%%%%%%%%%%%%%%%%%%%%%%%%%%%%%

% Typed Relations
\newcommand{\fulltrel}[7]{\rel{\typedprocess{#1}{#2}{#3}}{#4}{\typedprocess{#5}{#6}{#7}}}
\newcommand{\treld}[6]{\rel{\typedprocess{#1}{#2}{#3}}{#4}{\noGtypedprocess{#5}{#6}}}
\newcommand{\trel}[5]{\rel{#1 \proves #2}{#3}{\noGtypedprocess{#4}{#5}}}

\newcommand{\tcong}{\cong}
\newcommand{\twb}{\approx}
\newcommand{\govwb}{\approx_g}
\newcommand{\tequiv}{\approx}

%%%%%%%%%%%%%%%%%%%%%%%%%%%%%%%%%%%%%%%%%%%%%%%%%%%%%%%%%%%%%%%%%%%%%%%%%%%%%%%%%%%%%%%%%%%%%%%%%%%%
%                                    CONFIGURATION THEORY
%%%%%%%%%%%%%%%%%%%%%%%%%%%%%%%%%%%%%%%%%%%%%%%%%%%%%%%%%%%%%%%%%%%%%%%%%%%%%%%%%%%%%%%%%%%%%%%%%%%%

\newcommand{\confpair}[2]{(#1, #2)}
\newcommand{\uptoconfpair}[2]{[#1, #2]}


%%%%%%%%%%%%%%%%%%%%%%%%%%%%%%%%%%%%%%%%%%%%%%%%%%%%%%%%%%%%%%%%%%%%%%%%%%%%%%%%%%%%%%%%%%%%%%%%%%%%
%                                   CONFLUENCE DETERMINACY
%%%%%%%%%%%%%%%%%%%%%%%%%%%%%%%%%%%%%%%%%%%%%%%%%%%%%%%%%%%%%%%%%%%%%%%%%%%%%%%%%%%%%%%%%%%%%%%%%%%%

\newcommand{\sesstrans}[1]{\stackrel{#1}{\longrightarrow_{s}}}
\newcommand{\sessTrans}[1]{\stackrel{#1}{\Longrightarrow_{s}}}

\newcommand{\fulltypedsesstrans}[7]{\typedprocess{#1}{#2}{#3} \sesstrans{#4} \typedprocess{#5}{#6}{#7}}
\newcommand{\fulltypedsessTrans}[7]{\typedprocess{#1}{#2}{#3} \sessTrans{#4} \typedprocess{#5}{#6}{#7}}

\newcommand{\typedsesstrans}[6]{\typedprocess{#1}{#2}{#3} \sesstrans{#4} \noGtypedprocess{#5}{#6}}
\newcommand{\typedsessTrans}[6]{\typedprocess{#1}{#2}{#3} \sessTrans{#4} \noGtypedprocess{#5}{#6}}
\newcommand{\Gtypedsesstrans}[5]{\Gtypedprocess{#1}{#2} \sesstrans{#3} \noGtypedprocess{#4}{#5}}
\newcommand{\GtypedsessTrans}[5]{\Gtypedprocess{#1}{#2} \sessTrans{#3} \noGtypedprocess{#4}{#5}}

% Actions
\newcommand{\confact}[2]{#1 \lfloor #2}

%%%%%%%%%%%%%%%%%%%%%%%%%%%%%%%%%%%%%%%%%%%%%%%%%%%%%%%%%%%%%%%%%%%%%%%%%%%%%%%%%%%%%%%%%%%%%%%%%%%%
%                                    MAPPING AND ENCODINGS
%%%%%%%%%%%%%%%%%%%%%%%%%%%%%%%%%%%%%%%%%%%%%%%%%%%%%%%%%%%%%%%%%%%%%%%%%%%%%%%%%%%%%%%%%%%%%%%%%%%%

\newcommand{\map}[1]{[\!\![#1]\!\!]}
\newcommand{\umap}[1]{[\!\![#1]\!\!]^u}
\newcommand{\pmap}[2]{\ensuremath{[\!\![#1]\!\!]^#2}}
\newcommand{\pmapp}[3]{\ensuremath{[\!\![#1]\!\!]^#2_#3}}
\newcommand{\auxmap}[2]{\ensuremath{\{\!\{#1\}\!\}^#2}}
\newcommand{\tauxmap}[2]{\ensuremath{\{\!|#1|\!\}^#2}}
\newcommand{\auxmapp}[3]{\ensuremath{\big\lfloor\!\!\big\lfloor#1\big\rfloor\!\!\big\rfloor^#2_#3}}
\newcommand{\tmap}[2]{\ensuremath{(\!\!\langle#1\rangle\!\!)^{#2}}}
\newcommand{\vtmap}[2]{{\ensuremath{\big\lfloor #1\big\rfloor^{#2}}}}
\newcommand{\mapt}[1]{\ensuremath{(\!\!\langle#1\rangle\!\!)}}
\newcommand{\mapa}[1]{\ensuremath{\{\!\!\{#1\}\!\!\}}}
\newcommand{\namemap}[2]{#1\map{#2}}

\newcommand{\enc}[2]{\big\langle\map{#1}, \mapt{#2}\big\rangle}
\newcommand{\enco}[1]{\big\langle #1\big\rangle}
\newcommand{\encod}[3]{\lrangle{\map{#1}^{#3}, \mapt{#2}^{#3}}}
\newcommand{\fencod}[4]{\lrangle{\map{#1}^{#3}_{#4} \, , \, \mapt{#2}^{#3}}}

\newcommand{\calc}[5]{\lrangle{#1, #2, #3, #4, #5}}
\newcommand{\tyl}[1]{\ensuremath{\mathcal{#1}}}

%%%%%%%%%%%%%%%%%%%%%%%%%%%%%%%%%%%%%%%%%%%%%%%%%%%%%%%%%%%%%%%%%%%%%%%%%%%%%%%%%%%%%%%%%%%%%%%%%%%%
%                                    PI CONSTRUCTS
%%%%%%%%%%%%%%%%%%%%%%%%%%%%%%%%%%%%%%%%%%%%%%%%%%%%%%%%%%%%%%%%%%%%%%%%%%%%%%%%%%%%%%%%%%%%%%%%%%%%

\newcommand{\constrtype}[1]{\mathtt{#1}}


\newcommand{\Let}{\constrtype{let}\ }
\newcommand{\In}{\constrtype{in}\ }
\newcommand{\To}{\constrtype{to}\ }
\newcommand{\new}{\constrtype{new}\ }
\newcommand{\from}{\constrtype{from}\ }
\newcommand{\select}{\constrtype{select}\ }
\newcommand{\register}{\constrtype{register}\ }
\newcommand{\Update}{\constrtype{update}\ }

\newcommand{\selectfrom}[2]{\select #1\ \from #2\ \In}
\newcommand{\registerto}[2]{\register #1\ \To #2\ \In}

\newcommand{\newselector}[1]{\new \constrtype{sel}\ #1\ \In}
\newcommand{\newselectorT}[2]{\new \constrtype{sel}\lrangle{#2}\ #1\ \In}
\newcommand{\selecttype}[1]{\dual{\constrtype{sel}}\lrangle{#1}}
\newcommand{\sselecttype}[1]{\constrtype{sel}\lrangle{#1}}

\newcommand{\update}[3]{\Update(#1, #2, #3)\ \In}

\newcommand{\newenv}[1]{\new \mathtt{env}\ #1\ \In\ }
\newcommand{\Letin}[2]{\Let #1 = #2\ \In}

\newcommand{\selqueue}[2]{#1\lrangle{#2}}

%%%%%%%%%%%%%%%%%%%%%%%%%%%%%%%%%%%%%%%%%%%%%%%%%%%%%%%%%%%%%%%%%%%%%%%%%%%%%%%%%%%%%%%%%%%%%%%%%%%%
%                                    DUALITY
%%%%%%%%%%%%%%%%%%%%%%%%%%%%%%%%%%%%%%%%%%%%%%%%%%%%%%%%%%%%%%%%%%%%%%%%%%%%%%%%%%%%%%%%%%%%%%%%%%%%
\newcommand{\dualof}{\ \mathsf{dual}\ }


%%%%%%%%%%%%%%%%%%%%%%%%%%%%%%%%%%%%%%%%%%%%%%%%%%%%%%%%%%%%%%%%%%%%%%%%%%%%%%%%%%%%%%%%%%%%%%%%%%%%
%                                        lambda - CALCULUS
%%%%%%%%%%%%%%%%%%%%%%%%%%%%%%%%%%%%%%%%%%%%%%%%%%%%%%%%%%%%%%%%%%%%%%%%%%%%%%%%%%%%%%%%%%%%%%%%%%%%

\newcommand{\labs}[2]{\lambda #1. #2}

%%%%%%%%%%%%%%%%%%%%%%%%%%%%%%%%%%%%%%%%%%%%%%%%%%%%%%%%%%%%%%%%%%%%%%%%%%%%%%%%%%%%%%%%%%%%%%%%%%%%
%                                    HIGHER ORDER SESSION PI
%%%%%%%%%%%%%%%%%%%%%%%%%%%%%%%%%%%%%%%%%%%%%%%%%%%%%%%%%%%%%%%%%%%%%%%%%%%%%%%%%%%%%%%%%%%%%%%%%%%%
%\newcommand{\pHOp}{\ensuremath{\mathsf{HO}\pi_{\mathsf{p}}}\xspace}
%\newcommand{\pHOpnr}{\ensuremath{\mathsf{HO}\pi^{-\mu}_{\mathsf{p}}}\xspace}
\newcommand{\HOp}{\ensuremath{\mathsf{HO}\pi}\xspace}
%\newcommand{\sessp}{\ensuremath{\mathtt{SE}\pi}\xspace}
\newcommand{\sessp}{\ensuremath{\pi}\xspace}
\newcommand{\haskp}{\ensuremath{\pi^{\lambda}}\xspace}
\newcommand{\pHOp}{\ensuremath{\mathsf{HO}\tilde{\pi}}\xspace}
%\newcommand{\psesp}{\ensuremath{\mathtt{sess}\pi_{\mathsf{p}}}\xspace}
%\newcommand{\psespnr}{\ensuremath{\mathtt{sess}\pi^{-\mu}_{\mathsf{p}}}\xspace}
%\newcommand{\sespnr}{\ensuremath{\mathtt{sess}\pi^{-\mu}}\xspace}
\newcommand{\HO}{\ensuremath{\mathsf{HO}}\xspace}
\newcommand{\HOpp}{\ensuremath{\mathsf{HO\pi^{+}}}\xspace}
\newcommand{\PHOp}{\ensuremath{\mathsf{HO}\,{\widetilde{\pi}}}\xspace}
\newcommand{\PHOpp}{\ensuremath{\mathsf{HO}\,{\widetilde{\pi}}^{\,+}}\xspace}
\newcommand{\PHO}{\ensuremath{\vec{\mathsf{HO}}}\xspace}
\newcommand{\Psessp}{\ensuremath{\vec{\pi}}\xspace}


\newcommand{\CAL}{\ensuremath{\mathsf{C}}\xspace}

\newcommand{\pol}{\mathsf{p}}


\newcommand{\ST}{\mathsf{ST}}


%\newcommand{\pHO}{\mathsf{pure\ HO}}

%\newcommand{\HOp}{\HO^+}
%\newcommand{\pHOp}{\pHO^+}
%\newcommand{\ppi}{\mathsf{pure\ session\ }\pi}
%\newcommand{\spi}{\mathsf{session\ }\pi}

\newcommand{\Proc}{\ensuremath{\diamond}}


%\newcommand{\appl}[2]{#1\lrangle{#2}}
\newcommand{\appl}[2]{#1\, {#2}}
%\newcommand{\abs}[2]{(#1)#2}
\newcommand{\abs}[2]{\lambda #1.\,#2}

\newcommand{\lollipop}{\multimap}
\newcommand{\sharedop}{\rightarrow}
\newcommand{\logicop}{\multimapdot}

\newcommand{\lhot}[1]{#1\!\! \lollipop\!\! \diamond}
\newcommand{\shot}[1]{#1\!\! \sharedop\!\! \diamond}
\newcommand{\hot}[1]{#1 \logicop \diamond}

%\newcommand{\absmap}[1]{}
\newcommand{\vmap}[1]{|\!|#1|\!|}
\newcommand{\smap}[1]{(\!|\!|#1|\!|\!)^s}
\newcommand{\svmap}[1]{(\!|\!|#1|\!|\!)^{s\rightarrow v}}
\newcommand{\amap}[1]{\mathcal{A}\map{#1}}
\newcommand{\absmap}[2]{\mathcal{A}\map{#1}^{#2}}



%%%%% triggers

\newcommand{\hotrigger}[2]{\binp{#1}{x} \newsp{s}{\appl{x}{s} \Par \bout{\dual{s}}{#2} \inact}}
\newcommand{\fotrigger}[5]{\binp{#1}{#2} \newsp{#3}{\map{#4}^{#3} \Par \bout{\dual{#3}}{#5} \inact}}
%\newcommand{\fotrigger}[2]{\binp{#1}{X} \appl{X}{#2}}

%%%%%% Typed relations

\newcommand{\horel}[6]{#1; #2 \proves #3 #4 #5 \proves #6}
%\newcommand{\horel}[6]{#1; \es; #2 \proves #3 #4 #5 \proves #6}

\newcommand{\mhorel}[7]{
	\begin{array}{rcll}
		#1; \es; #2 &#4& #5 \proves& #3\\
			&#4& #6 & #7
	\end{array}
}

%%%%%%%%%%%%%%%%%%%%%%%%%%%%%%%%%%%%%%%%%%%%%%%%%%%%%%%%%%%%%%%%%%%%%%%%%%%%%%%%%%%%%%%%%%%%%%%%%%%%
%                                    LN TRANSFORM
%%%%%%%%%%%%%%%%%%%%%%%%%%%%%%%%%%%%%%%%%%%%%%%%%%%%%%%%%%%%%%%%%%%%%%%%%%%%%%%%%%%%%%%%%%%%%%%%%%%%

\newcommand{\Loop}{\mathsf{Loop}}
\newcommand{\CodeBlocks}{\mathsf{CodeBlocks}}

\newcommand{\lnmap}[1]{\namemap{LN}{#1}}
\newcommand{\lnrmap}[1]{\namemap{LNR}{#1}}
\newcommand{\lnblockmap}[1]{\namemap{\mathcal{B}}{#1}}
\newcommand{\lnnonblockmap}[2]{\map{#1, #2}}

\newcommand{\mapenv}[2]{\map{#1}_{#2}}

%%%%%%%%%%%%%%%%%%%%%%%%%%%%%%%%%%%%%%%%%%%%%%%%%%%%%%%%%%%%%%%%%%%%%%%%%%%%%%%%%%%%%%%%%%%%%%%%%%%%
%                                        GENERAL TYPES
%%%%%%%%%%%%%%%%%%%%%%%%%%%%%%%%%%%%%%%%%%%%%%%%%%%%%%%%%%%%%%%%%%%%%%%%%%%%%%%%%%%%%%%%%%%%%%%%%%%%

% Values
\newcommand{\true}{\sessionfont{tt}}
\newcommand{\false}{\sessionfont{ff}}

% Typed
\newcommand{\bool}{\sessionfont{bool}}
\newcommand{\nat}{\sessionfont{nat}}



%%%%%%%%%%%%%%%%%%%%%%%%%%%%%%%%%%%%%%%%%%%%%%%%%%%%%%%%%%%%%%%%%%%%%%%%%%%%%%%%%%%%%%%%%%%%%%%%%%%%
%                                        PROCESSES NAMES SESSIONS ETC
%%%%%%%%%%%%%%%%%%%%%%%%%%%%%%%%%%%%%%%%%%%%%%%%%%%%%%%%%%%%%%%%%%%%%%%%%%%%%%%%%%%%%%%%%%%%%%%%%%%%

% Processes
\newcommand{\PP}{\ensuremath{P}}
\newcommand{\Q}{\ensuremath{Q}}
\newcommand{\R}{\ensuremath{R}}
\newcommand{\OP}{\ensuremath{\mathsf{O}}}

% Global environments
\newcommand{\En}{\ensuremath{En}}

% Session channels
\newcommand{\s}{\ensuremath{s}}
\newcommand{\ds}{\ensuremath{\dual{s}}}
%\newcommand{\Ms}[2]{\ensuremath{s}\role{#1}\role{#2}}

%Dummy channels
\newcommand{\sd}{\mathtt{sd}}
\newcommand{\shd}{\mathtt{shd}}

% Names
\newcommand{\Ia}{\ensuremath{a}}
\newcommand{\Iu}{\ensuremath{u}}

% Variables, values, expressions
\newcommand{\x}{\ensuremath{x}}
\newcommand{\y}{\ensuremath{y}}
\newcommand{\ks}{\ensuremath{k}}
\newcommand{\cc}{\ensuremath{c}}
\newcommand{\va}{\ensuremath{v}}
\newcommand{\e}{\ensuremath{e}}
\newcommand{\n}{\ensuremath{n}}

% Process Variables

\newcommand{\X}{\varp{X}}
\newcommand{\Y}{\varp{Y}}

% Roles
\newcommand{\p}{\ensuremath{\mathtt{p}}}
\newcommand{\q}{\ensuremath{\mathtt{q}}}
\newcommand{\A}{\ensuremath{A}}

% Types
\newcommand{\G}{\ensuremath{G}}
\newcommand{\gG}{\globaltype{\G}}
\newcommand{\U}{\ensuremath{U}}
\newcommand{\So}{\ensuremath{S}}
\newcommand{\T}{\ensuremath{T}}

% Queue Types
\newcommand{\M}{\ensuremath{M}}
\newcommand{\I}{\ensuremath{M_\inputsym}}
\newcommand{\Om}{\ensuremath{M_\outputsym}}
\newcommand{\Typ}{\ensuremath{\mathsf{T}}}

% Queues values
\newcommand{\h}{\ensuremath{h}}

%barbs
\newcommand{\m}{\ensuremath{\mu}}

% Contexts
\newcommand{\C}{\ensuremath{{\Bbb C}}}
\newcommand{\E}{\ensuremath{E}}

% Congruence completness - Definibility
\newcommand{\TT}{\ensuremath{T}}
\newcommand{\suc}{\textrm{succ}}
\newcommand{\fail}{\textrm{fail}}


% Set selection labels
\newcommand{\lPi}{\set{l_i:\PP_i}_{i \in I}}
\newcommand{\lGi}{\set{l_i:\G_i}_{i \in I}}
\newcommand{\lTi}{\set{l_i:\T_i}_{i \in I}}
\newcommand{\lSi}{\set{l_i:\So_i}_{i \in I}}

% Selector proof

\newcommand{\SEL}{P_\mathit{Sel}}
\newcommand{\DSEL}{P_\mathit{DSel}}
\newcommand{\Sel}{\mathsf{Sel}}
\newcommand{\IfSel}{\mathsf{IfSel}}
\newcommand{\DSel}{\mathsf{DSel}}
\newcommand{\PSel}{\mathsf{PermSel}}
\newcommand{\PIfSel}{\mathsf{PermIfSel}}
\newcommand{\PDSel}{\mathsf{PermDSel}}

%%%%%%%%%%%%%%%%%%%%%%%%%%%%%%%%%%%%%%%%%%%%%%%%%%%%%%%%%%%%%%%%%%%%%%%%%%%%%%%%%%%%%%%%%%%%%%%%%%%%
%                                        ENVIRONMENTS
%%%%%%%%%%%%%%%%%%%%%%%%%%%%%%%%%%%%%%%%%%%%%%%%%%%%%%%%%%%%%%%%%%%%%%%%%%%%%%%%%%%%%%%%%%%%%%%%%%%%
\newtheorem{fact}{Fact}[section]
\newtheorem{notation}{Notation}

%\newenvironment{notation}{\paragraph{{\bf Notation}}}{}

%\newtheorem{proposition}[fact]{{\bf\em Proposition}}
%\newtheorem{example}[fact]{{\bf\em Example}}
%\newtheorem{lemma}[fact]{{\bf\em Lemma}}
%\newtheorem{corollary}[fact]{{\bf\em Corollary}}
%\newtheorem{definition}[fact]{{\bf\em Definition}}
%\newtheorem{theorem}[fact]{{\bf\em Theorem}}
%\newtheorem{remark}[fact]{{\bf\em Remark}}

\newcommand{\nonhosyntax}[1]{\colorbox{lightgray}{\ensuremath{#1}}}

\newenvironment{mytheorem}{%\vspace{-3pt}
	\begin{theorem}
}{%\vspace{-4pt}
	\end{theorem}
}

\newenvironment{myproposition}{
	\begin{proposition}%\vspace{-3pt}
}{%\vspace{-4pt}
	\end{proposition}
}

\newenvironment{mycorollary}{
	\begin{corollary}%\vspace{-3pt}
}{%\vspace{-4pt}
	\end{corollary}
}

\newenvironment{mylemma}{
	\begin{lemma}%\vspace{-3pt}
}{%\vspace{-4pt}
	\end{lemma}
}


\newenvironment{mydefinition}{%\vspace{-3pt}
	\begin{definition}
}{%\vspace{-3pt}
	\end{definition}
}

%\newenvironment{proof}{
%	{\em Proof.}
%}{}


%\newcommand{\qed}{\ensuremath{\square}}





%%%%%%%%%%%%%%%%%%%%%%%%%%%%%%%%%%%%%%%%%%%%%%%%%%%%%%%%%%%%%%%%%%%%%%%%%%%%%%%%%%%%%%%%%%%%%%%%%%%%
%                                        MISC
%%%%%%%%%%%%%%%%%%%%%%%%%%%%%%%%%%%%%%%%%%%%%%%%%%%%%%%%%%%%%%%%%%%%%%%%%%%%%%%%%%%%%%%%%%%%%%%%%%%%
\newcommand{\Appendix}[1]{Appendix \ref{#1}}

\newcommand{\dimcom}[1]{{\bf Comment: #1 \\}}

\newcommand{\hintcom}[1]{{\bf Hint: #1 \\}}

\newif\ifny\nyfalse
%\nytrue
\newcommand{\NY}[1]
{\ifny{\color{purple}{#1}}\else{#1}\fi}

\newcommand{\KH}[1]
{\ifny{\color{brown}{#1}}\else{#1}\fi}

\newif\ifdm\dmtrue
%\dmfalse
\newcommand{\dk}[1]
{\ifdm{\color{blue}{#1}}\else{#1}\fi}

\newif\ifrhu\rhutrue
%\rhufalse
\newcommand{\rh}[1]
{\ifdm{\color{red}{#1}}\else{#1}\fi}

\newif\ifjp\jptrue
%\jpfalse
\newcommand{\jp}[1]
{\ifjp{\color{red}{#1}}\else{#1}\fi}

\newif\ifjp\jptrue
%\jpfalse
\newcommand{\jpc}[1]
{\ifjp{\color{red}{#1}}\else{#1}\fi}

\newcommand{\ENCan}[1]{\langle #1 \rangle}
\newcommand{\NI}{\noindent}


\newcommand{\syntaxvspace}{\\[1mm]}

\newcommand{\TO}[2]{#1\to #2}
\newcommand{\GS}[3]{\TO{#1}{#2}\colon \!\ENCan{#3}}

\newcommand{\ASET}[1]{\{#1\}}
\newcommand{\participant}[1]{\mathtt{#1}}
\newcommand{\CODE}[1]{{\tt #1}}

\newcommand{\AT}[2]{#1 \! : \! #2}


\newcommand{\myrm}{}


\newcommand{\secref}[1]{\S\,\ref{#1}}
\newcommand{\defref}[1]{Def.~\ref{#1}}
\newcommand{\notref}[1]{Not.~\ref{#1}}
\newcommand{\defsref}[1]{Defs.~\ref{#1}}
\newcommand{\figref}[1]{Fig.~\ref{#1}}
\newcommand{\thmref}[1]{Thm.~\ref{#1}}
\newcommand{\thmsref}[1]{Thms.~\ref{#1}}
\newcommand{\exref}[1]{Ex.~\ref{#1}}
\newcommand{\propref}[1]{Prop.~\ref{#1}}
\newcommand{\propsref}[1]{Props.~\ref{#1}}
\newcommand{\appref}[1]{App.~\ref{#1}}
\newcommand{\lemref}[1]{Lem.~\ref{#1}}



\newcommand{\stytra}[6]{\ensuremath{#1; #3 \proves #4 \hby{#2} #5 \proves #6 }}
\newcommand{\stytraarg}[7]{\ensuremath{#1; #3 \proves_{#7} #4 \hby{#2} #5 \proves_{#7} #6 }}
\newcommand{\stytraargi}[8]{\ensuremath{#1; #3 \proves_{#7} #4 \hby{#2}_{#8} #5 \proves_{#7} #6 }}
\newcommand{\wtytra}[6]{\ensuremath{#1; #3 \proves #4 \Hby{#2}  #5 \proves #6}}
\newcommand{\wtytraarg}[7]{\ensuremath{#1; #3 \proves_{#7} #4 \Hby{#2}  #5 \proves_{#7} #6 }}
\newcommand{\wtytraargi}[8]{\ensuremath{#1; #3 \proves_{#7} #4 \Hby{#2}_{#8}  #5 \proves_{#7} #6 }}
\newcommand{\wbb}[6]{\ensuremath{#1; #3 \proves #4 \wb #5 \proves #6 }}
\newcommand{\wbbarg}[7]{\ensuremath{#1; #3 \proves_{#7} #4 \wb_{#7} #5 \proves_{#7} #6 }}

\newcommand{\minussh}{\ensuremath{\mathsf{-sh}}\xspace}

\definecolor{lightgray}{gray}{0.75}

\newcommand\greybox[1]{%
  \vskip\baselineskip%
  \par\noindent\colorbox{lightgray}{%
    \begin{minipage}{\textwidth}#1\end{minipage}%
  }%
  \vskip\baselineskip%
}

%\newcommand{\myparagraph}[1]{\paragraph{\bf #1}}

\newcommand{\mapchar}[2]{\ensuremath{[\!\!(#1)\!\!]^{#2}}}
\newcommand{\omapchar}[1]{\ensuremath{[\!\!(#1)\!\!]_{\mathsf{c}}}}

\newcommand{\trigger}[3]{#1 \leftarrow\!\!\!\!\!\!\!\leftarrow #2:#3 }
\newcommand{\htrigger}[2]{#1 \Leftarrow #2}
\newcommand{\ftrigger}[3]{#1 \Leftarrow \AT{#2}{#3}}

\newcommand{\btau}{\tau_{\beta}}
\newcommand{\stau}{\tau_{s}}
\newcommand{\dtau}{\tau_{d}}



%\newcommand{\HOpp}{\ensuremath{\mathsf{HO\pi^{+}}}\xspace}
%\newcommand{\PHOp}{\ensuremath{\mathsf{HO}{\vec{\pi}}}\xspace}
%\newcommand{\PHOpp}{\ensuremath{\mathsf{HO}{\vec{\pi}}^{+}}\xspace}



% As a general rule, do not put math, special symbols or citations
% in the abstract
\begin{abstract}
The abstract goes here.
\end{abstract}

% no keywords

% For peer review papers, you can put extra information on the cover
% page as needed:
% \ifCLASSOPTIONpeerreview
% \begin{center} \bfseries EDICS Category: 3-BBND \end{center}
% \fi
%
% For peerreview papers, this IEEEtran command inserts a page break and
% creates the second title. It will be ignored for other modes.
\IEEEpeerreviewmaketitle

\section{Introduction}
\label{sec:intro}

\section{Higher-Order Session $\pi$-Calculi}
\label{sec:calculus}
% !TEX root = main.tex
\noindent 
We introduce the 
\emph{Higher-Order Session $\pi$-Calculus} (\HOp).
\HOp includes both name- and abstraction-passing 
as well as recursion; it is a subcalculus 
of the language
studied 
in~\cite{tlca07}. 
Following the literature~\cite{JeffreyR05},
for simplicity of the presentation
we concentrate on the second-order call-by-value \HOp.  
(In \secref{sec:extension} we consider extensions of 
\HOp with higher-order abstractions 
and polyadicity in name-passing/abstractions.)
%We also introduce two subcalculi of \HOp. In particular, we define the 
%core higher-order session
%calculus (\HO), which 
%%. The \HO calculus is  minimal: it 
%includes constructs for shared name synchronisation and 
%%constructs for session establish\-ment/communication and 
%(monadic) name-abstraction, but lacks name-passing and recursion.

%Although minimal, in \secref{s:expr}
%the abstraction-passing capabilities of \HOp will prove 
%expressive enough to capture key features of session communication, 
%such as delegation and recursion.

\subsection{Syntax of \HOp}
\label{subsec:syntax}
\noindent\myparagraph{Values} The syntax of \HOp is defined in \figref{fig:syntax}
\begin{figure}[t]
\[ 
\begin{array}{lll}
u,w  ::=  n \ | \ x,y,z
& n ::= a,b  \ | \ s, \dual{s} 
& V,W  ::=   \nonhosyntax{u} \ | \ \abs{x}{P}
\end{array}
\]
\[
\begin{array}{rclllll}
P,Q \!\!\!\!\!\! & ::= & \!\! \bout{u}{V}{P}  \bnfbar  \binp{u}{x}{P} \bnfbar
 \bsel{u}{l} P \bnfbar \bbra{u}{l_i:P_i}_{i \in I}   \\[1mm]
 & \bnfbar & \!\!  \nonhosyntax{\rvar{X} \bnfbar \recp{X}{P}} \bnfbar \appl{V}{u} \bnfbar P\Par Q \bnfbar \news{n} P 
\bnfbar 
%ny
%\appl{x}{u}

 \inact
%\\[1mm]
 %    & \bnfbar & \nonhosyntax{\rvar{X} \bnfbar \recp{X}{P}}
\end{array}
\]
 \caption{Syntax of \HOp (\HO lacks the constructs in \nonhosyntax{\text{grey}}).}
\label{fig:syntax}
\Hlinefig
\end{figure}
We use $a,b,c, \dots$ (resp.~$s, \dual{s}, \dots$) 
to range over shared (resp. session) names. 
We use $m, n, t, \dots$ for session or shared names. 
We define the dual operation over names $n$ as $\dual{n}$ with
$\dual{\dual{s}} = s$ and $\dual{a} = a$.
Intuitively, names $s$ and $\dual{s}$ are dual (two) \emph{endpoints} while 
shared names represent shared (non-deterministic) points. 
Variables are denoted with $x, y, z, \dots$, 
and recursive variables are denoted with $\varp{X}, \varp{Y} \dots$.
An abstraction %(or higher-order value) 
$\abs{x}{P}$ is a process $P$ with name parameter $x$.
%Symbols $u, v, \dots$ range over identifiers; and  $V, W, \dots$ to denote values. 
Values $V,W$ include 
identifiers $u, v, \ldots$ %(first-order values) 
and 
abstractions $\abs{x}{P}$ (first- and higher-order values, resp.). 

\myparagraph{Terms} 
include the
$\pi$-calculus prefixes for sending and receiving values $V$.
Process $\bout{u}{V} P$ denotes the output of value $V$
over name $u$, with continuation $P$;
process $\binp{u}{x} P$ denotes the input prefix on name $u$ of a value
that 
will substitute variable $x$ in continuation $P$. 
Recursion is expressed by $\recp{X}{P}$,
which binds the recursive variable $\varp{X}$ in process $P$.
Process 
%ny
%$\appl{x}{u}$ 
$\appl{V}{u}$ 
is the application
which substitutes name $u$ on the abstraction~$V$. 
\dk{Typing  ensures \jpc{that} $V$ is not a name.}
Prefix $\bsel{u}{l} P$ selects label $l$ on name $u$ and then behaves as $P$.
%Given $i \in I$ 
Process $\bbra{u}{l_i: P_i}_{i \in I}$ offers a choice on labels $l_i$ with
continuation $P_i$, given that $i \in I$.
%Others are standard c
Constructs for 
inaction $\inact$,  parallel composition $P_1 \Par P_2$, and 
name restriction $\news{n} P$ are standard.
Session name restriction $\news{s} P$ simultaneously binds endpoints $s$ and $\dual{s}$ in $P$.
%A well-formed process relies on assumptions for
%guarded recursive processes.
We use $\fv{P}$ and $\fn{P}$ to denote a set of free 
%\jpc{recursion}
variables and names; 
and assume $V$ in $\bout{u}{V}{P}$ does not include free recursive 
variables $\rvar{X}$. 
If $\fv{P} = \emptyset$, we call $P$ {\em closed}.
%; and closed $P$ without 
%free session names a {\em program}. 

\subsection{Subcalculi of \HOp}
\label{subsec:subcalculi}
\noi
We define two subcalculi of \HOp. 
%We now define several sub-calculi of \HOp. 
The first is the 
{\em core higher-order session calculus} (denoted \HO),
which lacks recursion and name passing; its 
formal syntax is obtained from \figref{fig:syntax} by excluding 
constructs in \nonhosyntax{\text{grey}}.
The second subcalculus is 
the {\em session $\pi$-calculus} 
(denoted $\sessp$), which 
lacks  the
higher-order constructs
(i.e., abstraction passing and application), but includes recursion.
Let $\CAL \in \{\HOp, \HO, \sessp\}$. We write 
$\CAL^{-\mathsf{sh}}$ for $\CAL$ without shared names
(we delete $a,b$ from $n$). 
We shall demonstrate that 
$\HOp$, $\HO$, and $\sessp$ have the same expressivity.

\subsection{Operational Semantics}
\label{subsec:semantics}
\begin{figure}
\[
\begin{array}{rclrcrclr}
(\abs{x}{P}) \, u  & \red & P \subst{u}{x} 
& \orule{App}
		\\[1mm]
%\bout{a}{V} P \Par \binp{a}{x} Q & \red & P \Par Q \subst{V}{x} 
%& \orule{Com}
%		\\[1mm]
\bout{n}{V} P \Par \binp{\dual{n}}{x} Q & \red & P \Par Q \subst{V}{x} 
& \orule{Pass}
		\\[1mm]
		\bsel{n}{l_j} Q \Par \bbra{\dual{n}}{l_i : P_i}_{i \in I} & \red & Q \Par P_j ~~(j \in I)~~  & \orule{Sel}\\[1mm]
		P \red P' & \Rightarrow & \news{n} P  \red  \news{n} P'  & \orule{Res}\\[1mm]
			P \red P' & \Rightarrow  &  P \Par Q  \red   P' \Par Q  & \orule{Par}\\[1mm]
			P \scong Q \red Q' \scong P' & \Rightarrow & P  \red  P' & \orule{Cong}
	\end{array}
\]
{\small
\[
	\begin{array}{c}
		P \Par \inact \scong P
		\quad
		P_1 \Par P_2 \scong P_2 \Par P_1
		\quad
		P_1 \Par (P_2 \Par P_3) \scong (P_1 \Par P_2) \Par P_3
		%\quad
		%P \scong Q \textrm{ if } P \scong_\alpha Q
		\\[1mm]
		\news{n} \inact \scong \inact
		\quad 
		P \Par \news{n} Q \scong \news{n}(P \Par Q)
		\	(n \notin \fn{P})\quad 
		\recp{X}{P} \scong P\subst{\recp{X}{P}}{\rvar{X}}
		\\[1mm]
		P \scong Q \textrm{ if } P \scong_\alpha Q
%		\qquad
%		\dk{V \scong W \textrm{ if } V \scong_\alpha W
%\quad \abs{x}{P} \scong \abs{x}{Q} \textrm{ if } P \scong Q}
	\end{array}
\]
}
\caption{Operational Semantics of $\HOp$. 
\label{fig:reduction}}
\Hlinefig
\end{figure}
\noindent \figref{fig:reduction} defines the operational semantics 
of \HOp.
$\orule{App}$ is a name application; 
$\orule{Pass}$ defines a shared interaction at $n$ 
(\jpc{with} $\dual{n}=n$) or a session interaction;  
$\orule{Sel}$ is the standard rule for labelled choice/selection:
given an index set $I$, 
a process selects label $l_j$ on name $n$ over a set of
labels $\set{l_i}_{i \in I}$ offered by a branching 
on the dual endpoint $\dual{n}$; and other rules are standard.
Rules for \emph{structural congruence} are defined in \figref{fig:reduction} (bottom). 
\jpc{We assume the expected extension of $\scong$ to values $V$.}
We write $\red^\ast$ for a multi-step reduction. 


\section{Types and Sub-Calculi}
\label{sec:types}
\section{Types for $\HOp$}

We define a session type system for $\HOp$, which is based on the type system developed by Mostrous
and Yoshida in~\cite{tlca07}.

\subsection{Session Types}
We consider a minimal type structure, a fragment of that defined in~\cite{tlca07}.
The only (but fundamental) differences are in the types for values: we focus on having 
$\shot{S}$ and $\lhot{S}$, whereas the structure in~\cite{tlca07} supports general functions $U \sharedop T$ and 
$U \lollipop T$.
\[
	\begin{array}{lcl}
		\text{\emph{Values}} & U ::= & S \bnfbar \lhot{S} \bnfbar \shot{S} \bnfbar \chtype{S} \qquad \quad \text{\emph{Terms}} \quad T ::= U  \bnfbar  \Proc\\
		\text{\emph{Sessions}} \ & S ::= &  \btout{U} S \bnfbar \btinp{U} S
		\bnfbar		\btsel{l_i:S_i}_{i \in I} \bnfbar \btbra{l_i:S_i}_{i \in I} \bnfbar \trec{t}{S} \bnfbar \vart{t}  \bnfbar \tinact 
	\end{array}
\]

There are four different value types $U$; session value $S$, linear higher order value $\lhot{S}$, 
shared higher order value $\shot{S}$; shared channel $\chtype{S}$. Terms can either have a
value type $U$ or a process type $\Proc$.

Session types follow the standard binary session types syntax \cite{}. Session send prefix $\btout{U} S$ 
denotes a session type that sends a value of type $U$ and continues as $S$. Dually receive prefix $\btinp{U} S$
denotes a session type that receives a value of type $U$ and continues as $S$. 
Set $\mathsf{ST}$ is the space of all session types.

\begin{definition}[Duality]
	Let function $F(R): \mathsf{ST} \longrightarrow \mathsf{ST}$ to be defined as:

	\begin{tabular}{rcl}
		$F(R)$ &$=$&		$\set{(\tinact, \tinact), (\vart{t}, \vart{t})}$\\
			&$\cup$&	$\set{(\btout{U};S_1, \btinp{U}; S_2), (\btinp{U};S_1, \btout{U}; S_2) \bnfbar S_1\ R\ S_2}$\\
			&$\cup$&	$\set{(\btsel{l_i: S_i}_{i \in I}, \btbra{l_j: S_j'}_{j \in J}) \bnfbar I \subseteq J, S_i\ R\ S_j'}$\\
			&$\cup$&	$\set{(\btbra{l_i: S_i}_{i \in I}, \btsel{l_j: S_j'}_{j \in J}) \bnfbar J \subseteq I, S_j\ R\ S_i'}$\\
			&$\cup$&	$\set{(\trec{t}{S_1}, \trec{t}{S_2}) \cup (S_1 \subst{\trec{t}{S}}{\vart{t}}, S_2), (S_1, S_2\subst{\trec{t}{S}}{\vart{t}}) \bnfbar S_1\ R\ S_2)}$
	\end{tabular}

	Standard arguments ensure that $F$ is monotone, thus the greatest fix point
	of $F$ exists and let duality defined as $\dualof = \nu X. F(X)$.
\end{definition}


Following our decision of focusing on functions $\shot{S}$ and $\lhot{S}$,
our environments are also simpler than those in~\cite{tlca07}:
\[
	\begin{array}{lcl}
		\text{Shared} & \Gamma \bnfis & \emptyset \bnfbar \Gamma \cat \varp{X}: \shot{S} \bnfbar \Gamma \cat k: \chtype{S} \bnfbar \Gamma \cat \rvar{X}: \Sigma \\
		\text{Linear} & \Lambda \bnfis & \emptyset \bnfbar \Lambda \cat \varp{X}: \lhot{S}  \\
		\text{Session} \quad & \Sigma \bnfis & \emptyset \bnfbar \Sigma \cat k:S  
	\end{array}
\]


With these environments the shape of judgments is exactly the same as in Mostrous and Yoshida's system:
\[
	\begin{array}{c}
		\Gamma; \Lambda ; \Sigma \proves P \hastype T
%		\Gamma; \Lambda; \Sigma \proves V \hastype U
	\end{array}
\]
As expected, weakening, contraction, and exchange principles apply to $\Gamma$;
environments $\Lambda$ and $\Sigma$ behave linearly, and are only subject to exchange.
We require that the domains of $\Gamma, \Lambda$ and $\Sigma$ are pairwise distinct.
%We focus on \emph{well-formed} judgments, which do not share elements in their domains.
%\newcommand{\jrule}[3]{\displaystyle \frac{#1 }{#2} & \trule{#3}}
\newcommand{\jrule}[3]{\displaystyle \trule{#3}~~\frac{#1 }{#2}}
\newcommand{\addenv}{\circ}

\begin{figure}[!t]
\[
	\begin{array}{c}
%		\jrule{ }{\Gamma ; \emptyset; \emptyset \vdash \UnitV \hastype \Unit}{Unit} 
%		\qquad\quad  
		\trule{Session}~~\Gamma; \emptyset; \set{k:S} \proves k \hastype S 
		\\[2mm]
		\trule{Shared}~~\Gamma \cat a : \chtype{S}; \emptyset; \emptyset \proves a \hastype \chtype{S}
		\qquad
		\trule{LVar}~~\Gamma; \set{X: \lhot{S}}; \emptyset \proves X \hastype \lhot{S} 
		\\[2mm]
		\trule{Prom}~~\tree{
			\Gamma; \emptyset; \emptyset \proves V \hastype \lhot{S}
		}{
			\Gamma; \emptyset; \emptyset \proves V \hastype \shot{S}
		} 
		\qquad\quad  
		\trule{Derelic}~~\tree{
			\Gamma; \Lambda \cat X{:}\lhot{S}; \Sigma \proves P \hastype \Proc
		}{
			\Gamma \cat X:\shot{S}; \Lambda; \Sigma \proves P \hastype \Proc
		} 
		\\[4mm]
%		\trule{Subt}~~\tree{
%			\Gamma; \Lambda; \Sigma \proves P \hastype T \quad \Sigma \subt \Sigma' \quad T \subt T'
%		}{
%			\Gamma ; \Lambda; \Sigma' \vdash P \hastype T'
%		} 
%		\qquad\quad

		\trule{Abs}~~\tree{
			\Gamma; \Lambda; \Sigma \cat x: S \proves P \hastype \Proc
		}{
			\Gamma; \Lambda; \Sigma \proves \abs{x}{P} \hastype \lhot{S}
		}
		\\[4mm]

		\trule{App}~~\tree{(U = \lhot{S}) \lor (U = \shot{S}) \quad \Gamma; \Lambda_1; \Sigma_1 \proves X \hastype U  \quad \Gamma; \Lambda_2; \Sigma_2 \proves k \hastype S
		}{
			\Gamma; \Lambda_1 \cup \Lambda_2; \Sigma_1 \cup \Sigma_2 \proves \appl{X}{k} \hastype \Proc
		} 
		\\[4mm]

		\trule{Send}~~\tree{
			\Gamma; \Lambda_1; \Sigma_1 \proves P \hastype \Proc  \quad \Gamma; \Lambda_2; \Sigma_2 \vdash V \hastype U  \quad (k:S \in \Sigma_1 \cup \Sigma_2)
		}{
			\Gamma; \Lambda_1 \cup \Lambda_2; (\Sigma_1 \cup \Sigma_2)\backslash\set{k:S} \cat k:\btout{U} S \proves \bout{k}{V} P \hastype \Proc
		}

		\\[4mm]
		\trule{Conn}~~\tree{
			\Gamma; \Lambda; \Sigma \cat x:S \proves P \hastype \Proc  \quad \Gamma; \emptyset; \emptyset \proves a \hastype \chtype{S}
		}{
			\Gamma; \Lambda; \Sigma \proves \binp{a}{x} P \hastype \Proc
		}
		\\[4mm]
%		\trule{ConnDual}~~\tree{
%			\Gamma; \Lambda; \Sigma \cat x: S_1 \proves P \hastype \Proc  \quad \Gamma; \emptyset; \emptyset \proves k \hastype \chtype{S_2} \quad S_1 \dualof S_2
%		}{
%			\Gamma; \Lambda; \Sigma \proves \bout{k}{x} P \hastype \Proc
%		}
%		\\[4mm]

		\trule{ConnDual}~~\tree{
			\Gamma; \Lambda; \Sigma \cat \dual{s}: S_1 \proves P \hastype \Proc  \quad \Gamma; \emptyset; \emptyset \proves a \hastype \chtype{S_2} \quad S_1 \dualof S_2
		}{
			\Gamma; \Lambda; \Sigma  \proves \bout{a}{\dual{s}} P \hastype \Proc
		}

		\\[4mm]

		\trule{NewSh}~~\tree{
			\Gamma\cat a:\chtype{S} ; \Lambda; \Sigma \proves P \hastype \Proc
		}{
			\Gamma; \Lambda; \Sigma \proves \news{a} P \hastype \Proc}
		\qquad\quad
		\trule{NewSes}~~\tree{
			\Gamma; \Lambda; \Sigma \cat s:S_1 \cat \dual{s}: S_2 \proves P \hastype \Proc \quad S_1 \dualof S_2
		}{
			\Gamma; \Lambda; \Sigma \proves \news{s} P \hastype \Proc
		}
		\\[4mm]

		\trule{RecvS}~~\tree{
			\Gamma; \Lambda; \Sigma \cat k: S_1 \cat x: S_2 \proves P \hastype \Proc
		}{
			\Gamma; \Lambda; \Sigma, k: \btinp{S_2} S_1  \vdash \binp{k}{x}P \hastype \Proc
		}
		\quad\quad 
		\trule{RecvL}~~\tree{
			\Gamma; \Lambda \cat X: \lhot{S}; \Sigma \cat k: S_1  \proves P \hastype \Proc
		}{
			\Gamma; \Lambda; \Sigma \cat k:\btinp{\lhot{S}}S_1  \proves \binp{k}{X}P \hastype \Proc
		}
		\\[4mm]

		\trule{RecvSh}~~\tree{
			\Gamma \cat X: \shot{S}; \Lambda; \Sigma \cat k: S_1  \proves P \hastype \Proc
		}{
			\Gamma; \Lambda; \Sigma \cat k:\btinp{\shot{S}}S_1  \proves \binp{k}{X}P \hastype \Proc
		}
		\quad ~~
		\trule{RecvShN}~~\tree{
			\Gamma \cat x: \chtype{S}; \Lambda; \Sigma \cat k: S_1  \proves P \hastype \Proc
		}{
			\Gamma; \Lambda; \Sigma \cat k:\btinp{\chtype{S}}S_1  \proves \binp{k}{x}P \hastype \Proc
		}
		\\[4mm]
		\trule{Par}~~\tree{
			\Gamma; \Lambda_{1}; \Sigma_{1} \proves P_{1} \hastype \Proc \quad \Gamma; \Lambda_{2}; \Sigma_{2} \proves P_{2} \hastype \Proc
		}{
			\Gamma; \Lambda_{1} \cup \Lambda_2; \Sigma_{1} \cup \Sigma_2 \proves P_1 \Par P_2 \hastype \Proc
		}
		\qquad\quad
		\trule{Close}~~\tree{
			\Gamma; \Lambda; \Sigma  \proves P \hastype T \quad k \not\in \dom{\Gamma, \Lambda,\Sigma}
		}{
			\Gamma; \Lambda; \Sigma \cat k: \tinact  \proves P \hastype \Proc
		}
		\\[4mm]
		\trule{Bra}~~\tree{
			 \forall i \in I \quad \Gamma; \Lambda; \Sigma \cat k:S_i \proves P_i \hastype \Proc
		}{
			\Gamma; \Lambda; \Sigma \cat k: \btbra{l_i:S_i}_{i \in I} \proves \bbra{k}{l_i:P_i}_{i \in I}\hastype \Proc
		}
		\qquad\quad 
	 	\trule{Sel}~~\tree{
			\Gamma; \Lambda; \Sigma \cat k: S_j  \proves P \hastype \Proc \quad j \in I
		}{
			\Gamma; \Lambda; \Sigma \cat k:\btsel{l_i:S_i}_{i \in I} \proves \bsel{s}{l_j} P \hastype \Proc
		}
		\\[4mm]

		\trule{Nil}~~\Gamma; \emptyset; \emptyset \proves \inact \hastype \Proc
\qquad \quad
		\trule{Var}~~\tree{
	
		}{
			\Gamma \cat \rvar{X}: \Sigma; \emptyset; \emptyset  \proves \rvar{X} \hastype \Proc
		}
		\qquad\quad 
%	 	\trule{Rec}~~\tree{
%			\Gamma \cat \rvar{X}: \Sigma; \emptyset; \emptyset  \proves P \hastype \Proc
%		}{
%			\Gamma ; \emptyset; \emptyset  \proves \recp{X}{P} \hastype \Proc
%		}
%		\\[4mm]

	 	\trule{Rec}~~\tree{
			\Gamma \cat \rvar{X}: \Sigma; \emptyset; \Sigma  \proves P \hastype \Proc
		}{
			\Gamma ; \emptyset; \Sigma  \proves \recp{X}{P} \hastype \Proc
		}


	\end{array}
\]
\caption{Typing Rules for $\HOp$\label{fig:typerulesmy}}
\end{figure}


The typing rules for our system are in Fig.~\ref{fig:typerulesmy}. 


%%%%%%%%%%%%%%%%%%%%%%%%%%%%%%%%%%%%%%%%%%%%%%%%%%%%%%%%%%%%%%%%%%%%%%%%%%%%%
% Bibliography.
%%%%%%%%%%%%%%%%%%%%%%%%%%%%%%%%%%%%%%%%%%%%%%%%%%%%%%%%%%%%%%%%%%%%%%%%%%%%%

\bibliographystyle{IEEEtran}
\bibliography{session}


\end{document}


