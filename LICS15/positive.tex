% !TEX root = main.tex
\noi This section presents two encodabilty results:
(1)~The higher-order name-passing communication with recursions (\HOp) into 
the higher-order communication without name-passing nor 
recursions (\HO) (\secref{subsec:HOpi_to_HO}); and 
(2)~\HOp into the first-order name-passing communication
with recursions (\sessp) (\secref{subsec:HOp_to_sessp}). 

 


\subsection{From \HOp to \HO}
\label{subsec:HOpi_to_HO}
\noi We show that $\HO$ is expressive enough to
represent the full
 \HOp-calculus.
The main challenge is to encode (1) name passing 
and (2) recursions, 
for which 
we only use name abstraction passing. For (1), we pass  
an % simple 
abstraction which enables to use the name upon application. 
For (2), we 
copy a process upon reception; the case of linear abstraction passing
%presents a limitation 
is \NY{delicate} 
because 
linear abstractions cannot be copied.
To handle linearity, we define 
%a preliminary tool which is a mapping from
\jpc{an auxiliary mapping}
from processes \jpc{with free names} to processes without free names (but with free variables) (\defref{d:auxmap}). 
We first require an auxiliary definition.

\smallskip 

\begin{definition}\rm 
\label{def:hop_to_ho}
	Let $\vmap{\cdot}: 2^{\mathcal{N}} \longrightarrow \mathcal{V}^\omega$
	be a map of sequences of 
lexicographically ordered names to sequences of variables, defined
	inductively as: 
	$\vmap{\epsilon} = \epsilon$ and $\vmap{n \cat \tilde{m}} = x_n \cat \vmap{\tilde{m}}$. 
\end{definition}

\smallskip 

\noi The following auxiliary mapping transforms processes
with free names into abstractions and it is
used in \defref{d:enc:hopitoho}.

\smallskip 

\begin{definition}[Auxiliary Mapping] \label{d:trabs}\label{d:auxmap}
	Let $\sigma$ be a set of session names.
	\figref{f:auxmap} defines an auxiliary mapping
	$\auxmapp{\cdot}{{}}{\sigma}: \HO \to \HO$.
\end{definition}

%
\begin{figure}[t]
\[
\small
\begin{array}{rl}
	\auxmapp{\bout{n}{\abs{x}{Q}} P}{{}}{\sigma} &\!\!\!\!\!\!\defeq
		\bout{u}{\abs{x}{\auxmapp{Q}{{}}{\sigma}}} \auxmapp{P}{{}}{\sigma}
\\[1mm]
%\auxmapp{\bout{n}{m} P}{{}}{\sigma} \defeq
%	    \bout{u}{v}\auxmapp{P}{{}}{\sigma} 

	\auxmapp{\appl{x}{n}}{{}}{\sigma}  \defeq
		\appl{x}{u} &
	\auxmapp{\appl{(\lambda x.Q)}{n}}{{}}{\sigma}  \defeq
		\appl{(\lambda x.\auxmapp{Q}{{}}{\sigma})}{u} \\
	\auxmapp{\inact}{{}}{\sigma}  \defeq  \inact
 & 
			\auxmapp{\binp{n}{x} P}{{}}{\sigma}\defeq
		\binp{u}{x} \auxmapp{P}{{}}{\sigma} 
\\[1mm]
	\auxmapp{\bsel{n}{l} P}{{}}{\sigma} \defeq
		\bsel{u}{l} \auxmapp{P}{{}}{\sigma} 
 & 
	\auxmapp{\bbra{n}{l_i:P_i}_{i \in I}}{{}}{\sigma}  \defeq 
		\bbra{u}{l_i:\auxmapp{P_i}{{}}{\sigma}}_{i \in I}
	\vspace{1mm} \\
\auxmapp{\news{n} P}{{}}{\sigma}  \defeq  \news{n} \auxmapp{P}{{}}{{\sigma \cat n}}
 & 
	\auxmapp{P \Par Q}{{}}{\sigma}  \defeq  \auxmapp{P}{{}}{\sigma} \Par \auxmapp{Q}{{}}{\sigma} 
\end{array}
\]
%\[
%	\begin{array}{rcl}
%          \auxmapp{\news{n} P}{{}}{\sigma} &\bnfis& \news{n} \auxmapp{P}{{}}{{\sigma \cat n}}
%		\vspace{1mm} \\
%		\auxmapp{\bout{n}{\abs{x}{Q}} P}{{}}{\sigma} &\bnfis&
%		\left\{
%		\begin{array}{rl}
%			\bout{x_n}{\abs{(x,\vmap{\fn{P}})}{\auxmapp{Q}{{}}{\sigma}}} \auxmapp{P}{{}}{\sigma} & n \notin \sigma\\
%			\bout{n}{\abs{(x,\vmap{\fn{P}})}{\auxmapp{Q}{{}}{\sigma}}} \auxmapp{P}{{}}{\sigma} & n \in \sigma
%		\end{array}
%		\right.
%			\vspace{1mm}	\\ 
%		\auxmapp{\bout{n}{m} P}{{}}{\sigma} &\bnfis&
%		\left\{
%		\begin{array}{rl}
%		    \bout{n}{m}\auxmapp{P}{{}}{\sigma} & n, m \in \sigma \\
%		    \bout{x_n}{m}\auxmapp{P}{{}}{\sigma} & n \not\in \sigma, m \in \sigma \\
%		    \bout{n}{x_m}\auxmapp{P}{{}}{\sigma} & n \in \sigma, m \not\in \sigma \\
%		    \bout{x_n}{x_m}\auxmapp{P}{{}}{\sigma} & n, m \not\in \sigma 
%		\end{array}
%		\right.
%		\vspace{1mm} \\ 
%				\auxmapp{\binp{n}{X} P}{{}}{\sigma} &\bnfis&
%		\left\{
%		\begin{array}{rl}
%			\binp{x_n}{X} \auxmapp{P}{{}}{\sigma} & n \notin \sigma\\
%			\binp{n}{X} \auxmapp{P}{{}}{\sigma} & n \in \sigma
%		\end{array}
%		\right.
%			\vspace{1mm}	\\ 
%		\auxmapp{\binp{n}{x}P}{{}}{\sigma} &\bnfis&
%		\left\{
%		\begin{array}{rl}
%		    \binp{n}{x}\auxmapp{P}{{}}{\sigma} & n \in \sigma \\
%		    \binp{x_n}{x}\auxmapp{P}{{}}{\sigma} & n \not\in \sigma 
%		\end{array}
%		\right.
%		\vspace{1mm} \\ 
%		\auxmapp{\bsel{n}{l} P}{{}}{\sigma} &\bnfis&
%		\left\{
%		\begin{array}{rl}
%			\bsel{x_n}{l} \auxmapp{P}{{}}{\sigma} & n \notin \sigma\\
%			\bsel{n}{l} \auxmapp{P}{{}}{\sigma} & n \in \sigma
%		\end{array}
%		\right.
%		\vspace{1mm} \\
%		\auxmapp{\bbra{n}{l_i:P_i}_{i \in I}}{{}}{\sigma} &\bnfis&
%		%\auxmapp{\bsel{n}{l} P}{{}}{\sigma} &\bnfis&
%		\left\{
%		\begin{array}{rl}
%			\bbra{x_n}{l_i:\auxmapp{P_i}{{}}{\sigma}}_{i \in I}  & n \notin \sigma\\
%			\bbra{n}{l_i:\auxmapp{P_i}{{}}{\sigma}}_{i \in I}  & n \in \sigma
%		\end{array}
%		\right.
%		\vspace{1mm} \\
%		\auxmapp{\appl{\X}{n}}{{}}{\sigma} &\bnfis&
%		\left\{
%		\begin{array}{rl}
%			\appl{\X}{x_n} & n \notin \sigma\\
%			\appl{\X}{n} & n \in \sigma\\
%		\end{array}
%		\right. \\
%		\auxmapp{\inact}{{}}{\sigma} &\bnfis& \inact\\
%		\auxmapp{P \Par Q}{{}}{\sigma} &\bnfis& \auxmapp{P}{{}}{\sigma} \Par \auxmapp{Q}{{}}{\sigma} 
% \end{array}
%\]
%The auxiliary map (cf. \defref{d:auxmap}) 
%used in the encoding of the higher-order communication 
%with recursive definitions into higher-order communication 
%without recursive definitions and (\defref{d:enc:fotohorec}).
$u = n$ if $n\in \sigma$; otherwise $u = x_n$. \\
%The mapping is defined homomorphically for inaction and parallel composition.
\caption{\label{f:auxmap} Auxiliary mapping used to encode \HOp into \HO.}
\Hlinefig 
\end{figure}

\smallskip 

\noi Given a process $P$ with $\fn{P} = m_1, \cdots, m_n$, we are interested in its associated abstraction, which is defined as
$\abs{x_1\cdots x_n}{\auxmapp{P}{{}}{\emptyset} }$, where $\vmap{m_j} = x_j$, for all $j \in \{1, \ldots, n\}$. 

%This transformation from processes into abstractions can be reverted by
%using abstraction and application with an appropriate sequence of session names:
%%
%\begin{proposition}\rm
%	Let $P$ be a \HOp process and 
%	suppose $\tilde{x} = \vmap{\tilde{n}}$ where 
%$\tilde{n} = \fn{P}$.
%	Then $P \scong \appl{(\abs{\tilde{x}}{\auxmapp{P}{{}}{\emptyset}})}{\tilde{n}}$.
%%	$\appl{X}{\smap{\fn{P}}} \subst{(\vmap{\fn{P}}) \map{P}^{\emptyset}}{X} \scong P$
%\end{proposition}

\begin{figure}[t]
\[
\begin{array}{rcll}
\noindent{\bf Types:}\quad\quad\quad\quad\quad\quad
	\vtmap{{S}}{1}	&\!\!\!\!\!\!\defeq\!\!\!\!\!\!&	\lhot{(\btinp{\lhot{\tmap{S}{1}}} \tinact)} \\
	\vtmap{\chtype{S}}{1}&\!\!\!\!\!\!\defeq\!\!\!\!\!\!&	\lhot{(\btinp{\shot{\chtype{\tmap{S}{1}}}} \tinact)}  \\
	\vtmap{\chtype{L}}{1}&\!\!\!\!\!\!\defeq\!\!\!\!\!\!&	\lhot{(\btinp{\shot{\chtype{\tmap{L}{1}}}} \tinact)} \\
	\vtmap{\lhot{C}}{1} \!\!\defeq\!\! \lhot{\tmap{C}{1}}&& 
	\vtmap{\shot{C}}{1} \!\!\defeq\!\! \shot{\tmap{C}{1}}\\
	\tmap{\chtype{S}}{1} \!\!\defeq\!\!	\chtype{\tmap{S}{1}}  &&
	\tmap{\chtype{L}}{1} \!\!\defeq\!\!	\chtype{\tmap{L}{1}}  \\
	%		\tmap{\btout{S_1} {S} }{1}	&\!\!\defeq\!\!&	\bbtout{\lhot{\btinp{\lhot{\tmap{S_1}{1}}}\tinact}} \tmap{S}{1}  \\
	%		\tmap{\btinp{S_1} S }{1}	&\!\!\defeq\!\!&	\bbtinp{\lhot{\btinp{\lhot{\tmap{S_1}{1}}}\tinact}} \tmap{S}{1} \\
	%		\tmap{\bbtout{\chtype{U}} {S} }{1}	&\!\!\defeq\!\!&	\bbtout{\shot{\btinp{\shot{\chtype{\tmap{U}{1}}}}\tinact}} \tmap{S}{1}  \\
	%		\tmap{\bbtinp{\chtype{U}} {S} }{1}	&\!\!\defeq\!\!&	\bbtinp{\shot{\btinp{\shot{\chtype{\tmap{U}{1}}}}\tinact}} \tmap{S}{1} \\

	\tmap{\btout{U} S}{1} \!\!\defeq\!\! \btout{{\vtmap{U}{1}}} \tmap{S}{1}&&
	\tmap{\btinp{U} S}{1} \!\!\defeq\!\! \btinp{{\vtmap{U}{1}}} \tmap{S}{1}\\
	\tmap{\btsel{l_i: S_i}_{i \in I}}{1} &\!\!\!\!\defeq\!\!\!\!& \btsel{l_i: \tmap{S_i}{1}}_{i \in I}\\
			\tmap{\btbra{l_i: S_i}_{i \in I}}{1} &\!\!\!\!\defeq\!\!\!\!& \btbra{l_i: \tmap{S_i}{1}}_{i \in I}\\
	\tmap{\vart{t}}{1} \defeq \vart{t} \quad 
			\tmap{\trec{t}{S}}{1}  &\!\!\!\!\defeq\!\!\!\!&
	\trec{t}{\tmap{S}{1}}\quad 
	\tmap{\tinact}{1}  \defeq  \tinact\\[1mm]
	\hline
	%\end{array}
	%\]
	%\[
	%\begin{array}{rcll}
	\noindent{\bf Labels:} \quad \quad
		\mapa{(\nu \tilde{m})\bactout{n}{m}}^{1} &\!\!\!\!\defeq\!\!\!\!&   
(\nu \tilde{m})\bactout{n}{\abs{z}{\,\binp{z}{x} (\appl{x}{m})} } \\
		\mapa{\bactinp{n}{m}}^{1} &\!\!\!\!\defeq\!\!\!\!&   \bactinp{n}{\abs{z}{\,\binp{z}{x} (\appl{x}{m})} } \\
	\mapa{(\nu \tilde{m})\bactout{n}{\abs{{x}}{P}}}^{1} &\!\!\!\!\defeq\!\!\!\!& 
(\nu \tilde{m})\bactout{n}{\abs{{x}}{\pmapp{P}{1}{\es}}}\\
			\mapa{\bactinp{n}{\abs{{x}}{P}}}^{1} &\!\!\!\!\defeq\!\!\!\!& \bactinp{n}{\abs{{x}}{\pmapp{P}{1}{\es}}}\\
			\mapa{\bactsel{n}{l} }^{1} \!\!\defeq\!\! \bactsel{n}{l} 
	\quad 
			\mapa{\bactbra{n}{l} }^{1} &\!\!\!\!\defeq\!\!\!\!& \bactbra{n}{l} 
	\quad \quad 
			\mapa{\tau}^{1} \!\!\defeq\!\! \tau
\\[1mm]
\hline
\end{array}
\]
\noi{\bf Terms} : \\
$
\begin{array}{rcll}
  \pmapp{\bout{u}{w} P}{1}{f}	&\!\!\defeq\!\!&	\bout{u}{ \abs{z}{\,\binp{z}{x} (\appl{x}{w})} } \pmapp{P}{1}{f} \\
  \pmapp{\binp{u}{\AT{x}{C}} Q}{1}{f}	&\!\!\defeq\!\!&	\binp{u}{y} \newsp{s}{\appl{y}{s} \Par \bout{\dual{s}}{\abs{x}{\pmapp{Q}{1}{f}}} \inact} \\
		\pmapp{\bout{u}{\abs{{x}}{Q}} P}{1}{f}  
&\!\!\defeq\!\!& \bout{u}{\abs{{x}}{\pmapp{Q}{1}{f}}} \pmapp{P}{1}{f} \\
		\pmapp{\binp{u}{\AT{x}{L}} P}{1}{f} &\!\!\defeq\!\!& \binp{u}{x} \pmapp{P}{1}{f}\\
		\pmapp{\bsel{s}{l} P}{1}{f} &\!\!\defeq\!\!& \bsel{s}{l} \pmapp{P}{1}{f}\\
		\pmapp{\bbra{s}{l_i: P_i}_{i \in I}}{1}{f} &\!\!\defeq\!\!& \bbra{s}{l_i: \pmapp{P_i}{1}{f}}_{i \in I}\\
		\pmapp{\inact}{1}{f} \!\!\defeq\!\!\inact
& & 
		\pmapp{\news{n} P}{1}{f} \!\!\defeq\!\! \news{n} \pmapp{P}{1}{f}\\
\pmapp{{x}\, {u}}{1}{f}
 \!\!\defeq\!\!
{x}\, {u}
& & 		
\pmapp{\appl{(\lambda x.Q)}{u}}{1}{f}
 \!\!\defeq\!\!
\appl{(\lambda x.\pmapp{Q}{1}{f})}{u}
\\
\pmapp{P \Par Q}{1}{f} & \!\!\defeq\!\! & 
\pmapp{P}{1}{f} \Par \pmapp{Q}{1}{f} \\
		\pmapp{\recp{X}{P}}{1}{f} &\!\!\defeq\!\!&\!\!\!\!\!\!
	
\\
& &\hspace{-15mm}\!\!\!\!\!\!\newsp{s}{\bout{\dual{s}}{\abs{(\vmap{\tilde{n}}, y)} 
\,{\binp{y}{z_\X} \auxmapp{\pmapp{P\subst{z_\X}{\X}}{1}{{f,\{z_\rvar{X}\to \tilde{n}\}}}}{{}}{\es}}} \inact
\\ 
& & \hspace{-5mm}\!\!\!\!\!\!
 \Par 
\binp{s}{z_\X} \pmapp{P\subst{z_X}{X}}{1}{{f,\{z_\rvar{X}\to \tilde{n}\}}}
} 
\quad (\tilde{n} = \fn{P}) \\ 
\pmapp{z_\rvar{X}}{1}{f} &\!\!\defeq\!\!& \hspace{-3mm}\newsp{s}{
\appl{z_X}{(\tilde{n}, s)}\\
& & \hspace{-5mm} \Par \bbout{\dual{s}}{ \abs{(\vmap{\tilde{n}},y)}{\appl{z_X}{(\vmap{\tilde{n}}, y)}}} \inact}  \quad (\tilde{n} = f(z_\rvar{X})) \\
\end{array}
$\\[1mm]
%  
Above $\fn{P}$ denotes a 
lexicographically ordered
sequence 
of free names in $P$.
The input bound variable $x$ is annotated by a type to distinguish the first- and higher-order cases. \\
\caption{\label{f:enc:hopi_to_ho}Encoding of \HOp into \HO.}
%(cf.~\defref{d:enc:fotohorec}).
%Mappings 
%$\map{\cdot}^2$,
%$\mapt{\cdot}^2$, 
%and 
%$\mapa{\cdot}^2$
%are homomorphisms for the other processes/types/labels. 
\Hlinefig
\end{figure}

\smallskip 

\begin{definition}[Typed Encoding of \HOp into \HO]
\label{d:enc:hopitoho}
Let $f$ be a function from variables to sequences of name variables.
%
%Let $\tyl{L}_{\HOp}=\calc{\HOp}{{\cal{T}}_1}{\hby{\ell}}{\wb_H}{\proves}$
%and 
%$\tyl{L}_{\HO}=\calc{\HO}{{\cal{T}}_2}{\hby{\ell}}{\wb_H}{\proves}$. 
%where 
%${\cal{T}}_1$ and ${\cal{T}}_2$ are sets of types of $\HOp$ 
%and $\HO$, respectively, 
%the typing $\proves$ is defined in 
%\figref{fig:typerulesmy} 
%and $\hwb$ is defined in \defref{d:hbw}. 
We define the typed encoding 
$\enco{\map{\cdot}^{1}_f, \mapt{\cdot}^{1}, \mapa{\cdot}^{1}}: \tyl{L}_{\HOp} \to \tyl{L}_{\HO}$ in 
\figref{f:enc:hopi_to_ho}. 
We assume that the mapping $\mapt{\cdot}^{1}$ on types is extended to 
session environments $\Delta$
and
shared environments $\Gamma$ homomorphically with: 
\[
	\begin{array}{l}
%	    \mapt{\Delta \cat s: S}^{1} & =  & \mapt{\Delta}^{1} \cat s:\mapt{S}^{1} & \\
%		\mapt{\Gamma \cat u: \chtype{S}}^{1} & =  & \mapt{\Gamma}^{1} \cat u:\chtype{\mapt{S}^{1}} & \\
%		\mapt{\Gamma \cat u: \chtype{L}}^{1} & = &  \mapt{\Gamma}^{1} \cat u:\chtype{\mapt{L}^{1}} & \\
		\tmap{\Gamma \cat \varp{X}:\Delta}{1}  =  \tmap{\Gamma}{1} \cat x:\shot{(S_1,..,S_m,S^*)} \ 
	\end{array}
\]
with
$S^* = \trec{t}{\btinp{\shot{(S_1,...,S_m,\vart{t})}} \tinact}$
%and $\Delta = \{n_1:S_1, \ldots, n_m:S_m\}$. 
and $\Delta = \{n_i:S_i\}_{1\leq i\leq m}$. 
\end{definition}

\smallskip 
\noi Note that $\varp{X}:\Delta$ is mapped to a non-tail
recursive session type.
\dk{Non-tail
recursive session types \jpc{have been} studied in
\cite{DBLP:journals/corr/abs-1202-2086,TGC14};
\jpc{to our knowledge,}
this is the first application in the
context of higher-order session types.}
%which carries type variable as the last argument.  
For a simplicity of the presentation, we use the polyadic name abstraction and passing.
They will be formally encoded into \HO in \secref{sec:extension}.

\noi We explain the mapping in \figref{f:enc:hopi_to_ho}, focusing 
on {\em name passing} ($\pmapp{\bout{u}{w} P}{1}{f}$) and  
{\em recursion} ($\pmapp{\recp{X}{P}}{1}{f}$ and $\pmapp{z_\rvar{X}}{1}{f}$). 

\myparagraph{Name passing}
A name $w$ is being passed as an input guarded abstraction;
the abstraction receives a higher-order
value and continues with the application of $w$ over
the received higher-order value.
%A name $m$ is being passed as an input
%guarded abstraction. 
%The input prefix receives an abstraction and
%continues with the application of $n$ over the received abstraction.
On the receiver side $\binp{u}{x} P$ 
the encoding realises a mechanism that i) receives
the input guarded abstraction, then ii) applies it on a fresh session endpoint $s$, 
and iii) uses
the dual endpoint $\dual{s}$ to send the continuation $P$ as the abstraction
$\abs{x}{P}$. 
\NY{Then} name substitution is achieved via name application.

\myparagraph{Recursion}
The encoding of a recursive process $\recp{X}{P}$  is delicate, for it 
must preserve the linearity of session endpoints. To this end, we:
\NY{i) record a mapping from recursive variable $X$ to process variables $z_X$;
ii)~encode the recursion body $P$ as a name abstraction
in which free names of $P$ are converted into name variables;
iii)~this higher-order value is embedded in an input-guarded 
``duplicator'' process; and 
iv)~make the encoding of process variable $z_X$ to 
simulate recursion unfolding by 
invoking the duplicator in a by-need fashion,
i.e.,~upon reception, abstraction $\auxmapp{P}{{}}{\sigma}$ is duplicated
with one copy used to reconstitute the encoded recursion body $P$ through
the application of $\fn{P}$ and another copy used to re-invoke
the duplicator when needed. % to simulate recursion unfolding.
}
%The idea follows 
%a classical recursion encoding \cite{ThomsenB:plachoasgcfhop}.  
%A mapping of process $P$ is parallel composed, 
%and also being passed as an input
%guarded abstraction, parameterised also by a sequence of trigger names $\tilde{n}$. 
%We record a mapping from $z_X$ (which is a fresh variable of $X$) 
%to $\tilde{n}$, so that 
%when the abstraction is substituted to $z_\rvar{X}$ 
%(which occurs in the mapping of $P\subst{z_X}{X}$), 
%the correct $\tilde{n}$ is applied. In this way, we can 
%send and receive an abstraction which holds $P$, repeatedly. 

\smallskip 

\begin{theorem}[Precise Encoding of \HOp into \HO]
\label{f:enc:hopitoho}
The encoding from $\tyl{L}_{\HOp}$ into $\tyl{L}_{\HO}$ (cf.~\defref{d:enc:hopitoho})
is precise. 
\end{theorem}

\subsection{From \HOp to \sessp}
\label{subsec:HOp_to_sessp}
\noi 
We now discuss the encodability of  $\HO$ into $\sessp$ where
we essentially follow the representability result put forward by 
Sangiorgi~\cite{San92,SaWabook}, but casted in the setting of session-typed communications. 
Intuitively, such a strategy represents the exchange of a process with the exchange of a freshly generated \emph{trigger name}. 
Trigger names may then be used to activate copies of the process, which now becomes a persistent resource represented by an input-guarded replication.

%Consider the following (naive) adaptation of \cite{San92,SaWabook} 
%in which session names are used are triggers and 
%exchanged processes would be have to used exactly once:
%%
%\[
%\begin{array}{l}
%		\pmap{\bout{u}{\abs{x}{Q}} P}{n}  \defeq   \newsp{s}{\bout{u}{s} (\pmap{P}{n} \Par \binp{\dual{s}}{x} \pmap{Q}{n})} \\
%		\pmap{\binp{u}{x} P}{n}  \defeq \binp{u}{x} \pmap{P}{n}
%		\quad 
%		\pmap{\appl{x}{u}}{n}  \defeq  \bout{x}{u} \inact
%	\end{array}
%\]
%%
%with the remaining \HOp constructs being mapped homomorphically.
%Although $\pmap{\cdot}{n}$ captures the correct semantics when
%dealing with systems that allow only linear abstractions,
%it suffers from untypability in the presence
%of shared abstractions. For instance,
%mapping for $P = \bout{n}{\abs{x}{\bout{x}{m}\inact}} \inact \Par \binp{\dual{n}}{x} (\appl{x}{s_1} \Par \appl{x}{s_2})$
%would be:
%%
%\[
%	\pmap{P}{n} \defeq
%	\newsp{s}{\bout{n}{s} \binp{\dual{s}}{x} \bout{x}{m} \inact \Par \binp{\dual{n}}{x} (\bout{x}{s_1} \inact \Par \bout{x}{s_2} \inact)}
%\]
%%
%The above process is untypable since processes $(\bout{x}{s_1} \inact$ and $\bout{x}{s_2} \inact)$
%cannot be put in parallel because they do not have disjoint session environments.
A session name is, however, a linear resource and cannot be replicated.
The correct approach would be to use replicated names
as triggers for shared resources and non-replicated names
for linear resources.
%as triggers instead of session names, when dealing with shared abstractions. 

\smallskip 

\begin{definition}[Typed Encoding of \HOp into \sessp]
\label{d:enc:hopitopi}
%Let $\tyl{L}_{\sessp}=\calc{\sessp}{{\cal{T}}_3}{\hby{\ell}}{\fwb}{\proves}$ 
%where the typing is defined in 
%\figref{fig:typerulesmy} 
%and the equivalence $\fwb$ is defined in \defref{d:fwb}.
%${\cal{T}}_3$ is a set of types of $\sessp$.  
%%
We define the typed encoding 
$\enco{\map{\cdot}^{2}, \mapt{\cdot}^{2}, \mapa{\cdot}^{2}}: \tyl{L}_{\HOp} \to \tyl{L}_{\sessp}$  
%We define the mappings $\map{\cdot}^{2}$, $\mapt{\cdot}^{2}$, $\mapa{\cdot}^{2}$
in \figref{f:enc:ho_to_sessp}. 
\end{definition}

\smallskip 

\begin{figure}[t]
\[
\begin{array}{l}
	\begin{array}{rcl}
\noindent{\bf Types:}\hspace{0.5cm} 
		\tmap{\btout{\lhot{S}}S_1}{2} & \defeq & \bbtout{\chtype{\btinp{\tmap{S}{2}}\tinact}}\tmap{S_1}{2} \\
		\tmap{\btinp{\lhot{S}}S_1}{2} & \defeq & \bbtinp{\chtype{\btinp{\tmap{S}{2}}\tinact}}\tmap{S_1}{2} 
\\[1mm]
\hline
%\end{array}
%\]
%\[
%\begin{array}{rcll}
\noindent{\bf Labels:}\ 
		\mapa{(\nu \tilde{m})\bactout{n}{\abs{ x}{P}} }^2  & \defeq & \news{a} \bactout{n}{a} \\
		\mapa{\bactinp{n}{\abs{ x}{P}} }^2 &  \defeq & \bactinp{n}{a}
\\[1mm]
\hline
\end{array}
\end{array}
\]
\hspace{4mm}{\bf Terms} :\\
\noi$
\begin{array}{rcll}
		\pmap{\bout{u}{\abs{x}{Q}} P}{2} &\!\!\!\! \defeq \!\!\!\!\!\!\!\!&\!\!\!  \left\{
		\begin{array}{r}
		\!\!\!\!	\newsp{a}{\bout{u}{a} (\pmap{P}{2} \Par \repl{} \binp{a}{y} \binp{y}{x} \pmap{Q}{2})\,}\\
                  (s \notin \fn{Q}) \\
		\!\!\!\!	\newsp{a}{\bout{u}{a} (\pmap{P}{2} \Par \binp{a}{y} \binp{y}{x} \pmap{Q}{2})\,}\quad\\
            \textrm{(otherwise)} %\dk{Q \textrm{ linear}} \\
		\end{array}
		\right.
		\\
\pmap{\binp{u}{x} P}{2} &\!\!\!\! \defeq \!\!\!\! \!\!\!\! &  \binp{u}{x} \pmap{P}{2}\\
\pmap{\appl{x}{u}}{2} & \!\!\!\! \defeq \!\!\!\! \!\!\!\! & \newsp{s}{\bout{x}{s} \bout{\dual{s}}{u} \inact}\\
\pmap{\appl{V}{u}}{2} & \!\!\!\! \defeq \!\!\!\! \!\!\!\! & \newsp{s}{\bout{a}{s} \bout{\dual{s}}{u} \inact} \\

	\end{array}
$\\[3mm]
$\repl{} P$ means $\recp{X}{(P \Par \rvar{X})}$.
The rest of mappings are homomorphic.\\ 
%for others.  %types, labels and processes    
	\caption{
Encoding of \HOp into \sessp.
\label{f:enc:ho_to_sessp}
}
\Hlinefig
\end{figure}

\begin{theorem}[Precise Encoding of \HOp into \sessp]
\label{f:enc:hotopi}
The encoding from $\tyl{L}_{\HOp}$ into $\tyl{L}_{\sessp}$ (cf.~\defref{d:enc:hopitopi})
is precise. 
\end{theorem}

\begin{remark}
Operational correspondence for the encoding in~\defref{d:enc:hopitopi}
is different than that in~\defref{def:ep}, due to triggers. 
In particular,  completeness differs when $\ell_1 \neq \tau$.
This way, e.g., if  
$\stytraarg{\Gamma}{\ell_1}{\Delta}{P}{\Delta'}{P'}{}$
with $\ell_1 = (\nu \tilde{m})\bactout{n}{\abs{ x}{R}}$, 
then %$\exists \ell_2, Q$ s.t. 
$\stytraarg{\mapt{\Gamma}^2}{\ell_2}{\mapt{\Delta}^2}{\map{P}^2}{\mapt{\Delta'}^2}{Q}{}$,
where 
$\ell_2 = (\nu a)\bactout{n}{a}$ and
$Q = \pmap{P' \Par  \repl{} \binp{a}{y} \binp{y}{x} R}{2}$.
This 
statement, essential in proofs of full abstraction,
is the same given by Sangiorgi~\cite{SangiorgiD:expmpa}.
Completeness is as in~\defref{def:ep} when  $\ell_1 = \tau$.
See~\cite{KouzapasPY15} for details.
\end{remark}


