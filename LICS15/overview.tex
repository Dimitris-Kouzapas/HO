% !TEX root = main.tex
\noi
\begin{comment}
%In  
%\S\,\ref{subsec:intro:expr}
%and
%\S\,\ref{subsec:intro:bisimulation}
We motivate further our contributions and 
give details of the technical challenges involved. 
Some  notation, formally introduced shortly, is useful here:
$\bout{u}{V} P$
and
$\binp{u}{x} P$ denote input- and output-prefixed processes.
Values $V, W$ can be either a name $u$ or an (name) abstraction $\abs{x}Q$.
Processes $P \Par Q$ and $\inact$ denote the parallel composition and inactive processes, respectively.
Given a (linear) session name $s$, we write $\dual{s}$ for its \emph{dual}; 
they are the two \emph{endpoints} of the same session: the restriction operation  
$\news{s}P$ simultaneously covers $s$ and $\dual{s}$ in~$P$. 
The restriction for shared name $a$ in $P$ is denoted $\news{a}P$.
We write $S$ to range over session types; 
this way, e.g., session type $\btout{U} S'$ (resp. $\btinp{U} S'$) is
decrees that the output (resp. input) of a value of type $U$
must precede a protocol with type $S'$. 
The  terminated session is typed with $\tinact$.
%Given  type $U$, 
We write $\lhot{U}$ (resp. $\shot{U}$) for the 
linear (resp. unrestricted) functional type.

\subsection{Relative Expressiveness Results}
\label{subsec:intro:expr}
\myparagraph{Encoding Name Passing and Recursion into \HO.}
Our first encodability result highlights the expressiveness of 
the core higher-order calculus \HO, which lacks name passing and recursion. 
We encode \HOp into \HO, which entails also an encoding of \sessp into \HO.
The challenges in this encoding of concern exactly name passing and recursion.
To encode name output, we ``pack''
the name to be passed around into a suitable abstraction; 
upon reception, the receiver must ``unpack'' this object following a precise protocol.
The encoding formally is defined in Def.~\ref{d:enc:hopitoho}; we illustrate the encoding strategy below.
The encoding of name passing is:
\[
\begin{array}{rcll}
  \map{\bout{u}{w} P}	&=&	\bout{u}{ \abs{z}{\,\binp{z}{x} (\appl{x}{w})} } \map{P} \\
  \map{\binp{u}{x} Q}	&=&	\binp{u}{y} \newsp{s}{\appl{y}{s} \Par \bout{\dual{s}}{\abs{x}{\map{Q}}} \inact}
\end{array}
\]
and so we need 
exactly two (deterministic) reductions 
to unpack  name $w$.
The encoding of a recursive process $\recp{X}{P}$  is delicate, for it 
 must preserve the linearity of session endpoints. To this end, we
%\begin{enumerate}[i)]
%\item 
encode the recursion body $P$ as a (polyadic) name abstraction
in which free session names are converted into name variables.
This higher-order value is embedded in a sort of input-guarded 
``duplicator'' process; the encoding of process variable $X$ is then meant to 
invoke the duplicator in a by-need fashion to simulate recursion unfolding. 

%\item The recursion body $P$ is encoded in such a way that
%the  in $\map{P}$ (linear names) ; the obtained process
%is then used as the body of a  on those variables.
%\item Using a private session, the abstraction obtained in (i) is communicated to a
%process which instantiates the initial free session names in $P$, 
%in coordination with the encoding of the recursion variable $X$ (using a private session).
%\end{enumerate}
%The second step is also challenging:
%in essence, one should establish a private session with the encoding of the recursion  
%body in order to spawn copies of $\map{P}$ with appropriate free session names.
The use of polyadicity is crucial to the encoding; we shall get back to this point below.
It is worth noticing that the typing of the encoding requires 
a non tail recursive type of the form $\trec{t}{\btinp{\lhot{(S,\vart{t}}} \tinact}$
(see Def.~\ref{d:enc:hopitoho} for the precise formulation).

\smallskip 

\myparagraph{Other Encodings.}
We also give %the reverse of the previous encoding, namely 
an encoding of \HOp into \sessp. We rely on the well-known representability result of Sangiorgi~\cite{SangiorgiD:expmpa}. 
Since communicated processes may contain session names, in order to respect linearity and session protocols
the encoding enforces a distinction, depending on whether this kind of names is present in the communication object. If session names are present then a linear server trigger is deployed; otherwise, the replicated server  in~\cite{SangiorgiD:expmpa} can be used. 

As mentioned above, \HOp and \HO feature \emph{first-order} abstractions: 
only names can be used as arguments to abstractions.
Hence, given a (shared/linear) name $u$, in \HOp
we have the reduction $(\abs{x}{P}) \, u   \red  P \subst{u}{x}$.
We also consider \HOpp, an extension of \HOp with \emph{higher-order} abstractions.
 Thus, in \HOpp also an arbitrary value $V$ (possibly a process) can be an argument of an abstraction, 
 and one could have the reduction
 $(\abs{x}{P}) \, V   \red  P \subst{V}{x}$.
 We give an encoding of \HOpp into \HO: it  naturally extends that of \HOp into \HO;
see~\S\,\ref{subsec:hop}.

A well-known feature in process calculi is \emph{polyadicity}, i.e.,  
passing around tuples of values in communications. 
We consider the polyadic extension of \HOp, denoted \pHOp.
In \pHOp we have polyadicity in session communications and abstractions; 
polyadicity of shared names is ruled out by typing. 
This is enough for most purposes, including our encoding from \HOp into \HO.
In a session-typed setting, encoding polyadicity is straightforward, thanks to 
%polyadic arguments can be sent one by one, relying on 
the private character of 
(linear) session names --- see \S\,\ref{subsec:pho} for details.
%\[
%\begin{array}{rl}
%		\map{\binp{u}{x_1, \cdots, x_m} P}
%		 =  & \!\!\!\!
%		\binp{u}{x_1} \cdots ;  \binp{u}{x_m} \map{P}
%		\\
%%		\map{\bout{u}{u_1, \cdots, u_m} P}
%%		 =  & \!\!\!\!
%%		\bout{u}{u_1} \cdots ;  \bout{u}{u_m} \map{P}
%%		\\
%		\map{\bbout{u}{\abs{(x_1, \cdots, x_m)} Q} P}
%		= & \!\!\!\!
%		\bbout{u}{\abs{z}\binp{z}{x_1}\cdots ; \binp{z}{x_m} \map{Q}} \map{P}
%		\\ 
%		\map{\appl{x}{(u_1, \cdots, u_m)}}
%		= & \!\!\!\!
%		\newsp{s}{\appl{x}{s} \Par \bout{\dual{s}}{u_1} \cdots ; \bout{\dual{s}}{u_m} \inact} 
%	\end{array}
%\]
%Notice that encoding of polyadic abstraction/application requires an extra step, 
%in which a monadic abstraction is sent.

\smallskip

\myparagraph{A Non Encodability Result.}
We also show that shared names strictly add expressiveness to session calculi: that is,
there are (non deterministic) behaviours expressible with shared names not expressible using linear names only.
Although somewhat expected we do not know of a formal proof.
We propose such a formal proof, which relies crucially on the behavioural theory that we have introduced here
and on its determinacy properties. %\jp{EXPAND}.


\subsection{Tractable Bisimilarities for Session-Typed Processes}
\end{comment}

\label{subsec:intro:bisimulation}
\noi 
We outline our motivations and methods 
to show how session types are used for formulating 
two tractable bisimulations. 

\myparagraph{Overcoming Issues of Context Bisimilarity.}
%The characterisation of contextual congruence given by 
Context bisimilarity ($\wbc$, \defref{def:wbc}) is a too demanding relation on processes. 
%In the following we motivate our
%proposal for alternative, more tractable characterisations.  
%For the sake of clarity, and to emphasise the novelties of our approach, 
%we often omit type information. 
%Formal definitions including types are in \S\,\ref{sec:behavioural}.
To see the issue, we show 
the following clause for output.
Suppose $P \,\Re\, Q$, for some context bisimulation $\Re$. Then:

\smallskip 

\begin{enumerate}[$(\star)$]
	\item	Whenever 
		$P \by{\news{\tilde{m_1}} \bactout{n}{V}} P'$
		there exist
		$Q'$ and $W$
		such that 
		$Q \by{\news{\tilde{m_2}} \bactout{n}{W}} Q'$
		and, \emph{\textbf{for all} $R$}  with $\fv{R}=x$, 
		$\newsp{\tilde{m_1}}{P' \Par R\subst{V}{x}} \,\Re\, \newsp{\tilde{m_2}}{Q' \Par R\subst{W}{x}}$.
\end{enumerate}
\smallskip 
\noi 
Above, 
$\news{\tilde{m_1}} \bactout{n}{V}$ is the output label of 
value $V$ with extrusion of names in $\tilde{m_1}$.
To reduce the burden induced by 
universal quantification, we introduce \emph{higher-order}  and 
\emph{characteristic}  
bisimulations, two tractable equivalences denoted  $\hwb$ and $\fwb$, respectively.
As we work with an \emph{early} labelled transition system (LTS), 
%we shall also aim at limiting the input actions,  
%so to define a
%bisimulation relation for the output clause without observing
%infinitely many actions on the same input prefix. 
%To this end, 
%
we take the following two steps: 
%
\begin{enumerate}[(a)]
	\item We replace $(\star)$ with a clause involving a more tractable process closure.
	\item We refine the transition rule for input in the LTS,
	to avoid observing infinitely many actions on the same input prefix.
\end{enumerate}
%
\smallskip

\myparagraph{Trigger Processes with Session Communication.}
Concerning~(a), we exploit session types. 
We 
first 
observe that closure $R\subst{V}{x}$ 
in $(\star)$
is context bisimilar to the process:
\begin{equation}\label{equ:1}
	P = \newsp{s}{\appl{(\abs{z}{\binp{z}{x}{R}})}{s} \Par \bout{\dual{s}}{V} \inact}
\end{equation}
\noi 
%where $\binp{z}{x}{R}$ is an input and $\bout{\dual{s}}{V} \inact$
%is an output 
%on the endpoint $\dual{s}$ (the dual of $s$).
In fact,
we do have $P \wbc R\subst{V}{x}$, 
since 
application and reduction of dual endpoints 
%($s$ and $\dual{s}$) 
are deterministic.  
Now let us
consider process $T_{V}$ below, where $t$ is a fresh name:
\begin{equation}\label{equ:0}
T_{V} = \hotrigger{t}{V}
\end{equation}
%We call $\abs{z}{\binp{z}{x} R}$ a {\bf\em trigger value}. 
Let us write $P \by{\bactinp{n}{V}} P'$ to denote an input transition along $n$.
If $T_{V}$ inputs value $\abs{z}{\binp{z}{x} R}$ then
we can simulate the closure of $P$:
\begin{equation}\label{equ:2}
%\hotrigger{t}{V_1} 
T_{V}
\by{\bactinp{t}{\abs{z}{\binp{z}{x} R}}} P 
\wbc 
R\subst{V}{x}
\end{equation}
Processes such as $T_{V}$ 
offer a value at a fresh name; we will use this class of 
{\bf\em trigger processes} to define a
 refined bisimilarity without the demanding 
output clause $(\star)$. Given a fresh name $t$, 
we write $\htrigger{t}{V}$ to 
stand for a trigger process $T_{V}$ for value $V$.
We note that 
in contrast to previous approaches~\cite{SaWabook,JeffreyR05} 
our {trigger processes} do {\em not} use recursion or 
replication. This is crucial for preserving linearity of session names.  

\smallskip

%Then we can use 
%$\newsp{\tilde{m_1}}{P_1 \Par \htrigger{t}{V_1}}$ instead 
%of Clause 1) in Definition \ref{def:wbc} if we input 
%$\abs{z}{\binp{z}{x} R}$.   

\myparagraph{Characteristic Processes and Values.}
Concerning (b), we limit the possible 
input values (such as $\abs{z}{\binp{z}{x} R}$ above) %processes 
by exploiting session types.
The key concept is that of {\bf \emph{characteristic process/value}}
of a type,  
%The characteristic process of a session type $S$ is the process inhabiting $S$. 
the 
simplest term inhabiting that type (\defref{def:char}).
This way, e.g., let $S = \btinp{\shot{S_1}} \btout{S_2} \tinact$
be a session type: first
input an abstraction, %from values $S_1$ to processes, 
then output a value of type $S_2$.
Then, process $Q = \binp{u}{x} (\bout{u}{s_2} \inact \Par \appl{x}{s_1})$
is a characteristic process for $S$ 
\jpc{along name $u$.}
%Thus, characteristic processes follow the communication structures decreed by session types.
Given a session type $S$, we write $\mapchar{S}{u} $
for its characteristic process along name $u$
(cf.~\defref{def:char}).
Also, %Similarly, 
given value type $U$, we write 
$\omapchar{U}$ to denote its characteristic value.


We use the %characteristic %Precisely, we exploit  the
 characteristic value %$\lambda x.\mapchar{U}{x}$. %$\lambda x.\mapchar{U}{x}$. 
$\omapchar{U}$
 to limit input transitions.
Following the same reasoning as (\ref{equ:1})--(\ref{equ:2}), 
we can use an alternative trigger process, called
{\bf\em characteristic trigger process} with type 
$U$ to replace clause
% (1) in Definition~\ref{def:wbc}:
($\star$) in \defref{def:wbc}:
\begin{equation}
	\label{eq:4}
	\ftrigger{t}{V}{U} \defeq \fotrigger{t}{x}{s}{\btinp{U} \tinact}{V}
\end{equation}

%Note that if $U=L$, $\ftrigger{t}{V}{U}$ subsumes 
%$\htrigger{t}{V}$. 
\noi 
\jpc{Thus, in contrast to the trigger process~\eqref{equ:0}, the characteristic trigger process 
in~\eqref{eq:4}
does not involve a
higher-order communication on fresh name $t$.}
To refine the input transition system, we need to observe 
an additional value, 
$\abs{{x}}{\binp{t}{y} (\appl{y}{{x}})}$, 
called the {\bf\em trigger value}. 
This is necessary, because it turns out
that a characteristic value 
alone as the observable input 
is not enough to define a sound bisimulation.
Roughly speaking, the trigger value is used
to observe/simulate application processes.
%to {\em count} the number of free higher-order variables inside 
%the receiver. 
%\jpc{See Example~\ref{ex:motivation} for further details.}

\smallskip 
\myparagraph{Refined Input Transition Rule.}
Based on 
the above discussion, we refine 
the (early) transition rule for input actions. 
%We write $P \by{\bactinp{n}{V}} P'$ for the input transition along $n$.
The transition rule for input roughly becomes 
(see \defref{def:rlts} for details):
\[
		\tree {
%\begin{array}{c}
P \by{\bactinp{n}{V}} P' \quad  V = m \vee V \scong
(\abs{{x}}{\binp{t}{y} (\appl{y}{{x}})}
 \vee  \omapchar{U})  \textrm{ with } t \textrm{ fresh} 
		}{
			P' \hby{\bactinp{n}{V}} P'
		}
\]
Note the distinction between standard and refined transitions: $\by{\bactinp{n}{V}}$ vs. $\hby{\bactinp{n}{V}}$.
Using this rule, we define an alternative  LTS
with refined 
\jpc{(higher-order)}
input. %; all other rules are kept unchanged.
This refined LTS is used for 
both higher-order ($\hwb$) and characteristic ($\fwb$) bisimulations (Defs.~\ref{d:hbw} and~\ref{d:fwb}),
in which the demanding clause~$(\star)$ is replaced with 
more tractable clauses based on trigger processes 
\jpc{(cf.~\eqref{equ:0})} 
and characteristic 
trigger processes
\jpc{(cf.~\eqref{eq:4})},
respectively.
We show that $\hwb$ is useful for \HOp and \HO, and that~$\fwb$ 
can be uniformly used in all subcalculi, including \sessp. 

\NY{Our calculus lacks name matching, 
which is usually crucial to prove completeness of bisimilarity.
Instead of matching,  we use types:
a process trigger embeds a name into a characteristic
process so to observe its session behaviour.}

%\dk{We stress-out that our calculus lacks
%a matching construct
%which is usually a crucial element to prove completeness of bisimilarity.
%Nevertheless, we use types
%%and specifically the characteristic process
%to compensate;
%%for the absence of matching, i.e.~
%instead of name matching, a process trigger embeds a name into a characteristic
%process so to observe its session behavior.}
%Notice that while Definition \ref{d:hbw} is useful for 
%\HOp and its higher-order variants,
%Definition \ref{d:fwb} is useful for first-order sub-calculi of \HOp.




%%\myparagraph{Outline}
%\subsection{Outline}
%\noi \S\,\ref{sec:calculus} presents the calculi; 
%\S\,\ref{sec:types} presents types;
%the tractable bisimulations are in \S\,\ref{sec:behavioural};
%the notion of encoding is in \S\,\ref{s:expr};
%\S\,\ref{sec:positive} and \S\,\ref{sec:negative}
%present positive and negative encodability results, resp;
%\S\,\ref{sec:extension} discusses extensions; and 
%\S\,\ref{sec:relwork} concludes with related work;
%Appendix summarises the typing system. 
%The paper is self-contained. 
%{\bf\em Omitted definitions, additional related work and full proofs can be found 
%in a technical report, available from \cite{KouzapasPY15}.} 
