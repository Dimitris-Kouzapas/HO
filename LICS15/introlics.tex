% !TEX root = main.tex
\subsection{Key points}
\begin{enumerate}[1.]
	\item	Session $\pi$ calculus with process passing.
	\item	Identify session $\pi$ and process passing subcalculi and their polyadic variants.
	\item	Bisimulation theory for higher-order session semantics.
	\item	New triggered bisimulation, related to Jeffrey and Rathke's.
	\item   Elementary values key to characterizations of behavioral equivalence.
	\item	Types provide techniques to prove completeness without matching.
	\item	We are interested in encodings with properties a la Gorla. 
                We extended them to typed setting. 
	\item	Encode name-passing to pure process abstraction calculus, with name abstractions.
	\item	Type of the recursion encoding uses non tail recursive type $\trec{t}{\btinp{t} \tinact}$.
	\item	Encode higher-order semantics to first order semantics.
	\item	Negative result. Cannot encode shared names using only shared names.
	\item   Extensions with higher-order abstractions and polyadicity also explored.
\end{enumerate}

\smallskip 

\subsection{Important things to explain}
Explain our \HO is very small without containg name passing 
\[ 
\abs{x}.P \quad \appl{x}{u}
\]

Explain we input only characteristic processes.  

\[
\lambda x.\mapchar{S}{x}
\]

\subsection{Tractable Bisimilarities for Session-Typed Processes}
\noi 
\myparagraph{Overcoming the Problem of Contextual Bisimulation}
Contextual bisimilarity (we denote $\wbc$) is 
a straightforward labelled characterisation of contextual congruence 
\cite{SaWabook}. 
Unfortunately, it is a too demanding relation on processes. 
%In the following we motivate our
%proposal for alternative, more tractable characterisations.  
%For the sake of clarity, and to emphasise the novelties of our approach, 
%we often omit type information. 
%Formal definitions including types are in \S\,\ref{sec:behavioural}.
To see the problem, we show 
the following clause for output.
Suppose $P_1 \,\Re\, Q_1$, for some context bisimulation $\Re$. Then:

\smallskip 

\begin{enumerate}[$(\star)$]
\item Whenever 
$P_1 \by{\news{\tilde{m_1}} \bactout{n}{V_1}} P_2$
there exist
$Q_2$ and $ V_2$
such that 
$Q_1 \by{\news{\tilde{m_2}} \bactout{n}{V_2}} Q_2$
and, for all $R$ with $\fv{R}=x$, 
$\newsp{\tilde{m_1}}{P_2 \Par R\subst{V_1}{x}} \,\Re\, \newsp{\tilde{m_2}}{Q_2 \Par R\subst{V_2}{x}}$
\end{enumerate}
\smallskip 

\noi $\news{\tilde{m_1}} \bactout{n}{V_1}$ is the output label of 
value $V_1$ with fresh name $\tilde{m_1}$.
The universal quantification involved in this clause is too demanding. 
To reduce this burden, we introduce \emph{higher-order} $\hwb$ and 
\emph{characteristic} $\fwb$ 
bisimulations, two tractable equivalences. 
We shall aim at limiting the input 
labelled transition system (LTS)  so that we can define a
bisimulation relation for the output clause without observing
infinitely many actions on the same input prefix. 
To this end, we take the following two steps: 
%
\begin{enumerate}[(a)]
	\item We replace $(\star)$ with a clause involving a more tractable process closure.
	\item We refine the transition rule for input in the LTS.
\end{enumerate}
%
\myparagraph{Trigger Processes with Session Communication}
Concerning (a), we use the session type feature. 
We 
first 
observe that closure $R\subst{V_1}{x}$ 
in $(\star)$
is contextually bisimilar to the process
\begin{equation}\label{equ:1}
P = \newsp{s}{\appl{(\abs{z}{\binp{z}{x}{R}})}{s} \Par \bout{\dual{s}}{V_1} \inact}
\end{equation}
\noi where $\binp{z}{x}{R}$ is an input and $\bout{\dual{s}}{V_1} \inact$
is an output with session dual name $\dual{s}$. 
In fact,
we have $P \wbc R\subst{V_1}{x}$, 
since 
application and session reduction between two dual end points 
($s$ and $\dual{s}$) are deterministic.  
Now consider the process $T_{V_1}$ below, where $t$ is a fresh name:
\begin{equation}\label{equ:0}
T_{V_1} = \hotrigger{t}{V_1}
\end{equation}
%We call $\abs{z}{\binp{z}{x} R}$ a {\bf\em trigger value}. 
Then, if $T_{V_1}$ inputs
a value $\abs{z}{\binp{z}{x} R}$, 
we can simulate the closure of $P$:
\begin{equation}\label{equ:2}
%\hotrigger{t}{V_1} 
T_{V_1}
\by{\bactinp{t}{\abs{z}{\binp{z}{x} R}}} P 
\wbc 
R\subst{V_1}{x}
\end{equation}
Processes such as $T_{V_1}$ 
offer a value at a fresh name; we will use this class of 
{\bf\em trigger processes} to define a
 refined bisimilarity without the demanding 
output clause $(\star)$. Given a fresh name $t$, 
we write $\htrigger{t}{V_1}$ to 
stand for a trigger process $T_{V_1}$ for value $V_1$.
We note that our {trigger processes} do {\em not} use a recursion or 
replication unlike the traditional approaches \cite{SaWabook,JeffreyR05}.  
This is crucial for preserving linearity of session names.  

%Then we can use 
%$\newsp{\tilde{m_1}}{P_1 \Par \htrigger{t}{V_1}}$ instead 
%of Clause 1) in Definition \ref{def:wbc} if we input 
%$\abs{z}{\binp{z}{x} R}$.   

\myparagraph{Characteristic Processes}
Concerning (b), we shall limit the possible $R$ processes by
exploiting the structure of session types.
The key concept is that of a {\bf \emph{characteristic process}} 
(Definition~\ref{def:char}). 
Given a session type $S$, its characteristic process is the 
simplest process inhabiting $S$. 
This way, e.g., if $S = \btinp{\shot{S_1}} \btout{S_2} \tinact$
(input a function from values $S_1$ to processes, then output a value of type $S_2$)
then process $Q = \binp{u}{x} (\bout{u}{s_2} \inact \Par \appl{x}{s_1})$
is a characteristic process for $S$.
Thus, characteristic processes follow the communication structures decreed by session types.
Given a session type $S$, we write $\mapchar{S}{u} $ for its characteristic process along name $u$ (the definition
extends to types $U$).

We use characteristic processes to limit input transitions.
More precisely, we exploit  the
{\bf\em characteristic value}, denoted $\lambda x.\mapchar{U}{x}$. %$\lambda x.\mapchar{U}{x}$. 
If we follow the same reasoning as (\ref{equ:1})--(\ref{equ:2}), 
we can use alternative trigger process, called
{\bf\em characteristic trigger process} with type 
$U$ to replace clause (1) in Definition~\ref{def:wbc}:
\begin{equation}
	\label{eq:4}
	\ftrigger{t}{V}{U} \defeq \fotrigger{t}{x}{s}{\btinp{U} \tinact}{V}
\end{equation}

%Note that if $U=L$, $\ftrigger{t}{V}{U}$ subsumes 
%$\htrigger{t}{V}$. 
To refine the input transition system, we need to observe 
an additional value, 
$\abs{{x}}{\binp{t}{y} (\appl{y}{{x}})}$, 
called the {\bf\em trigger value}. 
This is necessary, because it turns out
that a characteristic value 
alone as the observable input 
is not enough to define meaningful bisimulations.
Roughly speaking, the trigger value is used 
to {\em count} the number of free higher-order variables inside 
the receiver (Example~\ref{ex:motivation} explains the details). 
%as the following example illustrates.
%\begin{example}[The Need for Refined Typed LTS]
%%\label{ex:motivation}
%%We show that either a characteristic value or 
%%a trigger value alone as the observable input 
%%is not enough to define meaningful bisimulations.
%%(to justify rule \eltsrule{RRcv} in Definition~\ref{def:rlts} in the next paragraph). 
%%
%First we demonstrate that a characteristic value is not sufficient. 
%Consider the typed processes $P_1, P_2$:
%%
%\begin{eqnarray}
%	P_1 = \binp{s}{x} (\appl{x}{s_1} \Par \appl{x}{s_2}) & ~~&
%	P_2 = \binp{s}{x} (\appl{x}{s_1} \Par \binp{s_2}{y} \inact) 
%	\label{equ:6}
%\end{eqnarray}
%%
%We can show that 
%the session type for $s$ in $P_1$ and $P_2$ is 
%$s: \btinp{\shot{\btinp{C} \tinact}} \tinact$.
%If $P_1$ and $P_2$ 
%receive 
%the characteristic value $\abs{z}{\binp{z}{y}} \inact$ along $x$, 
%then they will evolve into %(\ref{eq:5}) and (\ref{eq:6}) in become:
%%
%\begin{eqnarray*}
%	\binp{s_1}{y} \inact \Par \binp{s_2}{y} \inact \hastype \Proc
%\end{eqnarray*}
%%
%therefore becoming 
%contextually bisimilar after the input of
% $\abs{x}{\binp{x}{y}} \inact$.
%However, the processes in \eqref{equ:6}
%are clearly {\em not} contextually bisimilar: there exist many input actions
%which may be used to distinguish them.
%For example, if 
%$P_1$ and $P_2$ input 
% $\abs{x} \bout{a}{s_3} \binp{x}{y} \inact$ then
%their derivatives are not bisimilar. 

%Observing only the characteristic value 
%results in an over-discriminating bisimulation.
%However, if a trigger value is received, 
%then we can distinguish 
%processes in \eqref{equ:6}:  
%%
%\begin{eqnarray*}
%%	\Gamma; \es; \Delta &\proves& 
%	P_1 \by{\ell} \binp{t}{x} \appl{x}{s_1} \Par \binp{t}{x} \appl{x}{s_2} 
%%\hastype \Proc
%	\mbox{~~and~~}
%%	\Gamma; \es; \Delta &\proves& 
%	P_2 \by{\ell} \binp{t}{x} \appl{x}{s_1} \Par \binp{s_2}{y} \inact 
%%\hastype \Proc
%\end{eqnarray*}
%%\noi resulting two distinct processes.  
%%
%where $\ell$ stands for the input of the trigger value
%$\abs{{x}}{\binp{t}{y} (\appl{y}{{x}})}$.
%One question that arises here is whether the trigger value is enough
%to distinguish two processes, hence no need of 
%characteristic values as the input. 
%This is not the case, since considering the trigger value
%alone also results in an over-discriminating bisimulation relation.
%In fact, the input trigger can be observed on any input prefix
%of {\em any type}. For example, consider the following processes:
%%
%\begin{eqnarray}
%	 \newsp{s}{\binp{n}{x} \appl{x}{s} \Par \bout{\dual{s}}{\abs{x} P} \inact} \label{equ:7}
%	\\
%	 \newsp{s}{\binp{n}{x} \appl{x}{s} \Par \bout{\dual{s}}{\abs{x} Q} \inact} \label{equ:8}
%\end{eqnarray}
%%
%\noi 
%Upon reception of the trigger abstraction, 
%processes \eqref{equ:7} and \eqref{equ:8}
%would evolve to 
%%input the trigger value, we obtain: % they evolved to 
%\begin{eqnarray*}
%%\Gamma; \es; \Delta \proves 
%	\newsp{s}{\binp{t}{x} \appl{x}{s} \Par \bout{\dual{s}}{\abs{x} P} \inact} 
%%\hastype \Proc
%	\mbox{~~and~~}
%%      \\
%%\Gamma; \es; \Delta \proves 
%	\newsp{s}{\binp{t}{x} \appl{x}{s} \Par \bout{\dual{s}}{\abs{x} Q} \inact}
%%\hastype \Proc
%\end{eqnarray*}
%%
%%\noi It is easy to obtain a closure if allow only the
%%trigger value as the input value.
%Then, 
%if both these processes input the trigger value $\abs{z}{\binp{z}{x} (\appl{x}{m})}$,  
%then they would become:
%\begin{eqnarray*}
%\Gamma; \es; \Delta \proves \newsp{s}{\binp{s}{x} \appl{x}{m} \Par \bout{\dual{s}}{\abs{x} P} \inact} \wbc \Delta \proves P \subst{m}{x}
%	\\
%\Gamma; \es; \Delta \proves \newsp{s}{\binp{s}{x} \appl{x}{m} \Par \bout{\dual{s}}{\abs{x} Q} \inact} \wbc \Delta \proves Q \subst{m}{x}
%\end{eqnarray*}
%\noi which are not bisimilar if $P \subst{m}{x} \not\wb Q \subst{m}{x}$.
%%\qed
%%In conclusion, these examples explain a need of both 
%%trigger and characteristic values 
%%as an input observation in the input transition relation (\eltsrule{RRcv})
%%which will be defined in Definition~\ref{def:rlts}.  
%\end{example}

%%\smallskip 
\myparagraph{Refined Input Transition Rule}
The above discussion justifies the refinement of the transition 
rule for input actions. 
The new transition rule for input looks as the rule 
given below (see Definition~\ref{def:rlts} for a precise definition):
\[
		\tree {
%\begin{array}{c}
P_1 \by{\bactinp{n}{V}} P_2 \quad  V  \scong
(\abs{{x}}{\binp{t}{y} (\appl{y}{{x}})}
 \vee  \abs{{x}}{\map{U}^{{x}}}  \vee m)  \textrm{ with } t \textrm{ fresh} 
		}{
			P_1 \hby{\bactinp{n}{V}} P_2
		}
\]
Using this rule, we define an alternative (typed) LTS
with refined input; all other rules are kept unchanged.
This refined LTS is used for 
both higher-order ($\hwb$) and characteristic ($\fwb$) bisimulations
where where we replace the demanding clause $(\star)$ with 
more tractable clauses based on trigger processes and characteristic 
trigger processes, respectively (Definitions \ref{d:hbw} and \ref{d:fwb}).
Later we show $\hwb$ is useful for \HOp and \HO, but 
$\fwb$ is uniformly useful for all subcalculi including the 
first order session $\pi$-calculus. 
%Notice that while Definition \ref{d:hbw} is useful for 
%\HOp and its higher-order variants,
%Definition \ref{d:fwb} is useful for first-order sub-calculi of \HOp.

%\subsection{Expressiveness Results}
%TBD.

\smallskip 

\myparagraph{Outline}
\noi \S\,\ref{sec:calculus} presents the calculi; 
\S\,\ref{sec:types} presents types;
the tractable bisimulations are in \S\,\ref{sec:behavioural};
the notion of encoding is in \S\,\ref{s:expr};
\S\,\ref{sec:positive} and \S\,\ref{sec:negative}
present positive and negative encodability results, resp;
\S\,\ref{sec:extension} discusses extensions; and 
\S\,\ref{sec:relwork} concludes with related work;
Appendix summarises the typing system. 
The paper is self-contained. 
{\bf\em Omitted definitions, additional related work and full proofs can be found 
in a technical report, available from \cite{KouzapasPY15}.} 
