% !TEX root = main.tex
\myparagraph{Key points}
\begin{enumerate}[1.]
	\item	Session $\pi$ calculus with process passing. DONE
	\item	Identify session $\pi$ and process passing subcalculi and their polyadic variants. DONE
	\item	Bisimulation theory for higher-order session semantics. DONE
	\item	New triggered bisimulation, related to J\&R's. DONE
	\item   Elementary values key to characterizations of behavioral equivalence. DONE
	\item	Types provide techniques to prove completeness without matching. \jp{TBD}
	\item	We are interested in encodings with properties a la Gorla. 
                We extended them to typed setting. \jp{TBD}
	\item	Encode name-passing to pure process abstraction calculus, with name abstractions. DONE
	\item	Type of the recursion encoding uses non tail recursive type $\trec{t}{\btinp{t} \tinact}$. DONE
	\item	Encode higher-order semantics to first order semantics. DONE
	\item	Negative result. Cannot encode shared names using only shared names.
	\item   Extensions with higher-order abstractions and polyadicity also explored. DONE
\end{enumerate}

%\smallskip 
%
%\myparagraph{Important things to explain}
%Explain our \HO is very small without containg name passing 
%\[ 
%\abs{x}.P \quad \appl{x}{u}
%\]

%Explain we input only characteristic processes.  
%
%\[
%\lambda x.\mapchar{S}{x}
%\]

\subsection{Higher-Order Session Calculi}
\noi By combining features from the $\lambda$-calculus and the $\pi$-calculus, 
\emph{higher-order process calculi} enable the communication of values containing 
processes. In this paper, we consider higher-order calculi with \emph{session primitives},
thus enabling the specification of sequences of reciprocal exchanges (protocols)
which can be verified via type-checking using \emph{session types}~\cite{honda.vasconcelos.kubo:language-primitives}.
Therefore, the process languages that we consider here allow us to specify   
session protocols in which higher-order values 
(mobile code) can be exchanged; naturally governed by session types, 
such protocols cleanly distinguish between 
linear and unrestricted behaviors in 
directed %point-to-point 
communications.

The study of higher-order concurrency has received significant attention, 
from typed and untyped perspectives.
%in particular via  comparisons with the first-order mobility of the $\pi$-calculus~\cite{MilnerR:calmp1}. 
Although models of session-typed 
communications with features of higher-order concurrency exist~\cite{tlca07,DBLP:journals/jfp/GayV10},
several aspects of their theory 
have yet to be understood. To this end, here we study
 \emph{tractable behavioral equivalences} and \emph{relative expressiveness}
for higher-order session calculi. 
These two issues 
have been throughly studied
%are well-understood 
for higher-order languages without sessions,
but not for higher-order process calculi with sessions.
This is unfortunate, given the wide applicability of session-based concurrency; indeed,
session types are expressive enough to describe complex 
communication structures found in practical protocols,  expressible, e.g., via recursive session types.
Clarifying the status of typed equivalences and relative expressiveness for session languages
may play a vital role in justifying non-trivial protocol optimizations and in transferring key reasoning techniques between session calculi.

The main higher-order language in our work, denoted \HOp,
extends the higher-order $\pi$-calculus~\cite{SangiorgiD:expmpa} with session primitives:
it contains constructs for session establishment
(synchronization on shared names), 
recursion, 
name abstractions/applications (i.e., functions from name identifiers to processes, call-by-value style),
and session communication (value passing and
labeled choice using linear names). 
Two significant subcalculi of \HOp restrict to higher- and first-order mobility:
while the \HO calculus is \HOp without recursion and name passing,
the \sessp calculus is \HOp without abstractions and applications.
Thus, 
while $\sessp$ is in essence the calculus in~\cite{honda.vasconcelos.kubo:language-primitives}, 
\HO  is  a core calculus for higher-order session concurrency.

In the first part of the paper, we address tractable behavioral equivalences
for \HOp.
A well-studied behavioral equivalence in the higher-order setting is \emph{contextual bisimulation},
a labelled characterization of barbed congruence, 
which offers an appropriate discriminative power at the price of heavy universal quantifications in output clauses.
Obtaining alternative characterizations that alleviate this burden
is then a recurring and important issue 
in the study of higher-order equivalences.
Our approach to this problem 
exploits the protocol specifications given by session types to  limit 
the behavior of higher-order session processes. 
Exploiting elementary processes inhabiting session types, 
this limitation is formally enforced by 
a refined (typed) labelled transition system (LTS)
that narrows down the spectrum of allowed process behaviors, 
thus naturally enabling tractable reasoning techniques. 
Two tractable characterizations of bisimilarity, 
targeted to higher- and first-order session processes,
are shown to coincide with contextual bisimilarity.

Then, in the second part of the paper we assess the expressive 
power of \HOp, \HO, and \sessp as delineated by typing. 
We establish strong correspondences between 
these languages via fully abstract encodings. 
Here again having the name usage information given by session types is fundamental to define the encodings
and to state their semantic correspondences.
That is, our encodings relate source and target processes
endowed with proper communication structures as described by session types. 
We introduce a notion of encoding that 
requires the translation of both process and types and 
formally captures preservation of (session) typing,
both for processes and for typing judgments. 
A further expressivity result shows that shared names, 
as required in the session establishment phase,
 strictly add expressive power 
to session calculi. 
\smallskip

\myparagraph{Outline}
The next section gives an overview of results and approach.
\noi \S\,\ref{sec:calculus} presents the calculi; 
\S\,\ref{sec:types} presents types;
the tractable bisimulations are in \S\,\ref{sec:behavioural};
the notion of encoding is in \S\,\ref{s:expr};
\S\,\ref{sec:positive} and \S\,\ref{sec:negative}
present positive and negative encodability results, resp;
\S\,\ref{sec:extension} discusses extensions; and 
\S\,\ref{sec:relwork} concludes with related work;
Appendix summarises the typing system. 
The paper is self-contained. 
{\bf\em Omitted definitions, additional related work and full proofs can be found 
in a technical report, available from \cite{KouzapasPY15}.} 

\section{Overview}
\label{sec:overview}
\noi
In  \S\,\ref{subsec:intro:bisimulation}
and
\S\,\ref{subsec:intro:expr}
motivate further our contributions and 
give details of the technical challenges involved. 
STILL TO ADD: Notation for input/output processes, abstraction/applications, endpoints, 
session types, 


\subsection{Tractable Bisimilarities for Session-Typed Processes}
\label{subsec:intro:bisimulation}
\noi 
\myparagraph{Overcoming Limitations of Contextual Bisimulation.}
Contextual bisimilarity ($\wbc$, Def.~\ref{def:wbc}) is 
%a straightforward 
labelled characterisation of contextual congruence 
\cite{SaWabook}. 
Unfortunately, it is a too demanding relation on processes. 
%In the following we motivate our
%proposal for alternative, more tractable characterisations.  
%For the sake of clarity, and to emphasise the novelties of our approach, 
%we often omit type information. 
%Formal definitions including types are in \S\,\ref{sec:behavioural}.
To see the problem, we show 
the following clause for output.
Suppose $P \,\Re\, Q$, for some context bisimulation $\Re$. Then:

\smallskip 

\begin{enumerate}[$(\star)$]
\item Whenever 
$P \by{\news{\tilde{m_1}} \bactout{n}{V}} P'$
there exist
$Q'$ and $W$
such that 
$Q \by{\news{\tilde{m_2}} \bactout{n}{W}} Q'$
and, for all $R$ with $\fv{R}=x$, 
$\newsp{\tilde{m_1}}{P' \Par R\subst{V}{x}} \,\Re\, \newsp{\tilde{m_2}}{Q' \Par R\subst{W}{x}}$
\end{enumerate}
\smallskip 
\noi 
Above, 
$\news{\tilde{m_1}} \bactout{n}{V}$ is the output label of 
value $V$ with extrusion of names in $\tilde{m_1}$.
To reduce the burden induced by 
universal quantification, we introduce \emph{higher-order}  and 
\emph{characteristic}  
bisimulations, two tractable equivalences denoted  $\hwb$ and $\fwb$, respectively.
We shall aim at limiting the input 
labelled transition system (LTS)  so that we can define a
bisimulation relation for the output clause without observing
infinitely many actions on the same input prefix. 
To this end, we take the following two steps: 
%
\begin{enumerate}[(a)]
	\item We replace $(\star)$ with a clause involving a more tractable process closure.
	\item We refine the transition rule for input in the LTS.
\end{enumerate}
%
\smallskip

\myparagraph{Trigger Processes with Session Communication.}
Concerning~(a), we exploit session types. 
We 
first 
observe that closure $R\subst{V}{x}$ 
in $(\star)$
is contextually bisimilar to the process
\begin{equation}\label{equ:1}
P = \newsp{s}{\appl{(\abs{z}{\binp{z}{x}{R}})}{s} \Par \bout{\dual{s}}{V} \inact}
\end{equation}
\noi where $\binp{z}{x}{R}$ is an input and $\bout{\dual{s}}{V} \inact$
is an output 
on the endpoint $\dual{s}$ (the dual of $s$).
In fact,
we have $P \wbc R\subst{V}{x}$, 
since 
application and reduction of dual endpoints 
%($s$ and $\dual{s}$) 
are deterministic.  
Now consider the process $T_{V}$ below, where $t$ is a fresh name:
\begin{equation}\label{equ:0}
T_{V} = \hotrigger{t}{V}
\end{equation}
%We call $\abs{z}{\binp{z}{x} R}$ a {\bf\em trigger value}. 
If $T_{V}$ inputs $\abs{z}{\binp{z}{x} R}$
we can simulate the closure of $P$:
\begin{equation}\label{equ:2}
%\hotrigger{t}{V_1} 
T_{V}
\by{\bactinp{t}{\abs{z}{\binp{z}{x} R}}} P 
\wbc 
R\subst{V}{x}
\end{equation}
Processes such as $T_{V}$ 
offer a value at a fresh name; we will use this class of 
{\bf\em trigger processes} to define a
 refined bisimilarity without the demanding 
output clause $(\star)$. Given a fresh name $t$, 
we write $\htrigger{t}{V}$ to 
stand for a trigger process $T_{V}$ for value $V$.
We note that 
in contrast to previous approaches~\cite{SaWabook,JeffreyR05} 
our {trigger processes} do {\em not} use recursion or 
replication. This is crucial for preserving linearity of session names.  

\smallskip

%Then we can use 
%$\newsp{\tilde{m_1}}{P_1 \Par \htrigger{t}{V_1}}$ instead 
%of Clause 1) in Definition \ref{def:wbc} if we input 
%$\abs{z}{\binp{z}{x} R}$.   

\myparagraph{Characteristic Processes.}
Concerning (b), we shall limit the possible $R$ processes by
exploiting the structure of session types.
The key concept is that of {\bf \emph{characteristic process}} 
(Def.~\ref{def:char}). 
The characteristic process of 
a session type $S$ is the 
simplest process inhabiting $S$. 
This way, e.g., if $S = \btinp{\shot{S_1}} \btout{S_2} \tinact$
(input a function from values $S_1$ to processes, then output a value of type $S_2$)
then process $Q = \binp{u}{x} (\bout{u}{s_2} \inact \Par \appl{x}{s_1})$
is a characteristic process for $S$.
Thus, characteristic processes follow the communication structures decreed by session types.
Given a session type $S$, we write $\mapchar{S}{u} $ for its characteristic process along name $u$.
In general, 
the definition
extends to  value types $U$ and so we write $\mapchar{U}{u}$.

We use characteristic processes to limit input transitions.
Precisely, we exploit  the
{\bf\em characteristic value} $\lambda x.\mapchar{U}{x}$. %$\lambda x.\mapchar{U}{x}$. 
If we follow the same reasoning as (\ref{equ:1})--(\ref{equ:2}), 
we can use an alternative trigger process, called
{\bf\em characteristic trigger process} with type 
$U$ to replace clause
% (1) in Definition~\ref{def:wbc}:
($\star$) in Def.~\ref{def:wbc}:
\begin{equation}
	\label{eq:4}
	\ftrigger{t}{V}{U} \defeq \fotrigger{t}{x}{s}{\btinp{U} \tinact}{V}
\end{equation}

%Note that if $U=L$, $\ftrigger{t}{V}{U}$ subsumes 
%$\htrigger{t}{V}$. 
\noi To refine the input transition system, we need to observe 
an additional value, 
$\abs{{x}}{\binp{t}{y} (\appl{y}{{x}})}$, 
called the {\bf\em trigger value}. 
This is necessary, because it turns out
that a characteristic value 
alone as the observable input 
is not enough to define meaningful bisimulations.
Roughly speaking, the trigger value is used 
to {\em count} the number of free higher-order variables inside 
the receiver. In Example~\ref{ex:motivation} we explain the details. 

\smallskip 
\myparagraph{Refined Input Transition Rule.}
The above discussion justifies the refinement of the transition 
rule for input actions. 
We write $P \by{\bactinp{n}{V}} P'$ for the input transition along $n$ (early style).
The transition rule for input roughly becomes 
(see Def.~\ref{def:rlts} for a precise definition):
\[
		\tree {
%\begin{array}{c}
P \by{\bactinp{n}{V}} P' \quad  V  \scong
(\abs{{x}}{\binp{t}{y} (\appl{y}{{x}})}
 \vee  \abs{{x}}{\map{U}^{{x}}}  \vee m)  \textrm{ with } t \textrm{ fresh} 
		}{
			P' \hby{\bactinp{n}{V}} P'
		}
\]
Observe the notations for standard and refined transitions: $\by{\bactinp{n}{V}}$ vs. $\hby{\bactinp{n}{V}}$.
Using this rule, we define an alternative  LTS
with refined input; all other rules are kept unchanged.
This refined LTS is used for 
both higher-order ($\hwb$) and characteristic ($\fwb$) bisimulations,
which replace the demanding clause~$(\star)$ with 
more tractable clauses based on trigger processes and characteristic 
trigger processes, respectively (Defs.~\ref{d:hbw} and~\ref{d:fwb}).
Later we show $\hwb$ is useful for \HOp and \HO, but 
$\fwb$ is uniformly useful for all subcalculi including the 
first-order %session $\pi$-calculus.
calculus \sessp. 
%Notice that while Definition \ref{d:hbw} is useful for 
%\HOp and its higher-order variants,
%Definition \ref{d:fwb} is useful for first-order sub-calculi of \HOp.

\subsection{Relative Expressiveness Results}
\label{subsec:intro:expr}
\myparagraph{Encoding Name Passing and Recursion into \HO.}
Our first encodability result highlights the expressiveness of 
the core higher-order calculus \HO, which lacks name passing and recursion. 
We encode \HOp into \HO, which entails also an encoding of \sessp into \HO.
The challenges in this encoding of concern exactly name passing and recursion.
To encode name output, we ``pack''
the name to be passed around into a suitable abstraction; 
upon reception, the receiver must ``unpack'' this object following a precise protocol.
The encoding formally defined in Def.~\ref{d:enc:hopitoho}; we give intuitions of the strategy below.
The encoding of name passing is:
\[
\begin{array}{rcll}
  \map{\bout{u}{w} P}	&=&	\bout{u}{ \abs{z}{\,\binp{z}{x} (\appl{x}{w})} } \map{P} \\
  \map{\binp{u}{x} Q}	&=&	\binp{u}{y} \newsp{s}{\appl{y}{s} \Par \bout{\dual{s}}{\abs{x}{\map{Q}}} \inact}
\end{array}
\]
where it is easy to see that 
exactly two deterministic reductions are needed
to unpack the name $w$ sent by the sender.
A challenge in this encoding strategy is the encoding of recursion, 
for it must preserve the linearity of session endpoints. 
The encoding of $\recp{X}{P}$ is delicate and involves the following steps:
\begin{enumerate}[i)]
\item The recursion body $P$ is encoded in such a way that
the free session names in $\map{P}$ (linear names) are renamed as name variables; the obtained process
is then used as the body of a (polyadic) name abstraction on those variables.
\item Using a private session, the abstraction obtained in (i) is communicated to a
process which instantiates the initial free session names in $P$, 
in coordination with the encoding of the recursion variable $X$ (using a private session).
\end{enumerate}
%The second step is also challenging:
%in essence, one should establish a private session with the encoding of the recursion  
%body in order to spawn copies of $\map{P}$ with appropriate free session names.
The use of polyadicity is crucial to the encoding; we shall get back to this point below.
It is worth noticing that the typing of the encoding requires 
a non tail recursive type of the form $\trec{t}{\btinp{\lhot{(S,\vart{t}}} \tinact}$
(see Def.~\ref{d:enc:hopitoho} for the precise formulation).

\smallskip 

\myparagraph{Other Encodings.}
We also study the reverse of the previous encoding, namely 
an encoding of \HOp into \sessp. Here we rely on the well-known representability result of Sangiorgi~\cite{SangiorgiD:expmpa}. 
Since communicated processes may contain free session names, in order to respect session protocols
the encoding should distinguish between two cases, depending on whether this kind of linear names is present in the communication object. If session names are present then a linear server trigger is deployed; otherwise, the replicated server  in~\cite{SangiorgiD:expmpa} can be used. 

As mentioned earlier, \HOp and \HO feature \emph{first-order} abstractions. 
This means that
only  names can be used as arguments to abstractions.
Hence, given a (shared/linear) name $u$, in \HOp
we have the reduction $(\abs{x}{P}) \, u   \red  P \subst{u}{x}$.
We also consider \HOpp, an extension of \HOp with \emph{higher-order} abstractions.
 This means that in \HOpp also arbitrary values can be arguments of abstractions, 
 and one could have the reduction
 $(\abs{x}{P}) \, V   \red  P \subst{V}{x}$, where $V$ can be a process.
 We propose an encoding of \HOpp into \HO: it results as 
a natural extension of that of \HOp into \HO;
see \S\,\ref{subsec:hop} for a formal definition.

Another well-known feature in process calculi is \emph{polyadicity}, the ability 
of passing around tuples of values in communications. 
We consider a polyadic extension of \HOp, denoted \pHOp.
In \pHOp polyadicity appears in intra-session communications and abstractions; 
polyadicity of shared names is ruled out by typing. 
This is enough for most practical purposes, including our encoding from \HOp into \HO.
Encoding \pHOp into \HOp is straightforward
in a session-typed setting, as detailed in \S\,\ref{subsec:pho}.
For instance, we may define:
\[
\begin{array}{rl}
		\map{\binp{u}{x_1, \cdots, x_m} P}
		 =  & \!\!\!\!
		\binp{u}{x_1} \cdots ;  \binp{u}{x_m} \map{P}
		\\
		\map{\bout{u}{u_1, \cdots, u_m} P}
		 =  & \!\!\!\!
		\bout{u}{u_1} \cdots ;  \bout{u}{u_m} \map{P}
		\\
		\map{\bbout{u}{\abs{(x_1, \cdots, x_m)} Q} P}
		= & \!\!\!\!
		\bbout{u}{\abs{z}\binp{z}{x_1}\cdots ; \binp{z}{x_m} \map{Q}} \map{P}
		\\ 
		\map{\appl{x}{(u_1, \cdots, u_m)}}
		= & \!\!\!\!
		\newsp{s}{\appl{x}{s} \Par \bout{\dual{s}}{u_1} \cdots ; \bout{\dual{s}}{u_m} \inact} 
	\end{array}
\]
Thus, the polyadic arguments are communicated one by one, relying on the linear, private character of 
session names. The encoding of polyadic abstraction/application requires an extra step, 
in which a  monadic abstraction is passed around.

\smallskip

\myparagraph{A Non Encodability Result.}
We also show that shared names strictly add expressiveness to session calculi:
there are behaviors expressible with shared names not expressible using linear names only.
Although somewhat expected we do not know of a formal proof.
We propose such a formal proof, which relies crucially on the behavioral theory that we have introduced here
and its properties.


%%\myparagraph{Outline}
%\subsection{Outline}
%\noi \S\,\ref{sec:calculus} presents the calculi; 
%\S\,\ref{sec:types} presents types;
%the tractable bisimulations are in \S\,\ref{sec:behavioural};
%the notion of encoding is in \S\,\ref{s:expr};
%\S\,\ref{sec:positive} and \S\,\ref{sec:negative}
%present positive and negative encodability results, resp;
%\S\,\ref{sec:extension} discusses extensions; and 
%\S\,\ref{sec:relwork} concludes with related work;
%Appendix summarises the typing system. 
%The paper is self-contained. 
%{\bf\em Omitted definitions, additional related work and full proofs can be found 
%in a technical report, available from \cite{KouzapasPY15}.} 
