% !TEX root = main.tex
\myparagraph{Expressiveness.}
There is a vast literature on expressiveness studies for process calculi. 
For space reasons here we concentrate on closely related work; 
see~\cite{KouzapasPY15} for more detailed comparisons with other literature. 
%To formalise claims of (relative) expressiveness,
%%early works appealed to different definitions of encoding \cite{Palamidessi03}.
%%Later on, 
%abstract frameworks which define encodings and their 
%associated syntactic and semantic criteria 
%have been developed; 
%two proposals are~\cite{DBLP:journals/iandc/Gorla10,DBLP:journals/tcs/FuL10}. 
%These frameworks are applicable to different calculi, and 
%have shown useful to clarify known results and to derive new ones.
%Our formulation of (precise) typed encoding (\defref{def:goodenc}) 
%builds upon existing proposals (including~\cite{Palamidessi03,DBLP:journals/iandc/Gorla10,DBLP:conf/icalp/LanesePSS10})
%in order to account for the session type systems
%associated to \HOp and its variants/extensions.
%
%\myparagraph{Expressiveness of \emph{Higher-Order} Calculi.}
%Due to the close relationship between
%higher-order process calculi and functional calculi, works devoted to
%encoding (variants of) the $\lambda$-calculus into (variants of) the
%$\pi$-calculus~ (e.g.,~\cite{San92,DBLP:journals/tcs/Fu99}) are broadly related.
The encoding of process-passing into name-passing is well-known~\cite{SangiorgiD:expmpa};
an encoding in the reverse direction 
is given in~\cite{SaWabook} for an asynchronous, localised $\pi$-calculus
(only the output capability of names can be sent around).  The
work~\cite{San96int} studies hierarchies for calculi with
\emph{internal} first-order mobility and with higher-order mobility
without name-passing (like \HO). The
hierarchies are %based on expressivity: formally 
defined according to
the order of types needed in typing. 
%, they describe different ``degrees of mobility''.  
Via type-preserving fully abstract encodings, it is shown that 
name- and process-passing calculi with equal order of types have the
same expressiveness.  With respect to these previous results, our
approach based on session types has important consequences and
allows us to derive new results.  
Our study stresses the 
view of ``encodings as protocols'', namely session protocols which
enforce clean linear and shared disciplines for names, a distinction
not explored in~\cite{SangiorgiD:expmpa,DBLP:journals/tcs/Sangiorgi01}. In
turn, this distinction is central in proper definitions
of trigger processes, which are key to encodings
(\defref{d:enc:hopitopi}) and behavioural equivalences
(\defref{d:hbw} and~\ref{d:fwb}).  More interestingly, we showed that
$\HO$ suffices to encode  the session
calculus with name passing ($\sessp$) but also $\HOp$ and its extension with
higher-order applications ($\HOpp$). 
Thus, %using session types
all these session calculi are equally expressive with fully
abstract encodings.  To our knowledge, these are the first
expressivity results of this kind.

\jpc{Building upon~\cite{ThomsenB:plachoasgcfhop},
the work~\cite{XuActa2012} studies 
the (non)encodability of the $\pi$-calculus into 
a higher-order $\pi$-calculus with a powerful 
name relabelling operator, which is 
shown to be essential in encoding name-passing}. %, following \cite{Tho90}.
A core higher-order calculus is studied in~\cite{DBLP:journals/iandc/LanesePSS11}: 
it lacks restriction,  name passing, output prefix %(communication is asynchronous), 
and constructs for infinite behaviour. 
This calculus  has 
a simple notion of bisimilarity which coincides with 
%reduction-closed, barbed congruence.
contextual equivalence.
%be Turing complete, while 
%have a decidable notion of (strong) bisimilarity that coincides with barbed congruence. 
\jpc{
The absence of restriction plays a key role in the characterisations in~\cite{DBLP:journals/iandc/LanesePSS11};
hence, our characterisation of contextual equivalence for \HO (which has restriction)
cannot be derived from that in~\cite{DBLP:journals/iandc/LanesePSS11}. 
} 

%Our work is closely related in spirit to the expressiveness studies in~\cite{DBLP:conf/icalp/LanesePSS10,DBLP:conf/wsfm/XuYL13}.
In~\cite{DBLP:conf/icalp/LanesePSS10}
the core calculus in~\cite{DBLP:journals/iandc/LanesePSS11} is extended with restriction,
synchronous communication, and polyadicity. It is shown that 
synchronous communication can encode asynchronous communication, % (as in the first-order setting),
and that process passing polyadicity induces a hierarchy in expressive power.
Encodability criteria does not include full abstraction.
 % (unlike the first-order setting).
%A further extension with process abstractions of order one
%(functions from processes to processes)
% is shown to strictly add expressive power with respect to passing of processes only.
\jpc{The paper~\cite{DBLP:conf/wsfm/XuYL13} 
complements~\cite{DBLP:conf/icalp/LanesePSS10} 
by studying the expressivity %of second-order abstractions.
%with replication ($!P$).  
%The work \cite{DBLP:conf/wsfm/XuYL13} focuses  
%%name and process abstractions are distinguished and contrasted, also 
%on expressiveness of the hirarchy of polyadic abstraction parameters. 
%(the same kind of polyadicity present in \pHOp)
%By adapting the encodings in~\cite{DBLP:conf/icalp/LanesePSS10} 
%Polyadicity 
of 
second-order process abstractions.
Polyadicity is shown to induce an expressiveness hierarchy; 
also,
by adapting the encoding in~\cite{SangiorgiD:expmpa},
process abstractions are encoded into name abstractions.
In contrast, we 
give a fully abstract encoding of
 \PHOpp into \HO that preserves session types; this improves~\cite{DBLP:conf/icalp/LanesePSS10,DBLP:conf/wsfm/XuYL13}   
by enforcing linearity disciplines on process behaviour.
Also, the focus of~\cite{DBLP:conf/icalp/LanesePSS10,DBLP:conf/wsfm/XuYL13} is on 
the expressiveness of untyped, higher-order processes; they
%Moreover,~\cite{DBLP:conf/icalp/LanesePSS10,DBLP:conf/wsfm/XuYL13}
do not address 
tractable equivalences for processes  (such as 
$\hwb$ and $\fwb$) which only require observation of finite %number of 
%higher-order 
values,  
whose formulations rely on session types.}
%therefore, our work complements their  results. 
% by clarifying the status of typeful %, resource-aware 
%structured communications. % in trigger-based representations of process passing, both in encodings and  equivalences.

\myparagraph{Session Typed Processes.}
The works~\cite{DemangeonH11,Dardha:2012:STR:2370776.2370794} 
study encodings of binary session calculi into a linearly typed $\pi$-calculus. 
While~\cite{DemangeonH11}~gives a precise encoding of \sessp into a linear calculus 
(an extension of \cite{BHY}),  
the work~\cite{Dardha:2012:STR:2370776.2370794} 
gives operational correspondence (without full abstraction, cf.~\defref{def:sep}-4)
for the first- and higher-order 
$\pi$-calculi into \cite{LinearPi}. 
They investigate embeddability of two different typing systems;
by the result of \cite{DemangeonH11}, 
\HOpp is encodable  into the linearly typed $\pi$-calculi.     

The syntax of $\HOp$ is a subset of that in~\cite{tlca07,MostrousY15}.
The work~\cite{tlca07} develops a full higher-order session calculus
with process abstractions and applications; it admits the type 
$U=U_1 \rightarrow U_2 \dots U_n \rightarrow \Proc$ and its linear type 
$U^1$
which corresponds to $\shot{\tilde{U}}$ and $\lhot{\tilde{U}}$ in 
a super-calculus of $\HOpp$ and $\PHOp$. 
%in~\cite{MostrousY15} in the asynchronous setting.
%The session type
%system considered is influenced by the type systems for $\lambda$-calculi and
%uses type syntax of the form $U_1 \rightarrow U_2 \dots U_n \rightarrow \Proc$
%for shared values and $(U_1 \rightarrow U_2 \dots U_n \rightarrow \Proc)^{1}$
%for linear values.
%Such a type is expressed in $\HOpp$
%terms using the type $\shot{U}$ (respectively, $\lhot{U}$)
%with $U$ being a nested higher-order type; and 
%the $\HOp$ uses only types of the form
%$\shot{C}$ and $\lhot{C}$ with $C$ being a first-order channel type.
Our results show that
the calculus in~\cite{tlca07} is not only expressed but 
also reasoned in 
$\HO$ (with limited form of arrow types, $\shot{C}$ and $\lhot{C}$), via precise encodings. \dk{None of the above works proposes tractable 
bisimulations for higher-order processes.}  

\myparagraph{Typed Behavioural Equivalences.}
\NY{This work follows 
the 
%principles for
session type behavioural semantics in 
\cite{KYHH2015,KY2015,DBLP:journals/iandc/PerezCPT14}
where a bisimulation is defined on a LTS 
that assumes a session typed
observer.
%The bisimilarity is characterised by the corresponding
%reduction-closed, barbed congruence using techniques derived from~\cite{Hennessy07}.
Our theory for higher-order session types 
differentiates from 
the work in~\cite{KYHH2015,KY2015}, which 
considers the first-order
binary and multiparty session types, respectively.
The work \cite{DBLP:journals/iandc/PerezCPT14} gives a behavioural theory 
for a 
logically motivated
language of binary sessions 
without shared names.}
%Determinacy properties (confluence, $\tau$-inertness) are proven.

%The theory for higher-order session type quivalences is more challenging than
%their corresponding first-order bisimulation theory.
Our approach %for the higher-order 
to typed equivalences
builds upon techniques by Sangiorgi~\cite{SangiorgiD:expmpa,San96H}
and Jeffrey and Rathke~\cite{JeffreyR05}.
The work %Sangiorgi as part of his Ph.D.~research
%\cite{San96H,SangiorgiD:expmpa}
\cite{SangiorgiD:expmpa}
introduced the first fully-abstract encoding from the higher-order 
$\pi$-calculus into the $\pi$-calculus. 
Sangiorgi's encoding is based on the idea of a replicated input-guarded process 
(a trigger process). We use a similar 
replicated triggered process 
to encode \HOp into \sessp (\defref{d:enc:hopitopi}).
 Operational correspondence for
the triggered encoding is shown using a context bisimulation
with first-order labels.
%Although contextual bisimilarity has a satisfactory discriminative power,
%its use is hindered by the universal quantification on output clauses.
To deal with the issue of context bisimilarity, 
Sangiorgi proposes \emph{normal bisimilarity}, 
a tractable  equivalence without universal quantification. 
To prove that context and normal bisimilarities coincide,~\cite{SangiorgiD:expmpa} uses 
triggered processes.
%The encoding also motivates the definition of a form of
Triggered bisimulation is also defined on first-order labels
where the context bisimulation is restricted to arbitrary
trigger substitution. %rather than arbitrary process substitutions.
This
characterisation of context bisimilarity  was refined in~\cite{JeffreyR05} for
calculi with recursive types, not addressed in~\cite{San96H,SangiorgiD:expmpa} and
quite relevant in our work (cf. \defref{d:enc:hopitoho}).
The
bisimulation in~\cite{JeffreyR05}
is based on an LTS which is extended with trigger meta-notation.
%for a full higher-order $\pi$-calculus that allows
%higher-order applications.
As in~\cite{San96H,SangiorgiD:expmpa}, 
the LTS in~\cite{JeffreyR05}
observes first-order triggered values instead of
higher-order values, offering a more direct characterisation of contextual equivalence
and lifting the restriction to finite types.
We briefly contrast 
the approach in~\cite{JeffreyR05} and ours based on 
\dk{higher-order ($\hwb$) and} characteristic ($\fwb$) bisimilarities:
\begin{enumerate}[$\bullet$]
%\begin{enumerate}[i.]
\item 
The LTS in~\cite{JeffreyR05} is enriched with extra labels for triggers;
an output action transition emits a trigger and introduces a parallel replicated trigger.
Our 
approach retains usual labels/transitions; in  case of output,
%our bisimilarities 
$\hwb$ and $\fwb$
introduce a parallel
non-replicated trigger.
\item Higher-order input in~\cite{JeffreyR05} involves 
the input of a trigger which reduces after substitution.
Rather than a trigger name, %our bisimulations  
$\hwb$ and $\fwb$
decree the input of a triggered value $\abs{z}\binp{t}{x} \appl{x}{z}$.
\item Unlike~\cite{JeffreyR05}, 
%our 
$\fwb$ treats  
first- and higher-order values uniformly. %In the latter case, 
%Since the 
As the typed LTS distinguishes linear and shared values,
replicated closures are used only for shared values.

\item In~\cite{JeffreyR05} a matching construct is
crucial to prove completeness of bisimilarity,
while our calculi lack matching. 
%Contrarily 
\jpc{In contrast,} 
we use the characteristic
process interaction with the environment, exploiting 
session type structures, i.e., instead of matching 
a name is embedded into a process and then observe its behaviour.

%In~\cite{JeffreyR05}  a matching construct 
%is crucial to prove completeness of bisimilarity.
%Since our language lacks matching,
%we use session type information to obtain the simplest value that 
%enables interaction with the environment.

\end{enumerate}

%There are similarities and differences between of the characteristic bisimulation
%and the bisimulation as defined by Jeffrey and Rathke
%(below we use the meta-notation adopted in~\cite{JeffreyR05}):
%%
%\begin{enumerate}[i)]
%	\item	The output of a higher-order value $\abs{x}{Q}$ on name
%		$n$ in Jeffrey\&Rathke approach requires the output of
%		a fresh trigger name $t$ (notation $\tau_t$)
%		on name $n$ 
%		and then the introduction of a replicated triggered process
%		(notation $(t \Leftarrow (x) Q)$)
%		in the context of the acting process:
%		%
%		\[
%			P \by{\news{t} \bactout{n}{\tau_{t}}} P' \Par (t \Leftarrow (x) Q) \by{\bactinp{t}{v}} P' \Par \appl{(x) Q}{v} \Par (t \Leftarrow (x) Q) 
%		\]
%		%
%		In the characteristic bisimulation approach we only observe
%		an output of a value that can be either first- or higher-order:
%		%
%		\[
%			P \hby{\bactout{n}{V}} P' 
%		\]
%		%
%		with $V = \abs{x}{Q}$ or $V = m$.
%		A non-replicated triggered process appears in
%		the parallel context of the acting process when
%		we compare two processes for behavioural equality
%		(cf.~characteristic bisimulation \defref{d:fwb}),
%		$P' \Par \htrigger{t}{\abs{x}{Q}}$.
%		In fact using the LTS in
%		\defref{d:tlts} we can get:
%		%
%		\begin{eqnarray*}
%			P' \Par \htrigger{t}{\abs{x}{Q}}
%			&\by{\abs{z}{\binp{z}{y} \repl{} \binp{t}{x} \appl{y}{x}}}&
%			P' \Par \newsp{s}{\binp{s}{y} \repl{} \binp{t}{x} \appl{y}{x} \Par \bout{s}{\abs{x}{Q}} \inact}\\
%			&\by{\tau}&
%			P' \Par \repl{}\binp{t}{y} \appl{\abs{x}{Q}}{y}
%		\end{eqnarray*}
%		%
%		that simulates the Jeffrey\&Rathke approach.
%
%		The characteristic bisimulation differentiates from
%		the Jeffrey\&Rathke approach:
%		\begin{enumerate}[$\bullet$]
%			\item	The typed LTS predicts the case of linear
%				output values and will never allow replication
%				of such a value;
%				if $V$ is linear the input action would have no replication
%				operator, as
%				$\abs{z}{\binp{z}{y} \binp{t}{x} \appl{y}{x}}$.
%
%			\item	The characteristic bisimulation introduces a uniform approach
%				not only for
%				higher-order values but for first-order values
%				as well. A triggered process can accept any
%				process that can substitute a first-order value as well.
%				This is derived from the fact that the $\HOp$
%				calculus makes no use of a matching operator, in contrast
%				to the calculus defined in~\cite{JeffreyR05})
%				where name matching is crucial to prove completness
%				of the bisimilarity relation.
%				We chose not to include the matching operator
%				because of the requirement of a minimal calculus.
%				In the lack of matching we use types to inhabit
%				a value so we can observe its simplest interaction
%				with the process environment.
%
%			\item	The \HOp calculus requires only first-order
%				applications. Higher-order applications,
%				as in the Jeffrey\&Rathke work,
%				are presented as an extension in the \HOpp
%				calculus.
%
%			\item	The trigger process is non-replicated. In fact
%				the trigger process transforms guards the output
%				value with a higher-order input prefix. The
%				functionality of the input is used to
%				simulate the contextual bisimilarity that subsumes
%				the replicated trigger approach.
%				The transformation of an output action as an input
%				action allows for treating an output
%				using the restricted LTS \defref{def:rlts}:
%				%
%				\[
%					P' \Par \htrigger{t}{\abs{x}{Q}} \hby{\bactinp{t}{\abs{x}{\mapchar{U}{x}}}}
%					P' \Par \news{s}{ \appl{\mapchar{U}{x}}{s} \Par \bout{\dual{s}}{\abs{x}{Q}} \inact}
%				\]
%		\end{enumerate}
%		%
%		%In essence we are transforming a replicated trigger into a process
%		%that is input-prefixed on a fresh name that receives a higher-order
%		%value;
%
%	\item	The input of a higher-order value in the Jeffrey\&Rathke approach requires 
%		the input of a fresh trigger name, which is substituted on the application
%		variable, thus having a meta-suntax for triggered application instead
%		of higher-order applications:
%		%
%		\[
%			\binp{n}{x} P \by{\bactinp{n}{\tau_k}} \appl{\abs{x}{P}}{\tau_k} \by{\tau} P \subst{x}{\tau_k} 
%		\]
%		%
%		with every instance of process variable $x$ in $P$ being substituted
%		with trigger value $\tau_k$ to give a process of the form $\appl{\tau_k}{x}$.
%		The approach in the characteristic bisimulation observes the
%		triggered value
%		$\abs{z}\binp{t}{x} \appl{x}{z}$ as an input instead of the
%		trigger name:
%		%
%		\[
%			P \hby{\bactinp{n}{\abs{z}\binp{t}{x} \appl{x}{z}}} P \subst{\abs{z}\binp{t}{x} \appl{x}{z}}{x}
%		\]
%		%
%		with applications being transformed to
%		$\abs{z}{\binp{t}{x} \appl{x}{z}}{v}$
%		Note that in the characteristic bisimulation semantics
%		we can also observe a characteristic process as an input.
%		
%	\item 	Triggered application in the Jeffrey\&Rahtke
%		are observe using an output
%		lead into an output observation of the
%		application value over
%		the fresh trigger name.
%		%
%		\[
%			\appl{\tau_k}{v} \by{\bactout{k}{v}} \inact
%		\]
%		%
%		In the characteristic bisimulation instead of observing an 
%		application and its value as an action we observe:
%		i) the name of the trigger through the trigger value
%		application; and ii) the application
%		value by inhabiting it in the characteristic value
%		and observing the interaction of the corresponding
%		characteristic process with its environment.
%		%
%		\begin{eqnarray*}
%			\appl{\abs{z}{\binp{t}{x} \appl{x}{z}}}{v} &\by{\tau}& \binp{t}{x} \appl{x}{v}
%			\by{\bactinp{t}{\abs{x}{\mapchar{U}{x}}}}
%			\appl{\abs{x}{\mapchar{U}{x}}}{v}
%			\by{\tau} \mapchar{U}{x} \subst{n}{x}
%		\end{eqnarray*}
%		%
%\end{enumerate}

%The main differences of the triggered
%bisimulation approach comparing to our approach are:
%i) We use observe higher-order values on the LTS in contrast to first-order 
%values in~\cite{DBLP:journals/lmcs/JeffreyR05}.
%ii) In our approach we avoid the replicated triggered process,
%by transforming the output process into a higher-order guarded input.
%iii) The triggered bisimulation gives semantics for higher-order application,
%whereas in our approach we give semantics for first-order applications
%and show that higher-order applications are fully encodable.

%Boreale and Sangiorgi, 
%Deng and Hennessy, 
%Jeffrey and Rathke, Hennessy and Koutavas, Schmitt and Lenglet, Pi\E9rard and Sumii.
%Perez et al (bisimilarities for binary sessions), Kouzapas and Yoshida (bisimilarities for binary and multiparty sessions).
%Bisimilarities for HO processes: \cite{Xu07}.

\noi 
The \emph{environmental bisimulations} given in~\cite{DBLP:conf/lics/SangiorgiKS07} 
%which 
%Sangiorgi et al.~\cite{DBLP:conf/lics/SangiorgiKS07}, 
use a higher-order LTS 
to define a bisimulation that stores the observer's knowledge; hence, observed actions are based on this knowledge
at any given time. This approach is enhanced in~\cite{DBLP:journals/cl/KoutavasH12,DBLP:conf/esop/KoutavasH11}
where a mapping from constants to higher-order values is introduced. This 
allows to observe first-order values instead
of higher-order values. It differs from~\cite{San96H,JeffreyR05} in that 
the mapping between higher- and first-order values is no longer implicit.





