\noi This section defines the formal notion of \emph{encoding}, 
extending the existing untyped criteria studied in 
\cite{Palamidessi,Gorhan,AAA} to a typed settion. 
We first define a typed calculus parameterised by 
by a syntax, typing and semantics. 

\smallskip 

\begin{definition}[Typed Calculus]\label{d:tcalculus}\rm
	A \emph{typed calculus} $\tyl{L}$ is a tuple
          $\calc{\CAL}{\cal{T}}{\hby{\ell}}{\wb}{\proves}$
	where $\CAL$ and $\cal{T}$ are set of processes and types, 
respectively; and $\hby{\ell}$, $\wb$, and $\proves$ 
	denote a transition system, a typed process equivalence, and a type system for $\CAL$ processes, respectively. 
\end{definition}

\smallskip 

\noi Our notion of encoding considers a mapping on processes, 
types, and transition labels.  

\begin{definition}[Typed Encoding]\rm
        Let  $\tyl{L}_i=\calc{\CAL_i}{{\cal{T}}_i}{\hby{\ell}}{\wb_i}{\proves_i}$
        ($i=1,2$) be typed calculi. 
	Given mappings $\map{\cdot}: \CAL_1 \to \CAL_2$, 
	$\mapt{\cdot}: {\cal{T}}_1 \to {\cal{T}}_2$, and 
	$\mapa{\cdot}: \ell \to \ell$, 
	we write 
	$\enco{\map{\cdot}, \mapt{\cdot}, \mapa{\cdot}} : \tyl{L}_1 \to \tyl{L}_2$ to denote the \emph{typed encoding} of $\tyl{L}_1$ into $\tyl{L}_2$.
\end{definition}

\smallskip 

\noi We will often assume that  $\mapt{\cdot}$ extends to typing
environments as expected. This way, e.g., $\mapt{\Delta \cat u:S} = \mapt{\Delta} \cat u:\mapt{S}$.

For stating stronger positive results of encodings, 
we define the strong encoding criteria (called {\em precise}). Later 
they are weakened to prove a stronger negative result (called {\em good}). 
We first state the 
syntactic criteria which are used in the untyped setting \cite{}. 
Let $\sigma$ denote a substitution of names for names (a renaming, in the usual sense). Given environments $\Delta$ and $\Gamma$,
we write $\sigma(\Delta)$ and $\sigma(\Gamma)$ to denote the effect of applying $\sigma$ on the 
domains of $\Delta$ and $\Gamma$
(clearly, $\sigma(\Gamma)$ concerns only shared names in $\Gamma$: process and recursion variables in $\Gamma$ are not affected by $\sigma$). 

\smallskip 

\begin{definition}[Syntax Preserving]\rm
	\label{def:sep}
	We say that 
	the typed encoding 
	$\enco{\map{\cdot}, \mapt{\cdot}, \mapa{\cdot}}: \tyl{L}_1 \to \tyl{L}_2$ is \emph{syntax preserving}
	if it is:
	
	\begin{enumerate}[1.]
		\item	\emph{Homomorphic wrt parallel},   if 
		$\mapt{\Gamma}; \emptyset; \mapt{\Delta_1 \cat \Delta_2} \proves \map{P_1 \Par P_2} \hastype \Proc$
		then 
		$\mapt{\Gamma}; \emptyset; \mapt{\Delta_1} \cat \mapt{\Delta_2} \proves \map{P_1} \Par \map{P_2} \hastype \Proc$.

		\item	\emph{Compositional wrt restriction},  if 
		$\mapt{\Gamma}; \emptyset; \mapt{\Delta} \proves \map{\news{n}P} \hastype \Proc$
		then 
		$\mapt{\Gamma}; \emptyset; \mapt{\Delta} \proves \news{n}\map{P} \hastype \Proc$.
		
		\item \emph{Name invariant},   if
		$\mapt{\sigma(\Gamma)}; \emptyset; \mapt{\sigma(\Delta)} \proves \map{\sigma(P)} \hastype \Proc$
		then 
		$\sigma(\mapt{\Gamma}); \emptyset; \sigma(\mapt{\Delta}) \proves \sigma(\map{P}) \hastype \Proc$, 
		for any injective renaming  of names $\sigma$.
	\end{enumerate}
\end{definition}

\smallskip 

\noi Next we define semantic criteria for typed encodings.

\smallskip 

\begin{definition}[Semantic Preserving Encoding]\rm
\label{def:ep}
       Let  $\tyl{L}_i=\calc{\CAL_i}{{\cal{T}}_i}{\hby{\ell}}{\wb_i}{\proves_i}$
       ($i=1,2$) be typed calculi. 
We say that $\enco{\map{\cdot}, \mapt{\cdot}, \mapa{\cdot}}: \tyl{L}_1 \to \tyl{L}_2$ is a \emph{semantic preserving encoding}
if it satisfies the following properties:
	
	\begin{enumerate}[1.]
		\item \emph{Type Preservation}:
	if
	$\Gamma; \emptyset; \Delta \proves_1 P \hastype \Proc$ then 
	$\mapt{\Gamma}; \emptyset; \mapt{\Delta} \proves_2 \map{P} \hastype \Proc$,  
	for any   $P$ in $\CAL_1$.

	\item \emph{Operational Correspondence}: If $\Gamma; \emptyset; \Delta \proves_1 P \hastype \Proc$ then
		\begin{enumerate}
%			\item	Completeness: \\
%			    If $P \red_1 P'$ then $\exists\, \Delta'$ s.t.
%				$\map{P} \Red_2 \map{P'}$ and
%				$\mapt{\Gamma}; \emptyset; \mapt{\Delta'} \proves_2 \map{P'} \hastype \Proc$.
			\item	Completeness: 
			   If  
$\stytraarg{\Gamma}{\ell_1}{\Delta_1}{P}{\Delta'_1}{P'}{1}$
			   then $\exists \ell_2$ s.t. 

$\wtytraarg{\mapt{\Gamma}}{\ell_2}{\mapt{\Delta_1}}{\map{P}}{\mapt{\Delta'_1}}{\map{P'}}{2}$
			    and $\ell_2 = \mapa{\ell_1}$.
			    				
%			\item Soundness : \\
%			    If $\map{P} \red_2 Q$ then
%				$\exists P'$ s.t. $P \red_1 P'$ and 
%				$\mapt{\Gamma}; \mapt{\Delta_1} \wb_2 \mapt{\Delta_2} \proves_2 \map{P'} \wb_2 Q$.
				
			\item	Soundness:   
				If  $\wtytraarg{\mapt{\Gamma}}{\ell_2}{\mapt{\Delta}}{\map{P}}{\mapt{\Delta'}}{Q}{2}$
				then $\exists \ell_1, P'$ s.t.  
				(i)~$\stytraarg{\Gamma}{\ell_1}{\Delta}{P}{\Delta'}{P'}{1}$,
				(ii)~$\ell_2 = \mapa{\ell_1}$, and
				(iii)~
${\mapt{\Gamma}};{\mapt{\Delta'}}\proves_2 {\map{P'}}{\wb_2}
{\mapt{\Delta'}}\proves_2 {Q}$.

%			If  $\wtytraarg{\mapt{\Gamma}}{\ell_2}{\mapt{\Delta_1}}{\map{P}}{\mapt{\Delta'_1}}{Q}{2}$
%			   then $\exists \ell_1, P'$ s.t.  
%			    (i)~$\stytraarg{\Gamma}{\ell_1}{\Delta_1}{P}{\Delta'_1}{P'}{1}$,
%			    (ii)~~$\Hby{\mapa{\ell}} = \hby{\ell_1} \Hby{\ell_2}$ and 
%			    (iii)~
%				$\mapt{\Gamma}; \mapt{\Delta_1} 
%\proves_2 \map{P'}  \wb_2 \mapt{\Delta_2} \proves_2 Q$.
%$\wbb{\mapt{\Gamma}}{\ell}{\mapt{\Delta'_1}}{\map{P'}}{\mapt{\Delta'_1}}{Q}$.
		\end{enumerate}
		
		\item \emph{Full Abstraction:} 
%		$\Gamma; \Delta_1 \wb \Delta_2 \proves P \wb Q $ if and only if $\mapt{\Gamma}; \mapt{\Delta_1} \wb \mapt{\Delta_2} \proves \map{P} \wb \map{Q} $.
		\wbbarg{\Gamma}{}{\Delta_1}{P}{\Delta_2}{Q}{1}
		if and only if
		\wbbarg{\mapt{\Gamma}}{}{\mapt{\Delta_1}}{\map{P}}{\mapt{\Delta_2}}{\map{Q}}{2}.
	\end{enumerate}
\end{definition}

\smallskip 

\noi Type preservation enables us to focus on encodings which retain the communication structures denoted by (session) types.
Our operational correspondence builds upon analogous definitions in the untyped setting \cite{}. 
We used labeled transitions, and so correspondences involve the mapping $\mapa{\cdot}$.
Soundness ensures that the source process is mimicked 
by its associated encoding; and completeness concerns the opposite direction.
In the encodability results in the next section, 
we prove stronger statements, tailored to the specifics of each encoding.
Finally, we consider the full abstraction as a criteria to demonstrate 
a stronger positive result. The completeness direction 
of full abstraction is dropped when we prove the negative result.  
We now define:

\smallskip 


%\begin{remark}[Polyadic \HOp] We can trivially define 
%a semantic and type preserving encoding from 
%a polyadic version of \HOp (which extends $\HOp$ to 
%$V$ to $\tilde{n}$ and $\abs{\tilde{x}}{Q}$) into the monadic \HOp.  
%Hence we use the polyadic 
%version of \HOp in the encodings.  
%\end{remark}

\begin{definition}[Typed Encodings: Good and Basic]\label{def:goodenc}
We say that the typed encoding 
	$\enco{\map{\cdot}, \mapt{\cdot}, \mapa{\cdot}}: \tyl{L}_1 \to \tyl{L}_2$ is 
\begin{enumerate}
\item \emph{precise}, if it is syntax preserving and semantic preserving (cf. Definitions \ref{def:sep} and  \ref{def:ep}).
\item \emph{good}, if it is syntax preserving 
(cf.~Definition \ref{def:sep}),
barb preserving, and enjoys operational completeness (cf.~Definition \ref{def:ep}-2~a)).
\end{enumerate}
\end{definition}
Precise encodings offer more detailed criteria and used for positive 
encodability results (\S\,\ref{sec:positive}).
In contrast, good encodings contains only 
some of the criteria of precise encodings:    
this minimal notion will be used 
for the negative result in \S\,\ref{sec:negative}. Further we have:

\smallskip 

\begin{proposition}[Composability of Semantic Preserving Encodings]
	Let 
	$\enco{\map{\cdot}^{1}, \mapt{\cdot}^{1}, \mapa{\cdot}^{1}}: \tyl{L}_1 \to \tyl{L}_2$
	and 
	$\enco{\map{\cdot}^{2}, \mapt{\cdot}^{2}, \mapa{\cdot}^{2}}: \tyl{L}_2 \to \tyl{L}_3$
%	$\enco{\cdot}{\cdot}{1}: \tyl{L}_1 \to \tyl{L}_2$ and $\encod{\cdot}{\cdot}{2}: \tyl{L}_2 \to \tyl{L}_3$
	be two semantic preserving encodings.
	Then their composition, denoted 
	$\enco{\map{\cdot}^{1} \circ \map{\cdot}^{2}, \mapt{\cdot}^{1} \circ \mapt{\cdot}^{2}, \mapa{\cdot}^{1}\circ \mapa{\cdot}^{2}}: \tyl{L}_1 \to \tyl{L}_3$
	is also a semantic preserving encoding.
\end{proposition}


