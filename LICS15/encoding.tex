\begin{definition}[Typed Calculus]\label{d:tcalculus}\rm
	A \emph{typed calculus} $\tyl{L}$ is a tuple
          $\calc{\CAL}{\cal{T}}{\hby{\ell}}{\wb}{\proves}$
	where $\CAL$ and $\cal{T}$ are set of processes and types, 
respectively; and $\hby{\ell}$, $\wb$, and $\proves$ 
	denote a transition system, a typed process equivalence, and a type system for $\CAL$ processes, respectively. 
\end{definition}

\begin{definition}[Typed Encoding]\rm
        Let  $\tyl{L}_i=\calc{\CAL_i}{{\cal{T}}_i}{\hby{\ell}}{\wb_i}{\proves_i}$
        ($i=1,2$) be typed calculi. 
	Given mappings $\map{\cdot}: \CAL_1 \to \CAL_2$, 
	$\mapt{\cdot}: {\cal{T}}_1 \to {\cal{T}}_2$, and 
	$\mapa{\cdot}: \ell \to \ell$, 
	we write 
	$\enco{\map{\cdot}, \mapt{\cdot}, \mapa{\cdot}} : \tyl{L}_1 \to \tyl{L}_2$ to denote the \emph{typed encoding} of $\tyl{L}_1$ into $\tyl{L}_2$.
\end{definition}

%Below 
%we  assume that if 
%$P \hby{\ell} P'$ and $\mapa{\ell} = \{\ell_1, \ell_2,  \cdots, \ell_m\}$ then
%$\map{P} \Hby{\mapa{\ell}} \map{P'}$
%should be understood as
%$\map{P} \Hby{\ell_1} P_1 \Hby{\ell_2} P_2 \cdots \Hby{\ell_m} P_m =  \map{P'}$,
%for some
%$P_1, P_2, \ldots, P_m$.


\begin{definition}[Semantic Preserving Encoding]\rm
\label{def:ep}
       Let  $\tyl{L}_i=\calc{\CAL_i}{{\cal{T}}_i}{\hby{\ell}}{\wb_i}{\proves_i}$
       ($i=1,2$) be typed calculi. 
We say that $\enco{\map{\cdot}, \mapt{\cdot}, \mapa{\cdot}}: \tyl{L}_1 \to \tyl{L}_2$ is a \emph{semantic preserving encoding}
if it satisfies the following properties:
	
	\begin{enumerate}[1.]
		\item \emph{Type preservation}:
	if
	$\Gamma; \emptyset; \Delta \proves_1 P \hastype \Proc$ then 
	$\mapt{\Gamma}; \emptyset; \mapt{\Delta} \proves_2 \map{P} \hastype \Proc$,  
	for any   $P$ in $\CAL_1$.

	\item \emph{Operational Correspondence}: If $\Gamma; \emptyset; \Delta \proves_1 P \hastype \Proc$ then
		\begin{enumerate}[-]
%			\item	Completeness: \\
%			    If $P \red_1 P'$ then $\exists\, \Delta'$ s.t.
%				$\map{P} \Red_2 \map{P'}$ and
%				$\mapt{\Gamma}; \emptyset; \mapt{\Delta'} \proves_2 \map{P'} \hastype \Proc$.
			\item	Completeness: 
			   If  $\stytraarg{\Gamma}{\ell_1}{\Delta_1}{P}{\Delta'_1}{P'}{1}$
			   then $\exists \ell_2$ s.t. 
			    $\wtytraarg{\mapt{\Gamma}}{\ell_2}{\mapt{\Delta_1}}{\map{P}}{\mapt{\Delta'_1}}{\map{P'}}{2}$
			    and $\ell_2 = \mapa{\ell_1}$.
			    				
%			\item Soundness : \\
%			    If $\map{P} \red_2 Q$ then
%				$\exists P'$ s.t. $P \red_1 P'$ and 
%				$\mapt{\Gamma}; \mapt{\Delta_1} \wb_2 \mapt{\Delta_2} \proves_2 \map{P'} \wb_2 Q$.
				
			\item	Soundness:   
				If  $\horel{\mapt{\Gamma}}{\mapt{\Delta_1}}{\map{P}}{\hby{\ell_1}}{\Delta''_1}{Q'}$
				then $\exists \ell, P'$ s.t.  \\
				(i)~$\stytra{\Gamma}{\ell}{\Delta_1}{P}{\Delta'_1}{P'}$,
				(ii)~$\horel{\mapt{\Gamma}}{\Delta''_1}{Q'}{\Hby{\ell_2}}{\mapt{\Delta'_1}}{Q}$,
				(iii)~$\Hby{\mapa{\ell}} = \hby{\ell_1} \Hby{\ell_2}$, and
%				(iv)~$\ell_2 = \mapa{\ell_1}$, and
				(iv)~$\horel{\mapt{\Gamma}}{\mapt{\Delta'_1}}{\map{P'}}{\wb_2}{\mapt{\Delta'_1}}{Q}$.
%			If  $\wtytraarg{\mapt{\Gamma}}{\ell_2}{\mapt{\Delta_1}}{\map{P}}{\mapt{\Delta'_1}}{Q}{2}$
%			   then $\exists \ell_1, P'$ s.t.  
%			    (i)~$\stytraarg{\Gamma}{\ell_1}{\Delta_1}{P}{\Delta'_1}{P'}{1}$,
%			    (ii)~~$\Hby{\mapa{\ell}} = \hby{\ell_1} \Hby{\ell_2}$ and 
%			    (iii)~
%				$\mapt{\Gamma}; \mapt{\Delta_1} 
%\proves_2 \map{P'}  \wb_2 \mapt{\Delta_2} \proves_2 Q$.
%$\wbb{\mapt{\Gamma}}{\ell}{\mapt{\Delta'_1}}{\map{P'}}{\mapt{\Delta'_1}}{Q}$.
		\end{enumerate}
		
		\item \emph{Full Abstraction:} 
%		$\Gamma; \Delta_1 \wb \Delta_2 \proves P \wb Q $ if and only if $\mapt{\Gamma}; \mapt{\Delta_1} \wb \mapt{\Delta_2} \proves \map{P} \wb \map{Q} $.
		\wbbarg{\Gamma}{}{\Delta_1}{P}{\Delta_2}{Q}{1}
		if and only if
		\wbbarg{\mapt{\Gamma}}{}{\mapt{\Delta_1}}{\map{P}}{\mapt{\Delta_2}}{\map{Q}}{2}.
	\end{enumerate}
\end{definition}

\begin{remark}[Polyadic \HOp] We can trivially define 
a semantic and type preserving encoding from 
a polyadic version of \HOp (which extends $\HOp$ to 
$V$ to $\tilde{n}$ and $\abs{\tilde{x}}{Q}$) into the monadic \HOp.  
Hence we use the polyadic 
version of \HOp in the encodings.  
\end{remark}

\begin{proposition}[Composability of Semantic Preserving Encodings]
	Let 
	$\enco{\map{\cdot}^{1}, \mapt{\cdot}^{1}, \mapa{\cdot}^{1}}: \tyl{L}_1 \to \tyl{L}_2$
	and 
	$\enco{\map{\cdot}^{2}, \mapt{\cdot}^{2}, \mapa{\cdot}^{2}}: \tyl{L}_2 \to \tyl{L}_3$
%	$\enco{\cdot}{\cdot}{1}: \tyl{L}_1 \to \tyl{L}_2$ and $\encod{\cdot}{\cdot}{2}: \tyl{L}_2 \to \tyl{L}_3$
	be two semantic preserving encodings.
	Then their composition, denoted 
	$\enco{\map{\cdot}^{1} \circ \map{\cdot}^{2}, \mapt{\cdot}^{1} \circ \mapt{\cdot}^{2}, \mapa{\cdot}^{1}\circ \mapa{\cdot}^{2}}: \tyl{L}_1 \to \tyl{L}_3$
	is also a semantic preserving encoding.
\end{proposition}


