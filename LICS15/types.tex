% !TEX root = main.tex
\noi %Here we define 
This section defines 
a session typing system for
\HOp and establish its main properties. 
Ours is simpler than that in~\cite{tlca07,mostrous_phd}, distilling the key
features of higher-order sessions. %communications. %in a session-typed setting.

\smallskip 

%\subsection{Types}
%\label{subsec:types}
\myparagraph{Types.}
The syntax of types of \HOp is given below: 
\[
\begin{array}{lrl}
\text{(value)}	& U \bnfis &	\nonhosyntax{C} \bnfbar L\\[1mm]  % \bnfbar \Proc
\text{(chan)}   & C  \bnfis &	S \bnfbar \chtype{S}\bnfbar \chtype{L}
	\\[1mm]
\text{(lambda)} & L \bnfis &	\shot{C} \bnfbar \lhot{C}
	\\[1mm]
\text{(session)} &  S \bnfis & 	\btout{U} S \bnfbar \btinp{U} S 
\bnfbar \btsel{l_i:S_i}_{i \in I} \\ 
 & \bnfbar & \btbra{l_i:S_i}_{i \in I}
	  \bnfbar  \trec{t}{S} \bnfbar \vart{t}  \bnfbar \tinact
\end{array}
\]
\dk{Value type $U$ includes
the first-order types $C$ and the higher-order
types $L$. Session types are denoted with $S$ and
shared types with $\chtype{C}$ and $\chtype{L}$.
%Note that we dissallow type $\chtype{U}$, thus
%in the type discipline shared names cannot carry shared names.
}
%In name types, $\chtype{U}$ is shared name types 
%which are sent via shared names. 
Types $\shot{C}$ and $\lhot{C}$ denote
{\em shared} and {\em linear} higher-order 
\jpc{functional}
types, respectively.
%,
%used to type abstraction values.
%$\lhot{C}$ \cite{tlca07,mostrous_phd} ensures values which contain free 
%session names used once. 
 
We write $S$ to denote binary session types.  {\em Output type}
$\btout{U} S$ %is assigned to a name that 
first sends a value of
type $U$ and then follows the protocol described by $S$.  Dually,
$\btinp{U} S$ denotes an {\em input type}.  {\em Branching type}
$\btbra{l_i:S_i}_{i \in I}$ and {\em selection type}
$\btsel{l_i:S_i}_{i \in I}$ are the labelled choice. 
We assume {\em recursive type} $\trec{t}{S}$ is guarded,
i.e.  $\trec{t}{\vart{t}}$ is not allowed. 
%We stress that carried type $U$ in $\btout{U} S$ and
%$\btinp{U} S$ can contain free type variables, which is crucial
%to encode $\HOp$ into $\HO$.
Type $\tinact$ is the termination type. 

Types of \HO exclude $\nonhosyntax{C}$ from 
value types of \HOp; the types of \sessp exclude $L$. 
From each $\CAL \in \{\HOp, \HO, \pi \}$, $\CAL^{-\mathsf{sh}}$ 
excludes shared name types ($\chtype{S}$ and $\chtype{L}$), 
from name type $C$.

We write $S_1 \dualof S_2$ if 
$S_1$ is dual of $S_2$ following \cite{TGC14}.  
Intuitively, 
the duality of types is defined by 
dualising $!$ by $?$, $?$ by $!$, $\oplus$ by $\&$ and $\&$ by $\oplus$,  
incorporating the fixed point construction 
(see \defref{def:dual} in Appendix). 

\smallskip 

%\subsection{Typing System of \HOp}
%\label{subsec:typing}
\myparagraph{Typing Judgements of \HOp.}
\noi We define typing judgements for values $V$
and processes $P$:

\begin{tabular}{c}
	$\Gamma; \Lambda; \Delta \proves V \hastype U \qquad \qquad \qquad \Gamma; \Lambda; \Delta \proves P \hastype \Proc$
\end{tabular}

\noi The first judgement
states that under environments $\Gamma; \Lambda; \Delta$ value $V$
has type $U$, whereas the second judgement states that under
environments $\Gamma; \Lambda; \Delta$ process $P$ has the process type~$\Proc$. The environments are defined below:
\[
\begin{array}{l}
 \Lambda \bnfis  \emptyset \bnfbar \Lambda \cat \AT{x}{\lhot{C}}
\quad\quad \Delta  \ \bnfis  \ \emptyset \bnfbar \Delta \cat \AT{u}{S} \\
 \Gamma  \bnfis  \emptyset \bnfbar \Gamma \cat \varp{x}: \shot{C} \bnfbar \Gamma \cat u: \chtype{S} \bnfbar \Gamma \cat u: \chtype{L} 
        \bnfbar \Gamma \cat \rvar{X}: \Delta
\end{array}
\]
\noi 
$\Gamma$ is a mapping from variables, shared names and recursive 
process variables;  $\Lambda$ is a mapping from variables to 
the linear functional type; and $\Delta$ is a mapping from 
session names to session types. 
%As expected, weakening, contraction, and exchange principles apply to
%$\Gamma$; environments $\Lambda$ and $\Delta$ behave linearly, and are
%only subject to exchange.  
We require that the domains of $\Gamma,
\Lambda$ and $\Delta$ are pairwise distinct. $\Delta_1\cdot \Delta_2$ means 
a disjoint union of $\Delta_1$ and $\Delta_2$.  
We are interested in session environments that contain
dual endpoints typed with dual types.
%The following definition ensures two session endpoints 
%are dual each other. 

\smallskip

\begin{definition}[Balanced]\label{d:wtenv}%\rm
	%Let $\Delta$ be a session environment.
	We say that a session environment $\Delta$ is {\em balanced} if whenever
	$s: S_1, \dual{s}: S_2 \in \Delta$ then $S_1 \dualof S_2$.
\end{definition}

\smallskip

\myparagraph{The Typing System of \HOp.}
Since the typing system is similar to~\cite{tlca07,mostrous_phd}, we give 
its full description in \appref{app:types}.  The rest of the paper can
be read without knowing the details of the typing system. 
\jpc{We introduce an auxiliary notion before stating type soundness}.
%We list the key properties.

\smallskip

\begin{definition}[Reduction of Session Environment]%\rm
\label{def:ses_red}
We define the relation $\red$ on session environments as:

\begin{tabular}{l}
	$\Delta \cat s: \btout{U} S_1 \cat \dual{s}: \btinp{U} S_2 \red
	\Delta \cat s: S_1 \cat \dual{s}: S_2$\\[1mm]
	$\Delta \cat s: \btsel{l_i: S_i}_{i \in I} \cat \dual{s}: \btbra{l_i: S_i'}_{i \in I} \red \Delta \cat s: S_k \cat \dual{s}: S_k' \ (k \in I)$
\end{tabular}
%\begin{tabular}{rcl}
%	\setlength{\tabcolsep}{0pt}
%	$\Delta \cat s: \btout{U} S_1 \cat \dual{s}: \btinp{U} S_2$ & $\red$ & 
%	$\Delta \cat s: S_1 \cat \dual{s}: S_2$\\[1mm]
%	$\Delta \cat s: \btsel{l_i: S_i}_{i \in I} \cat \dual{s}: \btbra{l_i: S_i'}_{i \in I}$ & $\red$ & $\Delta \cat s: S_k \cat \dual{s}: S_k' \ (k \in I)$
%\end{tabular}
%\[
%\begin{array}{rcl}
%\Delta \cat s: \btout{U} S_1 \cat \dual{s}: \btinp{U} S_2 & \red & 
%\Delta \cat s: S_1 \cat \dual{s}: S_2\\[1mm]
%\Delta \cat s: \btsel{l_i: S_i}_{i \in I} \cat \dual{s}: \btbra{l_i: S_i'}_{i \in I} & \red & \Delta \cat s: S_k \cat \dual{s}: S_k' \ (k \in I)
%\end{array}
%\]
\end{definition}

\smallskip

%The following result %Theorem 7.3 in M\&Y
\noi We state the type soundness results of \HOp (which implies 
the type soundness of the sub-calculi, \HO, \sessp and $\CAL^{-\mathsf{sh}}$). 

\smallskip

\begin{theorem}[Type Soundness]\label{t:sr}%\rm
%	\begin{enumerate}[1.]
%		\item	(Subject Congruence) Suppose $\Gamma; \es; \Delta \proves P \hastype \Proc$.
%			Then $P \scong P'$ implies $\Gamma; \es; \Delta \proves P' \hastype \Proc$.
%
%		\item
%			(Subject Reduction)
			Suppose $\Gamma; \es; \Delta \proves P \hastype \Proc$
			with
			$\Delta$ balanced. 
			Then $P \red P'$ implies $\Gamma; \es; \Delta'  \proves P' \hastype \Proc$
			and $\Delta = \Delta'$ or $\Delta \red \Delta'$
			with $\Delta'$ balanced. 
%	\end{enumerate}
\end{theorem}
