This section defines a type system for
$\HOp$ and establishes its main properties. 
Our system is simpler than that in~\cite{tlca07}, in order to distill the key
features of higher-order communication in a session-typed setting.

\subsection{Types}
\label{subsec:types}
The syntax of types of \HOp is given below: 
\[
\begin{array}{lrl}
\text{(value)}	& U \bnfis &	\nonhosyntax{C} \bnfbar L\\[1mm]  % \bnfbar \Proc
\text{(chan)}   & C  \bnfis &	S \bnfbar \chtype{S}\bnfbar \chtype{L}
	\\[1mm]
\text{(lambda)} & L \bnfis &	\shot{C} \bnfbar \lhot{C}
	\\[1mm]
\text{(session)} &  S \bnfis & 	\btout{U} S \bnfbar \btinp{U} S 
\bnfbar \btsel{l_i:S_i}_{i \in I} \\ 
 & \bnfbar & \btbra{l_i:S_i}_{i \in I}
	  \bnfbar  \trec{t}{S} \bnfbar \vart{t}  \bnfbar \tinact
\end{array}
\]
The types of \HO excludes $\nonhosyntax{C}$ from 
value types of \HOp; and the types of \sessp exludes $L$. 
In channel types, $\chtype{U}$ is shared channel types 
which are sent via shared channels. 
$\shot{C}$ and $\lhot{C}$ denote
{\em shared} and {\em linear} function types, respectively.
$\lhot{C}$ \cite{tlca07} ensures values which contain free 
session channels used once. 
 
We write $S$ to denote binary session types.  {\em Output type}
$\btout{U} S$ is assigned to a channel that first sends a value of
type $U$ and then follows the protocol described by $S$.  Dually,
$\btinp{U} S$ denotes an {\em input type}.  {\em Branching type}
$\btbra{l_i:S_i}_{i \in I}$ and {\em selection type}
$\btsel{l_i:S_i}_{i \in I}$ are standard.  We assume {\em recursive
  type} $\trec{t}{S}$ is guarded, i.e.  $\trec{t}{\vart{t}}$ is not
allowed.  Note that we allow carried type $U$ in $\btout{U} S$ and
$\btinp{U} S$ contains free recursive type variables, which is crucial
to encode $\HOp$ into $\HO$. Type $\tinact$ represents the
terminaition. We define $\dual{S}$ as a type given by duallising $!$
by $?$, $?$ by $!$, $\oplus$ by $\&$ and $\&$ by $\oplus$. 
Then we write $S_1 \dualof S_2$ if 
$S_1$ is dual of $S_2$ following \cite{TGC14} 
(Definition~\ref{def:dual} in Appendix).

\subsection{Typing System}
\label{subsec:typing}
\paragraph{Typing Judgements}
We first define the environments where  
\[
\begin{array}{l}
 \Lambda \bnfis  \emptyset \bnfbar \Lambda \cat \AT{x}{\lhot{C}}
\quad\quad \Delta  \ \bnfis  \ \emptyset \bnfbar \Delta \cat \AT{u}{S} \\
 \Gamma  \bnfis  \emptyset \bnfbar \Gamma \cat \varp{x}: \shot{C} \bnfbar \Gamma \cat u: \chtype{S} \bnfbar \Gamma \cat u: \chtype{L} 
        \bnfbar \Gamma \cat \rvar{X}: \Delta
\end{array}
\]
\noi We define typing judgements for values $V$
and processes $P$:
%
\[	\begin{array}{c}
		\Gamma; \Lambda; \Delta \proves V \hastype U \qquad \qquad \qquad \Gamma; \Lambda; \Delta \proves P \hastype \Proc
	\end{array}
\]
%
\noi 
$\Gamma$ is a mapping from variables, shared channels and recursive 
process variables;  $\Lambda$ 
As expected, weakening, contraction, and exchange principles apply to
$\Gamma$; environments $\Lambda$ and $\Delta$ behave linearly, and are
only subject to exchange.  We require that the domains of $\Gamma,
\Lambda$ and $\Delta$ are pairwise distinct. $\Delta_1\cdot \Delta_2$ means 
a disjoint union of $\Delta_1$ and $\Delta_2$.  The first judgement
states that under environment $\Gamma; \Lambda; \Delta$ values $V$
have type $U$, whereas the second judgement states that under
environment $\Gamma; \Lambda; \Delta$ process $P$ has the process type
$\Proc$.


\paragraph{Typing System}
The judgements 
\[	\begin{array}{c}
	\Gamma; \Lambda; \Delta \proves V \hastype U \qquad \qquad \qquad \Gamma; \Lambda; \Delta \proves P \hastype \Proc
	\end{array}
\]

\begin{figure}[t]
\[
	\begin{array}{lc}
	\trule{Sess}& \Gamma; \emptyset; \set{u:S} \proves u \hastype S 
		\quad
		\trule{Sh}~~\Gamma \cat u : U; \emptyset; \emptyset \proves u \hastype U
\\[4mm]
	\trule{LVar}& \Gamma; \set{x: \lhot{C}}; \emptyset \proves x \hastype \lhot{C}
		\\[4mm]

		\trule{Prom}& \tree{
			\Gamma; \emptyset; \emptyset \proves V \hastype 
                         \lhot{C}
		}{
			\Gamma; \emptyset; \emptyset \proves V \hastype 
                         \shot{C}
		} 
		\quad
		\trule{EProm}\tree{
		\Gamma; \Lambda \cat x : \lhot{C}; \Delta \proves P \hastype \Proc
		}{
			\Gamma \cat x:\shot{C}; \Lambda; \Delta \proves P \hastype \Proc
		}
		\\[6mm]

%		\trule{Pol}~~\tree{
%			I = \set{i \setbar k_i \in \tilde{k}, C_i \in \tilde{C}}
%			\qquad
%			\forall i \in I \quad \Gamma; \Lambda_i; \Delta_i \proves k_i \hastype C_i
%		}{
%			\Gamma; \bigcup_{i \in I} \Lambda_i; \bigcup_{i \in I} \Delta_i \proves \tilde{k} \hastype \tilde{C}
%		}
%		\\[6mm]

		\trule{Abs}& \tree{
			\Gamma; \Lambda; \Delta_1 \proves P \hastype \Proc
			\quad
			\Gamma; \es; \Delta_2 \proves x \hastype C
		}{
			\Gamma\backslash x; \Lambda; \Delta_1 \backslash \Delta_2 \proves \abs{{x}}{P} \hastype \lhot{{C}}
		}
		\\[6mm]

		\trule{App}& \tree{
			\begin{array}{c}
				U = \lhot{C} \lor \shot{C}
				\quad
				\Gamma; \Lambda; \Delta_1 \proves V \hastype U
				\quad
				\Gamma; \es; \Delta_2 \proves u \hastype C
			\end{array}
		}{
			\Gamma; \Lambda; \Delta_1 \cat \Delta_2 \proves \appl{V}{u} \hastype \Proc
		} 
		\\[6mm]

%		\trule{Send}~~\tree{
%			\Gamma; \Lambda_1; \Delta_1 \proves P \hastype \Proc  \quad \Gamma; \Lambda_2; \Delta_2 \vdash V \hastype U  \quad (k:S \in \Delta_1 \cup \Delta_2)
%		}{
%			\Gamma; \Lambda_1 \cat \Lambda_2; (\Delta_1 \cat \Delta_2)\backslash\set{k:S} \cat k:\btout{U} S \proves \bout{k}{V} P \hastype \Proc
%		}
%		\\[4mm]

		\trule{Send}& \tree{
			\Gamma; \Lambda_1; \Delta_1 \proves P \hastype \Proc
			\quad
			\Gamma; \Lambda_2; \Delta_2 \proves V \hastype U
			\quad
			u:S \in \Delta_1 \cat \Delta_2
		}{
			\Gamma; \Lambda_1 \cat \Lambda_2; ((\Delta_1 \cat \Delta_2) \setminus u:S) \cat u:\btout{U} S \proves \bout{u}{V} P \hastype \Proc
		}
		\\[6mm]

		\trule{Rcv}& \tree{
		\Gamma; \Lambda_1; \Delta_1 \cat u: S \proves P \hastype \Proc
			\quad
			\Gamma; \Lambda_2; \Delta_2 \proves {x} \hastype {U}
		}{
			\Gamma \backslash x; \Lambda_1\cat \Lambda_2; \Delta_1\backslash \Delta_2 \cat u: \btinp{U} S \vdash \binp{u}{{x}} P \hastype \Proc
		}
\\[6mm]
%		\trule{RvH}& \tree{
%			\Gamma; \Lambda_1; \Delta \cat u: S \proves P \hastype \Proc
%			\quad
%			\Gamma; \Lambda_2; \es \proves x \hastype L
%		}{
%			\Gamma \backslash x; \Lambda_1\backslash\Lambda_2; \Delta \cat u: \btinp{L} S \proves \binp{u}{x} P \hastype \Proc
%		}
%		\\[6mm]

%		\trule{RvS}& \tree{
%		\Gamma; \Lambda; \Delta_1 \cat u: S \proves P \hastype \Proc
%			\quad
%			\Gamma; \es; \Delta_2 \proves {x} \hastype {C}
%		}{
%			\Gamma \backslash x; \Lambda; \Delta_1\backslash \Delta_2 \cat u: \btinp{C} S \vdash \binp{u}{{x}} P \hastype \Proc
%		}
%\\[6mm]
%		\trule{RvH}& \tree{
%			\Gamma; \Lambda_1; \Delta \cat u: S \proves P \hastype \Proc
%			\quad
%			\Gamma; \Lambda_2; \es \proves x \hastype L
%		}{
%			\Gamma \backslash x; \Lambda_1\backslash\Lambda_2; \Delta \cat u: \btinp{L} S \proves \binp{u}{x} P \hastype \Proc
%		}
%		\\[6mm]

%		\trule{RcvS}~~\tree{
%			\Gamma; \Lambda; \Delta \cat k: S_1 \cat x: S_2 \proves P \hastype \Proc
%		}{
%			\Gamma; \Lambda; \Delta, k: \btinp{S_2} S_1  \vdash \binp{k}{x}P \hastype \Proc
%		}
%		\quad\quad 
%		\trule{RcvL}~~\tree{
%			\Gamma; \Lambda \cat X: \lhot{U}; \Delta \cat k: S  \proves P \hastype \Proc
%		}{
%			\Gamma; \Lambda; \Delta \cat k:\btinp{\lhot{U}}S  \proves \binp{k}{X}P \hastype \Proc
%		}
%		\\[4mm]
%		\trule{RcvShN}~~\tree{
%			\Gamma \cat x: \chtype{U}; \Lambda; \Delta \cat k: S_1  \proves P \hastype \Proc
%		}{
%			\Gamma; \Lambda; \Delta \cat k:\btinp{\chtype{U}}S_1  \proves \binp{k}{x}P \hastype \Proc
%		}		
%		\quad ~~
%		\trule{RcvSh}~~\tree{
%			\Gamma \cat X: \shot{U}; \Lambda; \Delta \cat k: S_1  \proves P \hastype \Proc
%		}{
%			\Gamma; \Lambda; \Delta \cat k:\btinp{\shot{U}}S_1  \proves \binp{k}{X}P \hastype \Proc
%		}
%		\\[4mm]

		\trule{Req}& \tree{
			\begin{array}{c}
				\Gamma; \es; \es \proves u \hastype U_1
				\quad
				\Gamma; \Lambda; \Delta_1 \proves P \hastype \Proc
				\quad
				\Gamma; \es; \Delta_2 \proves V \hastype U_2
				\\
				(U_1 = \chtype{S} 
                                \land %\Leftrightarrow 
                                U_2 = S)
				\lor
				 (U_1 = \chtype{L} 
                                \land %\Leftrightarrow 
                                %\Leftrightarrow 
                                 U_2 = L)
			\end{array}
		}{
			\Gamma; \Lambda; \Delta_1 \cat \Delta_2 \proves \bout{u}{V} P \hastype \Proc
		}
		\\[6mm]

%		\trule{ReqH}~~\tree{
%			\Gamma; \es; \es \proves k \hastype \chtype{U}
%			\quad
%			\Gamma; \Lambda_1; \Delta_1 \proves P \hastype \Proc
%			\quad
%			\Gamma; \Lambda_2; \Delta_2 \proves (x) Q \hastype U
%		}{
%			\Gamma; \Lambda_1 \cat \Lambda_1; \Delta_1 \cat \Delta_2 \proves \bout{k}{(x) Q} P \hastype \Proc
%		}
%		\\[6mm]

		\trule{Acc}& \tree{
			\begin{array}{c}
			\Gamma; \emptyset; \emptyset \proves u \hastype 
U_1 
		\quad
		\Gamma; \Lambda_1; \Delta_1 \proves P \hastype \Proc
		\quad
		\Gamma; \Lambda_2; \Delta_2 \proves x \hastype U_2\\
				(U_1 = \chtype{S} 
                                \land %\Leftrightarrow 
                                U_2 = S)
				\lor
				 (U_1 = \chtype{L} 
                                \land %\Leftrightarrow 
                                %\Leftrightarrow 
                                 U_2 = L)
               \end{array}
		}{
			\Gamma\backslash x; \Lambda_1 \backslash \Lambda_2; \Delta_1 \backslash \Delta_2 \proves \binp{u}{x} P \hastype \Proc
		}
		\\[6mm]

%		\trule{AcS}& \tree{
%			\Gamma; \emptyset; \emptyset \proves u \hastype \chtype{S}
%			\quad
%			\Gamma; \Lambda_1; \Delta_1 \proves P \hastype \Proc
%			\quad
%			\Gamma; \Lambda_2; \Delta_2 \proves x \hastype S
%		}{
%			\Gamma; \Lambda_1 \backslash \Lambda_2; \Delta_1 \backslash \Delta_2 \proves \binp{u}{x} P \hastype \Proc
%		}
%		\\[6mm]

%		\trule{AcH}& \tree{
%			\Gamma; \emptyset; \emptyset \proves u \hastype \chtype{L}
%			\quad
%			\Gamma; \Lambda_1; \Delta \proves P \hastype \Proc
%			\quad
%			\Gamma; \Lambda_2; \es \proves x \hastype L
%		}{
%			\Gamma \backslash x; \Lambda_1 \backslash \Lambda_2; \Delta \proves \binp{u}{x} P \hastype \Proc
%		}
%		\\[6mm]

		\trule{Bra}& \tree{
			 \forall i \in I \quad \Gamma; \Lambda; \Delta \cat u:S_i \proves P_i \hastype \Proc
		}{
			\Gamma; \Lambda; \Delta \cat u: \btbra{l_i:S_i}_{i \in I} \proves \bbra{u}{l_i:P_i}_{i \in I}\hastype \Proc
		}
\\[6mm]
	 	\trule{Sel}& \tree{
			\Gamma; \Lambda; \Delta \cat u: S_j  \proves P \hastype \Proc \quad j \in I

		}{
			\Gamma; \Lambda; \Delta \cat u:\btsel{l_i:S_i}_{i \in I} \proves \bsel{u}{l_j} P \hastype \Proc
		}
		\\[6mm]

%		\trule{Conn}~~\tree{
%			\Gamma; \Lambda; \Delta \cat x:S \proves P \hastype \Proc  \quad \Gamma; \emptyset; \emptyset \proves a \hastype \chtype{S}
%		}{
%			\Gamma; \Lambda; \Delta \proves \binp{a}{x} P \hastype \Proc
%		}
%		\quad
%		\trule{ConnL}~~\tree{
%			\Gamma \cat a : \chtype{\lhot{U}}; \Lambda \cat X: \lhot{U}; \Delta \proves P \hastype \Proc
%		}{
%			\Gamma \cat a : \chtype{\lhot{U}}; \Lambda; \Delta \proves \binp{a}{X} P \hastype \Proc
%		}
%		\\[4mm]
%
%		\trule{ConnSh}~~\tree{
%			\Gamma  \cat x:\chtype{U}; \Lambda; \Delta \proves P \hastype \Proc  \quad \Gamma; \emptyset; \emptyset \proves a \hastype \chtype{U}
%		}{
%			\Gamma; \Lambda; \Delta \proves \binp{a}{x} P \hastype \Proc
%		}
%		\quad
%		\trule{ConnS}~~\tree{
%			\Gamma \cat a : \chtype{\shot{U}} \cat X: \shot{U}; \Lambda; \Delta \proves P \hastype \Proc
%		}{
%			\Gamma \cat a : \chtype{\shot{U}} \cat X: \shot{U}; \Lambda; \Delta \proves \binp{a}{X} P \hastype \Proc
%		}
%		\\[4mm]

		\trule{Res}& \tree{
			\Gamma\cat a:\chtype{S} ; \Lambda; \Delta \proves P \hastype \Proc
		}{
			\Gamma; \Lambda; \Delta \proves \news{a} P \hastype \Proc}
\\[6mm]
		\trule{ResS}& \tree{
			\Gamma; \Lambda; \Delta \cat s:S_1 \cat \dual{s}: S_2 \proves P \hastype \Proc \quad S_1 \dualof S_2
		}{
			\Gamma; \Lambda; \Delta \proves \news{s} P \hastype \Proc
		}
		\\[6mm]

		\trule{Par}& \tree{
			\Gamma; \Lambda_{1}; \Delta_{1} \proves P_{1} \hastype \Proc \quad \Gamma; \Lambda_{2}; \Delta_{2} \proves P_{2} \hastype \Proc
		}{
			\Gamma; \Lambda_{1} \cat \Lambda_2; \Delta_{1} \cat \Delta_2 \proves P_1 \Par P_2 \hastype \Proc
		}
\\[6mm]
		\trule{Weak}& \tree{
			\Gamma; \Lambda; \Delta  \proves P \hastype T \quad u \not\in \dom{\Gamma, \Lambda,\Delta}
		}{
			\Gamma; \Lambda; \Delta \cat u: \tinact  \proves P \hastype \Proc
		}
		\\[6mm]

		\trule{Nil}& \Gamma; \emptyset; \emptyset \proves \inact \hastype \Proc
\quad 
		\trule{RVar}~~\Gamma \cat \rvar{X}: \Delta; \emptyset; \Delta  \proves \rvar{X} \hastype \Proc
\\[4mm]
%	 	\trule{Rec}& \tree{
%			\Gamma \cat \rvar{X}: \Delta; \emptyset; \emptyset  \proves P \hastype \Proc
%		}{
%			\Gamma ; \emptyset; \emptyset  \proves \recp{X}{P} \hastype \Proc
%		}
%		\\[4mm]

	 	\trule{Rec}& \tree{
			\Gamma \cat \rvar{X}: \Delta; \emptyset; \Delta  \proves P \hastype \Proc
		}{
			\Gamma ; \emptyset; \Delta  \proves \recp{X}{P} \hastype \Proc
		}


%		\\[4mm]
%		\trule{PSend}~~\tree{
%			\Gamma; \Lambda; \Delta \cat n: S \proves P \hastype \Proc \qquad \forall i \in I, \Gamma; \es; \Delta_i \proves m_i \hastype C_i
%		}{
%			\Gamma; \Lambda; ((\Delta\cat\tilde{\Delta_i})\backslash n:S) \cat n: \btout{\tilde{C_i}_{i \in I}} S\proves \bout{n}{\tilde{m_i}_{i \in I}} P \hastype \Proc
%		}
%		\\[4mm]
%
%		\trule{PRcv}~~\tree{
%			\Gamma; \Lambda; \Delta \cat n: S \proves P \hastype \Proc \qquad \forall i \in I, \Gamma_i; \es; \Delta_i \proves x: C_i 
%		}{
%			\Gamma\backslash\tilde{\Gamma_i}; \Lambda; \Delta\backslash\tilde{\Delta_i} \cat n: \btinp{\tilde{C_i}_{i \in I}} S \proves \binp{n}{\tilde{x_i}_{i \in I}} P \hastype \Proc
%		}
%		\\[4mm]
%
%		\trule{PAbs}~~\tree{
%			\Gamma; \Lambda; \Delta \proves P \hastype \Proc \quad \forall i \in I, \Gamma; \es; \Delta_i \proves x_i \hastype C_i
%		}{
%			\Gamma; \Lambda; \Delta\backslash\tilde{\Delta_i} \proves \abs{\tilde{x_i}_{i \in I}}{P} \hastype \lhot{\tilde{C_i}_{i \in I}}
%		}
%		\\[4mm]
%
%		\trule{App}~~\tree{(U = \lhot{\tilde{C_i}}) \lor (U = \shot{\tilde{C_i}}) \quad
%			\Gamma; \Lambda; \Delta \proves X \hastype U  \quad \forall i \in I, \Gamma; \es; \Delta_2 \proves k_i \hastype C_i
%		}{
%			\Gamma; \Lambda; \Delta \cat \tilde{\Delta_i} \proves \appl{X}{\tilde{k_i}} \hastype \Proc
%		} 
%		\\[4mm]
	\end{array}
\]
\caption{Typing Rules for $\HOp$.\label{fig:typerulesmy}}
\end{figure}






\begin{definition}[Well-typed Session Environment]\label{d:wtenv}%\rm
	Let $\Delta$ be a session environment.
	We say that $\Delta$ is {\em well-typed} if whenever
	$s: S_1, \dual{s}: S_2 \in \Delta$ then $S_1 \dualof S_2$.
\end{definition}

\begin{definition}[Session Environment Reduction]%\rm
\label{def:ses_red}
	We define the relation $\red$ on session environments as:
	\begin{enumerate}[$-$]
		\item	$\Delta \cat s: \btout{U} S_1 \cat \dual{s}: \btinp{U} S_2 \red \Delta \cat s: S_1 \cat \dual{s}: S_2$
		\item	$\Delta \cat s: \btsel{l_i: S_i}_{i \in I} \cat \dual{s}: \btbra{l_i: S_i'}_{i \in I} \red \Delta \cat s: S_k \cat \dual{s}: S_k', \quad k \in I$.
	\end{enumerate}
\end{definition}

%Theorem 7.3 in M\&Y
\begin{theorem}[Type Soundness]\label{t:sr}%\rm
	\begin{enumerate}[1.]
		\item	(Subject Congruence) Suppose $\Gamma; \es; \Delta \proves P \hastype \Proc$.
			Then $P \scong P'$ implies $\Gamma; \es; \Delta \proves P' \hastype \Proc$.

		\item	(Subject Reduction) Suppose $\Gamma; \es; \Delta \proves P \hastype \Proc$
			with
			well-typed $\Delta$. 
			Then $P \red P'$ implies $\Gamma; \es; \Delta'  \proves P' \hastype \Proc$
			and $\Delta = \Delta'$ or $\Delta \red \Delta'$
with $\Delta'$ well-formed. 
	\end{enumerate}
\end{theorem}
