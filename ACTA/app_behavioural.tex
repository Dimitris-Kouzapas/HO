% !TEX root = main.tex

%\section{Behavioural Semantics}

We present the proofs for 
\thmref{the:coincidence} (Page \pageref{the:coincidence}).
We require an auxiliary result on 
deterministic transitions (\lemref{lem:up_to_deterministic_transition}).
Some notions needed to prove this auxiliary result are presented next.
%Then we present the proof of \thmref{the:coincidence}, based on \emph{higher-order bisimilarity}.

%As mentioned in the paper, 
%the proof of \thmref{the:coincidence}
%relies on an auxiliary typed behavioral equivalence, \emph{higher-order bisimilarity}:

%\begin{definition}[Higher-Order Bisimulation]\myrm
%	\label{def:bisim}
%	Typed relation
%	$\Re$ is a {\em higher-order bisimulation} if for all
%	$\horel{\Gamma}{\Delta_1}{P_1}{\ \Re\ }{\Delta_2}{Q_1}$, % implies:
%%
%	\begin{enumerate}[1.]
%		\item	%$\forall \news{\widetilde{m_1}} \bactout{n}{V_1}$ such that
%		   Whenever 
%			$
%				\horel{\Gamma}{\Delta_1}{P_1}{\hby{\news{\widetilde{m_1}} \bactout{n}{V_1}}}{\Delta_1'}{P_2}
%			$
%			there exist $Q_2$, $V_2$, $\Delta_2'$ such that
%			\[
%				\horel{\Gamma}{\Delta_2}{Q_1}{\Hby{\news{\widetilde{m_2}} \bactout{n}{V_2}}}{\Delta_2'}{Q_2}
%			\]
%			and, for a fresh $t$, 
%			$
%				\horel{\Gamma}{\Delta_1''}{\newsp{\widetilde{m_1}}{P_2 \Par \htrigger{t}{V_1}}}
%				{\ \Re\ }
%				{\Delta_2''}{}{\newsp{\widetilde{m_2}}{Q_2 \Par \htrigger{t}{V_2}}}$.
%			
%%
%		\item	For all 
%			$
%				\horel{\Gamma}{\Delta_1}{P_1}{\hby{\ell}}{\Delta_1'}{P_2}
%			$
%			such that $\ell \not= \news{\widetilde{m}} \bactout{n}{V}$, there exist
%			 $\exists Q_2$ and $\Delta_2'$ such that 
%			\[
%				\horel{\Gamma}{\Delta_1}{Q_1}{\Hby{\hat{\ell}}}{\Delta_2'}{Q_2}
%			\]
%			and
%			$\horel{\Gamma}{\Delta_1'}{P_2}{\ \Re\ }{\Delta_2'}{Q_2}$.
%
%		\item	The symmetric cases of 1 and 2.
%	\end{enumerate}
%	The Knaster-Tarski theorem ensures that the largest higher-order bisimulation exists;
%	it is called \emph{higher-order bisimilarity} and is denoted by $\hwb$.
%\end{definition}
In this appendix, we use the polyadic abstractions and name passing 
for shorthand notations. 

%\smallskip


%the theorem in \secref{sec:behavioural}.


%%%%%%%%%%%%%%%%%%%%%%%%%%%%%%%%%%%%%%%%%%%%%%%%%%%%%%%%%%%%%%
% tau - Innertness
%%%%%%%%%%%%%%%%%%%%%%%%%%%%%%%%%%%%%%%%%%%%%%%%%%%%%%%%%%%%%%

\subsection{Deterministic Transitions}
\label{app:sub_tau_inert}

%We now define internal deterministic transitions as those associated to session synchronizations or to 
%$\beta$-reductions: 
		
%\begin{definition}[Deterministic Transition]\myrm
%\label{def:dettrans}
%	Let  $\Gamma; \es; \Delta \proves P \hastype \Proc$ be a balanced \HOp process. 
%	Transition $\horel{\Gamma}{\Delta}{P}{\hby{\tau}}{\Delta'}{P'}$ is called:
%	\begin{enumerate}[$-$]
%		\item {\em \sesstran} whenever the untyped transition $P \by{\tau} P'$ 
%		is derived using 
%			rule~$\ltsrule{Tau}$ 
%		(where $\subj{\ell_1}$ and $\subj{\ell_2}$ in the premise 
%		are dual endpoints), 
%		possibly followed by uses of  $\ltsrule{Alpha}$, $\ltsrule{Res}$, $\ltsrule{Rec}$, or $\ltsrule{Par${}_L$}/
%		\ltsrule{Par${}_R$}$.
		
%		\item	{\em \betatran}	whenever the untyped transition $P \by{\tau} P'$
%			is derived using rule $\ltsrule{App}$,
%			possibly followed by uses of  $\ltsrule{Alpha}$, $\ltsrule{Res}$, $\ltsrule{Rec}$, or $\ltsrule{Par${}_L$}/
%		\ltsrule{Par${}_R$}$.
%	\end{enumerate}
%%
%Below we write
%$\horel{\Gamma}{\Delta}{P}{\hby{\stau}}{\Delta'}{P'}$
%and 
%$\horel{\Gamma}{\Delta}{P}{\hby{\btau}}{\Delta'}{P'}$
%to denote session and $\beta$-transitions, resp. 
%We also denote $\Hby{\dtau}$ for one or more deterministic transitions. 
%Also, 
%	 $\horel{\Gamma}{\Delta}{P}{\hby{\dtau}}{\Delta'}{P'}$ denotes
%	either a \sesstran or a \betatran.
%\end{definition}

%Deterministic transitions imply the $\tau$-inertness property, which
%is a property that ensures behavioural invariance on deterministic
%transitions.


\begin{proposition}[$\tau$-inertness]
	\label{app:lem:tau_inert}
	Let  $\Gamma; \es; \Delta \proves P \hastype \Proc$ be a balanced \HOp process.
	Then
	\begin{enumerate}[1.]
		\item	$\horel{\Gamma}{\Delta}{P}{\hby{\dtau}}{\Delta'}{P'}$ implies
			$\horel{\Gamma}{\Delta}{P}{\hwb}{\Delta'}{P'}$.
		\item	$\horel{\Gamma}{\Delta}{P}{\Hby{\dtau}}{\Delta'}{P'}$ implies
			$\horel{\Gamma}{\Delta}{P}{\hwb}{\Delta'}{P'}$.
	\end{enumerate}
\end{proposition}

%\begin{proposition}[$\tau$-inertness]\rm
%	Let balanced \HOp process $\Gamma; \es; \Delta \proves P \hastype \Proc$.
%	$\horel{\Gamma}{\Delta}{P}{\hby{\dtau}}{\Delta'}{P'}$ implies
%	$\horel{\Gamma}{\Delta}{P}{\wb}{\Delta'}{P'}$.
%\end{proposition}

%\jp{This proof seems to be by induction on deterministic transition; but then the analysis is on the structure of processes,
%which is confusing. In general: I would have done this proof by coinduction, constructing a closure containing $(P, P')$.}

\begin{proof}
	\noi 
	We prove Part 1 --- the proof for Part 2 follows straightforwardly.
	The proof is by induction on the structure of $\by{\tau}$
	which coincides the reduction $\red$.

	\noi Basic step:
	\begin{enumerate}
		\item %Case: $P = \appl{(\abs{x}{P})}{n}$:
	%
		\[
			\horel{\Gamma}{\Delta}{\appl{(\abs{x}{P})}{n}}{\hby{\btau}}{\Delta'}{P \subst{n}{x}}
		\]
	%
		\noi Bisimulation requirements hold because there is no other transition to observe than ${\hby{\btau}}$.

		\item %Case: $P = \bout{s}{V} P_1 \Par \binp{\dual{s}}{x} P_2$:
	%
		\[
			\horel{\Gamma}{\Delta}{\bout{s}{V} P_1 \Par \binp{\dual{s}}{x} P_2}{\hby{\stau}}{\Delta'}{P_1 \Par P_2}
		\]
	%
		\noi The proof follows from the fact that we can only observe a $\tau$
		action on typed process
		$\Gamma; \emptyset; \Delta \proves P \hastype \Proc$.
		Actions $\bactout{s}{V}$ and $\bactinp{\dual{s}}{V}$
		are forbidden by the LTS for typed environments;
		\dk{this is because
		$s: \btout{U} S_1 \cat \dual{s}: \btinp{U} S_2 \in \Delta$ and
		rule \eltsrule{SSnd} (resp., \eltsrule{SRv}) cannot be applied
		in order to observe action $(\Gamma; \es; \Delta) \by{\bactout{s}{V}} (\Gamma; \es; \Delta')$
		(resp., action $(\Gamma; \es; \Delta) \by{\bactinp{\dual{s}}{V}} (\Gamma; \es; \Delta'')$)
		because of the requirement $\dual{s} \notin \dom{\Delta}$ (resp., $s \notin \dom{\Delta}$).
		}

		\noi It is easy to conclude then that $\horel{\Gamma}{\Delta}{P}{\hwb}{\Delta'}{P'}$.

		\item %Case:
			\[
				\horel{\Gamma}{\Delta}{\bsel{s}{l_k} P \Par \bbra{\dual{s}}{l_i: P_i}_{i \in I}}{\hby{\stau}}{\Delta'}{P \Par P_k}
			\]

		\noi Similar arguments as the previous case.
	\end{enumerate}
	
	\noi Induction hypothesis:

	\noi If $P_1 \red P_2$ then $\horel{\Gamma_1}{\Delta_1}{P_1}{\hwb}{\Delta_2}{P_2}$.
	\noi Induction Step:
	\begin{enumerate}
		\item %Case: $P = \news{s} P_1$
	%
		\[
			\horel{\Gamma}{\Delta}{\news{s}{P_1}}{\hby{\stau}}{\Delta'}{\news{s} P_2}
		\]
	%
		\noi From the induction hypothesis and the fact that bisimulation is a congruence
		we get that $\horel{\Gamma}{\Delta}{P}{\hwb}{\Delta'}{P'}$.

		\item  %Case: $P = P_1 \Par P_3$
	%
		\[
			\horel{\Gamma}{\Delta}{P_1 \Par P_3}{\hby{\stau}}{\Delta'}{P_2 \Par P_3}
		\]
	%
		\noi From the induction hypothesis and the fact that bisimulation is a congruence
		we get that $\horel{\Gamma}{\Delta}{P}{\hwb}{\Delta'}{P'}$.

		\item   %Case:
			\[
				P \scong P_1 \text{ and }\horel{\Gamma}{\Delta}{P_1}{\hby{\stau}}{\Delta'}{P'}
			\]
%
		From the induction hypothesis and the fact that bisimulation is a congruence \dk{(\thmref{the:coincidence})}
		and structural congruence preserves $\hwb$
		we get that $\horel{\Gamma}{\Delta}{P}{\hwb}{\Delta'}{P'}$.
	\end{enumerate}
%	The proof for part two is an induction on the length of $\red^*$.
%	The basic step is trivial and the inductive step
%	deploys part 1 of this lemma and the fact that bisimulation is
%	transitive to conclude.
%	We can now conclude that
%	$P \wbc P'$ because $P \wbc P''$ and $P'' \wbc P'$.
	\qed
\end{proof}


%\begin{lemma}[Up-to Deterministic Transition]\myrm
%	\label{lem:up_to_deterministic_transition}
%	Let $\horel{\Gamma}{\Delta_1}{P_1}{\ \Re\ }{\Delta_2}{Q_1}$ such
%	that if whenever:
%%
%	\begin{enumerate}
%		\item	$\forall \news{\widetilde{m_1}} \bactout{n}{V_1}$ such that
%			$
%				\horel{\Gamma}{\Delta_1}{P_1}{\hby{\news{\widetilde{m_1}} \bactout{n}{V_1}}}{\Delta_3}{P_3}
%			$
%			implies that $\exists Q_2, V_2$ such that
%			\[
%				\horel{\Gamma}{\Delta_2}{Q_1}{\Hby{\news{\widetilde{m_2}} \bactout{n}{V_2}}}{\Delta_2'}{Q_2}
%			\]
%			and
%			\[
%				\horel{\Gamma}{\Delta_3}{P_3}{\Hby{\dtau}}{\Delta_1'}{P_2}
%			\]
%			and for fresh $t$:
%			\[
%				\horel{\Gamma}{\Delta_1''}{\newsp{\widetilde{m_1}}{P_2 \Par \htrigger{t}{V_1}}}
%				{\ \Re\ }
%				{\Delta_2''}{}{\newsp{\widetilde{m_2}}{Q_2 \Par \htrigger{t}{V_2}}}
%%				\mhorel{\Gamma}{\Delta_1''}{\newsp{\widetilde{m_1}}{P_2 \Par \hotrigger{t}{x}{s}{V_1}}}
%%				{\ \Re\ }
%%				{\Delta_2''}{}{\newsp{\widetilde{m_2}}{Q_2 \Par \hotrigger{t}{x}{s}{V_2}}}
%			\]
%%
%		\item	$\forall \ell \not= \news{\widetilde{m}} \bactout{n}{V}$ such that
%			$
%				\horel{\Gamma}{\Delta_1}{P_1}{\hby{\ell}}{\Delta_3}{P_3}
%			$
%			implies that $\exists Q_2$ such that 
%			\[
%				\horel{\Gamma}{\Delta_1}{Q_1}{\hat{\Hby{\ell}}}{\Delta_2'}{Q_2}
%			\]
%			and
%			\[
%				\horel{\Gamma}{\Delta_3}{P_3}{\Hby{\dtau}}{\Delta_1'}{P_2}
%			\]
%			and
%			$\horel{\Gamma}{\Delta_1'}{P_2}{\ \Re\ }{\Delta_2'}{Q_2}$
%
%		\item	The symmetric cases of 1 and 2.
%	\end{enumerate}
%	Then $\Re\ \subseteq\ \wb$.
%\end{lemma}
%
%
%\begin{proof}
%	The proof is easy by considering the closure
%	\[
%		\Re^{\Hby{\dtau}} = \set{ \horel{\Gamma}{\Delta_1'}{P_2}{,}{\Delta_2'}{Q_1} \setbar \horel{\Gamma}{\Delta_1}{P_1}{\ \Re\ }{\Delta_2'}{Q_1},
%		\horel{\Gamma}{\Delta_1}{P_1}{\Hby{\dtau}}{\Delta_1'}{P_2} }
%	\]
%	We verify that $\Re^{\Hby{\dtau}}$ is a bisimulation with
%	the use of \propref{app:lem:tau_inert}.
%	\qed
%\end{proof}

\subsection{Proof of \thmref{the:coincidence}}
\label{app:sub_coinc}


\noi We split the proof of \thmref{the:coincidence} (Page \pageref{the:coincidence}) into 
several lemmas:
\begin{enumerate}[$-$]
\item	\lemref{app:lem:wb_eq_wbf} establishes $\hwb\ =\ \fwb$.
\item	\lemref{app:lem:wb_is_wbc} exploits the process substitution result
		(\lemref{app:lem:proc_subst}) to prove that $\hwb \subseteq \wbc$.
\item	\lemref{app:lem:wbc_is_cong} shows that $\wbc$ is a congruence
		which implies $\wbc \subseteq \cong$.
\item	\lemref{app:lem:cong_is_wb} shows  that $\cong \subseteq \hwb$.
\end{enumerate}

%By the combination of the lemmas, we can obtain the theorem.

\noi
We now proceed to state and prove these lemmas, together with some auxiliary results.
%\thmref{app:thm:coincidence} (Page~\pageref{app:thm:coincidence}) summarises the coincidence result.

%\begin{lemma}
%	\label{app:lem_fresh_name}
%	if $\Gamma; \es; \Delta_1 \proves P_1 \Par R \hwb \Delta_2 \proves P_2 \Par R$, $t$
%	fresh and $x \notin \fn{R}$ then
%	$\Gamma; \es; \Delta_1' \proves P_1 \Par \binp{t}{x} R \hwb \Delta_2' \proves P_2 \Par \binp{t}{x} R$.
%\end{lemma}
%
%\begin{proof}
%	The proof distinguishes two cases.
%	\begin{itemize}
%		\item	$\Gamma; \es; \Delta_1 \proves P_1 \Par \binp{t}{x} R \hby{\ell_1} \Delta_1'' \proves P_1' \Par \binp{t}{x} R$.
%
%				Then by labelled transition rule \dk{put smth}
%				$\Gamma; \es; \Delta_3 \proves P_1 \hby{\ell_1} \Delta_3' \proves P_1'$
%				which implies
%				$\Gamma; \es; \Delta_1 \proves P_1 \Par R \hby{\ell_1} \Delta_1''' \proves P_1' \Par R$
%				which implies
%				$\Gamma; \es; \Delta_2 \proves P_2 \Par R \Hby{\ell_1} \Delta_1''' \proves P_2' \Par R'$
%	\end{itemize}
%	\dk{proofs}
%	\qed
%\end{proof}


%%%%%%%%%%%%%%%%%%%%%%%%%%%%%%%%%%%%%%%%%%%%%%%%%%%%%%%%%
%  HWB = FWB
%%%%%%%%%%%%%%%%%%%%%%%%%%%%%%%%%%%%%%%%%%%%%%%%%%%%%%%%%


\begin{lemma}
	\label{app:lem:wb_eq_wbf}
	$\hwb = \fwb$.
\end{lemma}

%\jp{The level of clarity/formality in the current proof does not convince me. I wrote an alternative version.
%
%\begin{proof}[Alternative]
%We only prove the direction $\hwb \subseteq \fwb$; 
%the	direction $\fwb \subseteq \hwb$ is similar.
%We show that the typed relation:
%	\[
%		\Re = \set{(P,Q) \setbar \horel{\Gamma}{\Delta_1}{P}{\hwb}{\Delta_2}{Q}}
%	\]
%\end{proof}
%is a characteristic bisimulation. 
%Observe that  $\hwb$ and $\fwb$ differ 
%only in the output clause (cf. \defref{d:hwb} and \defref{d:fwb}).
%We therefore focus our analysis in output actions. 
%
%Suppose $(P, Q) \in \Re$ and 
%$\horel{\Gamma}{\Delta_1}{P}{\hby{\news{\widetilde{m_1}} \bactout{n}{V_1}}}{\Delta_1'}{P'}$.
%Then, by definition of $\hwb$, there exist $Q, V_2$ such that 
%$\horel{\Gamma}{\Delta_2}{Q}{\Hby{\news{\widetilde{m_2}} \bactout{n}{V_2}}}{\Delta_2'}{Q'}$
%and, for a fresh $t$:
%\begin{equation*}
%		\horel{\Gamma}{\Delta_1'}{\newsp{\widetilde{m_1}}{P' \Par \htrigger{t}{V_1}}}
%		{~\Re~}
%		{\Delta_2}{}{\newsp{\widetilde{m_2}}{Q' \Par \htrigger{t}{V_2} }}
%\end{equation*}
%which, by expanding the definition of higher-order trigger,  can be equivalently written as
%\begin{equation}
%		\horel{\Gamma}{\Delta_1'}{\newsp{\widetilde{m_1}}{P' \Par \hotrigger{t}{x}{s}{V_1}}}
%		{~\Re~}
%		{\Delta_2}{}{\newsp{\widetilde{m_2}}{Q' \Par \hotrigger{t}{x}{s}{V_2} }}
%		\label{eq:altc1}
%\end{equation}
%We show that \eqref{eq:altc1} implies the existence of a pair in $\Re$ that corresponds to the shape 
%induced by the output clause of characteristic bisimilarity. 
%Suppose the following input action from 
%$\newsp{\widetilde{m_1}}{P' \Par \hotrigger{t}{x}{s}{V_1}}$:
%
%	\[
%		\mhorel{\Gamma}{\Delta_1'}{\newsp{\widetilde{m_1}}{P' \Par \hotrigger{t}{x}{s}{V_1}}}
%		{\hby{\bactinp{t}{\map{\btinp{U} \tinact}^{x}}}}
%		{\Delta_1''}{}{\newsp{\widetilde{m_1}}{P' \Par \newsp{s}{\map{\btinp{U} \tinact}^{s} \Par \bout{\dual{s}}{V_1} \inact}}}
%	\]
%
%By definition of $\hwb$, we know that process 
%$\newsp{\widetilde{m_2}}{Q' \Par \hotrigger{t}{x}{s}{V_2} }$ can match this action:
%	\[
%		\mhorel{\Gamma}{\Delta_2'}{\newsp{\widetilde{m_2}}{Q' \Par \hotrigger{t}{x}{s}{V_2}}}
%		{\Hby{\bactinp{t}{\map{\btinp{U} \tinact}^{x}}}}
%		{\Delta_2''}{}{\newsp{\widetilde{m_2}}{Q' \Par \newsp{s}{\map{\btinp{U} \tinact}^{s} \Par \bout{\dual{s}}{V_2} \inact}}}
%	\]
%with 
%	\[
%		\mhorel{\Gamma}{\Delta_1''}{\newsp{\widetilde{m_1}}{P' \Par \newsp{s}{\map{\btinp{U} \tinact}^{s} \Par \bout{\dual{s}}{V_1} \inact}}}
%		{~\Re~}
%		{\Delta_2''}{}{\newsp{\widetilde{m_2}}{Q'' \Par \newsp{s}{\map{\btinp{U} \tinact}^{s} \Par \bout{\dual{s}}{V_2} \inact}}}
%	\]
%which is precisely the shape required by $\fwb$ after an output action---cf. \defref{d:fwb} and \eqref{eqb:4} in Page \pageref{eqb:4}.
%Therefore, $\Re$ is a characteristic bisimulation and we are done.
%\qed
%}

\begin{proof}
	\noi
	We split the proof into the direction
	$\hwb \subseteq \fwb$ and the direction
	$\fwb \subseteq \hwb$.

	\begin{enumerate}
		\item	Direction $\hwb \subseteq \fwb$.

				\noi Consider the typed relation (for readability, we omit type information):
		%
				\[
					\Re = \set{
								%\horel{\Gamma}{\Delta_1}{P}{\ ,\ }{\Delta_2}{Q} 
								(P, Q) 
								\setbar
								\horel{\Gamma}{\Delta_1}{P}{\hwb}{\Delta_2}{Q}}
				\]
		%
				We show that $\Re$ is a characteristic bisimulation.
				Suppose
				$
						\horel{\Gamma}{\Delta_1}{P}{\hby{ \ell}}{\Delta_1'}{P'}
		%				\label{lem:wb_eq_wbf1}
				$.
				The proof does a case analysis on the transition label $\ell$.
				\begin{itemize}
					\item	Case $\ell = \news{\widetilde{m_1}} \bactout{n}{V_1}$ is the non-trivial case.

							\noi From the definition of $\Re$ we have that if:
						%
							\begin{eqnarray}
								\horel{\Gamma}{\Delta_1}{P}{\hby{\news{\widetilde{m_1}} \bactout{n}{V_1}}}{\Delta_1''}{P'}
								\label{lem:wb_eq_wbf1}
							\end{eqnarray}
							then $\exists Q, V_2$ such that:
						%
							\begin{eqnarray}
								\horel{\Gamma}{\Delta_2}{Q}{\Hby{\news{\widetilde{m_2}} \bactout{n}{V_2}}}{\Delta_2''}{Q'}
								\label{lem:wb_eq_wbf2}
							\end{eqnarray}
						%
							and for fresh $t$:
								\nhorel{\Gamma}{\Delta_1'}{\newsp{\widetilde{m_1}}{P' \Par \hotrigger{t}{x}{s}{V_1}}}
								{\hwb}
								{\Delta_2'}{\newsp{\widetilde{m_2}}{Q' \Par \hotrigger{t}{x}{s}{V_2} }}
								{lem:wb_eq_wbf3}
						%
							\noi 
							To show that $\Re$ is a characteristic bisimulation
							after the fact that transition~\eqref{lem:wb_eq_wbf1} implies transition~\eqref{lem:wb_eq_wbf2},
							we need to show that for fresh $t$:
						%
							\begin{eqnarray}
								\mhorel{\Gamma}{\Delta_3}{\newsp{\widetilde{m_1}}{P' \Par \fotrigger{t}{x}{s}{\btinp{U} \tinact}{V_1}}}
								{\Re}
								{\Delta_4}{}{\newsp{\widetilde{m_2}}{Q' \Par \fotrigger{t}{x}{s}{\btinp{U} \tinact}{V_2}}}
								\label{lem:wb_eq_wbf4}
							\end{eqnarray}
						%
							\noi
							From the higher-order bisimulation pair in \eqref{lem:wb_eq_wbf3} we can
							conclude that if:
							\[
								\mhorel{\Gamma}{\Delta_1'}{\newsp{\widetilde{m_1}}{P' \Par \hotrigger{t}{x}{s}{V_1}}}
								{\hby{\bactinp{t}{\map{\btinp{U} \tinact}^{x}}}}
								{\Delta_3'}{}{\newsp{\widetilde{m_1}}{P' \Par \newsp{s}{\map{\btinp{U} \tinact}^{s} \Par \bout{\dual{s}}{V_1} \inact}}}
							\]
						%
							then there exists $Q''$ such that:
							\begin{eqnarray}
								\begin{array}{crll}
									& \Gamma; \es; \Delta_2' & \proves &		
									\newsp{\widetilde{m_2}}{Q' \Par \hotrigger{t}{x}{s}{V_2}}
									\\
									\Hby{} &&&
									\newsp{\widetilde{m_2}}{Q_2 \Par \fotrigger{t}{x}{s}{V_2}}
									\\
									{\hby{\bactinp{t}{\map{\btinp{U} \tinact}^{x}}}}& & &
									\newsp{\widetilde{m_2}}{Q_2 \Par \newsp{s}{\map{\btinp{U} \tinact}^{s} \Par \bout{\dual{s}}{V_2} \inact}}
									\\
									\Hby{} & \Delta_4' & \proves & Q''
								\end{array}
								\label{lem:wb_eq_wbf_5}
							\end{eqnarray}
						%
							and
								\[
									\horel{\Gamma}
									{\Delta_3'}{\newsp{\widetilde{m_1}}{P' \Par \newsp{s}{\map{\btinp{U} \tinact}^{s} \Par \bout{\dual{s}}{V_1} \inact}}}
									{\hwb}
									{\Delta_4'}{Q''}
								\]
							which implies from the definition of $\Re$:
							\begin{eqnarray}
								\horel{\Gamma}
								{\Delta_3'}{\newsp{\widetilde{m_1}}{P' \Par \newsp{s}{\map{\btinp{U} \tinact}^{s} \Par \bout{\dual{s}}{V_1} \inact}}}
								{\ \Re\ }
								{\Delta_4'}{Q''}
								\label{lem:wb_eq_wbf6}
							\end{eqnarray}
						%
							From the pair in \eqref{lem:wb_eq_wbf4}, we get:
							\[
								\mhorel{\Gamma}{\Delta_3}{\newsp{\widetilde{m_1}}{P' \Par \fotrigger{t}{x}{s}{\btinp{U} \tinact}{V_1}}}
								{\hby{\bactinp{t}{m'}}}
								{\Delta_3'}{}{\newsp{\widetilde{m_1}}{P' \Par \newsp{s}{\map{\btinp{U} \tinact}^{s} \Par \bout{\dual{s}}{V_1} \inact}}}
							\]
						%
							The freshness of $t$ allows us to mimic the transitions
							in \eqref{lem:wb_eq_wbf_5} to get:
							\[
								\begin{array}{crll}
									& \Gamma; \es; \Delta_4 & \proves &		
									\newsp{\widetilde{m_2}}{Q' \Par \fotrigger{t}{x}{s}{\btinp{U} \tinact}{V_2}}
									\\
									\Hby{} &&&
									\newsp{\widetilde{m_2}}{Q_2 \Par \fotrigger{t}{x}{s}{\btinp{U} \tinact}{V_2}}
									\\
									\hby{\bactinp{t}{m'}}& & &
									\newsp{\widetilde{m_2}}{Q_2 \Par \newsp{s}{\map{\btinp{U} \tinact}^{s} \Par \bout{\dual{s}}{V_2} \inact}}
									\\
									\Hby{} & \Delta_4' & \proves & Q''
								\end{array}
							\]
						%
						This last implication together with the result in \eqref{lem:wb_eq_wbf6}
						concludes the proof of the case.

					\item	The other cases are trivial.
				\end{itemize}

		\item	Direction $\fwb \subseteq \hwb$.
				\noi Consider the typed relation (for readability, we omit type information):
		%
				\[
					\Re = \set{
								%\horel{\Gamma}{\Delta_1}{P}{\ ,\ }{\Delta_2}{Q} 
								(P, Q) 
								\setbar
								\horel{\Gamma}{\Delta_1}{P}{\fwb}{\Delta_2}{Q}}
				\]
		%
				We show that $\Re$ is a higher-order bisimulation.
				Suppose
				$
						\horel{\Gamma}{\Delta_1}{P}{\hby{ \ell}}{\Delta_1'}{P'}
		%				\label{lem:wb_eq_wbf1}
				$.
				The proof does a case analysis on the transition label $\ell$.
				\begin{itemize}
					\item	Case $\ell = \news{\widetilde{m_1}} \bactout{n}{V_1}$ is the non-trivial case.

							\noi From the definition of $\Re$ we have that if:
						%
							\begin{eqnarray}
								\horel{\Gamma}{\Delta_1}{P}{\hby{\news{\widetilde{m_1}} \bactout{n}{V_1}}}{\Delta_1''}{P'}
								\label{lem:fwb_sub_hbf1}
							\end{eqnarray}
							then $\exists Q, V_2$ such that:
						%
							\begin{eqnarray}
								\horel{\Gamma}{\Delta_2}{Q}{\Hby{\news{\widetilde{m_2}} \bactout{n}{V_2}}}{\Delta_2''}{Q'}
								\label{lem:fwb_sub_hbf2}
							\end{eqnarray}
						%
							and for fresh $t$:
								\nhorel{\Gamma}{\Delta_1'}{\newsp{\widetilde{m_1}}{P' \Par \fotrigger{t}{x}{s}{\btinp{U} \tinact}{V_1}}}
								{\hwb}
								{\Delta_2'}{\newsp{\widetilde{m_2}}{Q' \Par \fotrigger{t}{x}{s}{\btinp{U} \tinact}{V_2} }}
								{lem:fwb_sub_hbf3}
						%
							\noi 
							To show that $\Re$ is a higher-order bisimulation
							after the fact that transition~\eqref{lem:fwb_sub_hbf1} implies transition~\eqref{lem:fwb_sub_hbf2},
							we need to show that for fresh $t$:
						%
							\begin{eqnarray}
								\mhorel{\Gamma}{\Delta_3}{\newsp{\widetilde{m_1}}{P' \Par \hotrigger{t}{x}{s}{V_1}}}
								{\Re}
								{\Delta_4}{}{\newsp{\widetilde{m_2}}{Q' \Par \hotrigger{t}{x}{s}{V_2}}}
								\label{lem:fwb_sub_hbf4}
							\end{eqnarray}
						%
							\noi
							From the characteristic bisimulation pair in \eqref{lem:fwb_sub_hbf3}
							and fresh $m'$ we can conclude that:
							\[
								\mhorel{\Gamma}{\Delta_1'}{\newsp{\widetilde{m_1}}{P' \Par \fotrigger{t}{x}{s}{\btinp{U} \tinact}{V_1}}}
%								{\hby{\bactinp{t}{\map{\btinp{U} \tinact}^{x}}}}
								{\hby{\bactinp{t}{m'}}}
								{\Delta_3'}{}{\newsp{\widetilde{m_1}}{P' \Par \newsp{s}{\map{\btinp{U} \tinact}^{s} \Par \bout{\dual{s}}{V_1} \inact}}}
							\]
						%
							then there exists $Q''$ such that:
							\begin{eqnarray}
								\begin{array}{crll}
									& \Gamma; \es; \Delta_2' & \proves &		
									\newsp{\widetilde{m_2}}{Q' \Par \fotrigger{t}{x}{s}{\btinp{U} \tinact}{V_2}}
									\\
									\Hby{} &&&
									\newsp{\widetilde{m_2}}{Q_2 \Par \fotrigger{t}{x}{s}{\btinp{U} \tinact}{V_2}}
									\\
%									{\hby{\bactinp{t}{\map{\btinp{U} \tinact}^{x}}}}
									{\hby{\bactinp{t}{m'}}}& & &
									\newsp{\widetilde{m_2}}{Q_2 \Par \newsp{s}{\map{\btinp{U} \tinact}^{s} \Par \bout{\dual{s}}{V_2} \inact}}
									\\
									\Hby{} & \Delta_4' & \proves & Q''
								\end{array}
								\label{lem:fwb_sub_hbf5}
							\end{eqnarray}
						%
							and
								\[
									\horel{\Gamma}
									{\Delta_3'}{\newsp{\widetilde{m_1}}{P' \Par \newsp{s}{\map{\btinp{U} \tinact}^{s} \Par \bout{\dual{s}}{V_1} \inact}}}
									{\fwb}
									{\Delta_4'}{Q''}
								\]
							which implies from the definition of $\Re$:
							\begin{eqnarray}
								\horel{\Gamma}
								{\Delta_3'}{\newsp{\widetilde{m_1}}{P' \Par \newsp{s}{\map{\btinp{U} \tinact}^{s} \Par \bout{\dual{s}}{V_1} \inact}}}
								{\ \Re\ }
								{\Delta_4'}{Q''}
								\label{lem:fwb_sub_hbf6}
							\end{eqnarray}
						%
							From the pair in \eqref{lem:fwb_sub_hbf4}, we get:
							\[
								\mhorel{\Gamma}{\Delta_3}{\newsp{\widetilde{m_1}}{P' \Par \hotrigger{t}{x}{s}{V_1}}}
								{\hby{\bactinp{t}{\map{\btinp{U} \tinact}^{x}}}}
								{\Delta_3'}{}{\newsp{\widetilde{m_1}}{P' \Par \newsp{s}{\map{\btinp{U} \tinact}^{s} \Par \bout{\dual{s}}{V_1} \inact}}}
							\]
						%
							The freshness of $t$ allows us to mimic the transitions
							in \eqref{lem:fwb_sub_hbf5} to get:
							\[
								\begin{array}{crll}
									& \Gamma; \es; \Delta_4 & \proves &		
									\newsp{\widetilde{m_2}}{Q' \Par \hotrigger{t}{x}{s}{V_2}}
									\\
									\Hby{} &&&
									\newsp{\widetilde{m_2}}{Q_2 \Par \hotrigger{t}{x}{s}{V_2}}
									\\
									{\hby{\bactinp{t}{\map{\btinp{U} \tinact}^{x}}}}& & &
									\newsp{\widetilde{m_2}}{Q_2 \Par \newsp{s}{\map{\btinp{U} \tinact}^{s} \Par \bout{\dual{s}}{V_2} \inact}}
									\\
									\Hby{} & \Delta_4' & \proves & Q''
								\end{array}
							\]
						%
						This last implication together with the result in \eqref{lem:fwb_sub_hbf6}
						concludes the proof of the case.

					\item	The other cases are trivial.
				\end{itemize}



	\end{enumerate}
	\qed
\end{proof}


\begin{comment}
\begin{proof}
	\noi We only prove the direction $\hwb \subseteq \fwb$. The
	direction $\fwb \subseteq \hwb$ is similar.
Consider the typed relation (for readability, we omit type information):
%
	\[
		\Re = \set{%\horel{\Gamma}{\Delta_1}{P}{\ ,\ }{\Delta_2}{Q} 
		(P, Q) 
		\setbar \horel{\Gamma}{\Delta_1}{P}{\hwb}{\Delta_2}{Q}}
	\]
%
	We show that $\Re$ is a characteristic bisimulation.
	Suppose $\horel{\Gamma}{\Delta_1}{P}{\hby{ \ell}}{\Delta_1'}{P'}
					\label{lem:wb_eq_wbf1}
			$.
				The proof does a case analysis on the transition label $\ell$.

	\begin{enumerate}
		\item	$\ell = \news{\widetilde{m_1}} \bactout{n}{V_1}$ is the non-trivial case. If
			%
				\begin{eqnarray}
					\horel{\Gamma}{\Delta_1}{P}{\hby{\news{\widetilde{m_1}} \bactout{n}{V_1}}}{\Delta_1'}{P'}
					\label{lem:wb_eq_wbf1}
				\end{eqnarray}
				then $\exists Q, V_2$ such that
			%
				\begin{eqnarray}
					\horel{\Gamma}{\Delta_2}{Q}{\Hby{\news{\widetilde{m_2}} \bactout{n}{V_2}}}{\Delta_2'}{Q'}
					\label{lem:wb_eq_wbf2}
				\end{eqnarray}
			%
				and for fresh $t$:
				\[
					\mhorel{\Gamma}{\Delta_1'}{\newsp{\widetilde{m_1}}{P' \Par \hotrigger{t}{x}{s}{V_1}}}
					{\hwb}
					{\Delta_2}{}{\newsp{\widetilde{m_2}}{Q' \Par \hotrigger{t}{x}{s}{V_2} }}
				\]
			%
				From the last typed pair we can derive that for $\Gamma; \es; \Delta \proves V_1 \hastype U$:
			%
				\[
					\mhorel{\Gamma}{\Delta_1'}{\newsp{\widetilde{m_1}}{P' \Par \hotrigger{t}{x}{s}{V_1}}}
					{\hby{\bactinp{t}{\map{\btinp{U} \tinact}^{x}}}}
					{\Delta_1''}{}{\newsp{\widetilde{m_1}}{P' \Par \newsp{s}{\map{\btinp{U} \tinact}^{s} \Par \bout{\dual{s}}{V_1} \inact}}}
				\]
			%
				\noi implies
			%
				\[
					\mhorel{\Gamma}{\Delta_2'}{\newsp{\widetilde{m_2}}{Q' \Par \hotrigger{t}{x}{s}{V_2}}}
					{\hby{\bactinp{t}{\map{\btinp{U} \tinact}^{x}}}}
					{\Delta_2''}{}{\newsp{\widetilde{m_2}}{Q' \Par \newsp{s}{\map{\btinp{U} \tinact}^{s} \Par \bout{\dual{s}}{V_2} \inact}}}
				\]
			%
				\noi and $\Gamma; \es; \Delta' \proves V_2 \hastype U$.

				\noi To prove that $\Re$ is a characteristic bisimulation we
				have that transition~\eqref{lem:wb_eq_wbf1} implies transition~\eqref{lem:wb_eq_wbf2}.
				We still need to show that for fresh $t$:
			%
				\begin{eqnarray}
					\mhorel{\Gamma}{\Delta_1'}{\newsp{\widetilde{m_1}}{P' \Par \fotrigger{t}{x}{s}{\btinp{U} \tinact}{V_1}}}
					{\Re}
					{\Delta_2}{}{\newsp{\widetilde{m_2}}{Q' \Par \fotrigger{t}{x}{s}{\btinp{U} \tinact}{V_2}}}
					\label{lem:wb_eq_wbf3}
				\end{eqnarray}
			%
				\noi To show result \eqref{lem:wb_eq_wbf3} we need to prove that the processes % in \eqref{lem:wb_eq_wbf3}
				are higher-order bisimilar. We can show that by observing an input action on the fresh name $t$ on
				both processes:

				\noi The freshness of $t$ implies that
				\[
					\mhorel{\Gamma}{\Delta_1'}{\newsp{\widetilde{m_1}}{P' \Par \fotrigger{t}{x}{s}{\btinp{U} \tinact}{V_1}}}
					{\hby{\bactinp{t}{m'}}}
					{\Delta_1''}{}{\newsp{\widetilde{m_1}}{P' \Par \newsp{s}{\map{\btinp{U} \tinact}^{s} \Par \bout{\dual{s}}{V_1} \inact}}}
				\]
				\noi and
			%
				\[
					\mhorel{\Gamma}{\Delta_2'}{\newsp{\widetilde{m_2}}{Q' \Par \fotrigger{t}{x}{s}{\btinp{U} \tinact}{V_2}}}
					{\hby{\bactinp{t}{m'}}}
					{\Delta_2''}{}{\newsp{\widetilde{m_2}}{Q' \Par \newsp{s}{\map{\btinp{U} \tinact}^{s} \Par \bout{\dual{s}}{V_2} \inact}}}
				\]
			%
				\noi	thus processes in \eqref{lem:wb_eq_wbf3} are higher order bisimilar and
						thus they are included in $\Re$.
			%			which coincides with the transitions for $\hwb$,
			%			from where we can conclude that 
		

		\item	The other cases are trivial.
	\end{enumerate}

	\noi The direction $\fwb \subseteq \hwb$ is very similar to the
	direction $\hwb \subseteq \fwb$: it requires a case analysis
	on the transition label $\ell$. Again, the non-trivial case is
	$\ell = \news{\widetilde{m_1}} \bactout{n}{V_1}$.
	\qed
\end{proof}
\end{comment}


%%%%%%%%%%%%%%%%%%%%%%%%%%%%%%%%%%%%%%%%%%%%%%%%%%%%%%%%%
%  LINEAR SUBSTITUTION
%%%%%%%%%%%%%%%%%%%%%%%%%%%%%%%%%%%%%%%%%%%%%%%%%%%%%%%%%

The next lemma implies a process substitution lemma as a corollary.
Given two processes that are bisimilar under triggered substitution
and characteristic process substitution, we can prove that they are
bisimilar under every process substitution. This result is
the key result for proving the soundness of the bisimulation. 
%We prove for the case of the general polyadic abstractions. 

%We also use one of the equalities in the substitution lemmas. 
%However all results hold for other equivalences. 
%\jp{
%Below I slightly modified the statement and added some structure (enumerate).
%Some remaining issues:
%
%The statement should have some quantification on $P_1, Q_1$, I guess.
%
%The abstraction still mentions $\widetilde{x}$, it is better to have $\widetilde{z}$.
%
%The closure contains pairs of processes, not judgments: no need to mention $\Gamma$, $\Delta_1, \Delta_2$ in the 
%LHS of the definition of $\Re$. 

%I don't understand what is process 
%$\newsp{\widetilde{m_1}}{P_1 \subst{\abs{\widetilde{x}}{R}}{x} \Par P_2\subst{\abs{\widetilde{x}}{R}}{x} }$
%right before the case analysis. Do you mean
%$\newsp{\widetilde{m_1}}{P_1\Par P_2\subst{\abs{\widetilde{x}}{R}}{x} }$
%(without the substitution in $P_1$?) Is $x$ free also in $P_1$?

%The rationale for the case analysis is not clear: why the two cases 
%$P_2 = \appl{x}{\widetilde{n}}$ and $P_2 \not= \appl{x}{\widetilde{n}}$
%are the only relevant cases? Where do you get the information to narrow down the options??

%We need to mention what $C_1$ and $C_2$ are in the first case.

%}

\begin{lemma}[Linear Process Substitution]
	\label{app:lem:subst_equiv}
	Let $P_2$ and $Q_2$ be processes, with $\fpv{P_2} = \fpv{Q_2} = \set{x}$.
	If $\exists \Gamma$ and $x: U$ such that the following hold
%
	\begin{enumerate}
		\item	$\exists \Delta_1''', \Delta_2''',$ such that
				$\Gamma; x: U; \Delta_1''' \proves P_2 \hastype \Proc$ and $\Gamma; x: U; \Delta_2''' \proves Q_2 \hastype \Proc$.
		\item	$\exists P_1, Q_1$ such that
		\begin{enumerate}[i.]
			\item	$\exists \Delta_1'''', \Delta_2''''$ and fresh $t$ such that
					$\horel{\Gamma}{\Delta_1''''}{\newsp{\widetilde{m_1}}{P_1 \Par P_2 \subst{\auxtr{t}}{x}}}
					{\hwb}
					{\Delta_2''''}{\newsp{\widetilde{m_2}}{Q_1 \Par Q_2 \subst{\auxtr{t}}{x}}}$.

			\item	$\exists \Delta_1'', \Delta_2''$ such that
					$\horel{\Gamma}{\Delta_1''}{\newsp{\widetilde{m_1}}{P_1 \Par P_2 \subst{\omapchar{U}}{x}}}
					{\hwb}{\Delta_2''}{\newsp{\widetilde{m_2}}{Q_1 \Par Q_2 \subst{\omapchar{U}}{x}}}$.
%					for some $U$.
		\end{enumerate}
	\end{enumerate}
%
	then $\forall R$ such that $\fv{R} = \widetilde{z}$
\[
	\horel{\Gamma}{\Delta_1}{\newsp{\widetilde{m_1}}{P_1 \Par P_2 \subst{\abs{\widetilde{z}}{R}}{x}}}
	{\hwb}
	{\Delta_2}{\newsp{\widetilde{m_2}}{Q_1 \Par Q_2 \subst{\abs{\widetilde{z}}{R}}{x}}}
\]
\end{lemma}

\begin{proof}
	Let relation
%
	\begin{eqnarray*}
		\Re &=&
%			\set{\horel{\Gamma}{\Delta_1}{\newsp{\widetilde{m_1}}{P_1 \Par P_2 \subst{\abs{\widetilde{z}}{R}}{x}}}{,}
%			{\Delta_2}{\newsp{\widetilde{m_2}}{Q_1 \Par Q_2 \subst{\abs{\widetilde{z}}{R}}{x}}} \setbar \\
			\set{
				(\Gamma; \Delta_1 \proves \newsp{\widetilde{m_1}}{P_1 \Par P_2 \subst{\abs{\widetilde{z}}{R}}{x}} \hastype \Proc
				\ ,\ 
				\Gamma; \Delta_2 \proves \newsp{\widetilde{m_2}}{Q_1 \Par Q_2 \subst{\abs{\widetilde{z}}{R}}{x}} \hastype \Proc)
				\setbar
				\\
			&& \quad \forall R \textrm{ such that } \fv{R} = \widetilde{z}, \fpv{P_2} = \fpv{Q_2} = \set{x} \vee \exists U \text{ such that }
				\\
			&& \quad ((\exists \Delta_1''', \Delta_2''', \text{fresh } t, \Gamma; x: U; \Delta_1''' \proves P_2 \hastype \Proc, \Gamma; x: U; \Delta_2''' \proves Q_2 \hastype \Proc),
				\\
			&& \quad \land (\exists \Delta_1'''', \Delta_2'''', \\
			&& \quad \horel{\Gamma}{\Delta_1''''}{\newsp{\widetilde{m_1}}{P_1 \Par P_2 \subst{\auxtr{t}}{x}}}{\hwb}{\Delta_2''''}{\newsp{\widetilde{m_2}}{Q_1 \Par Q_2 \subst{\auxtr{t}}{x}}}),
				\\
			&& \quad \land (\exists \Delta_1'', \Delta_2'', \horel{\Gamma}{\Delta_1''}{\newsp{\widetilde{m_1}}{P_1 \Par P_2 \subst{\omapchar{U}}{x}}}{\hwb}{\Delta_2''}{\newsp{\widetilde{m_2}}{Q_1 \Par Q_2 \subst{\omapchar{U}}{x}}))} \\%\textrm{ for some } U}\\
			&&}
	\end{eqnarray*}
%
	\noi  We show that $\Re$ is a higher-order bisimulation up-to \betatran (\lemref{lem:up_to_deterministic_transition}).

	\noi We do a case analysis on the structure of processes that perform the transition:
%
	\[
		\horel{\Gamma}{\Delta_1}{\newsp{\widetilde{m_1}}{P_1 \Par P_2\subst{\abs{\widetilde{z}}{R}}{x} }}{\hby{\ell_1}}{\Delta_1'}{P_1'}
	\]
%
	\noi We distinguish two cases:
	i) either $P_2 = \appl{x}{\widetilde{n}}$ for some $\widetilde{n}$,
	where the substitution of $\abs{\widetilde{z}}{R}$
	on variable $x$ gives
	$P_2 = \appl{\abs{\widetilde{z}}{R}}{\widetilde{n}}$ so we can observe the
	application on abstraction $\abs{\widetilde{z}}{R}$; or
	ii) $P_2 \not= \appl{x}{\widetilde{n}}$, where we can observe
	an action on $P_2$ that does not involve $\abs{\widetilde{z}}{R}$.
%	and easily verify the closure $\Re$.

	%%%%%%%%%%%%%%%%%%%%%%%%%%%%%%
	%          Case
	%%%%%%%%%%%%%%%%%%%%%%%%%%%%%%
\begin{enumerate}
	\item Case: $P_2 \not= \appl{x}{\widetilde{n}}$ for some $\widetilde{n}$.
%
	\begin{eqnarray}
		&&	\horel{\Gamma}{\Delta_1}
		{\newsp{\widetilde{m_1}}{P_1 \Par P_2 \subst{\abs{\widetilde{z}}{R}}{x}}}
		{\hby{\ell_1}}
		{\Delta_1'}{\newsp{\widetilde{m_1'}}{P_1' \Par P_2' \subst{\abs{\widetilde{z}}{R}}{x}}}
		\label{lem:subst_equiv0}
	\end{eqnarray}

	\noi From the latter transition, the fact $P_2 \not= \appl{x}{\widetilde{n}}$ we obtain that:
%
\[
		\mhorel{\Gamma}{\Delta_1''''}{\newsp{\widetilde{m_1}}{P_1 \Par P_2 \subst{\auxtr{t}}{x}}}
		{\hby{\ell_1}}{\Delta_3}{P' \scong}{\newsp{\widetilde{m_1}}{P_1' \Par P_2' \subst{\auxtr{t}}{x}}}
\]
%
	\noi
		From the freshness of $t$ ($t \notin \fn{\ell_1}$) and
		from the definition of $\Re$, specifically, the requirement
		$\horel{\Gamma}{\Delta_1''''}
		{\newsp{\widetilde{m_1}}{P_1 \Par P_2 \subst{\auxtr{t}}{x} }}
		{\hwb}
		{\Delta_2''''}{\newsp{\widetilde{m_2}}{Q_1 \Par Q_2 \subst{\auxtr{t}}{x}}}$, we get that
		$\exists \newsp{\widetilde{m_2'}}{Q_1' \Par Q_2' \subst{\auxtr{t}}{x}}, \Delta_4$ such that:
%
	\begin{eqnarray}
		\Gamma; \es; &\Delta_2''''& \proves \newsp{\widetilde{m_2}}{Q_1 \Par Q_2 \subst{\auxtr{t}}{x}} \nonumber \\
%		\Hby{}&& \newsp{\widetilde{m_2}}{Q_1'' \Par Q_2'' \subst{\auxtr{t}}{x}} \nonumber\\
%		\hby{\ell_2}&& \newsp{\widetilde{m_2}}{Q_1''' \Par Q_2''' \subst{\auxtr{t}}{x}} \nonumber\\
		\Hby{\ell_2} &\Delta_4& \proves \newsp{\widetilde{m_2'}}{Q_1' \Par Q_2' \subst{\auxtr{t}}{x}}
		\label{lem:subst_equiv1}
		\\
		&&\horel{\Gamma}{\Delta_3}{\newsp{m_1'}{P' \Par C_1}}{\hwb}{\Delta_4}{\newsp{m_2'}{Q_1' \Par Q_2' \subst{\auxtr{t}}{x} \Par C_2}} \label{lem:subst_equiv2}
	\end{eqnarray}
%
	\noi with $C_1 = \htrigger{t'}{V_1}, C_2 = \htrigger{t'}{V_2}$ for fresh $t'$
	in the case where $\ell_1 = \news{m_1''}\bactout{n}{V_1}, \ell_2 = \news{m_2''}\bactout{n}{V_2}$ or $C_1 = C_2 = \inact$ otherwise.   
 
	\noi Furthermore, from the fact that $P_2 \not= \appl{x}{\widetilde{n}}$ we have:
%
\[
	\horel{\Gamma}{\Delta_1''}{\newsp{\widetilde{m_1}}{P_1 \Par P_2 \subst{\omapchar{U}}{x}}}{\hby{\ell_1}}
	{\Delta_5}{P'' \scong \newsp{\widetilde{m_1'}}{P_1' \Par P_2' \subst{\omapchar{U}}{x}}}
\]
%
	\noi
		From \ref{lem:subst_equiv0}, together with the latter transition consider process $C_1, C_2$ and action $\ell_2$.
		From the definition of $\Re$, specifically, the requirement
		$\horel{\Gamma}{\Delta_1''}{\newsp{\widetilde{m_1}}{P_1 \Par P_2 \subst{\omapchar{U}}{x}}}
		{\hwb}{\Delta_2''}{\newsp{\widetilde{m_2}}{Q_1 \Par Q_2 \subst{\omapchar{U}}{x}}))}$,
		we get that $\exists \newsp{\widetilde{m_2}'}{Q_1' \Par Q_2' \subst{\omapchar{U}}{x}}, \Delta_6$ such that:
%
	\begin{eqnarray}
		\Gamma; \es; &\Delta_2''& \proves \newsp{\widetilde{m_2}}{Q_1 \Par Q_2 \subst{\omapchar{U}}{x}} \nonumber \\
		\Hby{\ell_2} &\Delta_6& \proves \newsp{\widetilde{m_2}'}{Q_1' \Par Q_2' \subst{\omapchar{U}}{x}}
		\label{lem:subst_equiv3}
		\\
		&&\horel{\Gamma}{\Delta_5}{\newsp{m_1'}{P'' \Par C_1}}{\hwb}{\Delta_6}{\newsp{m_2'}{Q_1' \Par Q_2' \subst{\omapchar{U}}{x} \Par C_2}} \label{lem:subst_equiv4}
	\end{eqnarray}
%
	\noi From \eqref{lem:subst_equiv1} and \eqref{lem:subst_equiv3}, and $\forall R$ with $\fv{R} = \widetilde{z}$
	there exist $\Delta_2', \newsp{\widetilde{m_2}'}{Q_1' \Par Q_2' \subst{\abs{\widetilde{z}}{R}}{x}}$ such that:
%
	\[
		\horel{\Gamma}{\Delta_2}{\newsp{\widetilde{m_2}}{Q_1 \Par Q_2 \subst{\abs{\widetilde{z}}{R}}{x}}}
		{\Hby{\ell_2}}
		{\Delta_2'}{\newsp{\widetilde{m_2}'}{Q_1' \Par Q_2' \subst{\abs{\widetilde{z}}{R}}{x}}}
	\]
	%
	\noi
		Also if we combine~\eqref{lem:subst_equiv2} and~\eqref{lem:subst_equiv4} and the definition of $\Re$ we get that
		$\forall R$ with $\fv{R} = \widetilde{z}$:
	%
	\begin{eqnarray*}
		\horel{\Gamma}{\Delta_1'}{\newsp{\widetilde{m_1}'}{P_1' \Par P_2' \subst{\abs{\widetilde{z}}{R}}{x} \Par C_1}}
		{\ \Re\ }
		{\Delta_2'}{\newsp{\widetilde{m_2}'}{Q_1' \Par Q_2' \subst{\abs{\widetilde{z}}{R}}{x} \Par C_2}}
	\end{eqnarray*}
	%
	The last two equations conclude the requirement for
	$\Re$ being a higher-order bisimulation.
%

	%%%%%%%%%%%%%%%%%%%%%%%%%%%%%%
	%          Case
	%%%%%%%%%%%%%%%%%%%%%%%%%%%%%%

	\item	Case: $P_2 = \appl{x}{\widetilde{n}}$ for some $\widetilde{n}$.
			We distinguish two cases:
			\begin{enumerate}[i.]
				\item 	\noi $\forall R$ with $\fv{R} = \widetilde{z}, \exists \Delta_1'$ such that:
%
						\[
							\horel{\Gamma}{\Delta_1}{\newsp{\widetilde{m_1}}{P_1 \Par (\appl{x}{\widetilde{n}}) \subst{\abs{\widetilde{z}}{R}}{x}}}
							{\hby{\ell_1}}
							{\Delta_1'}{}{\newsp{\widetilde{m_1'}}{P_1' \Par (\appl{x}{\widetilde{n}}) \subst{\abs{\widetilde{z}}{R}}{x}}}
						\]
%
						\noi The case is very similar with Case 1 of this proof.

				\item	\noi $\forall R$ with $\fv{R} = \widetilde{z}$
%
						\begin{eqnarray}
							\horel{\Gamma}{\Delta_1}{\newsp{\widetilde{m_1}}{P_1 \Par (\appl{x}{\widetilde{n}}) \subst{\abs{\widetilde{z}}{R}}{x}}}
							{\hby{\btau}}
							{\Delta_1}{\newsp{\widetilde{m_1}}{P_1 \Par  R \subst{\widetilde{n}}{\widetilde{z}}}}
							\label{lem:subst_equiv5}
						\end{eqnarray}
%
						\noi
							From the latter transition we get that:
%
						\begin{eqnarray*}
							\horel{\Gamma}{\Delta_1''''}{\newsp{\widetilde{m_1}}{P_1 \Par \appl{x}{\widetilde{n}} \subst{\auxtr{t}}{x}}}
							{\hby{\btau}}
							{\Delta_1''''}{\newsp{\widetilde{m_1}}{P_1 \Par \binp{t}{y} \appl{y}{\widetilde{n}}}}
%							\label{lem:subst_equiv6}
						\end{eqnarray*}
%
						From the Determinacy Lemma (\lemref{app:lem:tau_inert}) we get
%
						\begin{eqnarray*}
							\horel{\Gamma}{\Delta_1''''}{\newsp{\widetilde{m_1}}{P_1 \Par \appl{x}{\widetilde{n}} \subst{\auxtr{t}}{x}}}
							{\hwb}
							{\Delta_1''''}{\newsp{\widetilde{m_1}}{P_1 \Par \binp{t}{y} \appl{y}{\widetilde{n}}}}
%							\label{lem:subst_equiv6}
						\end{eqnarray*}
%
						and from the definition of $\Re$, specifically, the requirement
						$\horel{\Gamma}{\Delta_1''''}
						{\newsp{\widetilde{m_1}}{P_1 \Par \appl{x}{\widetilde{n}} \subst{\auxtr{t}}{x} }}
						{\hwb}
						{\Delta_2''''}{\newsp{\widetilde{m_2}}{Q_1 \Par Q_2 \subst{\auxtr{t}}{x}}}$, and the
						transitivity of bisimulation we get:
%
						\begin{eqnarray}
							\horel{\Gamma}{\Delta_1''''}{\newsp{\widetilde{m_1}}{P_1 \Par \binp{t}{y} \appl{y}{\widetilde{n}}}}
							{\hwb}
							{\Delta_2''''}{\newsp{\widetilde{m_2}}{Q_1 \Par Q_2 \subst{\auxtr{t}}{x}}}
							\label{lem:subst_equiv7}
						\end{eqnarray}
%
						Furthermore, from \eqref{lem:subst_equiv5} we get:
%
						\begin{eqnarray*}
							\horel{\Gamma}{\Delta_1''}{\newsp{\widetilde{m_1}}{P_1 \Par \appl{x}{\widetilde{n}} \subst{\omapchar{U}}{x}}}
							{\hby{\btau}}
							{\Delta_1''}{\newsp{\widetilde{m_1}}{P_1 \Par \mapchar{U}{\widetilde{n}}}}% \binp{t}{y} \appl{y}{\widetilde{n}}}}
%								\label{lem:subst_equiv6}
						\end{eqnarray*}
%
						From the Determinacy Lemma (\lemref{app:lem:tau_inert}) we get
%
						\begin{eqnarray*}
							\horel{\Gamma}{\Delta_1''}{\newsp{\widetilde{m_1}}{P_1 \Par \appl{x}{\widetilde{n}} \subst{\omapchar{U}}{x}}}
							{\hwb}
							{\Delta_1''}{\newsp{\widetilde{m_1}}{P_1 \Par \mapchar{U}{\widetilde{n}}}}% \binp{t}{y} \appl{y}{\widetilde{n}}}}
%							\label{lem:subst_equiv6}
						\end{eqnarray*}
%
						From the definition of $\Re$, specifically, the requirement
						$\horel{\Gamma}{\Delta_1''}{\newsp{\widetilde{m_1}}{P_1 \Par \appl{x}{\widetilde{n}} \subst{\omapchar{U}}{x}}}
						{\hwb}{\Delta_2''}{\newsp{\widetilde{m_2}}{Q_1 \Par Q_2 \subst{\omapchar{U}}{x}}))}$,
						and the transitivity of $\hwb$ we get that:
%
						\begin{eqnarray}
							\horel{\Gamma}{\Delta_1''}{\newsp{\widetilde{m_1}}{P_1 \Par \mapchar{U}{\widetilde{n}}}}
							{\hwb}
							{\Delta_2''}{\newsp{\widetilde{m_2}}{Q_1 \Par Q_2 \subst{\omapchar{U}}{x}}}
							\label{lem:subst_equiv9}
						\end{eqnarray}
%
%							\noi From \eqref{lem:subst_equiv6} and \eqref{lem:subst_equiv8}, and $\forall R$ with $\fv{R} = \widetilde{z}$
%							there exist $\newsp{\widetilde{m_2}}{Q_1' \Par Q_2' \subst{\abs{\widetilde{z}}{R}}{x}}$ such that:
%						%
%							\[
%								\horel{\Gamma}{\Delta_2}{\newsp{\widetilde{m_2}}{Q_1 \Par Q_2 \subst{\abs{\widetilde{z}}{R}}{x}}}
%								{\Hby{}}
%								{\Delta_2}{\newsp{\widetilde{m_2}}{Q_1' \Par Q_2' \subst{\abs{\widetilde{z}}{R}}{x}}}
%							\]
%							%
						\noi
						If we combine~\eqref{lem:subst_equiv7} and~\eqref{lem:subst_equiv9} and the definition of $\Re$ we get that
						$\forall R$ with $\fv{R} = \widetilde{z}$:
						%
						\begin{eqnarray*}
							\horel{\Gamma}{\Delta_1}{\newsp{\widetilde{m_1}}{P_1 \Par \appl{x}{\widetilde{n}} \subst{\abs{\widetilde{z}}{R}}{x}  }}
							{\hby{\dtau} \ \Re\ }
							{\Delta_2}{\newsp{\widetilde{m_2}}{Q_1 \Par Q_2 \subst{\abs{\widetilde{z}}{R}}{x}}}
						\end{eqnarray*}
						%
						which coincides with the requirements of a $\Re$ being a bisimulation
						up-to determinate transition (\lemref{lem:up_to_deterministic_transition}).
			\end{enumerate}

%	\item Case: $P_2 = \appl{x}{\widetilde{n}}$ for some $\widetilde{n}$.
%
%%	\noi We do a case analysis on action $\ell$.
%%	\noi - Subcase: $\ell \not= \tau$.
%
%	\noi $\forall R$ with $\fv{R} = \widetilde{z}$
%%
%	\[
%		\mhorel{\Gamma}{\Delta_1}{\newsp{\widetilde{m_1}}{P_1 \Par (\appl{x}{\widetilde{n}}) \subst{\abs{\widetilde{z}}{R}}{x}}}
%		{\hby{\btau}}
%		{\Delta_1'}{}{\newsp{\widetilde{m_1'}}{P_1 \Par  R \subst{\widetilde{n}}{\widetilde{z}}}}
%	\]
%%
%	\noi From the latter transition we get that:
%%
%	\nhorel{\Gamma}{\Delta_1}{\newsp{\widetilde{m_1}}{P_1 \Par \appl{x}{\widetilde{n}} \subst{\auxtr{t}}{x}}}
%	{\hby{\btau} \hby{\bactinp{t}{\auxtr{t'}}}}
%	{\Delta_1'}{\newsp{\widetilde{m_1'}}{P_1 \Par \appl{x}{\widetilde{n}} \subst{\auxtr{t'}}{x}}}
%	{lem:subst_equiv5}
%%
%%	\begin{eqnarray}
%%		\Gamma; \es; \Delta_1 &\hby{\bactinp{t}{\auxtr{t'}}}& \Delta_1' \proves
%%		\newsp{\widetilde{m_1}}{P_1 \Par \appl{X}{\widetilde{n}} \subst{\auxtr{t}}{X}} \nonumber \\
%%		&\hby{\bactinp{t}{\auxtr{t'}}}& 
%%		\newsp{\widetilde{m_1'}}{P_1 \Par \appl{X}{\widetilde{n}} \subst{\auxtr{t'}}{X}}
%%		\label{lem:subst_equiv5}
%%	\end{eqnarray}
%%
%	\noi and $t'$ a fresh name. From the freshness of $t$,
%	the determinacy of the application transition
%	and the fact that $x$ is linear in $Q_2$
%	it has to be the case that:
%%
%	\[
%		\begin{array}{crll}
%			&\Gamma; \es; \Delta_2'& \proves &
%			\newsp{\widetilde{m_2'}}{Q_1 \Par Q_2 \subst{\auxtr{t}}{x}}\\
%			\Hby{} & & &
%			\newsp{\widetilde{m_2'}}{Q_1'' \Par Q_3 \Par \appl{x}{\widetilde{m}} \subst{\auxtr{t}}{x}} \\
%			\hby{\btau} \hby{\bactinp{t}{\auxtr{t'}}}
%			& \Delta_2''& \proves& \newsp{\widetilde{m_2'}}{Q_1' \Par \appl{x}{\widetilde{m}} \subst{\auxtr{t'}}{x}}
%		\end{array}
%	\]
%%
%%
%%	\[
%%		\begin{array}{rcll}
%%			\Gamma; \es; \Delta_2' &\Hby{\bactinp{t}{\auxtr{t'}}}& \Delta_2'' \proves &
%%			\newsp{\widetilde{m_2'}}{Q_1 \Par Q_2 \subst{\auxtr{t}}{X}}\\
%%			&\Hby{}& &
%%			\newsp{\widetilde{m_2'}}{Q_1'' \Par Q_3 \Par \appl{X}{\widetilde{m}} \subst{\auxtr{t}}{X}} \\
%%			&\hby{\bactinp{t}{\auxtr{t'}}}& &
%%			\newsp{\widetilde{m_2'}}{Q_1' \Par \appl{X}{\widetilde{m}} \subst{\auxtr{t'}}{X}} \\
%%		\end{array}
%%	\]
%%
%	\noi and
%%
%%	\nhorel{\Gamma}{\Delta_1'}{\newsp{\widetilde{m_1'}}{P_1 \Par \appl{x}{\widetilde{n}} \subst{\auxtr{t'}}{x}}}
%%	{\hwb}
%%	{\Delta_2'}{\newsp{\widetilde{m_2}'}{Q_1' \Par \appl{x}{\widetilde{m}} \subst{\auxtr{t'}}{x}}}
%%	{lem:subst_equiv6}
%%
%%
%%	\begin{eqnarray}
%%		\Gamma; \es; \Delta_1' &\hwb& \Delta_2' \proves 
%%		\newsp{\widetilde{m_1'}}{P_1 \Par \appl{X}{\widetilde{n}} \subst{\auxtr{t'}}{X}} \nonumber \\
%%		& \hwb &
%%		\newsp{\widetilde{m_2}'}{Q_1' \Par \appl{X}{\widetilde{m}} \subst{\auxtr{t'}}{X}}
%%		\label{lem:subst_equiv6}
%%	\end{eqnarray} 
%%
%	\noi From the latter transition we can conclude that $\forall R$ with $\fv{R} = \set{x}$:
%%
%	\[
%		\begin{array}{crll}
%			&\Gamma; \es; \Delta_2'& \proves & 
%			\newsp{\widetilde{m_2'}}{Q_1 \Par Q_2 \subst{\abs{\widetilde{z}}{R}}{x}}\\
%			\Hby{} & & &
%			\newsp{\widetilde{m_2'}}{Q_1' \Par \appl{x}{\widetilde{m}} \subst{\abs{\widetilde{z}}{R}}{x}} \\
%			\hby{\btau}
%			&\Delta_2''& \proves & \newsp{\widetilde{m_2'}}{Q_1' \Par R \subst{\widetilde{m}}{\widetilde{z}}}%\appl{x}{\widetilde{m}} \subst{\abs{\widetilde{x}}{R'}}{x}}
%		\end{array}
%	\]
%%
%%	\[
%%		\begin{array}{rcll}
%%			\Gamma; \es; \Delta_2' &\Hby{\ell}& \Delta_2'' \proves &
%%			\newsp{\widetilde{m_2'}}{Q_1 \Par Q_2 \subst{\abs{\widetilde{x}}{R}}{X}}\\
%%			&\Hby{}& &
%%			\newsp{\widetilde{m_2'}}{Q_1' \Par \appl{X}{\widetilde{m}} \subst{\abs{\widetilde{x}}{R}}{X}} \\
%%			&\hby{\ell}& &
%%			\newsp{\widetilde{m_2'}}{Q_1' \Par \appl{X}{\widetilde{m}} \subst{\abs{\widetilde{x}}{R'}}{X}} \\
%%		\end{array}
%%	\]
%%
%	\noi From the definition of $\Re$ and~\eqref{lem:subst_equiv6},
%	we also conclude that
%	\begin{eqnarray*}
%		&& \horel{\Gamma}
%		{\Delta_1'}{\newsp{\widetilde{m_1'}}{P_1 \Par R \subst{\widetilde{n}}{\widetilde{z}}}}
%		{\hby{\btau}\ \Re\ \stackrel{\btau}{\longmapsfrom}}
%		{\Delta_2'}{\newsp{\widetilde{m_2}'}{Q_1' \Par R \subst{\widetilde{m}}{\widetilde{z}}}}
%	\end{eqnarray*}
	\end{enumerate}
%
%	\noi the latter substitution can be rewritten as
%
%	\begin{eqnarray*}
%		&&\horel{\Gamma}
%		{\Delta_1'}{\newsp{\widetilde{m_1'}}{P_1 \Par \appl{X}{s} \subst{\abs{\widetilde{x}}{R'}}{X} \Par C}}
%		{\ \mathcal{S}\ }
%		{\Delta_2'}{\newsp{\widetilde{m_2}'}{Q_1' \Par \appl{X}{s'} \subst{\abs{\widetilde{x}}{R'}}{X} \Par C}}
%	\end{eqnarray*}
%
%	\noi with process $C$ being the process derived from action $\ell$
%	to complete the bisimulation closure.
	%\qed
\begin{comment}
	\noi - Subcase: $\ell = \tau$.

	\noi $\forall R$ with $\fv{R} = \widetilde{x}$
%
	\[
		\mhorel{\Gamma}{\Delta_1}{\newsp{\widetilde{m_1}}{P_1 \Par \appl{X}{\widetilde{n}} \subst{\abs{\widetilde{x}}{R}}{x}}}
		{\hby{\tau}}
		{\Delta_1'}{}{\newsp{\widetilde{m_1'}}{P_1' \Par \appl{X}{\widetilde{n}} \subst{\abs{\widetilde{x}}{R'}}{x}}}
	\]
%
	\noi The last transition implies
	\[
		\mhorel{\Gamma}{\Delta_1}{\newsp{\widetilde{m_1}}{P_1 \Par \appl{X}{\widetilde{n}} \subst{\abs{\widetilde{x}}{R}}{x}}}
		{\hby{\ell_1}}
		{\Delta_1'}{}{\newsp{\widetilde{m_1'}}{P_1 \Par \appl{X}{\widetilde{n}} \subst{\abs{\widetilde{x}}{R'}}{x}}}
	\]
%
	\noi and
%
	\[
		\mhorel{\Gamma}{\Delta_1}{\newsp{\widetilde{m_1}}{P_1 \Par \appl{X}{\widetilde{n}} \subst{\abs{\widetilde{x}}{R}}{x}}}
		{\hby{\ell_2}}
		{\Delta_1'}{}{\newsp{\widetilde{m_1'}}{P_1' \Par \appl{X}{\widetilde{n}} \subst{\abs{\widetilde{x}}{R}}{x}}}
	\]
%
	\noi We also get that:
%
	\[
		\begin{array}{crll}
			&\Gamma; \es; \Delta_1 & \proves &
			\newsp{\widetilde{m_1}}{P_1 \Par \appl{X}{\widetilde{n}} \subst{\auxtr{t}}{x}} \\
			\hby{\ell_1} & \Delta_1'' & \proves & \newsp{\widetilde{m_1}}{P_1' \Par \appl{x}{\widetilde{n}} \subst{\auxtr{t}}{x}} \\
			\hby{\bactinp{t}{\abs{\widetilde{x}}{\binp{t'}{y} \appl{y}{\widetilde{x}}}}} 
			&\Delta_1'& \proves & \newsp{\widetilde{m_1'}}{P_1' \Par \appl{x}{\widetilde{n}} \subst{\abs{\widetilde{x}}{\binp{t'}{y} \appl{y}{\widetilde{x}}}}{x}}
		\end{array}
	\]
%
%	\begin{eqnarray*}
%		\Gamma; \es; \Delta_1 &\hby{\bactinp{t}{\auxtr{t'}}}& \Delta_1' \proves
%		\newsp{\widetilde{m_1}}{P_1 \Par \appl{X}{\widetilde{n}} \subst{\auxtr{t}}{X}} \\
%		&\hby{\ell_1}& \newsp{\widetilde{m_1}}{P_1' \Par \appl{X}{\widetilde{n}} \subst{\auxtr{t}}{X}} \\
%		&\hby{\bactinp{t}{\auxtr{t'}}}& 
%		\newsp{\widetilde{m_1'}}{P_1' \Par \appl{X}{\widetilde{n}} \subst{\auxtr{t'}}{X}}
%		\label{lem:subst_equiv5}
%	\end{eqnarray*}
%
	\noi and $t'$ a fresh name. From the freshness of $t$ and the fact that $X$ is linear in $Q_2$
	it has to be the case that:
%
	\[
		\begin{array}{crll}
			& \Gamma; \es; \Delta_2'&  \proves &
			\newsp{\widetilde{m_2'}}{Q_1 \Par Q_2 \subst{\auxtr{t}}{x}}\\
			\Hby{\ell_2}& & &
			\newsp{\widetilde{m_2'}}{Q_1'' \Par Q_3 \Par \appl{x}{\widetilde{m}} \subst{\auxtr{t}}{x}} \\
			\hby{\bactinp{t}{\abs{\widetilde{x}}{\binp{t'}{y} \appl{y}{\widetilde{x}}}}}
			&\Delta_2''& \proves& \newsp{\widetilde{m_2'}}{Q_1' \Par \appl{x}{\widetilde{m}} \subst{\abs{\widetilde{x}}{\binp{t'}{y} \appl{y}{\widetilde{x}}}}{x}} \\
		\end{array}
	\]
%
%	\[
%		\begin{array}{rcll}
%			\Gamma; \es; \Delta_2' &\Hby{\bactinp{t}{\auxtr{t'}}}& \Delta_2'' \proves &
%			\newsp{\widetilde{m_2'}}{Q_1 \Par Q_2 \subst{\auxtr{t}}{X}}\\
%			&\Hby{\ell_2}& &
%			\newsp{\widetilde{m_2'}}{Q_1'' \Par Q_3 \Par \appl{X}{\widetilde{m}} \subst{\auxtr{t}}{X}} \\
%			&\hby{\bactinp{t}{\auxtr{t'}}}& &
%			\newsp{\widetilde{m_2'}}{Q_1' \Par \appl{X}{\widetilde{m}} \subst{\auxtr{t'}}{X}} \\
%		\end{array}
%	\]
%
	\noi From here the proof is similar with the previous case.
	%\qed
\end{comment}
\qed
\end{proof}

\noindent
We can generalise the result of the linear process substitution lemma
to prove process substitution (\lemref{app:lem:proc_subst}).
Intuitively, we can subsequently apply linear process substitution
to achieve process substitution.

%\jp{The following is so "loose" that I don't understand how can it help...}

We first need to show that any process has a normal form:
\begin{lemma}[Normal Form]
	\label{app:lem:nomral_form}
	For every process $P$, there exist (possibly empty) $\tilde{m}$
	and (possibly inactive) $P_1, P_2$ such that
	$P \scong \newsp{\tilde{m}}{P_1 \Par P_2}$.
\end{lemma}

\begin{proof}[Sketch]
	The proof is a simple induction on the structure
	of $P$ that uses structural congruence to show that
	$P \scong \newsp{\tilde{m}}{P_1 \Par P_2}$. 
\qed
\end{proof}


%%%%%%%%%%%%%%%%%%%%%%%%%%%%%%%%%%%%%%
%	Process Substitution
%%%%%%%%%%%%%%%%%%%%%%%%%%%%%%%%%%%%%%

%\jp{Some of the issues above apply verbatim also to this statement/proof.}

\begin{lemma}[Process Substitution]
	\label{app:lem:proc_subst}
	Let $P$ and $Q$ be processes. If 
%
	\begin{enumerate}
		\item	$\horel{\Gamma}{\Delta_1'}{P \subst{\auxtr{t}}{x}}{\hwb}{\Delta_2'}{Q \subst{\auxtr{t}}{x}}$,
				for some fresh $t$.

		\item	$\horel{\Gamma}{\Delta_1''}{P \subst{\omapchar{U}}{x}}{\hwb}{\Delta_2''}{Q \subst{\omapchar{U}}{x}}$,
				for some $U$.
	\end{enumerate}
%
	then $\forall R$ such that $\fv{R} = \widetilde{z}$
\[
	\horel{\Gamma}{\Delta_1}{P \subst{\abs{\widetilde{z}}{R}}{x}}{\hwb}{\Delta_2}{Q \subst{\abs{\widetilde{z}}{R}}{x}}
\]
\end{lemma}


\begin{proof}
	\noi We define a closure $\Re$ using the normal form (\lemref{app:lem:nomral_form}) of $P$ and $Q$:

	\[
		\begin{array}{rcll}
			\Re &=& \set{
			(\Gamma; \Delta_1 \proves \newsp{\widetilde{m_1}}{P_1 \subst{\abs{\widetilde{z}}{R}}{x} \Par P_2 \subst{\abs{\widetilde{z}}{R}}{x}} \hastype \Proc
			\ ,\ 
			\Gamma; \Delta_2 \proves \newsp{\widetilde{m_2}}{Q_1 \subst{\abs{\widetilde{z}}{R}}{x} \Par Q_2 \subst{\abs{\widetilde{z}}{R}}{x}} \hastype \Proc)
			\setbar \\
			&& \qquad \forall R \textrm{ such that } \fv{R} = \widetilde{z},\\
			&& \qquad \textrm{ for fresh } t,
			\mhorel{\Gamma}{\Delta_1'}
			{\newsp{\widetilde{m_1}}{P_1 \subst{\auxtr{t}}{x} \Par P_2 \subst{\auxtr{t}}{x}}}
			{\hwb}
			{\Delta_2'}{}{\newsp{\widetilde{m_2}}{Q_1 \subst{\auxtr{t}}{x} \Par Q_2 \subst{\auxtr{t}}{x}}}\\
			&& \qquad \textrm{ for some } U, 
			\mhorel{\Gamma}{\Delta_1''}
			{\newsp{\widetilde{m_1}}{P_1 \subst{\omapchar{U}}{x} \Par P_2 \subst{\omapchar{U}}{x}}}
			{\hwb}
			{\Delta_2''}{}{\newsp{\widetilde{m_2}}{Q_1 \subst{\omapchar{U}}{x} \Par Q_2 \subst{\omapchar{U}}{x}}} \\
			&&}
		\end{array}
	\]
%
%	\begin{eqnarray*}
%		\mathcal{S} &=& \set{\horel{\Gamma}{\Delta_1}{\newsp{\widetilde{m_1}}{P_1 \subst{\abs{\widetilde{x}}{R}}{X} \Par P_2 \subst{\abs{\widetilde{x}}{R}}{X}}}{,}{\Delta_2}{\newsp{\widetilde{m_2}}{Q_1 \subst{\abs{\widetilde{x}}{R}}{X} \Par Q_2 \subst{\abs{\widetilde{x}}{R}}{X}}} \setbar \\
%		&& \quad \forall R \textrm{ such that } \fv{R} = \widetilde{x},\\
%		&& \quad \textrm{ for fresh } t, \Gamma; \es; \Delta_1' \hwb \Delta_2' \proves \\
%		&& \quad \newsp{\widetilde{m_1}}{P_1 \subst{\auxtr{t}}{X} \Par P_2 \subst{\auxtr{t}}{X}} \hwb \newsp{\widetilde{m_2}}{Q_1 \subst{\auxtr{t}}{X} \Par Q_2 \subst{\auxtr{t}}{X}}, \\
%		&& \quad \textrm{ for some } U, \Gamma; \es; \Delta_1'' \hwb \Delta_2'' \proves \\
%		&& \quad \newsp{\widetilde{m_1}}{P_1 \subst{\abs{\widetilde{x}}{\map{U}^x}}{X} \Par P_2 \subst{\abs{\widetilde{x}}{\map{U}^x}}{X}} \hwb \newsp{\widetilde{m_2}}{Q_1 \subst{\abs{\widetilde{x}}{\map{U}^x}}{X} \Par Q_2 \subst{\abs{\widetilde{x}}{\map{U}^x}}{X}} \\
%		&&}
%	\end{eqnarray*}
%
	\noi We show that $\Re$ is a bisimulation up to \betatran (\lemref{app:lem:tau_inert}).

	\noi We distinguish two cases:
	i) either $P_2 = \appl{x}{\widetilde{n}}$ for some $\widetilde{n}$,
	where the substitution of $\abs{\widetilde{z}}{R}$
	on variable $x$ gives
	$P_2 = \appl{\abs{\widetilde{z}}{R}}{\widetilde{n}}$ so we can observe the
	application on abstraction $\abs{\widetilde{z}}{R}$; or
	ii) $P_2 \not= \appl{x}{\widetilde{n}}$, where we can observe
	an action on $P_2$ that does not involve $\abs{\widetilde{z}}{R}$
	and easily verify the closure $\Re$.

	\begin{enumerate} 
	\item Case: $P_2 \not= \appl{x}{\widetilde{n}}$ for some $\widetilde{n}$.
%
	\nhorel	{\Gamma}{\Delta_1}{\newsp{\widetilde{m_1}}{P_1 \subst{\abs{\widetilde{z}}{R}}{x} \Par P_2 \subst{\abs{\widetilde{z}}{R}}{x}}}
		{\hby{\ell_1}}
		{\Delta_1'}{\newsp{\widetilde{m_1'}}{P_1 \subst{\abs{\widetilde{z}}{R}}{x} \Par P_2' \subst{\abs{\widetilde{z}}{R}}{x}}}
		{lem:subst_equiv11}
%
	\noi The case is similar to the first case of \lemref{app:lem:subst_equiv}.

	\noi \item Case: $P_2 = \appl{x}{\widetilde{n}}$ for some $\widetilde{n}$.
%
\[
	\mhorel	{\Gamma}{\Delta_1}{\newsp{\widetilde{m_1}}{P_1 \subst{\abs{\widetilde{z}}{R}}{x} \Par \appl{x}{\widetilde{n}} \subst{\abs{\widetilde{z}}{R}}{x}}}
		{\hby{\btau}}{\Delta_1'}{}{\newsp{\widetilde{m_1'}}{P_1 \subst{\abs{\widetilde{z}}{R}}{x} \Par R \subst{\widetilde{n}}{\widetilde{z}}}} %\appl{x}{\widetilde{n}} \subst{\abs{\widetilde{x}}{R'}}{x}}}
%		{\hby{\ell_1}}{\Delta_1'}{}{\newsp{\widetilde{m_1'}}{P_1 \subst{\abs{\widetilde{x}}{R}}{x} \Par \appl{x}{\widetilde{n}} \subst{\abs{\widetilde{x}}{R'}}{x}}}
\]
%
	\noi From the latter transition we get that:
%
	\nhorel{\Gamma}{\Delta_1}
	{\newsp{\widetilde{m_1}}{P_1 \subst{\auxtr{t}}{x} \Par \appl{x}{\widetilde{n}} \subst{\auxtr{t}}{x}}}
	{\hby{\btau} \hby{\bactinp{t}{\auxtr{t'}}}}
	{\Delta_1'}{\newsp{\widetilde{m_1}'}{P_1 \subst{\auxtr{t}}{x} \Par \appl{y}{\widetilde{n}} \subst{\auxtr{t'}}{y}}}
%	{\hby{\bactinp{t}{\auxtr{t'}}}}
%	{\Delta_1'}{\newsp{\widetilde{m_1}'}{P_1 \subst{\auxtr{t}}{x} \Par \appl{y}{\widetilde{n}} \subst{\abs{\widetilde{x}}{\binp{t'}{y} \appl{y}{\widetilde{x}}}}{y}}}
	{cor:subst_equiv5}
%
%	\begin{eqnarray}
%		\Gamma; \es; \Delta_1 &\hby{\bactinp{t}{\auxtr{t'}}}& \Delta_1' \proves
%		\newsp{\widetilde{m_1}}{P_1 \subst{\auxtr{t}}{X} \Par \appl{X}{\widetilde{n}} \subst{\auxtr{t}}{X}}
%		\nonumber \\
%		&\hby{\bactinp{t}{\auxtr{t'}}}& 
%		\newsp{\widetilde{m_1}'}{P_1 \subst{\auxtr{t}}{X} \Par \appl{Y}{\widetilde{n}} \subst{\auxtr{t'}}{Y}}
%		\label{cor:subst_equiv5}
%	\end{eqnarray}
%
	\noi and $t'$ a fresh name. From the freshness of $t$
	and the determinacy of the application transition
	it has to be the case that:
%
	\[
		\begin{array}{crll}
			& \Gamma; \es; \Delta_2'& \proves &
			\newsp{\widetilde{m_2}'}{Q_1 \subst{\auxtr{t}}{x} \Par Q_2 \subst{\auxtr{t}}{x}}\\
			\Hby{} && &
			\newsp{\widetilde{m_2}'}{Q_1' \subst{\auxtr{t}}{x} \Par Q_2' \subst{\auxtr{t}}{x} \Par \\
			&&& \qquad \qquad \appl{x}{\widetilde{m}} \subst{\auxtr{t}}{x}} \\
			\hby{\btau} \hby{\bactinp{t}{\auxtr{t'}}}
			& \Delta_2''& \proves& \newsp{\widetilde{m_2}'}{(Q_1' \Par Q_2') \subst{\auxtr{t}}{x} \Par \appl{y}{\widetilde{m}} \subst{\auxtr{t'}}{y}}
%			\hby{\bactinp{t}{\abs{\widetilde{x}}{\binp{t'}{y} \appl{y}{\widetilde{x}}}}}
%			& \Delta_2''& \proves& \newsp{\widetilde{m_2}'}{(Q_1' \Par Q_2') \subst{\auxtr{t}}{x} \Par \appl{y}{\widetilde{m}} \subst{\abs{\widetilde{x}}{\binp{t'}{y} \appl{y}{\widetilde{x}}}}{y}}
		\end{array}
	\]
%
%	\[
%		\begin{array}{rcll}
%			\Gamma; \es; \Delta_2' &\Hby{\bactinp{t}{\auxtr{t'}}}& \Delta_2'' \proves &
%			\newsp{\widetilde{m_2}'}{Q_1 \subst{\auxtr{t}}{X} \Par Q_2 \subst{\auxtr{t}}{X}}\\
%			&\Hby{}& &
%			\newsp{\widetilde{m_2}'}{Q_1' \subst{\auxtr{t}}{X} \Par Q_2' \subst{\auxtr{t}}{X} \Par \appl{X}{\widetilde{m}} \subst{\auxtr{t}}{X}} \\
%			&\hby{\bactinp{t}{\auxtr{t'}}}& &
%			\newsp{\widetilde{m_2}'}{(Q_1' \Par Q_2') \subst{\auxtr{t}}{X} \Par \appl{Y}{\widetilde{m}} \subst{\auxtr{t'}}{Y}}
%		\end{array}
%	\]
%
	Let $Q_3$ such that
	\[
		\mhorel{\Gamma}{\Delta}{\newsp{\widetilde{m_2}'}{Q_1 \Par Q_3} \subst{\auxtr{t}}{x} \subst{\auxtr{t'}}{y}}
		{\Hby{}}
		{\Delta'}{}{\newsp{\widetilde{m_2}'}{(Q_1' \Par Q_2') \subst{\auxtr{t}}{x} \Par \appl{y}{\widetilde{m}} \subst{\auxtr{t'}}{y}}}
%		\mhorel{\Gamma}{\Delta}{\newsp{\widetilde{m_2}'}{Q_1 \Par Q_3} \subst{\auxtr{t}}{x} \subst{\abs{\widetilde{x}}{\binp{t'}{y} \appl{y}{\widetilde{x}}}}{y}}
%		{\Hby{}}
%		{\Delta'}{}{\newsp{\widetilde{m_2}'}{(Q_1' \Par Q_2') \subst{\auxtr{t}}{x} \Par \appl{y}{\widetilde{m}} \subst{\abs{\widetilde{x}}{\binp{t'}{y} \appl{y}{\widetilde{x}}}}{y}}}
	\]
%
	\noi From \lemref{app:lem:subst_equiv} we get that $\forall R$ with $\fv{R} = \widetilde{z}$
%
	\begin{eqnarray*}
		\mhorel{\Gamma}{\Delta_1'''}{\newsp{\widetilde{m_1}'}{P_1 \subst{\auxtr{t}}{x} \Par \appl{y}{\widetilde{n}} \subst{\abs{\widetilde{z}}{R}}{y}}}
		{\hwb}
		{\Delta'}{}{\newsp{\widetilde{m_2}'}{(Q_1 \Par Q_3) \subst{\auxtr{t}}{x} \subst{\abs{\widetilde{z}}{R}}{y}}}
%		\mhorel{\Gamma}{\Delta_1'''}{\newsp{\widetilde{m_1}'}{P_1 \subst{\auxtr{t}}{X} \Par \appl{y}{\widetilde{n}} \subst{\abs{\widetilde{x}}{R}}{y}}}
%		{\hwb}
%		{\Delta'}{}{\newsp{\widetilde{m_2}'}{(Q_1 \Par Q_3) \subst{\auxtr{t}}{X} \subst{\abs{\widetilde{x}}{R}}{y}}}
	\end{eqnarray*}
%
	\noi From~\eqref{lem:subst_equiv11} we get that
\[
	\mhorel{\Gamma}{\Delta'}{\newsp{\widetilde{m_1}'}{(Q_1 \Par Q_3) \subst{\auxtr{t}}{x} \subst{\abs{\widetilde{z}}{R}}{y}}}
	{\Hby{} \hby{\btau}}
	{\Delta''}{}{\newsp{\widetilde{m_2}'}{(Q_1' \Par Q_2') \subst{\auxtr{t}}{x} \Par R \subst{\widetilde{m}}{\widetilde{z}}}}%\appl{Y}{\widetilde{m}} \subst{\auxtr{t'}}{Y}}}
%	{\Hby{\ell_2}}
%	{\Delta''}{}{\newsp{\widetilde{m_2}'}{(Q_1' \Par Q_2') \subst{\auxtr{t}}{x} \Par R'}}%\appl{Y}{\widetilde{m}} \subst{\abs{\widetilde{x}}{\binp{t'}{Y} \appl{Y}{\widetilde{x}}}}{Y}}}
\]
	\noi and from the definition of $\Re$
%	\noi From \lemref{app:lem:subst_equiv} we get that $\forall R$ with $\fv{R} = x$
%
	\[
		\mhorel{\Gamma}{\Delta_1''}{\newsp{\widetilde{m_1}'}{P_1 \subst{\abs{\widetilde{z}}{R}}{x} \Par \appl{y}{\widetilde{n}} \subst{\abs{\widetilde{z}}{R}}{y}}}
		{\hby{\btau}\ \Re\ \stackrel{\btau}{\longleftarrow}}
		{\Delta_2''}{}{\newsp{\widetilde{m_2}'}{(Q_1' \Par Q_2') \subst{\abs{\widetilde{z}}{R}}{x} \Par \appl{y}{\widetilde{m}} \subst{\abs{\widetilde{z}}{R}}{y}}}
%		\mhorel{\Gamma}{\Delta_1''}{\newsp{\widetilde{m_1}'}{P_1 \subst{\abs{\widetilde{x}}{R}}{x} \Par \appl{Y}{\widetilde{n}} \subst{\abs{\widetilde{x}}{R'}}{y}}}
%		{\ \mathcal{S}\ }
%		{\Delta_2''}{}{\newsp{\widetilde{m_2}'}{(Q_1' \Par Q_2') \subst{\abs{\widetilde{x}}{R}}{x} \Par \appl{y}{\widetilde{m}} \subst{\abs{\widetilde{x}}{R'}}{y}}}
	\]
	\noi as required.
%	\noi From here we apply \lemref{app:lem:subst_equiv} for each substituting instance of
%	abstraction $\abs{\widetilde{x}}{R}$ to complete the proof.
\end{enumerate}
	\qed
\end{proof}

%%%%%%%%%%%%%%%%%%%%%%%%%%%%%%%%%%%%%%%%%%%%%%%%%%%%%%%%%
%  WB IS WBC
%%%%%%%%%%%%%%%%%%%%%%%%%%%%%%%%%%%%%%%%%%%%%%%%%%%%%%%%%

%\jp{Below we should be consistent and describe an $\Re$ that we use as closure.}
\begin{lemma}
	\label{app:lem:wb_is_wbc}
	$\hwb\ \subseteq\ \wbc$.
\end{lemma}

\begin{proof}
	Let $\Re$ be the typed relation (for readability, we omit typing judgements from the definition):
	\[
		\Re = \set{(P_1, Q_1) \setbar \horel{\Gamma}{\Delta_1}{P_1}{\hwb}{\Delta_2}{Q_1}}
	\]
	We show that $\Re$ is a contextual bisimulation.
	The proof is divided on cases on the label $\ell$ for the transition:
%
	\begin{eqnarray}
		\horel{\Gamma}{\Delta_1}{P_1}{\hby{\ell}}{\Delta_1'}{P_2}
		\label{lem:wb_is_wbc1}
	\end{eqnarray}
%
We distinguish four cases: $\ell$ is not an output or an input action; $\ell$ is an input action;
$\ell$ is an higher-order output; $\ell$ is a first-order output.
	\begin{enumerate}
		\item
				Case $\ell \notin \set{ \news{\widetilde{m_1}} \bactout{n}{\abs{\widetilde{x}}{P}},  \news{\widetilde{m_1}'} \bactout{n}{\widetilde{m_1}}, \bactinp{n}{\abs{\widetilde{x}}{P}} }$:

				\noi For the latter $\ell$ and transition in \eqref{lem:wb_is_wbc1} we conclude that:	
			%
				\[
					\horel{\Gamma}{\Delta_2}{Q_1}{\Hby{\ell}}{\Delta_2'}{Q_2}
				\]
			%
				\noi and
			%
				\[
					\horel{\Gamma}{\Delta_1'}{P_2}{\hwb}{\Delta_2'}{Q_2}
				\]
			%
				The above premise and conclusion coincides with defining cases for $\ell$ in $\wbc$.

		\item	Case $\ell = \bactinp{n}{\abs{\widetilde{x}}{P}}$:

				\noi Transition in \eqref{lem:wb_is_wbc1} concludes:
			%
				\[
					\begin{array}{l}
						\horel{\Gamma}{\Delta_1}{P_1}{\hby{\bactinp{n}{\abs{\widetilde{x}}{\mapchar{U}{\widetilde{x}}}}}}{\Delta_1'}{P_2 \subst{\abs{\widetilde{x}}{\mapchar{U}{\widetilde{x}}}}{x}}\\
						\horel{\Gamma}{\Delta_1}{P_1}{\hby{\bactinp{n}{\auxtr{t}}}}{\Delta_1''}{P_2 \subst{\auxtr{t}}{x}}
					\end{array}
				\]
			%
				\noi The last two transitions imply:
			%
			\[
				\begin{array}{l}
					\horel{\Gamma}{\Delta_2}{Q_1}{\Hby{\bactinp{n}{\abs{\widetilde{x}}{\mapchar{U}{\widetilde{x}}}}}}{\Delta_2'}{Q_2 \subst{\abs{\widetilde{x}}{\mapchar{U}{\widetilde{x}}}}{x}}\\
					\horel{\Gamma}{\Delta_2}{Q_1}{\Hby{\bactinp{n}{\auxtr{t}}}}{\Delta_2''}{Q_2 \subst{\auxtr{t}}{x}}
				\end{array}
			\]
			%
				\noi and
			%
			\[
				\begin{array}{l}
					\horel{\Gamma}{\Delta_1'}{P_2 \subst{\abs{\widetilde{x}}{\mapchar{U}{\widetilde{x}}}}{x}}{\hwb}{\Delta_2'}{Q_2 \subst{\abs{\widetilde{x}}{\mapchar{U}{\widetilde{x}}}}{x}}\\
					\horel{\Gamma}{\Delta_1''}{P_2 \subst{\auxtr{t}}{x}}{\hwb}{\Delta_2''}{Q_2 \subst{\auxtr{t}}{x}}
				\end{array}
			\]
			%
				\noi To conclude from (\lemref{app:lem:proc_subst}) that
				$\forall P$ with $\fv{P} = \widetilde{x}$
			%
			\[
				\horel{\Gamma}{\Delta_1'}{P_2 \subst{\abs{\widetilde{x}}{P}}{x}}{\hwb}{\Delta_2'}{Q_2 \subst{\abs{\widetilde{x}}{P}}{x}}
			\]
			%
				\noi as required.

		\item	Case $\ell = \news{\widetilde{m_1}} \bactout{n}{\abs{\widetilde{x}}{P}}$:

				\noi From transition \eqref{lem:wb_is_wbc1} we conclude:
			%
			\[
				\horel{\Gamma}{\Delta_2}{Q_1}{\Hby{\news{\widetilde{m_2}} \bactout{n}{\abs{\widetilde{x}}{Q}}}}{\Delta_2'}{Q_2}
			\]
			%
				\noi and for fresh $t$
			%
			\[
				\mhorel	{\Gamma}{\Delta_1'}{\newsp{\widetilde{m_1}}{P_2 \Par \binp{t}{x} \newsp{s}{\appl{x}{s} \Par \bout{\dual{s}}{\abs{\widetilde{x}}{P}} \inact}}}
					{\hwb}
					{\Delta_2'}{}{\newsp{\widetilde{m_2}}{Q_2 \Par \binp{t}{x} \newsp{s}{\appl{x}{s} \Par \bout{\dual{s}}{\abs{\widetilde{x}}{Q}} \inact}}}
			\]
			%
				\noi From the  second case of this proof we can conclude that $\forall R$ with $\fpv{R} = \set{x}$:
			%
			\[
				\begin{array}{rl}
					\Gamma; \es; &\Delta_1' \proves \newsp{\widetilde{m_1}}{P_2 \Par \binp{t}{x} \newsp{s}{\appl{x}{s} \Par \bout{\dual{s}}{\abs{\widetilde{x}}{P}} \inact}} \\
					\by{\bactinp{t}{\abs{z}{\binp{z}{x} R}}}& \newsp{\widetilde{m_1}}{P_2 \Par \newsp{s}{\binp{s}{x} R \Par \bout{\dual{s}}{\abs{\widetilde{x}}{P}} \inact}}\\
					\by{\tau} \quad &\Delta_1'' \proves \newsp{\widetilde{m_1}}{P_2 \Par  R \subst{\abs{\widetilde{x}}{P}}{x}}
				\end{array}
			\]
			%
				\noi and
			%
			\[
				\begin{array}{rl}
					\Gamma; \es; &\Delta_2' \proves \newsp{\widetilde{m_2}}{Q_2 \Par \binp{t}{x} \newsp{s}{\appl{x}{s} \Par \bout{\dual{s}}{\abs{\widetilde{x}}{Q}} \inact}} \\
					\by{\bactinp{t}{\abs{z}{\binp{z}{x} R}}} &\newsp{\widetilde{m_2}}{Q_2 \Par \newsp{s}{\binp{s}{x} R \Par \bout{\dual{s}}{\abs{\widetilde{x}}{Q}} \inact}}\\
					\by{\tau} &\Delta_2'' \proves \newsp{\widetilde{m_2}}{Q_2 \Par  R \subst{\abs{\widetilde{x}}{Q}}{x}}
				\end{array}
			\]
			%
				\noi and furthermore it is easy to see that $\forall R$ with $\fpv{R} = \set{x}$:
			%
			\[
				\horel{\Gamma}{\Delta_1''}{\newsp{\widetilde{m_1}}{P_2 \Par  R \subst{\abs{\widetilde{x}}{P}}{x}}}{\hwb}{\Delta_2}{\newsp{\widetilde{m_2}}{Q_2 \Par R \subst{\abs{\widetilde{x}}{Q}}{x}}}
			\]
			%
				\noi as required by the definition of $\wbc$.

		\item	Case $\ell = \news{\widetilde{m_1}'} \bactout{n}{\widetilde{m_1}}$: 

				This case shares a similar argumentation with the previous case.
	\end{enumerate}
	\qed
\end{proof}


%%%%%%%%%%%%%%%%%%%%%%%%%%%%%%%%%%%%%%%%%%%%%%%%%%%%%%%%%
%  WBC IS CONG
%%%%%%%%%%%%%%%%%%%%%%%%%%%%%%%%%%%%%%%%%%%%%%%%%%%%%%%%%

\begin{lemma}
	\label{app:lem:wbc_is_cong}
	$\wbc \subseteq \cong$.
\end{lemma}


\begin{proof}
	\noi We prove that $\wbc$ satisfies the three defining properties of $\cong$:
	reduction closure, barb preservation, congruence (cf. \defref{def:rc}).
%

\noi	{\bf Reduction Closed:}
	Let
		$\horel{\Gamma}{\Delta_1}{P_1}{\wbc}{\Delta_2}{P_2}$. The reduction

	\[
		\horel{\Gamma}{\Delta_1}{P_1}{\by{}}{\Delta_1'}{P_1'}
	\]
%
	\noi implies that 
	$\exists P_2'$ such that 
%
	\begin{eqnarray*}
		\horel{\Gamma}{\Delta_2}{P_2}{\By{}}{\Delta_2'}{P_2'}\\
		\horel{\Gamma}{\Delta_1}{P_1'}{\wbc}{\Delta_2'}{P_2'}
	\end{eqnarray*}
%
	\noi Same arguments hold for the symmetric case, thus $\wbc$ is reduction closed.

	\noi {\bf Barb Preservation:} We have that
%
	\begin{eqnarray*}
		\Gamma; \emptyset; \Delta_1 \proves P_1 \barb{n}
	\end{eqnarray*}
%
	implies that
	\begin{eqnarray*}
		P &\cong& \newsp{\widetilde{m}}{\bout{n}{V_1} P_3 \Par P_4}\\
		\dual{n} &\notin& \Delta_1
	\end{eqnarray*}
%
	\noi From the definition of $\wbc$ we get that
%
\[
	\horel	{\Gamma}{\Delta_1}{\newsp{\widetilde{m}}{\bout{n}{V_1} P_3 \Par P_4}}
		{\by{\news{s_1} \bactout{n}{V_1}}}
		{\Delta_1'}
		{\newsp{\widetilde{m'}}{P_3 \Par P_4}}
\]
%
	\noi implies
%
	\begin{eqnarray*}
		\horel{\Gamma}{\Delta_2}{P_2}{\By{\news{m_2} \bactout{n}{V_2}}}{\Delta_2'}{P_2'}\\
	\end{eqnarray*}
%
	\noi From the last result we obtain
%
	\begin{eqnarray*}
		\Gamma; \emptyset; \Delta_2 \proves P_2 \Barb{n}
	\end{eqnarray*}
%
	\noi as required.

	\noi {\bf Congruence:}

	\noi The congruence property requires that we check that $\wbc$
	is preserved under any context.
	The most interesting context case is parallel composition.

	\noi We construct a congruence relation. Let
	\[
	\begin{array}{rcl}
		\mathcal{S} &=&	\set{
				(\Gamma; \emptyset; \Delta_1 \cat \Delta_3 \proves \newsp{\widetilde{n_1}}{P_1 \Par R} \hastype \Proc,
				\Gamma; \emptyset; \Delta_2 \cat \Delta_3 \proves \newsp{\widetilde{n_2}}{P_2 \Par R})
				\setbar \\
		& &		\horel{\Gamma}{\Delta_1}{P_1}{\wbc}{\Delta_2}{P_2}, \forall \Gamma; \emptyset; \Delta_3 \proves R \hastype \Proc\\
		& &}
	\end{array}
	\]
	\noi We need to show that 
	%the above congruence is a bisimulation.
	%To show that 
	$\mathcal{S}$ is a bisimulation:  we do a case analysis on the structure
	of the $\by{\ell}$ transition. There are three main cases.


	\begin{enumerate}
	%%%%%%%%%%%%%%%
	% Case 1
	%%%%%%%%%%%%%%%

		\item Suppose
				%
				\[
					\horel{\Gamma}{\Delta_1 \cat \Delta_3}{\newsp{\widetilde{n_1}}{P_1 \Par R}}
					{\by{\ell}}
					{\Delta_1' \cat \Delta_3}{\newsp{\widetilde{n_1'}}{P_1' \Par R}}
				\]
				%
				\noi The case is divided into three subcases:

				\begin{enumerate}[i.]
					\item Sub-case	$\ell \notin \set{\news{\widetilde{m}} \bactout{n}{\abs{\widetilde{x}}{Q}}, \news{\widetilde{mm_1}} \bactout{n}{\widetilde{m_1}}}$:
					
							\noi From the definition of typed transition we get:
					%
							\[
								\horel{\Gamma}{\Delta_1}{P_1}{\by{\ell}}{\Delta_1'}{P_1'}
							\]
							\noi which implies that
							\begin{eqnarray}
								\horel{\Gamma}{\Delta_1}{P_2}{\By{\ell}}{\Delta_2'}{P_2'}
								\label{lem:wbc_is_cong1}\\
								\horel{\Gamma}{\Delta_1'}{P_1'}{\wbc}{\Delta_2''}{P_2'}
								\label{lem:wbc_is_cong2}
							\end{eqnarray}
					%
							\noi From transition in~\eqref{lem:wbc_is_cong1} we conclude that 
							\[
								\horel{\Gamma}{\Delta_2 \cat \Delta_3}{\newsp{\widetilde{n_2}}{P_2 \Par R}}
								{\By{\ell}}
								{\Delta_2' \cat \Delta_3}{\newsp{\widetilde{n_2}'}{P_2' \Par R}}
							\]
					%
							\noi Furthermore from \eqref{lem:wbc_is_cong2} and the definition of $\mathcal{S}$ we conclude that
					%
							\[
								\horel{\Gamma}{\Delta_1' \cat \Delta_3}{\newsp{\widetilde{n_1}'}{P_1' \Par R}}
								{\ \mathcal{S}\ }
								{\Delta_2' \cat \Delta_3}{\newsp{\widetilde{n_2}'}{P_2' \Par R}}
							\]

					\item	Sub-case $\ell = \news{\widetilde{m_1}} \bactout{n}{\abs{\widetilde{x}}{Q_1}}$:

							\noi From the definition of typed transition we get
							\[
								\horel{\Gamma}{\Delta_1}{P_1}
								{\by{\news{\widetilde{m_1}} \bactout{n}{\abs{\widetilde{x}}{Q_1}}}}
								{\Delta_1'}{P_1'}
							\]
							\noi which implies that
					%
							\begin{eqnarray}
								&& \horel{\Gamma}{\Delta_1}{P_2}{\By{\news{\widetilde{m_2}} \bactout{n}{\abs{\widetilde{x}}{Q_2}}}}{\Delta_2'}{P_2'}
								\label{lem:wbc_is_cong3} \\
								&&\forall Q, \set{x} \in \fpv{Q} \nonumber \\
						%		\forall s'
								&& \horel{\Gamma}{\Delta_1''}{\newsp{\widetilde{n_1}''}{P_1' \Par Q \subst{\abs{\widetilde{x}}{Q_1}}{x}}}
								{\ \wbc\ }
								{\Delta_2''}{\newsp{\widetilde{n_2}''}{P_2' \Par Q \subst{\abs{\widetilde{x}}{Q_2}}{x}}}
								\label{lem:wbc_is_cong4}
							\end{eqnarray}
					%
							\noi From transition~\eqref{lem:wbc_is_cong3} conclude that 
							\[
								\horel{\Gamma}{\Delta_2 \cat \Delta_3}{\newsp{\widetilde{n_2}}{P_2 \Par R}}
								{\By{\news{\widetilde{m_2}} \bactout{n}{\abs{\widetilde{x}}{Q_2}}}}
								{\Delta_2' \cat \Delta_3}{\newsp{\widetilde{n_2}'}{P_2' \Par R}}
							\]
					%
							\noi Furthermore from~\eqref{lem:wbc_is_cong4} we conclude that $\forall Q$ with $\set{x} = \fpv{Q}$
					%
							\[
								\horel{\Gamma}{\Delta_1'' \cat \Delta_3}{\newsp{\widetilde{n_1}''}{P_1' \Par Q \subst{(\widetilde{x}) Q_1}{x} \Par R}}
								{\ \mathcal{S}\ }
								{\Delta_2'' \cat \Delta_3}{\newsp{\widetilde{n_2}''}{P_2' \Par Q \subst{\abs{\widetilde{x}}{Q_2}}{x} \Par R}}
							\]
					%

					\item	Sub-case $\ell = \news{\widetilde{mm_1}} \bactout{n}{\widetilde{m_1}}$:

							\noi From the definition of typed transition we get that
							\[
								\horel{\Gamma}{\Delta_1}{P_1}
								{\by{\news{\widetilde{mm_1}} \bactout{n}{\widetilde{m_1}}}}
								{\Delta_1'}{P_1'}
							\]
							\noi which implies that $\exists P_2', s_2$ such that
					%
							\begin{eqnarray}
								&& \horel{\Gamma}{\Delta_1}{P_2}
								{\By{\news{\widetilde{mm_2}} \bactout{n}{\widetilde{m_2}}}}
								{\Delta_2'}{P_2'}
								\label{lem:wbc_is_cong5}\\
								&&\forall Q, x = \fn{Q}, \nonumber \\%  &&
								&& \horel{\Gamma}{\Delta_1''}{\newsp{\widetilde{n_1}}{P_1' \Par Q \subst{\widetilde{m_1}}{\widetilde{x}}}}
								{\ \wbc\ }
								{\Delta_2''}{\newsp{\widetilde{n_2}}{P_2' \Par Q \subst{\widetilde{m_2}}{\widetilde{x}}}}
								\label{lem:wbc_is_cong6}
							\end{eqnarray}
					%
						\noi From transition~\eqref{lem:wbc_is_cong5} conclude that 
						\[
							\horel{\Gamma}{\Delta_2 \cat \Delta_3}{\newsp{\widetilde{n_2}'}{P_2 \Par R}}
							{\By{\news{\widetilde{mm_2}} \bactout{n}{\widetilde{m_2}}}}
							{\Delta_2' \cat \Delta_3}{\newsp{\widetilde{n_2}'''}{P_2' \Par R}}
						\]
					%
						\noi Furthermore from~\eqref{lem:wbc_is_cong6} we conclude that $\forall Q, x = \fn{Q}$
					%
						\[
							\horel{\Gamma}{\Delta_1'' \cat \Delta_3}{\newsp{\widetilde{n_1}''}{P_1' \Par Q \subst{\widetilde{m_1}}{\widetilde{x}} \Par R}}
							{\ \mathcal{S}\ }
							{\Delta_2'' \cat \Delta_3}{\newsp{\widetilde{n_2}''}{P_2' \Par Q \subst{\widetilde{m_2}}{\widetilde{x}} \Par R}}
						\]
					%
				\end{enumerate}
	%%%%%%%%%%%%%%%
	% Case 2
	%%%%%%%%%%%%%%%

		\item Suppose
			%
				\[
					\horel{\Gamma}{\Delta_1 \cat \Delta_3}{\newsp{\widetilde{m_1}}{P_1 \Par R}}
					{\by{\ell}}
					{\Delta_1 \cat \Delta_3'}{\newsp{\widetilde{m_1}'}{P_1 \Par R'}}
				\]
				\noi This case is divided into three subcases:

				\begin{enumerate}[i.]
			%
					\item Sub-case 	$\ell \notin \set{\news{\widetilde{m}} \bactout{n}{\abs{\widetilde{x}}{Q}}, \news{\widetilde{mm_1}} \bactout{n}{\widetilde{m_1}}}$:

							\noi From the LTS we get that:
							\[
								\horel{\Gamma}{\Delta_3}{R}{\by{\ell}}{\Delta_3'}{R'}
							\]
						%
							\noi Which in turn implies
							\begin{eqnarray*}
								\horel{\Gamma}{\Delta_2 \cat \Delta_3}{\newsp{\widetilde{m_2}}{P_2 \Par R}}
								{\by{\ell}}
								{\Delta_2 \cat \Delta_3'}{\newsp{\widetilde{m_2}'}{P_2 \Par R'}}
							\end{eqnarray*}
						%
							\noi From the definition of $\mathcal{S}$ we conclude that
							\[
								\horel{\Gamma}{\Delta_1 \cat \Delta_3'}{\newsp{\widetilde{m_1}'}{P_1 \Par R'}}
								{\ \mathcal{S}\ }
								{\Delta_2 \cat \Delta_3''}{\newsp{\widetilde{m_2}'}{P_2 \Par R'}}
							\]
							\noi as required.

				\item	Sub-case $\ell = \news{\widetilde{m_1}} \bactout{n}{\abs{\widetilde{x}}{Q}}$:

						\noi From the LTS we get that:
						\begin{eqnarray}
							& &	\horel{\Gamma}{\Delta_3}{R}{\by{\ell}}{\Delta_3'}{R'}
								\label{lem:wbc_is_cong7}\\
							& & 	\forall R_1, \set{x} = \fpv{R_1},
								\nonumber\\
					%		\forall s'
							& &	\Gamma; \emptyset; \Delta_3'' \proves \newsp{\widetilde{m}'}{R' \Par R_1 \subst{\abs{\widetilde{x}}{Q}}{x}} \hastype \Proc
								\label{lem:wbc_is_cong8}
						\end{eqnarray}
					%
						\noi From~\eqref{lem:wbc_is_cong7} we get that
						\[
							\horel{\Gamma}{\Delta_2 \cat \Delta_3}{\newsp{\widetilde{m_2}'}{P_2 \Par R}}{\by{\ell}}{\Delta_2 \cat \Delta_3'}{\newsp{\widetilde{m_2}}{P_2 \Par R'}}
						\]
						\noi Furthermore from~\eqref{lem:wbc_is_cong8} and the definition of $\mathcal{S}$ we conclude that
						$\forall R_1$ with $\set{x} \in \fpv{R_1}$
						\[
							\horel{\Gamma}{\Delta_1 \cat \Delta_3''}{\newsp{\widetilde{m_1}}{P_1 \Par \newsp{\widetilde{m}'}{R' \Par R_1 \subst{\abs{\widetilde{x}}{Q}}{x}}}}
							{\ \mathcal{S}\ }
							{\Delta_2 \cup \Delta_3''}{\newsp{\widetilde{m_2}}{P_2 \Par \newsp{\widetilde{m}'}{R' \Par R_1 \subst{\abs{\widetilde{x}}{Q}}{x}}}}
						\]
						\noi as required.

				\item	Sub-case $\ell = \news{\widetilde{mm}} \bactout{n}{\widetilde{m}}$:

					\noi From the typed LTS we get that:
					\begin{eqnarray}
						& &	\horel{\Gamma}{\Delta_3}{R}{\by{\ell}}{\Delta_3'}{R'}
							\label{lem:wbc_is_cong9} \\
						& &	\forall Q, \widetilde{x} = \fn{Q}, \nonumber\\
						& &	\Gamma; \emptyset; \Delta_3'' \proves \newsp{\widetilde{m}'}{R' \Par Q \subst{\widetilde{m}}{\widetilde{x}}} \hastype \Proc
							\label{lem:wbc_is_cong10}
					\end{eqnarray}
				%
					\noi From~\eqref{lem:wbc_is_cong9}, we obtain that
					\[
						\horel{\Gamma}{\Delta_2 \cat \Delta_3}{\newsp{\widetilde{m_2}}{P_2 \Par R}}{\by{\ell}}{\Delta_2 \cat \Delta_3'}{\newsp{\widetilde{m_2}}{P_2 \Par R'}}
					\]
					\noi Furthermore from~\eqref{lem:wbc_is_cong10} and the definition of $\mathcal{S}$ we conclude that
					$\forall Q, \widetilde{x} = \fn{Q}$
					\[
						\horel{\Gamma}{\Delta_1 \cat \Delta_3''}{\newsp{\widetilde{m_1}}{P_1 \Par \newsp{\widetilde{m}}{R' \Par Q \subst{\widetilde{m}'}{\widetilde{x}}}}}
						{\ \mathcal{S}\ }
						{\Delta_2 \cat \Delta_3''}{\newsp{\widetilde{m_2}}{P_2 \Par \newsp{\widetilde{m}'}{R' \Par Q \subst{\widetilde{m}}{\widetilde{x}}}}}
					\]
					\noi as required.
			\end{enumerate}

	%%%%%%%%%%%%%%%
	% Case 3
	%%%%%%%%%%%%%%%

	\item For the last case, suppose:
			\[
				\horel{\Gamma}{\Delta_1 \cat \Delta_3}{\newsp{\widetilde{m_1}}{P_1 \Par R}}
				{\by{\tau}}
				{\Delta_1' \cat \Delta_3'}{\newsp{\widetilde{m_1}'}{P_1' \Par R'}}
			\]

			\noi This case is divided into three subcases:

			\begin{enumerate}[i.]

				\item	$\horel{\Gamma}{\Delta_1}{P_1}{\by{\ell}}{\Delta_1'}{P_1'}$
						and $\ell \notin \set{\news{\widetilde{m}} \bactout{n}{\abs{\widetilde{x}}{Q}}, \news{\widetilde{mm_1}} \bactout{n}{\widetilde{m_1}}}$
						implies
					%
						\begin{eqnarray}
							\horel{\Gamma}{\Delta_3}{R}{\by{\dual{\ell}}}{\Delta_3}{R'}
							\label{lem:wbc_is_cong11} \\
							\horel{\Gamma}{\Delta_2}{P_2}{\By{\hat{\ell}}}{\Delta_2'}{P_2'}
							\label{lem:wbc_is_cong12}\\
							\horel{\Gamma}{\Delta_1'}{P_1'}{\wbc}{\Delta_2'}{P_2'}
							\label{lem:wbc_is_cong13}
						\end{eqnarray}
					%
						\noi From~\eqref{lem:wbc_is_cong11} and~\eqref{lem:wbc_is_cong12} we get
						\[
							\horel{\Gamma}{\Delta_2 \cat \Delta_3}{\newsp{\widetilde{m_2}}{P_2 \Par R}}
							{\By{}}
							{\Delta_2' \cat \Delta_3'}{\newsp{\widetilde{m_2}'}{P_2' \Par R'}}
						\]
					%
						\noi From~\eqref{lem:wbc_is_cong13} and the definition of ($\mathcal{S}$) we get that
						\[
							\horel{\Gamma}{\Delta_1' \cat \Delta_3'}{\newsp{\widetilde{m_1}'}{P_1' \Par R'}}
							{\ \mathcal{S}\ }
							{\Delta_2' \cat \Delta_3}{\newsp{\widetilde{m_2}'}{P_2' \Par R'}}
						\]
						\noi as required.

				\item
						$\horel{\Gamma}{\Delta_1}{P_1}{\by{\news{\widetilde{m_1}} \bactout{n}{\abs{\widetilde{x}}{Q_1}}}}{\Delta_1'}{P_1'}$
						implies
					%
						\begin{eqnarray}
							& & \horel{\Gamma}{\Delta_3}{R}
							{\by{\bactinp{n}{\abs{\widetilde{x}} {Q_1}}}}{\Delta_3'}
							{R' \subst{\abs{\widetilde{x}}{Q_1}}{x}}
							\label{lem:wbc_is_cong14}\\
							& & \horel{\Gamma}{\Delta_1 \cat \Delta_3}{\newsp{\widetilde{m_1}}{P_1 \Par R}}
							{\by{}}{\Delta_1' \cat \Delta_3'}
							{\newsp{\widetilde{m_1}''}{P_1' \Par R' \subst{\abs{\widetilde{x}}{Q_1}}{x}}}
							\nonumber \\
							& & \horel{\Gamma}{\Delta_2}{P_2}
							{\By{\news{\widetilde{m_2}} \bactout{n}{\abs{\widetilde{x}}{Q_2}}}}
							{\Delta_2'}{P_2'}
							\label{lem:wbc_is_cong15}\\
							& & \forall Q, \set{x} = \fpv{Q}, \nonumber \\
							& & \horel{\Gamma}{\Delta_1''}{\newsp{\widetilde{m_1}'}{P_1' \Par Q \subst{\abs{\widetilde{x}}{Q_1}}{x}}}
							{\ \wbc\ }
							{\Delta_2''}{\newsp{\widetilde{m_2}'}{P_2' \Par Q \subst{\abs{\widetilde{x}}{Q_2}}{x}}}
							\label{lem:wbc_is_cong16}
						\end{eqnarray}
					%
						From~\eqref{lem:wbc_is_cong14} and the Substitution Lemma~(\lemref{l:subst}) we obtain that
						\[
							\horel{\Gamma}{\Delta_3}{R}{\by{\bactinp{n}{\abs{\widetilde{x}} {Q_2}}}}{\Delta_3''}{R' \subst{\abs{\widetilde{x}}{Q_2}}{x}}
						\]
						%\dk{(prove that $\forall V, R \by{\bactinp{s}{V}} R'\subst{V}{x}$)}
						\noi to combine with~\eqref{lem:wbc_is_cong15} and get
						\[
							\horel{\Gamma}{\Delta_2 \cat \Delta_3}{\newsp{\widetilde{m_2}}{P_2 \Par R}}
							{\By{}}
							{\Delta_2' \cat \Delta_3''}{\newsp{\widetilde{m_2}''}{P_2' \Par R' \subst{\abs{\widetilde{x}}{Q_2}}{X}}}
						\]
					%
						\noi In result in~\eqref{lem:wbc_is_cong16}, set $Q$ as $R'$ to obtain:
					%
					%	\noi From~\eqref{lem:wbc_is_cong16} and the definition of $\mathcal{S}$ we get that
						\[
							\horel{\Gamma}{\Delta_1''}{\newsp{\widetilde{m_1}'}{P_1' \Par R' \subst{\abs{\widetilde{x}}{Q_1}}{x}}}
							{\ \mathcal{S}\ }{ \Delta_2''}
							{\newsp{\widetilde{m_2}'}{P_2' \Par R' \subst{\abs{\widetilde{x}}{Q_2}}{x}}}
						\]

				\item
						$\horel{\Gamma}{\Delta_1}{P_1}{\by{\news{\widetilde{mm_1}} \bactout{n}{\widetilde{m_1}}}}{\Delta_1'}{P_1'}$
					%
						\begin{eqnarray}
							& & \horel{\Gamma}{\Delta_3}{R}
							{\by{\bactinp{n}{\widetilde{m_1}}}}
							{\Delta_3'}{R' \subst{\widetilde{m_1}}{\widetilde{x}}}
							\label{lem:wbc_is_cong24}\\
							& & \horel{\Gamma}{\Delta_1 \cup \Delta_3}{\newsp{\widetilde{m_1}}{P_1 \Par R}}
							{\by{}}
							{\Delta_1' \cup \Delta_3'}{\newsp{\widetilde{m_1}''}{P_1' \Par R' \subst{m_1}{x}}}
							\nonumber \\
							& & \horel{\Gamma}{\Delta_2}{P_2}
							{\By{\news{\widetilde{mm_2}} \bactout{n}{\widetilde{m_2}}}}
							{\Delta_2'}{P_2'}
							\label{lem:wbc_is_cong25}\\
							& & \forall Q, \set{x} = \fpv{Q}, \nonumber \\
							& & \horel{\Gamma}{\Delta_1''}{\newsp{\widetilde{m_1}'}{P_1' \Par Q \subst{\widetilde{m_1}}{\widetilde{x}}}}
							{\ \wbc\ }
							{\Delta_2''}{\newsp{\widetilde{m_2}'}{P_2' \Par Q \subst{\widetilde{m_2}}{\widetilde{x}}}}
							\label{lem:wbc_is_cong26}
						\end{eqnarray}
					%
						From~\eqref{lem:wbc_is_cong24} and the Substitution Lemma~(\lemref{l:subst}) we get that
						\[
							\horel{\Gamma}{\Delta_3}{R}{\by{\bactinp{n}{\widetilde{m_2}}}}{\Delta_3''}{R' \subst{\widetilde{m_2}}{\widetilde{x}}}
						\]
						%\dk{(prove that $\forall V, R \by{\bactinp{s}{V}} R'\subst{V}{x}$)}
						\noi to combine with~\eqref{lem:wbc_is_cong25} and get
						\[
							\horel{\Gamma}{\Delta_2 \cat \Delta_3}{\newsp{\widetilde{m_2}}{P_2 \Par R}}
							{\By{}}
							{\Delta_2' \cat \Delta_3''}{\newsp{\widetilde{m_2}''}{P_2' \Par R' \subst{\widetilde{m_2}}{\widetilde{x}}}}
						\]
					%
						\noi Set $Q$ as $R'$ in result in \eqref{lem:wbc_is_cong26} to obtain
					%
					%	\noi From~\eqref{lem:wbc_is_cong16} and the definition of $\mathcal{S}$ we get that
						\[
							\horel{\Gamma}{\Delta_1''}{\newsp{\widetilde{m_1}'}{P_1' \Par R' \subst{\widetilde{m_1}}{\widetilde{x}}}}
							{\ \mathcal{S}\ }
							{\Delta_2''}{\newsp{\widetilde{m_2}'}{P_2' \Par R' \subst{\widetilde{m_2}}{\widetilde{x}}}}
						\]
		\end{enumerate}
	\end{enumerate}
	\qed
\end{proof}

%%%%%%%%%%%%%%%%%%%%%%%%%%%%%%%%%%%%%%%%%%%%%%%%%%%%%%%%%
%  CONG IS WB
%%%%%%%%%%%%%%%%%%%%%%%%%%%%%%%%%%%%%%%%%%%%%%%%%%%%%%%%%

We prove the result $\cong \subseteq \hwb$ following
the technique developed in~\cite{Hennessy07} and
refined for session types in~\cite{KYHH2015,KY2015}.

%\jp{Below I slightly modify the structure of items.}

\begin{definition}[Definability]\myrm
	\label{app:def:definibility}
	Let $\Gamma; \emptyset; \Delta_1 \proves P \hastype \Proc$.
	A visible action $\ell$ is \emph{definable} whenever
	there exists (testing) process
	$\Gamma; \emptyset; \Delta_2 \proves T\lrangle{\ell, \suc} \hastype \Proc$
	with $\suc$ fresh name % and $N$ a set of names.
	such that:
%
	\begin{enumerate}
		\item	Let $\ell \in \set{\bactsel{n}{\ell}, \bactbra{n}{\ell}, \bactinp{n}{\widetilde{m}}, \bactinp{n}{(\widetilde{x}) Q}}$.
		
			\begin{enumerate}[i.]
				\item	If $\horel{\Gamma}{\Delta_1}{P}{\hby{\ell}}{\Delta_1'}{P'}$
				%		and
				%		$\ell \in \set{\bactsel{n}{\ell}, \bactbra{n}{\ell}, \bactinp{n}{\widetilde{m}}, \bactinp{n}{\abs{\widetilde{x}}{Q}}}$
						then:
				%
						\[
							P \Par T\lrangle{\ell, \suc} \red P' \Par \bout{\suc}{\dual{n}} \inact \textrm{ and }
							\Gamma; \emptyset; \Delta_1' \cat \Delta_2' \proves P' \Par \bout{\suc}{\dual{n}} \inact
						\]

				\item
						If $P \Par T\lrangle{\ell, \suc} \red Q$ with			
						$\Gamma; \emptyset; \Delta \proves Q \barb{\suc}$ then \\
						$\horel{\Gamma}{\Delta_1}{P}{\Hby{\ell}}{\Delta_1'}{P'}$
						and $Q \scong P' \Par \bout{\suc}{\dual{n}} \inact$.
			\end{enumerate}
%
 		\item Let	(i) $\ell = \news{\widetilde{m}}\bactout{n}{V}$,
					and (ii) fresh $t$%, and
					%(iii) $\widetilde{m}'$ such that $ \widetilde{m}' \subseteq \widetilde{m}$

			\begin{enumerate}[i.]
				\item	If $\horel{\Gamma}{\Delta_1}{P}{\hby{\news{\widetilde{m}}\bactout{n}{V}}}{\Delta_1'}{P'}$
						then:
%
						\begin{itemize}
							\item $P \Par T\lrangle{\news{\widetilde{m}}\bactout{n}{V}, \suc} \red
							\newsp{\widetilde{m}}{P' \Par \hotrigger{t}{x}{s}{V} \Par \bout{\suc}{\dual{n}, V} \inact}$
							\item $\Gamma; \emptyset; \Delta_1' \cat \Delta_2' \proves
							\newsp{\widetilde{m}}{P' \Par \hotrigger{t}{x}{s}{V} \Par  \bout{\suc}{\dual{n}, V} \inact} \hastype \Proc$
						\end{itemize}

				\item	If $P \Par T\lrangle{\news{\widetilde{m}}\bactout{n}{V}, \suc} \red Q$
						with $\Gamma; \emptyset; \Delta \proves Q \barb{\suc}$ then 
						\begin{itemize}
							\item $\horel{\Gamma}{\Delta_1}{P}{\Hby{\news{\widetilde{m}}\bactout{n}{V}}}{\Delta_1'}{P'}$
							\item $Q \scong \newsp{\widetilde{m}}{P' \Par \hotrigger{t}{x}{s}{V} \Par \bout{\suc}{\dual{n}, V} \inact}$
						\end{itemize}
			\end{enumerate}
	\end{enumerate}	
%
\end{definition}

We first show that every visible action $\ell$ is definable.

\begin{lemma}[Definability]
	\label{lem:definibility}
	Every action $\ell$ is definable.
\end{lemma}

\begin{proof}
	\noi We define $T\lrangle{\ell, \suc}$:
	\begin{eqnarray*}
		T\lrangle{\bactinp{n}{V}, \suc} &=&
		\bout{\dual{n}}{V} \bout{\suc}{\dual{n}} \inact
		\\
		T\lrangle{\bactbra{n}{l}, \suc} &=&
		\bsel{\dual{n}}{l} \bout{\suc}{\dual{n}} \inact
		\\
%		T\lrangle{\news{\widetilde{m}} \bactout{n}{\widetilde{m}}, \suc} &=&
%		\binp{\dual{n}}{\widetilde{y}} (\hotrigger{t}{x}{s}{\widetilde{y}} \Par \bout{\suc}{\dual{n}, \widetilde{y}} \inact)
%		\\
%		T\lrangle{\news{\widetilde{m}} \bactout{n}{\abs{\widetilde{x}}{Q}}, \suc} &=&
%		\binp{\dual{n}}{y} (\hotrigger{t}{x}{s}{\abs{\widetilde{x}}{(\appl{y}{\widetilde{x}}})} \Par \bout{\suc}{\dual{n}, y} \inact)
%		\\
		T\lrangle{\news{\widetilde{m}} \bactout{n}{V}, \suc} &=&
		\binp{\dual{n}}{y} (\hotrigger{t}{x}{s}{y} \Par \bout{\suc}{\dual{n}, y} \inact)
		\\
		T\lrangle{\bactsel{n}{l}, \suc} &=&
		\bbra{\dual{n}}{l: \bout{\suc}{\dual{n}} \inact), l_i: \newsp{a}{\binp{a}{y} \bout{\suc}{\dual{n}} \inact}}_{i \in I}
	\end{eqnarray*}
%		
	\noi Let process 
	\[
		\Gamma; \emptyset; \Delta \proves P \hastype \Proc
	\]
	%
	\noi	It is straightforward to do a case analysis
			on all actions $\ell$ such that
			\[
				\Gamma; \emptyset; \Delta \proves P \hby{\ell} \Delta' \proves P'
			\]
			to show that $\ell$ is definable.
%	\noi it is straightforward to verify that $\forall \ell$, $\ell$ is definable.
	\qed
\end{proof}

%\jp{Here again I think that the closure should not mention environments in the  LHS.}

\begin{lemma}[Extrusion]\rm
	\label{lem:extrusion}
	Let $m_1 = \fn{V_1}$ and $m_2 = \fn{V_2}$. 
	If 
	\[
		\horel{\Gamma}{\Delta_1'}{\newsp{\widetilde{m_1}}{P \Par \bout{\suc}{\dual{n}, V_1} \inact}}{\cong}{\Delta_2}{\newsp{\widetilde{m_2}}{Q \Par \bout{\suc}{\dual{n}, V_2} \inact}}
	\]
	then $\exists \Delta_1, \Delta_2$ such that
	\[
		\horel{\Gamma}{\Delta_1}{P}{\cong}{\Delta_2}{Q}.
	\]
\end{lemma}

\begin{proof}
	\noi Let
%
	\begin{eqnarray*}
		\mathcal{S}	&=&
					\set{(\Gamma; \es; \Delta_1 \proves P \hastype \Proc\ ,\ \Gamma; \es; \Delta_2 \proves Q \hastype \Proc) \setbar \\
				& &	\horel{\Gamma}{\Delta_1'}{\newsp{\widetilde{m_1}}{P \Par \bout{\suc}{\dual{n}, V_1} \inact}}
					{\cong}{\Delta_2}{\newsp{\widetilde{m_2}}{Q \Par \bout{\suc}{\dual{n}, V_2} \inact}},\\
				&&   \land m_1 = \fn{V_1} \land m_2 = \fn{V_2} \\
		&&}
	\end{eqnarray*}
%
	\noi We show that $\mathcal{S}$ is a reduction-closed, barbed congruence.


	\begin{itemize}
		\item	{\bf Reduction-closed:}

				$P \red P'$
				implies
				\[
					\newsp{\widetilde{m_1}}{P \Par \bout{\suc}{\dual{n}, V_1} \inact}
					\red
					\newsp{\widetilde{m_1}}{P' \Par \bout{\suc}{\dual{n}, V_1} \inact}
				\]
				which implies from the freshness of $\suc$
				\[
					\newsp{\widetilde{m_1}}{P \Par \bout{\suc}{\dual{n}, V_2} \inact}
					\red^{*}
					\newsp{\widetilde{m_1}}{Q' \Par \bout{\suc}{\dual{n}, V_2} \inact}
				\]
				which in turn implies
				$Q \red^{*} Q'$ as required.

	\item	{\bf Barb Preserving:}

			Let $\Gamma; \es; \Delta_1 \proves P \barb{m}$. We analyse three cases.
			%
		    \begin{itemize}
				\item	Case: $m \not= s$ ($m$ is not a session name)

						$\Gamma; \es; \Delta_1 \proves P \barb{m}$
						implies
					%
						\[
							\Gamma; \es; \Delta_1' \proves
							\newsp{\widetilde{m_1}}{P \Par \bout{\suc}{\dual{n}, V_1} \inact}
							\barb{m}
						\]
					%
						which implies
						\[
							\Gamma; \es; \Delta_2' \proves
							\newsp{\widetilde{m_2}}{Q \Par \bout{\suc}{\dual{n}, V_2} \inact}
							\Barb{m}
						\]
						which implies from the freshness of $\suc$ that
						$\Gamma; \es; \Delta_2 \proves Q \Barb{m}$ as required.

				\item	Case: $m = s$ ($m$ is a session name) and $m \not= n$.
						Similar proof as the previous case.

				\item	Case: $m = s$ ($m$ is a session name) and $m = n$ and
						$\Gamma; \es; \Delta_1 \proves P \barb{n}$
						
						The fact that $n$ is a session 
						implies that $n, \dual{n} \in \dom{\Delta_1'}$
						which implies from the definition
						of barbs (\defref{def:barbs}) that:
						%
						\[
							\Gamma; \es; \Delta_1' \proves
							\newsp{\widetilde{m_1}}{P \Par \bout{\suc}{\dual{n}, V_1} \inact}
							\not\barb{n}
						\]
						%
						This is because both endpoints of the session $n$
						are present in $\Delta_1'$.


						We compose $\Gamma; \es; \Delta_1 \proves P \hastype \Proc$ with
						$\binp{\dual{\suc}}{x, y} T\lrangle{\ell, \suc'}$
						with $\subj{\ell} = x$ and fresh $\suc'$ to get
						%
						\[
							\Gamma; \es; \Delta_1' \proves
							\newsp{\widetilde{m_1}}{P \Par \bout{\suc}{\dual{n}, V_1} \inact} \Par
							\binp{\dual{\suc}}{x, y} T\lrangle{\ell, \suc'}
						\]
						%
						The definition of definibility and the fact that $\Gamma; \es; \Delta_1 \proves P \barb{n}$
						implies that
						%
						\[
							\newsp{\widetilde{m_1}}{P \Par \bout{\suc}{\dual{n}, V_1} \inact} \Par
							\binp{\dual{\suc}}{x, \widetilde{y}} T\lrangle{\ell, \suc'}
							\red^{*} 
							\newsp{\widetilde{m_1}}{P' \Par \bout{\suc'}{\dual{n}, V_1'} \inact}
						\]
						%
						\noi and furthermore
						%
						\[
							\newsp{\widetilde{m_2}}{Q \Par \bout{\suc}{\dual{n}, V_2} \inact} \Par
							\binp{\dual{\suc}}{x, \widetilde{y}} T\lrangle{\ell, \suc'}
							\red^{*} 
							\newsp{\widetilde{m_2}}{Q' \Par \bout{\suc'}{\dual{n}, V_2'} \inact}
						\]
						%
						\noi The last reduction implies that
						$\Gamma; \es; \Delta_2 \proves Q \Barb{n}$ as required.
				\end{itemize}
    
		\item	{\bf Congruence:}

				The key case of congruence is parallel composition.
				The other cases are easier due to the fact that we are
				working with closed process terms (i.e.~input congruence is straightforward
				on closed process terms).
				We define relation $\mathcal{C}$ as
				%
				\begin{eqnarray*}
					\mathcal{C} &=&
					\set{	(\Gamma; \es; \Delta_1 \cat \Delta_3 \proves P \Par R \hastype \Proc,
							\Gamma; \es; \Delta_2 \cat \Delta_3 \proves Q \Par R \hastype \Proc) \setbar
					\\
					& &	\forall R \textrm{ such that } \exists \Delta_3, \Gamma;\es; \Delta_3 \proves R \hastype \Proc \land\\
					& &	\horel{\Gamma}{\Delta_1'}{\newsp{\widetilde{m_1}}{P \Par \bout{\suc}{\dual{n}, V_1} \inact}}
						{\cong}
						{\Delta_2'}{\newsp{\widetilde{m_2}}{Q \Par \bout{\suc}{\dual{n}, V_2} \inact}}}
				\end{eqnarray*}
		%
				We show that $\mathcal{C}$ is a congruence with respect to parallel composition.
				We distinguish two cases:
				\begin{itemize}
					\item	Case: $(\dual{n} \cup \fn{V_1} \cup \fn{V_2}) \cap \fn{R} = \es$

							From the contextual definition of $\cong$ we can deduce that
							$\forall \Gamma; \es; \Delta_3 \proves R \hastype \Proc$:
							%
							\[
								\horel{\Gamma}{\Delta_1' \cat \Delta_3}{\newsp{\widetilde{m_1}}{P \Par \bout{\suc}{\dual{n}, V_1} \inact} \Par R}
								{\cong}
								{\Delta_2' \cat \Delta_3}{\newsp{\widetilde{m_2}}{Q \Par \bout{\suc}{\dual{n}, V_2} \inact} \Par R}
							\]
							%
							Because of the requirement
							$(\dual{n} \cup \fn{V_1} \cup \fn{V_2}) \cap \fn{R} = \es$
							the above is up to structural congruence with
							\[
								\horel{\Gamma}{\Delta_1' \cat \Delta_3}{\newsp{\widetilde{m_1}}{P \Par \bout{\suc}{\dual{n}, V_1} \inact \Par R}}
								{\cong}
								{\Delta_2' \cat \Delta_3}{\newsp{\widetilde{m_2}}{Q \Par \bout{\suc}{\dual{n}, V_2} \inact \Par R}}
							\]
							From the definition of $\mathcal{C}$ the conclusion is trivial.

					\item	Case: $\widetilde{s} = \set{\dual{n}, \widetilde{m_1}} \cap \set{\dual{n}, \widetilde{m_2}} \in \fn{R}$.

							Let $R^{\widetilde{y}}$ such that $R = R^{y}\subst{\widetilde{s}}{\widetilde{\widetilde{y}}}$.


							From the contextual definition of $\cong$ we can deduce that for fresh $\suc'$
							$\forall \Gamma; \es; \Delta_3' \proves \binp{\dual{\suc}}{\widetilde{y}} (R^{\widetilde{y}} \Par \bout{\suc'}{\widetilde{y}} \inact) \hastype \Proc$:
							%
							\[
								\mhorel{\Gamma}
								{\Delta_1''}
									{\newsp{\widetilde{m_1}}{P \Par \bout{\suc}{\dual{n}, V_1} \inact}
									\Par \binp{\dual{\suc}}{\widetilde{y}} (R^{y} \Par \bout{\suc'}{\widetilde{y}} \inact)}
								{\cong}
								{\Delta_2''}{}
									{\newsp{\widetilde{m_2}}{Q \Par \bout{\suc}{\dual{n}, V_2} \inact}
									\Par \binp{\dual{\suc}}{\widetilde{y}} (R^{y} \Par \bout{\suc'}{\widetilde{y}} \inact)}
							\]

%							From the definition of $\mathcal{C}$
%							we can deduce that $\forall R^{y_1}$ such that $R = R^{y_1}\subst{\widetilde{s}}{\widetilde{y_1}}$
%							and $\suc'$ fresh and $\set{\widetilde{y}} = \set{\widetilde{y_1}} \cup \set{\widetilde{y_2}}$:
							%
%							\[
%								\mhorel{\Gamma}{\Delta_1''}{\newsp{\widetilde{m_1}'}{P \Par \bout{\suc}{\dual{n}, \widetilde{m_1}''} \inact} \Par \binp{\dual{\suc}}{\widetilde{y}} (R^{y_1} \Par \bout{\suc'}{\widetilde{y_2}} \inact)}
%								{\cong}
%								{\Delta_2''}{}{\newsp{\widetilde{m_2}'}{Q \Par \bout{\suc}{\dual{n}, \widetilde{m_2}''} \inact} \Par \binp{\dual{\suc}}{\widetilde{y}} (R^{y_1} \Par \bout{\suc'}{\widetilde{y_2}} \inact)}
%							\]
							%
							\noi Applying reduction closeness to the above pair we get:
							%
							\[
								\horel{\Gamma}{\Delta_1''}{\newsp{\widetilde{m_1}}{P \Par R \Par \bout{\suc'}{\dual{n}, V_1} \inact}}{\cong}{\Delta_2''}{\newsp{\widetilde{m_2}}{Q \Par R \Par \bout{\suc'}{\dual{n}, V_2} \inact}}
							\]
						%
						\noi The conclusion then follows from the definition of $\mathcal{C}$.
	    \end{itemize}
	\end{itemize}
	\qed
\end{proof}


\begin{lemma}\rm
	\label{app:lem:cong_is_wb}
	$\cong \subseteq \hwb$.
\end{lemma}

\begin{proof}
	\noi Let $\Re$ be the typed relation (we omit the typing information in the definition):
	\[
		\Re = \set{(P_1, P_2) \setbar \horel{\Gamma}{\Delta_1}{P_1}{\cong}{\Delta_2}{P_2}}
	\]

	To prove that $\Re$ is a higher-order bisimulation
	we do a case analysis on the transition:
	\[
		\horel{\Gamma}{\Delta_1}{P_1}{\by{\ell}}{\Delta_1'}{P_1'}
	\]
	We distinguish two cases: one for the $\tau$ transition and one case for
	the visible transitions $\ell$.

%% Case tau
\begin{enumerate}
	\item Suppose 
			\[
				\horel{\Gamma}{\Delta_1}{P_1}{\by{\tau}}{\Delta_1'}{P_1'}
			\]
			\noi The result follows the reduction closeness property of $\cong$ since
			\[
				\horel{\Gamma}{\Delta_2}{P_2}{\By{\tau}}{\Delta_2'}{P_2'}
			\]
			\noi and
			\[
				\horel{\Gamma}{\Delta_1'}{P_1'}{\cong}{\Delta_2'}{P_2'} \text{ implies } \horel{\Gamma}{\Delta_1'}{P_1'}{\ \Re\ }{\Delta_2'}{P_2'}
			\]

%% Case ell
	\item Suppose
		%
			\begin{eqnarray}
				\horel{\Gamma}{\Delta_1}{P_1}{\by{\ell}}{\Delta_1'}{P_1'}
				\label{lem:cong_is_wb1}
			\end{eqnarray}
		%
			\noi We choose test $T\lrangle{\ell, \suc}$ to get
		%
			\begin{eqnarray}
				\horel{\Gamma}{\Delta_1 \cat \Delta_3}{P_1 \Par T\lrangle{\ell, \suc}}{\cong}{\Delta_2 \cat \Delta_3}{P_2 \Par T\lrangle{\ell, \suc}}
				\label{lem:cong_is_wb2}
			\end{eqnarray}
		%
			\noi From this point we distinguish two subcases:

			\begin{enumerate}[i.]
			%% Subcase i
				\item	Sub-case $\ell \in \set{\bactinp{n}{V_1}, \bactsel{n}{l}, \bactbra{n}{l}}$:

						\noi By reducing~(\ref{lem:cong_is_wb1}), we obtain
					%
						\begin{eqnarray*}
							&& P_1 \Par T\lrangle{\ell, \suc} \red P_1' \Par \bout{\suc}{\dual{n}} \inact \\
							&& \Gamma; \es; \Delta_1' \cat \Delta_3' \proves P_1' \Par \bout{\suc}{\dual{n}} \inact \barb{\suc}
						\end{eqnarray*}
					%
						\noi implies from~(\ref{lem:cong_is_wb2})
					%
						\begin{eqnarray*}
							&& \Gamma; \es; \Delta_2 \cat \Delta_3 \proves P_2 \Par T\lrangle{\ell, \suc} \Barb{\suc}
						\end{eqnarray*}
					%
						\noi implies from Lemma~\ref{lem:definibility},
					%
						\begin{eqnarray*}
							&& \horel{\Gamma}{\Delta_2}{P_2}{\By{\ell}}{\Delta_2'}{P_2'}\\
							&& P_2 \Par T \lrangle{\ell, \suc} \red^{*} P_2' \Par \bout{\suc}{\dual{n}} \inact
						\end{eqnarray*}
					%
						\noi and
					%
						\[
							\horel{\Gamma}{\Delta_1' \cat \Delta_3'}{P_1' \Par \bout{\suc}{\dual{n}}\inact}{\cong}{\Delta_2' \cat \Delta_3'}{P_2' \Par \bout{\suc}{\dual{n}} \inact}
						\]
						We then apply \lemref{lem:extrusion} to get
					%
						\[
							\horel{\Gamma}{\Delta_1'}{P_1'}{\cong}{\Delta_2'}{P_2'} \text{ implies } \horel{\Gamma}{\Delta_1'}{P_1'}{\ \Re\ }{\Delta_2'}{P_2'}
						\]
					%
						\noi as required.

			%% Subcase ii
				\item	Sub-case $\ell = \news{\widetilde{m_1}} \bactout{n}{V_1}$:

						\noi Note that $T\lrangle{\news{\widetilde{m_1}} \bactout{n}{V_1}, \suc} = T\lrangle{\news{\widetilde{m_2}} \bactout{n}{V_2}, \suc}$

						\noi Transition~in (\ref{lem:cong_is_wb1}) becomes
					%
						\begin{eqnarray}
							\horel{\Gamma}{\Delta_1}{P_1}{\by{\news{\widetilde{m_1}} \bactout{n}{V_1}}}{\Delta_1'}{P_1'}
							\label{lem:cong_is_wb3}
						\end{eqnarray}
					%
						\noi If we use the test process $T\lrangle{\news{\widetilde{m_1}} \bactout{n}{V_1}, \suc}$ we reduce to:%~\ref{lem:cong_is_wb1} we get
					%
						\begin{eqnarray*}
							&& P_1 \Par T\lrangle{\news{\widetilde{m_1}} \bactout{n}{V_1}, \suc}
							\red
							\newsp{\widetilde{m_1}}{P_1' \Par \hotrigger{t}{x}{s}{V_1}} \Par \bout{\suc}{\dual{n}, V_1} \inact \\
							&& \Gamma; \es; \Delta_1' \cat \Delta_3' \proves \newsp{\widetilde{m_1}}{P_1' \Par \hotrigger{t}{x}{s}{V_1}} \Par \bout{\suc}{\dual{n}, V_1} \inact \barb{\suc}
						\end{eqnarray*}
					%
						\noi implies from~(\ref{lem:cong_is_wb2})
					%
						\[
							\Gamma; \es; \Delta_2 \cat \Delta_3 \proves P_2 \Par T\lrangle{\news{\widetilde{m_2}} \bactout{n}{V_2}, \suc} \Barb{\suc}
						\]
					%
						\noi implies from \lemref{lem:definibility}
					%
						\begin{eqnarray}
							&& \horel{\Gamma}{\Delta_2}{P_2}{\By{\news{\widetilde{m_2}} \bactout{n}{V_2}}}{\Delta_2'}{P_2'}
							\label{lem:cong_is_wb4}\\
							&& P_2 \Par T \lrangle{\ell, \suc} \red^{*} \newsp{\widetilde{m_2}}{P_2' \Par \hotrigger{t}{x}{s}{V_2}} \Par \bout{\suc}{\dual{n}, V_2} \inact \nonumber
						\end{eqnarray}
					%
						\noi and
					%
						\[
							\mhorel{\Gamma}{\Delta_1' \cat \Delta_3'}{\newsp{\widetilde{m_1}}{P_1' \Par \hotrigger{t}{x}{s}{\abs{\widetilde{x}}{Q_1}}} \Par \bout{\suc}{\dual{n}, V_1} \inact}
							{\cong}
							{\Delta_2' \cat \Delta_3'}{}{\newsp{\widetilde{m_2}}{P_2' \Par \hotrigger{t}{x}{s}{\abs{\widetilde{x}}{Q_2}}} \Par \bout{\suc}{\dual{n}, V_2} \inact}
						\]
					%
						\noi We then apply \lemref{lem:extrusion} to get:
					%
						\[
							\mhorel{\Gamma}{\Delta_1'}{\newsp{\widetilde{m_1}}{P_1' \Par \hotrigger{t}{x}{s}{V_1}}}
							{\cong}
							{\Delta_2'}{}{\newsp{\widetilde{m_2}}{P_2' \Par \hotrigger{t}{x}{s}{V_2}}}
						\]
					%
						\noi From the last result and definition of $\Re$ we get:
						\[
							\mhorel{\Gamma}{\Delta_1'}{\newsp{\widetilde{m_1}}{P_1' \Par \hotrigger{t}{x}{s}{V_1}}}
							{\ \Re\ }
							{\Delta_2'}{}{\newsp{\widetilde{m_2}}{P_2' \Par \hotrigger{t}{x}{s}{V_2}}}
						\]
						\noi as required.

%				\item	Sub-case $\ell = \news{\widetilde{s}} \bactout{n}{\widetilde{m}}$:
%
%						\noi Follows similar arguments as the previous case.
			\end{enumerate}
\end{enumerate}
	\qed
\end{proof}

%%%%%%%%%%%%%%%%%%%%%%%%%%%%%%%%%%%%%%%%%%%%%%%%%%%%%%%%%%%%%%
% Proof of the main theorem
%%%%%%%%%%%%%%%%%%%%%%%%%%%%%%%%%%%%%%%%%%%%%%%%%%%%%%%%%%%%%%

%\begin{theorem}[Concidence]\label{app:thm:coincidence} We have:
%	\begin{enumerate}
%		\item	$\wbc\ =\ \hwb$.
%		\item	$\wbc\ =\ \cong$.
%	\end{enumerate}
%\end{theorem}
%
%\begin{proof}
%	\noi	\lemref{app:lem:wb_eq_wbf} proves $\hwb\ =\ \fwb$.
%			\lemref{app:lem:cong_is_wb} proves $\cong\ \subseteq\ \hwb$.
%			\lemref{app:lem:wb_is_wbc} proves $\hwb\ \subseteq\ \wbc$.
%			\lemref{app:lem:wbc_is_cong} proves $\wbc\ \subseteq\ \cong$.
%			From the above results, we conclude $\cong\ \subseteq\ \hwb\ =\ \fwb\ \subseteq\ \wbc\ \subseteq\ \cong$. 
%			\qed
%\end{proof}



