% !TEX root = main.tex

 Since types can limit
contexts (environments) where processes can interact, typed equivalences
usually offer {\em coarser} semantics than untyped equivalences.
Pierce and Sangiorgi~\cite{PiSa96b} demonstrated that IO-subtyping \newc{can justify
the optimal encoding of the $\lambda$-calculus by Milner---this was not possible
in the untyped polyadic $\pi$-calculus~\cite{MilnerR:funp}.}
After~\cite{PiSa96b}, many works on typed $\pi$-calculi 
have investigated correctness of encodings of known concurrent and
sequential calculi in order to examine semantic
effects of proposed typing systems. 

A  type discipline closely related
to session types is a family of linear typing systems. Kobayashi, Pierce, and Turner~\cite{LinearPi} first proposed a linearly typed reduction-closed, barbed congruence and 
used to 
reason about a tail-call optimisation of higher-order functions 
encoded 
as processes. 
Yoshida~\cite{Yoshida96} 
used a bisimulation of graph-based types to prove the full abstraction
of encodings of the polyadic synchronous $\pi$-calculus into the
monadic synchronous $\pi$-calculus. 
Later typed equivalences of a
family of linear and affine calculi \cite{BHY,DBLP:journals/iandc/YoshidaBH04,BergerHY05} 
were used to encode 
PCF \cite{Plotkin1977223,Milner19771}, the simply typed $\lambda$-calculus with sums and products, and System F \cite{GirardJY:protyp}
fully abstractly (a fully abstract encoding of the $\lambda$-calculi 
was an open problem in \cite{MilnerR:funp}).  
Yoshida, Honda, and Berger~\cite{YHB02} proposed a new bisimilarity
method associated with linear type structure and strong
normalisation. 
It presented applications to reason about secrecy in programming languages. 
A subsequent work~\cite{HY02} adapted these results
to a practical direction, proposing new typing
systems for secure higher-order and multi-threaded programming 
languages. 
In these works, typed properties, linearity and liveness, 
play a fundamental role in the analysis. In general, linear types 
are suitable to encode ``sequentiality'' in the sense of 
\cite{HylandJME:fulapi,AbramskyS:fulap}.

 
 {Our work follows 
the 
%principles for
%session type 
behavioural semantics in 
\cite{KYHH2015,KY2015,DBLP:journals/iandc/PerezCPT14}
where a bisimulation is defined on an LTS 
that assumes a session typed
observer.
%The bisimilarity is characterised by the corresponding
%reduction-closed, barbed congruence using techniques derived from~\cite{Hennessy07}.
Our theory for higher-order sessions 
differentiates from 
the work in~\cite{KYHH2015} and \cite{KY2015}, which 
considers  (first-order)
binary and multiparty session types, respectively.
P\'{e}rez et al~\cite{DBLP:journals/iandc/PerezCPT14} studied typed equivalences
for a 
theory of binary sessions based on linear logic,
without shared names.}
%Determinacy properties (confluence, $\tau$-inertness) are proven.



%The theory for higher-order session type quivalences is more challenging than
%their corresponding first-order bisimulation theory.
Our approach %for the higher-order 
to typed equivalences
builds upon techniques developed by Sangiorgi~\cite{SangiorgiD:expmpa,San96H}
and Jeffrey and Rathke~\cite{JeffreyR05}.
%The work %Sangiorgi as part of his Ph.D.~research
%%\cite{San96H,SangiorgiD:expmpa}
%\cite{SangiorgiD:expmpa}
%introduced the first fully-abstract encoding from the higher-order 
%$\pi$-calculus into the $\pi$-calculus. 
%Sangiorgi's encoding is based on the idea of a replicated input-guarded process 
%(a trigger process). 
%%We use a similar  replicated triggered process to encode \HOp into \sessp (\defref{d:enc:hopitopi}).
% Operational correspondence for
%the triggered encoding is shown using a context bisimulation
%with first-order labels.
As we have discussed, although context bisimilarity has a satisfactory discriminative power,
its use is hindered by the universal quantification on output.
To deal with this, 
Sangiorgi proposes \emph{normal bisimilarity}, 
a tractable  equivalence without universal quantification. 
To prove that context and normal bisimilarities coincide,~\cite{SangiorgiD:expmpa} uses 
triggered processes.
%The encoding also motivates the definition of a form of
Triggered bisimulation is also defined on first-order labels
where the context bisimulation is restricted to arbitrary
trigger substitution. %rather than arbitrary process substitutions.
This
characterisation of context bisimilarity  was refined in~\cite{JeffreyR05} for
calculi with recursive types, not addressed in~\cite{San96H,SangiorgiD:expmpa} and
quite relevant in %our work (cf. \defref{d:enc:hopitoho}).
session-based concurrency.
The
bisimulation in~\cite{JeffreyR05}
is based on an LTS  extended with trigger meta-notation.
%for a full higher-order $\pi$-calculus that allows
%higher-order applications.
As in~\cite{San96H,SangiorgiD:expmpa}, 
the LTS in~\cite{JeffreyR05}
observes first-order triggered values instead of
higher-order values, offering a more direct characterisation of contextual equivalence
and lifting the restriction to finite types.
\emph{Environmental bisimulations}~\cite{DBLP:conf/lics/SangiorgiKS07} 
%which 
%Sangiorgi et al.~\cite{DBLP:conf/lics/SangiorgiKS07}, 
use a higher-order LTS 
to define a bisimulation that stores the observer's knowledge; hence, observed actions are based on this knowledge
at any given time. This approach is enhanced in~\cite{DBLP:journals/cl/KoutavasH12}
with a mapping from constants to higher-order values. This 
allows to observe first-order values instead
of higher-order values. It differs from~\cite{San96H,JeffreyR05} in that 
the mapping between higher- and first-order values is no longer implicit.

\paragraph{Comparison with respect to~\cite{JeffreyR05}.} 
We briefly contrast 
the approach in~\cite{JeffreyR05} and ours based on 
%\dk{higher-order ($\hwb$) and} 
characteristic  bisimilarity ($\fwb$):
\begin{enumerate}[$\bullet$]
%\begin{enumerate}[i.]
\item 
The LTS in~\cite{JeffreyR05} is enriched with extra labels for triggers;
an output action transition emits a trigger and introduces a parallel replicated trigger.
Our 
approach retains usual labels/transitions; in  case of output,
%our bisimilarities 
%$\hwb$ and 
$\fwb$
introduces a parallel
\emph{non-replicated} trigger.

\item Higher-order input in~\cite{JeffreyR05} involves 
the input of a trigger which reduces after substitution.
Rather than a trigger name, %our bisimulations  
%$\hwb$ and 
$\fwb$
decrees the input of a trigger value $\abs{z}\binp{t}{x} (\appl{x}{z})$.

\item Unlike~\cite{JeffreyR05}, 
%our 
$\fwb$ treats  
first- and higher-order values uniformly. %In the latter case, 
%Since the 
As the typed LTS distinguishes linear and shared values,
replicated closures are used only for shared values.

\item In~\cite{JeffreyR05}   name matching   is
crucial to prove completeness of bisimilarity.
In our case, \HOp lacks name matching and 
%Contrarily 
%\jpc{In contrast,} 
we use session types: a characteristic value inhabiting a type enables the simplest form of interactions with the environment.

%We use the characteristic process interaction with the environment, exploiting session types.
%, i.e., instead of matching a name, we embed it into a process and then observe its behaviour.

%In~\cite{JeffreyR05}  a matching construct 
%is crucial to prove completeness of bisimilarity.
%Since our language lacks matching,
%we use session type information to obtain the simplest value that 
%enables interaction with the environment.
\end{enumerate}
%\noi 
We compare our approach to that in~\cite{JeffreyR05} 
using a representative example.
%We considered the transitions and resulting processes involved in checking bisimilarity of process 
%$\bout{n}{\abs{x}{\appl{x}(\abs{y}{\bout{y}{m}} \inact)}} \inact$
%with itself.

%\section{Example}


\begin{example}
	Consider typed process
\[
	\Gamma; \es; \Delta \cat n: \btout{U} \tinact \proves \bout{n}{\abs{x}{\appl{x}(\abs{y}{\bout{y}{m}} \inact)}} \inact \hastype \Proc
\]
	with $U = \shot{(\shot{(\shot{(\btout{S} \tinact)})})}$
	We will count the number of steps require to check the bisimilarity
	of the above process with it self. Note that
%
	\begin{eqnarray*}
		\mapchar{\btinp{U} \inact}{s}\\
		= && \binp{s}{x} \mapchar{\shot{(\shot{(\shot{(\btout{S} \tinact)})})}}{x}\\
		= && \binp{s}{x} \appl{x}{\omapchar{\shot{(\shot{(\btout{S} \tinact)})}}}\\
		= && \binp{s}{x} \appl{x}{(\abs{y}{\mapchar{\shot{(\btout{S} \tinact)}}{y}})}\\
		= && \binp{s}{x} \appl{x}{(\abs{y}{\appl{y}{\omapchar{\btout{S} \tinact}})}}\\
		= && \binp{s}{x} \appl{x}{(\abs{y}{\appl{y}{a})}}\\
	\end{eqnarray*}
%
	The characteristic process of
	the type $\shot{(\shot{(\shot{(\btout{S} \tinact)})})}$ is

	\begin{tabular}{rrl}
		(1)& & $\Gamma; \es; \Delta \cat n: \btout{U} \tinact \proves \bout{n}{\abs{x}{\appl{x}(\abs{y}{\bout{y}{m}} \inact)}} \inact$ \\
		&$\by{\bactout{n}{\abs{x}{\appl{x}(\abs{y}{\bout{y}{m}} \inact)}}}$& $\Gamma; \es; \Delta \proves \inact$\\
		(2) & & $\Gamma; \es; \Delta \proves \binp{t}{z} \newsp{s}{\mapchar{\btinp{U} \inact}{s} \Par \bout{\dual{s}}{\abs{x}{\appl{x}(\abs{y}{\bout{y}{m}} \inact)}} \inact}$\\
		&$=$& $\Gamma; \es; \Delta \proves \binp{t}{z} \newsp{s}{\binp{s}{x} \appl{x}{(\abs{y}{\appl{y}{a})}} \Par \bout{\dual{s}}{\abs{x}{\appl{x}(\abs{y}{\bout{y}{m}} \inact)}} \inact}$\\
		(3) &$\by{\bactinp{t}{b}}$& $\Gamma; \es; \Delta \proves \newsp{s}{\binp{s}{x} \appl{x}{(\abs{y}{\appl{y}{a})}} \Par \bout{\dual{s}}{\abs{x}{\appl{x}(\abs{y}{\bout{y}{m}} \inact)}} \inact}$\\
		(4) &$\by{\tau}$& $\Gamma; \es; \Delta \proves \appl{\abs{x}{\appl{x}(\abs{y}{\bout{y}{m} \inact})}}{(\abs{y}{\appl{y}{a})}}$\\
		(5) &$\by{\tau}$& $\Gamma; \es; \Delta \proves \appl{(\abs{y}{\appl{y}{a}})}{(\abs{y}{\bout{y}{m} \inact})} $\\
		(6) &$\by{\tau}$& $\Gamma; \es; \Delta \proves \appl{(\abs{y}{\bout{y}{m} \inact})}{a}$\\
		(7) &$\by{\tau}$& $\Gamma; \es; \Delta \proves \bout{a}{m} \inact$
	\end{tabular}

On Jeffrey and Rathke Semantics:

	\begin{tabular}{rrl}
		(1)& & $\Gamma; \es; \Delta \cat n: \btout{U} \tinact \proves \bout{n}{\abs{x}{\appl{x}(\abs{y}{\bout{y}{m}} \inact)}} \inact$ \\
		&$\by{\bactout{n}{\abs{x}{\appl{x}(\abs{y}{\bout{y}{m}} \inact)}}}$& $\Gamma; \es; \Delta \proves \inact$\\
		(2) & & $\Gamma; \es; \Delta \proves \repl{} \binp{t}{x} \appl{x}(\abs{y}{\bout{y}{m}} \inact) $\\
		(3) &$\by{\bactinp{t}{\tau_l}}$& $\Gamma; \es; \Delta \proves \repl{} \binp{t}{x} \appl{x}(\abs{y}{\bout{y}{m}} \inact) \Par \appl{\abs{x}{\appl{x}(\abs{y}{\bout{y}{m}} \inact)}}{\tau_l}$\\
		(4) &$\by{\tau}$& $\Gamma; \es; \Delta \proves \repl{} \binp{t}{x} \appl{x}(\abs{y}{\bout{y}{m}} \inact) \Par \appl{\tau_l}{(\abs{y}{\bout{y}{m}} \inact)}$\\
		(5) &$\by{\bactout{l}{\tau_k}}$& $\Gamma; \es; \Delta \proves \repl{} \binp{t}{x} \appl{x}(\abs{y}{\bout{y}{m}} \inact) \Par \repl{} \binp{k}{y} \bout{y}{m} \inact $\\
		(6) &$\by{\bactinp{k}{a}}$& $\Gamma; \es; \Delta \proves \repl{} \binp{t}{x} \appl{x}(\abs{y}{\bout{y}{m}} \inact) \Par \repl{} \binp{k}{y} \bout{y}{m} \inact \Par \appl{\abs{y}{\bout{y}{m} \inact}}{a}$\\
		(7) &$\by{\tau}$& $\Gamma; \es; \Delta \proves \repl{} \binp{t}{x} \appl{x}(\abs{y}{\bout{y}{m}} \inact) \Par \repl{} \binp{k}{y} \bout{y}{m} \inact \Par \bout{a}{m} \inact$
	\end{tabular}


	Comparing
	\begin{itemize}
		\item	Same number of transitions.
		\item	J \& R more observable actions.
		\item	J \& R replicated processes.
	\end{itemize}

\end{example}




The previous comparison %, detailed in \appref{app:jandr}, 
reveals that our approach 
%based on %even if both techniques require the same number of transitions, 
%a refined LTS and characteristic bisimilarity 
requires less visible transitions and replicated processes. 
Therefore, linearity information does simplify analyses, 
as it enables simpler witnesses in  coinductive proofs.





%There are similarities and differences between of the characteristic bisimulation
%and the bisimulation as defined by Jeffrey and Rathke
%(below we use the meta-notation adopted in~\cite{JeffreyR05}):
%%
%\begin{enumerate}[i)]
%	\item	The output of a higher-order value $\abs{x}{Q}$ on name
%		$n$ in Jeffrey\&Rathke approach requires the output of
%		a fresh trigger name $t$ (notation $\tau_t$)
%		on name $n$ 
%		and then the introduction of a replicated triggered process
%		(notation $(t \Leftarrow (x) Q)$)
%		in the context of the acting process:
%		%
%		\[
%			P \by{\news{t} \bactout{n}{\tau_{t}}} P' \Par (t \Leftarrow (x) Q) \by{\bactinp{t}{v}} P' \Par \appl{(x) Q}{v} \Par (t \Leftarrow (x) Q) 
%		\]
%		%
%		In the characteristic bisimulation approach we only observe
%		an output of a value that can be either first- or higher-order:
%		%
%		\[
%			P \hby{\bactout{n}{V}} P' 
%		\]
%		%
%		with $V = \abs{x}{Q}$ or $V = m$.
%		A non-replicated triggered process appears in
%		the parallel context of the acting process when
%		we compare two processes for behavioural equality
%		(cf.~characteristic bisimulation \defref{d:fwb}),
%		$P' \Par \htrigger{t}{\abs{x}{Q}}$.
%		In fact using the LTS in
%		\defref{d:tlts} we can get:
%		%
%		\begin{eqnarray*}
%			P' \Par \htrigger{t}{\abs{x}{Q}}
%			&\by{\abs{z}{\binp{z}{y} \repl{} \binp{t}{x} \appl{y}{x}}}&
%			P' \Par \newsp{s}{\binp{s}{y} \repl{} \binp{t}{x} \appl{y}{x} \Par \bout{s}{\abs{x}{Q}} \inact}\\
%			&\by{\tau}&
%			P' \Par \repl{}\binp{t}{y} \appl{\abs{x}{Q}}{y}
%		\end{eqnarray*}
%		%
%		that simulates the Jeffrey\&Rathke approach.
%
%		The characteristic bisimulation differentiates from
%		the Jeffrey\&Rathke approach:
%		\begin{enumerate}[$\bullet$]
%			\item	The typed LTS predicts the case of linear
%				output values and will never allow replication
%				of such a value;
%				if $V$ is linear the input action would have no replication
%				operator, as
%				$\abs{z}{\binp{z}{y} \binp{t}{x} \appl{y}{x}}$.
%
%			\item	The characteristic bisimulation introduces a uniform approach
%				not only for
%				higher-order values but for first-order values
%				as well. A triggered process can accept any
%				process that can substitute a first-order value as well.
%				This is derived from the fact that the $\HOp$
%				calculus makes no use of a matching operator, in contrast
%				to the calculus defined in~\cite{JeffreyR05})
%				where name matching is crucial to prove completness
%				of the bisimilarity relation.
%				We chose not to include the matching operator
%				because of the requirement of a minimal calculus.
%				In the lack of matching we use types to inhabit
%				a value so we can observe its simplest interaction
%				with the process environment.
%
%			\item	The \HOp calculus requires only first-order
%				applications. Higher-order applications,
%				as in the Jeffrey\&Rathke work,
%				are presented as an extension in the \HOpp
%				calculus.
%
%			\item	The trigger process is non-replicated. In fact
%				the trigger process transforms guards the output
%				value with a higher-order input prefix. The
%				functionality of the input is used to
%				simulate the contextual bisimilarity that subsumes
%				the replicated trigger approach.
%				The transformation of an output action as an input
%				action allows for treating an output
%				using the restricted LTS \defref{def:rlts}:
%				%
%				\[
%					P' \Par \htrigger{t}{\abs{x}{Q}} \hby{\bactinp{t}{\abs{x}{\mapchar{U}{x}}}}
%					P' \Par \news{s}{ \appl{\mapchar{U}{x}}{s} \Par \bout{\dual{s}}{\abs{x}{Q}} \inact}
%				\]
%		\end{enumerate}
%		%
%		%In essence we are transforming a replicated trigger into a process
%		%that is input-prefixed on a fresh name that receives a higher-order
%		%value;
%
%	\item	The input of a higher-order value in the Jeffrey\&Rathke approach requires 
%		the input of a fresh trigger name, which is substituted on the application
%		variable, thus having a meta-suntax for triggered application instead
%		of higher-order applications:
%		%
%		\[
%			\binp{n}{x} P \by{\bactinp{n}{\tau_k}} \appl{\abs{x}{P}}{\tau_k} \by{\tau} P \subst{x}{\tau_k} 
%		\]
%		%
%		with every instance of process variable $x$ in $P$ being substituted
%		with trigger value $\tau_k$ to give a process of the form $\appl{\tau_k}{x}$.
%		The approach in the characteristic bisimulation observes the
%		triggered value
%		$\abs{z}\binp{t}{x} \appl{x}{z}$ as an input instead of the
%		trigger name:
%		%
%		\[
%			P \hby{\bactinp{n}{\abs{z}\binp{t}{x} \appl{x}{z}}} P \subst{\abs{z}\binp{t}{x} \appl{x}{z}}{x}
%		\]
%		%
%		with applications being transformed to
%		$\abs{z}{\binp{t}{x} \appl{x}{z}}{v}$
%		Note that in the characteristic bisimulation semantics
%		we can also observe a characteristic process as an input.
%		
%	\item 	Triggered application in the Jeffrey\&Rahtke
%		are observe using an output
%		lead into an output observation of the
%		application value over
%		the fresh trigger name.
%		%
%		\[
%			\appl{\tau_k}{v} \by{\bactout{k}{v}} \inact
%		\]
%		%
%		In the characteristic bisimulation instead of observing an 
%		application and its value as an action we observe:
%		i) the name of the trigger through the trigger value
%		application; and ii) the application
%		value by inhabiting it in the characteristic value
%		and observing the interaction of the corresponding
%		characteristic process with its environment.
%		%
%		\begin{eqnarray*}
%			\appl{\abs{z}{\binp{t}{x} \appl{x}{z}}}{v} &\by{\tau}& \binp{t}{x} \appl{x}{v}
%			\by{\bactinp{t}{\abs{x}{\mapchar{U}{x}}}}
%			\appl{\abs{x}{\mapchar{U}{x}}}{v}
%			\by{\tau} \mapchar{U}{x} \subst{n}{x}
%		\end{eqnarray*}
%		%
%\end{enumerate}

%The main differences of the triggered
%bisimulation approach comparing to our approach are:
%i) We use observe higher-order values on the LTS in contrast to first-order 
%values in~\cite{DBLP:journals/lmcs/JeffreyR05}.
%ii) In our approach we avoid the replicated triggered process,
%by transforming the output process into a higher-order guarded input.
%iii) The triggered bisimulation gives semantics for higher-order application,
%whereas in our approach we give semantics for first-order applications
%and show that higher-order applications are fully encodable.

%Boreale and Sangiorgi, 
%Deng and Hennessy, 
%Jeffrey and Rathke, Hennessy and Koutavas, Schmitt and Lenglet, Pi\E9rard and Sumii.
%Perez et al (bisimilarities for binary sessions), Kouzapas and Yoshida (bisimilarities for binary and multiparty sessions).
%Bisimilarities for HO processes: \cite{Xu07}.







