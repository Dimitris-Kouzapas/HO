% !TEX root = main.tex

\noi
We develop a theory for observational equivalence over
session typed \HOp processes that follows the principles
laid in our previous works~\cite{KYHH2015,KY2015}.
We introduce \emph{higher-order bisimulation} (\defref{d:hwb})
and
\emph{characteristic bisimulation} (\defref{d:fwb})
and prove that
they coincide with reduction-closed,
barbed congruence (\thmref{the:coincidence}).

We begin by defining an (early) labelled transition system (LTS) on
untyped processes~(\S\,\ref{ss:lts}). 
Then, using the \emph{environmental} transition semantics (\secref{ss:elts}), 
we define a typed LTS to formalise 
how a typed process interacts with a typed observer. 

\subsection{Labelled Transition System for Processes}
\label{ss:lts}

We define the interaction of processes with their evnviorment using action labels $\ell$:
%
\begin{center}
	\begin{tabular}{l}
		$\ell
			\bnfis  \tau 
			\bnfbar	\bactinp{n}{V} 
			\bnfbar	\news{\widetilde{m}} \bactout{n}{V}
			\bnfbar	\bactsel{n}{l} 
			\bnfbar	\bactbra{n}{l}$
	\end{tabular}
\end{center}
%
\noi 
Label $\tau$ defines internal actions.
Action
$\news{\widetilde{m}} \bactout{n}{V}$
denotes the sending of value $V$
over channel $n$ with a possible empty set of restricted names
$\widetilde{m}$ 
(we may write $\bactout{n}{V}$ when $\widetilde{m}$ is empty).
%and 
%$\news{\widetilde{m}} \bactout{n}{\AT{V}{U}}$
%when the type of $V$ is~$U$.
Dually, the action for value reception is 
$\bactinp{n}{V}$.
Actions for select and branch on
a label~$l$ are denoted $\bactsel{n}{l}$ and $\bactbra{n}{l}$, resp.
We write $\fn{\ell}$ and $\bn{\ell}$ to denote the
sets of free/bound names in $\ell$, resp.
%and set $\mathsf{n}(\ell)=\bn{\ell}\cup \fn{\ell}$. 
Given $\ell \neq \tau$, we write $\subj{\ell}$
to denote the \emph{subject} of $\ell$.

\emph{Dual actions}
occur on subjects that are dual between them and carry the same
object; thus, output is dual to input and 
selection is dual to branching.
Formally, duality \jpc{on actions}
is the symmetric relation $\asymp$ that satisfies:
%
\[
	\bactsel{n}{l} \asymp \bactbra{\dual{n}}{l}
	\qquad \qquad 
	\news{\widetilde{m}} \bactout{n}{V} \asymp \bactinp{\dual{n}}{V}
\]

%%%%%%%%%%%%%%%%%%%% LTS Figure %%%%%%%%%%%%%%%%%%%%%%%%%%%%%%%%%%%%%
\begin{figure}
	\begin{mathpar}
		\inferrule[\ltsrule{App}]{
		}{
			\appl{(\abs{x}{P})}{V} \by{\tau} P \subst{V}{x}
		}
		\and
		\inferrule[\ltsrule{Snd}]{
		}{
			\bout{n}{V} P \by{\bactout{n}{V}} P
		}
		\and
		\inferrule[\ltsrule{Rv}]{
		}{
			\binp{n}{x} P \by{\bactinp{n}{V}} P\subst{V}{x}
		}
		\and
		\inferrule[\ltsrule{Sel}]{
		}{
			\bsel{s}{l}{P} \by{\bactsel{s}{l}} P
		}
		\and
		\inferrule[\ltsrule{Bra}]{
		}{
			\bbra{s}{l_i:P_i}_{i \in I} \by{\bactbra{s}{l_j}} P_j \ (j\in I)
		}
		\and
		\inferrule[\ltsrule{Alpha}]{
			P \scong_\alpha Q
			\and
			Q\by{\ell} P'
		}{
			P \by{\ell} P'
		}
		\and
		\inferrule[\ltsrule{Res}]{
			P \by{\ell} P'
			\and
			n \notin \fn{\ell}
		}{
			\news{n} P \by{\ell} \news{n} P'
		}
		\and
		\inferrule[\ltsrule{New}]{
			P \by{\news{\widetilde{m}} \bactout{n}{V}} P'
			\and
			m \in \fn{V}
		}{
			\news{m} P \by{\news{m\cat\widetilde{m}'} \bactout{n}{V}} P'
		}
		\and
		\inferrule[\ltsrule{Par${}_L$}]{
			P \by{\ell} P'
			\and
			\bn{\ell} \cap \fn{Q} = \es
		}{
			P \Par Q \by{\ell} P' \Par Q
		}
		\and
		\inferrule[\ltsrule{Tau}]{
			P \by{\ell_1} P'
			\and
			Q \by{\ell_2} Q'
			\and
			\ell_1 \asymp \ell_2
		}{
			P \Par Q \by{\tau} \newsp{\bn{\ell_1} \cup \bn{\ell_2}}{P' \Par Q'}
		}
		\and
		\inferrule[\ltsrule{Rec}]{
			P\subst{\recp{X}{P}}{\rvar{X}} \by{\ell} P'
		}{
			\recp{X}{P}  \by{\ell} P'
		}
	\end{mathpar}
	%
	\caption{The Untyped LTS for \HOp processes. We omit rule $\ltsrule{Par${}_R$}$.  \label{fig:untyped_LTS}}
	%
\end{figure}

%%%%%%%%%%%%%%%%%%%% End LTS Figure %%%%%%%%%%%%%%%%%%%%%%%%%%%%%%%%%
%\paragraph{LTS over Untyped Processes.}
The labelled transition system
%(LTS) LTS
over \emph{untyped processes}
is given in
\figref{fig:untyped_LTS}. 
We write $P_1 \by{\ell} P_2$ with the usual meaning.
The rules are standard~\cite{KYHH2015,KY2015}.
A process with an output prefix can
interact with the environment with an output action that carries a value
$V$ (rule~$\ltsrule{Snd}$).  Dually, in rule $\ltsrule{Rv}$ a
receiver process can observe an input of an arbitrary value $V$.
Select and branch processes observe the select and branch
actions in rules $\ltsrule{Sel}$ and $\ltsrule{Bra}$, resp.
Rule $\ltsrule{Res}$ closes the LTS under restriction 
if the restricted name does not occur free in the
observable action. 
If a restricted name occurs free in
the carried value of an output action,
the process performs scope opening (rule~$\ltsrule{New}$).  
Rule~$\ltsrule{Rec}$ handles recursion unfolding.
Rule~$\ltsrule{Tau}$ 
states that two parallel processes which perform
dual actions can synchronise by an internal transition.
Rules $\ltsrule{Par${}_L$}$/$\ltsrule{Par${}_R$}$ 
and $\ltsrule{Alpha}$ close the LTS
under parallel composition and $\alpha$-renaming. 

\subsection{Environmental Labelled Transition System}
\label{ss:elts}

\noi \figref{fig:envLTS}
defines a labelled transition relation between 
a triple of environments, denoted
%
\[
	(\Gamma_1, \Lambda_1, \Delta_1) \by{\ell} (\Gamma_2, \Lambda_2, \Delta_2)
\]
%
It extends the LTSs
in \cite{KYHH2015,KY2015} 
to higher-order sessions. 
Notice that due to weakening %of shared environments 
we have 
$
	(\Gamma', \Lambda_1, \Delta_1) \hby{\ell} (\Gamma', \Lambda_2, \Delta_2)
$
if
$
	(\Gamma, \Lambda_1, \Delta_1) \hby{\ell} (\Gamma', \Lambda_2, \Delta_2)
$
%
\paragraph{Input Actions} 
are defined by 
rules~$\eltsrule{SRv}$ and $\eltsrule{ShRv}$.
In rule~$\eltsrule{SRv}$
the type of value $V$
and the type of the object associated to the session type on $s$ 
should coincide. 
The resulting type tuple must contain the environments 
associated to $V$. 
The
dual endpoint $\dual{s}$ cannot be
present in the session environment: if it were present
the only possible communication would be the interaction
between the two endpoints (cf. rule~$\eltsrule{Tau}$).
Rule~$\eltsrule{ShRv}$ is for shared names and follows similar principles.

\paragraph{Output Actions} are defined by rules~$\eltsrule{SSnd}$
and $\eltsrule{ShSnd}$.  
Rule $\eltsrule{SSnd}$ states the conditions for observing action
$\news{\widetilde{m}} \bactout{s}{V}$ on a type tuple 
$(\Gamma, \Lambda, \Delta\cdot \AT{s}{S})$. 
The session environment $\Delta$ with $\AT{s}{S}$ 
should include the session environment of the sent value $V$, 
{\em excluding} the session environments of names $m_j$ 
in $\widetilde{m}$ which restrict the scope of value $V$. 
Analogously, the linear variable environment 
$\Lambda'$ of $V$ should be included in $\Lambda$. 
Scope extrusion of session names in $\widetilde{m}$ requires
that the dual endpoints of $\widetilde{m}$ should appear in
the resulting session environment. Similarly for shared 
names in $\widetilde{m}$ that are extruded.  
All free values used for typing $V$ are subtracted from the
resulting type tuple. The prefix of session $s$ is consumed
by the action.
Rule $\eltsrule{ShSnd}$ is for output actions on shared names:
the name must be typed with $\chtype{U}$; conditions on $V$ are identical to those
on rule~$\eltsrule{SSnd}$.
%\NY{
Given a $V$ of type $U$, we sometimes annotate the output action 
$\news{\widetilde{m}} \bactout{n}{V}$
with the type of $V$ 
as $\news{\widetilde{m}} \bactout{n}{\AT{V}{U}}$.
%}

\paragraph{Other Actions}
Rules $\eltsrule{Sel}$ and $\eltsrule{Bra}$ describe actions for
select and branch.
%Both
%rules require the absence of the dual endpoint from the session
%environment.%, and the presence of the action labels in the type.
Rule $\eltsrule{Tau}$ defines
internal transitions: 
it keeps the session environment unchanged or 
reduces it (\defref{d:wtenvred}).

%A second environment LTS, denoted $\hby{\ell}$,
%is defined in the lower part of \figref{fig:envLTS}.
%The definition substitutes rules
%$\eltsrule{SRecv}$ and $\eltsrule{ShRecv}$
%of relation $\by{\ell}$ with rule $\eltsrule{RRcv}$.
%% the corresponding input cases
%%of $\by{\ell}$ with the definitions of $\hby{\ell}$.
%All other cases remain the same as the cases for
%relation $\by{\ell}$.
%Rule $\eltsrule{RRcv}$ restricts the higher-order input
%in relation $\hby{\ell}$;
%only characteristic processes and trigger processes
%are allowed to be received on a higher-order input.
%Names can still be received as in the definition of
%the $\by{\ell}$ relation.
%The conditions for input follow the conditions
%for the $\by{\ell}$ definition.

%%%%%%%%%%%%%%%%%%%% Environment LTS Figure %%%%%%%%%%%%%%%%%%%%%%%%%
\begin{figure}
\begin{mathpar}
	\inferrule[\eltsrule{SRv}]{
		\dual{s} \notin \dom{\Delta}
		\and
		\Gamma; \Lambda'; \Delta' \proves V \hastype U
	}{
		(\Gamma; \Lambda; \Delta \cat s: \btinp{U} S) \by{\bactinp{s}{V}} (\Gamma; \Lambda\cat\Lambda'; \Delta\cat\Delta' \cat s: S)
	}
	\and
	\inferrule[\eltsrule{ShRv}]{
		\Gamma; \es; \es \proves a \hastype \chtype{U}
		\and
		\Gamma; \Lambda'; \Delta' \proves V \hastype U
	}{
		(\Gamma; \Lambda; \Delta) \by{\bactinp{a}{{V}}} (\Gamma; \Lambda\cat\Lambda'; \Delta\cat\Delta')
	}
	\and
	\inferrule[\eltsrule{SSnd}]{
		\begin{array}{l}
			\Gamma \cat \Gamma'; \Lambda'; \Delta' \proves V \hastype U
			\and
			\Gamma'; \es; \Delta_j \proves m_j  \hastype U_j
			\and
			\dual{s} \notin \dom{\Delta}
			\\
			\Delta'\backslash \cup_j \Delta_j \subseteq (\Delta \cat s: S)
			\and
			\Gamma'; \es; \Delta_j' \proves \dual{m}_j  \hastype U_j'
			\and
			\Lambda' \subseteq \Lambda
		\end{array}
	}{
		(\Gamma; \Lambda; \Delta \cat s: \btout{U} S)
		\by{\news{\widetilde{m}} \bactout{s}{V}}
		(\Gamma \cat \Gamma'; \Lambda\backslash\Lambda'; (\Delta \cat s: S \cat \cup_j \Delta_j') \backslash \Delta')
	}
	\and
	\inferrule[\eltsrule{ShSnd}]{
		\begin{array}{l}
			\Gamma \cat \Gamma' ; \Lambda'; \Delta' \proves V \hastype U
			\and
			\Gamma'; \es; \Delta_j \proves m_j \hastype U_j
			\and
			\Gamma ; \es ; \es \proves a \hastype \chtype{U}
			\\
			\Delta'\backslash \cup_j \Delta_j \subseteq \Delta
			\and
			\Gamma'; \es; \Delta_j' \proves \dual{m}_j\hastype U_j'
			\and
			\Lambda' \subseteq \Lambda
		\end{array}
	}{
		(\Gamma ; \Lambda; \Delta) \by{\news{\widetilde{m}}
		\bactout{a}{V}}
		(\Gamma \cat \Gamma'; \Lambda\backslash\Lambda'; (\Delta \cat \cup_j \Delta_j') \backslash \Delta')
	}
	\and
	\inferrule[\eltsrule{Sel}]{
		\dual{s} \notin \dom{\Delta}
		\and
		j \in I
	}{
		(\Gamma; \Lambda; \Delta \cat s: \btsel{l_i: S_i}_{i \in I}) \by{\bactsel{s}{l_j}} (\Gamma; \Lambda; \Delta \cat s:S_j)
	}
	\and
	\inferrule[\eltsrule{Bra}]{
		\dual{s} \notin \dom{\Delta} \quad j \in I
	}{
		(\Gamma; \Lambda; \Delta \cat s: \btbra{l_i: T_i}_{i \in I}) \by{\bactbra{s}{l_j}} (\Gamma; \Lambda; \Delta \cat s:S_j)
	}
	\and
	\inferrule[\eltsrule{Tau}]{
		\Delta_1 \red \Delta_2 \vee \Delta_1 = \Delta_2
	}{
		(\Gamma; \Lambda; \Delta_1) \by{\tau} (\Gamma; \Lambda; \Delta_2)
	}
\end{mathpar}
%
\caption{Labelled Transition System for Typed Environments. 
\label{fig:envLTS}}
%
\end{figure}
%%%%%%%%%%%%%%%%%%%% End Environment LTS Figure %%%%%%%%%%%%%%%%%%%%%%

\begin{example}
	Consider environment tuple
	$
		(\Gamma; \es; s: \btout{\lhot{\btout{S} \tinact}} \tinact \cat s': S)
	$
	and typed value
	\[
		\Gamma; \es; s': S \cat m: \btinp{\tinact} \tinact \proves V \, \hastype \, \lhot{\btout{\tinact} \tinact}
	\]
	\noi with
	$
		V= \abs{x} \bout{x}{s'} \binp{m}{z} \inact
	$.
%
	\noi Let 
	$
		\Delta'_1=\set{\overline{m}: \btout{\tinact} \tinact}
	$
	and $U=\btout{\lhot{\btout{S} \tinact}} \tinact$.
	Then by $\eltsrule{SSnd}$, we can derive:
%
	\[
		(\Gamma; \es; s: \btout{\lhot{\btout{S} \tinact}} \tinact \cat s': S) \by{\news{m} \bactout{s}{V}} (\Gamma; \es; s: \tinact)
	\]
\end{example}

\noi
The typed LTS  combines
the LTSs in \figref{fig:untyped_LTS}
and \figref{fig:envLTS}. 

\begin{definition}[Typed Transition System]
	\label{d:tlts}\rm
	A {\em typed transition relation} is a typed relation
	$\horel{\Gamma}{\Delta_1}{P_1}{\by{\ell}}{\Delta_2}{P_2}$
	where:
%
	\begin{enumerate}
		\item
				$P_1 \by{\ell} P_2$ and 
		\item
				$(\Gamma, \emptyset, \Delta_1) \by{\ell} (\Gamma, \emptyset, \Delta_2)$ 
				with $\Gamma; \emptyset; \Delta_i \proves P_i \hastype \Proc$ ($i=1,2$).
%\dk{We sometimes annotated the output action with
%the type of value $V$ as in $\widetilde{m} \bactout{n}{V: U}$.}
%
% Efficient 
%\item 
%$\horel{\Gamma}{\Delta_1}{P_1}{\hby{\ell}}{\Delta_2}{P_2}$
%whenever: 
%$P_1 \by{\ell} P_2$, 
%$(\Gamma, \emptyset, \Delta_1) \hby{\ell} (\Gamma, \emptyset, \Delta_2)$, 
%and $\Gamma; \emptyset; \Delta_i \proves P_i \hastype \Proc$ 
%($i=1,2$)
	\end{enumerate}
%
	We extend to $\By{}$ and $\By{\hat{\ell}}$ 
	where we write  $\By{}$ for the reflexive and transitive closure of $\by{}$,
	$\By{\ell}$ for the transitions $\By{}\by{\ell}\By{}$, and $\By{\hat{\ell}}$
	for $\By{\ell}$ if $\ell\not = \tau$ otherwise $\By{}$. 
\end{definition}

\subsection{Reduction-Closed, Barbed Congruence ($\cong$)}
\label{subsec:rc}
\noi We now define \emph{typed relations} and \emph{contextual equivalence} (i.e., barbed congruence).  
We first define \emph{confluence}
over session environments $\Delta$:
%
\begin{definition}[Session Environment Confluence]
	Two session environments $\Delta_1$ and $\Delta_2$
	are confluent, denoted $\Delta_1 \bistyp \Delta_2$,
	if there exists $\Delta$ such that
	i)~$\Delta_1 \red^\ast \Delta$ and ii~$\Delta_2 \red^\ast \Delta$
	(here we write $\red^\ast$ for the multi-step reduction in \defref{d:wtenvred}).
\end{definition}

\begin{definition}[Typed Relation]
	We say that
	\[
		\Gamma; \emptyset; \Delta_1 \proves P_1 \hastype \Proc\ \Re \ \Gamma; \emptyset; \Delta_2 \proves P_2 \hastype \Proc
	\]
%
	\noi
	is a {\em typed relation} whenever:
	\begin{enumerate}
		\item	$P_1$ and $P_2$ are closed;
		\item	$\Delta_1$ and $\Delta_2$ are balanced; and
		\item	$\Delta_1 \bistyp \Delta_2$.
	\end{enumerate}
%
	We write
	\[
		\horel{\Gamma}{\Delta_1}{P_1}{\ \Re \ }{\Delta_2}{P_2}
	\]
%
	for the typed relation 
	\[
		\Gamma; \emptyset; \Delta_1 \proves P_1 \hastype \Proc\ \Re \ \Gamma; \emptyset; \Delta_2 \proves P_2 \hastype \Proc
	\]
\end{definition}

\noi Typed relations relate only closed terms whose
session environments are balanced  and confluent.
Next we define  {\em barbs}~\cite{MiSa92}
with respect to types. 

\begin{definition}[Barbs]\rm
	Let $P$ be a closed process. We write
	\begin{enumerate}
		\item
				\begin{enumerate}
					\item	$P \barb{n}$ if $P \scong \newsp{\tilde{m}}{\bout{n}{V} P_2 \Par P_3}$, with $n \notin \tilde{m}$.
					\item	Also: $P \Barb{n}$ if $P \red^* \barb{n}$.
				\end{enumerate}

		\item	Similarly, we write
				\begin{enumerate}
					\item	$\Gamma; \emptyset; \Delta \proves P \barb{n}$ if
							$\Gamma; \emptyset; \Delta \proves P \hastype \Proc$ with $P \barb{n}$ and $\dual{n} \notin \Delta$.
					\item	Also: $\Gamma; \emptyset; \Delta \proves P \Barb{n}$ if $P \red^* P'$ and
							$\Gamma; \emptyset; \Delta' \proves P' \barb{n}$.
				\end{enumerate}
	\end{enumerate}
\end{definition}

\noi A barb $\barb{n}$ is an observable on an output prefix with subject $n$;
a weak barb $\Barb{n}$ is a barb after a number of reduction steps.
Typed barbs $\barb{n}$ (resp.\ $\Barb{n}$)
occur on typed processes $\Gamma; \emptyset; \Delta \proves P \hastype \Proc$.
When $n$ is a session name we require that its dual endpoint $\dual{n}$ is not
present in the session environment $\Delta$.

To define a congruence relation, we introduce the family $\C$ of contexts:

\begin{definition}[Context]
	Context $\C$ is defined over the syntax:
%
	\begin{eqnarray*}
		\C & ::= & \hole \bnfbar \bout{u}{V} \C \bnfbar \binp{u}{x} \C \bnfbar \bout{u}{\lambda x.\C} P \bnfbar \news{n} \C
		(\lambda x.\C)u \bnfbar \recp{X}{\C}\\ 
		& \bnfbar & \C \Par P \bnfbar P \Par \C \bnfbar \bsel{u}{l} \C \bnfbar \bbra{u}{l_1:P_1,\cdots,l_i:\C,\cdots,l_n:P_n}
	\end{eqnarray*}
%
	Notation $\context{\C}{P}$ denotes the result of substituting 
	the hole $\hole$ in $\C$ with process $P$.
\end{definition}

\noi The first behavioural relation we define is reduction-closed, barbed congruence \cite{HondaKYoshida95}. 

\begin{definition}[Reduction-Closed, Barbed Congruence]\rm
\label{def:rc}
	Typed relation
	\[
		\horel{\Gamma}{\Delta_1}{P_1}{\ \Re\ }{\Delta_2}{P_2}
	\]
	is a {\em reduction-closed, barbed congruence} whenever:
%
	\begin{enumerate}[1)]
		\item
				\begin{enumerate}
					\item	If $P_1 \red P_1'$ then there exist $P_2', \Delta_2'$ such that $P_2 \red^* P_2'$ and\\
							$\horel{\Gamma}{\Delta_1'}{P_1'}{\ \Re\ }{\Delta_2'}{P_2'}$;
					\item	and the symmetric case;
				\end{enumerate}
		\item
				\begin{enumerate}
					\item	If $\Gamma;\Delta_1 \proves P_1 \barb{n}$ then $\Gamma;\Delta_2 \proves P_2 \Barb{n}$;
					\item	and the symmetric case;
				\end{enumerate}

		\item	For all $\C$, there exist $\Delta_1'',\Delta_2''$:
				$\horel{\Gamma}{\Delta_1''}{\context{\C}{P_1}}{\ \Re\ }{\Delta_2''}{\context{\C}{P_2}}$; 
	\end{enumerate}
%
	The largest such relation is denoted with $\cong$.
\end{definition}

\subsection{Context Bisimilarity ($\wbc$)}
\label{subsec:bisimulation}

\noi 
Following Sangiorgi~\cite{San96H}, 
we now define the standard (weak) context bisimilarity. 
%
\begin{definition}[Context Bisimilarity]\rm
	\label{def:wbc}
	A typed relation $\Re$ is {\em a context bisimulation} if
	for all $\horel{\Gamma}{\Delta_1}{P_1}{\ \Re \ }{\Delta_2}{Q_1}$,
%
	\begin{enumerate}[1)] 
		\item Whenever 
			$\horel{\Gamma}{\Delta_1}{P_1}{\by{\news{\widetilde{m_1}} \bactout{n}{V_1}}}{\Delta_1'}{P_2}$,
			there exist $Q_2$, $V_2$, $\Delta'_2$
			such that $\horel{\Gamma}{\Delta_2}{Q_1}{\By{\news{\widetilde{m_2}} \bactout{n}{V_2}}}{\Delta_2'}{Q_2}$ and 
			for all $R$ with $\fv{R}=x$:
%
\[
			\horel{\Gamma}{\Delta_1''}{\newsp{\widetilde{m_1}}{P_2 \Par R\subst{V_1}{x}}}
			{\ \Re\ }
			{\Delta_2''}{\newsp{\widetilde{m_2}}{Q_2 \Par R\subst{V_2}{x}}};
\]  
%\item	$\forall \news{\widetilde{m_1}'} \bactout{n}{\widetilde{m_1}}$ such that
%			\[
%				\horel{\Gamma}{\Delta_1}{P_1}{\by{\news{\widetilde{m_1}'} \bactout{n}{\widetilde{m_1}}}}{\Delta_1'}{P_2}
%			\]
%			implies that $\exists Q_2, \widetilde{m_2}$ such that
%			\[
%				\horel{\Gamma}{\Delta_2}{Q_1}{\By{\news{\widetilde{m_2}'} \bactout{n}{\widetilde{m_2}}}}{\Delta_2'}{Q_2}
%			\]
%			and $\forall R$ with $\widetilde{x} = \fn{R}$, 
%			then
%			\[
%				\horel{\Gamma}{\Delta_1''}{\newsp{\widetilde{m_1}'}{P_2 \Par R \subst{\widetilde{m_1}}{\widetilde{x}}}}
%				{\ \Re \ }
%				{\Delta_2''}{\newsp{\widetilde{m_2}'}{Q_2 \Par R \subst{\widetilde{m_2}}{\widetilde{x}}}}
%			\]
		\item	
				For all $\horel{\Gamma}{\Delta_1}{P_1}{\by{\ell}}{\Delta_1'}{P_2}$ such that 
				$\ell$ is not an output, there exist $Q_2$, $\Delta'_2$ such that 
				$\horel{\Gamma}{\Delta_2}{Q_1}{\By{\hat{\ell}}}{\Delta_2'}{Q_2}$
				and
				$\horel{\Gamma}{\Delta_1'}{P_2}{\ \Re \ }{\Delta_2'}{Q_2}$; and  

		\item
				The symmetric cases of 1 and 2.                
	\end{enumerate}
%
	The largest such bisimulation is called \emph{context bisimilarity} and is denoted by $\wbc$.
\end{definition}

\noi As hinted at in
%\secref{subsec:intro:bisimulation}, %\secref{sec:overview},
the Introduction,
in the general case,
context bisimilarity 
is hard to compute.
Below we introduce \emph{higher-order bisimulation} and \emph{characteristic bisimulation},
which are meant to offer a \emph{tractable} proof technique over session typed
processes with higher-order communication.
%$\hwb$ and  $\fwb$.
%due to: (1) the universal
%quantification over contexts in the output case;
%and (2) a higher-order input prefix which can observe
%infinitely many different input actions (since
%infinitely many different processes can match
%the session type of an input prefix).

\subsection{Higher-Order ($\hwb$) and  Characteristic  Bisimilarity ($\fwb$)}
\label{ss:hwb}

\noi 
We formalise the ideas given in % \secref{sec:overview}.
the introduction.
%Our main result is \thmref{the:coincidence}.
We define characteristic processes/values:

\begin{definition}[Characteristic Process and Values]\rm
\label{def:char}
	Let $u$ and $U$ be a name and a type, respectively.
	\figref{fig:char} defines the {\em characteristic process} 
	$\mapchar{U}{u}$ and the {\em characteristic value} $\omapchar{U}$.
\end{definition}

%%%%%%%%%%%%%%%%%%%%%%%% Characteristic Process Figure %%%%%%%%%%%%%%%%%%%%%%%%%
\begin{figure}
\[
	\begin{array}{rclcrcl}
		\mapchar{\btinp{U} S}{u}
		&\defeq&
		\binp{u}{x} (\mapchar{S}{u} \Par \mapchar{U}{x})
		&&
		\mapchar{\btout{U} S}{u}
		&\defeq&
		\bout{u}{\omapchar{U}} \mapchar{S}{u} %& & n \textrm{ fresh}
		\\

		\mapchar{\btsel{l : S}}{u}
		& \defeq &
		\bsel{u}{l} \mapchar{S}{u}
		&&
		\mapchar{\btbra{l_i: S_i}_{i \in I}}{u}
		& \defeq &
		\bbra{u}{l_i: \mapchar{S_i}{u}}_{i \in I}
		\\

		\mapchar{\tvar{t}}{u}
		&\defeq&
		\varp{X}_{\vart{t}}
		& & 
		\mapchar{\trec{t}{S}}{u}
		&\defeq&
		\recp{X_{\vart{t}}}{\mapchar{S}{u}}
		\\

		\mapchar{\tinact}{u}
		& \defeq &
		\inact
		& & 
		\mapchar{\chtype{S}}{u} 
		&\defeq&
		\bout{u}{\omapchar{S}} \inact
		\\

		\mapchar{\chtype{L}}{u}
		&\defeq&
		\bout{u}{\omapchar{L}} \inact
		&&
		\mapchar{\shot{U}}{u}
		&\defeq &
		\mapchar{\lhot{U}}{u}
		\, \defeq \,
		\appl{u}{\omapchar{U}}
	\end{array}
	\]
%
	\[
	\begin{array}{c}
		\omapchar{S}  \defeq  s ~~ (s \textrm{ fresh})
		\qquad
		\omapchar{\chtype{S}} \defeq \omapchar{\chtype{L}} \defeq a ~~ (a \textrm{ fresh})
		\qquad
		\omapchar{\shot{U}} \defeq \omapchar{\lhot{U}} \,\defeq\, \abs{x}{\mapchar{U}{x}}
	\end{array}
	\]
%
\caption{Characteristic Processes (top) and Characteristic Values (bottom).\label{fig:char}}
%
\end{figure}

%%%%%%%%%%%%%%%%%%%%%%%% End Characteristic Process Figure %%%%%%%%%%%%%%%%%%%%%


\noi We can verify that characteristic processes/values  
do inhabit their associated type.

\begin{proposition}[Characteristic Processes/Values Inhabit Their Types]
	\begin{enumerate}
		\item	
				Let $S$ be a session type.
				Then $\Gamma; \es; \Delta \cat s: S \proves \mapchar{S}{s} \hastype \Proc$.
		\item	
				Also, let $\chtype{U}$ be a first-order (channel) type.
				Then $\Gamma \cat a: \chtype{U}; \es; \Delta \proves \mapchar{\chtype{U}}{a} \hastype \Proc$.
	\end{enumerate}
\end{proposition}

\begin{proof}[Sketch]
	By induction on the definition of $\mapchar{S}{u}$
	and $\mapchar{U}{u}$. 
	\qed
\end{proof}

The following example motivates the refined 
LTS explained in %\secref{sec:overview}.
the introduction.


%\begin{proof}
%	By induction on the definition of $\mapchar{U}{n}$.
%\end{proof}

%\begin{example}
%	Consider the type $U = \shot{(\shot{(\shot{(\btout{S} \tinact)})})}$.
%	It is easy to verify that %the characteristic process %of the above type is:
%	$\mapchar{\btinp{U} \inact}{s} = \binp{s}{x} \appl{x}{(\abs{y}{\appl{y}{n})}}$, for some fresh $n$.
%%%
%%	\begin{eqnarray*}
%%		\mapchar{\btinp{U} \inact}{s} = && \binp{s}{x} \mapchar{\shot{(\shot{(\shot{(\btout{S} \tinact)})})}}{x}\\
%%		= && \binp{s}{x} \appl{x}{\omapchar{\shot{(\shot{(\btout{S} \tinact)})}}}\\
%%		= && \binp{s}{x} \appl{x}{(\abs{y}{\mapchar{\shot{(\btout{S} \tinact)}}{y}})}\\
%%		= && \binp{s}{x} \appl{x}{(\abs{y}{\appl{y}{\omapchar{\btout{S} \tinact}})}}\\
%%		= && \binp{s}{x} \appl{x}{(\abs{y}{\appl{y}{n})}}\\
%%	\end{eqnarray*}
%%%
%\end{example}


\begin{example}[The Need for Refined Typed LTS]
	\label{ex:motivation}
	We show that observing a characteristic value
	input alone is not enough
	to define a sound bisimulation closure.
	Consider   processes % $P_1, P_2$:
%
	\begin{eqnarray}
		P_1 = \binp{s}{x} (\appl{x}{s_1} \Par \appl{x}{s_2}) 
		&\qquad \qquad& 
		P_2 = \binp{s}{x} (\appl{x}{s_1} \Par \binp{s_2}{y} \inact) 
		\label{equ:6}
	\end{eqnarray}
%
	where
%
	\[
		\Gamma; \es; \Delta \cat s: \btinp{\shot{(\btinp{C} \tinact)}} \tinact \proves P_i \hastype \Proc \qquad (i \in \set{1,2})
	\]
%
	If $P_1$ and $P_2$ input 
	the characteristic value $\omapchar{\shot{(\btinp{C} \tinact)}} = \abs{x}{\binp{x}{y} \inact}$, 
	and substitute over $x$ then they evolve into:
%
	\[
		\Gamma; \es; \Delta \proves \binp{s_1}{y} \inact \Par \binp{s_2}{y} \inact \hastype \Proc
	\]
	\noi therefore becoming context bisimilar.
	However, the processes in (\ref{equ:6}) 
	are clearly {\em not} context bisimilar: many input actions
	may be used to distinguish them.
	For example, if  $P_1$ and $P_2$ input 
	$\abs{x} \newsp{s}{\bout{a}{s} \binp{x}{y} \inact}$ with
	$\Gamma; \es; \Delta \proves s \hastype \tinact$,
	then their derivatives are not bisimilar. 

	Observing only the characteristic value 
	results in an under-discriminating bisimulation.
	However, if a trigger value
	$\abs{{x}}{\binp{t}{y} (\appl{y}{{x}})}$ 
	is received on $s$, 
	we can distinguish $P_1$, $P_2$ in~\eqref{equ:6}:  
	%
	\begin{eqnarray*}
%		\Gamma; \es; \Delta &\proves& 
		P_1 &\By{\bactinp{s}{\abs{{x}}{\binp{t}{y} (\appl{y}{{x}})}}}& \binp{t}{x} (\appl{x}{s_1}) \Par \binp{t}{x} (\appl{x}{s_2})
%		\hastype \Proc
		\qquad \mbox{and}\\
%	\Gamma; \es; \Delta &\proves& 
		P_2 &\By{\bactinp{s}{\abs{{x}}{\binp{t}{y} (\appl{y}{{x}})}}}& \binp{t}{x} (\appl{x}{s_1}) \Par \binp{s_2}{y} \inact 
%	\hastype \Proc
%	\qquad \textrm{($\ell = $)}
	\end{eqnarray*}
%
%\normalsize
%\noi resulting two distinct processes.  
%
%\noi where 
%$\ell = s?\ENCan{\abs{{x}}{\binp{t}{y} (\appl{y}{{x}})}}$.
	One question is whether the trigger value is enough
	to distinguish two processes (hence no need of characteristic values). % as the input. 
	This is not the case: the trigger value
	alone also results in an under-discriminating bisimulation relation.
	In fact, the  trigger value can be observed on any input prefix
	of {\em any type}. For example, consider processes
%
	\begin{eqnarray}
%		\Gamma; \es; \Delta \proves 
		\newsp{s}{\binp{n}{x} (\appl{x}{s}) \Par \bout{\dual{s}}{\abs{x} R_1} \inact} 
%		\hastype \Proc
		\qquad \mbox{and} \qquad
%		\Gamma; \es; \Delta \proves 
		\newsp{s}{\binp{n}{x} (\appl{x}{s}) \Par \bout{\dual{s}}{\abs{x} R_2} \inact} 
%		\hastype \Proc
		\label{equ:7}\label{equ:8}
	\end{eqnarray}
%
	\noi If these processes %in \eqref{equ:7}/\eqref{equ:8}
	input the trigger value, we obtain: % they evolved to 
%
	\begin{eqnarray*}
%		\Gamma; \es; \Delta \proves 
		\newsp{s}{\binp{t}{x} (\appl{x}{s}) \Par \bout{\dual{s}}{\abs{x} R_1} \inact} 
%		\hastype \Proc
		\qquad \mbox{ and } \qquad
%		\\
%		\Gamma; \es; \Delta \proves 
		\newsp{s}{\binp{t}{x} (\appl{x}{s}) \Par \bout{\dual{s}}{\abs{x} R_2} \inact}
%		\hastype \Proc
	\end{eqnarray*}

	\noi thus we can easily derive a bisimulation closure if we 
	assume a bisimulation definition that allows only trigger value input.
%
%	\noi It is easy to obtain a closure if allow only the
%	trigger value as the input value. 
	But if processes in \eqref{equ:7}
	input the characteristic value $\abs{z}{\binp{z}{x} (\appl{x}{m})}$,  
	then they would become, under appropriate $\Gamma$ and $\Delta$:
%
	\begin{eqnarray*}
		\Gamma; \es; \Delta \proves \newsp{s}{\binp{s}{x} (\appl{x}{m}) \Par \bout{\dual{s}}{\abs{x} R_i} \inact}\ \wbc\ \Delta \proves R_i \subst{m}{x}
	\qquad (i=1,2)
%	\\
%	\Gamma; \es; \Delta \proves \newsp{s}{\binp{s}{x} (\appl{x}{m}) \Par \bout{\dual{s}}{\abs{x} Q} \inact} \wbc \Delta \proves Q \subst{m}{x}
	\end{eqnarray*}
%
	\noi which are not bisimilar if $R_1 \subst{m}{x} \not\wbc R_2 \subst{m}{x}$.
	%\qed
	In conclusion, these examples explain a need of both 
	trigger and characteristic values 
	as an input observation in the input transition relation (\eltsrule{RRcv})
	which will be defined in Definition~\ref{def:rlts}.
	\qed
\end{example}

%\noi We define the \emph{refined} typed LTS. 
\noi As explained in 
%\secref{sec:overview}, 
the introduction,
we define the
\emph{refined} typed LTS
by considering a transition rule for input in which admitted values are
trigger or characteristic values or names:

%\noi We define the \emph{refined} typed LTS. 
%As explained in \secref{subsec:intro:bisimulation}, this new LTS is defined 
%by considering a transition rule for input in which admitted values are
%trigger or characteristic values:
%\dk{(assume extension of the structural
%congruence to acommodate values: i) $\abs{x}{P} \scong \abs{x}{Q}$ if
%$P \scong Q$) and ii) $n \scong m$ if $n = n$)}: 

\begin{definition}[Refined Typed Labelled Transition Relation]
	\label{def:rlts}
	We define the environment transition rule for input actions 
	%restricted environment transition relation using the
	%following rule %using the environment transition relation defined in 
	using the input rules in \figref{fig:envLTS}:
	\begin{mathpar}
		\inferrule[\eltsrule{RRcv}]{
			(\Gamma_1; \Lambda_1; \Delta_1) \by{\bactinp{n}{V}} (\Gamma_2; \Lambda_2; \Delta_2)
			\and
			V = m 
			\vee  V \scong \omapchar{U}%\abs{{x}}{\map{U}^{{x}}}
			\vee V  \scong \abs{{x}}{\binp{t}{y} (\appl{y}{{x}})}
			\textrm{ {\small with $t$ fresh}} 
		}{
			(\Gamma_1; \Lambda_1; \Delta_1) \hby{\bactinp{n}{V}} (\Gamma_2; \Lambda_2; \Delta_2)
		}
	\end{mathpar}
	\noi Rule $\eltsrule{RRcv}$ is defined on top
	of rules $\eltsrule{SRv}$ and $\eltsrule{ShRv}$
	in \figref{fig:envLTS}.
%	uses the environment transition
%	$(\Gamma, \Lambda_1, \Delta_1) \hby{\ell} (\Gamma, \Lambda_2, \Delta_2)$
%	in \figref{fig:envLTS}. 
We  use the non-receiving rules in \figref{fig:envLTS}
	together with rule $\eltsrule{RRcv}$
	to define 
	$\horel{\Gamma}{\Delta_1}{P_1}{\hby{\ell}}{\Delta_2}{P_2}$
	as in \defref{d:tlts}.
%	by replacing $\by{\ell}$ by $\hby{\ell}$ in \defref{d:tlts}. 
\end{definition}

\noi Notice that
$\horel{\Gamma}{\Delta_1}{P_1}{\hby{\ell}}{\Delta_2}{P_2}$ (refined transition) implies  
$\horel{\Gamma}{\Delta_1}{P_1}{\by{\,\ell\,}}{\Delta_2}{P_2}$ (ordinary transition).
Below we sometimes write  
$\hby{\news{\widetilde{m}} \bactout{n}{\AT{V}{U}}}$
when the type of $V$ is~$U$.

%See \exref{ex:motivation} for the reason why {\em both} 
%the trigger values ($\lambda x.\binp{t}{y} (\appl{y}{{x}})$) 
%and characteristic values ($\lambda x.\map{U}^{{x}}$) are required 
%to define the following two bisimulations. 

\paragraph{Characterisation of contextual bisimilarity.}
We define \emph{higher-order Bisimularity} and 
\emph{characteristic bisimilarity} which are
two tractable bisimilarity relations of $\HOp$.
As explained in \secref{sec:overview},
the two bisimulations 
use trigger processes (cf.~\eqref{eq:4}):
%the key difference between them is in the trigger processes they use:
\begin{eqnarray*}
	\htrigger{t}{V_1}	& \defeq &	\hotrigger{t}{x}{s}{V}						\label{eqb:0} \\
	\ftrigger{t}{V}{U}	& \defeq &	\fotrigger{t}{x}{s}{\btinp{U} \tinact}{V}	\label{eqb:4}
\end{eqnarray*}
\noi
Notice that while 
in~\eqref{eqb:0} there is a higher-order input on $t$, 
in~\eqref{eqb:4} variable $x$ does not play any role.
% \dk{The two bisimulations differ on the fact that
%they use the different 
%trigger processes: $\htrigger{t}{V}$ and $\ftrigger{t}{V}{U}$.}

\begin{definition}[Higher-Order Bisimilarity]
	\label{d:hwb}
	A typed relation $\Re$ is a {\em  HO bisimulation} if 
	for all $\horel{\Gamma}{\Delta_1}{P_1}{\ \Re \ }{\Delta_2}{Q_1}$ 
%
	\begin{enumerate}[1)]
		\item 
				Whenever 
				$\horel{\Gamma}{\Delta_1}{P_1}{\hby{\news{\widetilde{m_1}} \bactout{n}{V_1}}}{\Delta_1'}{P_2}$, there exist 
				$Q_2$, $V_2$, $\Delta'_2$ such that 
				$\horel{\Gamma}{\Delta_2}{Q_1}{\Hby{\news{\widetilde{m_2}} \bactout{n}{V_2}}}{\Delta_2'}{Q_2}$ and, for fresh $t$, 
				\[
					\begin{array}{lrlll}
						\Gamma; \Delta''_1  \proves  {\newsp{\widetilde{m_1}}{P_2 \Par \htrigger{t}{V_1}}}
						\ \Re\ 
						\Delta''_2 \proves {\newsp{\widetilde{m_2}}{Q_2 \Par \htrigger{t}{V_2}}}
					\end{array}
				\]
		\item	
				For all $\horel{\Gamma}{\Delta_1}{P_1}{\hby{\ell}}{\Delta_1'}{P_2}$ such that 
				$\ell$ is not an output, 
				there exist $Q_2$, $\Delta'_2$ such that 
				$\horel{\Gamma}{\Delta_2}{Q_1}{\Hby{\hat{\ell}}}{\Delta_2'}{Q_2}$
				and
				$\horel{\Gamma}{\Delta_1'}{P_2}{\ \Re \ }{\Delta_2'}{Q_2}$; and 

		\item	The symmetric cases of 1 and 2.                
	\end{enumerate}
%
	The largest such bisimulation is called \emph{higher-order bisimilarity} and is denoted by $\hwb$.
\end{definition}

 
%We define characteristic bisimilarity:
% is given using characteristic trigger processes. 

\begin{definition}[Characteristic Bisimilarity]
\label{d:fwb}
	A typed relation $\Re$ is a {\em  characteristic bisimulation} if 
	for all $\horel{\Gamma}{\Delta_1}{P_1}{\ \Re \ }{\Delta_2}{Q_1}$, 
%
	\begin{enumerate}[1)]
		\item 
				Whenever 
				$\horel{\Gamma}{\Delta_1}{P_1}{\hby{\news{\widetilde{m_1}} \bactout{n}{\dk{V_1: U}}}}{\Delta_1'}{P_2}$ 
				then there exist 
				$Q_2$, $V_2$, $\Delta'_2$ such that 
				$\horel{\Gamma}{\Delta_2}{Q_1}{\Hby{\news{\widetilde{m_2}}\bactout{n}{\dk{V_2: U}}}}{\Delta_2'}{Q_2}$
				and, for fresh $t$,
%
				\[
					\Gamma; \Delta''_1  \proves  {\newsp{\widetilde{m_1}}{P_2 \Par \ftrigger{t}{V_1}{U_1}}}
	 				\ \Re\ 
					\Delta''_2 \proves {\newsp{\widetilde{m_2}}{Q_2 \Par \ftrigger{t}{V_2}{U_2}}}
				\]

		\item	
				For all $\horel{\Gamma}{\Delta_1}{P_1}{\hby{\ell}}{\Delta_1'}{P_2}$ such that 
				$\ell$ is not an output, there exist $Q_2$, $\Delta'_2$ such that 
				$\horel{\Gamma}{\Delta_2}{Q_1}{\Hby{\hat{\ell}}}{\Delta_2'}{Q_2}$
				and
				$\horel{\Gamma}{\Delta_1'}{P_2}{\ \Re \ }{\Delta_2'}{Q_2}$; and 

		\item	The symmetric cases of 1 and 2.                
	\end{enumerate}
%
	The largest such bisimulation is called \emph{characteristic bisimilarity} and is denoted by $\fwb$.
\end{definition}

\begin{lemma}
	\label{lem:wb_eq_wbf}
	$\fwb\ =\ \hwb$
\end{lemma}

\begin{proof}[Sketch]
	We show that
	trigger processes,
	$\fotrigger{t}{s}{x}{\btinp{U} \tinact}{V}$
	and
	$\hotrigger{t}{s}{x}{V}$,
	exhibit similar behaviour.
	The full proof is found in \lemref{app:lem:wb_eq_wbf}
	in \appref{app:beh}.
	\qed
\end{proof}

\begin{lemma}
	\label{lem:wb_is_wbc}
	$\hwb\ \subseteq\ \wb$
\end{lemma}

\begin{proof}[Sketch]
	\qed
\end{proof}

\begin{lemma}
	\label{lem:wbc_is_cong}
	$\wb \subseteq \cong$
\end{lemma}

\begin{proof}[Sketch]
	We show that $\wb$ satisfies the defining
	properties of $\cong$.
	It is easy to show that $\wb$ is reduction-closed
	and barb preserving. The challenging part is
	to show that $\wb$ is a congruence; we construct
	a congruence closure and we show that it is a
	context bisimulation. The full proof is found in
	\lemref{app:lem:wbc_is_cong} in \appref{app:beh}.
	\qed
\end{proof}

\begin{lemma}
	\label{lem:cong_is_wb}
	$\cong \subseteq \hwb$
\end{lemma}

\begin{proof}[Sketch]
	We prove the above result by exploiting
	a set of test processes defined in \defref{app:def:definibility}.
	These test processes allow for checking the interaction
	of reduction-closed barb congruence processes
	so to associate them with labelled transition semantics
	and check for the defining properties of $\hwb$.
	The full details can be found in \lemref{app:lem:cong_is_wb} in \appref{app:beh}.
	\qed
\end{proof}

\begin{theorem}[Coincidence]\rm
%	\label{the:coincidence}
%$\cong$, $\wbc$, $\hwb$ and $\fwb$ coincide in $\CAL\in \{\HOp, \HO\}$
%and 
%$\cong$, $\wbc$ and $\fwb$ coincide in $\CAL\in \{\HOp, \HO, \sessp\}$. 
	\label{the:coincidence}
	$\cong$, $\wbc$, $\hwb$ and $\fwb$ coincide in $\HOp$. 
\end{theorem}

\begin{proof}
	The proof is a direct consequence from
	\lemref{lem:wb_eq_wbf}, \lemref{lem:wb_is_wbc},\lemref{lem:wbc_is_cong}, and
	\lemref{lem:cong_is_wb}.
	\qed
\end{proof}

%\begin{proof}[Sketch]
%	We use \emph{higher-order bisimilarity} ($\hwb$, see \defref{def:bisim}), 
%	an auxiliary equivalence that is defined as $\fwb$ but by
%	using  trigger processes with higher-order communication (cf.~\eqref{equ:2}).
%	We first show that $\fwb$ and $\hwb$ coincide by using~\propref{lem:tau_inert}; 
%	then, we show that $\hwb$ coincides with $\wbc$ and $\cong$. 
%	A key result is a  substitution lemma which simplifies reasoning 
%	for $\hwb$ by exploiting characteristic processes/values.
%	See \appref{app:beh} and~\cite{KouzapasPY15} for full details.
%\end{proof}



%\smallskip 

%\begin{definition}[Characteristic Bisimilarity]\rm
%	\label{d:fwb}
%	Characteristic bisimilarity, denoted by $\fwb$, is defined \jpc{by} replacing 
%	Clause 1) in \defref{d:hbw} with the following clause:\\[1mm]
%	Whenever 
%	$\horel{\Gamma}{\Delta_1}{P_1}{\hby{\news{\widetilde{m_1}} \bactout{n}{\dk{V_1: U}}}}{\Delta_1'}{P_2}$ %with $\Gamma; \es; \Delta \proves V_1 \hastype U$,  
%	then there exist 
%	$Q_2$, $V_2$, $\Delta'_2$ such that 
%	$\horel{\Gamma}{\Delta_2}{Q_1}{\Hby{\news{\widetilde{m_2}}\bactout{n}{\dk{V_2: U}}}}{\Delta_2'}{Q_2}$ %with $\Gamma; \es; \Delta' \proves V_2 \hastype U$,  
%	and, for fresh $t$, \\[1mm]
%	$\begin{array}{lrlll}
%	\!\!\Gamma; \Delta''_1  \proves  {\newsp{\widetilde{m_1}}{P_2 \Par 
%	\ftrigger{t}{V_1}{U_1}}}
%	\ \!\!\Re\!\!
%	\ \Delta''_2 \proves {\newsp{\widetilde{m_2}}{Q_2 \Par \ftrigger{t}{V_2}{U_2}}}
%\end{array}
%$
%\end{definition}
%
%\smallskip 

\noi Internal transitions associated to session interactions or  
$\beta$-reductions are deterministic.  
		
\begin{definition}[Deterministic Transition]\myrm
\label{def:dettrans}
	Let  $\Gamma; \es; \Delta \proves P \hastype \Proc$ be a balanced \HOp process. 
	Transition $\horel{\Gamma}{\Delta}{P}{\hby{\tau}}{\Delta'}{P'}$ is called
%
	\begin{enumerate}[$-$]
		\item 
				a {\em session transition} whenever the   transition $P \by{\tau} P'$ 
				is derived using rule~$\ltsrule{Tau}$ 
				(where $\subj{\ell_1}$ and $\subj{\ell_2}$ in the premise are dual endpoints), 
				possibly followed by uses of  $\ltsrule{Alpha}$, $\ltsrule{Res}$, $\ltsrule{Rec}$, or $\ltsrule{Par${}_L$}/
				\ltsrule{Par${}_R$}$.

				We write $\horel{\Gamma}{\Delta}{P}{\hby{\stau}}{\Delta'}{P'}$ to denote a session transition.
		
		\item
%				Transition $\horel{\Gamma}{\Delta}{P}{\hby{\tau}}{\Delta'}{P'}$ is called
				a {\em \betatran} whenever the transition $P \by{\tau} P'$
				is derived using rule $\ltsrule{App}$,
				possibly followed by uses of  $\ltsrule{Alpha}$, $\ltsrule{Res}$, $\ltsrule{Rec}$, or $\ltsrule{Par${}_L$}/
				\ltsrule{Par${}_R$}$.

				We write $\horel{\Gamma}{\Delta}{P}{\hby{\btau}}{\Delta'}{P'}$ to denote a $\beta$-transition.

		\item	Also, $\horel{\Gamma}{\Delta}{P}{\hby{\dtau}}{\Delta'}{P'}$ denotes
				either a session transition or a \betatran.
	\end{enumerate}
%
We write $\Hby{\dtau}$ to denote a (possibly empty) sequence of deterministic steps $\hby{\dtau}$.
\end{definition}

Deterministic transitions imply the $\tau$-inertness property, which
is a property that ensures behavioural invariance on deterministic
transitions.

\begin{proposition}[$\tau$-inertness]\myrm
	\label{lem:tau_inert}
	Let  $\Gamma; \es; \Delta \proves P \hastype \Proc$ be a balanced \HOp process.
	Then
%	\begin{enumerate}[1.]
%		\item	
	$\horel{\Gamma}{\Delta}{P}{\hby{\dtau}}{\Delta'}{P'}$ implies
	$\horel{\Gamma}{\Delta}{P}{\fwb}{\Delta'}{P'}$.
%		\item	$\horel{\Gamma}{\Delta}{P}{\Hby{\dtau}}{\Delta'}{P'}$ implies
%			$\horel{\Gamma}{\Delta}{P}{\wb}{\Delta'}{P'}$.
%	\end{enumerate}
\end{proposition}
%
\noi 
See App.~\ref{app:sub_tau_inert} for the proofs. 
Our main theorem follows: %typed bisimilarities collapse for \HOp processes. 
it allows us to use $\fwb$ as a tractable reasoning %is the most tractable 
technique for higher-order processes with sessions.

Using the above determinacy properties, we can state the following up-to technique.
We write $\Hby{\dtau}$ to denote a (possibly empty) sequence of deterministic steps 
$\hby{\dtau}$.


\begin{lemma}[Up-to Deterministic Transition]
	\label{lem:up_to_deterministic_transition}
	Let $\horel{\Gamma}{\Delta_1}{P_1}{\ \Re\ }{\Delta_2}{Q_1}$ such
	that if whenever:
%
	\begin{enumerate}[1.]
		\item	$\forall \news{\tilde{m_1}} \bactout{n}{V_1}$ such that
			$
				\horel{\Gamma}{\Delta_1}{P_1}{\hby{\news{\tilde{m_1}} \bactout{n}{V_1}}}{\Delta_3}{P_3}
			$
			implies that $\exists Q_2, V_2$ such that
			$
				\horel{\Gamma}{\Delta_2}{Q_1}{\Hby{\news{\tilde{m_2}} \bactout{n}{V_2}}}{\Delta_2'}{Q_2}
			$
			and
			$
				\horel{\Gamma}{\Delta_3}{P_3}{\Hby{\dtau}}{\Delta_1'}{P_2}
			$
			and for fresh $t$:\\
			\[
				\horel{\Gamma}{\Delta_1''}{\newsp{\tilde{m_1}}{P_2 \Par \htrigger{t}{V_1}}}
				{\ \Re\ }
				{\Delta_2''}{}{\newsp{\tilde{m_2}}{Q_2 \Par \htrigger{t}{V_2}}}
%				\mhorel{\Gamma}{\Delta_1''}{\newsp{\tilde{m_1}}{P_2 \Par \hotrigger{t}{x}{s}{V_1}}}
%				{\ \Re\ }
%				{\Delta_2''}{}{\newsp{\tilde{m_2}}{Q_2 \Par \hotrigger{t}{x}{s}{V_2}}}
			\]
%
		\item	$\forall \ell \not= \news{\tilde{m}} \bactout{n}{V}$ such that
			$
				\horel{\Gamma}{\Delta_1}{P_1}{\hby{\ell}}{\Delta_3}{P_3}
			$
			implies that $\exists Q_2$  \\ such that 
			$
				\horel{\Gamma}{\Delta_1}{Q_1}{\hat{\Hby{\ell}}}{\Delta_2'}{Q_2}
			$
			and
			$
				\horel{\Gamma}{\Delta_3}{P_3}{\Hby{\dtau}}{\Delta_1'}{P_2}
			$
			and
			$\horel{\Gamma}{\Delta_1'}{P_2}{\ \Re\ }{\Delta_2'}{Q_2}$.

		\item	The symmetric cases of 1 and 2.
	\end{enumerate}
	Then $\Re\ \subseteq\ \hwb$.
\end{lemma}

%\smallskip 

\subsection{Running Example (\exref{exam:proc})}

Now we prove that  processes 
$\Client_1$ and $\Client_2$ 
in Example \ref{exam:proc}
are behaviourally equivalent.

\begin{proposition}\label{p:examp}
	Let
	\begin{eqnarray*}
		S &=& \btout{\rtype} \btinp{\Quote} \btsel{\accept: \btout{\creditc} \tinact, \reject: \tinact}\\
		\Delta &=& s_1: \btout{\lhot{S}} \tinact \cat s_2: \btout{\lhot{S}} \tinact
	\end{eqnarray*}
	Then
	$\horel{\es}{\Delta}{\Client_1}
	{\wbf}
	{\Delta}{\Client_2}$, where $\Client_1$, $\Client_2$ are defined in \exref{exam:proc}. 
\end{proposition}

\begin{proof}[Sketch]
	\noi We show a bisimulation closure by following transitions on each $\Client$.
	%We show the initial higher order transitions.
	See \appref{hotel_closure} for details.
	First, the characteristic process is given as:
	$\mapchar{\btinp{\lhot{S}} \tinact}{s} = \binp{s}{x} (\appl{x}{k})$.
	We show that the clients can simulate each other on
	the first two output transitions, that also generate the trigger
	processes:
%
\[
	\begin{array}{lll}
		&	\es; \es; \Delta \proves \Client_1
		&
			\by{\bactout{s_1}{\abs{x}{P_{xy} \subst{h_1}{y}}}}
			\qquad
			\by{\bactout{s_2}{\abs{x}{P_{xy} \subst{h_2}{y}}}}
		\\
		&	\es; \es; k_1: S \cat k_2: S \proves
		&
			\newsp{h_1, h_2}{\binp{\dual{h_1}}{x} \binp{\dual{h_2}}{y}
		\\
		&
		&	\If\ x \leq y\ \Then (\bsel{\dual{h_1}}{\accept} \bsel{\dual{h_2}}{\reject} \inact
			\Else \bsel{\dual{h_1}}{\reject} \bsel{\dual{h_2}}{\accept} \inact)
		\\
		&
		&	\Par \ftrigger{t_1}{\abs{x}{P_{xy} \subst{h_1}{y}}}{\lhot{S}} \Par \ftrigger{t_2}{\abs{x}{P_{xy} \subst{h_2}{y}}}{\lhot{S}}}
%		& \Par \binp{t_1}{x} \newsp{s}{\binp{s}{x} \appl{x}{k_1} \Par \bout{\dual{s}}{\abs{x}{P \subst{h_1}{y}}} \inact }\\
%		& \Par \binp{t_2}{x} \newsp{s}{\mapchar{\btinp{\lhot{S}}}{s} \Par \bout{\dual{s}}{\abs{x}{P \subst{h_2}{y}}} \inact }}
		\\[2mm]
		& \mbox{and}
		\\[2mm]
		&	\es; \es; \Delta \proves \Client_2
		&
			\by{\bactout{s_1}{\abs{x}{Q_1 \subst{h}{y}}}}
			\qquad
			\by{\bactout{s_2}{\abs{x}{Q_2 \subst{\dual{h}}{y}}}}
		\\
		&	\es; \es; k_1: S \cat k_2: S \proves & \newsp{h}{
			\ftrigger{t_1}{\abs{x}{Q_1 \subst{h}{y}}}{\lhot{S}} \Par \ftrigger{t_2}{\abs{x}{Q_2 \subst{\dual{h}}{y}}}{\lhot{S}}}
%		\binp{t_1}{x} \newsp{s}{\binp{s}{x} \appl{x}{k_1} \Par \bout{\dual{s}}{\abs{x}{P_1 \subst{h}{y}}} \inact }\\
%		&\Par \binp{t_2}{x} \newsp{s}{\binp{s}{x} \appl{x}{k_2} \Par \bout{\dual{s}}{\abs{x}{P_2 \subst{\dual{h}}{y}}} \inact }}
	\end{array}
\]
	\noi 
After these transitions, 
we can analyse that 
the resulting processes are behaviourally equivalent
since they have the same visible transitions; the rest 
is internal deterministic transitions. 
\end{proof}

%\smallskip 

%\noi 
%Thus, we may use $\hwb$ for tractable reasoning %is the most tractable 
%in the higher-order setting;  
%%the calculus is limited  
%%the $\sessp$-calculus, 
%%into~\jpc{\sessp}
%in the first-order setting of $\sessp$
%we can still use~$\fwb$. 



%PERHAPS REVISIT EXAMPLE \ref{ex:motivation}??


%\smallskip  
%
%\noi Processes that do not use shared names are inherently deterministic. 
%The following \jpc{determinacy property will be} useful 
%\dk{for proving our expressiveness results (\secref{sec:positive})}.
%%for both positive and negative results. 
%We require an auxiliary definition. 
%A transition $\horel{\Gamma}{\Delta}{P}{\hby{\tau}}{\Delta'}{P'}$ is said
%		{\em deterministic} if it is derived using~$\ltsrule{App}$ or~$\ltsrule{Tau}$ 
%		(where $\subj{\ell_1}$ and $\subj{\ell_2}$ in the premise 
%		are dual endpoints), 
%		possibly followed by uses of  $\ltsrule{Alpha}$, $\ltsrule{Res}$, $\ltsrule{Rec}$, or $\ltsrule{Par${}_L$}/\ltsrule{Par${}_R$}$.
%
%
%%\smallskip 
%
%\begin{lemma}[$\tau$-Inertness]\rm
%	\label{lem:tau_inert}
%	\begin{enumerate}[1)]
%		\item %(deterministic transitions) 
%		Let $\horel{\Gamma}{\Delta}{P}{\hby{\tau}}{\Delta'}{P'}$ be a deterministic transition,
%		with balanced $\Delta$. Then 
%		$\Gamma; \Delta \proves P \cong \Delta'\proves P'$ 
%		with $\Delta \red^\ast \Delta'$ balanced.
%%		Transition $\horel{\Gamma}{\Delta}{P}{\hby{\tau}}{\Delta'}{P'}$ is called
%%		{\em deterministic} if it is derived by $\ltsrule{App}$ or 
%%		$\ltsrule{Tau}$ where $\subj{\ell_1}$ and $\subj{\ell_2}$ in the premise 
%%		are dual session names. Suppose $\Delta$ is balanced. Then 
%%		$\Gamma; \Delta \proves P \cong \Delta'\proves P'$ 
%%		with $\Delta \red^\ast \Delta'$ balanced. 
%		\item 
%		%Let $P$ is the $\HOp^{-\mathsf{sh}}$-calculus. 
%		Let $P$ be an $\HOp^{-\mathsf{sh}}$ process. 
%		Assume $\Gamma; \emptyset; \Delta \proves P \hastype \Proc$. Then 
%		$P \red^\ast P'$ implies $\Gamma; \Delta \proves 
%		P \cong \Delta'\proves P'$ with $\Delta \red^\ast \Delta'$. 
%	\end{enumerate}
%\end{lemma}


%\smallskip 


%\begin{IEEEproof}
%	The full details of the proof are in Appendix~\ref{app:sub_coinc}.
%	The theorem is split into a hierarchy of Lemmas. Specifically
%	Lemma~\ref{lem:wb_eq_wbf} proves 
%	$\wb$ coincides with $\fwb$; 
%	Lemma~\ref{lem:wb_is_wbc} exploits the process substitution result
%	(Lemma~\ref{lem:proc_subst}) to prove that $\hwb \subseteq \wbc$.
%	Lemma~\ref{lem:wbc_is_cong} shows that $\wbc$ is a congruence
%	which implies $\wbc \subseteq \cong$.
%	The final result comes from Lemma~\ref{lem:cong_is_wb} where
%	we use label testing to show that $\cong \subseteq \fwb$ using
%	the technique in developed in~\cite{Hennessy07}. The formulation of input
%	triggers in the bisimulation relation allows us to prove
%	the latter result without using a matching operator.
%\end{IEEEproof}

%\smallskip 

%\noi Processes that do not use shared names, are inherently $\tau$-inert.

%\smallskip 

%\begin{lemma}[$\tau$-inertness]\rm
%	\label{lem:tau_inert}
%	Let $P$ is the $\HOp^{-\mathsf{sh}}$-calculus. 
%Assume $\Gamma; \emptyset; \Delta \proves P \hastype \Proc$. Then 
%$P \red^\ast P'$ implies $\Gamma; \Delta \proves 
%P \cong \Delta'\proves P'$ with $\Delta \red^\ast \Delta'$. 
%\end{lemma}


%\begin{IEEEproof}
%	The proof is relied on the fact that processes of the
%	form $\Gamma; \es; \Delta \proves_s \bout{s}{V} P_1 \Par \binp{k}{x} P_2$
%	cannot have any typed transition observables and the fact
%	that bisimulation is a congruence.
%	See details in Appendix~\ref{app:sub_tau_inert}.
%	\qed
%\end{IEEEproof}

