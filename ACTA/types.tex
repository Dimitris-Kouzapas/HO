% !TEX root = main.tex
\noi %Here we define 
We define a session typing system for \HOp and state its main properties. 
As we explain below, 
our system distils the key features of~\cite{tlca07,MostrousY15}.
%We give selected definitions;  see \appref{app:types} for a full description. 

%thus it is simpler.
%Our system is simpler than that in~\cite{tlca07,MostrousY15}, thus distilling the key
%features of higher-order sessions. %communications. %in a session-typed setting.

%\smallskip 

\subsection{Types}
%\label{subsec:types}
%\myparagraph{Types.}
The syntax of types of \HOp is given below:
\[
	\begin{array}{cc}
		\begin{array}{lcl}
			\textrm{(value)} & U \bnfis & C \bnfbar L
			\\[1mm]
			\textrm{(name)} & C \bnfis & S \bnfbar \chtype{S} \bnfbar \chtype{L}
			\\[1mm]
			\textrm{(abstractions)}~~ & L \bnfis & \shot{U} \bnfbar \lhot{U}
			\\[1mm]		
			\textrm{(session)} & S \bnfis &  \btout{U} S \bnfbar \btinp{U} S 
			\bnfbar \btsel{l_i:S_i}_{i \in I}  \bnfbar \btbra{l_i:S_i}_{i \in I}
			\\[1mm]					
			& \bnfbar & \trec{t}{S} \bnfbar \vart{t}  \bnfbar \tinact
			
		\end{array}
	\end{array}
	\]
Value types $U$ include
the first-order types $C$ and the higher-order
types $L$. Session types are denoted with $S$ and
shared types with $\chtype{S}$ and $\chtype{L}$.
\newc{We write $\Proc$ to denote the \emph{process type}.}
%Note that we dissallow type $\chtype{U}$, thus
%in the type discipline shared names cannot carry shared names.
%In name types, $\chtype{U}$ is shared name types 
%which are sent via shared names. 
\newc{The functional} types $\shot{U}$ and $\lhot{U}$ denote
{\em shared} and {\em linear} higher-order 
%\jpc{functional}
types, respectively.
%,
%used to type abstraction values.
%$\lhot{C}$ \cite{tlca07,mostrous_phd} ensures values which contain free 
%session names used once. 
Session types have the meaning already motivated in \secref{sec:overview}.
%We write $S$ to denote %binary 
%session types. 
The {\em output type}
$\btout{U} S$ %is assigned to a name that 
first sends a value of
type $U$ and then follows the type described by $S$.  Dually,
$\btinp{U} S$ denotes an {\em input type}. 
The {\em selection type}
$\btsel{l_i:S_i}_{i \in I}$ 
and the 
{\em branching type}
$\btbra{l_i:S_i}_{i \in I}$ 
define labelled choice, \newc{implemented at the level of processes by internal and external choice mechanisms, respectively.}
Type $\tinact$ is the termination type. 
We assume the {\em recursive type} $\trec{t}{S}$ is guarded,
i.e., type variables only appear under prefixes. 
This way, 
e.g.,  the type $\trec{t}{\vart{t}}$ is not allowed. 
%We stress that carried type $U$ in $\btout{U} S$ and
%$\btinp{U} S$ can contain free type variables, which is crucial
%to encode $\HOp$ into $\HO$.


%Types of \HO exclude $\nonhosyntax{C}$ from 
%value types of \HOp; the types of \sessp exclude $L$. 
%From each $\CAL \in \{\HOp, \HO, \pi \}$, $\CAL^{-\mathsf{sh}}$ 
%excludes shared name types ($\chtype{S}$ and $\chtype{L}$), 
%from name type $C$.

%Following \cite{TGC14},

Our type system is strictly included in that considered in~\cite{tlca07,MostrousY15}, which admits asynchronous communication and arbitrary nesting in functional types, i.e., their types are of the form 
$U\!\! \lollipop\!T$ 
(resp. $U\!\! \sharedop\!T$),
where $T$ ranges over $U$ and the process type $\Proc$. 
In contrast, our functional types are of the form $\lhot{U}$ (resp. $\shot{U}$). 

\newc{We rely on notions of \emph{duality} and \emph{equivalence} for types.}
Let us write $S_1 \sim S_2$
to denote that $S_1$ and $S_2$ are \emph{type-equivalent} (see 
\defref{def:iso} in the Appendix).
\newc{This notion extends to value types as expected; in the following, we write $U_1 \sim U_2$ to denote that $U_1$ and $U_2$ are type-equivalent.}
We write $S_1 \dualof S_2$ if 
$S_1$ is the \emph{dual} of $S_2$.   
Intuitively, 
duality
converts $!$ into $?$ and $\oplus$ into $\&$ (and vice-versa).
More formally, following~\cite{TGC14},
%for a co-inductive definition of duality.
%More formally, we 
we have a co-inductive definition for 
%\emph{type equivalence} and 
type {duality}:

%\begin{definition}[Type Equivalence]
%\label{def:iso}
%Let $\mathsf{ST}$ a set of closed session types. 
%Two types $S$ and $S'$ are said to be {\em isomorphic} if a pair $(S,S')$ is 
%in the largest fixed point of the monotone function
%$F:\mathcal{P}(\mathsf{ST}\times \mathsf{ST}) \to 
%\mathcal{P}(\mathsf{ST}\times \mathsf{ST})$ defined by: \\
%%\hspace{-0.5cm}
%\begin{tabular}{rcl}
%$F(\Re)$ &$\!\!=\!\!$&	$\set{(\tinact, \tinact)}$\\
%         &$\!\!\cup\!\!$&	$\set{(\btout{U_1} S_1, \btout{U_2} S_2)
%\bnfbar (S_1, S_2),(U_1, U_2)\in \Re}$\\ 
%       &$\!\!\cup\!\!$&	$\set{(\btinp{U_1} S_1, \btinp{U_2} S_2)
%\bnfbar(S_1, S_2),(U_1, U_2)\in \Re}$\\ 
%	&$\!\!\cup\!\!$&	$\set{(\btbra{l_i: S_i}_{i \in I} \,,\, \btbra{l_i: S_i'}_{i \in I}) \bnfbar \forall i\in I. (S_i, S_i')\in \Re}$\\
%	&$\!\!\cup\!\!$&	$\set{(\btsel{l_i: S_i}_{i \in I}\,,\, \btsel{l_i: S_i'}_{i \in I}) \bnfbar \forall i\in I. (S_i, S_i')\in \Re}$\\
%	&$\!\!\cup\!\!$&	$\set{(\trec{t}{S}, S')
%\bnfbar (S\subst{\trec{t}{S}}{\vart{t}},S')\in \Re}$\\
%	&$\!\!\cup\!\!$&	$\set{(S,\trec{t}{S'})
%\bnfbar (S,S'\subst{\trec{t}{S'}}{\vart{t}})\in \Re}$
%\end{tabular}
%	
%\noindent
%Standard arguments ensure that $F$ is monotone, thus the greatest fixed point
%of $F$ exists. We write $S_1 \sim S_2$ if  $(S_1,S_2)\in \Re$. 
%\end{definition}


\begin{definition}[Duality]
\label{def:dual}
Let $\mathsf{ST}$ be a set of closed session types. 
Two types $S$ and $S'$ are said to be {\em dual} if the pair $(S,S')$ is 
in the largest fixed point of the monotone function
$F:\mathcal{P}(\mathsf{ST}\times \mathsf{ST}) \to 
\mathcal{P}(\mathsf{ST}\times \mathsf{ST})$ defined by:\\ %[1mm]
\begin{tabular}{rcl}
$F(\Re)$ &$\!\!=\!\!$&	$\set{(\tinact, \tinact)}$\\
         &$\!\!\cup\!\!$&	$\set{(\btout{U_1} S_1, \btinp{U_2} S_2)
\bnfbar(S_1, S_2)\in \Re, \  U_1 \sim U_2 }$\\ 
       &$\!\!\cup\!\!$&	$\set{(\btinp{U_1} S_1, \btout{U_2} S_2)
\bnfbar(S_1, S_2)\in \Re, \ U_1 \sim U_2}$\\ 
	&$\!\!\cup\!\!$&	$\set{(\btsel{l_i: S_i}_{i \in I} \,,\, \btbra{l_i: S_i'}_{i \in I}) \bnfbar \forall i\in I. (S_i, S_i')\in \Re}$\\
	&$\!\!\cup\!\!$&	$\set{(\btbra{l_i: S_i}_{i \in I}\,,\, \btsel{l_i: S_i'}_{i \in I}) \bnfbar \forall i\in I. (S_i, S_i')\in \Re}$\\
	&$\!\!\cup\!\!$&	$\set{(\trec{t}{S}, S')
\bnfbar (S\subst{\trec{t}{S}}{\vart{t}},S')\in \Re}$\\
	&$\!\!\cup\!\!$&	$\set{(S,\trec{t}{S'})
\bnfbar (S,S'\subst{\trec{t}{S'}}{\vart{t}})\in \Re}$\\[1mm]
\end{tabular}

\noindent
%where $U_1 \sim U_2$ means $U_1$ is type equivalent to $U_2$ \cite{yoshida.vasconcelos:language-primitives}.
Standard arguments ensure that $F$ is monotone, thus the greatest fixed point
of $F$ exists. We write $S_1 \dualof S_2$ if  $(S_1,S_2)\in \Re$. 
\end{definition}

%\smallskip 

%\label{subsec:typing}
\subsection{Typing Environments and Judgements}
%\myparagraph{Typing Environments and Judgements.}
\noi Typing \emph{environments} are defined below:
%\[
\begin{eqnarray*}
	\Gamma  & \bnfis  &\emptyset \bnfbar \Gamma \cat \varp{x}: \shot{U} \bnfbar \Gamma \cat u: \chtype{S} \bnfbar \Gamma \cat u: \chtype{L} 
        \bnfbar \Gamma \cat \rvar{X}: \Delta \\
	\Lambda &\bnfis & \emptyset \bnfbar \Lambda \cat \AT{x}{\lhot{U}} \\
	%\qquad\qquad
	\Delta   &\bnfis &  \emptyset \bnfbar \Delta \cat \AT{u}{S}
\end{eqnarray*}
%]
\newc{Typing environments $\Gamma$, $\Lambda$, and $\Delta$ satisfy different structural principles.
Intuitively, the \emph{exchange} principle indicates that the ordering of type assignments does not matter. 
\emph{Weakening} says that type assignments need not be used. 
Finally, \emph{contraction}  says that type assignments may be duplicated.}

Environment $\Gamma$ maps variables and shared names to value types, and recursive 
variables to session environments;  
it admits weakening, contraction, and exchange principles.
Given $\Gamma$, we write $\Gamma \backslash x$ to denote the environment obtained from 
 $\Gamma$ by removing the assignment $x:\shot{U}$, for some $U$.
While  $\Lambda$ maps variables to 
%the
 linear %functional 
higher-order
types, $\Delta$ maps
session names to session types. 
Both $\Lambda$ and $\Delta$ %behave linearly: they 
are
only subject to exchange.  
%We require that t
The domains of $\Gamma,
\Lambda$ and $\Delta$ are assumed pairwise distinct. 
$\Delta_1\cdot \Delta_2$ means 
the disjoint union of $\Delta_1$ and $\Delta_2$.  
We define \emph{typing judgements} for values $V$
and processes $P$:
%\begin{center}
%\begin{tabular}{c}
	$$\Gamma; \Lambda; \Delta \proves V \hastype U \qquad \qquad \qquad \qquad \qquad \Gamma; \Lambda; \Delta \proves P \hastype \Proc$$
%\end{tabular}
%\end{center}
While the judgement on the left
says that under environments $\Gamma$, $\Lambda$, and $\Delta$ value $V$
has type $U$; the  judgement on the right says that under
environments $\Gamma$, $\Lambda$, and $\Delta$ process $P$ has the process type~$\Proc$.
The type soundness result for \HOp (Thm.~\ref{t:sr})
relies on two auxiliary notions on session environments: 
%that contain dual endpoints typed with dual types.
%The following definition ensures two session endpoints 
%are dual each other. 

%\smallskip

\begin{definition}[Session Environments: Balanced/Reduction]\label{d:wtenvred}%\rm
	Let $\Delta$ be a session environment.
	\begin{enumerate}[$\bullet$]
	\item  $\Delta$ is {\em balanced} if whenever
	$s: S_1, \dual{s}: S_2 \in \Delta$ then $S_1 \dualof S_2$.
	\item We define the reduction relation $\red$ on session environments as: %\\ %[-2mm]
\begin{eqnarray*}
	\Delta \cat s: \btout{U} S_1 \cat \dual{s}: \btinp{U} S_2  & \red & 
	\Delta \cat s: S_1 \cat \dual{s}: S_2  \\
	\Delta \cat s: \btsel{l_i: S_i}_{i \in I} \cat \dual{s}: \btbra{l_i: S_i'}_{i \in I} &\red& \Delta \cat s: S_k \cat \dual{s}: S_k' \ (k \in I)
\end{eqnarray*}
\end{enumerate}
\end{definition}

\begin{figure}[t!]
	\begin{mathpar}
		\inferrule[\trule{Sess}]{
		}{
			\Gamma; \emptyset; \{u:S\} \proves u \hastype S
		}
		\and
		\inferrule[\trule{Sh}]{
		}{\Gamma \cat u : U; \emptyset; \emptyset \proves u \hastype U}
		\and
		\inferrule[\trule{LVar}]{ }
		{\Gamma; \set{x: \lhot{U}}; \emptyset \proves x \hastype \lhot{U}}
		\and
		\inferrule[\trule{Prom}]{
			\Gamma; \emptyset; \emptyset \proves V \hastype 
                         \lhot{U}
		}{
			\Gamma; \emptyset; \emptyset \proves V \hastype 
                         \shot{U}
		} 
		\and
		\inferrule[\trule{EProm}]{
		\Gamma; \Lambda \cat x : \lhot{U}; \Delta \proves P \hastype \Proc
		}{
			\Gamma \cat x:\shot{U}; \Lambda; \Delta \proves P \hastype \Proc
		}
		\and
		\inferrule[\trule{Abs}]{
			\Gamma; \Lambda; \Delta_1 \proves P \hastype \Proc
			\quad
			\Gamma; \es; \Delta_2 \proves x \hastype U
		}{
			\Gamma\backslash x; \Lambda; \Delta_1 \backslash \Delta_2 \proves \abs{{x}}{P} \hastype \lhot{{U}}
		}
		\and
		\inferrule[\trule{App}]{
			\begin{array}{c}
				U = \lhot{U'} \lor \shot{U'}
				\quad
				\Gamma; \Lambda; \Delta_1 \proves V \hastype U
				\quad
				\Gamma; \es; \Delta_2 \proves W \hastype U'
			\end{array}
		}{
			\Gamma; \Lambda; \Delta_1 \cat \Delta_2 \proves \appl{V}{W} \hastype \Proc
		} 
		\and
		\inferrule[\trule{Send}]{
			\Gamma; \Lambda_1; \Delta_1 \proves P \hastype \Proc
			\quad
			\Gamma; \Lambda_2; \Delta_2 \proves V \hastype U
			\quad
			u:S \in \Delta_1 \cat \Delta_2
		}{
			\Gamma; \Lambda_1 \cat \Lambda_2; ((\Delta_1 \cat \Delta_2) \setminus u:S) \cat u:\btout{U} S \proves \bout{u}{V} P \hastype \Proc
		}
		\and
				\inferrule[\trule{Rcv}]{
		\Gamma; \Lambda_1; \Delta_1 \cat u: S \proves P \hastype \Proc
			\quad
			\Gamma; \Lambda_2; \Delta_2 \proves {x} \hastype {U}
		}{
			\Gamma \backslash x; \Lambda_1\cat \Lambda_2; \Delta_1\backslash \Delta_2 \cat u: \btinp{U} S \vdash \binp{u}{{x}} P \hastype \Proc
		}
\and \inferrule[\trule{Req}]{
			\begin{array}{c}
				\Gamma; \es; \es \proves u \hastype \dk{\chtype{U}}%_1
				\qquad
				\Gamma; \Lambda; \Delta_1 \proves P \hastype \Proc
				\qquad
				\Gamma; \es; \Delta_2 \proves V \hastype \dk{U}
%				\qquad
%				U_1 = \chtype{U_2}
%				(U_1 = \chtype{S} 
%                                \land %\Leftrightarrow 
%                                U_2 = S)
%				\lor
%				 (U_1 = \chtype{L} 
%                               \land %\Leftrightarrow 
%                                %\Leftrightarrow 
%                                 U_2 = L)
			\end{array}
		}{
			\Gamma; \Lambda; \Delta_1 \cat \Delta_2 \proves \bout{u}{V} P \hastype \Proc
		}
\and
		\inferrule[\trule{Acc}]{
			\begin{array}{c}
			\Gamma; \emptyset; \emptyset \proves u \hastype \dk{\chtype{U}}%_1 
		\qquad
		\Gamma; \Lambda_1; \Delta_1 \proves P \hastype \Proc
		\qquad
		\Gamma; \Lambda_2; \Delta_2 \proves x \hastype \dk{U}%_2
%		\qquad
%		U_1 = \chtype{U_2}
%				(U_1 = \chtype{S} 
%                                \land %\Leftrightarrow 
%                                U_2 = S)
%				\lor
%				 (U_1 = \chtype{L} 
%                                \land %\Leftrightarrow 
%                                %\Leftrightarrow 
%                                 U_2 = L)
               \end{array}
		}{
			\Gamma\backslash x; \Lambda_1 \backslash \Lambda_2; \Delta_1 \backslash \Delta_2 \proves \binp{u}{x} P \hastype \Proc
		}
		\and
\inferrule[\trule{Bra}]{
			 \forall i \in I \quad \Gamma; \Lambda; \Delta \cat u:S_i \proves P_i \hastype \Proc
		}{
			\Gamma; \Lambda; \Delta \cat u: \btbra{l_i:S_i}_{i \in I} \proves \bbra{u}{l_i:P_i}_{i \in I}\hastype \Proc
		}
\and
	 	\inferrule[\trule{Sel}]{
		%a
			\Gamma; \Lambda; \Delta \cat u: S_j  \proves P \hastype \Proc \quad j \in I
		}{
			\Gamma; \Lambda; \Delta \cat u:\btsel{l_i:S_i}_{i \in I} \proves \bsel{u}{l_j} P \hastype \Proc
		}
		\and
		\inferrule[\trule{ResS}]{
			\Gamma; \Lambda; \Delta \cat s:S_1 \cat \dual{s}: S_2 \proves P \hastype \Proc \qquad S_1 \dualof S_2
		}{
			\Gamma; \Lambda; \Delta \proves \news{s} P \hastype \Proc
		}
		\and
		\inferrule[\trule{Res}]{
			\Gamma\cat a:\chtype{S} ; \Lambda; \Delta \proves P \hastype \Proc
		}{
			\Gamma; \Lambda; \Delta \proves \news{a} P \hastype \Proc}
		\and
		\inferrule[\trule{Par}]{
			\Gamma; \Lambda_{i}; \Delta_{i} \proves P_{i} \hastype \Proc \qquad i=1,2
		}{
			\Gamma; \Lambda_{1} \cat \Lambda_2; \Delta_{1} \cat \Delta_2 \proves P_1 \Par P_2 \hastype \Proc
		}
				\and
				\inferrule[\trule{End}]{
			\Gamma; \Lambda; \Delta  \proves P \hastype \Proc \quad u \not\in \dom{\Gamma, \Lambda,\Delta}
		}{
			\Gamma; \Lambda; \Delta \cat u: \tinact  \proves P \hastype \Proc
		}
\and 
	 	\inferrule[\trule{Rec}]{
			\Gamma \cat \rvar{X}: \Delta; \emptyset; \Delta  \proves P \hastype \Proc
		}{
			\Gamma ; \emptyset; \Delta  \proves \recp{X}{P} \hastype \Proc
		}
		\and
	\inferrule[\trule{RVar}]{ }{\Gamma \cat \rvar{X}: \Delta; \emptyset; \Delta  \proves \rvar{X} \hastype \Proc
} 
\and
		\inferrule[\trule{Nil}]{ }{\Gamma; \emptyset; \emptyset \proves \inact \hastype \Proc}
\end{mathpar}
\caption{Typing Rules for $\HOp$.\label{fig:typerulesmy}}
\end{figure}
		
%\begin{figure}[t!]
%\[
%	\begin{array}{lc}
%		\trule{Sess}& \hspace{0cm}\Gamma; \emptyset; \set{u:S} \proves u \hastype S 
%		\quad  
%		\trule{Sh}~~\Gamma \cat u : U; \emptyset; \emptyset \proves u \hastype U
%		%\\[4mm]
%		\quad
%		\trule{LVar}~~ \Gamma; \set{x: \lhot{U}}; \emptyset \proves x \hastype \lhot{U}
%		\\[4mm]
%
%		\trule{Prom}&\hspace{-0.5cm} \tree{
%			\Gamma; \emptyset; \emptyset \proves V \hastype 
%                         \lhot{U}
%		}{
%			\Gamma; \emptyset; \emptyset \proves V \hastype 
%                         \shot{U}
%		} 
%		\qquad
%		\trule{EProm}\tree{
%		\Gamma; \Lambda \cat x : \lhot{U}; \Delta \proves P \hastype \Proc
%		}{
%			\Gamma \cat x:\shot{U}; \Lambda; \Delta \proves P \hastype \Proc
%		}
%		\\[6mm]
%
%		\trule{Abs}&\hspace{-0.5cm} \tree{
%			\Gamma; \Lambda; \Delta_1 \proves P \hastype \Proc
%			\quad
%			\Gamma; \es; \Delta_2 \proves x \hastype U
%		}{
%			\Gamma\backslash x; \Lambda; \Delta_1 \backslash \Delta_2 \proves \abs{{x}}{P} \hastype \lhot{{U}}
%		}
%		\\[6mm]
%
%		\trule{App}&\hspace{-0.5cm} \tree{
%			\begin{array}{c}
%				U = \lhot{U'} \lor \shot{U'}
%				\quad
%				\Gamma; \Lambda; \Delta_1 \proves V \hastype U
%				\quad
%				\Gamma; \es; \Delta_2 \proves W \hastype U'
%			\end{array}
%		}{
%			\Gamma; \Lambda; \Delta_1 \cat \Delta_2 \proves \appl{V}{W} \hastype \Proc
%		} 
%		\\[6mm]
%
%
%		\trule{Send}& \hspace{-0.5cm}\tree{
%			\Gamma; \Lambda_1; \Delta_1 \proves P \hastype \Proc
%			\quad
%			\Gamma; \Lambda_2; \Delta_2 \proves V \hastype U
%			\quad
%			u:S \in \Delta_1 \cat \Delta_2
%		}{
%			\Gamma; \Lambda_1 \cat \Lambda_2; ((\Delta_1 \cat \Delta_2) \setminus u:S) \cat u:\btout{U} S \proves \bout{u}{V} P \hastype \Proc
%		}
%		\\[6mm]
%
%		\trule{Rcv}&\hspace{-0.5cm} \tree{
%		\Gamma; \Lambda_1; \Delta_1 \cat u: S \proves P \hastype \Proc
%			\quad
%			\Gamma; \Lambda_2; \Delta_2 \proves {x} \hastype {U}
%		}{
%			\Gamma \backslash x; \Lambda_1\cat \Lambda_2; \Delta_1\backslash \Delta_2 \cat u: \btinp{U} S \vdash \binp{u}{{x}} P \hastype \Proc
%		}
%\\[6mm]
%
%		\trule{Req}& \hspace{-0.5cm}\tree{
%			\begin{array}{c}
%				\Gamma; \es; \es \proves u \hastype U_1
%				\quad
%				\Gamma; \Lambda; \Delta_1 \proves P \hastype \Proc
%				\quad
%				\Gamma; \es; \Delta_2 \proves V \hastype U_2
%				\\
%				(U_1 = \chtype{S} 
%                                \land %\Leftrightarrow 
%                                U_2 = S)
%				\lor
%				 (U_1 = \chtype{L} 
%                                \land %\Leftrightarrow 
%                                %\Leftrightarrow 
%                                 U_2 = L)
%			\end{array}
%		}{
%			\Gamma; \Lambda; \Delta_1 \cat \Delta_2 \proves \bout{u}{V} P \hastype \Proc
%		}
%		\\[6mm]
%
%
%		\trule{Acc}&\hspace{-0.5cm} \tree{
%			\begin{array}{c}
%			\Gamma; \emptyset; \emptyset \proves u \hastype 
%U_1 
%		\quad
%		\Gamma; \Lambda_1; \Delta_1 \proves P \hastype \Proc
%		\quad
%		\Gamma; \Lambda_2; \Delta_2 \proves x \hastype U_2\\
%				(U_1 = \chtype{S} 
%                                \land %\Leftrightarrow 
%                                U_2 = S)
%				\lor
%				 (U_1 = \chtype{L} 
%                                \land %\Leftrightarrow 
%                                %\Leftrightarrow 
%                                 U_2 = L)
%               \end{array}
%		}{
%			\Gamma\backslash x; \Lambda_1 \backslash \Lambda_2; \Delta_1 \backslash \Delta_2 \proves \binp{u}{x} P \hastype \Proc
%		}
%		\\[6mm]
%
%
%
%		\trule{Bra}& \hspace{-0.5cm}\tree{
%			 \forall i \in I \quad \Gamma; \Lambda; \Delta \cat u:S_i \proves P_i \hastype \Proc
%		}{
%			\Gamma; \Lambda; \Delta \cat u: \btbra{l_i:S_i}_{i \in I} \proves \bbra{u}{l_i:P_i}_{i \in I}\hastype \Proc
%		}
%\\[6mm]
%	 	\trule{Sel}& \hspace{-0.5cm}\tree{
%			\Gamma; \Lambda; \Delta \cat u: S_j  \proves P \hastype \Proc \quad j \in I
%
%		}{
%			\Gamma; \Lambda; \Delta \cat u:\btsel{l_i:S_i}_{i \in I} \proves \bsel{u}{l_j} P \hastype \Proc
%		}
%		\\[6mm]
%
%
%
%		\trule{ResS}&\hspace{-0.5cm} \tree{
%			\Gamma; \Lambda; \Delta \cat s:S_1 \cat \dual{s}: S_2 \proves P \hastype \Proc \quad S_1 \dualof S_2
%		}{
%			\Gamma; \Lambda; \Delta \proves \news{s} P \hastype \Proc
%		}
%		\\[6mm]
%		\trule{Res}&\hspace{-0.5cm}\tree{
%			\Gamma\cat a:\chtype{S} ; \Lambda; \Delta \proves P \hastype \Proc
%		}{
%			\Gamma; \Lambda; \Delta \proves \news{a} P \hastype \Proc}
%			\qquad 
%
%		\trule{Par} \tree{
%			\Gamma; \Lambda_{i}; \Delta_{i} \proves P_{i} \hastype \Proc \quad i=1,2
%		}{
%			\Gamma; \Lambda_{1} \cat \Lambda_2; \Delta_{1} \cat \Delta_2 \proves P_1 \Par P_2 \hastype \Proc
%		}
%\\[6mm]
%		\trule{End}& \hspace{-0.5cm}\tree{
%			\Gamma; \Lambda; \Delta  \proves P \hastype T \quad u \not\in \dom{\Gamma, \Lambda,\Delta}
%		}{
%			\Gamma; \Lambda; \Delta \cat u: \tinact  \proves P \hastype \Proc
%		}
%\qquad 
%	 	\trule{Rec} \tree{
%			\Gamma \cat \rvar{X}: \Delta; \emptyset; \Delta  \proves P \hastype \Proc
%		}{
%			\Gamma ; \emptyset; \Delta  \proves \recp{X}{P} \hastype \Proc
%		}
%		\\[6mm]
%\trule{RVar}& \Gamma \cat \rvar{X}: \Delta; \emptyset; \Delta  \proves \rvar{X} \hastype \Proc
%\qquad 
%		\trule{Nil}~~\Gamma; \emptyset; \emptyset \proves \inact \hastype \Proc
%	\end{array}
%\]
%%\caption{Typing Rules for $\HOp$.\label{fig:typerulesmy}}
%\Hline
%\end{figure}

\noi We rely on a typing system that is similar to the one developed in~\cite{tlca07,MostrousY15}. 
The typing system is defined in \figref{fig:typerulesmy}.
%Types for session names/variables $u$ and
%directly derived from the linear part of the typing
%environment, i.e.~type maps $\Delta$ and $\Lambda$.
Rules \trule{\textsc{Sess}}, \trule{\textsc{Sh}}, \trule{\textsc{LVar}} are name and variable introduction rules. 
%The type $\shot{U}$ %for shared higher order values $V$
%is derived using 
Rule $\trule{\textsc{Prom}}$
\newc{allows a value with a linear type $\lhot{U}$
to be used as $\shot{U}$ if its linear environment is empty.
%where we require a value with linear type to be typed without a linear environment to be typed as a shared type.
Rule $\trule{\textsc{EProm}}$ allows to freely use a shared
type variable in a linear way.}

Abstraction values are typed with Rule $\trule{\textsc{Abs}}$.
The key type for an abstraction is the type for
the bound variable of the abstraction, i.e., for
a bound variable with type $C$ the corresponding abstraction has type $\lhot{C}$.
The dual of abstraction typing is application typing,
governed by Rule $\trule{\textsc{App}}$: we expect
the type $U$ of an application value $W$ 
to match the type $\lhot{U}$ or $\shot{U}$
of the application variable $x$.

%A process prefixed with a session send operator $\bout{k}{V} P$
%is typed using rule $\trule{Send}$.
In Rule $\trule{\textsc{Send}}$, 
the type $U$ of the sent value $V$ should appear as a prefix
on the session type $\btout{U} S$ of $u$.
Rule $\trule{\textsc{Rcv}}$ is its dual.  
%defined the typing for the 
%reception of values $\binp{u}{V} P$.
%the type $U$ of a receive value should 
%appear as a prefix on the session type $\btinp{U} S$ of $u$.
We use a similar approach with session prefixes
to type interaction between shared names as defined 
in Rules $\trule{\textsc{Req}}$ and $\trule{\textsc{Acc}}$,
where the type of the sent/received object 
($S$ and $L$, respectively) should
match the type of the sent/received subject
($\chtype{S}$ and $\chtype{L}$, respectively).
Rules 
$\trule{\textsc{Sel}}$ and $\trule{\textsc{Bra}}$ for selection and branching are standard: 
both
rules prefix the session type with the selection
type $\btsel{l_i: S_i}_{i \in I}$ and
$\btbra{l_i:S_i}_{i \in I}$, respectively.

A
shared name creation $a$ creates and restricts
$a$ in environment $\Gamma$ as defined in 
Rule \trule{\textsc{Res}}. 
Creation of a session name $s$
creates and restricts two endpoints with dual types 
%and restricts
%them by removing them from the session environment
%$\Delta$ as defined 
in Rule \trule{\textsc{ResS}}. 
Rule \trule{\textsc{Par}}, 
combines the environments
$\Lambda$ and $\Delta$ of
the parallel components of a parallel process.
%to create a type for the entire parallel process.
The disjointness of environments $\Lambda$ and $\Delta$
is implied. Rule \trule{\textsc{End}} adds 
a name with type $\tinact$ in $\Delta$.  
The recursion requires that the body process 
matches the type of the recursive
variable as in Rule \trule{\textsc{Rec}}.
The recursive variable is typed
directly from the shared environment $\Gamma$ as
in Rule~\trule{\textsc{RVar}}.
Rule \trule{\textsc{Nil}} says that 
the inactive process $\inact$ is typed with empty linear environments  $\Lambda$ and $\Delta$.


We state the type soundness result for \HOp processes.


%\smallskip

\begin{theorem}[Type Soundness]\label{t:sr}%\rm
			Suppose $\Gamma; \es; \Delta \proves P \hastype \Proc$
			with
			$\Delta$ balanced. 
			Then $P \red P'$ implies $\Gamma; \es; \Delta'  \proves P' \hastype \Proc$
			and $\Delta = \Delta'$ or $\Delta \red \Delta'$
			with $\Delta'$ balanced. 
\end{theorem}
\begin{proof} Following standard lines.
See \appref{app:types} for details. \qed % of the associated proofs.
\end{proof}

\begin{example}[The Hotel Booking Example, Revisited]\label{exam:type}
We give types to the client
processes of~\secref{exam:proc}.
Assume 
\begin{eqnarray*}
S & = & \btout{\Quote} \btbra{\accept: \tinact, \reject: \tinact} \\
U & = & \btout{\rtype} \btinp{\Quote} \btsel{\accept: \btout{\creditc} \tinact, \reject: \tinact }
\end{eqnarray*}
While the typing for $\abs{x}{P_{xy}}$ is $\es; \es; y: S \proves \abs{x}{P_{xy}} \hastype \lhot{U}$,
the typing for $\Client_1$ is
$~~
	\es; \es; s_1: \btout{\lhot{U}} \tinact \cat s_2: \btout{\lhot{U}} \tinact \proves \Client_1 \hastype \Proc
$.


The typings for $Q_1$ and $Q_2$ are
$	\es; \es; y: \btout{\Quote} \btinp{\Quote} \tinact \proves \abs{x}{Q_i} \hastype \lhot{U}
$ ($i=1,2$)
and the type for $\Client_2$ is
$~~
	\es; \es; s_1: \btout{\lhot{U}} \tinact \cat s_2: \btout{\lhot{U}} \tinact \proves \Client_2 \hastype \Proc
$.
\end{example}

%%\begin{definition}[Balanced]\label{d:wtenv}%\rm
%%	%Let $\Delta$ be a session environment.
%%	We say that %a session environment 
%%	$\Delta$ is {\em balanced} if whenever
%%	$s: S_1, \dual{s}: S_2 \in \Delta$ then $S_1 \dualof S_2$.
%%\end{definition}
%%
%%\smallskip
%%
%%\myparagraph{The Typing System of \HOp.}
%Since the typing system is similar to~\cite{tlca07,MostrousY15}, 
%we fully describe it  in \appref{app:types}.  The %rest of the 
%paper can
%be read without knowing the details of the typing system. 
%\jpc{Type soundness relies on the following auxiliary notion}.
%%We list the key properties.
%
%\smallskip
%
%\begin{definition}[Reduction of Session Environment]%\rm
%\label{def:ses_red}
%We define the relation $\red$ on session environments as:\\[-2mm]
%\begin{center}
%\begin{tabular}{l}
%	$\Delta \cat s: \btout{U} S_1 \cat \dual{s}: \btinp{U} S_2 \red
%	\Delta \cat s: S_1 \cat \dual{s}: S_2$\\[1mm]
%	$\Delta \cat s: \btsel{l_i: S_i}_{i \in I} \cat \dual{s}: \btbra{l_i: S_i'}_{i \in I} \red \Delta \cat s: S_k \cat \dual{s}: S_k' \ (k \in I)$
%\end{tabular}
%\end{center}
%%\begin{tabular}{rcl}
%%	\setlength{\tabcolsep}{0pt}
%%	$\Delta \cat s: \btout{U} S_1 \cat \dual{s}: \btinp{U} S_2$ & $\red$ & 
%%	$\Delta \cat s: S_1 \cat \dual{s}: S_2$\\[1mm]
%%	$\Delta \cat s: \btsel{l_i: S_i}_{i \in I} \cat \dual{s}: \btbra{l_i: S_i'}_{i \in I}$ & $\red$ & $\Delta \cat s: S_k \cat \dual{s}: S_k' \ (k \in I)$
%%\end{tabular}
%%\[
%%\begin{array}{rcl}
%%\Delta \cat s: \btout{U} S_1 \cat \dual{s}: \btinp{U} S_2 & \red & 
%%\Delta \cat s: S_1 \cat \dual{s}: S_2\\[1mm]
%%\Delta \cat s: \btsel{l_i: S_i}_{i \in I} \cat \dual{s}: \btbra{l_i: S_i'}_{i \in I} & \red & \Delta \cat s: S_k \cat \dual{s}: S_k' \ (k \in I)
%%\end{array}
%%\]
%\end{definition}
%
%\smallskip
%
%%The following result %Theorem 7.3 in M\&Y
%\noi We state the type soundness results of \HOp: %; it implies 
%%the type soundness of the sub-calculi \HO, \sessp, and $\CAL^{-\mathsf{sh}}$. 
%
%\smallskip
%
%\begin{theorem}[Type Soundness]\label{t:sr}%\rm
%%	\begin{enumerate}[1.]
%%		\item	(Subject Congruence) Suppose $\Gamma; \es; \Delta \proves P \hastype \Proc$.
%%			Then $P \scong P'$ implies $\Gamma; \es; \Delta \proves P' \hastype \Proc$.
%%
%%		\item
%%			(Subject Reduction)
%			Suppose $\Gamma; \es; \Delta \proves P \hastype \Proc$
%			with
%			$\Delta$ balanced. 
%			Then $P \red P'$ implies $\Gamma; \es; \Delta'  \proves P' \hastype \Proc$
%			and $\Delta = \Delta'$ or $\Delta \red \Delta'$
%			with $\Delta'$ balanced. 
%%	\end{enumerate}
%\end{theorem}
