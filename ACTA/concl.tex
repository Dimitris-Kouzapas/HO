% !TEX root = main.tex

Obtaining tractable characterizations of context equivalence 
is a long-standing issue for higher-order  languages. 
In this paper, we have addressed this challenge for a higher-order language 
which integrates functional constructs and  features from concurrent processes (name and process passing), and 
whose interactions are governed by \emph{session types}, a behavioral type discipline for structured communications.
The main result of our study is the development of \emph{characteristic bisimilarity}, 
a relation on session typed processes which fully characterizes contextual equivalence.

Compared to 
the well-known 
context bisimilarity, our notion of 
characteristic bisimilarity 
enables more tractable analyses without sacrificing distinguishing power. 
Our approach to simplified analysis rests upon two simple mechanisms. 
First, 
using \emph{trigger processes} 
we lighten the requirements for the closures involved in output clauses. 
In particular, we can lift the need for heavy universal quantifications. 
Second, using \emph{characteristic processes and values} we refine the requirements for input clauses.
Formally supported by a refined LTS, the use of characteristic processes and values effectively narrows down
input actions.
Session type information (which includes linearity requirements on reciprocal communications), naturally available in scenarios of interacting processes, is crucial to define these two new mechanisms. Indeed, session types inform and enable technical simplifications in our developments.
As already discussed, our coincidence result is insightful also in the light of previous works on labeled equivalences 
for higher-order processes, in particular with respect to characterizations 
by Sangiorgi~\cite{SangiorgiD:expmpa,San96H} and by Jeffrey and Rathke~\cite{JeffreyR05}. Our main result 
combines several technical innovations, including, e.g., 
up-to techniques for deterministic behaviors (cf. \lemref{lem:up_to_deterministic_transition})
and an intermediate behavioral equivalence, called \emph{higher-order bisimilarity}
(denoted $\hwb$, cf. \defref{d:hwb}), which 
uses simpler trigger processes and 
is applicable to processes without first-order passing. 

In addition to their intrinsic significance, our study 
has important consequences and applications in other aspects of the theory of higher-order   processes. 
In particular,  we have recently explored the \emph{relative expressivity}
of higher-order sessions.  
%tractable behavioral equivalences developed here 
%  have enabled the study of properties of encoding correctness, in variants 
%of \HOp defined according to 
% a wide spectrum of features intrinsic to higher-order concurrency: 
 Both characteristic and higher-order bisimilarities play an important role in establishing 
 tight correctness properties (e.g., operational correspondence and full abstraction) for encodability results connecting different variants of \HOp.
 Such variants cover
 features such as 
 pure process passing (with first- and higher-order abstractions), pure name passing, polyadicity, linear/shared communication. 
 See the  
technical report~\cite{KouzapasPY15} for details on these additional results.

\paragraph{Acknowledgments} 
We are grateful to the CONCUR'15 reviewers and attendees for their valuable feedback.
This work has been partially sponsored by the The Doctoral Prize Fellowship, EPSRC EP/K011715/1, EPSRC EP/K034413/1, and EPSRC EP/L00058X/1, EU project FP7-612985 UpScale, and EU COST Action IC1201 BETTY. P\'{e}rez is also affiliated to the NOVA Laboratory for Computer Science and Informatics (NOVA LINCS), Universidade Nova de Lisboa, Portugal.