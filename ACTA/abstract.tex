% !TEX root = main.tex
\begin{abstract}
%Obtaining usefulcharacterisations of 
For higher-order (process) languages, 
characterising 
contextual equivalence is a long-standing issue. 
In the setting of a higher-order $\pi$-calculus with \emph{session types}, 
we develop %efficient 
\emph{characteristic bisimilarity}, 
a typed bisimilarity  
which fully characterises contextual equivalence.
To our knowledge, ours is the first characterisation of its kind.
Using simple values inhabiting 
(session) types, 
%our type syntax %(which includes recursive types),
our approach distinguishes from untyped methods for 
characterising contextual equivalence in
%untyped 
higher-order processes:
we show that observing as inputs
only a precise finite set of higher-order values suffices 
to reason about higher-order session processes. 
We demonstrate how characteristic bisimilarity can be used to justify optimisations in session protocols with mobile code communication.

%In fact, our characterisation result
%demonstrates that observing as inputs
%only a specific finite set of higher-order values (which inhabit session types) suffices 
%to reason about \HOp processes. 
\end{abstract}