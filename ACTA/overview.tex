% !TEX root = main.tex


%\smallskip 
We explain how we exploit session types to 
define characteristic bisimilarity.
Key notions are \emph{triggered} and \emph{characteristic processes/values}.
We first informally introduce some basic notation and terminology; formal definitions will be given in \secref{sec:calculus}.


\paragraph{Preliminaries.}
\newc{The syntax of \HOp considered in this paper is the following:
	\[
		\begin{array}{rcll}
			\text{Values}~~ V,W  & \bnfis & u & \text{names (shared and linear)} \\
			& \bnfbar &  \abs{x}{P} & \text{abstractions}
			\\[2mm]
			\text{Processes}~~ P,Q & \bnfis & \bout{u}{V}{P}  \bnfbar   \binp{u}{x}{P}  & \text{output and input} \\
			& \bnfbar &	 \bsel{u}{l} P \bnfbar \bbra{u}{l_i:P_i}_{i \in I}~~  & \text{labeled choice} \\
			& \bnfbar & \rvar{X} \bnfbar \recp{X}{P} & \text{recursion} \\
			& \bnfbar & \appl{V}{W} & \text{value application} \\
			& \bnfbar & P\Par Q \bnfbar \news{n} P \bnfbar \inact  & \text{composition, restriction, inaction}
			\\[2mm]
		\end{array}
	\] 
}
We write $n$ to range over shared names $a,b,\ldots$ and session (linear) names $s, \dual{s}, \ldots$.
Also, $u, w$ denotes either a name or a name variable.
Session names are sometimes called \emph{endpoints}; this way, e.g., 
$\bout{\dual{s}}{V} Q$ denotes 
for an output of value $V$ along endpoint $\dual{s}$ (the \emph{dual} of~$s$).
The higher-order character of \HOp thus comes from the fact that $V$ can be an abstraction.
The semantics of \HOp can be given in terms of a labelled transition system (LTS),
denoted $P \by{\ell} P'$, where $\ell$ denotes a transition label.
This way, e.g., 
$P \by{\bactinp{n}{V}} P'$ denotes an input transition along $n$
and
$P \by{\news{\widetilde{m}} \bactout{n}{V}} P'$
denotes an output transition along $n$, 
sending value $V$, and extruding names $\widetilde{m}$. 
Weak transitions, written 
$P \By{\ell} Q'$, abstract from internal actions in the usual way.
Throughout the paper, we write $\Re, \Re',\ldots$ to denote binary relations on (typed) processes.

\newc{
\HOp processes are meant to be disciplined by \emph{session types}, denoted $S, S', \ldots$, 
which we informally describe next:
		\[
		\begin{array}{lcll}
		S & \bnfis &  \btout{U} S \!\! \bnfbar \!\! \btinp{U} S &  \text{output/input value of type $U$, continue as $S$}
			\\[1mm]
			&\bnfbar& \btsel{l_i:S_i}_{i \in I} \!\! \bnfbar \!\!\btbra{l_i:S_i}_{i \in I}~  & \text{internal/external labeled choice of an $S_i$}
			\\[1mm]
			& \bnfbar & \trec{t}{S} \bnfbar \vart{t} & \text{recursive protocol}
			\\[1mm]
			&\bnfbar&  \tinact  & \text{completed protocol}
		\end{array}
		\]
As we will see, type $U$ denotes first-order values (i.e., shared and session names)
but also shared and linear functional types, denoted $\shot{U}$ and $\lhot{U}$, respectively
(where $\Proc$ is the type for processes).}


%\smallskip
\paragraph{Issues of Context Bisimilarity.}
Context bisimilarity ($\wbc$, \defref{def:wbc}) is an overly demanding relation on higher-order processes. 
There are two issues, associated to demanding clauses for output and input actions. 
%The first demanding issue arises from the universal quantification on the output clause of context bisimilarity:
A \emph{first issue} %appears in 
is 
the universal quantification in the output clause of context bisimilarity.
Suppose $P \,\Re\, Q$, for some context bisimulation~$\Re$. We have:
\begin{enumerate}[$(\star)$]
	\item	Whenever 
		$P \by{\news{\widetilde{m_1}} \bactout{n}{V}} P'$
		there exist
		$Q'$, $W$
		such that 
		$Q \By{\news{\widetilde{m_2}} \bactout{n}{W}} Q'$
		and, \\ \emph{\textbf{for all} $R$}  with $\fv{R}=x$, 
		$\newsp{\widetilde{m_1}}{P' \Par R\subst{V}{x}} \,\Re\, \newsp{\widetilde{m_2}}{Q' \Par R\subst{W}{x}}$.
\end{enumerate}
Intuitively, 
in the above clause process $R$ stands for any possible \emph{context} to which the emitted value
($V$ and $W$) is supposed to go. 
As explained in~\cite{San96H}, considering 
all possible contexts $R$ is key to achieve an adequate distinguishing power.

The \emph{second issue} is due to inputs: it  
%appears in input clauses, and 
follows from 
the fact that we work with an \emph{early}
labelled transition system (LTS). Thus, %a higher-order 
an input prefix may observe
infinitely many different values.

To alleviate these issues,
%induced by 
%universal quantification,
in %our notion of
\emph{characteristic
bisimilarity} ($\fwb$)
we take two (related) steps: 
\begin{enumerate}[(a)]
	\item We replace $(\star)$ with a clause involving a context \emph{more tractable} than $R\subst{V}{x}$; and 
	\item We refine  inputs % transitions 
	to avoid observing infinitely many actions on the same input prefix.
\end{enumerate}


\begin{comment}
\myparagraph{Issues of Context Bisimilarity.}
%The characterisation of contextual congruence given by 
Context bisimilarity ($\wbc$, \defref{def:wbc}) is an overly demanding relation on higher-order processes. 
%In the following we motivate our
%proposal for alternative, more tractable characterisations.  
%For the sake of clarity, and to emphasise the novelties of our approach, 
%we often omit type information. 
%Formal definitions including types are in \S\,\ref{sec:behavioural}.
This issue may be better appreciated by examining the output clause of context bisimilarity.
% see the issue, we show its output clause.
Suppose $P \,\Re\, Q$, for some context bisimulation $\Re$. Then:
\begin{enumerate}[$(\star)$]
	\item	Whenever 
		$P \by{\news{\widetilde{m_1}} \bactout{n}{V}} P'$
		there exist
		$Q'$ and $W$
		such that 
		$Q \by{\news{\widetilde{m_2}} \bactout{n}{W}} Q'$
		and, \emph{\textbf{for all} $R$}  with $\fv{R}=x$, 
		$\newsp{\widetilde{m_1}}{P' \Par R\subst{V}{x}} \,\Re\, \newsp{\widetilde{m_2}}{Q' \Par R\subst{W}{x}}$.
\end{enumerate}
\smallskip 
\noi 
Above, 
$\news{\widetilde{m_1}} \bactout{n}{V}$ is the output label of 
value $V$ with extrusion of names in $\widetilde{m_1}$.
To reduce the burden induced by 
universal quantification, we introduce 
%\emph{higher-order}  and 
\emph{characteristic}  
bisimilarity ($\fwb$). %, a tractable equivalence.
As we work with an \emph{early} labelled transition system (LTS), 
%we shall also aim at limiting the input actions,  
%so to define a
%bisimulation relation for the output clause without observing
%infinitely many actions on the same input prefix. 
%To this end, 
%
we take the following two steps: 
%
\begin{enumerate}[(a)]
	\item replace $(\star)$ with a clause involving a more tractable process closure; and 
	\item refine the input LTS rule 
	to avoid observing infinitely many actions on the same input prefix.
\end{enumerate}
%
%\smallskip
\end{comment}
%\noi
%Before 
%explaining how we address (a) and (b), we
%introduce some  notation. We  


\paragraph{Trigger Processes.} % with Session Communication.}
To address~(a), we exploit session types. 
We first 
observe that %closure 
$R\subst{V}{x}$ 
in $(\star)$
is context bisimilar to the process
%\begin{equation}\label{equ:1}
	$$P = \newsp{s}{\appl{(\abs{z}{\binp{z}{x}{R}})}{s} \Par \bout{\dual{s}}{V} \inact}$$
%\end{equation}
%\noi 
In fact,
we do have $P \wbc R\subst{V}{x}$:
\begin{eqnarray*}%\label{equ:1}
	P & \red &  \newsp{s}{\appl{\binp{s}{x}{R}} \Par \bout{\dual{s}}{V} \inact} \\
	 & \red &  R\subst{V}{x} \Par \inact
\end{eqnarray*}
where it is worth noticing that 
application and endpoint reduction
%($s$ and $\dual{s}$) 
are deterministic.  

Now let us
consider process $T_{V}$ below, where $t$ is a fresh name:
\begin{equation}\label{equ:2}
%\hotrigger{t}{V_1} 
T_{V} = \hotrigger{t}{x}{s}{V} 
\end{equation}
If $T_{V}$ inputs value $\abs{z}{\binp{z}{x} R}$ then
we have:
\begin{equation*}\label{equ:2a}
T_{V}
\by{\bactinp{t}{\abs{z}{\binp{z}{x} R}}} 
R\subst{V}{x}
\wbc 
P
\end{equation*}
Processes such as $T_{V}$ 
offer a value at a fresh name; this class of 
{\bf\em trigger processes} 
already suggests a tractable formulation of 
bisimilarity without the demanding 
output 
clause $(\star)$. 
%However, 
%Trigger p
Process $T_{V}$ in~\eqref{equ:2} requires a higher-order communication along $t$.
%As we explain now, we can do better than this; the key is using \emph{elementary inhabitants} of session types.
As we explain below, we can give an alternative trigger process; the key is using \emph{elementary inhabitants} of session types.
%Given a fresh name $t$, we write $\htrigger{t}{V}$ to  stand for a trigger process $T_{V}$ for value $V$.

%\smallskip

%Then we can use 
%$\newsp{\tilde{m_1}}{P_1 \Par \htrigger{t}{V_1}}$ instead 
%of Clause 1) in Definition \ref{def:wbc} if we input 
%$\abs{z}{\binp{z}{x} R}$.   

\paragraph{Characteristic Processes and Values.}
To address (b), we limit the possible 
input values (such as $\abs{z}{\binp{z}{x} R}$ above) %processes 
by exploiting session types.
The key concept is that of {\bf \emph{characteristic process/value}}
of a type,  
%The characteristic process of a session type $S$ is the process inhabiting $S$. 
the 
simplest term inhabiting that type (\defref{def:char}).
%Thus, characteristic processes follow the communication structures decreed by session types.

This way, for instance, let $S = \btinp{\shot{S_1}} \btout{S_2} \tinact$
be the session type that first
inputs an abstraction, %from values $S_1$ to processes, 
and then outputs a value of type $S_2$.
Then, process $\binp{u}{x} (\bout{u}{s_2} \inact \Par \appl{x}{s_1})$
is a \emph{characteristic process} for $S$ 
\jpc{along $u$.}


Given a session type $S$, we write $\mapchar{S}{u} $
for its characteristic process along   $u$
(cf.~\defref{def:char}).
Also, %Similarly, 
given value type $U$ (i.e., a type for channels or abstractions), then
$\omapchar{U}$ denotes its \emph{characteristic value}.
As we explain now, we use 
%the %characteristic %Precisely, we exploit  the
% characteristic value %$\lambda x.\mapchar{U}{x}$. %$\lambda x.\mapchar{U}{x}$. 
$\omapchar{U}$
 to limit input transitions.

\paragraph{Refined Input Transitions.}
To refine  input transitions, we need to observe 
an additional value, 
$\abs{{x}}{\binp{t}{y} (\appl{y}{{x}})}$, 
called the {\bf\em trigger value}. 
This is necessary: it turns out
that a characteristic value 
alone as the observable input 
is not enough to define a sound bisimulation (cf. \exref{ex:motivation}, page \pageref{ex:motivation}).
Intuitively, the trigger value is used
to observe/simulate application processes.
%to {\em count} the number of free higher-order variables inside 
%the receiver. 
%\jpc{See Example~\ref{ex:motivation} for further details.}
Based on the above discussion, we refine 
the transition rule for input actions (cf. \defref{def:rlts}). 
%We write $P \by{\bactinp{n}{V}} P'$ for the input transition along $n$.
Roughly, the 
refined rule for  \jpc{(higher-order)} input only  admits names, trigger values, and characteristic values:
%\[
%\tree {
%	P \by{\bactinp{n}{V}} P' \quad 
%	V = m \vee V \scong \abs{{x}}{\binp{t}{y} (\appl{y}{{x}})}
%	\vee  V \scong \omapchar{U}  \textrm{ with } t \textrm{ fresh} 
%}{
%	P' \hby{\bactinp{n}{V}} P'
%}
%\]
$$
%\boxed
{
~~P \by{\bactinp{n}{V}} P' ~\land~ (V = m \vee V \scong \abs{{x}}{\binp{t}{y} (\appl{y}{{x}})} \vee  V \scong \omapchar{U})
  %\textrm{ with } t \textrm{ fresh}) 
  ~~\Rightarrow~~
P \hby{\bactinp{n}{V}} P'~~
}
$$
for some fresh $t$.
Note the distinction between standard and refined transitions: $\by{\bactinp{n}{V}}$ vs. $\hby{\bactinp{n}{V}}$.
%Our refined  rule for \jpc{(higher-order)} input admits only names, trigger values, and characteristic values.
Using this rule, we define an alternative, refined  LTS on typed processes: 
we use it to define
%both higher-order ($\hwb$) and 
characteristic  bisimulation 
%(Defs.~\ref{d:hbw} and~\ref{d:fwb}),
($\fwb$, \defref{d:fwb}),
in which the demanding clause~$(\star)$ is replaced with 
a more tractable output clause based on 
%trigger processes \jpc{(cf.~\eqref{equ:2})}  and 
characteristic 
trigger processes
\jpc{(cf.~\eqref{eq:4})}.
%respectively.
%We show that $\hwb$ is useful for \HOp and \HO, and that~$\fwb$ can be uniformly used in all subcalculi, including \sessp. 


\paragraph{Characteristic Triggers.}
Following the same reasoning as (\ref{equ:2}), 
we can use an alternative trigger process, called
{\bf\em characteristic trigger process} with type 
$U$ to replace clause
% (1) in Definition~\ref{def:wbc}:
($\star$): % in \defref{def:wbc}:
\begin{equation}
	\label{eq:4}
	 {
	~~\ftrigger{t}{V}{U} \defeq \fotrigger{t}{x}{s}{\btinp{U} \tinact}{V}~~
	}
\end{equation}

%Note that if $U=L$, $\ftrigger{t}{V}{U}$ subsumes 
%$\htrigger{t}{V}$. 
\noi
This is justified because in~\eqref{equ:2} $T_V \hby{\bactinp{t}{\omapchar{\btinp{U} \tinact}}} \wbc \newsp{s}{\mapchar{\btinp{U} \tinact}{s} \Par \bout{\dual{s}}{V} \inact }$.
\noi 
\jpc{Thus, unlike process~\eqref{equ:2}, the characteristic trigger process 
in~\eqref{eq:4}
does not involve a
higher-order communication on %fresh name 
$t$.}
In contrast to previous approaches~\cite{SangiorgiD:expmpa,JeffreyR05} 
our %{trigger processes} 
 characteristic trigger processes 
do {\em not} use recursion or 
replication. This is key to preserve linearity of session endpoints.  


\begin{comment}
%\smallskip 
\myparagraph{Refined Input Transitions.}
Based on 
the above discussion, we refine 
the (early) transition rule for input actions. 
%We write $P \by{\bactinp{n}{V}} P'$ for the input transition along $n$.
The refined transition rule for input roughly becomes 
(see \defref{def:rlts} for details):
%\[
%\tree {
%	P \by{\bactinp{n}{V}} P' \quad 
%	V = m \vee V \scong \abs{{x}}{\binp{t}{y} (\appl{y}{{x}})}
%	\vee  V \scong \omapchar{U}  \textrm{ with } t \textrm{ fresh} 
%}{
%	P' \hby{\bactinp{n}{V}} P'
%}
%\]
$$
\boxed{
~~P \by{\bactinp{n}{V}} P' ~\land~ (V = m \vee V \scong \abs{{x}}{\binp{t}{y} (\appl{y}{{x}})} \vee  V \scong \omapchar{U}  \textrm{ with } t \textrm{ fresh}) ~~\Rightarrow~~
P' \hby{\bactinp{n}{V}} P'~~}
$$
\noi
Thus, our refined LTS admits only names, trigger values, and characteristic values in inputs.
Note the distinction between standard and refined transitions: $\by{\bactinp{n}{V}}$ vs. $\hby{\bactinp{n}{V}}$.
Using this rule, we define an alternative  LTS
with refined 
\jpc{(higher-order)}
input. %; all other rules are kept unchanged.
This refined LTS is used for 
%both higher-order ($\hwb$) and 
characteristic  bisimulation 
%(Defs.~\ref{d:hbw} and~\ref{d:fwb}),
($\fwb$, Def.~\ref{d:fwb}),
in which the demanding clause~$(\star)$ is replaced with 
a more tractable output clause based on 
%trigger processes \jpc{(cf.~\eqref{equ:2})}  and 
characteristic 
trigger processes
\jpc{(cf.~\eqref{eq:4})}.
%respectively.
%We show that $\hwb$ is useful for \HOp and \HO, and that~$\fwb$ can be uniformly used in all subcalculi, including \sessp. 
\end{comment}

\NY{It is also noteworthy that \HOp lacks name matching, 
which is usually crucial to prove completeness of bisimilarity---see, e.g.,~\cite{JeffreyR05}.
Instead of matching, we use types:
a process trigger embeds a name into a characteristic
process so to observe its session behaviour.}

%\dk{We stress-out that our calculus lacks
%a matching construct
%which is usually a crucial element to prove completeness of bisimilarity.
%Nevertheless, we use types
%%and specifically the characteristic process
%to compensate;
%%for the absence of matching, i.e.~
%instead of name matching, a process trigger embeds a name into a characteristic
%process so to observe its session behavior.}
%Notice that while Definition \ref{d:hbw} is useful for 
%\HOp and its higher-order variants,
%Definition \ref{d:fwb} is useful for first-order sub-calculi of \HOp.




%%\myparagraph{Outline}
%\subsection{Outline}
%\noi \S\,\ref{sec:calculus} presents the calculi; 
%\S\,\ref{sec:types} presents types;
%the tractable bisimulations are in \S\,\ref{sec:behavioural};
%the notion of encoding is in \S\,\ref{s:expr};
%\S\,\ref{sec:positive} and \S\,\ref{sec:negative}
%present positive and negative encodability results, resp;
%\S\,\ref{sec:extension} discusses extensions; and 
%\S\,\ref{sec:relwork} concludes with related work;
%Appendix summarises the typing system. 
%The paper is self-contained. 
%{\bf\em Omitted definitions, additional related work and full proofs can be found 
%in a technical report, available from \cite{KouzapasPY15}.} 
