% !TEX root = main.tex
%\myparagraph{Key points}
%\begin{enumerate}[1.]
%%	\item	Session $\pi$ calculus with process passing. DONE
%%	\item	Identify session $\pi$ and process passing subcalculi and their polyadic variants. DONE
%%	\item	Bisimulation theory for higher-order session semantics. DONE
%%	\item	New triggered bisimulation, related to J\&R's. DONE
%%	\item   Elementary values key to characterizations of behavioural equivalence. DONE
%	\item	Types provide techniques to prove completeness without matching. \jp{TBD}
%	\item	We are interested in encodings with properties a la Gorla. 
%                We extended them to typed setting. \jp{TBD}
%%	\item	Encode name-passing to pure process abstraction calculus, with name abstractions. DONE
%%	\item	Type of the recursion encoding uses non tail recursive type $\trec{t}{\btinp{t} \tinact}$. DONE
%%	\item	Encode higher-order semantics to first order semantics. DONE
%%	\item	Negative result. Cannot encode shared names using only shared names.
%%	\item   Extensions with higher-order abstractions and polyadicity also explored. DONE
%\end{enumerate}

%\smallskip 
%
%\myparagraph{Important things to explain}
%Explain our \HO is very small without containg name passing 
%\[ 
%\abs{x}.P \quad \appl{x}{u}
%\]

%Explain we input only characteristic processes.  
%
%\[
%\lambda x.\mapchar{S}{x}
%\]

%\subsection{Higher-Order Session Calculi}
%\noindent
\paragraph{Context.}
In \emph{higher-order process calculi} 
communicated
values %exchanged in communications 
may contain  processes.
Higher-order concurrency has received significant attention 
from untyped and typed perspectives; see, e.g.,~\cite{SangiorgiD:expmpa,JeffreyR05,DBLP:journals/iandc/LanesePSS11,DBLP:journals/cl/KoutavasH12,MostrousY15}.
%=== One alternative
%The combination of features from the $\lambda$-calculus and the $\pi$-calculus
%enables \emph{higher-order process calculi} to exchange values that may contain  processes.
%=== Dimitris version: 
%The combination of features from the $\lambda$-calculus and the $\pi$-calculus,
%in \emph{higher-order process calculi} allows for exchanged values to contain  processes. 
%=== Previous version: 
%By combining features from the $\lambda$-calculus and the $\pi$-calculus, 
%in \emph{higher-order process calculi} exchanged values may contain  processes. 
In this work, we consider \HOp, a higher-order process calculus with \emph{session communication}:
it 
combines functional constructs (abstractions/applications, as in the call-by-value $\lambda$-calculus)
and 
concurrent primitives (synchronisation on shared names, 
communication on linear names, 
  %(value passing, labelled choice), 
recursion).
\newc{By amalgamating  functional and concurrent constructs, 
\HOp may specify %reciprocal exchanges (protocols) 
complex session protocols that 
include higher-order  processes (process passing)
and that
 can be 
 type-checked 
 using \emph{session types}~\cite{honda.vasconcelos.kubo:language-primitives}.
 Session types ensure that process specifications conform to prescribed protocols by 
enforcing usage policies for \emph{shared}  and \emph{linear resources}.
This distinction is important, for session-based concurrency can be seen as involving two distinct phases:
the first one is non-deterministic and uses shared names, as it represents the interaction of processes seeking compatible protocol partners;
the second phase proceeds deterministically along linear names, as it specifies the concurrent execution of the session protocols established in the first phase.} 


%These calculi allow us to specify   
%session protocols in which higher-order values 
%(mobile code) can be exchanged in a type-safe manner. 
%; 
%governed by session types, 
%such protocols cleanly distinguish between 
%linear and unrestricted behaviors in 
%%directed %point-to-point 
%communications.
%in particular via  comparisons with the first-order mobility of the $\pi$-calculus~\cite{MilnerR:calmp1}. 
Although models of higher-order concurrency with session 
communication % higher-order features 
have been already developed~\cite{tlca07,DBLP:journals/jfp/GayV10},
their \emph{behavioural equivalences} 
remain little understood. 
Clarifying the status of these equivalences is essential to, e.g., 
justify non-trivial optimisations in protocols involving both name and process passing.
\newc{
An important aspect the development of these typed equivalences is that typed semantics are usually {\em coarser} than untyped semantics. 
Indeed, since (session) types limit the contexts (environments) in which processes can interact, 
typed equivalences admit stronger properties than their untyped counterpart.
}
%for higher-order session calculi. 
%these two issues 
%have been thoroughly studied
%%are well-understood 
%for higher-order languages without sessions \cite{},
%but not for higher-order process calculi with sessions.
%This is unfortunate, given the wide applicability of session-based concurrency; indeed,
%session types are expressive enough to describe complex 
%communication structures found in practical protocols,  expressible, e.g., via recursive session types.
%Clarifying the status of typed equivalences and relative expressiveness for session languages

%but also for transferring key reasoning techniques between (higher-order) session calculi. 
%Our discovery is that \emph{linearity} of session types plays a vital role to 
%offer equalities/characterisations
%% and fully abstract encodability, 
%which to our knowledge have not been proposed before.   


%In this paper we study
%%address  behavioural equivalences for 
%\HOp, 
%%study behavioral equalities for \HOp, 
%an extension of the higher-order $\pi$-calculus~\cite{SangiorgiD:expmpa} with session primitives:
%\HOp contains constructs for 
%%session establishment
%synchronisation on shared names, 
%recursion, 
% (linear) session communication (value passing and
%labelled choice),
%abstractions and applications. 
%Abstractions are functions from values to processes, 
%\jpc{denoted}
%$\lambda x.P$; applications are 
%denoted $(\lambda x.P)V$, where the value $V$ is either a name or an abstraction.
%We study two significant subcalculi of \HOp, 
%\jpc{which}
%distil higher- and first-order mobility:
%the \HO-calculus, which is \HOp without recursion and name passing, and 
%the session \sessp-calculus \jpc{(here denoted~\sessp)}, which is \HOp without abstractions and applications.  
%While \sessp is, 
%in essence, the calculus in~\cite{honda.vasconcelos.kubo:language-primitives}, 
%this paper shows that \HO  is a new core calculus 
%for higher-order session concurrency.

A well-known behavioural equivalence for higher-order processes
is \emph{context bisimilarity}~\cite{San96H}. This 
 characterisation of %reduction-closed, 
barbed congruence 
offers an adequate distinguishing power at the price of heavy universal quantifications in output clauses.
Obtaining alternative 
characterisations of context bisimilarity
is thus a recurring, important problem 
for higher-order calculi---see, e.g.,~\cite{SangiorgiD:expmpa,San96H,JeffreyR05,DBLP:journals/cl/KoutavasH12,DBLP:journals/corr/Xu13a,lenglet_et_al:LIPIcs:2015:5364}. 
In particular, Sangiorgi~\cite{SangiorgiD:expmpa,San96H} has 
given %important 
%useful
characterisations of context bisimilarity
for higher-order processes; such 
characterisations, however,  %in~\cite{SangiorgiD:expmpa,San96H} 
do not scale to  
  calculi with \emph{recursive types}, which %in our experience 
  are essential to %the practice of 
  express practical protocols in 
session-based concurrency. A characterisation  
%context bisimilarity 
that solves this limitation was developed by Jeffrey and Rathke in~\cite{JeffreyR05};
their solution, however, does not consider \emph{linearity}---an important aspect in session-based concurrency, as explained above.

%\smallskip

\paragraph{This Work.}
Building upon~\cite{SangiorgiD:expmpa,San96H,JeffreyR05}, 
our discovery is that {linearity} as induced by session types plays a vital role 
%to 
%offer equalities and characterisations
% and fully abstract encodability, 
%which to our knowledge have not been proposed before. 
% 
in 
solving 
the %long-standing, 
open problem 
of characterising context bisimilarity for higher-order mobile processes with session communication.
Our approach is to exploit 
%protocol specifications given by session types to limit 
the coarser semantics induced by session types to limit
the behaviour of higher-order session processes. 
 Formally, we enforce this limitation by defining
a \emph{refined} labelled transition system (LTS)
which effectively 
narrows down the spectrum of allowed process behaviours, 
exploiting elementary processes inhabiting session types.
%thus enabling tractable reasoning techniques. 
We then introduce \emph{characteristic bisimilarity}: this  
 new notion of typed bisimilarity   is 
\emph{more tractable} than context bisimilarity, in that 
it relies on the refined LTS for input actions and, more importantly, 
does not appeal to universal quantifications on output actions. 
%shown to coincide with context bisimilarity.

Our main result is that characteristic  %tractable
bisimilarity coincides with context bisimilarity.
Besides confirming the value of characteristic bisimilarity as a useful reasoning technique for 
higher-order processes with sessions,
%for  specifications of trivial practical scenarios, 
this result is 
%also technically 
remarkable 
also from a technical perspective, for associated 
completeness proofs do not require 
operators for 
name matching,
% in the process language, 
in contrast to Jeffrey and Rathke's technique for higher-order processes
with recursive types~\cite{JeffreyR05}.
%Remarkably session type structures enable to provide 
%a coincidence without name-matching operators in the calculi.



%\smallskip

\paragraph{Outline.} 
%This paper  is structured as follows.
%\begin{enumerate}[$\bullet$]
%\item 
Next,
%%\secref{sec:overview} overviews 
we overview the
key ideas of characteristic bisimilarity, 
our 
characterisation of contextual equivalence.
%%\item 
Then, \secref{sec:calculus}  presents 
%we present
the %higher-order 
session calculus \HOp. 
%A small example is given in \S\,X.
\secref{sec:types} gives the session type system for \HOp
and states type soundness.
%\item 
\secref{sec:behavioural} 
develops %\emph{higher-order} and 
\emph{characteristic} bisimilarity and 
%which alleviates the issues of context bisimilarity~\cite{San96H} and 
states our main result: characteristic bisimilarity and contextual equivalence coincide for 
%is shown  to coincide for 
well-typed \HOp processes (\thmref{the:coincidence}).
In \secref{sec:relwork} we discuss related works, while
%The appendix summarises the typing system. 
%\end{enumerate}
%\noi
%The paper is self-contained. 
\secref{sec:concl}~collects some concluding remarks. 
%\textbf{The Appendix contains omitted definitions/proofs.}

This paper is an extended version of the conference paper~\cite{kouzapas_et_al:LIPIcs:2015:5365}.
This presentation includes full technical details---definitions and proofs, collected in the Appendix.
In particular, we introduce \emph{higher-order bisimilarity} (an auxiliary labeled 
bisimilarity) 
and highlight its role
in the proof of \thmref{the:coincidence}. 
We also elaborate further  on the 
use case scenario 
for characteristic bisimilarity 
given in~\cite{kouzapas_et_al:LIPIcs:2015:5365} (the Hotel Booking scenario).
We develop an additional example, given in  
\secref{sec:relwork}, and use it to compare our approach with Jeffrey and Rathke's~\cite{JeffreyR05}. 
Moreover, we offer extended discussions of related works.

