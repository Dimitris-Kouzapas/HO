% !TEX root = main.tex


\label{subsec:intro:bisimulation}
\noi 
We explain how our approach exploits session types to 
formulate characteristic bisimilarity.

\smallskip 

\myparagraph{Issues of Context Bisimilarity.}
%The characterisation of contextual congruence given by 
Context bisimilarity ($\wbc$, \defref{def:wbc}) is an overly demanding relation on higher-order processes. 
%In the following we motivate our
%proposal for alternative, more tractable characterisations.  
%For the sake of clarity, and to emphasise the novelties of our approach, 
%we often omit type information. 
%Formal definitions including types are in \S\,\ref{sec:behavioural}.
This issue may be better appreciated by examining the output clause of context bisimilarity.
% see the issue, we show its output clause.
Suppose $P \,\Re\, Q$, for some context bisimulation $\Re$. Then:
\begin{enumerate}[$(\star)$]
	\item	Whenever 
		$P \by{\news{\widetilde{m_1}} \bactout{n}{V}} P'$
		there exist
		$Q'$ and $W$
		such that 
		$Q \by{\news{\widetilde{m_2}} \bactout{n}{W}} Q'$
		and, \emph{\textbf{for all} $R$}  with $\fv{R}=x$, 
		$\newsp{\widetilde{m_1}}{P' \Par R\subst{V}{x}} \,\Re\, \newsp{\widetilde{m_2}}{Q' \Par R\subst{W}{x}}$.
\end{enumerate}
\smallskip 
\noi 
Above, 
$\news{\widetilde{m_1}} \bactout{n}{V}$ is the output label of 
value $V$ with extrusion of names in $\widetilde{m_1}$.
To reduce the burden induced by 
universal quantification, we introduce 
%\emph{higher-order}  and 
\emph{characteristic}  
bisimilarity ($\fwb$). %, a tractable equivalence.
As we work with an \emph{early} labelled transition system (LTS), 
%we shall also aim at limiting the input actions,  
%so to define a
%bisimulation relation for the output clause without observing
%infinitely many actions on the same input prefix. 
%To this end, 
%
we take the following two steps: 
%
\begin{enumerate}[(a)]
	\item We replace $(\star)$ with a clause involving a more tractable process closure.
	\item We refine the transition rule for input in the LTS,
	to avoid observing infinitely many actions on the same input prefix.
\end{enumerate}
%
\smallskip

\myparagraph{Trigger Processes.} % with Session Communication.}
We  write $\binp{s}{x}{P}$ 
for an input on the (linear) endpoint $s$, and 
and $\bout{\dual{s}}{V} Q$
for an output of value $V$ 
on endpoint $\dual{s}$ (the dual of~$s$).
Concerning~(a), we exploit session types. 
We 
first 
observe that closure $R\subst{V}{x}$ 
in $(\star)$
is context bisimilar to the process:
\begin{equation}\label{equ:1}
	P = \newsp{s}{\appl{(\abs{z}{\binp{z}{x}{R}})}{s} \Par \bout{\dual{s}}{V} \inact}
\end{equation}
%\noi 
In fact,
we do have $P \wbc R\subst{V}{x}$, 
since 
application and reduction of dual endpoints 
%($s$ and $\dual{s}$) 
are deterministic.  
Now let us
consider process $T_{V}$ below, where $t$ is a fresh name:
\begin{equation}\label{equ:0}
T_{V} = \hotrigger{t}{V}
\end{equation}
%We call $\abs{z}{\binp{z}{x} R}$ a {\bf\em trigger value}. 
We write $P \by{\bactinp{n}{V}} P'$ to denote an input transition along $n$.
If $T_{V}$ inputs value $\abs{z}{\binp{z}{x} R}$ then
we can simulate the closure of $P$:
\begin{equation}\label{equ:2}
%\hotrigger{t}{V_1} 
T_{V}
\by{\bactinp{t}{\abs{z}{\binp{z}{x} R}}} P 
\wbc 
R\subst{V}{x}
\end{equation}
Processes such as $T_{V}$ 
offer a value at a fresh name; this class of 
{\bf\em trigger processes} 
already suggests a tractable formulation of 
bisimilarity without the demanding 
output clause $(\star)$. 
Simple inhabitants of session types will be essential here, as we explain now.
%Given a fresh name $t$, we write $\htrigger{t}{V}$ to  stand for a trigger process $T_{V}$ for value $V$.

\smallskip

%Then we can use 
%$\newsp{\tilde{m_1}}{P_1 \Par \htrigger{t}{V_1}}$ instead 
%of Clause 1) in Definition \ref{def:wbc} if we input 
%$\abs{z}{\binp{z}{x} R}$.   

\myparagraph{Characteristic Processes and Values.}
Concerning (b), we limit the possible 
input values (such as $\abs{z}{\binp{z}{x} R}$ above) %processes 
by exploiting session types.
The key concept is that of {\bf \emph{characteristic process/value}}
of a type,  
%The characteristic process of a session type $S$ is the process inhabiting $S$. 
the 
simplest term inhabiting that type (\defref{def:char}).
This way, e.g., let $S = \btinp{\shot{S_1}} \btout{S_2} \tinact$
be a session type: first
input an abstraction, %from values $S_1$ to processes, 
then output a value of type $S_2$.
Then, process $Q = \binp{u}{x} (\bout{u}{s_2} \inact \Par \appl{x}{s_1})$
is a characteristic process for $S$ 
\jpc{along name $u$.}
%Thus, characteristic processes follow the communication structures decreed by session types.
Given a session type $S$, we write $\mapchar{S}{u} $
for its characteristic process along name $u$
(cf.~\defref{def:char}).
Also, %Similarly, 
given value type $U$, we write 
$\omapchar{U}$ to denote its characteristic value.
We use the %characteristic %Precisely, we exploit  the
 characteristic value %$\lambda x.\mapchar{U}{x}$. %$\lambda x.\mapchar{U}{x}$. 
$\omapchar{U}$
 to limit input transitions.
Following the same reasoning as (\ref{equ:1})--(\ref{equ:2}), 
we can use an alternative trigger process, called
{\bf\em characteristic trigger process} with type 
$U$ to replace clause
% (1) in Definition~\ref{def:wbc}:
($\star$): % in \defref{def:wbc}:
\begin{equation}
	\label{eq:4}
	\ftrigger{t}{V}{U} \defeq \fotrigger{t}{x}{s}{\btinp{U} \tinact}{V}
\end{equation}

%Note that if $U=L$, $\ftrigger{t}{V}{U}$ subsumes 
%$\htrigger{t}{V}$. 
\noi 
\jpc{Thus, in contrast to the trigger process~\eqref{equ:0}, the characteristic trigger process 
in~\eqref{eq:4}
does not involve a
higher-order communication on fresh name $t$.}
We note that 
in contrast to previous approaches~\cite{SaWabook,JeffreyR05} 
our %{trigger processes} 
 characteristic trigger processes 
do {\em not} use recursion or 
replication. This is crucial for preserving linearity of session endpoints.  


To refine the input transition system, we need to observe 
an additional value, 
$\abs{{x}}{\binp{t}{y} (\appl{y}{{x}})}$, 
called the {\bf\em trigger value}. 
This is necessary, because it turns out
that a characteristic value 
alone as the observable input 
is not enough to define a sound bisimulation.
Roughly speaking, the trigger value is used
to observe/simulate application processes.
%to {\em count} the number of free higher-order variables inside 
%the receiver. 
%\jpc{See Example~\ref{ex:motivation} for further details.}



\smallskip 
\myparagraph{Refined Input Transitions.}
Based on 
the above discussion, we refine 
the (early) transition rule for input actions. 
%We write $P \by{\bactinp{n}{V}} P'$ for the input transition along $n$.
The transition rule for input roughly becomes 
(see \defref{def:rlts} for details):
\[
\tree {
	P \by{\bactinp{n}{V}} P' \quad 
	V = m \vee V \scong \abs{{x}}{\binp{t}{y} (\appl{y}{{x}})}
	\vee  V \scong \omapchar{U}  \textrm{ with } t \textrm{ fresh} 
}{
	P' \hby{\bactinp{n}{V}} P'
}
\]
Thus, our refined LTS admits only names, trigger values, and characteristic values in inputs.
Note the distinction between standard and refined transitions: $\by{\bactinp{n}{V}}$ vs. $\hby{\bactinp{n}{V}}$.
Using this rule, we define an alternative  LTS
with refined 
\jpc{(higher-order)}
input. %; all other rules are kept unchanged.
This refined LTS is used for 
%both higher-order ($\hwb$) and 
characteristic  bisimulation 
%(Defs.~\ref{d:hbw} and~\ref{d:fwb}),
($\fwb$, Def.~\ref{d:fwb}),
in which the demanding clause~$(\star)$ is replaced with 
a more tractable output clause based on 
%trigger processes \jpc{(cf.~\eqref{equ:0})}  and 
characteristic 
trigger processes
\jpc{(cf.~\eqref{eq:4})}.
%respectively.
%We show that $\hwb$ is useful for \HOp and \HO, and that~$\fwb$ can be uniformly used in all subcalculi, including \sessp. 

\NY{It is also noteworthy that \HOp lacks name matching, 
which is usually crucial to prove completeness of bisimilarity.
Instead of matching, we use types:
a process trigger embeds a name into a characteristic
process so to observe its session behaviour.}

%\dk{We stress-out that our calculus lacks
%a matching construct
%which is usually a crucial element to prove completeness of bisimilarity.
%Nevertheless, we use types
%%and specifically the characteristic process
%to compensate;
%%for the absence of matching, i.e.~
%instead of name matching, a process trigger embeds a name into a characteristic
%process so to observe its session behavior.}
%Notice that while Definition \ref{d:hbw} is useful for 
%\HOp and its higher-order variants,
%Definition \ref{d:fwb} is useful for first-order sub-calculi of \HOp.




%%\myparagraph{Outline}
%\subsection{Outline}
%\noi \S\,\ref{sec:calculus} presents the calculi; 
%\S\,\ref{sec:types} presents types;
%the tractable bisimulations are in \S\,\ref{sec:behavioural};
%the notion of encoding is in \S\,\ref{s:expr};
%\S\,\ref{sec:positive} and \S\,\ref{sec:negative}
%present positive and negative encodability results, resp;
%\S\,\ref{sec:extension} discusses extensions; and 
%\S\,\ref{sec:relwork} concludes with related work;
%Appendix summarises the typing system. 
%The paper is self-contained. 
%{\bf\em Omitted definitions, additional related work and full proofs can be found 
%in a technical report, available from \cite{KouzapasPY15}.} 
