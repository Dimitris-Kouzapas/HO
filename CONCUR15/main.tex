%%%%%%%%%%%%%%%%%%%%%%%%%%%%%%%%%%%%%%%%%%%%%%%%%%%%%%%%%
%%% CONCUR VERSION
%%%%%%%%%%%%%%%%%%%%%%%%%%%%%%%%%%%%%%%%%%%%%%%%%%%%%%%%%

\documentclass[a4paper,UKenglish]{lipics}
%This is a template for producing LIPIcs articles. 
%See lipics-manual.pdf for further information.
%for A4 paper format use option "a4paper", for US-letter use option "letterpaper"
%for british hyphenation rules use option "UKenglish", for american hyphenation rules use option "USenglish"
% for section-numbered lemmas etc., use "numberwithinsect"
 
\usepackage{microtype,xspace,enumerate,comment,stmaryrd}%if unwanted, comment out or use option "draft"
\usepackage{times}
\usepackage{mathpartir}
%\graphicspath{{./graphics/}}%helpful if your graphic files are in another directory
\input{macros}
%\bibliographystyle{plain}% the recommended bibstyle

% Author macros::begin %%%%%%%%%%%%%%%%%%%%%%%%%%%%%%%%%%%%%%%%%%%%%%%%
\title{Characteristic Bisimulations for Higher-Order Session Processes}
\titlerunning{Characteristic Bisimulations for Higher-Order Session Processes} %optional, in case that the title is too long; the running title should fit into the top page column

\author{Dimitrios Kouzapas$^{\text{1,3}}$}
\author{Jorge A. P\'{e}rez$^{\text{2}}$}
\author{Nobuko Yoshida$^{\text{1}}$}
\affil{1~~~ Imperial College London \qquad 2~~~ University of Groningen \qquad 3~~~ University of Glasgow}
%\affil[2]{University of Groningen}
%\affil[3]{University of Glasgow}
\authorrunning{D. Kouzapas and J.\,A. P\'{e}rez and N. Yoshida} %mandatory. First: Use abbreviated first/middle names. Second (only in severe cases): Use first author plus 'et. al.'

\Copyright{D. Kouzapas and J.\,A. P\'{e}rez and N. Yoshida}%mandatory, please use full first names. LIPIcs license is "CC-BY";  http://creativecommons.org/licenses/by/3.0/

%\subjclass{Dummy classification -- please refer to \url{http://www.acm.org/about/class/ccs98-html}}% mandatory: Please choose ACM 1998 classifications from http://www.acm.org/about/class/ccs98-html . E.g., cite as "F.1.1 Models of Computation". 
%\keywords{Dummy keyword -- please provide 1--5 keywords}% mandatory: Please provide 1-5 keywords
%% Author macros::end %%%%%%%%%%%%%%%%%%%%%%%%%%%%%%%%%%%%%%%%%%%%%%%%%
%
%%Editor-only macros:: begin (do not touch as author)%%%%%%%%%%%%%%%%%%%%%%%%%%%%%%%%%%
%\serieslogo{}%please provide filename (without suffix)
%\volumeinfo%(easychair interface)
%  {Billy Editor and Bill Editors}% editors
%  {2}% number of editors: 1, 2, ....
%  {Conference title on which this volume is based on}% event
%  {1}% volume
%  {1}% issue
%  {1}% starting page number
%\EventShortName{}
%\DOI{10.4230/LIPIcs.xxx.yyy.p}% to be completed by the volume editor
%% Editor-only macros::end %%%%%%%%%%%%%%%%%%%%%%%%%%%%%%%%%%%%%%%%%%%%%%%

\begin{document}

\maketitle

% !TEX root = main.tex
\begin{abstract}
This work proposes %efficient 
tractable
bisimulations 
for the higher-order $\pi$-calculus with session primitives~(\HOp).
We develop three typed bisimulations, which are shown to 
coincide with contextual equivalence.
These characterisations  
demonstrate that observing as inputs
only a specific finite set of higher-order values (which inhabit session types) suffices 
to reason about \HOp processes. 
\end{abstract}

%\begin{abstract}
%Lorem ipsum dolor sit amet, consectetur adipiscing elit. Praesent convallis orci arcu, eu mollis dolor. Aliquam eleifend suscipit lacinia. Maecenas quam mi, porta ut lacinia sed, convallis ac dui. Lorem ipsum dolor sit amet, consectetur adipiscing elit. Suspendisse potenti. 
% \end{abstract}

\section{Introduction}
\label{sec:intro}
% !TEX root = main.tex
%\myparagraph{Key points}
%\begin{enumerate}[1.]
%%	\item	Session $\pi$ calculus with process passing. DONE
%%	\item	Identify session $\pi$ and process passing subcalculi and their polyadic variants. DONE
%%	\item	Bisimulation theory for higher-order session semantics. DONE
%%	\item	New triggered bisimulation, related to J\&R's. DONE
%%	\item   Elementary values key to characterizations of behavioural equivalence. DONE
%	\item	Types provide techniques to prove completeness without matching. \jp{TBD}
%	\item	We are interested in encodings with properties a la Gorla. 
%                We extended them to typed setting. \jp{TBD}
%%	\item	Encode name-passing to pure process abstraction calculus, with name abstractions. DONE
%%	\item	Type of the recursion encoding uses non tail recursive type $\trec{t}{\btinp{t} \tinact}$. DONE
%%	\item	Encode higher-order semantics to first order semantics. DONE
%%	\item	Negative result. Cannot encode shared names using only shared names.
%%	\item   Extensions with higher-order abstractions and polyadicity also explored. DONE
%\end{enumerate}

%\smallskip 
%
%\myparagraph{Important things to explain}
%Explain our \HO is very small without containg name passing 
%\[ 
%\abs{x}.P \quad \appl{x}{u}
%\]

%Explain we input only characteristic processes.  
%
%\[
%\lambda x.\mapchar{S}{x}
%\]

%\subsection{Higher-Order Session Calculi}
%\noindent
\paragraph{Context.}
In \emph{higher-order process calculi} 
communicated
values %exchanged in communications 
may contain  processes.
Higher-order concurrency has received significant attention 
from untyped and typed perspectives; see, e.g.,~\cite{SangiorgiD:expmpa,JeffreyR05,DBLP:journals/iandc/LanesePSS11,DBLP:journals/cl/KoutavasH12,MostrousY15}.
%=== One alternative
%The combination of features from the $\lambda$-calculus and the $\pi$-calculus
%enables \emph{higher-order process calculi} to exchange values that may contain  processes.
%=== Dimitris version: 
%The combination of features from the $\lambda$-calculus and the $\pi$-calculus,
%in \emph{higher-order process calculi} allows for exchanged values to contain  processes. 
%=== Previous version: 
%By combining features from the $\lambda$-calculus and the $\pi$-calculus, 
%in \emph{higher-order process calculi} exchanged values may contain  processes. 
In this work, we consider \HOp, a higher-order process calculus with \emph{session communication}:
it 
combines functional constructs (abstractions/applications, as in the call-by-value $\lambda$-calculus)
and 
concurrent primitives (synchronisation on shared names, 
communication on linear names, 
  %(value passing, labelled choice), 
recursion).
\newc{By amalgamating  functional and concurrent constructs, 
\HOp may specify %reciprocal exchanges (protocols) 
complex session protocols that 
include higher-order  processes (process passing)
and that
 can be 
 type-checked 
 using \emph{session types}~\cite{honda.vasconcelos.kubo:language-primitives}.
 Session types ensure that process specifications conform to prescribed protocols by 
enforcing usage policies for \emph{shared}  and \emph{linear resources}.
This distinction is important, for session-based concurrency can be seen as involving two distinct phases:
the first one is non-deterministic and uses shared names, as it represents the interaction of processes seeking compatible protocol partners;
the second phase proceeds deterministically along linear names, as it specifies the concurrent execution of the session protocols established in the first phase.} 


%These calculi allow us to specify   
%session protocols in which higher-order values 
%(mobile code) can be exchanged in a type-safe manner. 
%; 
%governed by session types, 
%such protocols cleanly distinguish between 
%linear and unrestricted behaviors in 
%%directed %point-to-point 
%communications.
%in particular via  comparisons with the first-order mobility of the $\pi$-calculus~\cite{MilnerR:calmp1}. 
Although models of higher-order concurrency with session 
communication % higher-order features 
have been already developed~\cite{tlca07,DBLP:journals/jfp/GayV10},
their \emph{behavioural equivalences} 
remain little understood. 
Clarifying the status of these equivalences is essential to, e.g., 
justify non-trivial optimisations in protocols involving both name and process passing.
\newc{
An important aspect the development of these typed equivalences is that typed semantics are usually {\em coarser} than untyped semantics. 
Indeed, since (session) types limit the contexts (environments) in which processes can interact, 
typed equivalences admit stronger properties than their untyped counterpart.
}
%for higher-order session calculi. 
%these two issues 
%have been thoroughly studied
%%are well-understood 
%for higher-order languages without sessions \cite{},
%but not for higher-order process calculi with sessions.
%This is unfortunate, given the wide applicability of session-based concurrency; indeed,
%session types are expressive enough to describe complex 
%communication structures found in practical protocols,  expressible, e.g., via recursive session types.
%Clarifying the status of typed equivalences and relative expressiveness for session languages

%but also for transferring key reasoning techniques between (higher-order) session calculi. 
%Our discovery is that \emph{linearity} of session types plays a vital role to 
%offer equalities/characterisations
%% and fully abstract encodability, 
%which to our knowledge have not been proposed before.   


%In this paper we study
%%address  behavioural equivalences for 
%\HOp, 
%%study behavioral equalities for \HOp, 
%an extension of the higher-order $\pi$-calculus~\cite{SangiorgiD:expmpa} with session primitives:
%\HOp contains constructs for 
%%session establishment
%synchronisation on shared names, 
%recursion, 
% (linear) session communication (value passing and
%labelled choice),
%abstractions and applications. 
%Abstractions are functions from values to processes, 
%\jpc{denoted}
%$\lambda x.P$; applications are 
%denoted $(\lambda x.P)V$, where the value $V$ is either a name or an abstraction.
%We study two significant subcalculi of \HOp, 
%\jpc{which}
%distil higher- and first-order mobility:
%the \HO-calculus, which is \HOp without recursion and name passing, and 
%the session \sessp-calculus \jpc{(here denoted~\sessp)}, which is \HOp without abstractions and applications.  
%While \sessp is, 
%in essence, the calculus in~\cite{honda.vasconcelos.kubo:language-primitives}, 
%this paper shows that \HO  is a new core calculus 
%for higher-order session concurrency.

A well-known behavioural equivalence for higher-order processes
is \emph{context bisimilarity}~\cite{San96H}. This 
 characterisation of %reduction-closed, 
barbed congruence 
offers an adequate distinguishing power at the price of heavy universal quantifications in output clauses.
Obtaining alternative 
characterisations of context bisimilarity
is thus a recurring, important problem 
for higher-order calculi---see, e.g.,~\cite{SangiorgiD:expmpa,San96H,JeffreyR05,DBLP:journals/cl/KoutavasH12,DBLP:journals/corr/Xu13a,lenglet_et_al:LIPIcs:2015:5364}. 
In particular, Sangiorgi~\cite{SangiorgiD:expmpa,San96H} has 
given %important 
%useful
characterisations of context bisimilarity
for higher-order processes; such 
characterisations, however,  %in~\cite{SangiorgiD:expmpa,San96H} 
do not scale to  
  calculi with \emph{recursive types}, which %in our experience 
  are essential to %the practice of 
  express practical protocols in 
session-based concurrency. A characterisation  
%context bisimilarity 
that solves this limitation was developed by Jeffrey and Rathke in~\cite{JeffreyR05};
their solution, however, does not consider \emph{linearity}---an important aspect in session-based concurrency, as explained above.

%\smallskip

\paragraph{This Work.}
Building upon~\cite{SangiorgiD:expmpa,San96H,JeffreyR05}, 
our discovery is that {linearity} as induced by session types plays a vital role 
%to 
%offer equalities and characterisations
% and fully abstract encodability, 
%which to our knowledge have not been proposed before. 
% 
in 
solving 
the %long-standing, 
open problem 
of characterising context bisimilarity for higher-order mobile processes with session communication.
Our approach is to exploit 
%protocol specifications given by session types to limit 
the coarser semantics induced by session types to limit
the behaviour of higher-order session processes. 
 Formally, we enforce this limitation by defining
a \emph{refined} labelled transition system (LTS)
which effectively 
narrows down the spectrum of allowed process behaviours, 
exploiting elementary processes inhabiting session types.
%thus enabling tractable reasoning techniques. 
We then introduce \emph{characteristic bisimilarity}: this  
 new notion of typed bisimilarity   is 
\emph{more tractable} than context bisimilarity, in that 
it relies on the refined LTS for input actions and, more importantly, 
does not appeal to universal quantifications on output actions. 
%shown to coincide with context bisimilarity.

Our main result is that characteristic  %tractable
bisimilarity coincides with context bisimilarity.
Besides confirming the value of characteristic bisimilarity as a useful reasoning technique for 
higher-order processes with sessions,
%for  specifications of trivial practical scenarios, 
this result is 
%also technically 
remarkable 
also from a technical perspective, for associated 
completeness proofs do not require 
operators for 
name matching,
% in the process language, 
in contrast to Jeffrey and Rathke's technique for higher-order processes
with recursive types~\cite{JeffreyR05}.
%Remarkably session type structures enable to provide 
%a coincidence without name-matching operators in the calculi.



%\smallskip

\paragraph{Outline.} 
%This paper  is structured as follows.
%\begin{enumerate}[$\bullet$]
%\item 
Next,
%%\secref{sec:overview} overviews 
we overview the
key ideas of characteristic bisimilarity, 
our 
characterisation of contextual equivalence.
%%\item 
Then, \secref{sec:calculus}  presents 
%we present
the %higher-order 
session calculus \HOp. 
%A small example is given in \S\,X.
\secref{sec:types} gives the session type system for \HOp
and states type soundness.
%\item 
\secref{sec:behavioural} 
develops %\emph{higher-order} and 
\emph{characteristic} bisimilarity and 
%which alleviates the issues of context bisimilarity~\cite{San96H} and 
states our main result: characteristic bisimilarity and contextual equivalence coincide for 
%is shown  to coincide for 
well-typed \HOp processes (\thmref{the:coincidence}).
In \secref{sec:relwork} we discuss related works, while
%The appendix summarises the typing system. 
%\end{enumerate}
%\noi
%The paper is self-contained. 
\secref{sec:concl}~collects some concluding remarks. 
%\textbf{The Appendix contains omitted definitions/proofs.}

This paper is an extended version of the conference paper~\cite{kouzapas_et_al:LIPIcs:2015:5365}.
This presentation includes full technical details---definitions and proofs, collected in the Appendix.
In particular, we introduce \emph{higher-order bisimilarity} (an auxiliary labeled 
bisimilarity) 
and highlight its role
in the proof of \thmref{the:coincidence}. 
We also elaborate further  on the 
use case scenario 
for characteristic bisimilarity 
given in~\cite{kouzapas_et_al:LIPIcs:2015:5365} (the Hotel Booking scenario).
We develop an additional example, given in  
\secref{sec:relwork}, and use it to compare our approach with Jeffrey and Rathke's~\cite{JeffreyR05}. 
Moreover, we offer extended discussions of related works.




\section{Overview: Characteristic Bisimulations}
\label{sec:overview}
% !TEX root = main.tex
\noi
\begin{comment}
%In  
%\S\,\ref{subsec:intro:expr}
%and
%\S\,\ref{subsec:intro:bisimulation}
We motivate further our contributions and 
give details of the technical challenges involved. 
Some  notation, formally introduced shortly, is useful here:
$\bout{u}{V} P$
and
$\binp{u}{x} P$ denote input- and output-prefixed processes.
Values $V, W$ can be either a name $u$ or an (name) abstraction $\abs{x}Q$.
Processes $P \Par Q$ and $\inact$ denote the parallel composition and inactive processes, respectively.
Given a (linear) session name $s$, we write $\dual{s}$ for its \emph{dual}; 
they are the two \emph{endpoints} of the same session: the restriction operation  
$\news{s}P$ simultaneously covers $s$ and $\dual{s}$ in~$P$. 
The restriction for shared name $a$ in $P$ is denoted $\news{a}P$.
We write $S$ to range over session types;
this way, e.g., session type $\btout{U} S'$ (resp. $\btinp{U} S'$) is
decrees that the output (resp. input) of a value of type $U$
must precede a protocol with type $S'$. 
The  terminated session is typed with $\tinact$.
%Given  type $U$, 
We write $\lhot{U}$ (resp. $\shot{U}$) for the 
linear (resp. unrestricted) functional type.

\subsection{Relative Expressiveness Results}
\label{subsec:intro:expr}
\myparagraph{Encoding Name Passing and Recursion into \HO.}
Our first encodability result highlights the expressiveness of 
the core higher-order calculus \HO, which lacks name passing and recursion. 
We encode \HOp into \HO, which entails also an encoding of \sessp into \HO.
The challenges in this encoding of concern exactly name passing and recursion.
To encode name output, we ``pack''
the name to be passed around into a suitable abstraction; 
upon reception, the receiver must ``unpack'' this object following a precise protocol.
The encoding formally is defined in Def.~\ref{d:enc:hopitoho}; we illustrate the encoding strategy below.
The encoding of name passing is:
\[
\begin{array}{rcll}
  \map{\bout{u}{w} P}	&=&	\bout{u}{ \abs{z}{\,\binp{z}{x} (\appl{x}{w})} } \map{P} \\
  \map{\binp{u}{x} Q}	&=&	\binp{u}{y} \newsp{s}{\appl{y}{s} \Par \bout{\dual{s}}{\abs{x}{\map{Q}}} \inact}
\end{array}
\]
and so we need 
exactly two (deterministic) reductions 
to unpack  name $w$.
The encoding of a recursive process $\recp{X}{P}$  is delicate, for it 
 must preserve the linearity of session endpoints. To this end, we
%\begin{enumerate}[i)]
%\item 
encode the recursion body $P$ as a (polyadic) name abstraction
in which free session names are converted into name variables.
This higher-order value is embedded in a sort of input-guarded 
``duplicator'' process; the encoding of process variable $X$ is then meant to 
invoke the duplicator in a by-need fashion to simulate recursion unfolding. 

%\item The recursion body $P$ is encoded in such a way that
%the  in $\map{P}$ (linear names) ; the obtained process
%is then used as the body of a  on those variables.
%\item Using a private session, the abstraction obtained in (i) is communicated to a
%process which instantiates the initial free session names in $P$, 
%in coordination with the encoding of the recursion variable $X$ (using a private session).
%\end{enumerate}
%The second step is also challenging:
%in essence, one should establish a private session with the encoding of the recursion  
%body in order to spawn copies of $\map{P}$ with appropriate free session names.
The use of polyadicity is crucial to the encoding; we shall get back to this point below.
It is worth noticing that the typing of the encoding requires 
a non tail recursive type of the form $\trec{t}{\btinp{\lhot{(S,\vart{t}}} \tinact}$
(see Def.~\ref{d:enc:hopitoho} for the precise formulation).

\smallskip 

\myparagraph{Other Encodings.}
We also give %the reverse of the previous encoding, namely 
an encoding of \HOp into \sessp. We rely on the well-known representability result of Sangiorgi~\cite{SangiorgiD:expmpa}. 
Since communicated processes may contain session names, in order to respect linearity and session protocols
the encoding enforces a distinction, depending on whether this kind of names is present in the communication object. If session names are present then a linear server trigger is deployed; otherwise, the replicated server  in~\cite{SangiorgiD:expmpa} can be used. 

As mentioned above, \HOp and \HO feature \emph{first-order} abstractions: 
only names can be used as arguments to abstractions.
Hence, given a (shared/linear) name $u$, in \HOp
we have the reduction $(\abs{x}{P}) \, u   \red  P \subst{u}{x}$.
We also consider \HOpp, an extension of \HOp with \emph{higher-order} abstractions.
 Thus, in \HOpp also an arbitrary value $V$ (possibly a process) can be an argument of an abstraction, 
 and one could have the reduction
 $(\abs{x}{P}) \, V   \red  P \subst{V}{x}$.
 We give an encoding of \HOpp into \HO: it  naturally extends that of \HOp into \HO;
see~\S\,\ref{subsec:hop}.

A well-known feature in process calculi is \emph{polyadicity}, i.e.,  
passing around tuples of values in communications. 
We consider the polyadic extension of \HOp, denoted \pHOp.
In \pHOp we have polyadicity in session communications and abstractions; 
polyadicity of shared names is ruled out by typing. 
This is enough for most purposes, including our encoding from \HOp into \HO.
In a session-typed setting, encoding polyadicity is straightforward, thanks to 
%polyadic arguments can be sent one by one, relying on 
the private character of 
(linear) session names --- see \S\,\ref{subsec:pho} for details.
%\[
%\begin{array}{rl}
%		\map{\binp{u}{x_1, \cdots, x_m} P}
%		 =  & \!\!\!\!
%		\binp{u}{x_1} \cdots ;  \binp{u}{x_m} \map{P}
%		\\
%%		\map{\bout{u}{u_1, \cdots, u_m} P}
%%		 =  & \!\!\!\!
%%		\bout{u}{u_1} \cdots ;  \bout{u}{u_m} \map{P}
%%		\\
%		\map{\bbout{u}{\abs{(x_1, \cdots, x_m)} Q} P}
%		= & \!\!\!\!
%		\bbout{u}{\abs{z}\binp{z}{x_1}\cdots ; \binp{z}{x_m} \map{Q}} \map{P}
%		\\ 
%		\map{\appl{x}{(u_1, \cdots, u_m)}}
%		= & \!\!\!\!
%		\newsp{s}{\appl{x}{s} \Par \bout{\dual{s}}{u_1} \cdots ; \bout{\dual{s}}{u_m} \inact} 
%	\end{array}
%\]
%Notice that encoding of polyadic abstraction/application requires an extra step, 
%in which a monadic abstraction is sent.

\smallskip

\myparagraph{A Non Encodability Result.}
We also show that shared names strictly add expressiveness to session calculi: that is,
there are (non deterministic) behaviors expressible with shared names not expressible using linear names only.
Although somewhat expected we do not know of a formal proof.
We propose such a formal proof, which relies crucially on the behavioral theory that we have introduced here
and on its determinacy properties. %\jp{EXPAND}.


\subsection{Tractable Bisimilarities for Session-Typed Processes}
\end{comment}

\label{subsec:intro:bisimulation}
\noi 
We outlines our motivations and methods 
to show how session types are used for formulating 
two tractable bisimulations. 

\myparagraph{Overcoming Issues of Context Bisimilarity.}
%The characterisation of contextual congruence given by 
Context bisimilarity ($\wbc$, Def.~\ref{def:wbc}) is a too demanding relation on processes. 
%In the following we motivate our
%proposal for alternative, more tractable characterisations.  
%For the sake of clarity, and to emphasise the novelties of our approach, 
%we often omit type information. 
%Formal definitions including types are in \S\,\ref{sec:behavioural}.
To see the issue, we show 
the following clause for output.
Suppose $P \,\Re\, Q$, for some context bisimulation $\Re$. Then:

\smallskip 

\begin{enumerate}[$(\star)$]
	\item	Whenever 
		$P \by{\news{\tilde{m_1}} \bactout{n}{V}} P'$
		there exist
		$Q'$ and $W$
		such that 
		$Q \by{\news{\tilde{m_2}} \bactout{n}{W}} Q'$
		and, for all $R$ with $\fv{R}=x$, 
		$\newsp{\tilde{m_1}}{P' \Par R\subst{V}{x}} \,\Re\, \newsp{\tilde{m_2}}{Q' \Par R\subst{W}{x}}$.
\end{enumerate}
\smallskip 
\noi 
Above, 
$\news{\tilde{m_1}} \bactout{n}{V}$ is the output label of 
value $V$ with extrusion of names in $\tilde{m_1}$.
To reduce the burden induced by 
universal quantification, we introduce \emph{higher-order}  and 
\emph{characteristic}  
bisimulations, two tractable equivalences denoted  $\hwb$ and $\fwb$, respectively.
As we work with an \emph{early} labelled transition system (LTS), 
we shall also aim at limiting the input actions,  
so to define a
bisimulation relation for the output clause without observing
infinitely many actions on the same input prefix. 
To this end, we take the following two steps: 
%
\begin{enumerate}[(a)]
	\item We replace $(\star)$ with a clause involving a more tractable process closure.
	\item We refine the transition rule for input in the LTS.
\end{enumerate}
%
\smallskip

\myparagraph{Trigger Processes with Session Communication.}
Concerning~(a), we exploit session types. 
We 
first 
observe that closure $R\subst{V}{x}$ 
in $(\star)$
is contextually bisimilar to the process:
\begin{equation}\label{equ:1}
	P = \newsp{s}{\appl{(\abs{z}{\binp{z}{x}{R}})}{s} \Par \bout{\dual{s}}{V} \inact}
\end{equation}
\noi 
%where $\binp{z}{x}{R}$ is an input and $\bout{\dual{s}}{V} \inact$
%is an output 
%on the endpoint $\dual{s}$ (the dual of $s$).
In fact,
we have $P \wbc R\subst{V}{x}$, 
since 
application and reduction of dual endpoints 
%($s$ and $\dual{s}$) 
are deterministic.  
Now let us
consider process $T_{V}$ below, where $t$ is a fresh name:
\begin{equation}\label{equ:0}
T_{V} = \hotrigger{t}{V}
\end{equation}
%We call $\abs{z}{\binp{z}{x} R}$ a {\bf\em trigger value}. 
If $T_{V}$ inputs $\abs{z}{\binp{z}{x} R}$
we can simulate the closure of $P$:
\begin{equation}\label{equ:2}
%\hotrigger{t}{V_1} 
T_{V}
\by{\bactinp{t}{\abs{z}{\binp{z}{x} R}}} P 
\wbc 
R\subst{V}{x}
\end{equation}
Processes such as $T_{V}$ 
offer a value at a fresh name; we will use this class of 
{\bf\em trigger processes} to define a
 refined bisimilarity without the demanding 
output clause $(\star)$. Given a fresh name $t$, 
we write $\htrigger{t}{V}$ to 
stand for a trigger process $T_{V}$ for value $V$.
We note that 
in contrast to previous approaches~\cite{SaWabook,JeffreyR05} 
our {trigger processes} do {\em not} use recursion or 
replication. This is crucial for preserving linearity of session names.  

\smallskip

%Then we can use 
%$\newsp{\tilde{m_1}}{P_1 \Par \htrigger{t}{V_1}}$ instead 
%of Clause 1) in Definition \ref{def:wbc} if we input 
%$\abs{z}{\binp{z}{x} R}$.   

\myparagraph{Characteristic Processes and Values.}
Concerning (b), we limit the possible $R$ processes by
exploiting session types.
The key concept is that of {\bf \emph{characteristic process/values}}
of a type (Def.~\ref{def:char}),  
%The characteristic process of a session type $S$ is the process inhabiting $S$. 
the 
simplest value inhabiting that type.
This way, e.g., let $S = \btinp{\shot{S_1}} \btout{S_2} \tinact$
be a session type: first
input a function, %from values $S_1$ to processes, 
then output a value of type $S_2$.
Then, process $Q = \binp{u}{x} (\bout{u}{s_2} \inact \Par \appl{x}{s_1})$
is a characteristic process for $S$ 
\jpc{along name $u$.}
%Thus, characteristic processes follow the communication structures decreed by session types.
Given a session type $S$, we write $\mapchar{S}{u} $
for its characteristic process along name $u$
(cf.~\defref{def:char}).
Also, %Similarly, 
given value type $U$, we write 
$\omapchar{U}$ to denote its characteristic value.


We use the characteristic %Precisely, we exploit  the
 characteristic value %$\lambda x.\mapchar{U}{x}$. %$\lambda x.\mapchar{U}{x}$. 
$\omapchar{U}$
 to limit input transitions.
Following the same reasoning as (\ref{equ:1})--(\ref{equ:2}), 
we can use an alternative trigger process, called
{\bf\em characteristic trigger process} with type 
$U$ to replace clause
% (1) in Definition~\ref{def:wbc}:
($\star$) in Def.~\ref{def:wbc}:
\begin{equation}
	\label{eq:4}
	\ftrigger{t}{V}{U} \defeq \fotrigger{t}{x}{s}{\btinp{U} \tinact}{V}
\end{equation}

%Note that if $U=L$, $\ftrigger{t}{V}{U}$ subsumes 
%$\htrigger{t}{V}$. 
\noi 
\jpc{Thus, in contrast to the trigger process~\eqref{equ:0}, the characteristic trigger process 
in~\eqref{eq:4}
does not involve a
higher-order communication on $t$.}
To refine the input transition system, we need to observe 
an additional value, 
$\abs{{x}}{\binp{t}{y} (\appl{y}{{x}})}$, 
called the {\bf\em trigger value}. 
This is necessary, because it turns out
that a characteristic value 
alone as the observable input 
is not enough to define a sound bisimulation.
Roughly speaking, the trigger value is used
to observe/simulate application processes.
%to {\em count} the number of free higher-order variables inside 
%the receiver. 
%\jpc{See Example~\ref{ex:motivation} for further details.}

\smallskip 
\myparagraph{Refined Input Transition Rule.}
Based on 
the above discussion, we refine 
the (early) transition rule for input actions. 
We write $P \by{\bactinp{n}{V}} P'$ for the input transition along $n$.
The transition rule for input roughly becomes 
(see Def.~\ref{def:rlts} for details):
\[
		\tree {
%\begin{array}{c}
P \by{\bactinp{n}{V}} P' \quad  V  \scong
(\abs{{x}}{\binp{t}{y} (\appl{y}{{x}})}
 \vee  \omapchar{U}  \vee m)  \textrm{ with } t \textrm{ fresh} 
		}{
			P' \hby{\bactinp{n}{V}} P'
		}
\]
Note the distinction between standard and refined transitions: $\by{\bactinp{n}{V}}$ vs. $\hby{\bactinp{n}{V}}$.
Using this rule, we define an alternative  LTS
with refined 
\jpc{(higher-order)}
input. %; all other rules are kept unchanged.
This refined LTS is used for 
both higher-order ($\hwb$) and characteristic ($\fwb$) bisimulations,
which replace the demanding clause~$(\star)$ with 
more tractable clauses based on trigger processes 
\jpc{(cf.~\eqref{equ:0})} 
and characteristic 
trigger processes
\jpc{(cf.~\eqref{eq:4})},
respectively (Defs.~\ref{d:hbw} and~\ref{d:fwb}).
Later we show $\hwb$ is useful for \HOp and \HO, but 
$\fwb$ can be uniformly used in all subcalculi, including \sessp. 

\jpc{For the sake of minimality our calculus lacks a matching construct,
which is usually crucial to prove completeness of bisimilarity.
To compensate for this absence, we use types:
instead of name matching, a process trigger embeds a name into a characteristic
process so to observe its session behavior.}

%\dk{We stress-out that our calculus lacks
%a matching construct
%which is usually a crucial element to prove completeness of bisimilarity.
%Nevertheless, we use types
%%and specifically the characteristic process
%to compensate;
%%for the absence of matching, i.e.~
%instead of name matching, a process trigger embeds a name into a characteristic
%process so to observe its session behavior.}
%Notice that while Definition \ref{d:hbw} is useful for 
%\HOp and its higher-order variants,
%Definition \ref{d:fwb} is useful for first-order sub-calculi of \HOp.




%%\myparagraph{Outline}
%\subsection{Outline}
%\noi \S\,\ref{sec:calculus} presents the calculi; 
%\S\,\ref{sec:types} presents types;
%the tractable bisimulations are in \S\,\ref{sec:behavioural};
%the notion of encoding is in \S\,\ref{s:expr};
%\S\,\ref{sec:positive} and \S\,\ref{sec:negative}
%present positive and negative encodability results, resp;
%\S\,\ref{sec:extension} discusses extensions; and 
%\S\,\ref{sec:relwork} concludes with related work;
%Appendix summarises the typing system. 
%The paper is self-contained. 
%{\bf\em Omitted definitions, additional related work and full proofs can be found 
%in a technical report, available from \cite{KouzapasPY15}.} 


\section{A Higher-Order Session $\pi$-Calculus}
\label{sec:calculus}
% !TEX root = main.tex
\section{A Higher Order Session Calculus}

We define a session calculus augmented with higher order semantics.

\subsection{Syntax}

We assume the countable sets:

\begin{tabular}{rcllcrcll}
	$S$ &$=$& $\set{s, s_1, \dots}$ & Sessions
	&$\qquad$&
	$\dual{S}$ &$=$& $\set{\dual{s} \setbar s \in S}$ & Dual Sessions
	\\

	Var &$=$& $\set{x,y,z, \dots}$ & Variables
	&$\qquad$&
	PVar &$=$& $\set{\varp{X}, \varp{Y}, \varp{Z}, \dots}$ & Process Variables\\

	$\mathcal{R}$ &$=$& $\set{\varp{r}, \varp{r_1}, \dots}$ & Recursive Variables
	&$\qquad$&
	Abs &$=$& $\set{\abs{x}{P} \setbar P \textrm{ is a process}}$
\end{tabular}

with the set of names $\mathcal{N} = S \cup \dual{S}$ and let $k \in \mathcal{N} \cup Var$,
$V \in \mathcal{V} \cup PVar \cup Abs$.
Also for convenience we sometimes denote shared names with $a, b, \dots$ although $a,b,\dots \in N$.

\paragraph{Processes}

The syntax of processes follows:

\begin{tabular}{rcl}
	$P$	&$\bnfis$&	$\bout{k}{k'} P \bnfbar \binp{k}{x} P \bnfbar \varp{r} \bnfbar \recp{r}{P}$\\
		&$\bnfbar$&	$\bout{k}{\abs{x} Q} P \bnfbar \binp{k}{X}P \bnfbar \appl{X}{k}$\\ 
		&$\bnfbar$&	$\bsel{s}{l} P \bnfbar \bbra{s}{l_i:P_i}_{i \in I} \bnfbar 
				P_1 \Par P_2 \bnfbar \news{s} P \bnfbar \inact$
\end{tabular}

We say that processes are program if they contain
no free variables or free process variables.

\subsection{Reduction Relation}

\subsubsection{Structural Congruence}
\[
	\begin{array}{c}
		P \Par \inact \scong P \qquad P_1 \Par P_2 \scong P_2 \Par P_1 \qquad P_1 \Par (P_2 \Par P_3)
		\qquad (P_1 \Par P_2) \Par P_3 \qquad \news{s} \inact \scong \inact\\
		s \notin \fn{P_1} \Rightarrow P_1 \Par \news{s} P_2 \scong \news{s}(P_1 \Par P_2)
		\qquad \rec{r}{P} \scong P\subst{\rec{r}{P}}{r}
	\end{array}
\]

\subsubsection{Process Variable Substitution}
\[
	\begin{array}{rclcrcl}
			\appl{X}{k} \subst{\abs{x}{Q}}{\X} & = & Q \subst{k}{x} \\
		(\bout{s}{\abs{y}{P_1}} P_2) \subst{\abs{x}{Q}}{\X} &=& \bout{s}{\abs{y}{P_1 \subst{\abs{x}{Q}}{\X}}} (P_2 \subst{\abs{x}{Q}}{\X}) \\
		(\binp{s}{\Y} P) \subst{\abs{x}{Q}}{\X} &=& \binp{s}{\Y} (P \subst{\abs{x}{Q}}{\X}) \\
		(\bsel{s}{l} P) \subst{\abs{x}{Q}}{\X} &=& \bsel{s}{l} (P \subst{\abs{x}{Q}}{\X})\\
		(\bbra{s}{l_i : P_i}_{i \in I}) \subst{\abs{x}{Q}}{\X} &=& \bbra{s}{l_i : P_i \subst{\abs{x}{Q}}{\X}}_{i \in I}\\
		(P_1 \Par P_2) \subst{\abs{x}{Q}}{\X} & = & P_1 \subst{\abs{x}{Q}}{\X} \Par P_2 \subst{\abs{x}{Q}}{\X}\\
		(\news{s} P) \subst{\abs{x}{Q}}{\X} & = & \news{s} (P \subst{\abs{x}{Q}}{\X})\\
		\inact \subst{\abs{x}{Q}}{\X} & = & \inact
	\end{array}
\]

\subsubsection{Operational Semantics}
\[
	\begin{array}{rcl}
%		\spi & \breq{a}{s} P_1 \Par \bacc{a}{x} P_2 &\red& P_1 \Par P_2 \subst{s}{x}
	\end{array}
\]
\[
	\begin{array}{rclcrcl}
		&& && \bout{s}{\abs{x}{P}} P_1 \Par \binp{s}{\X} P_2 &\red& P_1 \Par P_2 \subst{\abs{x}{P}}{\X} \\
		&& && \bout{s}{s'} P_1 \Par \binp{s}{x} P_2 &\red& P_1 \Par P_2 \subst{s'}{x}\\
		&& && \bsel{s}{l_k} P \Par \bbra{s}{l_i : P_i}_{i \in I} &\red& P \Par P_k \qquad k \in I \\
		P_1 &\red& P_1' &\Rightarrow& P_1 \Par P_2 &\red& P_1' \Par P_2\\
		P &\red& P' &\Rightarrow& \news{s} P &\red& \news{s} P'\\
		P &\scong \red \scong& P'&\Rightarrow& P &\red& P' 		
	\end{array}
\]

\subsection{Subcalculi}

We identify two subcalculi of the Higher Order Session Calculus:
\begin{enumerate}
	\item	$\pHO$ uses only the semantics that allow abstraction passing
		(i.e.\ is not using the syntax in the first line of the definition of $P$).
	\item	$\spi$ uses only the semantics that allow name passing
		(i.e.\ is not using the syntax in the second line of the definition of $P$).
\end{enumerate}

Later in this paper we will identify a third typed subcalculi derived
from $\spi$ that is defined on the non-usage of shared sessions.

\begin{proposition}[Normalisation]
	\label{prop:normal_form}
	Let $P$ a Higher Orser Session Calculus process, then
	$P \scong \newsp{\tilde{s}}{P_1 \Par \dots \Par P_n}$ with
	$P_1, \dots, P_n$ session prefixed processes
	%\dk{recursion $\recp{r}{P}$}
	or application process $\appl{X}{k}$.
\end{proposition}

\begin{proof}
	The proof is a simple induction on the syntax of $P$.
\end{proof}


\section{Example: Processes}
\newcommand{\rtype}{\mathsf{roomtype}}
\newcommand{\Quote}{\mathsf{quote}}
\newcommand{\accept}{\mathsf{accept}}
\newcommand{\reject}{\mathsf{reject}}
\newcommand{\creditc}{\mathsf{credit}}

\newcommand{\Client}{\mathsf{Client}}

\begin{example}[Program Equivalence]

We introduce the usecase scenario where a Client wants to book
a hotel for her holidays in a remote island
(cf.~\cite{MostrousY15} for a similar usecase).
The Client narrows down her choice to two hotels.
It then requires a quote from the two hotels in order to
make her choice. The Round Trip Time required for
taking quotes from the two hotels in not optimal (cf.~\cite{MostrousY15}),
so she decides to send remote codes to the hotels
to automatically negotiate and book the hotel for
her:
%
\[
	\begin{array}{rcl}
		P &\defeq& \bout{x}{\rtype} \binp{x}{\Quote} \bout{y}{\Quote}
		y \triangleleft \left\{
				\begin{array}{l}
					\accept: \bsel{x}{\accept} \bout{x}{\creditc} \inact,\\
					\reject: \bsel{x}{\reject} \inact
				\end{array}
				\right\}
		\\[6mm]
		\Client_1 &\defeq& \newsp{h_1, h_2}{\bout{s_1}{\abs{x}{P \subst{h_1}{y}}} \bout{s_2}{\abs{x}{P \subst{h_2}{y}}} \inact \Par\\
			&&
			\begin{array}{lll}
				\binp{\dual{h_1}}{x} \binp{\dual{h_2}}{y} & \If\ x \leq y\ \Then & \bsel{\dual{h_1}}{\accept} \bsel{\dual{h_2}}{\reject} \inact\\
				& \Else& \bsel{\dual{h_1}}{\reject} \bsel{\dual{h_2}}{\accept} \inact
			\end{array}
		}
	\end{array}
\]
%
\begin{itemize}
	\item	Process $P$ is the remote code responsible for negotiation with a hotel.
		Channel $x$ is intended to be instatiated by the hotel as the negotiating
		endpoint. Channel $y$ is used to interact with $\Client_1$.

	\item	Process $P$
		i) sends the room requirements to the hotel;
		ii) receives a quote from the hotel;
		iii) sends the quote to the $\Client_1$;
		iv) expects a choice from the $\Client_1$ whether to accept or reject the offer and;
		v) If the choice is accept it informs the hotel and performs the booking,
		if the choice is reject it informs the hotel and ends the session.

	\item	$\Client_1$ instantiates two copies of process $P$ as abstractions
		on session $x$. It further uses two
		fresh endpoints $h_1, h_2$ to substitute channels $y$, respectively,
		in order for the two instances of $P$ to be able to interact
		with $\Client_1$.
	
	\item	$\Client_1$ then sends the two abstractions instances of $P$
		to the two hotels via sessions $s_1$ and $s_2$, respectively.

	\item	$\Client_1$ uses the dual endpoints $\dual{h_1}$ and $\dual{h_2}$
		to receive the negotiation
		result from the two remote instances of $P$ and then inform the two
		processes for the final booking decision.
\end{itemize}

The above scenario does not add a significant gain
to the time needed for the entire protocol to take
place, since the two remote processes are required
to send and receive data to $\Client_1$.

As an alternative we can propose a different implementation
of the same scenario that requires from the two
remote processes to interact with each other,
instead of $\Client_1$, to reach to a consensus:
%
\[
	\begin{array}{rcl}
		P_1 &\defeq&	\bout{x}{\rtype} \binp{x}{\Quote_1} \bout{y}{\Quote_1} \binp{y}{\Quote_2}\\
			&&
				\begin{array}{ll}
					\If\ \Quote_1 \leq \Quote_2\ \Then & \bsel{x}{\accept} \bout{x}{\creditc} \inact\\
					\Else & \bsel{x}{\reject} \inact
				\end{array}
		\\
		P_2 &\defeq&	\bout{x}{\rtype} \binp{x}{\Quote_2} \bout{y}{\Quote_1} \bout{y}{\Quote_1}\\
			&&
				\begin{array}{ll}
					\If\ \Quote_1 \leq \Quote_2\ \Then & \bsel{x}{\accept} \bout{x}{\creditc} \inact\\
					\Else & \bsel{x}{\reject} \inact
				\end{array}
		\\
		\Client_2 &\defeq& \newsp{h}{\bout{s_1}{\abs{x}{P_1 \subst{h}{y}}} \bout{s_2}{\abs{x}{P_2 \subst{\dual{h}}{y}}} \inact}
	\end{array}
\]
\end{example}

\begin{itemize}
	\item	Processes $P_1$ and $P_2$ are responsible for negotiating a quote from the
		hotel in the same fashion as process $P$ in the previous implementation.

	\item	The difference with process $P$ is that the channel $y$ is used for
		interaction between process $P_1$ and $P_2$. Both processes send
		there quotes to each other and then internally follow the same
		logic to reach to a decision.

	\item	The role of $\Client_2$ is to instantiate $P_1$ and $P_2$ as abstractions
		on name $x$. It further substitutes
		the two endpoints of a fresh channel $h$ to channels $y$ respectively,
		in order for the two instances to be able to communicate with each other.

	\item	Process $\Client_2$ then uses sessions $s_1$ and $s_2$ to send the two
		instances of $P_1$ and $P_2$ to the two hotels.
\end{itemize}

We can show that process $\Client_1$ and $\Client_2$
are interchangeable by showing that they are bisimilar.
We use characteristic bisimulation to give a bisimulation
closure.
%
%\begin{eqnarray*}
%	S &=& \set{
%		\Client_1, \Client_2\\
%	&&	\newsp{h_1, h_2}{\bout{s_1}{\abs{x}{P \subst{h_1}{y}}} \bout{s_2}{\abs{x}{P \subst{h_2}{y}}} \inact \Par\\
%	&&
%		\begin{array}{lll}
%			\binp{\dual{h_1}}{x} \binp{\dual{h_2}}{y} & \If\ x \leq y\ \Then & \bsel{\dual{h_1}}{\accept} \bsel{\dual{h_2}}{\reject} \inact\\
%			& \Else& \bsel{\dual{h_1}}{\reject} \bsel{\dual{h_2}}{\accept} \inact
%		\end{array}}
%		}
%\end{eqnarray*}

\subsection{Types}

Assuming $S = \btout{\Quote} \btbra{\accept: \tinact, \reject: \tinact}$ and
$U = \btout{\rtype} \btinp{\Quote} \btsel{\accept: \btout{\creditc} \tinact, \reject: \tinact }$.
The type for $\abs{x}{P}$ is
%
\[
	\es; \es; y: S \proves \abs{x}{P} \hastype \lhot{U}
\]
%
\noi and the type for $\Client_1$ is
%
\[
	\es; \es; s_1: \btout{\lhot{U}} \tinact \cat s_2: \btout{\lhot{U}} \tinact \proves \Client_1 \hastype \Proc
\]
%
The types for $P_1$ and $P_2$ are
%
\begin{eqnarray*}
	\es; \es; y: \btout{\Quote} \btinp{\Quote} \tinact \proves \abs{x}{P_1} \hastype \lhot{U}\\
	\es; \es; y: \btinp{\Quote} \btout{\Quote} \tinact \proves \abs{x}{P_2} \hastype \lhot{U}
\end{eqnarray*}
%
\noi ant the type for $\Client_2$ is
%
\[
	\es; \es; s_1: \btout{\lhot{U}} \tinact \cat s_2: \btout{\lhot{U}} \tinact \proves \Client_2 \hastype \Proc
\]
%


\section{Types and Typing}
\label{sec:types}
% !TEX root = main.tex
%\newpage
\section{Session Types for $\HOp$}
\label{sec:types}

In this section we define a session typing system for
$\HOp$ and establish its main properties. We use as
a reference the type system for higher-order session processes 
developed by Mostrous and Yoshida~\cite{tlca07,mostrous09sessionbased}.
Our system is simpler than that in~\cite{tlca07}, in order to distil the key
features of higher-order communication in a session-typed setting.

%%%%%%%%%%%%%%%%%%%%%%%%%%%%%%%%%%%%%%%%%%%%%%%%%%
%  SYNTAX FOR TYPES
%%%%%%%%%%%%%%%%%%%%%%%%%%%%%%%%%%%%%%%%%%%%%%%%%%

\subsection{Syntax}

We define the syntax of session types for \HOp.

\begin{definition}[Syntax of Types]\rm
	\label{def:types}
	The syntax of types is defined on the types for sessions $S$,
	and the types for values $U$:
	%
	\[
	\begin{array}{lrcl}
		\textrm{(value)} & U & \bnfis &		C \bnfbar L 
		\\

		\textrm{(names)} & C & \bnfis &	S \bnfbar \chtype{S} \bnfbar \chtype{L}
		\\

		\textrm{(lambda)} & L & \bnfis &	\shot{C} \bnfbar \lhot{C}
		\\

		\textrm{(session)} & S,T & \bnfis & 	\btout{U} S \bnfbar \btinp{U} S
							\bnfbar \btsel{l_i:S_i}_{i \in I} \bnfbar \btbra{l_i:S_i}_{i \in I}\\
					&&\bnfbar&	\trec{t}{S} \bnfbar \vart{t}  \bnfbar \tinact
	\end{array}
	\]
\end{definition}
%
\noi \myparagraph{Types for Values}
Types for values range over symbol $U$ which includes
first-order types $C$ and higher-order types $L$.
First-order types $C$ are used to type names;
session types $S$ type session names and shared types
$\chtype{S}$ or $\chtype{L}$ type shared names that
carry session values and higher-order values, respectively.
Higher-order types $L$ are used to type abstraction values;
$\shot{C}$ and $\lhot{C}$ denote
shared and linear abstraction types, respectively.

\myparagraph{Session Types}
The syntax of session types $S$ follows the usual
(binary) session types with
recursion~\cite{honda.vasconcelos.kubo:language-primitives,GH05}.
{\em Output type} $\btout{U} S$ is assigned to a name that
first sends a value of type $U$ and then follows
the type described by $S$.
Dually, the {\em input type} $\btinp{U} S$ is assigned to a name
that first receives a value of type $U$ and then continues as $S$. 
Session types for labelled choice and selection, 
%are standard: they are 
written $\btbra{l_i:S_i}_{i \in I}$ and $\btsel{l_i:S_i}_{i \in I}$, respectively,
require a set of types $\set{S_i}_{i \in I}$ that correspond to a set of
labels $\set{i \in I}_{i \in I}$. 
{\em Recursive session types} are defined using the primitive recursor.
We that a {\em type variable} is guarded, i.e.~$\trec{t}{\vart{t}}$ is not allowed.
We let $\mathsf{T}$ to be the set of all well-formed types and
Type $\tinact$ is the termination type.
\ST to be the set of all well-formed session types.

\begin{remark}[Restriction on Types for Values]
	The syntax for value types is restricted
	to dissallow types of the form:
	\begin{enumerate}[$\bullet$]
		\item	$\chtype{\chtype{U}}$: shared names
			cannot carry shared names; and

		\item  $\shot{U}$: abstractions do not
			bind higher-order variables.
	\end{enumerate}
\end{remark}

The difference between the syntax of process
in \HOp with the syntax of processes in~\cite{tlca07}
is also reflected on the two corresponding type syntax;
the type structure  in~\cite{tlca07}, 
supports the arrow types of the form $U \sharedop T$ and 
$U \lollipop T$, where $T$ denotes an arbitrary type of a term 
(i.e.~a value or a process).

%%%%%%%%%%%%%%%%%%%%%%%%%%%%%%%%%%%%%%%%%%%%%%%%%%
%  DUALITY
%%%%%%%%%%%%%%%%%%%%%%%%%%%%%%%%%%%%%%%%%%%%%%%%%%

\subsection{Duality}

Duality is defined following the co-inductive
approach following~\cite{GH05,TGC14}.
We first require the notion of type equivalence.
%
\begin{definition}[Type Equivalence]\rm
\label{def:type_equiv}
	Define function $F(\Re): \mathsf{T} \longrightarrow \mathsf{T}$:
	%
	\[
		\begin{array}{rcl}
			F(\Re) 	&=&	\set{(\tinact, \tinact)} \\
				&\cup&	\set{(\chtype{S}, \chtype{T}) \bnfbar S\ \Re\ T} \cup \set{(\chtype{L_1}, \chtype{L_2}) \bnfbar L_1\ \Re\ L_2}\\
				&\cup&	\set{(\shot{C_1}, \shot{C_2}), (\lhot{C_1}, \lhot{C_2}) \bnfbar C_1\ \Re\ C_2}\\
				&\cup&	\set{(\btout{U_1} S, \btout{U_2} T)\,,\, (\btinp{U_1} S, \btinp{U_1} T) \bnfbar U_1\ \Re\ U_2, S\ \Re\ T}\\
				&\cup&	\set{(\btsel{l_i: S_i}_{i \in I} \,,\, \btsel{l_i: T_i}_{i \in I}) \bnfbar  S_i\ \Re\ T_i}\\
				&\cup&	\set{(\btbra{l_i: S_i}_{i \in I}\,,\, \btbra{l_i: T_i}_{i \in I}) \bnfbar S_i\ \Re\ T_i}\\
				&\cup&	\set{(S\,,\, T) \bnfbar S\subst{\trec{t}{S}}{\vart{t}}\ \Re\ T)}
				\cup	\set{(S\,,\, T) \bnfbar S\ \Re\ T\subst{\trec{t}{T}}{\vart{t}})}
		\end{array}
	\]	
	\noi Standard arguments ensure that $F$ is monotone, thus the greatest fixed point
	of $F$ exists. Let type equivalence be defined as $\iso = \nu X. F(X)$.
\end{definition}
%
\noi Type equivalence is in a essence a co-inductive definition that
equates types up-to recursive unfolding.

The duality relation is defined in terms of type equivalence.
%
\begin{definition}[Duality]\rm
	Define function $F(\Re): \mathsf{ST} \longrightarrow \mathsf{ST}$:
	%
	\[
		\begin{array}{rcl}
			F(\Re) 	&=&	\set{(\tinact, \tinact)}\\
				&\cup&	\set{(\btout{U_1} S, \btinp{U_2} T)\,,\, (\btinp{U} S, \btout{U} T) \bnfbar U_1 \iso U_2, S\ \Re\ T}\\
				&\cup&	\set{(\btsel{l_i: S_i}_{i \in I} \,,\, \btbra{l_i: T_i}_{i \in I}) \bnfbar  S_i\ \Re\ T_i}\\
				&\cup&	\set{(\btbra{l_i: S_i}_{i \in I}\,,\, \btsel{l_i: T_i}_{i \in I}) \bnfbar S_i\ \Re\ T_i}\\
				&\cup&	\set{(S\,,\, T) \bnfbar S\subst{\trec{t}{S}}{\vart{t}}\ \Re\ T)}\\
				&\cup&	\set{(S\,,\, T) \bnfbar S\ \Re\ T\subst{\trec{t}{T}}{\vart{t}})}
		\end{array}
	\]	
	\noi Standard arguments ensure that $F$ is monotone, thus the greatest fixed point
	of $F$ exists. Let duality be defined as $\dualof = \nu X. F(X)$.
\end{definition}
%
Duality is applied co-inductively to session types
up-to recursive unfolding.
Dual session types are prefixed
on dual session type constructors
($!$ is dual to $?$ and $\oplus$ is dual to $\&$)
that carry equivalent types.
%The co-inductive definition of duality relates
%session types up-to recursive unfolding.

%%%%%%%%%%%%%%%%%%%%%%%%%%%%%%%%%%%%%%%%%%%%%%%%%%
%  TYPING SYSTEM
%%%%%%%%%%%%%%%%%%%%%%%%%%%%%%%%%%%%%%%%%%%%%%%%%%

\subsection{Type Environments and Judgements}
We follow~\cite{tlca07},
to defined the typing environment.
%
\begin{definition}[Typing environment]\rm
	We define the {\em shared type environment} $\Gamma$,
	the {\em linear type environment} $\Lambda$, and
	the {\em session type environment} $\Delta$ as:
	\[
	\begin{array}{llcl}
		\text{(Shared)}		& \Gamma  & \bnfis &	\emptyset \bnfbar \Gamma \cat x: \shot{C} \bnfbar \Gamma \cat u: \chtype{S} \bnfbar
								\Gamma \cat u: \chtype{L} \bnfbar \Gamma \cat \varp{X}: \Delta
		\\
		\text{(Linear)}		& \Lambda & \bnfis &	\emptyset \bnfbar \Lambda \cat x: \lhot{C}
		\\
		\text{(Session)}	& \Delta  & \bnfis &	\emptyset \bnfbar \Delta \cat u:S
	\end{array}
	\]
	We imply
	\begin{enumerate}[i.]
		\item	Domains of $\Gamma, \Lambda, \Delta$ are pairwise distinct.
		\item	Weakening, contraction and exchange apply to shared environment $\Gamma$.
		\item	Exchange applies to linear environments $\Lambda$ and $\Delta$. 
	\end{enumerate}
\end{definition}
%
\noi We define typing judgements for values $V$
and processes $P$:
%
\[	\begin{array}{c}
		\Gamma; \Lambda; \Delta \proves V \hastype U \qquad \qquad \qquad \qquad \Gamma; \Lambda; \Delta \proves P \hastype \Proc
	\end{array}
\]
%
\noi The first judgement asserts that under environment $\Gamma; \Lambda; \Delta$
values $V$ have type $U$,
whereas the second judgement asserts that under environment $\Gamma; \Lambda; \Delta$
process $P$ has the typed process type $\Proc$.

%%%%%%%%%%%
%	Type system description
%%%%%%%%%%%

\subsection{Typing Rules}

\begin{figure}[!t]
\[
	\begin{array}{c}
%		\jrule{ }{\Gamma ; \emptyset; \emptyset \vdash \UnitV \hastype \Unit}{Unit} 
%		\qquad\quad  
		\trule{Session}~~\Gamma; \emptyset; \set{k:S} \proves k \hastype S 
		\\[2mm]
		\trule{Shared}~~\Gamma \cat a : \chtype{S}; \emptyset; \emptyset \proves a \hastype \chtype{S}
		\qquad
		\trule{LVar}~~\Gamma; \set{X: \lhot{S}}; \emptyset \proves X \hastype \lhot{S} 
		\\[2mm]
		\trule{Prom}~~\tree{
			\Gamma; \emptyset; \emptyset \proves V \hastype \lhot{S}
		}{
			\Gamma; \emptyset; \emptyset \proves V \hastype \shot{S}
		} 
		\qquad\quad  
		\trule{Derelic}~~\tree{
			\Gamma; \Lambda \cat X{:}\lhot{S}; \Sigma \proves P \hastype \Proc
		}{
			\Gamma \cat X:\shot{S}; \Lambda; \Sigma \proves P \hastype \Proc
		} 
		\\[4mm]
%		\trule{Subt}~~\tree{
%			\Gamma; \Lambda; \Sigma \proves P \hastype T \quad \Sigma \subt \Sigma' \quad T \subt T'
%		}{
%			\Gamma ; \Lambda; \Sigma' \vdash P \hastype T'
%		} 
%		\qquad\quad

		\trule{Abs}~~\tree{
			\Gamma; \Lambda; \Sigma \cat x: S \proves P \hastype \Proc
		}{
			\Gamma; \Lambda; \Sigma \proves \abs{x}{P} \hastype \lhot{S}
		}
		\\[4mm]

		\trule{App}~~\tree{(U = \lhot{S}) \lor (U = \shot{S}) \quad \Gamma; \Lambda_1; \Sigma_1 \proves X \hastype U  \quad \Gamma; \Lambda_2; \Sigma_2 \proves k \hastype S
		}{
			\Gamma; \Lambda_1 \cup \Lambda_2; \Sigma_1 \cup \Sigma_2 \proves \appl{X}{k} \hastype \Proc
		} 
		\\[4mm]

		\trule{Send}~~\tree{
			\Gamma; \Lambda_1; \Sigma_1 \proves P \hastype \Proc  \quad \Gamma; \Lambda_2; \Sigma_2 \vdash V \hastype U  \quad (k:S \in \Sigma_1 \cup \Sigma_2)
		}{
			\Gamma; \Lambda_1 \cup \Lambda_2; (\Sigma_1 \cup \Sigma_2)\backslash\set{k:S} \cat k:\btout{U} S \proves \bout{k}{V} P \hastype \Proc
		}

		\\[4mm]
		\trule{Conn}~~\tree{
			\Gamma; \Lambda; \Sigma \cat x:S \proves P \hastype \Proc  \quad \Gamma; \emptyset; \emptyset \proves a \hastype \chtype{S}
		}{
			\Gamma; \Lambda; \Sigma \proves \binp{a}{x} P \hastype \Proc
		}
		\\[4mm]
%		\trule{ConnDual}~~\tree{
%			\Gamma; \Lambda; \Sigma \cat x: S_1 \proves P \hastype \Proc  \quad \Gamma; \emptyset; \emptyset \proves k \hastype \chtype{S_2} \quad S_1 \dualof S_2
%		}{
%			\Gamma; \Lambda; \Sigma \proves \bout{k}{x} P \hastype \Proc
%		}
%		\\[4mm]

		\trule{ConnDual}~~\tree{
			\Gamma; \Lambda; \Sigma \proves P \hastype \Proc  \quad \Gamma; \emptyset; \emptyset \proves a \hastype \chtype{S_2} \quad S_1 \dualof S_2
		}{
			\Gamma; \Lambda; \Sigma \cat k: S_1  \proves \bout{a}{k} P \hastype \Proc
		}

		\\[4mm]

		\trule{NewSh}~~\tree{
			\Gamma\cat a:\chtype{S} ; \Lambda; \Sigma \proves P \hastype \Proc
		}{
			\Gamma; \Lambda; \Sigma \proves \news{a} P \hastype \Proc}
		\qquad\quad
		\trule{NewSes}~~\tree{
			\Gamma; \Lambda; \Sigma \cat s:S_1 \cat \dual{s}: S_2 \proves P \hastype \Proc \quad S_1 \dualof S_2
		}{
			\Gamma; \Lambda; \Sigma \proves \news{s} P \hastype \Proc
		}
		\\[4mm]

		\trule{RecvS}~~\tree{
			\Gamma; \Lambda; \Sigma \cat k: S_1 \cat x: S_2 \proves P \hastype \Proc
		}{
			\Gamma; \Lambda; \Sigma, k: \btinp{S_2} S_1  \vdash \binp{k}{x}P \hastype \Proc
		}
		\quad\quad 
		\trule{RecvL}~~\tree{
			\Gamma; \Lambda \cat X: \lhot{S}; \Sigma \cat k: S_1  \proves P \hastype \Proc
		}{
			\Gamma; \Lambda; \Sigma \cat k:\btinp{\lhot{S}}S_1  \proves \binp{k}{X}P \hastype \Proc
		}
		\\[4mm]
		\trule{RecvShN}~~\tree{
			\Gamma \cat x: \chtype{S}; \Lambda; \Sigma \cat k: S_1  \proves P \hastype \Proc
		}{
			\Gamma; \Lambda; \Sigma \cat k:\btinp{\chtype{S}}S_1  \proves \binp{k}{x}P \hastype \Proc
		}
		
		\quad ~~
		\trule{RecvSh}~~\tree{
			\Gamma \cat X: \shot{S}; \Lambda; \Sigma \cat k: S_1  \proves P \hastype \Proc
		}{
			\Gamma; \Lambda; \Sigma \cat k:\btinp{\shot{S}}S_1  \proves \binp{k}{X}P \hastype \Proc
		}
		\\[4mm]
		\trule{Par}~~\tree{
			\Gamma; \Lambda_{1}; \Sigma_{1} \proves P_{1} \hastype \Proc \quad \Gamma; \Lambda_{2}; \Sigma_{2} \proves P_{2} \hastype \Proc
		}{
			\Gamma; \Lambda_{1} \cup \Lambda_2; \Sigma_{1} \cup \Sigma_2 \proves P_1 \Par P_2 \hastype \Proc
		}
		\qquad\quad
		\trule{Close}~~\tree{
			\Gamma; \Lambda; \Sigma  \proves P \hastype T \quad k \not\in \dom{\Gamma, \Lambda,\Sigma}
		}{
			\Gamma; \Lambda; \Sigma \cat k: \tinact  \proves P \hastype \Proc
		}
		\\[4mm]
		\trule{Bra}~~\tree{
			 \forall i \in I \quad \Gamma; \Lambda; \Sigma \cat k:S_i \proves P_i \hastype \Proc
		}{
			\Gamma; \Lambda; \Sigma \cat k: \btbra{l_i:S_i}_{i \in I} \proves \bbra{k}{l_i:P_i}_{i \in I}\hastype \Proc
		}
		\qquad\quad 
	 	\trule{Sel}~~\tree{
			\Gamma; \Lambda; \Sigma \cat k: S_j  \proves P \hastype \Proc \quad j \in I
		}{
			\Gamma; \Lambda; \Sigma \cat k:\btsel{l_i:S_i}_{i \in I} \proves \bsel{s}{l_j} P \hastype \Proc
		}
		\\[4mm]

		\trule{Nil}~~\Gamma; \emptyset; \emptyset \proves \inact \hastype \Proc
\qquad \quad
		\trule{Var}~~\tree{
	
		}{
			\Gamma \cat \rvar{X}: \Sigma; \emptyset; \emptyset  \proves \rvar{X} \hastype \Proc
		}
		\qquad\quad 
%	 	\trule{Rec}~~\tree{
%			\Gamma \cat \rvar{X}: \Sigma; \emptyset; \emptyset  \proves P \hastype \Proc
%		}{
%			\Gamma ; \emptyset; \emptyset  \proves \recp{X}{P} \hastype \Proc
%		}
%		\\[4mm]

	 	\trule{Rec}~~\tree{
			\Gamma \cat \rvar{X}: \Sigma; \emptyset; \Sigma  \proves P \hastype \Proc
		}{
			\Gamma ; \emptyset; \Sigma  \proves \recp{X}{P} \hastype \Proc
		}


	\end{array}
\]
\caption{Typing Rules for $\HOp$\label{fig:typerulesmy}}
\end{figure}


The type relation is defined in \figref{fig:typerulesmy}.
%Types for session names/variables $u$ and
%directly derived from the linear part of the typing
%environment, i.e.~type maps $\Delta$ and $\Lambda$.
Rules $\trule{Session}$ and $\trule{LVar}$ require,
respectively, the minimal:
i) session environment $\Delta$ to type session 
$u$ with type $S$; and,
ii) linear environment $\Lambda$ to type 
higher-order variable $x$ with type $\shot{C}$.
Rule $\trule{Shared}$
assigns the value type $U$
to shared names or shared variables $u$ 
if the map $u:U$ exists in environment
$\Gamma$. Rule $\trule{Shared}$ also requires 
that the linear environment is
be minimal, i.e.~empty.
The type $\shot{C}$ for shared higher-order values $V$
is derived using rule $\trule{Prom}$, where we require
a value with linear type to be typed without a linear
environment present in order to be used as a shared type.
Rule $\trule{EProm}$ allows to freely use a linear
type variable as shared type variable. 
%A value consisting of a tuple of names/variables is typed using the $\trule{Pol}$ rule,
%which requires theto type and combine each value in the tuple.
Abstraction values are typed with rule $\trule{Abs}$.
The key type for an abstraction is the type for
the bound variables of the abstraction, i.e.~for
bound variable with type $C$ the abstraction
has type $\lhot{C}$.
The dual of abstraction typing is application typing
governed by rule $\trule{App}$, where we expect
the type $C$ of an application name $u$ 
to match the type $\lhot{C}$ or $\shot{C}$
of the application variable $x$.

A process prefixed with a session send operator $\bout{u}{V} P$
is typed using rule $\trule{Send}$.
The type $U$ of a send value $V$ should appear as a prefix
on the session type $\btout{U} S$ of $s$.
Rule $\trule{Rcv}$
defines the typing for the 
reception of values $\binp{u}{V} P$.
The type $U$ of a receive value should 
appear as a prefix on the session type $\btinp{U} S$ of $u$.
We use a similar approach with session prefixes
to type interaction between shared channels as defined 
in rules $\trule{Req}$ and $\trule{Acc}$,
where the type of the sent/received object 
($S$ and $L$, respectively) should
match the type of the sent/received subject
($\chtype{S}$ and $\chtype{L}$, respectively).
%In the case of rule $\trule{Req}$ we require
%a duality condition for the communication of session names.
Select and branch prefixes are typed using the rules
$\trule{Sel}$ and $\trule{Bra}$ respectively. Both
rules prefix the session type with the selection
type $\btsel{l_i: S_i}_{i \in I}$ and
$\btbra{l_i:S_i}_{i \in I}$.

The creation of a
shared name $a$ requires to add
its type in environment $\Gamma$ as defined in 
rule \trule{Res}. 
Creation of a session name $s$
creates two endpoints with dual types and adds them to
the session environment 
$\Delta$ as defined in rule \trule{ResS}. 
Rule \trule{Par} concatanates the linear environment of
the parallel components of a parallel operator
to create a type for the entire process.
The disjointness of environments $\Lambda$ and $\Delta$
is implied. Rule \trule{End} allows a form of weakening 
for the session environment $\Delta$, provided that
the name added in $\Delta$ has the inactive
type $\tinact$. The inactive process $\inact$ has no
linear environment. The recursive variable is typed
directly from the shared environment $\Gamma$ as
in rule \trule{RVar}.
The recursive operator requires that the body of
a recursive process matches the type of the recursive
variable as in rule \trule{Rec}.

\begin{comment}
\subsection{Order of Types}

In~\cite{tlca07} the type syntax for values includes the definition
$U_1 \sharedop U_2$ and $U_1 \lollipop U_2$, that
allows us to define types of arbitrary order $k$.
An abstraction of $k$-order types requires to extend the syntax
to include higher-order applications:
\[
	\abs{z}{\binp{z}{x} \appl{x}{\abs{y} Q}}
\]
with with the type of $\abs{y}{Q}$ being of order
$k-1$. The type of of such an abstraction in the current setting would
be $\shot{U}$ (or $\lhot{U}$) with the order of the type being defined
as the number of nested higher-order types~\cite{San96int}.

In the type system we develop for the \HOp we only have
types of the form $\shot{C}$.
If we maintain the definition of counting the order
of the type as the nesting of higher-order types we
can still express $k$-order types, e.g:
\[
	\shot{(\btinp{U} \tinact)}
\]
with $U$ being of order $k-1$.
An $k$-order abstraction in \HOp would be:
\[
	\abs{z}{\binp{z}{x} \binp{x}{y} \appl{y}{n}}
\]
with $y$ being of order $k-1$.

\begin{definition}[Order of Value Type]\rm
	\label{def:order_type}
	Let type $U$ and value $V$ such that $\Gamma; \Lambda; \Delta \proves V \hastype U$.
	The order of $U$ is the number of using rule $\trule{Abs}$
	in the typing derivation $\Gamma; \Lambda; \Delta \proves V \hastype U$.
\end{definition}
\end{comment}

\subsection{Type Soundness}

%We state results for type safety:
Type safety result are instances of more general
statements already proved by
Mostrous and Yoshida~\cite{tlca07} in the asynchronous case.
%
\begin{lemma}[Substitution Lemma - Lemma C.10 in M\&Y]\rm
	\label{lem:subst}
	\begin{enumerate}[1.]
		\item	$\Gamma; \Lambda; \Delta \cat x:S  \proves P \hastype \Proc$ and
			$u \not\in \dom{\Gamma, \Lambda, \Delta}$
			implies
			$\Gamma; \Lambda; \Delta \cat u:S  \proves P\subst{u}{x} \hastype \Proc$.

		\item	$\Gamma \cat x:\chtype{U}; \Lambda; \Delta \proves P \hastype \Proc$ and
			$a \notin \dom{\Gamma, \Lambda, \Delta}$
			implies
			$\Gamma \cat a:\chtype{U}; \Lambda; \Delta \proves P\subst{a}{x} \hastype \Proc$.

		\item	If $\Gamma; \Lambda_1 \cat x:\lhot{C}; \Delta_1  \proves P \hastype \Proc$ 
			and $\Gamma; \Lambda_2; \Delta_2  \proves V \hastype \lhot{C}$ with 
			$\Lambda_1 \cat \Lambda_2$ and $\Delta_1 \cat \Delta_2$ defined,
			then $\Gamma; \Lambda_1 \cat \Lambda_2; \Delta_1 \cat \Delta_2  \proves P\subst{V}{x} \hastype \Proc$.

		\item	$\Gamma \cat x:\shot{C}; \Lambda; \Delta  \proves P \hastype \Proc$ and
			$\Gamma; \emptyset ; \emptyset \proves V \hastype \shot{C}$
			implies
			$\Gamma; \Lambda; \Delta \proves P\subst{V}{x} \hastype \Proc$.
		\end{enumerate}
\end{lemma}
%
\begin{proof}
	By induction on the typing for $P$, with a case analysis on the last used rule. 
	\qed
\end{proof}

We are interested in session environments that whenever they contain dual endpoints
their types are dual:
%
\begin{definition}[Balanced Session Environment]\label{d:wtenv}\rm
	We say that session environment $\Delta$ is {\em balanced} if
	$s: S_1, \dual{s}: S_2 \in \Delta$ implies $S_1 \dualof S_2$.
\end{definition}
%
The type soundness relies on the following auxiliary definition:
%
\begin{definition}[Session Environment Reduction]\rm
	\label{def:ses_red}
	The reduction relation $\red$ on session environments is defined as:
%
\[
	\begin{array}{rcl}
		\Delta \cat s: \btout{U} S_1 \cat \dual{s}: \btinp{U} S_2 &\red& \Delta \cat s: S_1 \cat \dual{s}: S_2
		\\
		\Delta \cat s: \btsel{l_i: S_i}_{i \in I} \cat \dual{s}: \btbra{l_i: S_i'}_{i \in I} &\red& \Delta \cat s: S_k \cat \dual{s}: S_k', \quad k \in I
	\end{array}
\]
%
	We write $\red^\ast$ for the multistep environment reduction.
\end{definition}
%
We now state the main soundness result as an instance
of type soundness from the system in~\cite{tlca07}.
It is worth noticing that in~\cite{tlca07} has a slightly richer
definition of structural congruence.
Also, their statement for subject reduction relies on an
ordering on typing associated to queues and other 
runtime elements. %(such extended typing is denoted as $\Delta$ by M\&Y).
Since we are dealing with synchronous semantics we can omit such an ordering.
The type soundness result implies soundness for the sub-calculi
\HO, \sessp, and $\CAL^{\minussh}$

\begin{theorem}[Type Soundness - Theorem 7.3 in M\&Y]\rm
	\label{thm:sr}
%
	\begin{enumerate}[1.]
		\item	(Subject Congruence)
			$\Gamma; \es; \Delta \proves P \hastype \Proc$
			and
			$P \scong P'$
			implies
			$\Gamma; \es; \Delta \proves P' \hastype \Proc$.

		\item	(Subject Reduction)
			$\Gamma; \es; \Delta \proves P \hastype \Proc$
			with
			balanced $\Delta$
			and
			$P \red P'$
			implies $\Gamma; \es; \Delta'  \proves P' \hastype \Proc$
			and either (i)~$\Delta = \Delta'$ or (ii)~$\Delta \red \Delta'$
			with $\Delta'$ balanced.
	\end{enumerate}
\end{theorem}

\begin{proof}
	See \appref{app:ts}.
	\qed
\end{proof}


\section{Characteristic Session Bisimulation}
\label{sec:behavioural}
\section{Behavioral Semantics}

In this section we define a theory for observational equivalence over
session typed $\HOp$ processes. The theory follows the principles
laid by the previous work of the authors
\cite{DBLP:conf/forte/KouzapasYH11,KY13,dkphdthesis}.
We require a bisimulation relation over typed processes that
is also characterised by the corresponding typed, reduction-closed,
barbed congruence relation.

\dk{(Jorge, I think you have a paper we can cite over session typed bisimulations)}

\subsection{Labelled Transition Semantics}

We define a relation $(P_1, \lambda, P_2) \in R$ over
(untyped) processes, that allows us to follow how a process may
interact with a process in its enviroment. The interaction
is defined on action $\lambda$:

\begin{tabular}{rcl}
		$\lambda$ &$\bnfis$& $\tau \bnfbar \bactout{s}{s'} \bnfbar \bactout{s}{\abs{x} P} \bnfbar\bactinp{s}{s} \bnfbar \bactinp{s}{\abs{x} P}$ \\
		&	$\bnfbar$ & $\bactsel{s}{l} \bnfbar \bactbra{s}{l} \bnfbar \news{\tilde{s}} \bactout{s}{s'} \bnfbar \news{\tilde{s}} \bactout{s}{\abs{x} P}$\\
		&$\bnfbar$ &	\dk{$\bactout{s}{\tilde{s}} \bnfbar \bactout{s}{\abs{\tilde{x}} P} \bactinp{s}{\tilde{s'}} \bnfbar \bactinp{s}{\abs{\tilde{x}} P}$}
\end{tabular}

The internal action is defined on label $\tau$.
Action $\bactout{s}{s'}$ denotes the sending of name $s'$ over channel $s$.
Similarly action $\bactout{s}{\abs{x}{P}}$ is the sending of abstraction $\abs{x}{P}$
over channel $s$. Dually actions for the reception of names and abstractions are
$\bactinp{s}{s'}$ and $\bactinp{s}{\abs{x}{P}}$ respectively. We also defined
actions for selecting a label $l$, $\bactsel{s}{l}$ and branching on a label
$l$, $\bactbra{s}{l}$. When output actions carry name restrictions a scope
opening is implied.

We define the notion of dual actions as the symmetric relation $\asymp$, that satisfies the rules:
\[
	\bactsel{s}{l} \asymp \bactbra{\dual{s}}{l} \qquad \news{\tilde{s}} \bactout{s}{\abs{x} P} \asymp \bactinp{\dual{s}}{\abs{x} P} \qquad \bactout{s}{s'} \asymp \bactinp{\dual{s}}{s'}
\]

Dual actions happen on subjects that are dual between them; they carry the same
object; and furthermore output action is dual with input action and 
select action is dual with branch action.

\paragraph{Untyped Labelled Transition System}

\begin{figure}
	\[
	\begin{array}{c}
		\bout{n}{\tilde{m}} P \by{\bactout{n}{\tilde{m}}} P\ \ltsrule{OutN}
		\qquad
		\binp{n}{\tilde{x}} P \by{\bactinp{n}{\tilde{m}}} P\subst{\tilde{m}}{\tilde{x}}\ \ltsrule{InN}
		\qquad
		\bout{n}{\abs{x}{Q}} P \by{\bactout{n}{\abs{x}{Q}}} P\ \ltsrule{OutA}
		\\[4mm]

		\binp{n}{X} P \by{\bactinp{n}{\abs{x}{Q}}} P\subst{\abs{x}Q}{X}\ \ltsrule{InA}
		\qquad
		\bsel{n}{l}{P} \by{\bactsel{n}{l}} P \ltsrule{Sel}
		\qquad
		\tree{
			j \in I
		}
		{
			\bbra{n}{l_i:P_i}_{i \in I} \by{\bactbra{n}{l_j}} P_j
		}\ \ltsrule{Bra}
		\\[6mm]

		\tree{
			P \by{\ell} P' \quad n \notin \fn{\ell}
		}{
			\news{n} P \by{\ell} \news{n} P' 
		}\ \ltsrule{Res}
		\qquad
		\tree{
			P \scong_\alpha P'' \quad P'' \by{\ell} P'
		}{
			P \by{\ell} P'
		}\ \ltsrule{Alpha}
		\qquad
		\tree{
			P \by{\news{\tilde{s}} \bactout{n}{\tilde{m}}} P' \quad s \in \tilde{m}
		}{
			\news{s} P \by{\news{s\cat\tilde{s}} \bactout{n}{\tilde{m}}} P'
		}\ \ltsrule{ScopeN}
		\\[6mm]

		\tree{
			P \by{\news{\tilde{m}} \bactout{n}{\abs{x} Q}} P' \quad s \in \fn{\abs{x} Q}
		}{
			\news{s} P \by{\news{s\cat\tilde{m}} \bactout{n}{\abs{x} Q}} P'
		}\ \ltsrule{ScopeA}
		\qquad
		\tree{
			P \by{\ell_1} P' \qquad Q \by{\ell_2} Q' \qquad \ell_1 \asymp \ell_2
		}{
			P \Par Q \by{\tau} \newsp{\bn{\ell_1} \cup \bn{\ell_2}}{P' \Par Q'}
		}\ \ltsrule{Tau}
		\\[6mm]

		\tree{

			P \by{\ell} P' \quad \bn{\ell} \cap \fn{Q} = \es
		}{
			P \Par Q \by{\ell} P' \Par Q
		}\ \ltsrule{LPar}
		\qquad
		\tree{
			Q \by{\ell} Q' \quad \bn{\ell} \cap \fn{P} = \es
		}{
			P \Par Q \by{\ell} P \Par Q'
		}\ \ltsrule{RPar}
%		\\[6mm]
	\end{array}
	\]
	\caption{The untyped Labelled Transition System \label{fig:untyped_LTS}}
\end{figure}


The labelled transition system, LTS, is defined in Figure~\ref{fig:untyped_LTS}.
A process with a send prefix can interact with the environment with a send
action that carries a name $s'$ or an abstraction $\abs{x}{Q}$. Dually
on a received prefixed process we can observe a receive action of a name or
an abstraction. Select and branch prefixed processes can trigger select
and branch actions respectively. The LTS is closed under the name creation
operator provided that the restricted name does not occur free in the LTS action.
If the restricted name occurs free in the LTS action then we observe a bound name
action and the continuation process performs scope opening. Similarly the LTS 
is closed on the parallel operator provided that the LTS action does not shared
any bound names with parallel processes. If two parallale processes can perform
dual actions then the two actions can synchronise to observe an internal transition.
Finally the LTS is closed under structural congruence.


\paragraph{Labeled Transition System for Typed Environments}

We define a relation
$((\Gamma, \Lambda_1, \Sigma_1), \lambda, (\Gamma, \Lambda_2, \Sigma_2)) \in R$
over type tuples, that allows us to follow the progress of types over actions $\lambda$.

\begin{figure}
	\[
	\begin{array}{c}
		\tree{
			\dual{s} \notin \dom{\Sigma} \quad \Gamma; \Lambda_1; \Sigma_1 \proves V \hastype U \quad \Sigma_1 \subseteq \Sigma \quad \Lambda_1 \subseteq \Lambda
		}{
			(\Gamma; \Lambda; \Sigma \cat s: \btout{U} S) \by{\bactout{s}{V}} (\Gamma; \Lambda\backslash\Lambda_1; \Sigma\backslash\Sigma_1 \cat s: S)
		}
		\\[6mm]
		\tree{
			\dual{s} \notin \dom{\Sigma} \quad  \Gamma; \Lambda_1; \Sigma_1 \proves V \hastype U
		}{
			(\Gamma; \Lambda; \Sigma \cat s: \btinp{U} S) \by{\bactinp{s}{U}} (\Gamma; \Lambda \cup \Lambda_1; \Sigma \cup \Sigma_1 \cat s: S)
		}
		\\[6mm]
		\tree{
			\dual{s} \notin \dom{\Sigma} \quad k \in I
		}{
			(\Gamma; \Lambda; \Sigma \cat s: \btsel{l_i: S_i}_{i \in I}) \by{\bactsel{s}{l_k}} (\Gamma; \Lambda; \Sigma \cat s:S_k)
		}
		\quad
		\tree{
			\dual{s} \notin \dom{\Sigma} \quad k \in I
		}{
			(\Gamma; \Lambda; \Sigma \cat s: \btbra{l_i: T_i}_{i \in I}) \by{\bactbra{s}{l_k}} (\Gamma; \Lambda; \Sigma \cat s:S_k)
		}
		\\[6mm]

		\tree{
			(\Gamma; \Lambda_1; \Sigma_1) \by{\news{\tilde{s}} \bactout{s}{V}} (\Gamma; \Lambda_2; \Sigma_2)
	%		\quad S_1 \dualof S_2
		}{
			(\Gamma; \Lambda_1; \Sigma_1) \by{\news{s' \cat \tilde{s'}} \bactout{s}{V}} (\Gamma; \Lambda_2; \Sigma_2 \cat \dual{s'}: S)
		}
		\quad
		\tree{
			\Sigma_1 \red \Sigma_2
		}{
			(\Gamma; \Lambda; \Sigma_1) \by{\tau} (\Gamma; \Lambda; \Sigma_2)
		}
	\end{array}
	\]
	\caption{Labelled Transition Semantics for Typed Enviroments \label{fig:envLTS}}
\end{figure}


\dk{describe env LTS}

\paragraph{Typed Transition System}

The transition system over processes is defined as a combination
of the untyped LTS and the LTS for typed environments:

\begin{definition}[Typed Transition System]\rm
	We write $\Gamma; \emptyset; \Sigma_1 \proves P_1 \hastype \Proc \by{\lambda} \Gamma; \emptyset; \Sigma_1 \proves P_2 \hastype \Proc$
	whenever:
	\begin{itemize}
		\item	$P_1 \by{\lambda} P_2$.
		\item	$(\Gamma, \emptyset, \Sigma_1) \by{\lambda} (\Gamma, \emptyset, \Sigma_2)$.
	\end{itemize}
\end{definition}

For notational convenience we can write
$\Gamma; \emptyset; \Sigma_1 \by{\lambda} \Sigma_2 \proves P_1 \by{\lambda} P_2$,
instead of $\Gamma; \emptyset; \Sigma_1 \proves P_1 \hastype \Proc \by{\lambda} \Gamma; \emptyset; \Sigma_1 \proves P_2 \hastype \Proc$.
We extend to $\By{}$ and $\By{\hat{\lambda}}$ in the \dk{standard way}.

The next invariant clarifies the soundness of the
typed transition system.

\begin{lemma}[Invariant]
	\begin{itemize}
		\item	If $\Gamma; \emptyset; \Sigma_1 \proves P_1 \hastype \Proc$ and
			$P_1 \by{\lambda} P_2$ and $(\Gamma; \emptyset; \Sigma_1) \by{\lambda} (\Gamma; \emptyset; \Sigma_2)$
			then $\Gamma; \emptyset; \Sigma_2 \proves P_2 \hastype \Proc$.
	\end{itemize}
\end{lemma}

\begin{proof}
	\dk{TODO}
\end{proof}

\subsection{Behavioural Semantics}

We use the typed labelled transition semantics to define
a set of relations over typed processes that allow us to compare
typed processes over a notion of observational equivalence.


We begin with a definition of a notion of confluence
over session environments $\Sigma$:
\begin{definition}[Session Environment Confluence]\rm
	We denote $\Sigma_1 \bistyp \Sigma_2$ whenever $\exists \Sigma$ such that
	$\Sigma_1 \red^* \Sigma$ and $\Sigma_2 \red^* \Sigma$.
\end{definition}

%\jp{The following definition is a bit too "loose". Need to add conditions on $\Sigma_1,\Sigma_2$, and a better notation not involving the empty $\Lambda$.}

A typed relation is a relation over typed programs:

\begin{definition}[Typed Relation]\rm
	We say that
	$\Gamma; \emptyset; \Sigma_1 \proves P_1 \hastype \Proc\ R\ \Gamma; \emptyset; \Sigma_2 \proves P_2 \hastype \Proc$
	is a typed relation whenever
	\begin{itemize}
		\item	$P_1$ and $P_2$ are programs.
		\item	$\Sigma_1$ and $\Sigma_2$ are well typed.
		\item	$\Sigma_1 \bistyp \Sigma_2$.
	\end{itemize}
\end{definition}

We require that relate only programs (i.e.\ processes with no free variables) with
well typed session environments and furthermore we require that two related processes
have confluent session environments.

For notational convenience we write $\Gamma; \emptyset; \Sigma_1\ R\ \Sigma_2 \proves P_1\ R\ P_2$
for expressing the typed relation $\Gamma; \emptyset; \Sigma_1 \proves P_1 \hastype \Proc\ R\ \Gamma; \emptyset; \Sigma_2 \proves P_2 \hastype \Proc$.

We define the notions of barb and typed barb.

\begin{definition}[Barbs]\rm
	Let program $P$.
	\begin{enumerate}
%		\item	We write $P \barb{s}$ if $P \scong \newsp{\tilde{s}}{\bout{s}{\abs{x} P_1} P_2 \Par P_3}, s \notin \tilde{s}$.
%			We write $P \Barb{s}$ if $P \red^* \barb{s}$.

		\item	We write $P \barb{s}$ if $P \scong \newsp{\tilde{s}}{\bout{s}{V} P_2 \Par P_3}, s \notin \tilde{s}$.
			We write $P \Barb{s}$ if $P \red^* \barb{s}$.

		\item	We write $\Gamma; \emptyset; \Sigma \proves P \barb{s}$ if
			$\Gamma; \emptyset; \Sigma \proves P \hastype \Proc$ with $P \barb{s}$ and $\dual{s} \notin \Sigma$.
			We write $\Gamma; \emptyset; \Sigma \proves P \Barb{s}$ if $P \red^* P'$ and
			$\Gamma; \emptyset; \Sigma' \proves P' \barb{s}$.			
	\end{enumerate}
\end{definition}

A barb $\barb{s}$ is an observable on an output prefix with subject $s$.
Similarly a weak barb $\Barb{s}$ is a barb after a number of reduction steps.
Typed barbs $\barb{s}$ (resp.\ $\Barb{s}$)
happen on typed processes $\Gamma; \emptyset; \Sigma \proves P \hastype \Proc$
where we require that the corresponding dual endpoint $\dual{s}$ is not present
in the session type $\Sigma$.

We define the notion of the context:

\begin{definition}[Context]\rm
	$C$ is a context defined on the grammar:

	\begin{tabular}{rcl}
		$C$ &$=$& $\hole \bnfbar P \bnfbar \bout{k}{V} C \bnfbar \binp{k}{X} C \bnfbar \binp{k}{x} C \bnfbar \news{s} C \bnfbar C \Par C \bnfbar \bsel{k}{l} C \bnfbar \bbra{k}{l_i:C_i}_{i \in I}$
	\end{tabular}
	Notation $\context{C}{P}$ replaces every $\hole$ in $C$ with $P$.
\end{definition}

A context is a function that takes a process and returns a new process
according to the above syntax.

%We extend the notion of context to the notion of typed context:
%\begin{definition}[Typed Context]
%	Let program $\Gamma; \emptyset; \Sigma \proves P \hastype \Proc$ then	
%\end{definition}

The first equivalence relation we define is reduction-closed, barbed congruence:
\begin{definition}[Reduction-closed, Barbed Congruence]\rm
	Typed relation $\Gamma; \emptyset; \Sigma_1\ R\ \Sigma \proves P_1 \ R\ P_2$ is a barbed congruence
	whenever:
	\begin{enumerate}
		\item
		\begin{itemize}
			\item	If $P_1 \red P_1'$ then $\exists P_2', P_2 \red^* P_1'$ and $\Gamma; \emptyset; \Sigma_1' \proves P_1'\ R\ \Gamma; \emptyset; \Sigma_2' \proves P_2' \hastype \Proc$.
			\item	If $P_2 \red P_2'$ then $\exists P_1', P_1 \red^* P_1'$ and $\Gamma; \emptyset; \Sigma_1' \proves P_1'\ R\ \Gamma; \emptyset; \Sigma_2' \proves P_2' \hastype \Proc$.
		\end{itemize}
		\item
		\begin{itemize}
			\item	If $\Gamma;\emptyset;\Sigma \proves P_1 \barb{s}$ then $\Gamma;\emptyset;\Sigma \proves P_2 \Barb{s}$.
			\item	If $\Gamma;\emptyset;\Sigma \proves P_2 \barb{s}$ then $\Gamma;\emptyset;\Sigma \proves P_1 \Barb{s}$.
		\end{itemize}
		\item	$\forall C, \Gamma; \emptyset; \Sigma_1'\ R\ \Sigma_2' \proves \context{C}{P_1}\ R\ \context{C}{P_2}$.
	\end{enumerate}
	The largest such congruence is denoted with $\cong$.
\end{definition}

Reduction-closed, barbed congruence has closed reduction semantics and 
preserves barbs under any context. In a sense no barb observer can distinguish
between two related processes.

We can use a session type to define the simplest process that is typed
under the given session type.
\begin{definition}[Simple Process]
	Let session type $S$ then we define a process $\map{S}^{x}$:
	\[
	\begin{array}{l}
		\map{\tinact}^{x} = \inact \qquad \map{\btinp{S'} S}^{x} = \binp{x}{y} (\map{S}^{x} \Par \map{S'}^{y}) \qquad
		\map{\btout{U} S}^{x} = \binp{x}{\dk{\map{U}}} \map{S}^{x}\\
		\map{\btsel{l : S}}^{x} = \bsel{x}{l} \map{S}^{x} \qquad \map{\btbra{l_i: S_i}_{i \in I}}^{x} = \bbra{x}{l_i: \map{S_i}^{x}}_{i \in I}\\
		\map{\tvar{t}}^{x} = \rvar{T} \qquad \map{\trec{t}{S}}^{x} = \recp{T}{\map{S}^{x}}
	\end{array}
	\] 
\end{definition}

The second equivalent relation is a bisimulation relation called
contextual bisimulation:
\begin{definition}[Contextual Bisimulation]\rm
	Let typed relation $\mathcal{R}$ such that $\Gamma; \emptyset; \Sigma_1\ \mathcal{R}\ \Sigma_2 \proves P_1\ \mathcal{R}\ Q_1$.
	$\mathcal{R}$ is a {\em contextual bisimulation} whenever:
	\begin{enumerate}
		\item	$\forall \news{\tilde{s}} \bactout{s}{\abs{x} P}$ such that
			\[
				\Gamma; \emptyset; \Sigma_1 \by{\news{\tilde{s}} \bactout{s}{\abs{x} P}} \Sigma_1' \proves P_1 \by{\news{\tilde{s}} \bactout{s}{\abs{x} P}} P_2
			\]
			$\exists Q_2, \abs{x}{Q}$ such that
			\[
				\Gamma; \emptyset; \Sigma_2 \By{\news{\tilde{s'}} \bactout{s}{\abs{x} Q}} \Sigma_2' \proves Q_1 \By{\news{\tilde{s'}} \bactout{s}{\abs{x} Q}} Q_2
			\]
			and $\forall C, s'$, %such that
%			\begin{eqnarray*}
%				\Gamma; \emptyset; \Sigma_1'' \proves \newsp{\tilde{s}}{P_2 \Par \context{C}{P \subst{s'}{x}}} \hastype \Proc \\
%				\Gamma; \emptyset; \Sigma_2'' \proves \newsp{\tilde{s}}{Q_2 \Par \context{C}{Q \subst{s'}{x}}} \hastype \Proc
%			\end{eqnarray*}
			then
			\[
				\Gamma; \emptyset; \Sigma_1''\ \mathcal{R}\ \Sigma_2'' \proves \newsp{\tilde{s}}{P_2 \Par \context{C}{P \subst{s'}{x}}}\ \mathcal{R}\ 
				\newsp{\tilde{s'}}{Q_2 \Par \context{C}{Q \subst{s'}{x}}}
			\]
		\item	$\forall \news{\tilde{s}} \bactout{s}{s_1}$ such that
			\[
				\Gamma; \emptyset; \Sigma_1 \by{\news{\tilde{s}} \bactout{s}{s_1}} \Sigma_1' \proves P_1 \by{\news{\tilde{s}} \bactout{s}{s_1}} P_2
			\]
			$\exists Q_2, s_2$ such that
			\[
				\Gamma; \emptyset; \Sigma_2 \By{\news{\tilde{s'}} \bactout{s}{s_2}} \Sigma_2' \proves Q_1 \By{\news{\tilde{s'}} \bactout{s}{s_2}} Q_2
			\]
			and $\forall R$ with $\set{x} = \fn{R}$, %such that
%			\begin{eqnarray*}
%				\Gamma; \emptyset; \Sigma_1'' \proves \newsp{\tilde{s}}{P_2 \Par \context{C}{P \subst{s'}{x}}} \hastype \Proc \\
%				\Gamma; \emptyset; \Sigma_2'' \proves \newsp{\tilde{s}}{Q_2 \Par \context{C}{Q \subst{s'}{x}}} \hastype \Proc
%			\end{eqnarray*}
			then
			\[
				\Gamma; \emptyset; \Sigma_1''\ \mathcal{R}\ \Sigma_2'' \proves \newsp{\tilde{s}}{P_2 \Par R \subst{s_1}{x}}\ \mathcal{R}\ 
				\newsp{\tilde{s'}}{Q_2 \Par R \subst{s_2}{x}}
			\]

		\item	$\forall \lambda \notin \set{\news{\tilde{s}} \bactout{s}{s'}, \news{\tilde{s}} \bactout{s}{\abs{x} P}}s$ such that
			\[
				\Gamma; \emptyset; \Sigma_1 \by{\lambda} \Sigma_1' \proves P_1 \by{\lambda} P_2
			\]
			$\exists Q_2$ such that 
			\[
				\Gamma; \emptyset; \Sigma_2 \by{\lambda} \Sigma_2' \proves Q_1 \By{\hat{\lambda}} Q_2
			\]
			and
			$\Gamma; \emptyset; \Sigma_1\ \mathcal{R}\ \Sigma_2 \proves P_2\ \mathcal{R}\ Q_2$.

		\item	The symmetric cases of 1, 2 and 3.
	\end{enumerate}
	The Knaster Tarski theorem ensures that the largest contextual bisimulation exists and is denoted by $\wb^c$.
\end{definition}






\section{Related Work}
\label{sec:relwork}
% !TEX root = main.tex

 Since types can limit
contexts (environments) where processes can interact, typed equivalences
usually offer {\em coarser} semantics than untyped equivalences.
Pierce and Sangiorgi~\cite{PiSa96b} demonstrated that IO-subtyping \newc{can justify
the optimal encoding of the $\lambda$-calculus by Milner---this was not possible
in the untyped polyadic $\pi$-calculus~\cite{MilnerR:funp}.}
After~\cite{PiSa96b}, many works on typed $\pi$-calculi 
have investigated correctness of encodings of known concurrent and
sequential calculi in order to examine semantic
effects of proposed typing systems. 

A  type discipline closely related
to session types is a family of linear typing systems. Kobayashi, Pierce, and Turner~\cite{LinearPi} first proposed a linearly typed reduction-closed, barbed congruence and 
used to 
reason about a tail-call optimisation of higher-order functions 
encoded 
as processes. 
Yoshida~\cite{Yoshida96} 
used a bisimulation of graph-based types to prove the full abstraction
of encodings of the polyadic synchronous $\pi$-calculus into the
monadic synchronous $\pi$-calculus. 
Later typed equivalences of a
family of linear and affine calculi \cite{BHY,DBLP:journals/iandc/YoshidaBH04,BergerHY05} 
were used to encode 
PCF \cite{Plotkin1977223,Milner19771}, the simply typed $\lambda$-calculus with sums and products, and System F \cite{GirardJY:protyp}
fully abstractly (a fully abstract encoding of the $\lambda$-calculi 
was an open problem in \cite{MilnerR:funp}).  
Yoshida, Honda, and Berger~\cite{YHB02} proposed a new bisimilarity
method associated with linear type structure and strong
normalisation. 
It presented applications to reason about secrecy in programming languages. 
A subsequent work~\cite{HY02} adapted these results
to a practical direction, proposing new typing
systems for secure higher-order and multi-threaded programming 
languages. 
In these works, typed properties, linearity and liveness, 
play a fundamental role in the analysis. In general, linear types 
are suitable to encode ``sequentiality'' in the sense of 
\cite{HylandJME:fulapi,AbramskyS:fulap}.

 
 {Our work follows 
the 
%principles for
%session type 
behavioural semantics in 
\cite{KYHH2015,KY2015,DBLP:journals/iandc/PerezCPT14}
where a bisimulation is defined on an LTS 
that assumes a session typed
observer.
%The bisimilarity is characterised by the corresponding
%reduction-closed, barbed congruence using techniques derived from~\cite{Hennessy07}.
Our theory for higher-order sessions 
differentiates from 
the work in~\cite{KYHH2015} and \cite{KY2015}, which 
considers  (first-order)
binary and multiparty session types, respectively.
P\'{e}rez et al~\cite{DBLP:journals/iandc/PerezCPT14} studied typed equivalences
for a 
theory of binary sessions based on linear logic,
without shared names.}
%Determinacy properties (confluence, $\tau$-inertness) are proven.



%The theory for higher-order session type quivalences is more challenging than
%their corresponding first-order bisimulation theory.
Our approach %for the higher-order 
to typed equivalences
builds upon techniques developed by Sangiorgi~\cite{SangiorgiD:expmpa,San96H}
and Jeffrey and Rathke~\cite{JeffreyR05}.
%The work %Sangiorgi as part of his Ph.D.~research
%%\cite{San96H,SangiorgiD:expmpa}
%\cite{SangiorgiD:expmpa}
%introduced the first fully-abstract encoding from the higher-order 
%$\pi$-calculus into the $\pi$-calculus. 
%Sangiorgi's encoding is based on the idea of a replicated input-guarded process 
%(a trigger process). 
%%We use a similar  replicated triggered process to encode \HOp into \sessp (\defref{d:enc:hopitopi}).
% Operational correspondence for
%the triggered encoding is shown using a context bisimulation
%with first-order labels.
As we have discussed, although context bisimilarity has a satisfactory discriminative power,
its use is hindered by the universal quantification on output.
To deal with this, 
Sangiorgi proposes \emph{normal bisimilarity}, 
a tractable  equivalence without universal quantification. 
To prove that context and normal bisimilarities coincide,~\cite{SangiorgiD:expmpa} uses 
triggered processes.
%The encoding also motivates the definition of a form of
Triggered bisimulation is also defined on first-order labels
where the context bisimulation is restricted to arbitrary
trigger substitution. %rather than arbitrary process substitutions.
This
characterisation of context bisimilarity  was refined in~\cite{JeffreyR05} for
calculi with recursive types, not addressed in~\cite{San96H,SangiorgiD:expmpa} and
quite relevant in %our work (cf. \defref{d:enc:hopitoho}).
session-based concurrency.
The
bisimulation in~\cite{JeffreyR05}
is based on an LTS  extended with trigger meta-notation.
%for a full higher-order $\pi$-calculus that allows
%higher-order applications.
As in~\cite{San96H,SangiorgiD:expmpa}, 
the LTS in~\cite{JeffreyR05}
observes first-order triggered values instead of
higher-order values, offering a more direct characterisation of contextual equivalence
and lifting the restriction to finite types.
\emph{Environmental bisimulations}~\cite{DBLP:conf/lics/SangiorgiKS07} 
%which 
%Sangiorgi et al.~\cite{DBLP:conf/lics/SangiorgiKS07}, 
use a higher-order LTS 
to define a bisimulation that stores the observer's knowledge; hence, observed actions are based on this knowledge
at any given time. This approach is enhanced in~\cite{DBLP:journals/cl/KoutavasH12}
with a mapping from constants to higher-order values. This 
allows to observe first-order values instead
of higher-order values. It differs from~\cite{San96H,JeffreyR05} in that 
the mapping between higher- and first-order values is no longer implicit.

\paragraph{Comparison with respect to~\cite{JeffreyR05}.} 
We briefly contrast 
the approach in~\cite{JeffreyR05} and ours based on 
%\dk{higher-order ($\hwb$) and} 
characteristic  bisimilarity ($\fwb$):
\begin{enumerate}[$\bullet$]
%\begin{enumerate}[i.]
\item 
The LTS in~\cite{JeffreyR05} is enriched with extra labels for triggers;
an output action transition emits a trigger and introduces a parallel replicated trigger.
Our 
approach retains usual labels/transitions; in  case of output,
%our bisimilarities 
%$\hwb$ and 
$\fwb$
introduces a parallel
\emph{non-replicated} trigger.

\item Higher-order input in~\cite{JeffreyR05} involves 
the input of a trigger which reduces after substitution.
Rather than a trigger name, %our bisimulations  
%$\hwb$ and 
$\fwb$
decrees the input of a trigger value $\abs{z}\binp{t}{x} (\appl{x}{z})$.

\item Unlike~\cite{JeffreyR05}, 
%our 
$\fwb$ treats  
first- and higher-order values uniformly. %In the latter case, 
%Since the 
As the typed LTS distinguishes linear and shared values,
replicated closures are used only for shared values.

\item In~\cite{JeffreyR05}   name matching   is
crucial to prove completeness of bisimilarity.
In our case, \HOp lacks name matching and 
%Contrarily 
%\jpc{In contrast,} 
we use session types: a characteristic value inhabiting a type enables the simplest form of interactions with the environment.

%We use the characteristic process interaction with the environment, exploiting session types.
%, i.e., instead of matching a name, we embed it into a process and then observe its behaviour.

%In~\cite{JeffreyR05}  a matching construct 
%is crucial to prove completeness of bisimilarity.
%Since our language lacks matching,
%we use session type information to obtain the simplest value that 
%enables interaction with the environment.
\end{enumerate}
%\noi 
We compare our approach to that in~\cite{JeffreyR05} 
using a representative example.
%We considered the transitions and resulting processes involved in checking bisimilarity of process 
%$\bout{n}{\abs{x}{\appl{x}(\abs{y}{\bout{y}{m}} \inact)}} \inact$
%with itself.

% !TEX root = main.tex
%\section{Example}


\begin{example}
	Consider process
	$\Gamma; \es; \Delta \cat n: \btout{U} \tinact \proves \bout{n}{\abs{x}{\appl{x}(\abs{y}{\bout{y}{m}} \inact)}} \inact \hastype \Proc$
	with $U = \shot{(\shot{(\shot{(\btout{S} \tinact)})})}$. 
	We will compare with~\cite{JeffreyR05} by counting the steps required to check the bisimilarity
	of this process with itself. Note that
	$\mapchar{\btinp{U} \inact}{s} = \binp{s}{x} \appl{x}{(\abs{y}{\appl{y}{a})}}$, for some fresh $a$.
%
%	\begin{eqnarray*}
%		\mapchar{\btinp{U} \inact}{s}\\
%		= && \binp{s}{x} \mapchar{\shot{(\shot{(\shot{(\btout{S} \tinact)})})}}{x}\\
%		= && \binp{s}{x} \appl{x}{\omapchar{\shot{(\shot{(\btout{S} \tinact)})}}}\\
%		= && \binp{s}{x} \appl{x}{(\abs{y}{\mapchar{\shot{(\btout{S} \tinact)}}{y}})}\\
%		= && \binp{s}{x} \appl{x}{(\abs{y}{\appl{y}{\omapchar{\btout{S} \tinact}})}}\\
%		= && \binp{s}{x} \appl{x}{(\abs{y}{\appl{y}{a})}}\\
%	\end{eqnarray*}
%
%	The characteristic process of
%	the type $\shot{(\shot{(\shot{(\btout{S} \tinact)})})}$ is

\noi In our approach, we would have the following typed transitions:

	\begin{tabular}{rrl}
		(1)& & $\Gamma; \es; \Delta \cat n: \btout{U} \tinact \proves \bout{n}{\abs{x}{\appl{x}(\abs{y}{\bout{y}{m}} \inact)}} \inact$ \\
		&$\by{\bactout{n}{\abs{x}{\appl{x}(\abs{y}{\bout{y}{m}} \inact)}}}$& $\Gamma; \es; \Delta \proves \inact$\\
		(2) & & $\Gamma; \es; \Delta \proves \binp{t}{z} \newsp{s}{\mapchar{\btinp{U} \inact}{s} \Par \bout{\dual{s}}{\abs{x}{\appl{x}(\abs{y}{\bout{y}{m}} \inact)}} \inact}$\\
		&$=$& $\Gamma; \es; \Delta \proves \binp{t}{z} \newsp{s}{\binp{s}{x} \appl{x}{(\abs{y}{\appl{y}{a})}} \Par \bout{\dual{s}}{\abs{x}{\appl{x}(\abs{y}{\bout{y}{m}} \inact)}} \inact}$\\
		(3) &$\by{\bactinp{t}{b}}$& $\Gamma; \es; \Delta \proves \newsp{s}{\binp{s}{x} \appl{x}{(\abs{y}{\appl{y}{a})}} \Par \bout{\dual{s}}{\abs{x}{\appl{x}(\abs{y}{\bout{y}{m}} \inact)}} \inact}$\\
		(4) &$\by{\tau}$& $\Gamma; \es; \Delta \proves \appl{\abs{x}{\appl{x}(\abs{y}{\bout{y}{m} \inact})}}{(\abs{y}{\appl{y}{a})}}$\\
		(5) &$\by{\tau}$& $\Gamma; \es; \Delta \proves \appl{(\abs{y}{\appl{y}{a}})}{(\abs{y}{\bout{y}{m} \inact})} $\\
		(6) &$\by{\tau}$& $\Gamma; \es; \Delta \proves \appl{(\abs{y}{\bout{y}{m} \inact})}{a}$\\
		(7) &$\by{\tau}$& $\Gamma; \es; \Delta \proves \bout{a}{m} \inact$
	\end{tabular}

\noi In the approach defined in~\cite{JeffreyR05} we would have:

	\begin{tabular}{rrl}
		(1)& & $\Gamma; \es; \Delta \cat n: \btout{U} \tinact \proves \bout{n}{\abs{x}{\appl{x}(\abs{y}{\bout{y}{m}} \inact)}} \inact$ \\
		&$\by{\bactout{n}{\abs{x}{\appl{x}(\abs{y}{\bout{y}{m}} \inact)}}}$& $\Gamma; \es; \Delta \proves \inact$\\
		(2) & & $\Gamma; \es; \Delta \proves \repl{} \binp{t}{x} \appl{x}(\abs{y}{\bout{y}{m}} \inact) $\\
		(3) &$\by{\bactinp{t}{\tau_l}}$& $\Gamma; \es; \Delta \proves \repl{} \binp{t}{x} \appl{x}(\abs{y}{\bout{y}{m}} \inact) \Par \appl{\abs{x}{\appl{x}(\abs{y}{\bout{y}{m}} \inact)}}{\tau_l}$\\
		(4) &$\by{\tau}$& $\Gamma; \es; \Delta \proves \repl{} \binp{t}{x} \appl{x}(\abs{y}{\bout{y}{m}} \inact) \Par \appl{\tau_l}{(\abs{y}{\bout{y}{m}} \inact)}$\\
		(5) &$\by{\bactout{l}{\tau_k}}$& $\Gamma; \es; \Delta \proves \repl{} \binp{t}{x} \appl{x}(\abs{y}{\bout{y}{m}} \inact) \Par \repl{} \binp{k}{y} \bout{y}{m} \inact $\\
		(6) &$\by{\bactinp{k}{a}}$& $\Gamma; \es; \Delta \proves \repl{} \binp{t}{x} \appl{x}(\abs{y}{\bout{y}{m}} \inact) \Par \repl{} \binp{k}{y} \bout{y}{m} \inact \Par \appl{\abs{y}{\bout{y}{m} \inact}}{a}$\\
		(7) &$\by{\tau}$& $\Gamma; \es; \Delta \proves \repl{} \binp{t}{x} \appl{x}(\abs{y}{\bout{y}{m}} \inact) \Par \repl{} \binp{k}{y} \bout{y}{m} \inact \Par \bout{a}{m} \inact$
	\end{tabular}
	
\noi This simple example shows how both approaches feature the same number of (typed) transitions.
It is interesting to see how our approach based on refined LTS and characteristic bisimilarity requires less observable actions than that in~\cite{JeffreyR05}.
Also, as we are able to distinguish between linear and shared names, we require less replicated processes than in~\cite{JeffreyR05}. 

%	Comparing
%	\begin{itemize}
%		\item	Same number of transitions.
%		\item	J \& R more observable actions.
%		\item	J \& R replicated processes.
%	\end{itemize}

\end{example}




The previous comparison %, detailed in \appref{app:jandr}, 
shows how our approach 
%based on %even if both techniques require the same number of transitions, 
%a refined LTS and characteristic bisimilarity 
requires less visible transitions and replicated processes. 
Therefore, linearity information does simplify analyses, 
as it enables simpler witnesses in  coinductive proofs.





%There are similarities and differences between of the characteristic bisimulation
%and the bisimulation as defined by Jeffrey and Rathke
%(below we use the meta-notation adopted in~\cite{JeffreyR05}):
%%
%\begin{enumerate}[i)]
%	\item	The output of a higher-order value $\abs{x}{Q}$ on name
%		$n$ in Jeffrey\&Rathke approach requires the output of
%		a fresh trigger name $t$ (notation $\tau_t$)
%		on name $n$ 
%		and then the introduction of a replicated triggered process
%		(notation $(t \Leftarrow (x) Q)$)
%		in the context of the acting process:
%		%
%		\[
%			P \by{\news{t} \bactout{n}{\tau_{t}}} P' \Par (t \Leftarrow (x) Q) \by{\bactinp{t}{v}} P' \Par \appl{(x) Q}{v} \Par (t \Leftarrow (x) Q) 
%		\]
%		%
%		In the characteristic bisimulation approach we only observe
%		an output of a value that can be either first- or higher-order:
%		%
%		\[
%			P \hby{\bactout{n}{V}} P' 
%		\]
%		%
%		with $V = \abs{x}{Q}$ or $V = m$.
%		A non-replicated triggered process appears in
%		the parallel context of the acting process when
%		we compare two processes for behavioural equality
%		(cf.~characteristic bisimulation \defref{d:fwb}),
%		$P' \Par \htrigger{t}{\abs{x}{Q}}$.
%		In fact using the LTS in
%		\defref{d:tlts} we can get:
%		%
%		\begin{eqnarray*}
%			P' \Par \htrigger{t}{\abs{x}{Q}}
%			&\by{\abs{z}{\binp{z}{y} \repl{} \binp{t}{x} \appl{y}{x}}}&
%			P' \Par \newsp{s}{\binp{s}{y} \repl{} \binp{t}{x} \appl{y}{x} \Par \bout{s}{\abs{x}{Q}} \inact}\\
%			&\by{\tau}&
%			P' \Par \repl{}\binp{t}{y} \appl{\abs{x}{Q}}{y}
%		\end{eqnarray*}
%		%
%		that simulates the Jeffrey\&Rathke approach.
%
%		The characteristic bisimulation differentiates from
%		the Jeffrey\&Rathke approach:
%		\begin{enumerate}[$\bullet$]
%			\item	The typed LTS predicts the case of linear
%				output values and will never allow replication
%				of such a value;
%				if $V$ is linear the input action would have no replication
%				operator, as
%				$\abs{z}{\binp{z}{y} \binp{t}{x} \appl{y}{x}}$.
%
%			\item	The characteristic bisimulation introduces a uniform approach
%				not only for
%				higher-order values but for first-order values
%				as well. A triggered process can accept any
%				process that can substitute a first-order value as well.
%				This is derived from the fact that the $\HOp$
%				calculus makes no use of a matching operator, in contrast
%				to the calculus defined in~\cite{JeffreyR05})
%				where name matching is crucial to prove completness
%				of the bisimilarity relation.
%				We chose not to include the matching operator
%				because of the requirement of a minimal calculus.
%				In the lack of matching we use types to inhabit
%				a value so we can observe its simplest interaction
%				with the process environment.
%
%			\item	The \HOp calculus requires only first-order
%				applications. Higher-order applications,
%				as in the Jeffrey\&Rathke work,
%				are presented as an extension in the \HOpp
%				calculus.
%
%			\item	The trigger process is non-replicated. In fact
%				the trigger process transforms guards the output
%				value with a higher-order input prefix. The
%				functionality of the input is used to
%				simulate the contextual bisimilarity that subsumes
%				the replicated trigger approach.
%				The transformation of an output action as an input
%				action allows for treating an output
%				using the restricted LTS \defref{def:rlts}:
%				%
%				\[
%					P' \Par \htrigger{t}{\abs{x}{Q}} \hby{\bactinp{t}{\abs{x}{\mapchar{U}{x}}}}
%					P' \Par \news{s}{ \appl{\mapchar{U}{x}}{s} \Par \bout{\dual{s}}{\abs{x}{Q}} \inact}
%				\]
%		\end{enumerate}
%		%
%		%In essence we are transforming a replicated trigger into a process
%		%that is input-prefixed on a fresh name that receives a higher-order
%		%value;
%
%	\item	The input of a higher-order value in the Jeffrey\&Rathke approach requires 
%		the input of a fresh trigger name, which is substituted on the application
%		variable, thus having a meta-suntax for triggered application instead
%		of higher-order applications:
%		%
%		\[
%			\binp{n}{x} P \by{\bactinp{n}{\tau_k}} \appl{\abs{x}{P}}{\tau_k} \by{\tau} P \subst{x}{\tau_k} 
%		\]
%		%
%		with every instance of process variable $x$ in $P$ being substituted
%		with trigger value $\tau_k$ to give a process of the form $\appl{\tau_k}{x}$.
%		The approach in the characteristic bisimulation observes the
%		triggered value
%		$\abs{z}\binp{t}{x} \appl{x}{z}$ as an input instead of the
%		trigger name:
%		%
%		\[
%			P \hby{\bactinp{n}{\abs{z}\binp{t}{x} \appl{x}{z}}} P \subst{\abs{z}\binp{t}{x} \appl{x}{z}}{x}
%		\]
%		%
%		with applications being transformed to
%		$\abs{z}{\binp{t}{x} \appl{x}{z}}{v}$
%		Note that in the characteristic bisimulation semantics
%		we can also observe a characteristic process as an input.
%		
%	\item 	Triggered application in the Jeffrey\&Rahtke
%		are observe using an output
%		lead into an output observation of the
%		application value over
%		the fresh trigger name.
%		%
%		\[
%			\appl{\tau_k}{v} \by{\bactout{k}{v}} \inact
%		\]
%		%
%		In the characteristic bisimulation instead of observing an 
%		application and its value as an action we observe:
%		i) the name of the trigger through the trigger value
%		application; and ii) the application
%		value by inhabiting it in the characteristic value
%		and observing the interaction of the corresponding
%		characteristic process with its environment.
%		%
%		\begin{eqnarray*}
%			\appl{\abs{z}{\binp{t}{x} \appl{x}{z}}}{v} &\by{\tau}& \binp{t}{x} \appl{x}{v}
%			\by{\bactinp{t}{\abs{x}{\mapchar{U}{x}}}}
%			\appl{\abs{x}{\mapchar{U}{x}}}{v}
%			\by{\tau} \mapchar{U}{x} \subst{n}{x}
%		\end{eqnarray*}
%		%
%\end{enumerate}

%The main differences of the triggered
%bisimulation approach comparing to our approach are:
%i) We use observe higher-order values on the LTS in contrast to first-order 
%values in~\cite{DBLP:journals/lmcs/JeffreyR05}.
%ii) In our approach we avoid the replicated triggered process,
%by transforming the output process into a higher-order guarded input.
%iii) The triggered bisimulation gives semantics for higher-order application,
%whereas in our approach we give semantics for first-order applications
%and show that higher-order applications are fully encodable.

%Boreale and Sangiorgi, 
%Deng and Hennessy, 
%Jeffrey and Rathke, Hennessy and Koutavas, Schmitt and Lenglet, Pi\E9rard and Sumii.
%Perez et al (bisimilarities for binary sessions), Kouzapas and Yoshida (bisimilarities for binary and multiparty sessions).
%Bisimilarities for HO processes: \cite{Xu07}.









% !TEX root = main.tex
%\section{Example}


\begin{example}
	Consider process
	$\Gamma; \es; \Delta \cat n: \btout{U} \tinact \proves \bout{n}{\abs{x}{\appl{x}(\abs{y}{\bout{y}{m}} \inact)}} \inact \hastype \Proc$
	with $U = \shot{(\shot{(\shot{(\btout{S} \tinact)})})}$. 
	We will compare with~\cite{JeffreyR05} by counting the steps required to check the bisimilarity
	of this process with itself. Note that
	$\mapchar{\btinp{U} \inact}{s} = \binp{s}{x} \appl{x}{(\abs{y}{\appl{y}{a})}}$, for some fresh $a$.
%
%	\begin{eqnarray*}
%		\mapchar{\btinp{U} \inact}{s}\\
%		= && \binp{s}{x} \mapchar{\shot{(\shot{(\shot{(\btout{S} \tinact)})})}}{x}\\
%		= && \binp{s}{x} \appl{x}{\omapchar{\shot{(\shot{(\btout{S} \tinact)})}}}\\
%		= && \binp{s}{x} \appl{x}{(\abs{y}{\mapchar{\shot{(\btout{S} \tinact)}}{y}})}\\
%		= && \binp{s}{x} \appl{x}{(\abs{y}{\appl{y}{\omapchar{\btout{S} \tinact}})}}\\
%		= && \binp{s}{x} \appl{x}{(\abs{y}{\appl{y}{a})}}\\
%	\end{eqnarray*}
%
%	The characteristic process of
%	the type $\shot{(\shot{(\shot{(\btout{S} \tinact)})})}$ is

\noi In our approach, we would have the following typed transitions:

	\begin{tabular}{rrl}
		(1)& & $\Gamma; \es; \Delta \cat n: \btout{U} \tinact \proves \bout{n}{\abs{x}{\appl{x}(\abs{y}{\bout{y}{m}} \inact)}} \inact$ \\
		&$\by{\bactout{n}{\abs{x}{\appl{x}(\abs{y}{\bout{y}{m}} \inact)}}}$& $\Gamma; \es; \Delta \proves \inact$\\
		(2) & & $\Gamma; \es; \Delta \proves \binp{t}{z} \newsp{s}{\mapchar{\btinp{U} \inact}{s} \Par \bout{\dual{s}}{\abs{x}{\appl{x}(\abs{y}{\bout{y}{m}} \inact)}} \inact}$\\
		&$=$& $\Gamma; \es; \Delta \proves \binp{t}{z} \newsp{s}{\binp{s}{x} \appl{x}{(\abs{y}{\appl{y}{a})}} \Par \bout{\dual{s}}{\abs{x}{\appl{x}(\abs{y}{\bout{y}{m}} \inact)}} \inact}$\\
		(3) &$\by{\bactinp{t}{b}}$& $\Gamma; \es; \Delta \proves \newsp{s}{\binp{s}{x} \appl{x}{(\abs{y}{\appl{y}{a})}} \Par \bout{\dual{s}}{\abs{x}{\appl{x}(\abs{y}{\bout{y}{m}} \inact)}} \inact}$\\
		(4) &$\by{\tau}$& $\Gamma; \es; \Delta \proves \appl{\abs{x}{\appl{x}(\abs{y}{\bout{y}{m} \inact})}}{(\abs{y}{\appl{y}{a})}}$\\
		(5) &$\by{\tau}$& $\Gamma; \es; \Delta \proves \appl{(\abs{y}{\appl{y}{a}})}{(\abs{y}{\bout{y}{m} \inact})} $\\
		(6) &$\by{\tau}$& $\Gamma; \es; \Delta \proves \appl{(\abs{y}{\bout{y}{m} \inact})}{a}$\\
		(7) &$\by{\tau}$& $\Gamma; \es; \Delta \proves \bout{a}{m} \inact$
	\end{tabular}

\noi In the approach defined in~\cite{JeffreyR05} we would have:

	\begin{tabular}{rrl}
		(1)& & $\Gamma; \es; \Delta \cat n: \btout{U} \tinact \proves \bout{n}{\abs{x}{\appl{x}(\abs{y}{\bout{y}{m}} \inact)}} \inact$ \\
		&$\by{\bactout{n}{\abs{x}{\appl{x}(\abs{y}{\bout{y}{m}} \inact)}}}$& $\Gamma; \es; \Delta \proves \inact$\\
		(2) & & $\Gamma; \es; \Delta \proves \repl{} \binp{t}{x} \appl{x}(\abs{y}{\bout{y}{m}} \inact) $\\
		(3) &$\by{\bactinp{t}{\tau_l}}$& $\Gamma; \es; \Delta \proves \repl{} \binp{t}{x} \appl{x}(\abs{y}{\bout{y}{m}} \inact) \Par \appl{\abs{x}{\appl{x}(\abs{y}{\bout{y}{m}} \inact)}}{\tau_l}$\\
		(4) &$\by{\tau}$& $\Gamma; \es; \Delta \proves \repl{} \binp{t}{x} \appl{x}(\abs{y}{\bout{y}{m}} \inact) \Par \appl{\tau_l}{(\abs{y}{\bout{y}{m}} \inact)}$\\
		(5) &$\by{\bactout{l}{\tau_k}}$& $\Gamma; \es; \Delta \proves \repl{} \binp{t}{x} \appl{x}(\abs{y}{\bout{y}{m}} \inact) \Par \repl{} \binp{k}{y} \bout{y}{m} \inact $\\
		(6) &$\by{\bactinp{k}{a}}$& $\Gamma; \es; \Delta \proves \repl{} \binp{t}{x} \appl{x}(\abs{y}{\bout{y}{m}} \inact) \Par \repl{} \binp{k}{y} \bout{y}{m} \inact \Par \appl{\abs{y}{\bout{y}{m} \inact}}{a}$\\
		(7) &$\by{\tau}$& $\Gamma; \es; \Delta \proves \repl{} \binp{t}{x} \appl{x}(\abs{y}{\bout{y}{m}} \inact) \Par \repl{} \binp{k}{y} \bout{y}{m} \inact \Par \bout{a}{m} \inact$
	\end{tabular}
	
\noi This simple example shows how both approaches feature the same number of (typed) transitions.
It is interesting to see how our approach based on refined LTS and characteristic bisimilarity requires less observable actions than that in~\cite{JeffreyR05}.
Also, as we are able to distinguish between linear and shared names, we require less replicated processes than in~\cite{JeffreyR05}. 

%	Comparing
%	\begin{itemize}
%		\item	Same number of transitions.
%		\item	J \& R more observable actions.
%		\item	J \& R replicated processes.
%	\end{itemize}

\end{example}





%%%%%%%%%%%%%%%%%%%%%%%%%%%%%%%%%%%%%%%%%%%%%%%%%%%%%%%%%%%%%%%%%%%%%%%%%%%%%
% Bibliography.
%%%%%%%%%%%%%%%%%%%%%%%%%%%%%%%%%%%%%%%%%%%%%%%%%%%%%%%%%%%%%%%%%%%%%%%%%%%%%

%\bibliographystyle{IEEEtran}
%\bibliographystyle{plain}
\bibliographystyle{abbrv}% the recommended bibstyle
{\bibliography{session}}

\newpage
\appendix 
\section{Appendix: the Typing System of \HOp}
\label{app:types}
%\begin{definition}[Type Equivalence]
%\label{def:iso}
%Let $\mathsf{ST}$ a set of closed session types. 
%Two types $S$ and $S'$ are said to be {\em isomorphic} if a pair $(S,S')$ is 
%in the largest fixed point of the monotone function
%$F:\mathcal{P}(\mathsf{ST}\times \mathsf{ST}) \to 
%\mathcal{P}(\mathsf{ST}\times \mathsf{ST})$ defined by:

%\begin{tabular}{rcl}
%$F(\Re)$ &$\!\!=\!\!$&	$\set{(\tinact, \tinact)}$\\
%         &$\!\!\cup\!\!$&	$\set{(\btout{U_1} S_1, \btout{U_2} S_2)
%\bnfbar (S_1, S_2),(U_1, U_2)\in \Re}$\\ 
%       &$\!\!\cup\!\!$&	$\set{(\btinp{U} S_1, \btinp{U} S_2)
%\bnfbar(S_1, S_2),,(U_1, U_2)\in \Re}$\\ 
%	&$\!\!\cup\!\!$&	$\set{(\btbra{l_i: S_i}_{i \in I} \,,\, \btbra{l_i: S_i'}_{i \in I}) \bnfbar \forall i\in I. (S_i, S_i')\in \Re}$\\
%	&$\!\!\cup\!\!$&	$\set{(\btsel{l_i: S_i}_{i \in I}\,,\, \btsel{l_i: S_i'}_{i \in I}) \bnfbar \forall i\in I. (S_i, S_i')\in \Re}$\\
%	&$\!\!\cup\!\!$&	$\set{(\trec{t}{S}, S')
%\bnfbar (S\subst{\trec{t}{S}}{\vart{t}},S')\in \Re}$\\
%	&$\!\!\cup\!\!$&	$\set{(S,\trec{t}{S'})
%\bnfbar (S,S'\subst{\trec{t}{S'}}{\vart{t}})\in \Re}$
%\end{tabular}
	
%\noindent
%Standard arguments ensure that $F$ is monotone, thus the greatest fixed point
%of $F$ exists. We write $S_1 \wb S_2$ if  $(S_1,S_2)\in \Re$. 
%\end{definition}

\begin{definition}[Duality]
\label{def:dual}
Let $\mathsf{ST}$ a set of closed session types. 
Two types $S$ and $S'$ are said to be {\em dual} if a pair $(S,S')$ is 
in the largest fixed point of the monotone function
$F:\mathcal{P}(\mathsf{ST}\times \mathsf{ST}) \to 
\mathcal{P}(\mathsf{ST}\times \mathsf{ST})$ defined by:
\begin{tabular}{rcl}
$F(\Re)$ &$\!\!=\!\!$&	$\set{(\tinact, \tinact)}$\\
         &$\!\!\cup\!\!$&	$\set{(\btout{U_1} S_1, \btinp{U_2} S_2)
\bnfbar(S_1, S_2)\in \Re, \  U_1 \wb U_2 }$\\ 
       &$\!\!\cup\!\!$&	$\set{(\btinp{U_1} S_1, \btout{U_2} S_2)
\bnfbar(S_1, S_2)\in \Re, \ U_1 \wb U_2}$\\ 
	&$\!\!\cup\!\!$&	$\set{(\btsel{l_i: S_i}_{i \in I} \,,\, \btbra{l_i: S_i'}_{i \in I}) \bnfbar \forall i\in I. (S_i, S_i')\in \Re}$\\
	&$\!\!\cup\!\!$&	$\set{(\btbra{l_i: S_i}_{i \in I}\,,\, \btsel{l_i: S_i'}_{i \in I}) \bnfbar \forall i\in I. (S_i, S_i')\in \Re}$\\
	&$\!\!\cup\!\!$&	$\set{(\trec{t}{S}, S')
\bnfbar (S\subst{\trec{t}{S}}{\vart{t}},S')\in \Re}$\\
	&$\!\!\cup\!\!$&	$\set{(S,\trec{t}{S'})
\bnfbar (S,S'\subst{\trec{t}{S'}}{\vart{t}})\in \Re}$
\end{tabular}
\noindent
where $U_1 \wb U_2$ means $U_1$ is type equivalent to $U_2$ \cite{yoshida.vasconcelos:language-primitives}.
Standard arguments ensure that $F$ is monotone, thus the greatest fixed point
of $F$ exists. We write $S_1 \dualof S_2$ if  $(S_1,S_2)\in \Re$. 
\end{definition}




\end{document}

\end{document}
