%%%%%%%%%%%%%%%%%%%%%%%%%%%%%%%%%%%%%%%%%%%%%%%%%%%%%%%%%
%%% CONCUR VERSION
%%%%%%%%%%%%%%%%%%%%%%%%%%%%%%%%%%%%%%%%%%%%%%%%%%%%%%%%%

\documentclass[a4paper,UKenglish]{lipics}
%This is a template for producing LIPIcs articles. 
%See lipics-manual.pdf for further information.
%for A4 paper format use option "a4paper", for US-letter use option "letterpaper"
%for british hyphenation rules use option "UKenglish", for american hyphenation rules use option "USenglish"
% for section-numbered lemmas etc., use "numberwithinsect"
 
\usepackage{microtype,xspace,enumerate,comment,stmaryrd}%if unwanted, comment out or use option "draft"
%\usepackage{times}
\usepackage{mathpartir}
%\graphicspath{{./graphics/}}%helpful if your graphic files are in another directory
%%%%%%%%%%%%%%%%%%%%%%%%%%%%%%%%%%%%%%%%%%%%%%%%%%%%%%%%%%%%%%%%%%%%%%%%%%%%%%%%%%%%%%%%%%%%%%%%%%%%
% Contents
% --------

% 1.  Formating
% 2.  Maths - Theorems
% 3.  The pi Calculus
% 4.  Session Syntax
% 5.  Subject Reduction
% 6.  Global Session Types
% 7.  Global Session Types Equivalence
% 8.  Projection
% 9.  Local Session Types
% 10. Behavioural Theory
% 11. Typed Transitions - Reductions
% 12. Typed Relations
% 13. Confluence Determinacy
% 14. Mapping
% 15. pi Constructs
% 16. LN Transform
% 17. General Types Processes Names Sessions ETC
% 18. newtheorem - newenvironment
% 19. Misc
%%%%%%%%%%%%%%%%%%%%%%%%%%%%%%%%%%%%%%%%%%%%%%%%%%%%%%%%%%%%%%%%%%%%%%%%%%%%%%%%%%%%%%%%%%%%%%%%%%%%


%%%%%%%%%%%%%%%%%%%%%%%%%%%%%%%%%%%%%%%%%%%%%%%%%%%%%%%%%%%%%%%%%%%%%%%%%%%%%%%%%%%%%%%%%%%%%%%%%%%%
%                                       FORMATING
%%%%%%%%%%%%%%%%%%%%%%%%%%%%%%%%%%%%%%%%%%%%%%%%%%%%%%%%%%%%%%%%%%%%%%%%%%%%%%%%%%%%%%%%%%%%%%%%%%%%

% Symbols
\newcommand{\semicolon}{:}
%\newcommand{\colon}{;}
\newcommand{\lrangle}[1]{\langle #1 \rangle}
\newcommand{\blrangle}[1]{\big\langle #1 \big\rangle}

%Tags
\newcommand{\parenthtext}[1]{(\textrm{\small #1})}
\newcommand{\brtext}[1]{[\textrm{\small #1}]}
\newcommand{\textinmath}[1]{\textrm{#1}}
\newcommand{\srule}[1]{\parenthtext{#1}}
\newcommand{\strule}[1]{\textrm{#1}}
\newcommand{\stypes}[1]{{\footnotesize \parenthtext{#1}}}
\newcommand{\ltsrule}[1]{{\footnotesize \lrangle{\textrm{#1}}}}
\newcommand{\eltsrule}[1]{{\footnotesize [\textrm{#1}]}}
\newcommand{\trule}[1]{{\footnotesize\brtext{#1}}}
\newcommand{\orule}[1]{{\scriptsize{\brtext{#1}}}}
\newcommand{\mrule}[1]{{\footnotesize{\parenthtext{#1}}}}

\newcommand{\iftag}{{\textrm{if }}}

% General
\newcommand{\noi}{\noindent}
\newcommand{\Hline}{\rule{\linewidth}{.5pt}}
\newcommand{\Hlinefig}{\rule{\linewidth}{.5pt}\vspace{-4mm}}
\newcommand{\myparagraph}[1]{\noindent{\textbf{#1}\ }}
\newcommand{\jparagraph}[1]{\paragraph{\textbf{#1}}}

%%%%%%%%%%%%%%%%%%%%%%%%%%%%%%%%%%%%%%%%%%%%%%%%%%%%%%%%%%%%%%%%%%%%%%%%%%%%%%%%%%%%%%%%%%%%%%%%%%%%
%                                       MATHS - THEOREMS
%%%%%%%%%%%%%%%%%%%%%%%%%%%%%%%%%%%%%%%%%%%%%%%%%%%%%%%%%%%%%%%%%%%%%%%%%%%%%%%%%%%%%%%%%%%%%%%%%%%%

%\newtheorem{notation}[definition]{Notation}

% BNF form
\newcommand{\bnfis}{\;\;::=\;\;}
\newcommand{\bnfbar}{\;\;\;|\;\;\;}
\newcommand{\sbnfbar}{\;\;|\;\;}

% Proof
\newcommand{\Case}[1]{\noi {\bf Case: }#1\\}
\newcommand{\proofend}{\qed}
%\newcommand{\proofend}{}
\newcommand{\Proof}{\noi {\bf Proof: }}

% Logic
\newcommand{\LogAnd}{\texttt{ and }}
\newcommand{\LogOr}{\texttt{ or }}

% Induction

\newcommand{\basic}{\noi {\bf Basic Step:}\\}
\newcommand{\inductive}{\noi {\bf Inductive Hypothesis:}\\}
\newcommand{\induction}{\noi {\bf Induction Step:}\\}

% Tree

\newcommand{\tree}[2]{
\ensuremath{\displaystyle
		\frac
		{
			%%\raisebox{0.0mm}{$\displaystyle{#1}$}
			#1
			%\vspace{0mm}
		}{
			%\vspace{2mm}
			#2
			%\raisebox{-0.4mm}{$\displaystyle{#2}$}
		}
	}
}


%\newcommand{\tree}[2]{
%\begin{prooftree}
%	#1
%	\justifies
%	#2
%\end{prooftree}
%}

\newcommand{\treeusing}[3]{
\begin{prooftree}
	#1
	\justifies
	#2
	\using
	#3
\end{prooftree}}

% Vectors
\newcommand{\vect}[1]{\tilde{#1}}
\newcommand{\mytilde}[1]{\widetilde{#1}}

% Functions - Set theory
\newcommand{\set}[1]{\{#1\}}
\newcommand{\es}{\emptyset}
\newcommand{\maxset}[1]{\max(#1)}
\newcommand{\setbar}{\ \ |\ \ }
\newcommand{\tuple}[2]{(#1, #2)}
\newcommand{\suchthat}{\cdot}
\newcommand{\powerset}[1]{\mathcal{P}(#1)}
\newcommand{\product}{\times}

\newcommand{\eval}{\downarrow}

\newcommand{\setsubtr}[2]{#1 \backslash #2}

\newcommand{\func}[2]{#1(#2)}
\newcommand{\dom}[1]{\mathtt{dom}(#1)}
\newcommand{\codom}[1]{\mathtt{codom}(#1)}

\newcommand{\funcbr}[2]{#1\lrangle{#2}}

\newcommand{\entails}{\text{implies}}


%%%%%%%%%%%%%%%%%%%%%%%%%%%%%%%%%%%%%%%%%%%%%%%%%%%%%%%%%%%%%%%%%%%%%%%%%%%%%%%%%%%%%%%%%%%%%%%%%%%%
%                                        pi - CALCULUS
%%%%%%%%%%%%%%%%%%%%%%%%%%%%%%%%%%%%%%%%%%%%%%%%%%%%%%%%%%%%%%%%%%%%%%%%%%%%%%%%%%%%%%%%%%%%%%%%%%%%

% Free-Bound notation
\newcommand{\freev}[1]{\lrangle{#1}}
\newcommand{\boundv}[1]{(#1)}

% General pi calculus Syntax
\newcommand{\send}[1]{\overline{#1}}
\newcommand{\ol}[1]{\overline{#1}}
\newcommand{\receive}[1]{#1.}
\newcommand{\inact}{\mathbf{0}}
\newcommand{\If}{\sessionfont{if}\ }
\newcommand{\Then}{\sessionfont{then}\ }
\newcommand{\Else}{\sessionfont{else}\ }
\newcommand{\ifthen}[2]{\If #1\ \Then #2\ }
\newcommand{\ifthenelse}[3]{\ifthen{#1}{#2} \Else #3}
\newcommand{\Par}{\;|\;}
\newcommand{\news}[1]{(\nu\, #1)}
\newcommand{\newsp}[2]{(\nu\, #1)(#2)}
\newcommand{\varp}[1]{#1}
%\newcommand{\rvar}[1]{\mathcal{#1}}
\newcommand{\rvar}[1]{#1}
%\newcommand{\rec}[2]{\mu #1. #2}
\newcommand{\recp}[2]{\mu \rvar{#1}. #2}

\newcommand{\Def}{\sessionfont{def}\ }

\newcommand{\defeq}{\stackrel{\Def}{=}}

\newcommand{\repl}{\ast\,}
\newcommand{\parcomp}[2]{\prod_{#1}{#2}}

% Free-Bound-Names sets
\newcommand{\bn}[1]{\mathtt{bn}(#1)}
\newcommand{\fn}[1]{\mathtt{fn}(#1)}
\newcommand{\ofn}[1]{\mathsf{ofn}(#1)}
\newcommand{\fv}[1]{\mathtt{fv}(#1)}
\newcommand{\bv}[1]{\mathtt{bv}(#1)}
\newcommand{\fs}[1]{\mathtt{fs}(#1)}
\newcommand{\fpv}[1]{\mathtt{fpv}(#1)}
\newcommand{\nam}[1]{\mathtt{n}(#1)}

%Subject - Object
\newcommand{\subj}[1]{\mathtt{subj}(#1)}
\newcommand{\obj}[1]{\mathtt{obj}(#1)}

% Relations
\newcommand{\relfont}[1]{\mathcal{#1}}
\newcommand{\rel}[3]{#1\ \relfont{#2}\ #3}

\newcommand{\scong}{\equiv}
\newcommand{\acong}{\scong_{\alpha}}
\newcommand{\wb}{\approx}
\newcommand{\fwb}{\approx^C}
\newcommand{\hwb}{\approx^H}
\newcommand{\swb}{\approx^{s}}
\newcommand{\wbc}{\approx}
\newcommand{\WB}{\approx}

\newcommand{\red}{\longrightarrow}
\newcommand{\Red}{\rightarrow\!\!\!\!\!\rightarrow}
\newcommand{\Redleft}{\leftarrow\!\!\!\!\!\leftarrow}

%\newcommand{\subst}[2]{\set{#1/#2 }}
\def\subst#1#2{\{\raisebox{.5ex}{\small$#1$}\! / \mbox{\small$#2$}\}}

% Context
\newcommand{\hole}{-}
\newcommand{\context}[2]{#1[#2]}
\newcommand{\Ccontext}[1]{\C[#1]}

% Expression Context
\newcommand{\Econtext}[1]{\E[#1]}

% Barbs
\newcommand{\barb}[1]{\downarrow_{#1}}
\newcommand{\Barb}[1]{\Downarrow_{#1}}
\newcommand{\nbarb}[1]{\not\downarrow_{#1}}
\newcommand{\nBarb}[1]{\not\Downarrow_{#1}}

% General
%\newcommand{\ESP}{\ensuremath{\mathbf{ESP}}}
\newcommand{\ESP}{\text{ESP}}
\newcommand{\ESPsel}{\ESP^+}

%%%%%%%%%%%%%%%%%%%%%%%%%%%%%%%%%%%%%%%%%%%%%%%%%%%%%%%%%%%%%%%%%%%%%%%%%%%%%%%%%%%%%%%%%%%%%%%%%%%%
%                                        SESSION SYNTAX
%%%%%%%%%%%%%%%%%%%%%%%%%%%%%%%%%%%%%%%%%%%%%%%%%%%%%%%%%%%%%%%%%%%%%%%%%%%%%%%%%%%%%%%%%%%%%%%%%%%%

% Session font
\newcommand{\sessionfont}[1]{\mathtt{#1}}
\newcommand{\vart}[1]{\mathsf{#1}}

% General Session symbols
\newcommand{\ssep}{;}
\newcommand{\shsep}{.}
\newcommand{\outses}{!}
\newcommand{\inpses}{?}
\newcommand{\selses}{\triangleleft}
\newcommand{\brases}{\triangleright}
\newcommand{\dual}[1]{\overline{#1}}
\newcommand{\cat}{\cdot}

\newcommand{\allstypes}{\mathcal{S}}

% Binary Session Syntax


\newcommand{\bacc}[2]{#1 \boundv{#2} \shsep}
\newcommand{\breq}[2]{\send{#1} \freev{#2} \shsep}
\newcommand{\bareq}[2]{\send{#1} \freev{#2}}

\newcommand{\breqt}[3]{\send{#1} \boundv{#2:#3} \shsep}
\newcommand{\bacct}[3]{#1 \boundv{#2:#3} \shsep}

\newcommand{\bout}[2]{#1 \outses \freev{#2} \shsep}
\newcommand{\bbout}[2]{#1 \outses \blrangle{#2} \shsep}
\newcommand{\binp}[2]{#1 \inpses \boundv{#2} \shsep}
\newcommand{\bsel}[2]{#1 \selses #2 \shsep}

%\newcommand{\bout}[2]{#1 \outses \freev{#2} \ssep}
%\newcommand{\bbout}[2]{#1 \outses \blrangle{#2} \ssep}
%\newcommand{\binp}[2]{#1 \inpses \boundv{#2} \ssep}
%\newcommand{\bsel}[2]{#1 \selses #2 \ssep}
\newcommand{\bbra}[2]{#1 \brases \set{#2}}
\newcommand{\bbras}[2]{#1 \brases #2}
\newcommand{\bbraP}[1]{#1 \brases \lPi}

% Multiparty Session syntax

\newcommand{\role}[1]{[#1]}

\newcommand{\srole}[2]{#1\role{#2}}
\newcommand{\sqrole}[2]{#1^{[]}\role{#2}}

\newcommand{\fromto}[2]{\role{#1} \role{#2}}
\newcommand{\sfromto}[3]{#1\fromto{#2}{#3}}

\newcommand{\sout}[3]{\srole{#1}{#2} \outses \freev{#3} \ssep}
\newcommand{\sinp}[3]{\srole{#1}{#2} \inpses \boundv{#3} \ssep}
\newcommand{\sdel}[4]{\srole{#1}{#2} \outses \freev{\srole{#3}{#4}} \ssep}
\newcommand{\ssel}[3]{\srole{#1}{#2} \selses #3 \ssep}
\newcommand{\sbra}[3]{\srole{#1}{#2} \brases \set{#3}}
\newcommand{\sbras}[3]{\srole{#1}{#2} \brases #3}
\newcommand{\sbraP}[2]{\srole{#1}{#2} \brases \lPi}

\newcommand{\acc}[3]{#1 \role{#2} \boundv{#3} \shsep}
\newcommand{\req}[3]{\send{#1} \role{#2} \boundv{#3} \shsep}
\newcommand{\areq}[3]{\send{#1} \role{#2} \freev{#3}}

\newcommand{\out}[4]{\sfromto{#1}{#2}{#3} \outses \freev{#4} \ssep}
\newcommand{\inp}[4]{\sfromto{#1}{#2}{#3} \inpses \boundv{#4} \ssep}
\newcommand{\del}[5]{\sfromto{#1}{#2}{#3} \outses \freev{\srole{#4}{#5}} \ssep}
\newcommand{\sel}[4]{\sfromto{#1}{#2}{#3} \selses #4 \ssep}
\newcommand{\bra}[4]{\sfromto{#1}{#2}{#3} \brases \set{#4}}
\newcommand{\bras}[4]{\sfromto{#1}{#2}{#3} \brases #4}
\newcommand{\braP}[3]{\sfromto{#1}{#2}{#3} \brases \lPi}

% Arrive construct
\newcommand{\arrivetext}{\mathtt{arrive}}
\newcommand{\arrive}[1]{\arrivetext\ #1}
\newcommand{\arrivem}[2]{\arrivetext\ #1\ #2}

% Typecase construct
\newcommand{\typecasetext}{\mathtt{typecase}}
\newcommand{\oftext}{\mathtt{of}}
\newcommand{\typecase}[2]{\typecasetext\ #1\ \oftext\ \set{#2}}

% IO symbols
\newcommand{\inputsym}{\mathtt{i}}
\newcommand{\outputsym}{\mathtt{o}}

%%%%%%%%%%%%%%%%%%%%%%%%%%%%%%%%%%%%%%%%%%%%%%%%%%%%%%%%%%%%%%%%%%%%%%%%%%%%%%%%%%%%%%%%%%%%%%%%%%%%
%                                      SUBJECT REDUCTION
%%%%%%%%%%%%%%%%%%%%%%%%%%%%%%%%%%%%%%%%%%%%%%%%%%%%%%%%%%%%%%%%%%%%%%%%%%%%%%%%%%%%%%%%%%%%%%%%%%%%

% typing reduction
\newcommand{\typingred}{\red}
\newcommand{\typingRed}{\Red}

\newcommand{\wellconf}[1]{\mathtt{wc}(#1)}
\newcommand{\cohses}[2]{\mathtt{co}(#1(#2))}
\newcommand{\coherent}[1]{\mathtt{co}(#1)}
\newcommand{\fcoherent}[1]{\mathtt{fco}(#1)}

%%%%%%%%%%%%%%%%%%%%%%%%%%%%%%%%%%%%%%%%%%%%%%%%%%%%%%%%%%%%%%%%%%%%%%%%%%%%%%%%%%%%%%%%%%%%%%%%%%%%
%                                      SESSION ENDPOINTS
%%%%%%%%%%%%%%%%%%%%%%%%%%%%%%%%%%%%%%%%%%%%%%%%%%%%%%%%%%%%%%%%%%%%%%%%%%%%%%%%%%%%%%%%%%%%%%%%%%%%

% Asynchronous syntax
%\newcommand{\mareq}[4]{\newsp{\srole{#2}{#3}, \dots, \srole{#2}{#4}}{\send{#1}[#3] \freev{#2} \Par \dots \Par \send{#1}[#4]\freev{#2}}}

%\newcommand{\areqs}[3]{\send{#1}[\set{#2}] \freev{#3}}

% Queues
\newcommand{\emp}{\epsilon}
\newcommand{\squeue}[3]{\srole{#1}{#2}:#3}
\newcommand{\srqueue}[4]{\srole{#1}{#2}[\inputsym: #3, \outputsym: #4]}
\newcommand{\srqueuei}[3]{\srole{#1}{#2}[\inputsym: #3]}
\newcommand{\srqueueo}[3]{\srole{#1}{#2}[\outputsym: #3]}
\newcommand{\srqueueio}[4]{\srole{#1}{#2}[\inputsym: #3, \outputsym: #4]}
\newcommand{\sgqueue}[2]{\srole{#1}:#2}

% Shared Names Queues
\newcommand{\shqueue}[2]{#1[#2]}
%\newcommand{\shqueuet}[3]{#1[#2, #3]}

% IO Queues
\newcommand{\squeueio}[3]{#1[\inputsym: #2, \outputsym: #3]}
\newcommand{\squeuei}[2]{#1[\inputsym: #2]}
\newcommand{\squeueo}[2]{#1[\outputsym: #2]}

\newcommand{\squeuetio}[4]{#1[#2, \inputsym: #3, \outputsym: #4]}
\newcommand{\squeueto}[3]{#1[#2, \outputsym: #3]}
\newcommand{\squeueti}[3]{#1[#2, \inputsym: #3]}
\newcommand{\squeuet}[2]{#1[#2]}

% Queue Values

\newcommand{\queuev}[2]{\role{#1}(#2)}
\newcommand{\queuel}[2]{\role{#1} #2}
\newcommand{\queues}[3]{\role{#1}(\srole{#2}{#3})}

%%%%%%%%%%%%%%%%%%%%%%%%%%%%%%%%%%%%%%%%%%%%%%%%%%%%%%%%%%%%%%%%%%%%%%%%%%%%%%%%%%%%%%%%%%%%%%%%%%%%
%                                        GLOBAL SESSION TYPES
%%%%%%%%%%%%%%%%%%%%%%%%%%%%%%%%%%%%%%%%%%%%%%%%%%%%%%%%%%%%%%%%%%%%%%%%%%%%%%%%%%%%%%%%%%%%%%%%%%%%

\newcommand{\gtfont}[1]{\mathtt{#1}}
\newcommand{\gsep}{.}

\newcommand{\globaltype}[1]{\lrangle{#1}}
\newcommand{\parties}[1]{\mathtt{\p}(#1)}
\newcommand{\roles}[1]{\mathtt{roles}(#1)}

\newcommand{\fromtogt}[2]{#1 \rightarrow #2 \semicolon}

\newcommand{\valuegt}[3]{\fromtogt{#1}{#2} \lrangle{#3} \gsep}
\newcommand{\selgt}[3]{\fromtogt{#1}{#2} \set{#3}}
\newcommand{\selgtG}[2]{\fromtogt{#1}{#2} \lGi}
\newcommand{\recgt}[2]{\mu \vart{#1}. #2}
\newcommand{\vargt}[1]{\vart{#1}}
\newcommand{\inactgt}{\gtfont{end}}

%%%%%%%%%%%%%%%%%%%%%%%%%%%%%%%%%%%%%%%%%%%%%%%%%%%%%%%%%%%%%%%%%%%%%%%%%%%%%%%%%%%%%%%%%%%%%%%%%%%%
%                              GLOBAL SESSION TYPES EQUIVALENCE
%%%%%%%%%%%%%%%%%%%%%%%%%%%%%%%%%%%%%%%%%%%%%%%%%%%%%%%%%%%%%%%%%%%%%%%%%%%%%%%%%%%%%%%%%%%%%%%%%%%%

\newcommand{\projset}[1]{\mathtt{proj}(#1)}
\newcommand{\aprojset}[1]{\mathtt{aproj}\ #1 }
\newcommand{\gcong}{\equiv}
\newcommand{\govcong}{\cong_g}
\newcommand{\gperm}{\simeq}

%%%%%%%%%%%%%%%%%%%%%%%%%%%%%%%%%%%%%%%%%%%%%%%%%%%%%%%%%%%%%%%%%%%%%%%%%%%%%%%%%%%%%%%%%%%%%%%%%%%%
%                                        PROJECTION
%%%%%%%%%%%%%%%%%%%%%%%%%%%%%%%%%%%%%%%%%%%%%%%%%%%%%%%%%%%%%%%%%%%%%%%%%%%%%%%%%%%%%%%%%%%%%%%%%%%%

\newcommand{\projsymb}{\lceil}
\newcommand{\proj}[2]{#1 \projsymb #2}

%%%%%%%%%%%%%%%%%%%%%%%%%%%%%%%%%%%%%%%%%%%%%%%%%%%%%%%%%%%%%%%%%%%%%%%%%%%%%%%%%%%%%%%%%%%%%%%%%%%%
%                                        LOCAL SESSION TYPES
%%%%%%%%%%%%%%%%%%%%%%%%%%%%%%%%%%%%%%%%%%%%%%%%%%%%%%%%%%%%%%%%%%%%%%%%%%%%%%%%%%%%%%%%%%%%%%%%%%%%

\newcommand{\tfont}[1]{\mathtt{#1}}
\newcommand{\tsep}{;}

\newcommand{\chtype}[1]{\lrangle{#1}}
\newcommand{\chtypei}[1]{\inputsym \lrangle{#1}}
\newcommand{\chtypeo}[1]{\outputsym \lrangle{#1}}
\newcommand{\chtypeio}[1]{\inputsym \outputsym \lrangle{#1}}

\newcommand{\outtype}{\outses}
\newcommand{\inptype}{\inpses}
\newcommand{\seltype}{\selses}
\newcommand{\bratype}{\brases}

\newcommand{\trec}[2]{\mu\vart{#1}.#2}
\newcommand{\tvar}[1]{\vart{#1}}
%\newcommand{\settype}[1]{\set{#1}}
\newcommand{\tset}[1]{\set{#1}}
\newcommand{\tinact}{\tfont{end}}

%\newcommand{\sminus}[1]{#1^-}
\newcommand{\sminus}[1]{#1^{\text{--}}}

\newcommand{\subt}{\leq}
\newcommand{\supt}{\geq}

% Multiparty Local Session Types
\newcommand{\tout}[2]{\role{#1} \outtype \lrangle{#2} \tsep}
\newcommand{\tinp}[2]{\role{#1} \inptype (#2) \tsep}
\newcommand{\tsel}[2]{\role{#1} \seltype \set{#2}}
\newcommand{\tsels}[2]{\role{#1} \seltype #2}
\newcommand{\tselT}[1]{\role{#1} \seltype \lTi}
\newcommand{\tbra}[2]{\role{#1} \bratype \set{#2}}
\newcommand{\tbras}[2]{\role{#1} \bratype #2}
\newcommand{\tbraT}[1]{\role{#1} \bratype \lTi}

% Binary Session Types
\newcommand{\btout}[1]{\outtype \lrangle{#1} \tsep}
\newcommand{\bbtout}[1]{\outtype \big\langle{#1}\big\rangle \tsep}
\newcommand{\btinp}[1]{\inptype (#1) \tsep}
\newcommand{\bbtinp}[1]{\inptype \big({#1}\big) \tsep}
\newcommand{\btsel}[1]{\oplus \set{#1}}
\newcommand{\btselS}{\oplus \lSi}
\newcommand{\btbra}[1]{\& \set{#1}}
\newcommand{\btbraS}{\& \lSi}

% Queue Typing

\newcommand{\mtout}[2]{\role{#1} \outtype \lrangle{#2}}
\newcommand{\mtinp}[2]{\role{#1} \inptype (#2)}
\newcommand{\mtsel}[2]{\role{#1} \seltype #2}
\newcommand{\mtbra}[2]{\role{#1} \bratype #2}


% Binary Queue Typing

\newcommand{\bmtout}[1]{\outtype \lrangle{#1}}
\newcommand{\bmtinp}[1]{\inptype (#1)}
\newcommand{\bmtsel}[1]{\seltype #1}
\newcommand{\bmtbra}[1]{\bratype #1}

% Message concatanation
\newcommand{\mcat}{\;*\;}
\newcommand{\icat}{\;\circ\;}


%%%%%%%%%%%%%%%%%%%%%%%%%%%%%%%%%%%%%%%%%%%%%%%%%%%%%%%%%%%%%%%%%%%%%%%%%%%%%%%%%%%%%%%%%%%%%%%%%%%%
%                                        TYPED PROCESSES
%%%%%%%%%%%%%%%%%%%%%%%%%%%%%%%%%%%%%%%%%%%%%%%%%%%%%%%%%%%%%%%%%%%%%%%%%%%%%%%%%%%%%%%%%%%%%%%%%%%%

\newcommand{\Ga}{\Gamma}
\newcommand{\De}{\Delta}
\newcommand{\proves}{\vdash}
\newcommand{\hastype}{\triangleright}

\newcommand{\Decat}[1]{\De \cat #1}
\newcommand{\Gacat}[1]{\Ga \cat #1}

\newcommand{\tcat}{\circ}

\newcommand{\typed}[1]{#1:}
\newcommand{\typedrole}[2]{\typed{\srole{#1}{#2}}}
%\newcommand{\typedqrole}[2]{\typed{\srole{#1^{[]}}{#2}}}

\newcommand{\typedprocess}[3]{#1 \proves #2 \hastype #3}

\newcommand{\Eproves}[3]{#1 \proves \typed{#2} #3}
\newcommand{\Gproves}[2]{\Eproves{\Ga}{#1}{#2}}

\newcommand{\tprocess}[3]{#1 \proves #2 \hastype #3}
\newcommand{\Gtprocess}[2]{\tprocess{\Ga}{#1}{#2}}
\newcommand{\Gptprocess}[2]{\tprocess{\Ga'}{#1}{#2}}

\newcommand{\noGtprocess}[2]{#1 \hastype #2}

%%%%%%%%%%%%%%%%%%%%%%%%%%%%%%%%%%%%%%%%%%%%%%%%%%%%%%%%%%%%%%%%%%%%%%%%%%%%%%%%%%%%%%%%%%%%%%%%%%%%
%                                    MULTIPARTY TYPED THEORY
%%%%%%%%%%%%%%%%%%%%%%%%%%%%%%%%%%%%%%%%%%%%%%%%%%%%%%%%%%%%%%%%%%%%%%%%%%%%%%%%%%%%%%%%%%%%%%%%%%%%

\newcommand{\globalenv}[1]{\set{#1}}
%\newcommand{\globalenvI}{\set{\typed{s_i} \G_i}_{i \in I}}
\newcommand{\globalenvI}{E}
\newcommand{\globalenvJ}{\set{\typed{s_j} \G_j}_{j \in J}}

\newcommand{\Gltprocess}[4]{\tprocess{#1}{#2}{#3, #4}}

\newcommand{\geI}{\globalenvI}
\newcommand{\geJ}{\globalenvJ}

\newcommand{\Stprocess}[3]{\tprocess{\geI, #1}{#2}{#3}}
\newcommand{\SGtprocess}[2]{\Gtprocess{#1}{#2, \globalenvI}}
\newcommand{\SJGtprocess}[2]{\globalenvJ, \Gtprocess{#1}{#2}}

\newcommand{\Observer}[2]{\mathsf{Observer}(#1, #2)}
\newcommand{\ObserverG}[1]{\mathsf{Observer}(\globalenvI, #1)}

\newcommand{\Obs}{\ensuremath{\mathsf{Obs}}}

%%%%%%%%%%%%%%%%%%%%%%%%%%%%%%%%%%%%%%%%%%%%%%%%%%%%%%%%%%%%%%%%%%%%%%%%%%%%%%%%%%%%%%%%%%%%%%%%%%%%
%                                        BEHAVIOURAL THEORY
%%%%%%%%%%%%%%%%%%%%%%%%%%%%%%%%%%%%%%%%%%%%%%%%%%%%%%%%%%%%%%%%%%%%%%%%%%%%%%%%%%%%%%%%%%%%%%%%%%%%

\newcommand{\fromtolts}[2]{\fromto{#1}{#2}}

\newcommand{\outlts}{\outses}
\newcommand{\inplts}{\inpses}
\newcommand{\sellts}{\oplus}
\newcommand{\bralts}{\&}

% Multiparty Labels
\newcommand{\actreq}[3]{\send{#1} \role{#2} \boundv{#3}}
%\newcommand{\actbreq}[3]{\send{#1} \role{#2} \boundv{#3}}

\newcommand{\actreqs}[3]{\send{#1} \role{\set{#2}} \boundv{#3}}
%\newcommand{\actbreqs}[3]{\send{#1} \role{\set{#2}} \boundv{#3}}

\newcommand{\actacc}[3]{#1 \role{#2} \boundv{#3}}
\newcommand{\actaccs}[3]{#1 \role{\set{#2}} \boundv{#3}}

\newcommand{\actout}[4]{#1 \fromtolts{#2}{#3} \outlts \freev{#4}}
\newcommand{\actqout}[4]{#1^{[]} \fromtolts{#2}{#3} \outlts \freev{#4}}
\newcommand{\actbout}[4]{#1 \fromtolts{#2}{#3} \outlts \boundv{#4}}

\newcommand{\actdel}[5]{#1 \fromtolts{#2}{#3} \outlts \freev{\srole{#4}{#5}}}
\newcommand{\actbdel}[5]{#1 \fromtolts{#2}{#3} \outlts \boundv{\srole{#4}{#5}}}
\newcommand{\actqdel}[5]{#1^{[]} \fromtolts{#2}{#3} \outlts \boundv{\srole{#4}{#5}}}

\newcommand{\actinp}[4]{#1 \fromtolts{#2}{#3} \inplts \freev{#4}}
\newcommand{\actqinp}[4]{#1^{[]} \fromtolts{#2}{#3} \inplts \freev{#4}}

\newcommand{\actsel}[4]{#1 \fromtolts{#2}{#3} \sellts #4}
\newcommand{\actqsel}[4]{#1^{[]} \fromtolts{#2}{#3} \sellts #4}

\newcommand{\actbra}[4]{#1 \fromtolts{#2}{#3} \bralts #4}
\newcommand{\actqbra}[4]{#1^{[]} \fromtolts{#2}{#3} \bralts #4}

\newcommand{\actval}[4]{#1: #2 \rightarrow #3:#4}
\newcommand{\actgsel}[4]{#1: #2 \rightarrow #3:#4}

% Binary Labels
\newcommand{\bactreq}[2]{\send{#1} \freev{#2}}
\newcommand{\bactbreq}[2]{\send{#1} \boundv{#2}}
\newcommand{\bactacc}[2]{#1 \freev{#2}}

\newcommand{\bactout}[2]{#1 \outlts \freev{#2}}
\newcommand{\bactbout}[2]{#1\outlts \boundv{#2}}
\newcommand{\bactinp}[2]{#1 \inplts \freev{#2}}
\newcommand{\bactsel}[2]{#1 \sellts #2}
\newcommand{\bactbra}[2]{#1 \bralts #2}

% Labelled transition relations
\newcommand{\by}[1]{\stackrel{#1}{\longrightarrow}}
\newcommand{\By}[1]{\stackrel{#1}{\Longrightarrow}}

\newcommand{\hby}[1]{\stackrel{#1}{\longmapsto}}
\newcommand{\Hby}[1]{\stackrel{#1}{\Longmapsto}}


% Session barbs
\newcommand{\barbreq}[1]{\barb{#1}}
%\newcommand{\barbacc}[2]{\barb{#1\role{\set{#2}}}}
\newcommand{\barbout}[3]{\barb{\sfromto{#1}{#2}{#3}}}
%\newcommand{\barbinp}[3]{\barb{#1\fromto{#2}{#3}\inpses}}

\newcommand{\Barbreq}[2]{\Barb{\send{#1}\role{#2}}}
%\newcommand{\Barbacc}[2]{\Barb{#1\role{\set{#2}}}}
%\newcommand{\Barbout}[3]{\Barb{\sfromto{#1}{#2}{#3}\outses}}
\newcommand{\Barbout}[3]{\Barb{\sfromto{#1}{#2}{#3}}}
%\newcommand{\Barbinp}[3]{\Barb{#1\fromto{#2}{#3}\inpses}}

% Binary Session barbs
\newcommand{\bbarbreq}[1]{\barb{#1}}
%\newcommand{\bbarbacc}[1]{\barb{#1}}
\newcommand{\bbarbout}[1]{\barb{#1}}
%\newcommand{\bbarbinp}[1]{\barb{#1\inpses}}

\newcommand{\bBarbreq}[1]{\Barb{\send{#1}}}
%\newcommand{\bBarbacc}[2]{\Barb{#1}}
\newcommand{\bBarbout}[1]{\Barb{#1\outses}}
%\newcommand{\bBarbinp}[1]{\Barb{#1\inpses}}

\newcommand{\comp}{\asymp}
\newcommand{\coh}{\asymp}
\newcommand{\bistyp}{\rightleftharpoons}

\newcommand{\typingbeh}{\leftrightarrow}

\newcommand{\bufrel}{\succ}

\newcommand{\ordercup}{\bowtie}

%%%%%%%%%%%%%%%%%%%%%%%%%%%%%%%%%%%%%%%%%%%%%%%%%%%%%%%%%%%%%%%%%%%%%%%%%%%%%%%%%%%%%%%%%%%%%%%%%%%%
%                                TYPED TRANSITIONS - REDUCTIONS
%%%%%%%%%%%%%%%%%%%%%%%%%%%%%%%%%%%%%%%%%%%%%%%%%%%%%%%%%%%%%%%%%%%%%%%%%%%%%%%%%%%%%%%%%%%%%%%%%%%%

% Environment Transitions
\newcommand{\envtrans}[1]{\by{#1}}
\newcommand{\envTrans}[1]{\by{#1}}
\newcommand{\typedtrans}[1]{\by{#1}}
\newcommand{\typedTrans}[1]{\By{#1}}
\newcommand{\typedred}{\red}
\newcommand{\typedRed}{\Red}

% Typed Environment Transitions - Binary case
\newcommand{\benv}[2]{(#1, #2)}
\newcommand{\bGenv}[1]{\benv{\Ga}{#1}}
\newcommand{\bGDenv}{\envtyp{\Ga}{\De}}

\newcommand{\envby}[5]{\benv{#1}{#2} \envtrans{#3} \benv{#4}{#5}}
\newcommand{\Genvby}[3]{\envby{\Ga}{#1}{#2}{\Ga}{#3}}

% Typed Environment Transitions - Multiparty case

\newcommand{\env}[3]{(#1, #2, #3)}
\newcommand{\Genv}[2]{\env{\globalenvI}{#1}{#2}}
\newcommand{\GGenv}[1]{\env{\globalenvI}{\Ga}{#1}}
\newcommand{\GGDenv}{\env{\globalenvI}{\Ga}{\De}}

% Typed Process Transitions - Binary case

\newcommand{\ftby}[7]{\tprocess{#1}{#2}{#3} \typedtrans{#4} \tprocess{#5}{#6}{#7}}
\newcommand{\ftBy}[7]{\tprocess{#1}{#2}{#3} \typedTrans{#4} \tprocess{#5}{#6}{#7}}

\newcommand{\tpby}[6]{\tprocess{#1}{#2}{#3} \typedtrans{#4} \noGtprocess{#5}{#6}}
\newcommand{\tpBy}[6]{\tprocess{#1}{#2}{#3} \typedTrans{#4} \noGtprocess{#5}{#6}}
\newcommand{\Gtpby}[5]{\Gtprocess{#1}{#2} \typedtrans{#3} \noGtprocess{#4}{#5}}
\newcommand{\GtpBy}[5]{\Gtprocess{#1}{#2} \typedTrans{#3} \noGtprocess{#4}{#5}}

\newcommand{\GGtpby}[5]{\tprocess{\globalenvI, \Ga}{#1}{#2} \typedtrans{#3} \noGtprocess{#4}{#5}}
\newcommand{\GGtpBy}[5]{\tprocess{\globalenvI, \Ga}{#1}{#2} \typedTrans{#3} \noGtprocess{#4}{#5}}

% Typed Reductions - Binary case

\newcommand{\ftpred}[6]{\tprocess{#1}{#2}{#3} \typedred \tprocess{#4}{#5}{#6}}
\newcommand{\ftpRed}[6]{\tprocess{#1}{#2}{#3} \typedRed \tprocess{#4}{#5}{#6}}

\newcommand{\tpred}[5]{\tprocess{#1}{#2}{#3} \typedred \noGtprocess{#4}{#5}}
\newcommand{\tpRed}[5]{\tprocess{#1}{#2}{#3} \typedRed \noGtprocess{#4}{#5}}
\newcommand{\Gtpred}[4]{\Gtprocess{#1}{#2} \typedred \noGtprocess{#3}{#4}}
\newcommand{\GtpRed}[4]{\Gtprocess{#1}{#2} \typedRed \noGtprocess{#3}{#4}}

% Observer Reductions

\newcommand{\obsred}{\red_{obs}}
\newcommand{\obsRed}{\Red_{obs}}


%%%%%%%%%%%%%%%%%%%%%%%%%%%%%%%%%%%%%%%%%%%%%%%%%%%%%%%%%%%%%%%%%%%%%%%%%%%%%%%%%%%%%%%%%%%%%%%%%%%%
%                                    TYPED RELATIONS
%%%%%%%%%%%%%%%%%%%%%%%%%%%%%%%%%%%%%%%%%%%%%%%%%%%%%%%%%%%%%%%%%%%%%%%%%%%%%%%%%%%%%%%%%%%%%%%%%%%%

% Typed Relations
\newcommand{\fulltrel}[7]{\rel{\typedprocess{#1}{#2}{#3}}{#4}{\typedprocess{#5}{#6}{#7}}}
\newcommand{\treld}[6]{\rel{\typedprocess{#1}{#2}{#3}}{#4}{\noGtypedprocess{#5}{#6}}}
\newcommand{\trel}[5]{\rel{#1 \proves #2}{#3}{\noGtypedprocess{#4}{#5}}}

\newcommand{\tcong}{\cong}
\newcommand{\twb}{\approx}
\newcommand{\govwb}{\approx_g}
\newcommand{\tequiv}{\approx}

%%%%%%%%%%%%%%%%%%%%%%%%%%%%%%%%%%%%%%%%%%%%%%%%%%%%%%%%%%%%%%%%%%%%%%%%%%%%%%%%%%%%%%%%%%%%%%%%%%%%
%                                    CONFIGURATION THEORY
%%%%%%%%%%%%%%%%%%%%%%%%%%%%%%%%%%%%%%%%%%%%%%%%%%%%%%%%%%%%%%%%%%%%%%%%%%%%%%%%%%%%%%%%%%%%%%%%%%%%

\newcommand{\confpair}[2]{(#1, #2)}
\newcommand{\uptoconfpair}[2]{[#1, #2]}


%%%%%%%%%%%%%%%%%%%%%%%%%%%%%%%%%%%%%%%%%%%%%%%%%%%%%%%%%%%%%%%%%%%%%%%%%%%%%%%%%%%%%%%%%%%%%%%%%%%%
%                                   CONFLUENCE DETERMINACY
%%%%%%%%%%%%%%%%%%%%%%%%%%%%%%%%%%%%%%%%%%%%%%%%%%%%%%%%%%%%%%%%%%%%%%%%%%%%%%%%%%%%%%%%%%%%%%%%%%%%

\newcommand{\sesstrans}[1]{\stackrel{#1}{\longrightarrow_{s}}}
\newcommand{\sessTrans}[1]{\stackrel{#1}{\Longrightarrow_{s}}}

\newcommand{\fulltypedsesstrans}[7]{\typedprocess{#1}{#2}{#3} \sesstrans{#4} \typedprocess{#5}{#6}{#7}}
\newcommand{\fulltypedsessTrans}[7]{\typedprocess{#1}{#2}{#3} \sessTrans{#4} \typedprocess{#5}{#6}{#7}}

\newcommand{\typedsesstrans}[6]{\typedprocess{#1}{#2}{#3} \sesstrans{#4} \noGtypedprocess{#5}{#6}}
\newcommand{\typedsessTrans}[6]{\typedprocess{#1}{#2}{#3} \sessTrans{#4} \noGtypedprocess{#5}{#6}}
\newcommand{\Gtypedsesstrans}[5]{\Gtypedprocess{#1}{#2} \sesstrans{#3} \noGtypedprocess{#4}{#5}}
\newcommand{\GtypedsessTrans}[5]{\Gtypedprocess{#1}{#2} \sessTrans{#3} \noGtypedprocess{#4}{#5}}

% Actions
\newcommand{\confact}[2]{#1 \lfloor #2}

%%%%%%%%%%%%%%%%%%%%%%%%%%%%%%%%%%%%%%%%%%%%%%%%%%%%%%%%%%%%%%%%%%%%%%%%%%%%%%%%%%%%%%%%%%%%%%%%%%%%
%                                    MAPPING AND ENCODINGS
%%%%%%%%%%%%%%%%%%%%%%%%%%%%%%%%%%%%%%%%%%%%%%%%%%%%%%%%%%%%%%%%%%%%%%%%%%%%%%%%%%%%%%%%%%%%%%%%%%%%

\newcommand{\map}[1]{[\!\![#1]\!\!]}
\newcommand{\umap}[1]{[\!\![#1]\!\!]^u}
\newcommand{\pmap}[2]{\ensuremath{[\!\![#1]\!\!]^#2}}
\newcommand{\pmapp}[3]{\ensuremath{[\!\![#1]\!\!]^#2_#3}}
\newcommand{\auxmap}[2]{\ensuremath{\{\!\{#1\}\!\}^#2}}
\newcommand{\tauxmap}[2]{\ensuremath{\{\!|#1|\!\}^#2}}
\newcommand{\auxmapp}[3]{\ensuremath{\big\lfloor\!\!\big\lfloor#1\big\rfloor\!\!\big\rfloor^#2_#3}}
\newcommand{\tmap}[2]{\ensuremath{(\!\!\langle#1\rangle\!\!)^{#2}}}
\newcommand{\vtmap}[2]{{\ensuremath{\big\lfloor #1\big\rfloor^{#2}}}}
\newcommand{\mapt}[1]{\ensuremath{(\!\!\langle#1\rangle\!\!)}}
\newcommand{\mapa}[1]{\ensuremath{\{\!\!\{#1\}\!\!\}}}
\newcommand{\namemap}[2]{#1\map{#2}}

\newcommand{\enc}[2]{\big\langle\map{#1}, \mapt{#2}\big\rangle}
\newcommand{\enco}[1]{\big\langle #1\big\rangle}
\newcommand{\encod}[3]{\lrangle{\map{#1}^{#3}, \mapt{#2}^{#3}}}
\newcommand{\fencod}[4]{\lrangle{\map{#1}^{#3}_{#4} \, , \, \mapt{#2}^{#3}}}

\newcommand{\calc}[5]{\lrangle{#1, #2, #3, #4, #5}}
\newcommand{\tyl}[1]{\ensuremath{\mathcal{#1}}}

%%%%%%%%%%%%%%%%%%%%%%%%%%%%%%%%%%%%%%%%%%%%%%%%%%%%%%%%%%%%%%%%%%%%%%%%%%%%%%%%%%%%%%%%%%%%%%%%%%%%
%                                    PI CONSTRUCTS
%%%%%%%%%%%%%%%%%%%%%%%%%%%%%%%%%%%%%%%%%%%%%%%%%%%%%%%%%%%%%%%%%%%%%%%%%%%%%%%%%%%%%%%%%%%%%%%%%%%%

\newcommand{\constrtype}[1]{\mathtt{#1}}


\newcommand{\Let}{\constrtype{let}\ }
\newcommand{\In}{\constrtype{in}\ }
\newcommand{\To}{\constrtype{to}\ }
\newcommand{\new}{\constrtype{new}\ }
\newcommand{\from}{\constrtype{from}\ }
\newcommand{\select}{\constrtype{select}\ }
\newcommand{\register}{\constrtype{register}\ }
\newcommand{\Update}{\constrtype{update}\ }

\newcommand{\selectfrom}[2]{\select #1\ \from #2\ \In}
\newcommand{\registerto}[2]{\register #1\ \To #2\ \In}

\newcommand{\newselector}[1]{\new \constrtype{sel}\ #1\ \In}
\newcommand{\newselectorT}[2]{\new \constrtype{sel}\lrangle{#2}\ #1\ \In}
\newcommand{\selecttype}[1]{\dual{\constrtype{sel}}\lrangle{#1}}
\newcommand{\sselecttype}[1]{\constrtype{sel}\lrangle{#1}}

\newcommand{\update}[3]{\Update(#1, #2, #3)\ \In}

\newcommand{\newenv}[1]{\new \mathtt{env}\ #1\ \In\ }
\newcommand{\Letin}[2]{\Let #1 = #2\ \In}

\newcommand{\selqueue}[2]{#1\lrangle{#2}}

%%%%%%%%%%%%%%%%%%%%%%%%%%%%%%%%%%%%%%%%%%%%%%%%%%%%%%%%%%%%%%%%%%%%%%%%%%%%%%%%%%%%%%%%%%%%%%%%%%%%
%                                    DUALITY
%%%%%%%%%%%%%%%%%%%%%%%%%%%%%%%%%%%%%%%%%%%%%%%%%%%%%%%%%%%%%%%%%%%%%%%%%%%%%%%%%%%%%%%%%%%%%%%%%%%%
\newcommand{\dualof}{\ \mathsf{dual}\ }


%%%%%%%%%%%%%%%%%%%%%%%%%%%%%%%%%%%%%%%%%%%%%%%%%%%%%%%%%%%%%%%%%%%%%%%%%%%%%%%%%%%%%%%%%%%%%%%%%%%%
%                                        lambda - CALCULUS
%%%%%%%%%%%%%%%%%%%%%%%%%%%%%%%%%%%%%%%%%%%%%%%%%%%%%%%%%%%%%%%%%%%%%%%%%%%%%%%%%%%%%%%%%%%%%%%%%%%%

\newcommand{\labs}[2]{\lambda #1. #2}

%%%%%%%%%%%%%%%%%%%%%%%%%%%%%%%%%%%%%%%%%%%%%%%%%%%%%%%%%%%%%%%%%%%%%%%%%%%%%%%%%%%%%%%%%%%%%%%%%%%%
%                                    HIGHER ORDER SESSION PI
%%%%%%%%%%%%%%%%%%%%%%%%%%%%%%%%%%%%%%%%%%%%%%%%%%%%%%%%%%%%%%%%%%%%%%%%%%%%%%%%%%%%%%%%%%%%%%%%%%%%
%\newcommand{\pHOp}{\ensuremath{\mathsf{HO}\pi_{\mathsf{p}}}\xspace}
%\newcommand{\pHOpnr}{\ensuremath{\mathsf{HO}\pi^{-\mu}_{\mathsf{p}}}\xspace}
\newcommand{\HOp}{\ensuremath{\mathsf{HO}\pi}\xspace}
%\newcommand{\sessp}{\ensuremath{\mathtt{SE}\pi}\xspace}
\newcommand{\sessp}{\ensuremath{\pi}\xspace}
\newcommand{\haskp}{\ensuremath{\pi^{\lambda}}\xspace}
\newcommand{\pHOp}{\ensuremath{\mathsf{HO}\tilde{\pi}}\xspace}
%\newcommand{\psesp}{\ensuremath{\mathtt{sess}\pi_{\mathsf{p}}}\xspace}
%\newcommand{\psespnr}{\ensuremath{\mathtt{sess}\pi^{-\mu}_{\mathsf{p}}}\xspace}
%\newcommand{\sespnr}{\ensuremath{\mathtt{sess}\pi^{-\mu}}\xspace}
\newcommand{\HO}{\ensuremath{\mathsf{HO}}\xspace}
\newcommand{\HOpp}{\ensuremath{\mathsf{HO\pi^{+}}}\xspace}
\newcommand{\PHOp}{\ensuremath{\mathsf{HO}\,{\widetilde{\pi}}}\xspace}
\newcommand{\PHOpp}{\ensuremath{\mathsf{HO}\,{\widetilde{\pi}}^{\,+}}\xspace}
\newcommand{\PHO}{\ensuremath{\vec{\mathsf{HO}}}\xspace}
\newcommand{\Psessp}{\ensuremath{\vec{\pi}}\xspace}


\newcommand{\CAL}{\ensuremath{\mathsf{C}}\xspace}

\newcommand{\pol}{\mathsf{p}}


\newcommand{\ST}{\mathsf{ST}}


%\newcommand{\pHO}{\mathsf{pure\ HO}}

%\newcommand{\HOp}{\HO^+}
%\newcommand{\pHOp}{\pHO^+}
%\newcommand{\ppi}{\mathsf{pure\ session\ }\pi}
%\newcommand{\spi}{\mathsf{session\ }\pi}

\newcommand{\Proc}{\ensuremath{\diamond}}


%\newcommand{\appl}[2]{#1\lrangle{#2}}
\newcommand{\appl}[2]{#1\, {#2}}
%\newcommand{\abs}[2]{(#1)#2}
\newcommand{\abs}[2]{\lambda #1.\,#2}

\newcommand{\lollipop}{\multimap}
\newcommand{\sharedop}{\rightarrow}
\newcommand{\logicop}{\multimapdot}

\newcommand{\lhot}[1]{#1\!\! \lollipop\!\! \diamond}
\newcommand{\shot}[1]{#1\!\! \sharedop\!\! \diamond}
\newcommand{\hot}[1]{#1 \logicop \diamond}

%\newcommand{\absmap}[1]{}
\newcommand{\vmap}[1]{|\!|#1|\!|}
\newcommand{\smap}[1]{(\!|\!|#1|\!|\!)^s}
\newcommand{\svmap}[1]{(\!|\!|#1|\!|\!)^{s\rightarrow v}}
\newcommand{\amap}[1]{\mathcal{A}\map{#1}}
\newcommand{\absmap}[2]{\mathcal{A}\map{#1}^{#2}}



%%%%% triggers

\newcommand{\hotrigger}[2]{\binp{#1}{x} \newsp{s}{\appl{x}{s} \Par \bout{\dual{s}}{#2} \inact}}
\newcommand{\fotrigger}[5]{\binp{#1}{#2} \newsp{#3}{\map{#4}^{#3} \Par \bout{\dual{#3}}{#5} \inact}}
%\newcommand{\fotrigger}[2]{\binp{#1}{X} \appl{X}{#2}}

%%%%%% Typed relations

\newcommand{\horel}[6]{#1; #2 \proves #3 #4 #5 \proves #6}
%\newcommand{\horel}[6]{#1; \es; #2 \proves #3 #4 #5 \proves #6}

\newcommand{\mhorel}[7]{
	\begin{array}{rcll}
		#1; \es; #2 &#4& #5 \proves& #3\\
			&#4& #6 & #7
	\end{array}
}

%%%%%%%%%%%%%%%%%%%%%%%%%%%%%%%%%%%%%%%%%%%%%%%%%%%%%%%%%%%%%%%%%%%%%%%%%%%%%%%%%%%%%%%%%%%%%%%%%%%%
%                                    LN TRANSFORM
%%%%%%%%%%%%%%%%%%%%%%%%%%%%%%%%%%%%%%%%%%%%%%%%%%%%%%%%%%%%%%%%%%%%%%%%%%%%%%%%%%%%%%%%%%%%%%%%%%%%

\newcommand{\Loop}{\mathsf{Loop}}
\newcommand{\CodeBlocks}{\mathsf{CodeBlocks}}

\newcommand{\lnmap}[1]{\namemap{LN}{#1}}
\newcommand{\lnrmap}[1]{\namemap{LNR}{#1}}
\newcommand{\lnblockmap}[1]{\namemap{\mathcal{B}}{#1}}
\newcommand{\lnnonblockmap}[2]{\map{#1, #2}}

\newcommand{\mapenv}[2]{\map{#1}_{#2}}

%%%%%%%%%%%%%%%%%%%%%%%%%%%%%%%%%%%%%%%%%%%%%%%%%%%%%%%%%%%%%%%%%%%%%%%%%%%%%%%%%%%%%%%%%%%%%%%%%%%%
%                                        GENERAL TYPES
%%%%%%%%%%%%%%%%%%%%%%%%%%%%%%%%%%%%%%%%%%%%%%%%%%%%%%%%%%%%%%%%%%%%%%%%%%%%%%%%%%%%%%%%%%%%%%%%%%%%

% Values
\newcommand{\true}{\sessionfont{tt}}
\newcommand{\false}{\sessionfont{ff}}

% Typed
\newcommand{\bool}{\sessionfont{bool}}
\newcommand{\nat}{\sessionfont{nat}}



%%%%%%%%%%%%%%%%%%%%%%%%%%%%%%%%%%%%%%%%%%%%%%%%%%%%%%%%%%%%%%%%%%%%%%%%%%%%%%%%%%%%%%%%%%%%%%%%%%%%
%                                        PROCESSES NAMES SESSIONS ETC
%%%%%%%%%%%%%%%%%%%%%%%%%%%%%%%%%%%%%%%%%%%%%%%%%%%%%%%%%%%%%%%%%%%%%%%%%%%%%%%%%%%%%%%%%%%%%%%%%%%%

% Processes
\newcommand{\PP}{\ensuremath{P}}
\newcommand{\Q}{\ensuremath{Q}}
\newcommand{\R}{\ensuremath{R}}
\newcommand{\OP}{\ensuremath{\mathsf{O}}}

% Global environments
\newcommand{\En}{\ensuremath{En}}

% Session channels
\newcommand{\s}{\ensuremath{s}}
\newcommand{\ds}{\ensuremath{\dual{s}}}
%\newcommand{\Ms}[2]{\ensuremath{s}\role{#1}\role{#2}}

%Dummy channels
\newcommand{\sd}{\mathtt{sd}}
\newcommand{\shd}{\mathtt{shd}}

% Names
\newcommand{\Ia}{\ensuremath{a}}
\newcommand{\Iu}{\ensuremath{u}}

% Variables, values, expressions
\newcommand{\x}{\ensuremath{x}}
\newcommand{\y}{\ensuremath{y}}
\newcommand{\ks}{\ensuremath{k}}
\newcommand{\cc}{\ensuremath{c}}
\newcommand{\va}{\ensuremath{v}}
\newcommand{\e}{\ensuremath{e}}
\newcommand{\n}{\ensuremath{n}}

% Process Variables

\newcommand{\X}{\varp{X}}
\newcommand{\Y}{\varp{Y}}

% Roles
\newcommand{\p}{\ensuremath{\mathtt{p}}}
\newcommand{\q}{\ensuremath{\mathtt{q}}}
\newcommand{\A}{\ensuremath{A}}

% Types
\newcommand{\G}{\ensuremath{G}}
\newcommand{\gG}{\globaltype{\G}}
\newcommand{\U}{\ensuremath{U}}
\newcommand{\So}{\ensuremath{S}}
\newcommand{\T}{\ensuremath{T}}

% Queue Types
\newcommand{\M}{\ensuremath{M}}
\newcommand{\I}{\ensuremath{M_\inputsym}}
\newcommand{\Om}{\ensuremath{M_\outputsym}}
\newcommand{\Typ}{\ensuremath{\mathsf{T}}}

% Queues values
\newcommand{\h}{\ensuremath{h}}

%barbs
\newcommand{\m}{\ensuremath{\mu}}

% Contexts
\newcommand{\C}{\ensuremath{{\Bbb C}}}
\newcommand{\E}{\ensuremath{E}}

% Congruence completness - Definibility
\newcommand{\TT}{\ensuremath{T}}
\newcommand{\suc}{\textrm{succ}}
\newcommand{\fail}{\textrm{fail}}


% Set selection labels
\newcommand{\lPi}{\set{l_i:\PP_i}_{i \in I}}
\newcommand{\lGi}{\set{l_i:\G_i}_{i \in I}}
\newcommand{\lTi}{\set{l_i:\T_i}_{i \in I}}
\newcommand{\lSi}{\set{l_i:\So_i}_{i \in I}}

% Selector proof

\newcommand{\SEL}{P_\mathit{Sel}}
\newcommand{\DSEL}{P_\mathit{DSel}}
\newcommand{\Sel}{\mathsf{Sel}}
\newcommand{\IfSel}{\mathsf{IfSel}}
\newcommand{\DSel}{\mathsf{DSel}}
\newcommand{\PSel}{\mathsf{PermSel}}
\newcommand{\PIfSel}{\mathsf{PermIfSel}}
\newcommand{\PDSel}{\mathsf{PermDSel}}

%%%%%%%%%%%%%%%%%%%%%%%%%%%%%%%%%%%%%%%%%%%%%%%%%%%%%%%%%%%%%%%%%%%%%%%%%%%%%%%%%%%%%%%%%%%%%%%%%%%%
%                                        ENVIRONMENTS
%%%%%%%%%%%%%%%%%%%%%%%%%%%%%%%%%%%%%%%%%%%%%%%%%%%%%%%%%%%%%%%%%%%%%%%%%%%%%%%%%%%%%%%%%%%%%%%%%%%%
\newtheorem{fact}{Fact}[section]
\newtheorem{notation}{Notation}

%\newenvironment{notation}{\paragraph{{\bf Notation}}}{}

%\newtheorem{proposition}[fact]{{\bf\em Proposition}}
%\newtheorem{example}[fact]{{\bf\em Example}}
%\newtheorem{lemma}[fact]{{\bf\em Lemma}}
%\newtheorem{corollary}[fact]{{\bf\em Corollary}}
%\newtheorem{definition}[fact]{{\bf\em Definition}}
%\newtheorem{theorem}[fact]{{\bf\em Theorem}}
%\newtheorem{remark}[fact]{{\bf\em Remark}}

\newcommand{\nonhosyntax}[1]{\colorbox{lightgray}{\ensuremath{#1}}}

\newenvironment{mytheorem}{%\vspace{-3pt}
	\begin{theorem}
}{%\vspace{-4pt}
	\end{theorem}
}

\newenvironment{myproposition}{
	\begin{proposition}%\vspace{-3pt}
}{%\vspace{-4pt}
	\end{proposition}
}

\newenvironment{mycorollary}{
	\begin{corollary}%\vspace{-3pt}
}{%\vspace{-4pt}
	\end{corollary}
}

\newenvironment{mylemma}{
	\begin{lemma}%\vspace{-3pt}
}{%\vspace{-4pt}
	\end{lemma}
}


\newenvironment{mydefinition}{%\vspace{-3pt}
	\begin{definition}
}{%\vspace{-3pt}
	\end{definition}
}

%\newenvironment{proof}{
%	{\em Proof.}
%}{}


%\newcommand{\qed}{\ensuremath{\square}}





%%%%%%%%%%%%%%%%%%%%%%%%%%%%%%%%%%%%%%%%%%%%%%%%%%%%%%%%%%%%%%%%%%%%%%%%%%%%%%%%%%%%%%%%%%%%%%%%%%%%
%                                        MISC
%%%%%%%%%%%%%%%%%%%%%%%%%%%%%%%%%%%%%%%%%%%%%%%%%%%%%%%%%%%%%%%%%%%%%%%%%%%%%%%%%%%%%%%%%%%%%%%%%%%%
\newcommand{\Appendix}[1]{Appendix \ref{#1}}

\newcommand{\dimcom}[1]{{\bf Comment: #1 \\}}

\newcommand{\hintcom}[1]{{\bf Hint: #1 \\}}

\newif\ifny\nyfalse
%\nytrue
\newcommand{\NY}[1]
{\ifny{\color{purple}{#1}}\else{#1}\fi}

\newcommand{\KH}[1]
{\ifny{\color{brown}{#1}}\else{#1}\fi}

\newif\ifdm\dmtrue
%\dmfalse
\newcommand{\dk}[1]
{\ifdm{\color{blue}{#1}}\else{#1}\fi}

\newif\ifrhu\rhutrue
%\rhufalse
\newcommand{\rh}[1]
{\ifdm{\color{red}{#1}}\else{#1}\fi}

\newif\ifjp\jptrue
%\jpfalse
\newcommand{\jp}[1]
{\ifjp{\color{red}{#1}}\else{#1}\fi}

\newif\ifjp\jptrue
%\jpfalse
\newcommand{\jpc}[1]
{\ifjp{\color{red}{#1}}\else{#1}\fi}

\newcommand{\ENCan}[1]{\langle #1 \rangle}
\newcommand{\NI}{\noindent}


\newcommand{\syntaxvspace}{\\[1mm]}

\newcommand{\TO}[2]{#1\to #2}
\newcommand{\GS}[3]{\TO{#1}{#2}\colon \!\ENCan{#3}}

\newcommand{\ASET}[1]{\{#1\}}
\newcommand{\participant}[1]{\mathtt{#1}}
\newcommand{\CODE}[1]{{\tt #1}}

\newcommand{\AT}[2]{#1 \! : \! #2}


\newcommand{\myrm}{}


\newcommand{\secref}[1]{\S\,\ref{#1}}
\newcommand{\defref}[1]{Def.~\ref{#1}}
\newcommand{\notref}[1]{Not.~\ref{#1}}
\newcommand{\defsref}[1]{Defs.~\ref{#1}}
\newcommand{\figref}[1]{Fig.~\ref{#1}}
\newcommand{\thmref}[1]{Thm.~\ref{#1}}
\newcommand{\thmsref}[1]{Thms.~\ref{#1}}
\newcommand{\exref}[1]{Ex.~\ref{#1}}
\newcommand{\propref}[1]{Prop.~\ref{#1}}
\newcommand{\propsref}[1]{Props.~\ref{#1}}
\newcommand{\appref}[1]{App.~\ref{#1}}
\newcommand{\lemref}[1]{Lem.~\ref{#1}}



\newcommand{\stytra}[6]{\ensuremath{#1; #3 \proves #4 \hby{#2} #5 \proves #6 }}
\newcommand{\stytraarg}[7]{\ensuremath{#1; #3 \proves_{#7} #4 \hby{#2} #5 \proves_{#7} #6 }}
\newcommand{\stytraargi}[8]{\ensuremath{#1; #3 \proves_{#7} #4 \hby{#2}_{#8} #5 \proves_{#7} #6 }}
\newcommand{\wtytra}[6]{\ensuremath{#1; #3 \proves #4 \Hby{#2}  #5 \proves #6}}
\newcommand{\wtytraarg}[7]{\ensuremath{#1; #3 \proves_{#7} #4 \Hby{#2}  #5 \proves_{#7} #6 }}
\newcommand{\wtytraargi}[8]{\ensuremath{#1; #3 \proves_{#7} #4 \Hby{#2}_{#8}  #5 \proves_{#7} #6 }}
\newcommand{\wbb}[6]{\ensuremath{#1; #3 \proves #4 \wb #5 \proves #6 }}
\newcommand{\wbbarg}[7]{\ensuremath{#1; #3 \proves_{#7} #4 \wb_{#7} #5 \proves_{#7} #6 }}

\newcommand{\minussh}{\ensuremath{\mathsf{-sh}}\xspace}

\definecolor{lightgray}{gray}{0.75}

\newcommand\greybox[1]{%
  \vskip\baselineskip%
  \par\noindent\colorbox{lightgray}{%
    \begin{minipage}{\textwidth}#1\end{minipage}%
  }%
  \vskip\baselineskip%
}

%\newcommand{\myparagraph}[1]{\paragraph{\bf #1}}

\newcommand{\mapchar}[2]{\ensuremath{[\!\!(#1)\!\!]^{#2}}}
\newcommand{\omapchar}[1]{\ensuremath{[\!\!(#1)\!\!]_{\mathsf{c}}}}

\newcommand{\trigger}[3]{#1 \leftarrow\!\!\!\!\!\!\!\leftarrow #2:#3 }
\newcommand{\htrigger}[2]{#1 \Leftarrow #2}
\newcommand{\ftrigger}[3]{#1 \Leftarrow \AT{#2}{#3}}

\newcommand{\btau}{\tau_{\beta}}
\newcommand{\stau}{\tau_{s}}
\newcommand{\dtau}{\tau_{d}}



%\newcommand{\HOpp}{\ensuremath{\mathsf{HO\pi^{+}}}\xspace}
%\newcommand{\PHOp}{\ensuremath{\mathsf{HO}{\vec{\pi}}}\xspace}
%\newcommand{\PHOpp}{\ensuremath{\mathsf{HO}{\vec{\pi}}^{+}}\xspace}


%\bibliographystyle{plain}% the recommended bibstyle
\theoremstyle{definition}

% Author macros::begin %%%%%%%%%%%%%%%%%%%%%%%%%%%%%%%%%%%%%%%%%%%%%%%%
\title{Characteristic Bisimulations for Higher-Order Session Processes}
\titlerunning{Characteristic Bisimulations for Higher-Order Session Processes} %optional, in case that the title is too long; the running title should fit into the top page column

\author{Dimitrios Kouzapas$^{\text{1}}$}
\author{Jorge A. P\'{e}rez$^{\text{2}}$}
\author{Nobuko Yoshida$^{\text{1}}$}
\affil{1~~~ Imperial College London 
\qquad 
2~~~ University of Groningen} 
%\qquad 2~~~ University of Glasgow}
%\affil[2]{University of Groningen}
%\affil[3]{University of Glasgow}
\authorrunning{D. Kouzapas and J.\,A. P\'{e}rez and N. Yoshida} %mandatory. First: Use abbreviated first/middle names. Second (only in severe cases): Use first author plus 'et. al.'

\Copyright{D. Kouzapas and J.\,A. P\'{e}rez and N. Yoshida}%mandatory, please use full first names. LIPIcs license is "CC-BY";  http://creativecommons.org/licenses/by/3.0/

%\subjclass{Dummy classification -- please refer to \url{http://www.acm.org/about/class/ccs98-html}}% mandatory: Please choose ACM 1998 classifications from http://www.acm.org/about/class/ccs98-html . E.g., cite as "F.1.1 Models of Computation". 
%\keywords{Dummy keyword -- please provide 1--5 keywords}% mandatory: Please provide 1-5 keywords
%% Author macros::end %%%%%%%%%%%%%%%%%%%%%%%%%%%%%%%%%%%%%%%%%%%%%%%%%
%
%%Editor-only macros:: begin (do not touch as author)%%%%%%%%%%%%%%%%%%%%%%%%%%%%%%%%%%
%\serieslogo{}%please provide filename (without suffix)
%\volumeinfo%(easychair interface)
%  {Billy Editor and Bill Editors}% editors
%  {2}% number of editors: 1, 2, ....
%  {Conference title on which this volume is based on}% event
%  {1}% volume
%  {1}% issue
%  {1}% starting page number
%\EventShortName{}
%\DOI{10.4230/LIPIcs.xxx.yyy.p}% to be completed by the volume editor
%% Editor-only macros::end %%%%%%%%%%%%%%%%%%%%%%%%%%%%%%%%%%%%%%%%%%%%%%%

\usepackage{tikz}
\usetikzlibrary{calc}


\begin{document}

\maketitle

% !TEX root = main.tex
\begin{abstract}
\noi \textbf{\abstractname} \ 
%In \emph{higher-order process calculi} exchanged values may contain processes;
%these languages integrate elements from the $\lambda$-calculus and 
%the $\pi$-calculus to specify and reason about mobile code.
%In the setting of structured communications as delineated by \emph{session types},
This work proposes %efficient 
tractable
bisimulations 
for the higher-order $\pi$-calculus with session primitives (\HOp) and 
offers a complete 
study of the expressivity of its most significant subcalculi.
%fully exploring features of  session communications. 
First we develop three typed bisimulations, which are shown to 
coincide with contextual equivalence.
These characterisations  
demonstrate that observing 
only a specific finite set of higher-order values (which inhabit session types) suffices 
to reason about \HOp. 
Next, we identify \HO, 
a minimal, second-order  subcalculus of \HOp in which 
%does {\em not} equip with 
higher-order applications, name-passing
and recursion are absent.
We show that 
%two fully abstract encodings:
\HO can encode the $n$-order \HOp, 
%fully abstractly; 
and 
that
the first-order session $\pi$-calculus can encode 
\HOp. %, fully abstractly.  
Both encodings are fully abstract.
We also 
prove that 
%then prove a  non-encodability result from 
the session $\pi$-calculus
with passing of shared names 
cannot be encoded 
into \HOp without shared names. 
Our results highlight how the expressiveness of \HO
results into more effective 
%reduces a burden of 
reasoning about typed
process equivalences for 
higher-order % name-passing 
processes. 
%separate the expressivity of higher-order session calculi
%and 
%make it more difficult to reason about process equivalences. 
%First, we identify a core higher-order,
%session-typed calculus and develop its characterisations of typed
%contextual equivalences as labeled bisimilarities.  Second, we
%formalize (non) encodability results between session calculi with
%either name-passing or process-passing mechanisms. Our results
%clarify the relationship between name- and process-passing in terms of
%high-level communication structures based on types.


%In calculi for \emph{higher-order} concurrency, values exchanged as communication objects may include processes. Based on process passing, this paradigm is in contrast with the \emph{first-order} (or name passing) concurrency of the $\pi$-calculus. %based on name passing. 
%%The higher-order paradigm is sometimes consideredBy combining mechanisms for functional and concurrent computation, these languages offer a unified account of different forms of interaction. 
%Although previous works have related calculi for higher-order and first-order concurrency, little is known about the nature of this relationship when interactions are disciplined by \emph{types for structured communications}. Here we tackle this challenge from the perspective of \emph{session types}, focusing on \emph{typed behavioral equivalences} and issues of \emph{relative expressiveness}. First, we identify a core higher-order, session-typed calculus and develop for it  characterizations of typed contextual equivalences as labeled bisimilarities. 
%Second, we formalize (non) encodability results between session calculi with either name-passing or process-passing mechanisms. 
%Our results clarify the relationship between name- and process-passing in terms of high-level communication structures based on types. 
\end{abstract}

%\begin{abstract}
%Lorem ipsum dolor sit amet, consectetur adipiscing elit. Praesent convallis orci arcu, eu mollis dolor. Aliquam eleifend suscipit lacinia. Maecenas quam mi, porta ut lacinia sed, convallis ac dui. Lorem ipsum dolor sit amet, consectetur adipiscing elit. Suspendisse potenti. 
% \end{abstract}

\section{Introduction}
\label{sec:intro}
% !TEX root = main.tex
%\myparagraph{Key points}
%\begin{enumerate}[1.]
%%	\item	Session $\pi$ calculus with process passing. DONE
%%	\item	Identify session $\pi$ and process passing subcalculi and their polyadic variants. DONE
%%	\item	Bisimulation theory for higher-order session semantics. DONE
%%	\item	New triggered bisimulation, related to J\&R's. DONE
%%	\item   Elementary values key to characterizations of behavioural equivalence. DONE
%	\item	Types provide techniques to prove completeness without matching. \jp{TBD}
%	\item	We are interested in encodings with properties a la Gorla. 
%                We extended them to typed setting. \jp{TBD}
%%	\item	Encode name-passing to pure process abstraction calculus, with name abstractions. DONE
%%	\item	Type of the recursion encoding uses non tail recursive type $\trec{t}{\btinp{t} \tinact}$. DONE
%%	\item	Encode higher-order semantics to first order semantics. DONE
%%	\item	Negative result. Cannot encode shared names using only shared names.
%%	\item   Extensions with higher-order abstractions and polyadicity also explored. DONE
%\end{enumerate}

%\smallskip 
%
%\myparagraph{Important things to explain}
%Explain our \HO is very small without containg name passing 
%\[ 
%\abs{x}.P \quad \appl{x}{u}
%\]

%Explain we input only characteristic processes.  
%
%\[
%\lambda x.\mapchar{S}{x}
%\]

%\subsection{Higher-Order Session Calculi}
\noindent
\myparagraph{Context.}
By combining features from the $\lambda$-calculus and the $\pi$-calculus, 
in \emph{higher-order process calculi} exchanged values may contain  processes. 
Here we consider \HOp, a higher-order calculus with \emph{session primitives}:
in addition to 
functional
abstractions/applications, \HOp 
contains constructs for 
synchronisation on shared names, 
  session communication on linear names, 
  %(value passing, labelled choice), 
  and recursion.
Thus, \HOp processes may specify %reciprocal exchanges (protocols) 
protocols
for higher-order  processes that
 can be 
 %verified via type-checking 
 type-checked 
 using \emph{session types}~\cite{honda.vasconcelos.kubo:language-primitives}.
%These calculi allow us to specify   
%session protocols in which higher-order values 
%(mobile code) can be exchanged in a type-safe manner. 
%; 
%governed by session types, 
%such protocols cleanly distinguish between 
%linear and unrestricted behaviors in 
%%directed %point-to-point 
%communications.
Higher-order concurrency has received significant attention 
from untyped and typed perspectives (cf., e.g.,~\cite{SangiorgiD:expmpa,JeffreyR05,DBLP:journals/iandc/LanesePSS11,DBLP:journals/cl/KoutavasH12,MostrousY15}).
%in particular via  comparisons with the first-order mobility of the $\pi$-calculus~\cite{MilnerR:calmp1}. 
Although models of session 
communication with  higher-order features exist~\cite{tlca07,DBLP:journals/jfp/GayV10},
their  \emph{behavioural equivalences} 
%and \emph{relative expressiveness}
remain little understood. 
Since types can limit the contexts (environments) in which processes can interact, typed equivalences
usually offer {\em coarser} semantics than untyped semantics.
%for higher-order session calculi. 
%these two issues 
%have been throughly studied
%%are well-understood 
%for higher-order languages without sessions \cite{},
%but not for higher-order process calculi with sessions.
%This is unfortunate, given the wide applicability of session-based concurrency; indeed,
%session types are expressive enough to describe complex 
%communication structures found in practical protocols,  expressible, e.g., via recursive session types.
%Clarifying the status of typed equivalences and relative expressiveness for session languages
Hence, clarifying the status of these equivalences is key to, e.g., 
justify non-trivial  optimisations in protocols involving both name- and process-passing.
%but also for transferring key reasoning techniques between (higher-order) session calculi. 
%Our discovery is that \emph{linearity} of session types plays a vital role to 
%offer equalities/characterisations
%% and fully abstract encodability, 
%which to our knowledge have not been proposed before.   


%In this paper we study
%%address  behavioural equivalences for 
%\HOp, 
%%study behavioral equalities for \HOp, 
%an extension of the higher-order $\pi$-calculus~\cite{SangiorgiD:expmpa} with session primitives:
%\HOp contains constructs for 
%%session establishment
%synchronisation on shared names, 
%recursion, 
% (linear) session communication (value passing and
%labelled choice),
%abstractions and applications. 
%Abstractions are functions from values to processes, 
%\jpc{denoted}
%$\lambda x.P$; applications are 
%denoted $(\lambda x.P)V$, where the value $V$ is either a name or an abstraction.
%We study two significant subcalculi of \HOp, 
%\jpc{which}
%distil higher- and first-order mobility:
%the \HO-calculus, which is \HOp without recursion and name passing, and 
%the session \sessp-calculus \jpc{(here denoted~\sessp)}, which is \HOp without abstractions and applications.  
%While \sessp is, 
%in essence, the calculus in~\cite{honda.vasconcelos.kubo:language-primitives}, 
%this paper shows that \HO  is a new core calculus 
%for higher-order session concurrency.

A well-studied behavioural equivalence for higher-order processes
is \emph{context bisimilarity}~\cite{San96H}. This 
labelled characterisation of %reduction-closed, 
barbed congruence 
offers an adequate distinguishing power at the price of heavy universal quantifications in output clauses.
Obtaining alternative 
characterisations of context bisimilarity
is thus a recurring, important problem 
for higher-order calculi---see, e.g.,~\cite{SangiorgiD:expmpa,San96H,JeffreyR05,DBLP:journals/cl/KoutavasH12,DBLP:journals/corr/Xu13a}. 
In particular, Sangiorgi~\cite{SangiorgiD:expmpa,San96H} has 
studied %important 
useful
characterizations of context bisimilarity
for higher-order processes; such 
characterisations, however,  %in~\cite{SangiorgiD:expmpa,San96H} 
do not scale to  
  calculi with \emph{recursive types}, which in our experience are essential to the practice of 
session-based concurrency. A characterisation  
%context bisimilarity 
that solves this limitation was developed by Jeffrey and Rathke in~\cite{JeffreyR05}.

\smallskip

\myparagraph{This Work.}
Building upon~\cite{SangiorgiD:expmpa,San96H,JeffreyR05}, 
our discovery is that \emph{linearity} of session types plays a vital role 
%to 
%offer equalities and characterisations
% and fully abstract encodability, 
%which to our knowledge have not been proposed before. 
% 
in 
solving 
the %long-standing, 
open problem 
of characterising context bisimilarity for higher-order mobile processes with session communications.
Our approach is to exploit 
%protocol specifications given by session types to limit 
the coarser semantics induced by session types to limit
the behaviour of higher-order session processes. 
 Formally, we enforce this limitation by defining
a refined labelled transition system (LTS)
which effectively 
narrows down the spectrum of allowed process behaviours, 
exploiting elementary processes inhabiting session types.
%thus enabling tractable reasoning techniques. 
We then introduce \emph{characteristic bisimilarity}: this  
 new notion of typed bisimilarity   is 
\emph{tractable}, in that 
it relies on the refined LTS for input actions and, more importantly, 
does not appeal to universal quantifications on output actions. 
%shown to coincide with context bisimilarity.
Our main result is that characteristic  %tractable
bisimilarity coincides with context bisimilarity.
Besides confirming the value of characteristic bisimilarity as an useful reasoning technique for 
higher-order processes with sessions,
%for  specifications of trivial practical scenarios, 
this result is 
%also technically 
remarkable 
also from a technical perspective, for associated 
completeness proofs do not require 
operators for 
name-matching in the process language, in contrast to untyped methods for  higher-order processes
with recursive types~\cite{JeffreyR05}.
%Remarkably session type structures enable to provide 
%a coincidence without name-matching operators in the calculi.

% !TEX root = main.tex
\noi
\begin{comment}
%In  
%\S\,\ref{subsec:intro:expr}
%and
%\S\,\ref{subsec:intro:bisimulation}
We motivate further our contributions and 
give details of the technical challenges involved. 
Some  notation, formally introduced shortly, is useful here:
$\bout{u}{V} P$
and
$\binp{u}{x} P$ denote input- and output-prefixed processes.
Values $V, W$ can be either a name $u$ or an (name) abstraction $\abs{x}Q$.
Processes $P \Par Q$ and $\inact$ denote the parallel composition and inactive processes, respectively.
Given a (linear) session name $s$, we write $\dual{s}$ for its \emph{dual}; 
they are the two \emph{endpoints} of the same session: the restriction operation  
$\news{s}P$ simultaneously covers $s$ and $\dual{s}$ in~$P$. 
The restriction for shared name $a$ in $P$ is denoted $\news{a}P$.
We write $S$ to range over session types; 
this way, e.g., session type $\btout{U} S'$ (resp. $\btinp{U} S'$) is
decrees that the output (resp. input) of a value of type $U$
must precede a protocol with type $S'$. 
The  terminated session is typed with $\tinact$.
%Given  type $U$, 
We write $\lhot{U}$ (resp. $\shot{U}$) for the 
linear (resp. unrestricted) functional type.

\subsection{Relative Expressiveness Results}
\label{subsec:intro:expr}
\myparagraph{Encoding Name Passing and Recursion into \HO.}
Our first encodability result highlights the expressiveness of 
the core higher-order calculus \HO, which lacks name passing and recursion. 
We encode \HOp into \HO, which entails also an encoding of \sessp into \HO.
The challenges in this encoding of concern exactly name passing and recursion.
To encode name output, we ``pack''
the name to be passed around into a suitable abstraction; 
upon reception, the receiver must ``unpack'' this object following a precise protocol.
The encoding formally is defined in Def.~\ref{d:enc:hopitoho}; we illustrate the encoding strategy below.
The encoding of name passing is:
\[
\begin{array}{rcll}
  \map{\bout{u}{w} P}	&=&	\bout{u}{ \abs{z}{\,\binp{z}{x} (\appl{x}{w})} } \map{P} \\
  \map{\binp{u}{x} Q}	&=&	\binp{u}{y} \newsp{s}{\appl{y}{s} \Par \bout{\dual{s}}{\abs{x}{\map{Q}}} \inact}
\end{array}
\]
and so we need 
exactly two (deterministic) reductions 
to unpack  name $w$.
The encoding of a recursive process $\recp{X}{P}$  is delicate, for it 
 must preserve the linearity of session endpoints. To this end, we
%\begin{enumerate}[i)]
%\item 
encode the recursion body $P$ as a (polyadic) name abstraction
in which free session names are converted into name variables.
This higher-order value is embedded in a sort of input-guarded 
``duplicator'' process; the encoding of process variable $X$ is then meant to 
invoke the duplicator in a by-need fashion to simulate recursion unfolding. 

%\item The recursion body $P$ is encoded in such a way that
%the  in $\map{P}$ (linear names) ; the obtained process
%is then used as the body of a  on those variables.
%\item Using a private session, the abstraction obtained in (i) is communicated to a
%process which instantiates the initial free session names in $P$, 
%in coordination with the encoding of the recursion variable $X$ (using a private session).
%\end{enumerate}
%The second step is also challenging:
%in essence, one should establish a private session with the encoding of the recursion  
%body in order to spawn copies of $\map{P}$ with appropriate free session names.
The use of polyadicity is crucial to the encoding; we shall get back to this point below.
It is worth noticing that the typing of the encoding requires 
a non tail recursive type of the form $\trec{t}{\btinp{\lhot{(S,\vart{t}}} \tinact}$
(see Def.~\ref{d:enc:hopitoho} for the precise formulation).

\smallskip 

\myparagraph{Other Encodings.}
We also give %the reverse of the previous encoding, namely 
an encoding of \HOp into \sessp. We rely on the well-known representability result of Sangiorgi~\cite{SangiorgiD:expmpa}. 
Since communicated processes may contain session names, in order to respect linearity and session protocols
the encoding enforces a distinction, depending on whether this kind of names is present in the communication object. If session names are present then a linear server trigger is deployed; otherwise, the replicated server  in~\cite{SangiorgiD:expmpa} can be used. 

As mentioned above, \HOp and \HO feature \emph{first-order} abstractions: 
only names can be used as arguments to abstractions.
Hence, given a (shared/linear) name $u$, in \HOp
we have the reduction $(\abs{x}{P}) \, u   \red  P \subst{u}{x}$.
We also consider \HOpp, an extension of \HOp with \emph{higher-order} abstractions.
 Thus, in \HOpp also an arbitrary value $V$ (possibly a process) can be an argument of an abstraction, 
 and one could have the reduction
 $(\abs{x}{P}) \, V   \red  P \subst{V}{x}$.
 We give an encoding of \HOpp into \HO: it  naturally extends that of \HOp into \HO;
see~\S\,\ref{subsec:hop}.

A well-known feature in process calculi is \emph{polyadicity}, i.e.,  
passing around tuples of values in communications. 
We consider the polyadic extension of \HOp, denoted \pHOp.
In \pHOp we have polyadicity in session communications and abstractions; 
polyadicity of shared names is ruled out by typing. 
This is enough for most purposes, including our encoding from \HOp into \HO.
In a session-typed setting, encoding polyadicity is straightforward, thanks to 
%polyadic arguments can be sent one by one, relying on 
the private character of 
(linear) session names --- see \S\,\ref{subsec:pho} for details.
%\[
%\begin{array}{rl}
%		\map{\binp{u}{x_1, \cdots, x_m} P}
%		 =  & \!\!\!\!
%		\binp{u}{x_1} \cdots ;  \binp{u}{x_m} \map{P}
%		\\
%%		\map{\bout{u}{u_1, \cdots, u_m} P}
%%		 =  & \!\!\!\!
%%		\bout{u}{u_1} \cdots ;  \bout{u}{u_m} \map{P}
%%		\\
%		\map{\bbout{u}{\abs{(x_1, \cdots, x_m)} Q} P}
%		= & \!\!\!\!
%		\bbout{u}{\abs{z}\binp{z}{x_1}\cdots ; \binp{z}{x_m} \map{Q}} \map{P}
%		\\ 
%		\map{\appl{x}{(u_1, \cdots, u_m)}}
%		= & \!\!\!\!
%		\newsp{s}{\appl{x}{s} \Par \bout{\dual{s}}{u_1} \cdots ; \bout{\dual{s}}{u_m} \inact} 
%	\end{array}
%\]
%Notice that encoding of polyadic abstraction/application requires an extra step, 
%in which a monadic abstraction is sent.

\smallskip

\myparagraph{A Non Encodability Result.}
We also show that shared names strictly add expressiveness to session calculi: that is,
there are (non deterministic) behaviours expressible with shared names not expressible using linear names only.
Although somewhat expected we do not know of a formal proof.
We propose such a formal proof, which relies crucially on the behavioural theory that we have introduced here
and on its determinacy properties. %\jp{EXPAND}.


\subsection{Tractable Bisimilarities for Session-Typed Processes}
\end{comment}

\label{subsec:intro:bisimulation}
\noi 
We outline our motivations and methods 
to show how session types are used for formulating 
two tractable bisimulations. 

\myparagraph{Overcoming Issues of Context Bisimilarity.}
%The characterisation of contextual congruence given by 
Context bisimilarity ($\wbc$, \defref{def:wbc}) is a too demanding relation on processes. 
%In the following we motivate our
%proposal for alternative, more tractable characterisations.  
%For the sake of clarity, and to emphasise the novelties of our approach, 
%we often omit type information. 
%Formal definitions including types are in \S\,\ref{sec:behavioural}.
To see the issue, we show 
the following clause for output.
Suppose $P \,\Re\, Q$, for some context bisimulation $\Re$. Then:

\smallskip 

\begin{enumerate}[$(\star)$]
	\item	Whenever 
		$P \by{\news{\tilde{m_1}} \bactout{n}{V}} P'$
		there exist
		$Q'$ and $W$
		such that 
		$Q \by{\news{\tilde{m_2}} \bactout{n}{W}} Q'$
		and, \emph{\textbf{for all} $R$}  with $\fv{R}=x$, 
		$\newsp{\tilde{m_1}}{P' \Par R\subst{V}{x}} \,\Re\, \newsp{\tilde{m_2}}{Q' \Par R\subst{W}{x}}$.
\end{enumerate}
\smallskip 
\noi 
Above, 
$\news{\tilde{m_1}} \bactout{n}{V}$ is the output label of 
value $V$ with extrusion of names in $\tilde{m_1}$.
To reduce the burden induced by 
universal quantification, we introduce \emph{higher-order}  and 
\emph{characteristic}  
bisimulations, two tractable equivalences denoted  $\hwb$ and $\fwb$, respectively.
As we work with an \emph{early} labelled transition system (LTS), 
%we shall also aim at limiting the input actions,  
%so to define a
%bisimulation relation for the output clause without observing
%infinitely many actions on the same input prefix. 
%To this end, 
%
we take the following two steps: 
%
\begin{enumerate}[(a)]
	\item We replace $(\star)$ with a clause involving a more tractable process closure.
	\item We refine the transition rule for input in the LTS,
	to avoid observing infinitely many actions on the same input prefix.
\end{enumerate}
%
\smallskip

\myparagraph{Trigger Processes with Session Communication.}
Concerning~(a), we exploit session types. 
We 
first 
observe that closure $R\subst{V}{x}$ 
in $(\star)$
is context bisimilar to the process:
\begin{equation}\label{equ:1}
	P = \newsp{s}{\appl{(\abs{z}{\binp{z}{x}{R}})}{s} \Par \bout{\dual{s}}{V} \inact}
\end{equation}
\noi 
%where $\binp{z}{x}{R}$ is an input and $\bout{\dual{s}}{V} \inact$
%is an output 
%on the endpoint $\dual{s}$ (the dual of $s$).
In fact,
we do have $P \wbc R\subst{V}{x}$, 
since 
application and reduction of dual endpoints 
%($s$ and $\dual{s}$) 
are deterministic.  
Now let us
consider process $T_{V}$ below, where $t$ is a fresh name:
\begin{equation}\label{equ:0}
T_{V} = \hotrigger{t}{V}
\end{equation}
%We call $\abs{z}{\binp{z}{x} R}$ a {\bf\em trigger value}. 
Let us write $P \by{\bactinp{n}{V}} P'$ to denote an input transition along $n$.
If $T_{V}$ inputs value $\abs{z}{\binp{z}{x} R}$ then
we can simulate the closure of $P$:
\begin{equation}\label{equ:2}
%\hotrigger{t}{V_1} 
T_{V}
\by{\bactinp{t}{\abs{z}{\binp{z}{x} R}}} P 
\wbc 
R\subst{V}{x}
\end{equation}
Processes such as $T_{V}$ 
offer a value at a fresh name; we will use this class of 
{\bf\em trigger processes} to define a
 refined bisimilarity without the demanding 
output clause $(\star)$. Given a fresh name $t$, 
we write $\htrigger{t}{V}$ to 
stand for a trigger process $T_{V}$ for value $V$.
We note that 
in contrast to previous approaches~\cite{SaWabook,JeffreyR05} 
our {trigger processes} do {\em not} use recursion or 
replication. This is crucial for preserving linearity of session names.  

\smallskip

%Then we can use 
%$\newsp{\tilde{m_1}}{P_1 \Par \htrigger{t}{V_1}}$ instead 
%of Clause 1) in Definition \ref{def:wbc} if we input 
%$\abs{z}{\binp{z}{x} R}$.   

\myparagraph{Characteristic Processes and Values.}
Concerning (b), we limit the possible 
input values (such as $\abs{z}{\binp{z}{x} R}$ above) %processes 
by exploiting session types.
The key concept is that of {\bf \emph{characteristic process/value}}
of a type,  
%The characteristic process of a session type $S$ is the process inhabiting $S$. 
the 
simplest term inhabiting that type (\defref{def:char}).
This way, e.g., let $S = \btinp{\shot{S_1}} \btout{S_2} \tinact$
be a session type: first
input an abstraction, %from values $S_1$ to processes, 
then output a value of type $S_2$.
Then, process $Q = \binp{u}{x} (\bout{u}{s_2} \inact \Par \appl{x}{s_1})$
is a characteristic process for $S$ 
\jpc{along name $u$.}
%Thus, characteristic processes follow the communication structures decreed by session types.
Given a session type $S$, we write $\mapchar{S}{u} $
for its characteristic process along name $u$
(cf.~\defref{def:char}).
Also, %Similarly, 
given value type $U$, we write 
$\omapchar{U}$ to denote its characteristic value.


We use the %characteristic %Precisely, we exploit  the
 characteristic value %$\lambda x.\mapchar{U}{x}$. %$\lambda x.\mapchar{U}{x}$. 
$\omapchar{U}$
 to limit input transitions.
Following the same reasoning as (\ref{equ:1})--(\ref{equ:2}), 
we can use an alternative trigger process, called
{\bf\em characteristic trigger process} with type 
$U$ to replace clause
% (1) in Definition~\ref{def:wbc}:
($\star$) in \defref{def:wbc}:
\begin{equation}
	\label{eq:4}
	\ftrigger{t}{V}{U} \defeq \fotrigger{t}{x}{s}{\btinp{U} \tinact}{V}
\end{equation}

%Note that if $U=L$, $\ftrigger{t}{V}{U}$ subsumes 
%$\htrigger{t}{V}$. 
\noi 
\jpc{Thus, in contrast to the trigger process~\eqref{equ:0}, the characteristic trigger process 
in~\eqref{eq:4}
does not involve a
higher-order communication on fresh name $t$.}
To refine the input transition system, we need to observe 
an additional value, 
$\abs{{x}}{\binp{t}{y} (\appl{y}{{x}})}$, 
called the {\bf\em trigger value}. 
This is necessary, because it turns out
that a characteristic value 
alone as the observable input 
is not enough to define a sound bisimulation.
Roughly speaking, the trigger value is used
to observe/simulate application processes.
%to {\em count} the number of free higher-order variables inside 
%the receiver. 
%\jpc{See Example~\ref{ex:motivation} for further details.}

\smallskip 
\myparagraph{Refined Input Transition Rule.}
Based on 
the above discussion, we refine 
the (early) transition rule for input actions. 
%We write $P \by{\bactinp{n}{V}} P'$ for the input transition along $n$.
The transition rule for input roughly becomes 
(see \defref{def:rlts} for details):
\[
		\tree {
%\begin{array}{c}
P \by{\bactinp{n}{V}} P' \quad  V = m \vee V \scong
(\abs{{x}}{\binp{t}{y} (\appl{y}{{x}})}
 \vee  \omapchar{U})  \textrm{ with } t \textrm{ fresh} 
		}{
			P' \hby{\bactinp{n}{V}} P'
		}
\]
Note the distinction between standard and refined transitions: $\by{\bactinp{n}{V}}$ vs. $\hby{\bactinp{n}{V}}$.
Using this rule, we define an alternative  LTS
with refined 
\jpc{(higher-order)}
input. %; all other rules are kept unchanged.
This refined LTS is used for 
both higher-order ($\hwb$) and characteristic ($\fwb$) bisimulations (Defs.~\ref{d:hbw} and~\ref{d:fwb}),
in which the demanding clause~$(\star)$ is replaced with 
more tractable clauses based on trigger processes 
\jpc{(cf.~\eqref{equ:0})} 
and characteristic 
trigger processes
\jpc{(cf.~\eqref{eq:4})},
respectively.
We show that $\hwb$ is useful for \HOp and \HO, and that~$\fwb$ 
can be uniformly used in all subcalculi, including \sessp. 

\NY{Our calculus lacks name matching, 
which is usually crucial to prove completeness of bisimilarity.
Instead of matching,  we use types:
a process trigger embeds a name into a characteristic
process so to observe its session behaviour.}

%\dk{We stress-out that our calculus lacks
%a matching construct
%which is usually a crucial element to prove completeness of bisimilarity.
%Nevertheless, we use types
%%and specifically the characteristic process
%to compensate;
%%for the absence of matching, i.e.~
%instead of name matching, a process trigger embeds a name into a characteristic
%process so to observe its session behavior.}
%Notice that while Definition \ref{d:hbw} is useful for 
%\HOp and its higher-order variants,
%Definition \ref{d:fwb} is useful for first-order sub-calculi of \HOp.




%%\myparagraph{Outline}
%\subsection{Outline}
%\noi \S\,\ref{sec:calculus} presents the calculi; 
%\S\,\ref{sec:types} presents types;
%the tractable bisimulations are in \S\,\ref{sec:behavioural};
%the notion of encoding is in \S\,\ref{s:expr};
%\S\,\ref{sec:positive} and \S\,\ref{sec:negative}
%present positive and negative encodability results, resp;
%\S\,\ref{sec:extension} discusses extensions; and 
%\S\,\ref{sec:relwork} concludes with related work;
%Appendix summarises the typing system. 
%The paper is self-contained. 
%{\bf\em Omitted definitions, additional related work and full proofs can be found 
%in a technical report, available from \cite{KouzapasPY15}.} 


\smallskip
\myparagraph{Outline.} 
%This paper  is structured as follows.
%\begin{enumerate}[$\bullet$]
%\item 
Next, 
%%\secref{sec:overview} overviews 
%we overview the
%key ideas of characteristic bisimilarity, 
%our 
%characterisation of contextual equivalence.
%%\item 
%Then, \secref{sec:calculus}  presents 
we present
the higher-order session calculus \HOp. 
%A small example is given in \S\,X.
\secref{sec:types} gives the session type system
and states type soundness for \HOp.
%\item 
\secref{sec:behavioural} 
develops %\emph{higher-order} and 
\emph{characteristic} bisimilarity and 
%which alleviates the issues of context bisimilarity~\cite{San96H} and 
states our main result: characteristic and context bisimilarities coincide for 
%is shown  to coincide for 
well-typed \HOp processes (\thmref{the:coincidence}).
\secref{sec:relwork}~concludes with related works. 
%The appendix summarises the typing system. 
%\end{enumerate}
%\noi
%The paper is self-contained. 
\textbf{Omitted definitions and proofs %and additional related work/examples  
can be found in Appendix.
%; and 
%additional expressive results and related work/examples in \cite{KouzapasPY15}.
} 




%\section{Overview: Characteristic Bisimulations}
%\label{sec:overview}
%% !TEX root = main.tex
\noi
\begin{comment}
%In  
%\S\,\ref{subsec:intro:expr}
%and
%\S\,\ref{subsec:intro:bisimulation}
We motivate further our contributions and 
give details of the technical challenges involved. 
Some  notation, formally introduced shortly, is useful here:
$\bout{u}{V} P$
and
$\binp{u}{x} P$ denote input- and output-prefixed processes.
Values $V, W$ can be either a name $u$ or an (name) abstraction $\abs{x}Q$.
Processes $P \Par Q$ and $\inact$ denote the parallel composition and inactive processes, respectively.
Given a (linear) session name $s$, we write $\dual{s}$ for its \emph{dual}; 
they are the two \emph{endpoints} of the same session: the restriction operation  
$\news{s}P$ simultaneously covers $s$ and $\dual{s}$ in~$P$. 
The restriction for shared name $a$ in $P$ is denoted $\news{a}P$.
We write $S$ to range over session types; 
this way, e.g., session type $\btout{U} S'$ (resp. $\btinp{U} S'$) is
decrees that the output (resp. input) of a value of type $U$
must precede a protocol with type $S'$. 
The  terminated session is typed with $\tinact$.
%Given  type $U$, 
We write $\lhot{U}$ (resp. $\shot{U}$) for the 
linear (resp. unrestricted) functional type.

\subsection{Relative Expressiveness Results}
\label{subsec:intro:expr}
\myparagraph{Encoding Name Passing and Recursion into \HO.}
Our first encodability result highlights the expressiveness of 
the core higher-order calculus \HO, which lacks name passing and recursion. 
We encode \HOp into \HO, which entails also an encoding of \sessp into \HO.
The challenges in this encoding of concern exactly name passing and recursion.
To encode name output, we ``pack''
the name to be passed around into a suitable abstraction; 
upon reception, the receiver must ``unpack'' this object following a precise protocol.
The encoding formally is defined in Def.~\ref{d:enc:hopitoho}; we illustrate the encoding strategy below.
The encoding of name passing is:
\[
\begin{array}{rcll}
  \map{\bout{u}{w} P}	&=&	\bout{u}{ \abs{z}{\,\binp{z}{x} (\appl{x}{w})} } \map{P} \\
  \map{\binp{u}{x} Q}	&=&	\binp{u}{y} \newsp{s}{\appl{y}{s} \Par \bout{\dual{s}}{\abs{x}{\map{Q}}} \inact}
\end{array}
\]
and so we need 
exactly two (deterministic) reductions 
to unpack  name $w$.
The encoding of a recursive process $\recp{X}{P}$  is delicate, for it 
 must preserve the linearity of session endpoints. To this end, we
%\begin{enumerate}[i)]
%\item 
encode the recursion body $P$ as a (polyadic) name abstraction
in which free session names are converted into name variables.
This higher-order value is embedded in a sort of input-guarded 
``duplicator'' process; the encoding of process variable $X$ is then meant to 
invoke the duplicator in a by-need fashion to simulate recursion unfolding. 

%\item The recursion body $P$ is encoded in such a way that
%the  in $\map{P}$ (linear names) ; the obtained process
%is then used as the body of a  on those variables.
%\item Using a private session, the abstraction obtained in (i) is communicated to a
%process which instantiates the initial free session names in $P$, 
%in coordination with the encoding of the recursion variable $X$ (using a private session).
%\end{enumerate}
%The second step is also challenging:
%in essence, one should establish a private session with the encoding of the recursion  
%body in order to spawn copies of $\map{P}$ with appropriate free session names.
The use of polyadicity is crucial to the encoding; we shall get back to this point below.
It is worth noticing that the typing of the encoding requires 
a non tail recursive type of the form $\trec{t}{\btinp{\lhot{(S,\vart{t}}} \tinact}$
(see Def.~\ref{d:enc:hopitoho} for the precise formulation).

\smallskip 

\myparagraph{Other Encodings.}
We also give %the reverse of the previous encoding, namely 
an encoding of \HOp into \sessp. We rely on the well-known representability result of Sangiorgi~\cite{SangiorgiD:expmpa}. 
Since communicated processes may contain session names, in order to respect linearity and session protocols
the encoding enforces a distinction, depending on whether this kind of names is present in the communication object. If session names are present then a linear server trigger is deployed; otherwise, the replicated server  in~\cite{SangiorgiD:expmpa} can be used. 

As mentioned above, \HOp and \HO feature \emph{first-order} abstractions: 
only names can be used as arguments to abstractions.
Hence, given a (shared/linear) name $u$, in \HOp
we have the reduction $(\abs{x}{P}) \, u   \red  P \subst{u}{x}$.
We also consider \HOpp, an extension of \HOp with \emph{higher-order} abstractions.
 Thus, in \HOpp also an arbitrary value $V$ (possibly a process) can be an argument of an abstraction, 
 and one could have the reduction
 $(\abs{x}{P}) \, V   \red  P \subst{V}{x}$.
 We give an encoding of \HOpp into \HO: it  naturally extends that of \HOp into \HO;
see~\S\,\ref{subsec:hop}.

A well-known feature in process calculi is \emph{polyadicity}, i.e.,  
passing around tuples of values in communications. 
We consider the polyadic extension of \HOp, denoted \pHOp.
In \pHOp we have polyadicity in session communications and abstractions; 
polyadicity of shared names is ruled out by typing. 
This is enough for most purposes, including our encoding from \HOp into \HO.
In a session-typed setting, encoding polyadicity is straightforward, thanks to 
%polyadic arguments can be sent one by one, relying on 
the private character of 
(linear) session names --- see \S\,\ref{subsec:pho} for details.
%\[
%\begin{array}{rl}
%		\map{\binp{u}{x_1, \cdots, x_m} P}
%		 =  & \!\!\!\!
%		\binp{u}{x_1} \cdots ;  \binp{u}{x_m} \map{P}
%		\\
%%		\map{\bout{u}{u_1, \cdots, u_m} P}
%%		 =  & \!\!\!\!
%%		\bout{u}{u_1} \cdots ;  \bout{u}{u_m} \map{P}
%%		\\
%		\map{\bbout{u}{\abs{(x_1, \cdots, x_m)} Q} P}
%		= & \!\!\!\!
%		\bbout{u}{\abs{z}\binp{z}{x_1}\cdots ; \binp{z}{x_m} \map{Q}} \map{P}
%		\\ 
%		\map{\appl{x}{(u_1, \cdots, u_m)}}
%		= & \!\!\!\!
%		\newsp{s}{\appl{x}{s} \Par \bout{\dual{s}}{u_1} \cdots ; \bout{\dual{s}}{u_m} \inact} 
%	\end{array}
%\]
%Notice that encoding of polyadic abstraction/application requires an extra step, 
%in which a monadic abstraction is sent.

\smallskip

\myparagraph{A Non Encodability Result.}
We also show that shared names strictly add expressiveness to session calculi: that is,
there are (non deterministic) behaviours expressible with shared names not expressible using linear names only.
Although somewhat expected we do not know of a formal proof.
We propose such a formal proof, which relies crucially on the behavioural theory that we have introduced here
and on its determinacy properties. %\jp{EXPAND}.


\subsection{Tractable Bisimilarities for Session-Typed Processes}
\end{comment}

\label{subsec:intro:bisimulation}
\noi 
We outline our motivations and methods 
to show how session types are used for formulating 
two tractable bisimulations. 

\myparagraph{Overcoming Issues of Context Bisimilarity.}
%The characterisation of contextual congruence given by 
Context bisimilarity ($\wbc$, \defref{def:wbc}) is a too demanding relation on processes. 
%In the following we motivate our
%proposal for alternative, more tractable characterisations.  
%For the sake of clarity, and to emphasise the novelties of our approach, 
%we often omit type information. 
%Formal definitions including types are in \S\,\ref{sec:behavioural}.
To see the issue, we show 
the following clause for output.
Suppose $P \,\Re\, Q$, for some context bisimulation $\Re$. Then:

\smallskip 

\begin{enumerate}[$(\star)$]
	\item	Whenever 
		$P \by{\news{\tilde{m_1}} \bactout{n}{V}} P'$
		there exist
		$Q'$ and $W$
		such that 
		$Q \by{\news{\tilde{m_2}} \bactout{n}{W}} Q'$
		and, \emph{\textbf{for all} $R$}  with $\fv{R}=x$, 
		$\newsp{\tilde{m_1}}{P' \Par R\subst{V}{x}} \,\Re\, \newsp{\tilde{m_2}}{Q' \Par R\subst{W}{x}}$.
\end{enumerate}
\smallskip 
\noi 
Above, 
$\news{\tilde{m_1}} \bactout{n}{V}$ is the output label of 
value $V$ with extrusion of names in $\tilde{m_1}$.
To reduce the burden induced by 
universal quantification, we introduce \emph{higher-order}  and 
\emph{characteristic}  
bisimulations, two tractable equivalences denoted  $\hwb$ and $\fwb$, respectively.
As we work with an \emph{early} labelled transition system (LTS), 
%we shall also aim at limiting the input actions,  
%so to define a
%bisimulation relation for the output clause without observing
%infinitely many actions on the same input prefix. 
%To this end, 
%
we take the following two steps: 
%
\begin{enumerate}[(a)]
	\item We replace $(\star)$ with a clause involving a more tractable process closure.
	\item We refine the transition rule for input in the LTS,
	to avoid observing infinitely many actions on the same input prefix.
\end{enumerate}
%
\smallskip

\myparagraph{Trigger Processes with Session Communication.}
Concerning~(a), we exploit session types. 
We 
first 
observe that closure $R\subst{V}{x}$ 
in $(\star)$
is context bisimilar to the process:
\begin{equation}\label{equ:1}
	P = \newsp{s}{\appl{(\abs{z}{\binp{z}{x}{R}})}{s} \Par \bout{\dual{s}}{V} \inact}
\end{equation}
\noi 
%where $\binp{z}{x}{R}$ is an input and $\bout{\dual{s}}{V} \inact$
%is an output 
%on the endpoint $\dual{s}$ (the dual of $s$).
In fact,
we do have $P \wbc R\subst{V}{x}$, 
since 
application and reduction of dual endpoints 
%($s$ and $\dual{s}$) 
are deterministic.  
Now let us
consider process $T_{V}$ below, where $t$ is a fresh name:
\begin{equation}\label{equ:0}
T_{V} = \hotrigger{t}{V}
\end{equation}
%We call $\abs{z}{\binp{z}{x} R}$ a {\bf\em trigger value}. 
Let us write $P \by{\bactinp{n}{V}} P'$ to denote an input transition along $n$.
If $T_{V}$ inputs value $\abs{z}{\binp{z}{x} R}$ then
we can simulate the closure of $P$:
\begin{equation}\label{equ:2}
%\hotrigger{t}{V_1} 
T_{V}
\by{\bactinp{t}{\abs{z}{\binp{z}{x} R}}} P 
\wbc 
R\subst{V}{x}
\end{equation}
Processes such as $T_{V}$ 
offer a value at a fresh name; we will use this class of 
{\bf\em trigger processes} to define a
 refined bisimilarity without the demanding 
output clause $(\star)$. Given a fresh name $t$, 
we write $\htrigger{t}{V}$ to 
stand for a trigger process $T_{V}$ for value $V$.
We note that 
in contrast to previous approaches~\cite{SaWabook,JeffreyR05} 
our {trigger processes} do {\em not} use recursion or 
replication. This is crucial for preserving linearity of session names.  

\smallskip

%Then we can use 
%$\newsp{\tilde{m_1}}{P_1 \Par \htrigger{t}{V_1}}$ instead 
%of Clause 1) in Definition \ref{def:wbc} if we input 
%$\abs{z}{\binp{z}{x} R}$.   

\myparagraph{Characteristic Processes and Values.}
Concerning (b), we limit the possible 
input values (such as $\abs{z}{\binp{z}{x} R}$ above) %processes 
by exploiting session types.
The key concept is that of {\bf \emph{characteristic process/value}}
of a type,  
%The characteristic process of a session type $S$ is the process inhabiting $S$. 
the 
simplest term inhabiting that type (\defref{def:char}).
This way, e.g., let $S = \btinp{\shot{S_1}} \btout{S_2} \tinact$
be a session type: first
input an abstraction, %from values $S_1$ to processes, 
then output a value of type $S_2$.
Then, process $Q = \binp{u}{x} (\bout{u}{s_2} \inact \Par \appl{x}{s_1})$
is a characteristic process for $S$ 
\jpc{along name $u$.}
%Thus, characteristic processes follow the communication structures decreed by session types.
Given a session type $S$, we write $\mapchar{S}{u} $
for its characteristic process along name $u$
(cf.~\defref{def:char}).
Also, %Similarly, 
given value type $U$, we write 
$\omapchar{U}$ to denote its characteristic value.


We use the %characteristic %Precisely, we exploit  the
 characteristic value %$\lambda x.\mapchar{U}{x}$. %$\lambda x.\mapchar{U}{x}$. 
$\omapchar{U}$
 to limit input transitions.
Following the same reasoning as (\ref{equ:1})--(\ref{equ:2}), 
we can use an alternative trigger process, called
{\bf\em characteristic trigger process} with type 
$U$ to replace clause
% (1) in Definition~\ref{def:wbc}:
($\star$) in \defref{def:wbc}:
\begin{equation}
	\label{eq:4}
	\ftrigger{t}{V}{U} \defeq \fotrigger{t}{x}{s}{\btinp{U} \tinact}{V}
\end{equation}

%Note that if $U=L$, $\ftrigger{t}{V}{U}$ subsumes 
%$\htrigger{t}{V}$. 
\noi 
\jpc{Thus, in contrast to the trigger process~\eqref{equ:0}, the characteristic trigger process 
in~\eqref{eq:4}
does not involve a
higher-order communication on fresh name $t$.}
To refine the input transition system, we need to observe 
an additional value, 
$\abs{{x}}{\binp{t}{y} (\appl{y}{{x}})}$, 
called the {\bf\em trigger value}. 
This is necessary, because it turns out
that a characteristic value 
alone as the observable input 
is not enough to define a sound bisimulation.
Roughly speaking, the trigger value is used
to observe/simulate application processes.
%to {\em count} the number of free higher-order variables inside 
%the receiver. 
%\jpc{See Example~\ref{ex:motivation} for further details.}

\smallskip 
\myparagraph{Refined Input Transition Rule.}
Based on 
the above discussion, we refine 
the (early) transition rule for input actions. 
%We write $P \by{\bactinp{n}{V}} P'$ for the input transition along $n$.
The transition rule for input roughly becomes 
(see \defref{def:rlts} for details):
\[
		\tree {
%\begin{array}{c}
P \by{\bactinp{n}{V}} P' \quad  V = m \vee V \scong
(\abs{{x}}{\binp{t}{y} (\appl{y}{{x}})}
 \vee  \omapchar{U})  \textrm{ with } t \textrm{ fresh} 
		}{
			P' \hby{\bactinp{n}{V}} P'
		}
\]
Note the distinction between standard and refined transitions: $\by{\bactinp{n}{V}}$ vs. $\hby{\bactinp{n}{V}}$.
Using this rule, we define an alternative  LTS
with refined 
\jpc{(higher-order)}
input. %; all other rules are kept unchanged.
This refined LTS is used for 
both higher-order ($\hwb$) and characteristic ($\fwb$) bisimulations (Defs.~\ref{d:hbw} and~\ref{d:fwb}),
in which the demanding clause~$(\star)$ is replaced with 
more tractable clauses based on trigger processes 
\jpc{(cf.~\eqref{equ:0})} 
and characteristic 
trigger processes
\jpc{(cf.~\eqref{eq:4})},
respectively.
We show that $\hwb$ is useful for \HOp and \HO, and that~$\fwb$ 
can be uniformly used in all subcalculi, including \sessp. 

\NY{Our calculus lacks name matching, 
which is usually crucial to prove completeness of bisimilarity.
Instead of matching,  we use types:
a process trigger embeds a name into a characteristic
process so to observe its session behaviour.}

%\dk{We stress-out that our calculus lacks
%a matching construct
%which is usually a crucial element to prove completeness of bisimilarity.
%Nevertheless, we use types
%%and specifically the characteristic process
%to compensate;
%%for the absence of matching, i.e.~
%instead of name matching, a process trigger embeds a name into a characteristic
%process so to observe its session behavior.}
%Notice that while Definition \ref{d:hbw} is useful for 
%\HOp and its higher-order variants,
%Definition \ref{d:fwb} is useful for first-order sub-calculi of \HOp.




%%\myparagraph{Outline}
%\subsection{Outline}
%\noi \S\,\ref{sec:calculus} presents the calculi; 
%\S\,\ref{sec:types} presents types;
%the tractable bisimulations are in \S\,\ref{sec:behavioural};
%the notion of encoding is in \S\,\ref{s:expr};
%\S\,\ref{sec:positive} and \S\,\ref{sec:negative}
%present positive and negative encodability results, resp;
%\S\,\ref{sec:extension} discusses extensions; and 
%\S\,\ref{sec:relwork} concludes with related work;
%Appendix summarises the typing system. 
%The paper is self-contained. 
%{\bf\em Omitted definitions, additional related work and full proofs can be found 
%in a technical report, available from \cite{KouzapasPY15}.} 


\section{A Higher-Order Session $\pi$-Calculus}
\label{sec:calculus}
% !TEX root = main.tex
\noindent 
We introduce the 
\emph{Higher-Order Session $\pi$-Calculus} (\HOp).
\HOp includes both name- and abstraction-passing 
as well as recursion; it is a subcalculus 
of the language
studied 
in~\cite{tlca07}. 
Following the literature~\cite{JeffreyR05},
for simplicity of the presentation
we concentrate on the second-order call-by-value \HOp.  
(In \secref{sec:extension} we consider extensions of 
\HOp with higher-order abstractions 
and polyadicity in name-passing/abstractions.)
%We also introduce two subcalculi of \HOp. In particular, we define the 
%core higher-order session
%calculus (\HO), which 
%%. The \HO calculus is  minimal: it 
%includes constructs for shared name synchronisation and 
%%constructs for session establish\-ment/communication and 
%(monadic) name-abstraction, but lacks name-passing and recursion.

%Although minimal, in \secref{s:expr}
%the abstraction-passing capabilities of \HOp will prove 
%expressive enough to capture key features of session communication, 
%such as delegation and recursion.

\subsection{Syntax of \HOp}
\label{subsec:syntax}
\noindent\myparagraph{Values} The syntax of \HOp is defined in \figref{fig:syntax}
\begin{figure}[t]
\[ 
\begin{array}{lll}
u,w  ::=  n \ | \ x,y,z
& n ::= a,b  \ | \ s, \dual{s} 
& V,W  ::=   \nonhosyntax{u} \ | \ \abs{x}{P}
\end{array}
\]
\[
\begin{array}{rclllll}
P,Q \!\!\!\!\!\! & ::= & \!\! \bout{u}{V}{P}  \bnfbar  \binp{u}{x}{P} \bnfbar
 \bsel{u}{l} P \bnfbar \bbra{u}{l_i:P_i}_{i \in I}   \\[1mm]
 & \bnfbar & \!\!  \nonhosyntax{\rvar{X} \bnfbar \recp{X}{P}} \bnfbar \appl{V}{u} \bnfbar P\Par Q \bnfbar \news{n} P 
\bnfbar 
%ny
%\appl{x}{u}

 \inact
%\\[1mm]
 %    & \bnfbar & \nonhosyntax{\rvar{X} \bnfbar \recp{X}{P}}
\end{array}
\]
 \caption{Syntax of \HOp (\HO lacks the constructs in \nonhosyntax{\text{grey}}).}
\label{fig:syntax}
\Hlinefig
\end{figure}
We use $a,b,c, \dots$ (resp.~$s, \dual{s}, \dots$) 
to range over shared (resp. session) names. 
We use $m, n, t, \dots$ for session or shared names. 
We define the dual operation over names $n$ as $\dual{n}$ with
$\dual{\dual{s}} = s$ and $\dual{a} = a$.
Intuitively, names $s$ and $\dual{s}$ are dual (two) \emph{endpoints} while 
shared names represent shared (non-deterministic) points. 
Variables are denoted with $x, y, z, \dots$, 
and recursive variables are denoted with $\varp{X}, \varp{Y} \dots$.
An abstraction %(or higher-order value) 
$\abs{x}{P}$ is a process $P$ with name parameter $x$.
%Symbols $u, v, \dots$ range over identifiers; and  $V, W, \dots$ to denote values. 
Values $V,W$ include 
identifiers $u, v, \ldots$ %(first-order values) 
and 
abstractions $\abs{x}{P}$ (first- and higher-order values, resp.). 

\myparagraph{Terms} 
include the
$\pi$-calculus prefixes for sending and receiving values $V$.
Process $\bout{u}{V} P$ denotes the output of value $V$
over name $u$, with continuation $P$;
process $\binp{u}{x} P$ denotes the input prefix on name $u$ of a value
that 
will substitute variable $x$ in continuation $P$. 
Recursion is expressed by $\recp{X}{P}$,
which binds the recursive variable $\varp{X}$ in process $P$.
Process 
%ny
%$\appl{x}{u}$ 
$\appl{V}{u}$ 
is the application
which substitutes name $u$ on the abstraction~$V$. 
\dk{Typing  ensures \jpc{that} $V$ is not a name.}
Prefix $\bsel{u}{l} P$ selects label $l$ on name $u$ and then behaves as $P$.
%Given $i \in I$ 
Process $\bbra{u}{l_i: P_i}_{i \in I}$ offers a choice on labels $l_i$ with
continuation $P_i$, given that $i \in I$.
%Others are standard c
Constructs for 
inaction $\inact$,  parallel composition $P_1 \Par P_2$, and 
name restriction $\news{n} P$ are standard.
Session name restriction $\news{s} P$ simultaneously binds endpoints $s$ and $\dual{s}$ in $P$.
%A well-formed process relies on assumptions for
%guarded recursive processes.
We use $\fv{P}$ and $\fn{P}$ to denote a set of free 
%\jpc{recursion}
variables and names; 
and assume $V$ in $\bout{u}{V}{P}$ does not include free recursive 
variables $\rvar{X}$. 
If $\fv{P} = \emptyset$, we call $P$ {\em closed}.
%; and closed $P$ without 
%free session names a {\em program}. 

\subsection{Subcalculi of \HOp}
\label{subsec:subcalculi}
\noi
We define two subcalculi of \HOp. 
%We now define several sub-calculi of \HOp. 
The first is the 
{\em core higher-order session calculus} (denoted \HO),
which lacks recursion and name passing; its 
formal syntax is obtained from \figref{fig:syntax} by excluding 
constructs in \nonhosyntax{\text{grey}}.
The second subcalculus is 
the {\em session $\pi$-calculus} 
(denoted $\sessp$), which 
lacks  the
higher-order constructs
(i.e., abstraction passing and application), but includes recursion.
Let $\CAL \in \{\HOp, \HO, \sessp\}$. We write 
$\CAL^{-\mathsf{sh}}$ for $\CAL$ without shared names
(we delete $a,b$ from $n$). 
We shall demonstrate that 
$\HOp$, $\HO$, and $\sessp$ have the same expressivity.

\subsection{Operational Semantics}
\label{subsec:semantics}
\begin{figure}
\[
\begin{array}{rclrcrclr}
(\abs{x}{P}) \, u  & \red & P \subst{u}{x} 
& \orule{App}
		\\[1mm]
%\bout{a}{V} P \Par \binp{a}{x} Q & \red & P \Par Q \subst{V}{x} 
%& \orule{Com}
%		\\[1mm]
\bout{n}{V} P \Par \binp{\dual{n}}{x} Q & \red & P \Par Q \subst{V}{x} 
& \orule{Pass}
		\\[1mm]
		\bsel{n}{l_j} Q \Par \bbra{\dual{n}}{l_i : P_i}_{i \in I} & \red & Q \Par P_j ~~(j \in I)~~  & \orule{Sel}\\[1mm]
		P \red P' & \Rightarrow & \news{n} P  \red  \news{n} P'  & \orule{Res}\\[1mm]
			P \red P' & \Rightarrow  &  P \Par Q  \red   P' \Par Q  & \orule{Par}\\[1mm]
			P \scong Q \red Q' \scong P' & \Rightarrow & P  \red  P' & \orule{Cong}
	\end{array}
\]
{\small
\[
	\begin{array}{c}
		P \Par \inact \scong P
		\quad
		P_1 \Par P_2 \scong P_2 \Par P_1
		\quad
		P_1 \Par (P_2 \Par P_3) \scong (P_1 \Par P_2) \Par P_3
		%\quad
		%P \scong Q \textrm{ if } P \scong_\alpha Q
		\\[1mm]
		\news{n} \inact \scong \inact
		\quad 
		P \Par \news{n} Q \scong \news{n}(P \Par Q)
		\	(n \notin \fn{P})\quad 
		\recp{X}{P} \scong P\subst{\recp{X}{P}}{\rvar{X}}
		\\[1mm]
		P \scong Q \textrm{ if } P \scong_\alpha Q
%		\qquad
%		\dk{V \scong W \textrm{ if } V \scong_\alpha W
%\quad \abs{x}{P} \scong \abs{x}{Q} \textrm{ if } P \scong Q}
	\end{array}
\]
}
\caption{Operational Semantics of $\HOp$. 
\label{fig:reduction}}
\Hlinefig
\end{figure}
\noindent \figref{fig:reduction} defines the operational semantics 
of \HOp.
$\orule{App}$ is a name application; 
$\orule{Pass}$ defines a shared interaction at $n$ 
(\jpc{with} $\dual{n}=n$) or a session interaction;  
$\orule{Sel}$ is the standard rule for labelled choice/selection:
given an index set $I$, 
a process selects label $l_j$ on name $n$ over a set of
labels $\set{l_i}_{i \in I}$ offered by a branching 
on the dual endpoint $\dual{n}$; and other rules are standard.
Rules for \emph{structural congruence} are defined in \figref{fig:reduction} (bottom). 
\jpc{We assume the expected extension of $\scong$ to values $V$.}
We write $\red^\ast$ for a multi-step reduction. 


%\section{Example: Processes}
%\newcommand{\rtype}{\mathsf{roomtype}}
\newcommand{\Quote}{\mathsf{quote}}
\newcommand{\accept}{\mathsf{accept}}
\newcommand{\reject}{\mathsf{reject}}
\newcommand{\creditc}{\mathsf{credit}}

\newcommand{\Client}{\mathsf{Client}}

\begin{example}[Program Equivalence]
\label{ex:hotelbooking}
We introduce the usecase scenario where a Client wants to book
a hotel for her holidays in a remote island
(cf.~\cite{MostrousY15} for a similar usecase).
The Client narrows down her choice to two hotels.
It then requires a quote from the two hotels in order to
make her choice. The Round Trip Time required for
taking quotes from the two hotels in not optimal (cf.~\cite{MostrousY15}),
so she decides to send remote codes to the hotels
to automatically negotiate and book the hotel for
her:
%
\[
	\begin{array}{rcl}
		P &\defeq& \bout{x}{\rtype} \binp{x}{\Quote} \bout{y}{\Quote}
		y \triangleleft \left\{
				\begin{array}{l}
					\accept: \bsel{x}{\accept} \bout{x}{\creditc} \inact,\\
					\reject: \bsel{x}{\reject} \inact
				\end{array}
				\right\}
		\\[6mm]
		\Client_1 &\defeq& \newsp{h_1, h_2}{\bout{s_1}{\abs{x}{P \subst{h_1}{y}}} \bout{s_2}{\abs{x}{P \subst{h_2}{y}}} \inact \Par\\
			&&
			\begin{array}{lll}
				\binp{\dual{h_1}}{x} \binp{\dual{h_2}}{y} & \If\ x \leq y\ \Then & \bsel{\dual{h_1}}{\accept} \bsel{\dual{h_2}}{\reject} \inact\\
				& \Else& \bsel{\dual{h_1}}{\reject} \bsel{\dual{h_2}}{\accept} \inact
			\end{array}
		}
	\end{array}
\]
%
\begin{itemize}
	\item	Process $P$ is the remote code responsible for negotiation with a hotel.
		Channel $x$ is intended to be instatiated by the hotel as the negotiating
		endpoint. Channel $y$ is used to interact with $\Client_1$.

	\item	Process $P$
		i) sends the room requirements to the hotel;
		ii) receives a quote from the hotel;
		iii) sends the quote to the $\Client_1$;
		iv) expects a choice from the $\Client_1$ whether to accept or reject the offer and;
		v) If the choice is accept it informs the hotel and performs the booking,
		if the choice is reject it informs the hotel and ends the session.

	\item	$\Client_1$ instantiates two copies of process $P$ as abstractions
		on session $x$. It further uses two
		fresh endpoints $h_1, h_2$ to substitute channels $y$, respectively,
		in order for the two instances of $P$ to be able to interact
		with $\Client_1$.
	
	\item	$\Client_1$ then sends the two abstractions instances of $P$
		to the two hotels via sessions $s_1$ and $s_2$, respectively.

	\item	$\Client_1$ uses the dual endpoints $\dual{h_1}$ and $\dual{h_2}$
		to receive the negotiation
		result from the two remote instances of $P$ and then inform the two
		processes for the final booking decision.
\end{itemize}

The above scenario does not add a significant gain
to the time needed for the entire protocol to take
place, since the two remote processes are required
to send and receive data to $\Client_1$.

As an alternative we can propose a different implementation
of the same scenario that requires from the two
remote processes to interact with each other,
instead of $\Client_1$, to reach to a consensus:
%
\[
	\begin{array}{rcl}
		P_1 &\defeq&	\bout{x}{\rtype} \binp{x}{\Quote_1} \bout{y}{\Quote_1} \binp{y}{\Quote_2}\\
			&&
				\begin{array}{ll}
					\If\ \Quote_1 \leq \Quote_2\ \Then & \bsel{x}{\accept} \bout{x}{\creditc} \inact\\
					\Else & \bsel{x}{\reject} \inact
				\end{array}
		\\
		P_2 &\defeq&	\bout{x}{\rtype} \binp{x}{\Quote_2} \bout{y}{\Quote_1} \bout{y}{\Quote_1}\\
			&&
				\begin{array}{ll}
					\If\ \Quote_1 \leq \Quote_2\ \Then & \bsel{x}{\accept} \bout{x}{\creditc} \inact\\
					\Else & \bsel{x}{\reject} \inact
				\end{array}
		\\
		\Client_2 &\defeq& \newsp{h}{\bout{s_1}{\abs{x}{P_1 \subst{h}{y}}} \bout{s_2}{\abs{x}{P_2 \subst{\dual{h}}{y}}} \inact}
	\end{array}
\]
\end{example}

\begin{itemize}
	\item	Processes $P_1$ and $P_2$ are responsible for negotiating a quote from the
		hotel in the same fashion as process $P$ in the previous implementation.

	\item	The difference with process $P$ is that the channel $y$ is used for
		interaction between process $P_1$ and $P_2$. Both processes send
		there quotes to each other and then internally follow the same
		logic to reach to a decision.

	\item	The role of $\Client_2$ is to instantiate $P_1$ and $P_2$ as abstractions
		on name $x$. It further substitutes
		the two endpoints of a fresh channel $h$ to channels $y$ respectively,
		in order for the two instances to be able to communicate with each other.

	\item	Process $\Client_2$ then uses sessions $s_1$ and $s_2$ to send the two
		instances of $P_1$ and $P_2$ to the two hotels.
\end{itemize}

We can show that process $\Client_1$ and $\Client_2$
are interchangeable by showing that they are bisimilar.
We use characteristic bisimulation to give a bisimulation
closure.
%
%\begin{eqnarray*}
%	S &=& \set{
%		\Client_1, \Client_2\\
%	&&	\newsp{h_1, h_2}{\bout{s_1}{\abs{x}{P \subst{h_1}{y}}} \bout{s_2}{\abs{x}{P \subst{h_2}{y}}} \inact \Par\\
%	&&
%		\begin{array}{lll}
%			\binp{\dual{h_1}}{x} \binp{\dual{h_2}}{y} & \If\ x \leq y\ \Then & \bsel{\dual{h_1}}{\accept} \bsel{\dual{h_2}}{\reject} \inact\\
%			& \Else& \bsel{\dual{h_1}}{\reject} \bsel{\dual{h_2}}{\accept} \inact
%		\end{array}}
%		}
%\end{eqnarray*}

\subsection{Types}

Assuming $S = \btout{\Quote} \btbra{\accept: \tinact, \reject: \tinact}$ and
$U = \btout{\rtype} \btinp{\Quote} \btsel{\accept: \btout{\creditc} \tinact, \reject: \tinact }$.
The type for $\abs{x}{P}$ is
%
\[
	\es; \es; y: S \proves \abs{x}{P} \hastype \lhot{U}
\]
%
\noi and the type for $\Client_1$ is
%
\[
	\es; \es; s_1: \btout{\lhot{U}} \tinact \cat s_2: \btout{\lhot{U}} \tinact \proves \Client_1 \hastype \Proc
\]
%
The types for $P_1$ and $P_2$ are
%
\begin{eqnarray*}
	\es; \es; y: \btout{\Quote} \btinp{\Quote} \tinact \proves \abs{x}{P_1} \hastype \lhot{U}\\
	\es; \es; y: \btinp{\Quote} \btout{\Quote} \tinact \proves \abs{x}{P_2} \hastype \lhot{U}
\end{eqnarray*}
%
\noi ant the type for $\Client_2$ is
%
\[
	\es; \es; s_1: \btout{\lhot{U}} \tinact \cat s_2: \btout{\lhot{U}} \tinact \proves \Client_2 \hastype \Proc
\]
%

%
\newcommand{\Hotel}{\mathsf{Hotel}}
\newcommand{\Code}{\mathsf{Code}}

%\makeatletter
%\newcommand{\gettikzxy}[3]{%
%  \tikz@scan@one@point\pgfutil@firstofone#1\relax
%  \edef#2{\the\pgf@x}%
%  \edef#3{\the\pgf@y}%
%}
%\makeatother
%	\gettikzxy{(Client1.south)}{\ax}{\ay}
%	\draw[dashed]			(Client1.south) -- (\ax, 0);


%\begin{center}

\begin{tikzpicture}
%	\draw[help lines]		(0, 0) grid (13, 10);

	%%%%%%%%%%%%%%%%%%%%% Scenario 1

	%%%% Nodes
	\node	(Client1)	at	(0, 10) {\footnotesize $\Client_1$};
	\node	(Hotel1)	at	(2.5, 10) {\footnotesize $\Hotel_1$};
	\node	(Hotel2)	at	(5, 10) {\footnotesize $\Hotel_2$};

	\node	(Code1)		at	(1.25, 8.8) {\footnotesize $\Code_1$};
	\node	(Code2)		at	(3.75, 8.8) {\footnotesize $\Code_2$};

	%%%% Lines for Nodes
%	\draw[dashed]		(Client1.south west) -- (Client1.south east);
	\draw
		let	\p1 = (Client1.south)
		in
			(\x1, \y1) -- (\x1, 2);

%	\draw[dashed]		(Hotel1.south west) -- (Hotel1.south east);
	\draw
		let
			\p1 = (Hotel1.south)
		in
			(\x1, \y1) -- (\x1, 2);


%	\draw[dashed]		(Hotel2.south west) -- (Hotel2.south east);
	\draw
		let
			\p1 = (Hotel2.south)
		in
			(\x1, \y1) -- (\x1, 2);


%	\draw[dashed]		(Code1.south west) -- (Code1.south east);
	\draw
		let
			\p1 = (Code1.south)
		in
			(\x1, \y1) -- (\x1, 2);

%	\draw[dashed]		(Code2.south west) -- (Code2.south east);
	\draw
		let
			\p1 = (Code2.south)
		in
			(\x1, \y1) -- (\x1, 2);


	%%%% Arrows
	\draw[->]
		let
			\p1 = (Client1),
			\p2 = (Hotel1)
		in
			(\x1, 9.6) to node[above] {\scriptsize $\mathsf{Code1}$} (\x2, 9.6);

	\draw[->]
		let
			\p1 = (Client1),
			\p2 = (Hotel2)
		in
			(\x1, 9.2) to node[above] {\scriptsize $\mathsf{Code1}$} (\x2, 9.2);

	\draw[->]
		let
			\p1 = (Hotel1)
		in
			(\x1, 9.6) -- (Code1.north);

	\draw[->]
		let
			\p1 = (Hotel2)
		in
			(\x1, 9.2) -- (Code2.north);


	\draw[->]
		let
			\p1 = (Code1),
			\p2 = (Hotel1)
		in
			(\x1, 8.4) to node[above] {\scriptsize $\rtype$} (\x2, 8.4);

	\draw[->]
		let
			\p1 = (Code1),
			\p2 = (Hotel1)
		in
			(\x2, 8) to node[above] {\scriptsize $\Quote$} (\x1, 8);

	\draw[->]
		let
			\p1 = (Code2),
			\p2 = (Hotel2)
		in
			(\x1, 8.4) to node[above] {\scriptsize $\rtype$} (\x2, 8.4);
	\draw[->]
		let
			\p1 = (Code2),
			\p2 = (Hotel2)
		in
			(\x2, 8) to node[above] {\scriptsize $\Quote$} (\x1, 8);

	\draw[->]
		let
			\p1 = (Code1),
			\p2 = (Client1)
		in
			(\x1, 7.6) to node[above] {\scriptsize $\Quote$} (\x2, 7.6);

	\draw[->]
		let
			\p1 = (Code2),
			\p2 = (Client1)
		in
			(\x1, 7.2) to node[above] {\scriptsize $\Quote$} (\x2, 7.2);


	%%%% Choice 
	%% Client1 --> Code1
	\draw[dashed]
		let
			\p1 = (Client1),
			\p2 = (Code1)
		in
			(\x1, 6.8) -- (\x2, 6.8)
			(\x1, 4.4) -- (\x2, 4.4)
			(\x1, 2.4) -- (\x2, 2.4);

	\draw[->]
		let
			\p1 = (Client1),
			\p2 = (Code1)
		in
			(\x1, 6.4) to node[above] {\scriptsize $\accept$} (\x2, 6.4);

	\draw[dashed]
		let
			\p1 = (Code1),
			\p2 = (Hotel1)
		in
			(\x1, 6) -- (\x2, 6)
			(\x1, 4.8) -- (\x2, 4.8)
			(\x1, 3.6) -- (\x2, 3.6)
			(\x1, 2.8) -- (\x2, 2.8);

	\draw[->]
		let
			\p1 = (Code1),
			\p2 = (Hotel1)
		in
			(\x1, 5.6) to node[above] {\scriptsize $\accept$} (\x2, 5.6);

	\draw[->]
		let
			\p1 = (Code1),
			\p2 = (Hotel1)
		in
			(\x1, 5.2) to node[above] {\scriptsize $\creditc$} (\x2, 5.2);

	\draw[->]
		let
			\p1 = (Client1),
			\p2 = (Code1)
		in
			(\x1, 4) to node[above] {\scriptsize $\reject$} (\x2, 4);

	\draw[->]
		let
			\p1 = (Code1),
			\p2 = (Hotel1)
		in
			(\x1, 3.2) to node[above] {\scriptsize $\reject$} (\x2, 3.2);


	%%%% Choice 
	%% Client1 --> Code1
	\draw[dashed]
		let
			\p1 = (Client1),
			\p2 = (Code2)
		in
			(\x1, 6.8) -- (\x2, 6.8)
			(\x1, 4.4) -- (\x2, 4.4)
			(\x1, 2.4) -- (\x2, 2.4);

	\draw[->]
		let
			\p1 = (Client1),
			\p2 = (Code2)
		in
			(\x1, 6.2) to node[above] {\scriptsize $\accept$} (\x2, 6.2);

	\draw[dashed]
		let
			\p1 = (Code2),
			\p2 = (Hotel2)
		in
			(\x1, 6) -- (\x2, 6)
			(\x1, 4.8) -- (\x2, 4.8)
			(\x1, 3.6) -- (\x2, 3.6)
			(\x1, 2.8) -- (\x2, 2.8);

	\draw[->]
		let
			\p1 = (Code2),
			\p2 = (Hotel2)
		in
			(\x1, 5.6) to node[above] {\scriptsize $\accept$} (\x2, 5.6);

	\draw[->]
		let
			\p1 = (Code2),
			\p2 = (Hotel2)
		in
			(\x1, 5.2) to node[above] {\scriptsize $\creditc$} (\x2, 5.2);

	\draw[->]
		let
			\p1 = (Client1),
			\p2 = (Code2)
		in
			(\x1, 3.8) to node[above] {\scriptsize $\reject$} (\x2, 3.8);

	\draw[->]
		let
			\p1 = (Code2),
			\p2 = (Hotel2)
		in
			(\x1, 3.2) to node[above] {\scriptsize $\reject$} (\x2, 3.2);


	%%%%%%%%%%%%%%%%%%%%% Scenario 2

	%%%% Nodes
	\node	(Client1)	at	(7.5, 10) {\footnotesize $\Client_1$};
	\node	(Hotel1)	at	(10, 10) {\footnotesize $\Hotel_1$};
	\node	(Hotel2)	at	(12.5, 10) {\footnotesize $\Hotel_2$};

	\node	(Code1)		at	(8.75, 8.8) {\footnotesize $\Code_1$};
	\node	(Code2)		at	(11.25, 8.8) {\footnotesize $\Code_2$};

	%%%% Lines for Nodes
%	\draw[dashed]		(Client1.south west) -- (Client1.south east);
	\draw
		let	\p1 = (Client1.south)
		in
			(\x1, \y1) -- (\x1, 4);

%	\draw[dashed]		(Hotel1.south west) -- (Hotel1.south east);
	\draw
		let
			\p1 = (Hotel1.south)
		in
			(\x1, \y1) -- (\x1, 4);


%	\draw[dashed]		(Hotel2.south west) -- (Hotel2.south east);
	\draw
		let
			\p1 = (Hotel2.south)
		in
			(\x1, \y1) -- (\x1, 4);


%	\draw[dashed]		(Code1.south west) -- (Code1.south east);
	\draw
		let
			\p1 = (Code1.south)
		in
			(\x1, \y1) -- (\x1, 4);

%	\draw[dashed]		(Code2.south west) -- (Code2.south east);
	\draw
		let
			\p1 = (Code2.south)
		in
			(\x1, \y1) -- (\x1, 4);


	%%%% Arrows
	\draw[->]
		let
			\p1 = (Client1),
			\p2 = (Hotel1)
		in
			(\x1, 9.6) to node[above] {\scriptsize $\mathsf{Code1}$} (\x2, 9.6);

	\draw[->]
		let
			\p1 = (Client1),
			\p2 = (Hotel2)
		in
			(\x1, 9.2) to node[above] {\scriptsize $\mathsf{Code1}$} (\x2, 9.2);

	\draw[->]
		let
			\p1 = (Hotel1)
		in
			(\x1, 9.6) -- (Code1.north);

	\draw[->]
		let
			\p1 = (Hotel2)
		in
			(\x1, 9.2) -- (Code2.north);


	\draw[->]
		let
			\p1 = (Code1),
			\p2 = (Hotel1)
		in
			(\x1, 8.4) to node[above] {\scriptsize $\rtype$} (\x2, 8.4);

	\draw[->]
		let
			\p1 = (Code1),
			\p2 = (Hotel1)
		in
			(\x2, 8) to node[above] {\scriptsize $\Quote$} (\x1, 8);

	\draw[->]
		let
			\p1 = (Code2),
			\p2 = (Hotel2)
		in
			(\x1, 8.4) to node[above] {\scriptsize $\rtype$} (\x2, 8.4);
	\draw[->]
		let
			\p1 = (Code2),
			\p2 = (Hotel2)
		in
			(\x2, 8) to node[above] {\scriptsize $\Quote$} (\x1, 8);

	\draw[->]
		let
			\p1 = (Code1),
			\p2 = (Code2)
		in
			(\x1, 7.6) to node[above] {\scriptsize $\Quote$} (\x2, 7.6);

	\draw[->]
		let
			\p1 = (Code2),
			\p2 = (Code1)
		in
			(\x1, 7.2) to node[above] {\scriptsize $\Quote$} (\x2, 7.2);


	%%%% Choice
	% Client1 --> Hotel1
	\draw[dashed]
		let
			\p1 = (Code1),
			\p2 = (Hotel1)
		in
			(\x1, 6.8) -- (\x2, 6.8)
			(\x1, 5.6) -- (\x2, 5.6)
			(\x1, 4.8) -- (\x2, 4.8);

	\draw[->]
		let
			\p1 = (Code1),
			\p2 = (Hotel1)
		in
			(\x1, 6.4) to node[above] {\scriptsize $\accept$} (\x2, 6.4);

	\draw[->]
		let
			\p1 = (Code1),
			\p2 = (Hotel1)
		in
			(\x1, 6) to node[above] {\scriptsize $\creditc$} (\x2, 6);

	\draw[->]
		let
			\p1 = (Code1),
			\p2 = (Hotel1)
		in
			(\x1, 5.2) to node[above] {\scriptsize $\reject$} (\x2, 5.2);

	% Client2 --> Hotel2
	\draw[dashed]
		let
			\p1 = (Code2),
			\p2 = (Hotel2)
		in
			(\x1, 6.8) -- (\x2, 6.8)
			(\x1, 5.6) -- (\x2, 5.6)
			(\x1, 4.8) -- (\x2, 4.8);

	\draw[->]
		let
			\p1 = (Code2),
			\p2 = (Hotel2)
		in
			(\x1, 6.4) to node[above] {\scriptsize $\accept$} (\x2, 6.4);

	\draw[->]
		let
			\p1 = (Code2),
			\p2 = (Hotel2)
		in
			(\x1, 6) to node[above] {\scriptsize $\creditc$} (\x2, 6);

	\draw[->]
		let
			\p1 = (Code2),
			\p2 = (Hotel2)
		in
			(\x1, 5.2) to node[above] {\scriptsize $\reject$} (\x2, 5.2);

\end{tikzpicture}
%\end{center}


\section{Types and Typing}
\label{sec:types}
\section{Types for $\HOp$}

We define a session type system for $\HOp$, which is based on the type system developed by Mostrous
and Yoshida in~\cite{tlca07}.

\subsection{Session Types}
We consider a minimal type structure, a fragment of that defined in~\cite{tlca07}.
The only (but fundamental) differences are in the types for values: we focus on having 
$\shot{S}$ and $\lhot{S}$, whereas the structure in~\cite{tlca07} supports general functions $U \sharedop T$ and 
$U \lollipop T$.
\[
	\begin{array}{lcl}
		\text{\emph{Values}} & U ::= & S \bnfbar \lhot{S} \bnfbar \shot{S} \bnfbar \chtype{S} \qquad \quad \text{\emph{Terms}} \quad T ::= U  \bnfbar  \Proc\\
		\text{\emph{Sessions}} \ & S ::= &  \btout{U} S \bnfbar \btinp{U} S
		\bnfbar		\btsel{l_i:S_i}_{i \in I} \bnfbar \btbra{l_i:S_i}_{i \in I} \bnfbar \trec{t}{S} \bnfbar \vart{t}  \bnfbar \tinact 
	\end{array}
\]

There are four different value types $U$; session value $S$, linear higher order value $\lhot{S}$, 
shared higher order value $\shot{S}$; shared channel $\chtype{S}$. Terms can either have a
value type $U$ or a process type $\Proc$.

Session types follow the standard binary session types syntax \cite{}. Session send prefix $\btout{U} S$ 
denotes a session type that sends a value of type $U$ and continues as $S$. Dually receive prefix $\btinp{U} S$
denotes a session type that receives a value of type $U$ and continues as $S$. 
Set $\mathsf{ST}$ is the space of all session types.

\begin{definition}[Duality]
	Let function $F(R): \mathsf{ST} \longrightarrow \mathsf{ST}$ to be defined as:

	\begin{tabular}{rcl}
		$F(R)$ &$=$&		$\set{(\tinact, \tinact), (\vart{t}, \vart{t})}$\\
			&$\cup$&	$\set{(\btout{U};S_1, \btinp{U}; S_2), (\btinp{U};S_1, \btout{U}; S_2) \bnfbar S_1\ R\ S_2}$\\
			&$\cup$&	$\set{(\btsel{l_i: S_i}_{i \in I}, \btbra{l_j: S_j'}_{j \in J}) \bnfbar I \subseteq J, S_i\ R\ S_j'}$\\
			&$\cup$&	$\set{(\btbra{l_i: S_i}_{i \in I}, \btsel{l_j: S_j'}_{j \in J}) \bnfbar J \subseteq I, S_j\ R\ S_i'}$\\
			&$\cup$&	$\set{(\trec{t}{S_1}, \trec{t}{S_2}) \cup (S_1 \subst{\trec{t}{S}}{\vart{t}}, S_2), (S_1, S_2\subst{\trec{t}{S}}{\vart{t}}) \bnfbar S_1\ R\ S_2)}$
	\end{tabular}

	Standard arguments ensure that $F$ is monotone, thus the greatest fix point
	of $F$ exists and let duality defined as $\dualof = \nu X. F(X)$.
\end{definition}


Following our decision of focusing on functions $\shot{S}$ and $\lhot{S}$,
our environments are also simpler than those in~\cite{tlca07}:
\[
	\begin{array}{lcl}
		\text{Shared} & \Gamma \bnfis & \emptyset \bnfbar \Gamma \cat \varp{X}: \shot{S} \bnfbar \Gamma \cat k: \chtype{S} \bnfbar \Gamma \cat \rvar{X}: \Sigma \\
		\text{Linear} & \Lambda \bnfis & \emptyset \bnfbar \Lambda \cat \varp{X}: \lhot{S}  \\
		\text{Session} \quad & \Sigma \bnfis & \emptyset \bnfbar \Sigma \cat k:S  
	\end{array}
\]


With these environments the shape of judgments is exactly the same as in Mostrous and Yoshida's system:
\[
	\begin{array}{c}
		\Gamma; \Lambda ; \Sigma \proves P \hastype T
%		\Gamma; \Lambda; \Sigma \proves V \hastype U
	\end{array}
\]
As expected, weakening, contraction, and exchange principles apply to $\Gamma$;
environments $\Lambda$ and $\Sigma$ behave linearly, and are only subject to exchange.
We require that the domains of $\Gamma, \Lambda$ and $\Sigma$ are pairwise distinct.
%We focus on \emph{well-formed} judgments, which do not share elements in their domains.
%\newcommand{\jrule}[3]{\displaystyle \frac{#1 }{#2} & \trule{#3}}
\newcommand{\jrule}[3]{\displaystyle \trule{#3}~~\frac{#1 }{#2}}
\newcommand{\addenv}{\circ}

\begin{figure}[!t]
\[
	\begin{array}{c}
%		\jrule{ }{\Gamma ; \emptyset; \emptyset \vdash \UnitV \hastype \Unit}{Unit} 
%		\qquad\quad  
		\trule{Session}~~\Gamma; \emptyset; \set{k:S} \proves k \hastype S 
		\\[2mm]
		\trule{Shared}~~\Gamma \cat a : \chtype{S}; \emptyset; \emptyset \proves a \hastype \chtype{S}
		\qquad
		\trule{LVar}~~\Gamma; \set{X: \lhot{S}}; \emptyset \proves X \hastype \lhot{S} 
		\\[2mm]
		\trule{Prom}~~\tree{
			\Gamma; \emptyset; \emptyset \proves V \hastype \lhot{S}
		}{
			\Gamma; \emptyset; \emptyset \proves V \hastype \shot{S}
		} 
		\qquad\quad  
		\trule{Derelic}~~\tree{
			\Gamma; \Lambda \cat X{:}\lhot{S}; \Sigma \proves P \hastype \Proc
		}{
			\Gamma \cat X:\shot{S}; \Lambda; \Sigma \proves P \hastype \Proc
		} 
		\\[4mm]
%		\trule{Subt}~~\tree{
%			\Gamma; \Lambda; \Sigma \proves P \hastype T \quad \Sigma \subt \Sigma' \quad T \subt T'
%		}{
%			\Gamma ; \Lambda; \Sigma' \vdash P \hastype T'
%		} 
%		\qquad\quad

		\trule{Abs}~~\tree{
			\Gamma; \Lambda; \Sigma \cat x: S \proves P \hastype \Proc
		}{
			\Gamma; \Lambda; \Sigma \proves \abs{x}{P} \hastype \lhot{S}
		}
		\\[4mm]

		\trule{App}~~\tree{(U = \lhot{S}) \lor (U = \shot{S}) \quad \Gamma; \Lambda_1; \Sigma_1 \proves X \hastype U  \quad \Gamma; \Lambda_2; \Sigma_2 \proves k \hastype S
		}{
			\Gamma; \Lambda_1 \cup \Lambda_2; \Sigma_1 \cup \Sigma_2 \proves \appl{X}{k} \hastype \Proc
		} 
		\\[4mm]

		\trule{Send}~~\tree{
			\Gamma; \Lambda_1; \Sigma_1 \proves P \hastype \Proc  \quad \Gamma; \Lambda_2; \Sigma_2 \vdash V \hastype U  \quad (k:S \in \Sigma_1 \cup \Sigma_2)
		}{
			\Gamma; \Lambda_1 \cup \Lambda_2; (\Sigma_1 \cup \Sigma_2)\backslash\set{k:S} \cat k:\btout{U} S \proves \bout{k}{V} P \hastype \Proc
		}

		\\[4mm]
		\trule{Conn}~~\tree{
			\Gamma; \Lambda; \Sigma \cat x:S \proves P \hastype \Proc  \quad \Gamma; \emptyset; \emptyset \proves a \hastype \chtype{S}
		}{
			\Gamma; \Lambda; \Sigma \proves \binp{a}{x} P \hastype \Proc
		}
		\\[4mm]
%		\trule{ConnDual}~~\tree{
%			\Gamma; \Lambda; \Sigma \cat x: S_1 \proves P \hastype \Proc  \quad \Gamma; \emptyset; \emptyset \proves k \hastype \chtype{S_2} \quad S_1 \dualof S_2
%		}{
%			\Gamma; \Lambda; \Sigma \proves \bout{k}{x} P \hastype \Proc
%		}
%		\\[4mm]

		\trule{ConnDual}~~\tree{
			\Gamma; \Lambda; \Sigma \cat \dual{s}: S_1 \proves P \hastype \Proc  \quad \Gamma; \emptyset; \emptyset \proves a \hastype \chtype{S_2} \quad S_1 \dualof S_2
		}{
			\Gamma; \Lambda; \Sigma  \proves \bout{a}{\dual{s}} P \hastype \Proc
		}

		\\[4mm]

		\trule{NewSh}~~\tree{
			\Gamma\cat a:\chtype{S} ; \Lambda; \Sigma \proves P \hastype \Proc
		}{
			\Gamma; \Lambda; \Sigma \proves \news{a} P \hastype \Proc}
		\qquad\quad
		\trule{NewSes}~~\tree{
			\Gamma; \Lambda; \Sigma \cat s:S_1 \cat \dual{s}: S_2 \proves P \hastype \Proc \quad S_1 \dualof S_2
		}{
			\Gamma; \Lambda; \Sigma \proves \news{s} P \hastype \Proc
		}
		\\[4mm]

		\trule{RecvS}~~\tree{
			\Gamma; \Lambda; \Sigma \cat k: S_1 \cat x: S_2 \proves P \hastype \Proc
		}{
			\Gamma; \Lambda; \Sigma, k: \btinp{S_2} S_1  \vdash \binp{k}{x}P \hastype \Proc
		}
		\quad\quad 
		\trule{RecvL}~~\tree{
			\Gamma; \Lambda \cat X: \lhot{S}; \Sigma \cat k: S_1  \proves P \hastype \Proc
		}{
			\Gamma; \Lambda; \Sigma \cat k:\btinp{\lhot{S}}S_1  \proves \binp{k}{X}P \hastype \Proc
		}
		\\[4mm]

		\trule{RecvSh}~~\tree{
			\Gamma \cat X: \shot{S}; \Lambda; \Sigma \cat k: S_1  \proves P \hastype \Proc
		}{
			\Gamma; \Lambda; \Sigma \cat k:\btinp{\shot{S}}S_1  \proves \binp{k}{X}P \hastype \Proc
		}
		\quad ~~
		\trule{RecvShN}~~\tree{
			\Gamma \cat x: \chtype{S}; \Lambda; \Sigma \cat k: S_1  \proves P \hastype \Proc
		}{
			\Gamma; \Lambda; \Sigma \cat k:\btinp{\chtype{S}}S_1  \proves \binp{k}{x}P \hastype \Proc
		}
		\\[4mm]
		\trule{Par}~~\tree{
			\Gamma; \Lambda_{1}; \Sigma_{1} \proves P_{1} \hastype \Proc \quad \Gamma; \Lambda_{2}; \Sigma_{2} \proves P_{2} \hastype \Proc
		}{
			\Gamma; \Lambda_{1} \cup \Lambda_2; \Sigma_{1} \cup \Sigma_2 \proves P_1 \Par P_2 \hastype \Proc
		}
		\qquad\quad
		\trule{Close}~~\tree{
			\Gamma; \Lambda; \Sigma  \proves P \hastype T \quad k \not\in \dom{\Gamma, \Lambda,\Sigma}
		}{
			\Gamma; \Lambda; \Sigma \cat k: \tinact  \proves P \hastype \Proc
		}
		\\[4mm]
		\trule{Bra}~~\tree{
			 \forall i \in I \quad \Gamma; \Lambda; \Sigma \cat k:S_i \proves P_i \hastype \Proc
		}{
			\Gamma; \Lambda; \Sigma \cat k: \btbra{l_i:S_i}_{i \in I} \proves \bbra{k}{l_i:P_i}_{i \in I}\hastype \Proc
		}
		\qquad\quad 
	 	\trule{Sel}~~\tree{
			\Gamma; \Lambda; \Sigma \cat k: S_j  \proves P \hastype \Proc \quad j \in I
		}{
			\Gamma; \Lambda; \Sigma \cat k:\btsel{l_i:S_i}_{i \in I} \proves \bsel{s}{l_j} P \hastype \Proc
		}
		\\[4mm]

		\trule{Nil}~~\Gamma; \emptyset; \emptyset \proves \inact \hastype \Proc
\qquad \quad
		\trule{Var}~~\tree{
	
		}{
			\Gamma \cat \rvar{X}: \Sigma; \emptyset; \emptyset  \proves \rvar{X} \hastype \Proc
		}
		\qquad\quad 
%	 	\trule{Rec}~~\tree{
%			\Gamma \cat \rvar{X}: \Sigma; \emptyset; \emptyset  \proves P \hastype \Proc
%		}{
%			\Gamma ; \emptyset; \emptyset  \proves \recp{X}{P} \hastype \Proc
%		}
%		\\[4mm]

	 	\trule{Rec}~~\tree{
			\Gamma \cat \rvar{X}: \Sigma; \emptyset; \Sigma  \proves P \hastype \Proc
		}{
			\Gamma ; \emptyset; \Sigma  \proves \recp{X}{P} \hastype \Proc
		}


	\end{array}
\]
\caption{Typing Rules for $\HOp$\label{fig:typerulesmy}}
\end{figure}


The typing rules for our system are in Fig.~\ref{fig:typerulesmy}. 


\section{Characteristic Session Bisimulation}
\label{sec:behavioural}
% !TEX root = main.tex
\noi We develop a theory for observational equivalence over
session typed \HOp processes. The theory follows the principles
laid by the previous work of the authors
\cite{KYHH2015,KY2015}.
We introduce three different bisimulations and prove
\jpc{that}
all of them coincide with typed, reduction-closed,
barbed congruence. 

\subsection{Labelled Transition System for Processes}\label{ss:lts}
\myparagraph{Labels}
\noi We define a labelled transition system (LTS) over
untyped processes. 
Later on, using the \emph{environmental} transition semantics, 
we can define a typed transition relation to formalise 
how a process interacts with a process in its environment. The interaction
is defined on action $\ell$:

\begin{tabular}{l}
	$\ell	\bnfis   \tau 
		\bnfbar	\bactinp{n}{V} 
		\bnfbar	\news{\tilde{m}} \bactout{n}{V}
		\bnfbar	\bactsel{n}{l} 
		\bnfbar	\bactbra{n}{l} $
\end{tabular}

\noi The internal action is defined by label $\tau$.
Action $\news{\tilde{m}} \bactout{n}{V}$ denotes the sending of value $V$
over channel $n$ with
a possible empty set of names $\tilde{m}$ being restricted
(we may also write $\bactout{n}{V}$ when $\tilde{m}$ is empty).
Dually, the action for value reception is 
$\bactinp{n}{V}$.
We also have actions for selecting 
and branching on
a label $l$ ($\bactsel{n}{l}$ and $\bactbra{n}{l}$, resp.)
$\fn{\ell}$ and $\bn{\ell}$ denote 
 sets of free/bound names in $\ell$, resp.
%and set $\mathsf{n}(\ell)=\bn{\ell}\cup \fn{\ell}$. 

Dual actions, defined below, occur on subjects that are dual between them and carry the same
object. Thus, output is dual with input and 
selection is dual with branching.
Formally, duality 
\jpc{on actions}
is the symmetric relation $\asymp$ that satisfies:
\jpc{(i) $\bactsel{n}{l} \asymp \bactbra{\dual{n}}{l}$ 
and (ii) $\news{\tilde{m}} \bactout{n}{V} \asymp \bactinp{\dual{n}}{V}$}.
%
%\begin{tabular}{c}
%	$\bactsel{n}{l} \asymp \bactbra{\dual{n}}{l}
%	\qquad \qquad \qquad
%	\news{\tilde{m}} \bactout{n}{V} \asymp \bactinp{\dual{n}}{V}$s
%
%\end{tabular}
\smallskip

\begin{figure}[t]
\[
\ltsrule{App} \quad 
(\abs{x}{P}) \, u   \by{\tau}  P \subst{u}{x} 
\]
	\[
	\begin{array}{ll}
\ltsrule{Out}\	\bout{n}{V} P \by{\bactout{n}{V}} P 
&
\ltsrule{In}\	\binp{n}{x} P \by{\bactinp{n}{V}} P\subst{V}{x} 
\\[3mm]
 \ltsrule{Sel}\ \bsel{s}{l}{P} \by{\bactsel{s}{l}} P
&
\hspace{-1cm}
\ltsrule{Bra}\ \bbra{s}{l_i:P_i}_{i \in I} \by{\bactbra{s}{l_j}} P_j
\quad (j\in I)
\\[3mm]
\ltsrule{Alpha}
		\tree{
			P \scong_\alpha Q \quad Q\by{\ell} P'
		}{
			P \by{\ell} P'
		}
&
 \ltsrule{Res}	\tree{
			P \by{\ell} P' \quad n \notin \fn{\ell}
		}{
			\news{n} P \by{\ell} \news{n} P' 
		}
\end{array}
\]
\[
\begin{array}{ll}
\ltsrule{New}&	\tree{
		P \by{\news{\tilde{m}} \bactout{n}{V}} P' \quad 
               m \in \fn{V}
		}{
			\news{m} P \by{\news{m\cat\tilde{m}'} 
\bactout{n}{V}} P'
		}
		\\[6mm]
\ltsrule{Tau}	& \tree{
			P \by{\ell_1} P' \qquad Q \by{\ell_2} Q' \qquad \ell_1 \asymp \ell_2
		}{
			P \Par Q \by{\tau} \newsp{\bn{\ell_1} \cup \bn{\ell_2}}{P' \Par Q'}
		} 
		\\[6mm]
 \ltsrule{Par${}_L$}	& \tree{

			P \by{\ell} P' \quad \bn{\ell} \cap \fn{Q} = \es
		}{
			P \Par Q \by{\ell} P' \Par Q
		}

%\\[3mm]
%		\tree{
%			Q \by{\ell} Q' \quad \bn{\ell} \cap \fn{P} = \es
%		}{
%			P \Par Q \by{\ell} P \Par Q'
%		}\ \ltsrule{RPar}
	\end{array}
	\]
%We omit $\ltsrule{Par${}_R$}$. 
	\caption{The Untyped (Early) Labelled Transition System. We omit rule $\ltsrule{Par${}_R$}$.  \label{fig:untyped_LTS}}
\Hlinefig
\end{figure}
\myparagraph{LTS over Untyped Processes}
The labelled transition system (LTS) over untyped processes
is given in
\figref{fig:untyped_LTS}. 
We write $P_1 \by{\ell} P_2$ with the usual meaning.
The rules are standard~\cite{KYHH2015,KY2015}.
A process with a send prefix can
interact with the environment with a send action that carries a value
$V$ as in rule~$\ltsrule{Out}$.  Dually, in rule $\ltsrule{In}$ a
received process can observe an input action of a value $V$.
Selection and branching processes observe the select and branch
actions in rules $\ltsrule{Sel}$ and $\ltsrule{Bra}$, respectively.
Rule $\ltsrule{Res}$ closes the LTS under the name creation operator
if the restricted name does not occur free in the
observable action. 
%If a restricted name occurs free,  
Otherwise, 
the process performs scope opening (rule $\ltsrule{New}$).  
Rule $\ltsrule{Tau}$ states that if two parallel processes can perform
dual actions then the two actions can synchronise by 
an internal transition. Rules $\ltsrule{Par${}_L$}$/$\ltsrule{Par${}_R$}$ 
and $\ltsrule{Alpha}$ close the LTS
under parallel  and $\alpha$-renaming. 
%provided that the observable
%action does not share any bound names with the parallel processes.

\subsection{Environmental Labelled Transition System}
\label{ss:elts}
\noi 
\figref{fig:envLTS}
defines a labelled transition relation between 
a triple of environments, 
denoted
$(\Gamma, \Lambda_1, \Delta_1) \by{\ell} (\Gamma, \Lambda_2, \Delta_2)$.
It extends the transition systems
in \cite{KYHH2015,KY2015} 
to higher-order sessions. 

\myparagraph{Input Actions} 
are defined by 
$\eltsrule{SRv}$ and $\eltsrule{ShRv}$
%describe the input action
($n$ session or shared channel respectively $\bactinp{n}{V}$). 
We require the value $V$ has
the same type as name $s$ and $a$, respectively.  Furthermore we
expect the resulting type tuple to contain the values that consist
with value $V$. The condition $\dual{s} \notin \dom{\Delta}$
in $\eltsrule{SRv}$ ensures that 
the dual channel $\dual{s}$ should not be
present in the session environment, since if it were present
the only communication that could take place is the interaction
between the two endpoints (using $\eltsrule{Tau}$ below).

\myparagraph{Output Actions} are defined by $\eltsrule{SSnd}$
and $\eltsrule{ShSnd}$.  
Rule $\eltsrule{SSnd}$ states the conditions for observing action
$\news{\tilde{m}} \bactout{s}{V}$ on a type tuple 
$(\Gamma, \Lambda, \Delta\cdot \AT{s}{S})$. 
The session environment $\Delta$ with $\AT{s}{S}$ 
should include the session environment of sent value $V$, 
{\em excluding} the session environments of the name $n_j$ 
in $\tilde{m}$ which restrict the scope of value $V$. 
Similarly the linear variable environment 
$\Lambda'$ of $V$ should be included in $\Lambda$. 
Scope extrusion of session names in $\tilde{m}$ requires
that the dual endpoints of $\tilde{m}$ appear in
the resulting session environment. Similarly for shared 
names in $\tilde{m}$ that are extruded.  
All free values used for typing $V$ are subtracted from the
resulting type tuple. The prefix of session $s$ is consumed
by the action.
Similarly, an output on a shared name is described
by rule $\eltsrule{ShSnd}$ where we require that the name
is typed with $\chtype{U}$. Conditions for
the output $V$ are identical to those
% the requirements 
for rule~$\eltsrule{SSnd}$.

\myparagraph{Other Actions}
Rules $\eltsrule{Sel}$ and $\eltsrule{Bra}$ describe actions for
select and branch. The only requirements for both
rules is that the dual endpoint is not present in the session
environment and the action labels are present
in the type.
Hidden transitions defined by rule $\eltsrule{Tau}$ 
do not change the session environment or they follow the reduction on session
environments (\defref{def:ses_red}). 


%A second environment LTS, denoted $\hby{\ell}$,
%is defined in the lower part of \figref{fig:envLTS}.
%The definition substitutes rules
%$\eltsrule{SRecv}$ and $\eltsrule{ShRecv}$
%of relation $\by{\ell}$ with rule $\eltsrule{RRcv}$.
%% the corresponding input cases
%%of $\by{\ell}$ with the definitions of $\hby{\ell}$.
%All other cases remain the same as the cases for
%relation $\by{\ell}$.
%Rule $\eltsrule{RRcv}$ restricts the higher-order input
%in relation $\hby{\ell}$;
%only characteristic processes and trigger processes
%are allowed to be received on a higher-order input.
%Names can still be received as in the definition of
%the $\by{\ell}$ relation.
%The conditions for input follow the conditions
%for the $\by{\ell}$ definition.


\begin{figure}[t]
\[
\begin{array}{lc}
	\eltsrule{SRv}&\tree{
			\dual{s} \notin \dom{\Delta} \quad \Gamma; \Lambda'; \Delta' \proves V \hastype U
		}{
			(\Gamma; \Lambda; \Delta \cat s: \btinp{U} S) \by{\bactinp{s}{V}} (\Gamma; \Lambda\cat\Lambda'; \Delta\cat\Delta' \cat s: S)
		}
		\\[8mm]
		\eltsrule{ShRv}&\tree{
			\Gamma; \es; \es \proves a \hastype \chtype{U}
			\quad
			\Gamma; \Lambda'; \Delta' \proves V \hastype U
		}{
			(\Gamma; \Lambda; \Delta) \by{\bactinp{a}{{V}}} (\Gamma; \Lambda\cat\Lambda'; \Delta\cat\Delta')
		}
%		\eltsrule{RRcv}&\tree {
%\begin{array}{c}
%(\Gamma_1; \Lambda_1; \Delta_1) \by{\bactinp{n}{V}} (\Gamma_2; \Lambda_2; \Delta_2)
%\\
%			\begin{array}{lll}
%				 V  =  
%(\abs{{x}}{\binp{t}{y} (\appl{y}{{x}})}
% \vee  \abs{{x}}{\map{U}^{{x}}}  \vee m)  \textrm{ with } t \textrm{ fresh} 
%			\end{array}
%			\end{array}
%		}{
%			(\Gamma_1; \Lambda_1; \Delta_1) \hby{\bactinp{n}{V}} (\Gamma_2; \Lambda_1; \Delta_2)
%		}
	\end{array}
	\]
	\[
	\begin{array}{l}
		\eltsrule{SSnd}\\
\tree{
			\begin{array}{lll}
			\Gamma \cat \Gamma'; \Lambda'; \Delta' \proves V \hastype U
&				
				\Gamma'; \es; \Delta_j \proves m_j  \hastype U_j
& 
				\dual{s} \notin \dom{\Delta}
\\
						\Delta'\backslash \cup_j \Delta_j \subseteq (\Delta \cat s: S)
& 
	\Gamma'; \es; \Delta_j' \proves \dual{m}_j  \hastype U_j'
& 
				\Lambda' \subseteq \Lambda
%				\dual{s} \notin \dom{\Delta}
%				\qquad 
%				\Gamma \cat \Gamma'; \Lambda'; \Delta_1 \cat \Delta_2 \proves V \hastype U
%				\qquad
%				\tilde{m} = m_1 \dots m_n
%				\\
%				\Gamma'; \es; \Delta_2 \proves m_1 \dots m_n \hastype U_1
%				\qquad
%				\Gamma'; \es; \Delta_3 \proves \dual{m}_1 \dots \dual{m}_n \hastype U_2
%				\qquad
%				\Lambda' \subseteq \Lambda
%				\qquad
%				\Delta_1 \subseteq (\Delta \cat s: S)
			\end{array}
		}{
			(\Gamma; \Lambda; \Delta \cat s: \btout{U} S) \by{\news{\tilde{m}} \bactout{s}{V}} (\Gamma \cat \Gamma'; \Lambda\backslash\Lambda';
			(\Delta \cat s: S \cat \cup_j \Delta_j') \backslash \Delta')
		}
\\[6mm]
\eltsrule{ShSnd}\\
\tree{
		\begin{array}{lll}
			\Gamma \cat \Gamma' ; \Lambda'; \Delta' \proves V \hastype U &  
		\Gamma'; \es; \Delta_j \proves m_j \hastype U_j
& \Gamma ; \es ; \es \proves a \hastype \chtype{U}
				\\
			\Delta'\backslash \cup_j \Delta_j 
                         \subseteq \Delta
& 
		\Gamma'; \es; \Delta_j' \proves \dual{m}_j\hastype U_j'
& 
				\Lambda' \subseteq \Lambda
			\end{array}
%			\begin{array}{c}
%				\Gamma \cat \Gamma' \cat a: \chtype{U}; \Lambda'; \Delta_1 \cat \Delta_2 \proves V \hastype U
%				\qquad
%				\tilde{m} = m_1 \dots m_n
%				\\
%				\Gamma'; \es; \Delta_2 \proves m_1 \dots m_n \hastype U_1
%				\qquad
%				\Gamma'; \es; \Delta_3 \proves \dual{m}_1 \dots \dual{m}_n \hastype U_2
%				\qquad
%				\Lambda' \subseteq \Lambda
%				\qquad
%				\Delta_1 \subseteq \Delta
%			\end{array}
		}{
			(\Gamma ; \Lambda; \Delta) \by{\news{\tilde{m}} \bactout{a}{V}} (\Gamma \cat \Gamma' ; \Lambda\backslash\Lambda';
			(\Delta \cat \cup_j \Delta_j') \backslash \Delta')
		}
\end{array}
\]
\[
\begin{array}{lc}
		\eltsrule{Sel}&\tree{
			\dual{s} \notin \dom{\Delta} \quad j \in I
		}{
			(\Gamma; \Lambda; \Delta \cat s: \btsel{l_i: S_i}_{i \in I}) \by{\bactsel{s}{l_j}} (\Gamma; \Lambda; \Delta \cat s:S_j)
		}
\\[8mm]
		\eltsrule{Bra}&\tree{
			\dual{s} \notin \dom{\Delta} \quad j \in I
		}{
			(\Gamma; \Lambda; \Delta \cat s: \btbra{l_i: T_i}_{i \in I}) \by{\bactbra{s}{l_j}} (\Gamma; \Lambda; \Delta \cat s:S_j)
		}
		\\[8mm]
		\eltsrule{Tau}&\tree{
			\Delta_1 \red \Delta_2 \vee \Delta_1 = \Delta_2
		}{
			(\Gamma; \Lambda; \Delta_1) \by{\tau} (\Gamma; \Lambda; \Delta_2)
		}
%\\[6mm]
%		\eltsrule{RRcv}&\tree {
%\begin{array}{c}
%(\Gamma_1; \Lambda_1; \Delta_1) \by{\bactinp{n}{V}} (\Gamma_2; \Lambda_2; \Delta_2)
%\\
%			\begin{array}{lll}
%				 V  =  
%(\abs{{x}}{\binp{t}{y} (\appl{y}{{x}})}
% \vee  \abs{{x}}{\map{U}^{{x}}}  \vee m)  \textrm{ with } t \textrm{ fresh} 
%			\end{array}
%			\end{array}
%		}{
%			(\Gamma_1; \Lambda_1; \Delta_1) \hby{\bactinp{n}{V}} (\Gamma_2; \Lambda_1; \Delta_2)
%		}
	\end{array}
	\]
\caption{Labelled Transition Systen for Typed Environments. 
\label{fig:envLTS}}
\Hlinefig
\end{figure}

Below we define the typed LTS combining 
the LTS of processes and the LTS of environments. 

\smallskip

\begin{definition}[Typed Transition Systems]\label{d:tlts}\rm
A {\em typed transition relation} is a typed relation
%\begin{enumerate}
%\item 
$\horel{\Gamma}{\Delta_1}{P_1}{\by{\ell}}{\Delta_2}{P_2}$
%$\Gamma; \emptyset; \Delta_1 \proves P_1 \hastype \Proc \by{\ell} \Gamma; \emptyset; \Delta_2 \proves P_2 \hastype \Proc$
	where:
%
(1) $P_1 \by{\ell} P_2$ and (2) 
$(\Gamma, \emptyset, \Delta_1) \by{\ell} (\Gamma, \emptyset, \Delta_2)$ 
with $\Gamma; \emptyset; \Delta_i \proves P_i \hastype \Proc$ 
($i=1,2$). 
%
% Efficient 
%\item 
%$\horel{\Gamma}{\Delta_1}{P_1}{\hby{\ell}}{\Delta_2}{P_2}$
%whenever: 
%$P_1 \by{\ell} P_2$, 
%$(\Gamma, \emptyset, \Delta_1) \hby{\ell} (\Gamma, \emptyset, \Delta_2)$, 
%and $\Gamma; \emptyset; \Delta_i \proves P_i \hastype \Proc$ 
%($i=1,2$)
%\end{enumerate}
%
We extend to $\By{}$ 
%(resp.\ $\Hby{}$) and  
and $\By{\hat{\ell}}$ 
%(resp.\ $\Hby{\hat{\ell}}$) 
where we write 
$\By{}$ for the reflexive and
transitive closure of $\by{}$, $\By{\ell}$ for the transitions
$\By{}\by{\ell}\By{}$ and $\By{\hat{\ell}}$ for $\By{\ell}$ if
$\ell\not = \tau$ otherwise $\By{}$. 
%We extend to $\By{}$ 
%(resp.\ $\Hby{}$) and  and 
%$\By{\hat{\ell}}$ 
%(resp.\ $\Hby{\hat{\ell}}$) 
%in the standard way.
\end{definition}

\subsection{Reduction-Closed, Barbed Congruence}
\label{subsec:rc}
\noi We define the typed relation and the contextual equivalence.  
%\begin{definition}[Session Environment Confluence]\rm
We begin with a definition of a notion of confluence
over session environments $\Delta$:
we denote $\Delta_1 \bistyp \Delta_2$ if there exists $\Delta$ such that
	$\Delta_1 \red^\ast \Delta$ and $\Delta_2 \red^\ast \Delta$
	\jpc{(here we write $\red^\ast$ for the multistep environment reduction in \defref{def:ses_red})}.
%\end{definition}

\smallskip 

\begin{definition}\rm %[Typed Relation]\rm
	We say that
	$\Gamma; \emptyset; \Delta_1 \proves P_1 \hastype \Proc\ \Re \ \Gamma; \emptyset; \Delta_2 \proves P_2 \hastype \Proc$
	is a {\em typed relation} whenever $P_1$ and $P_2$ are closed;
		$\Delta_1$ and $\Delta_2$ are balanced; and 
		$\Delta_1 \bistyp \Delta_2$.
We write
$\horel{\Gamma}{\Delta_1}{P_1}{\ \Re \ }{\Delta_2}{P_2}$
for the typed relation $\Gamma; \emptyset; \Delta_1 \proves P_1 \hastype \Proc\ \Re \ \Gamma; \emptyset; \Delta_2 \proves P_2 \hastype \Proc$.
\end{definition}

\smallskip 

\noi Type relations relate only closed terms with
balanced session environments and the two session environments
are confluent.
Next we define the {\em barb} \cite{MiSa92} 
with respect to types. 

\smallskip 

\begin{definition}[Barbs]\rm
Let $P$ closed. We define:
\begin{enumerate}
		\item	$P \barb{n}$ if $P \scong \newsp{\tilde{m}}{\bout{n}{V} P_2 \Par P_3}, n \notin \tilde{m}$. %; $P \Barb{n}$ if $P \red^* \barb{n}$.

		\item	$\Gamma; \Delta \proves P \barb{n}$ if
			$\Gamma; \emptyset; \Delta \proves P \hastype \Proc$ with $P \barb{n}$ and $\dual{n} \notin \dom{\Delta}$.

	$\Gamma; \Delta \proves P \Barb{n}$ if $P \red^* P'$ and
			$\Gamma; \Delta' \proves P' \barb{n}$.			
	\end{enumerate}
\end{definition}

\smallskip 

\noi A barb $\barb{n}$ is an observable on an output prefix with subject $n$.
Similarly a weak barb $\Barb{n}$ is a barb after a number of reduction steps.
Typed barbs $\barb{n}$ (resp.\ $\Barb{n}$)
happen on typed processes $\Gamma; \emptyset; \Delta \proves P \hastype \Proc$
where we require that the corresponding dual endpoint $\dual{n}$ is not present
in the session type $\Delta$.

To define a congruence relation, we introduce the context $\C$:\\  

%\begin{definition}[Context]\rm
%	A context $\C$ is defined as:
\noi 
\begin{tabular}{rl}
	$\C::=$\!\!\!\! & $\hole \bnfbar \bout{u}{V} \C \bnfbar \binp{u}{x} \C
\bnfbar \bout{u}{\lambda x.\C} P
\bnfbar \news{n} \C$\\
             $\bnfbar$\!\!\!\!& $(\lambda x.\C)u \bnfbar \recp{X}{\C} \bnfbar \C \Par P \bnfbar P \Par \C$\\ 
$\bnfbar$\!\!\!\!& $\bsel{u}{l} \C \bnfbar \bbra{u}{l_1:P_1,..,l_i:\C,..l_n:P_n}$\\
	\end{tabular}
\smallskip 

\noi 
Notation $\context{\C}{P}$ replaces 
\jpc{the hole}
$\hole$ in $\C$ with $P$.
%\end{definition}

\smallskip 

\noi The first behavioural relation we define is reduction-closed, barbed congruence \cite{HondaKYoshida95}. 

\smallskip 

\begin{definition}[Reduction-Closed, Barbed Congruence]\rm
\label{def:rc}
	Typed relation
	$\horel{\Gamma}{\Delta_1}{P_1}{\ \Re\ }{\Delta_2}{P_2}$
	is a {\em reduction-closed, barbed congruence} whenever:
	\begin{enumerate}
		\item	If $P_1 \red P_1'$ then there exists $P_2'$ such that $P_2 \red^* P_2'$ and
			$\horel{\Gamma}{\Delta_1'}{P_1'}{\ \Re\ }{\Delta_2'}{P_2'}$; 
			and its symmetric case;
%		\item	If $P_2 \red P_2'$ then $\exists P_1', P_1 \red^* P_1'$ and
%		$\horel{\Gamma}{\Delta_1'}{P_1'}{\ \Re\ }{\Delta_2'}{P_2'}$
%		\end{itemize}

%		\item
%		\begin{itemize}
			\item	If $\Gamma;\Delta \proves P_1 \barb{n}$ then $\Gamma;\Delta \proves P_2 \Barb{n}$; and its symmetric case; 

%			\item	If $\Gamma;\emptyset;\Delta \proves P_2 \barb{s}$ then $\Gamma;\emptyset;\Delta \proves P_1 \Barb{s}$.
%		\end{itemize}

		\item	for all $\C$, $\horel{\Gamma}{\Delta_1'}{\context{\C}{P_1}}{\ \Re\ }{\Delta_2'}{\context{\C}{P_2}}$
	\end{enumerate}
	The largest such congruence is denoted with $\cong$.
\end{definition}


\subsection{Contextual Bisimulation ($\wbc$)}
\label{subsec:bisimulation}
\noi The first bisimulation which we define 
is the standard contextual bisimulation~\cite{San96H}. 
%
\begin{definition}[Contextual Bisimulation]\rm
\label{def:wbc}
A typed relation $\Re$ is {\em a contextual bisimulation} if
for all $\horel{\Gamma}{\Delta_1}{P_1}{\ \Re \ }{\Delta_2}{Q_1}$, 
	\begin{enumerate}[1)] 
	\item	whenever 
$\horel{\Gamma}{\Delta_1}{P_1}
        {\by{\news{\tilde{m_1}} \bactout{n}{V_1}}}{\Delta_1'}{P_2}$,
there exists $\horel{\Gamma}{\Delta_2}{Q_1}{\By{\news{\tilde{m_2}} \bactout{n}{V_2}}}{\Delta_2'}{Q_2}$ such that, 
for all $R$ with $\fv{R}=x$:
\[\horel{\Gamma}{\Delta_1''}{\newsp{\tilde{m_1}}{P_2 \Par R\subst{V_1}{x}}}
				{\ \Re\ }
				{\Delta_2''}{\newsp{\tilde{m_2}}{Q_2 \Par R\subst{V_2}{x}}};\]  
%\item	$\forall \news{\tilde{m_1}'} \bactout{n}{\tilde{m_1}}$ such that
%			\[
%				\horel{\Gamma}{\Delta_1}{P_1}{\by{\news{\tilde{m_1}'} \bactout{n}{\tilde{m_1}}}}{\Delta_1'}{P_2}
%			\]
%			implies that $\exists Q_2, \tilde{m_2}$ such that
%			\[
%				\horel{\Gamma}{\Delta_2}{Q_1}{\By{\news{\tilde{m_2}'} \bactout{n}{\tilde{m_2}}}}{\Delta_2'}{Q_2}
%			\]
%			and $\forall R$ with $\tilde{x} = \fn{R}$, 
%			then
%			\[
%				\horel{\Gamma}{\Delta_1''}{\newsp{\tilde{m_1}'}{P_2 \Par R \subst{\tilde{m_1}}{\tilde{x}}}}
%				{\ \Re \ }
%				{\Delta_2''}{\newsp{\tilde{m_2}'}{Q_2 \Par R \subst{\tilde{m_2}}{\tilde{x}}}}
%			\]
		\item	
for all $\horel{\Gamma}{\Delta_1}{P_1}{\by{\ell}}{\Delta_1'}{P_2}$ such that 
$\ell$ is not an output, 
 there exists $Q_2$ such that 
$\horel{\Gamma}{\Delta_2}{Q_1}{\By{\ell}}{\Delta_2'}{Q_2}$
			and
			$\horel{\Gamma}{\Delta_1'}{P_2}{\ \Re \ }{\Delta_2'}{Q_2}$; and  

                      \item	The symmetric cases of 1 and 2.                
	\end{enumerate}
	The largest such bisimulation is called contextual bisimilarity \jpc{and} denoted by $\wbc$.
\end{definition}

\smallskip 

\noi As explained in \secref{subsec:intro:bisimulation}, 
in the general case,
contextual bisimulation 
is hard to compute. Below we introduce $\hwb$ and $\fwb$.
%due to: (1) the universal
%quantification over contexts in the output case;
%and (2) a higher-order input prefix which can observe
%infinitely many different input actions (since
%infinitely many different processes can match
%the session type of an input prefix).

\subsection{Higher-Order  and  
Characteristic  Bisimulations ($\hwb$/$\fwb$)}\label{ss:hwb}
\noi 
We now formalise the novelties motivated in \secref{subsec:intro:bisimulation}.
Our main result is \thmref{the:coincidence}.
We define characteristic processes/values:

\begin{definition}[Characteristic Process and Values]\rm
\label{def:char}
%	Let names $\tilde{k}$ and type $\tilde{C}$; then we define a {\em characteristic process}:
%	$\map{\tilde{C}}^{\tilde{k}}$:
%	\[
%		\map{C_1, \cdots, C_n}^{k_1 \cdots k_n} = \map{C_1}^{k_1} \Par \dots \Par \map{C_n}^{k_n}		
%	\]
%	with 
	Let name $u$ and type $U$. 
	\figref{fig:char} defines the {\em characteristic process} 
	$\mapchar{U}{u}$ and the {\em characteristic value} $\omapchar{U}$.
\end{definition}

\smallskip

\begin{figure}[t]
	\[
	\begin{array}{c}
		\begin{array}{rclcl}
			\mapchar{\btinp{U} S}{u} &\!\!\defeq\!\!
& \binp{u}{x} (\mapchar{S}{u} \Par \mapchar{U}{x})
			\\
			\mapchar{\btout{U} S}{u} &\!\!\defeq\!\!& \bout{u}{\omapchar{U}} \mapchar{S}{u} %& & n \textrm{ fresh}
			\\
			\mapchar{\btsel{l : S}}{u} &\!\!\defeq\!\!& \bsel{u}{l} \mapchar{S}{u}
			\\
			\mapchar{\btbra{l_i: S_i}_{i \in I}}{u} &\!\!\defeq\!\!& \bbra{u}{l_i: \mapchar{S_i}{u}}_{i \in I}
			\\
		\mapchar{\tvar{t}}{u} \!\defeq\! \varp{X}_{\vart{t}}
& & 
			\mapchar{\trec{t}{S}}{u} \!\defeq\! \recp{X_{\vart{t}}}{\mapchar{S}{u}}
			\\
			\mapchar{\tinact}{u} &\!\!\defeq\!\!& \inact
			\\
\mapchar{\chtype{S}}{u} \!\defeq\! \bout{u}{\omapchar{S}} \inact & & 
\quad\mapchar{\chtype{L}}{u} \!\defeq\! \bout{u}{\omapchar{L}} \inact
			\\
\mapchar{\shot{C}}{u} & \!\!\defeq\!\! & \mapchar{\lhot{C}}{u} \!\defeq\! 
\appl{u}{\omapchar{C}}
\end{array}
\\
\Hline
\\
		\begin{array}{rcll}
\omapchar{S} &\!\!\defeq\!\!& s & (s \textrm{ fresh})
			\\
\omapchar{\chtype{S}} \defeq \omapchar{\chtype{L}} &\!\!\defeq\!\!& a & 
(a \textrm{ fresh})\quad\quad
			\\
			\omapchar{\shot{C}} \defeq \omapchar{\lhot{C}} &\!\!\defeq\!\!& \abs{x}{\mapchar{C}{x}}
		\end{array}
	\end{array}
	\]
\caption{Characteristic Processes \jpc{(top)} and Values \jpc{(bottom)}.\label{fig:char}}
\Hlinefig
\end{figure}

%	\[
%	\begin{array}{rcllrcll}
%		\map{\btinp{C} S}^{u} &=& \binp{u}{x} (\map{S}^{u} \Par 
%\map{{C}}^{x})\\
%		\map{\btout{C} S}^{u} &=& \bout{u}{n} \map{S}^{u} & n\textrm{ fresh}
%		\\
%		\map{\btsel{l : S}}^{u} &=& \bsel{u}{l} \map{S}^{u}
%\\
%		\map{\btbra{l_i: S_i}_{i \in I}}^{u} &=& \bbra{k}{l_i: \map{S_i}^{u}}_{i \in I}
%		\\

%		\map{\tvar{t}}^{u} &=& \rvar{X}_{\tvar{t}}
%\\
%		\map{\trec{t}{S}}^{u} &=& \mu \rvar{X}_{\tvar{t}}.\map{S}^{u}
%		\\

%		\map{\btout{\lhot{C}} S}^{u} &=& \bout{u}{\abs{x}{\map{C}^{x}}} \map{S}^u
%\\
%		\map{\btinp{\lhot{C}} S}^{u} &=& \binp{u}{x} (\map{S}^u \Par \appl{x}{n}) & {n}\textrm{ fresh}
%		\\
%		\map{\btout{\shot{C}} S}^{u} &=& \bout{u}{\abs{x}{\map{C}^{x}}}
%\map{S}^u \\
%		\map{\btinp{\shot{C}} S}^{u} &=& \binp{u}{x} (\map{S}^u \Par \appl{x}{n}) & n\textrm{ fresh}
%		\\

%%		\map{\btinp{\chtype{S}} S}^{k} &=& \binp{k}{x} (\map{S}^k \Par \map{\chtype{S}}^y)
%%		&&
%%		\map{\btout{\chtype{S}} S}^{k} &=& \bout{k}{a} \map{S}^k  & a\textrm{ fresh}
%%		\\
%		\map{\tinact}^{u} &=& \inact
%\\
%		\map{\chtype{S}}^{u} &=& \bout{u}{s} \inact &s\textrm{ fresh}
%\\
%		\map{\chtype{\lhot{C}}}^{u} &=& \bout{u}{\abs{x} \map{C}^{x}} \inact
%		\\
%		\map{\chtype{\shot{C}}}^{u} &=& \bout{u}{\abs{x} \map{C}^{x}} \inact
%	\end{array}
%	\]
%\end{definition}


%\noi Characteristic processes are inhabitants of their associated type:

%\begin{proposition}[Characteristic Processes]\rm
%\label{pro:characteristic}
%\begin{enumerate}
%\item $\Gamma; \emptyset; \Delta \cat u:S \proves \mapchar{S}{u} \hastype \Proc$; and $\Gamma \cat u:\chtype{S}; \emptyset; \Delta \proves \mapchar{\chtype{S}}{u} \hastype \Proc$; and 
%\item  	If $\Gamma; \emptyset; \Delta \proves \mapchar{C}{u} \hastype \Proc$
%	then
%	$\Gamma; \es; \Delta \proves u \hastype C$.
%\end{enumerate}
%\end{proposition}
%%\begin{IEEEproof}
%%	By induction on $\mapchar{S}{u}$, $\mapchar{\chtype{S}}{u}$
%%and $\mapchar{C}{u}$. 
%%\end{IEEEproof}

\noi We can easily verify characteristic processes are
inhabitants of their associated type. 
The example below explains the motivation of the refined 
LTS explained in \secref{subsec:intro:bisimulation}.
\smallskip  

\begin{example}[The Need for Refined Typed LTS]
\label{ex:motivation}
First we demonstrate that observing a characteristic value
input alone is not sufficient
\dk{to define a sound bisimulation closure}.
Consider typed processes $P_1, P_2$:
%
\begin{eqnarray}
	P_1 = \binp{s}{x} (\appl{x}{s_1} \Par \appl{x}{s_2}) 
	& & 
	P_2 = \binp{s}{x} (\appl{x}{s_1} \Par \binp{s_2}{y} \inact) 
	\label{equ:6}
\end{eqnarray}
%
We can show that $\Gamma; \es; \Delta \cat s: \btinp{\shot{(\btinp{C} \tinact)}} \tinact \proves P_i \hastype \Proc$ ($i \in \{1,2\}$).
If the above processes input and substitute over $x$
the characteristic value $\dk{\omapchar{\shot{(\btinp{C} \tinact})} =} \abs{x}{\binp{x}{y} \inact}$, 
then both processes evolve into:%(\ref{eq:5}) and (\ref{eq:6}) in become:

\begin{tabular}{c}
	$\Gamma; \es; \Delta \proves \binp{s_1}{y} \inact \Par \binp{s_2}{y} \inact \hastype \Proc$
\end{tabular}

\noi therefore becoming 
contextually bisimilar.
%after the input of $\abs{x}{\binp{x}{y}} \inact$.
However, the processes in (\ref{equ:6}) 
are clearly {\em not} contextually bisimilar: there exist many input actions
which may be used to distinguish them.
For example, if 
$P_1$ and $P_2$ input 
$\abs{x} \newsp{s_3}{\bout{a}{s_3} \binp{x}{y} \inact}$ with
$\Gamma; \es; \Delta \proves s,\dual{s} \hastype \tinact, \tinact$
then their derivatives are not contextually bisimilar. 

Observing only the characteristic value 
results in an over-discriminating bisimulation.
However, if a trigger value, 
$\abs{{x}}{\binp{t}{y} (\appl{y}{{x}})}$ 
is received on $s$, 
then we can distinguish 
processes in \eqref{equ:6}:  
%
\small
\begin{eqnarray*}
%	\Gamma; \es; \Delta &\proves& 
	P_1 \By{\ell} \binp{t}{x} (\appl{x}{s_1}) \Par 
\binp{t}{x} (\appl{x}{s_2})
%\hastype \Proc
	\mbox{~~and~~}
%	\Gamma; \es; \Delta &\proves& 
	P_2 \By{\ell} \binp{t}{x} (\appl{x}{s_1}) \Par \binp{s_2}{y} \inact 
%\hastype \Proc
\end{eqnarray*}
\normalsize
%\noi resulting two distinct processes.  
%
\noi where 
$\ell = s?\ENCan{\abs{{x}}{\binp{t}{y} (\appl{y}{{x}})}}$.
One question that arises here is whether the trigger value is enough
to distinguish two processes, hence no need of 
characteristic values as the input. 
This is not the case since the trigger value
alone also results in an over-discriminating bisimulation relation.
In fact the input trigger can be observed on any input prefix
of {\em any type}. For example, consider the following processes:
%
\begin{eqnarray}
	\Gamma; \es; \Delta \proves \newsp{s}{\binp{n}{x} \appl{x}{s} \Par \bout{\dual{s}}{\abs{x} P} \inact} \hastype \Proc\label{equ:7}
	\\
	\Gamma; \es; \Delta \proves \newsp{s}{\binp{n}{x} \appl{x}{s} \Par \bout{\dual{s}}{\abs{x} Q} \inact} \hastype \Proc\label{equ:8}
\end{eqnarray}
%
\noi if processes in \eqref{equ:7}/\eqref{equ:8}
input the trigger abstraction, we obtain: % they evolved to 
\begin{eqnarray*}
%\Gamma; \es; \Delta \proves 
	\newsp{s}{\binp{t}{x} \appl{x}{s} \Par \bout{\dual{s}}{\abs{x} P} \inact} 
%\hastype \Proc
	\mbox{ and }
%      \\
%\Gamma; \es; \Delta \proves 
	\newsp{s}{\binp{t}{x} \appl{x}{s} \Par \bout{\dual{s}}{\abs{x} Q} \inact}
%\hastype \Proc
\end{eqnarray*}

\noi thus we can easily derive a bisimulation closure if we 
assume a bisimulation definition that allows only trigger value input.
%
%\noi It is easy to obtain a closure if allow only the
%trigger value as the input value. 
But if processes in \eqref{equ:7}/\eqref{equ:8}
input the characteristic value $\abs{z}{\binp{z}{x} (\appl{x}{m})}$,  
then they would become:
%
\begin{eqnarray*}
	\Gamma; \es; \Delta \proves \newsp{s}{\binp{s}{x} \appl{x}{m} \Par \bout{\dual{s}}{\abs{x} P} \inact} \wbc \Delta \proves P \subst{m}{x}
	\\
	\Gamma; \es; \Delta \proves \newsp{s}{\binp{s}{x} \appl{x}{m} \Par \bout{\dual{s}}{\abs{x} Q} \inact} \wbc \Delta \proves Q \subst{m}{x}
\end{eqnarray*}
\noi which are not bisimilar if $P \subst{m}{x} \not\wb Q \subst{m}{x}$.
%\qed
%In conclusion, these examples explain a need of both 
%trigger and characteristic values 
%as an input observation in the input transition relation (\eltsrule{RRcv})
%which will be defined in Definition~\ref{def:rlts}.  
\end{example}

\smallskip 
\noi We now define the \emph{refined} typed LTS. 
As explained in \secref{sec:intro}, this new LTS is defined 
by considering a transition rule for input in which admitted values are
trigger or characteristic values:
%\dk{(assume extension of the structural
%congruence to acommodate values: i) $\abs{x}{P} \scong \abs{x}{Q}$ if
%$P \scong Q$) and ii) $n \scong m$ if $n = n$)}: 

\smallskip 

\begin{definition}[Refined Typed Labelled Transition Relation]
	\label{def:rlts}
	We define the environment transition rule for input actions in
	%restricted environment transition relation using the
	%following rule %using the environment transition relation defined in 
	using the input rules in \figref{fig:envLTS}:
	\[
	\begin{array}{rl}
			\eltsrule{RRcv}&\tree {
	\begin{array}{c}
	(\Gamma_1; \Lambda_1; \Delta_1) \by{\bactinp{n}{V}} (\Gamma_2; \Lambda_2; \Delta_2)
	\\
				\begin{array}{lll}
					V  \scong
					(\abs{{x}}{\binp{t}{y} (\appl{y}{{x}})}
					\vee  \omapchar{U}%\abs{{x}}{\map{U}^{{x}}}
					\vee m)  \textrm{ with } t \textrm{ fresh} 
				\end{array}
				\end{array}
			}{
				(\Gamma_1; \Lambda_1; \Delta_1) \hby{\bactinp{n}{V}} (\Gamma_2; \Lambda_1; \Delta_2)
			}
	\end{array}
	\]
	\noi $\eltsrule{RRcv}$ is defined on top
	of rules $\eltsrule{SRv}$ and $\eltsrule{ShRv}$
	in \figref{fig:envLTS}.
%	uses the environment transition
%	$(\Gamma, \Lambda_1, \Delta_1) \hby{\ell} (\Gamma, \Lambda_2, \Delta_2)$
%	in \figref{fig:envLTS}. 
\dk{	We then use the non-receiving rules in \figref{fig:envLTS}
	together with rule $\eltsrule{RRcv}$
	to define 
	$\horel{\Gamma}{\Delta_1}{P_1}{\hby{\ell}}{\Delta_2}{P_2}$
	as in \defref{d:tlts}.}
%	by replacing $\by{\ell}$ by $\hby{\ell}$ in \defref{d:tlts}. 
\end{definition}

\smallskip 

\noi Note 
$\horel{\Gamma}{\Delta_1}{P_1}{\hby{\ell}}{\Delta_2}{P_2}$ implies  
$\horel{\Gamma}{\Delta_1}{P_1}{\by{\ell}}{\Delta_2}{P_2}$.
%See \exref{ex:motivation} for the reason why {\em both} 
%the trigger values ($\lambda x.\binp{t}{y} (\appl{y}{{x}})$) 
%and characteristic values ($\lambda x.\map{U}^{{x}}$) are required 
%to define the following two bisimulations. 

\smallskip 

\myparagraph{The Two Bisimulations.} We now define 
higher-order bisimulation, 
a more tractable bisimulation for $\HO$ and $\HOp$.%-calculi. 
\dk{The two bisimulations differ on the fact that
they use the different (but equivalent)
trigger processes: $\htrigger{t}{V}$ and $\ftrigger{t}{V}{U}$.}

\smallskip 

\begin{definition}[Higher-Order Bisimulation]\rm
	\label{d:hbw}
A typed relation $\Re$ is {\em the HO bisimulation} if 
for all $\horel{\Gamma}{\Delta_1}{P_1}{\ \Re \ }{\Delta_2}{Q_1}$ 
\begin{enumerate}[1)]
\item 
whenever 
$\horel{\Gamma}{\Delta_1}{P_1}{\hby{\news{\tilde{m_1}} \bactout{n}{V_1}}}{\Delta_1'}{P_2}$, there exits 
$\horel{\Gamma}{\Delta_2}{Q_1}{\Hby{\news{\tilde{m_2}} \bactout{n}{V_2}}}{\Delta_2'}{Q_2}$ such that, for fresh $t$, 
\[
\begin{array}{lrlll}
\Gamma; \Delta''_1  \proves  {\newsp{\tilde{m_1}}{P_2 \Par 
\htrigger{t}{V_1}}}
\ \Re 
\ \Delta''_2 \proves {\newsp{\tilde{m_2}}{Q_2 \Par \htrigger{t}{V_2}}}
\end{array}
\]
		\item	
for all $\horel{\Gamma}{\Delta_1}{P_1}{\hby{\ell}}{\Delta_1'}{P_2}$ such that 
$\ell$ is not an output, 
 there exists $Q_2$ such that 
$\horel{\Gamma}{\Delta_2}{Q_1}{\Hby{\ell}}{\Delta_2'}{Q_2}$
			and
			$\horel{\Gamma}{\Delta_1'}{P_2}{\ \Re \ }{\Delta_2'}{Q_2}$; and 

                      \item	The symmetric cases of 1 and 2.                
	\end{enumerate}
	The largest such bisimulation
	is called Higher-Order bisimilarity \jpc{and} denoted by $\hwb$.
\end{definition}

\smallskip 

\noi The characteristic bisimulation is given using 
characteristic trigger processes. 

\smallskip 

\begin{definition}[Characteristic Bisimulation]\rm
\label{d:fwb}
The first order bisimilarity, denoted by $\fwb$, is defined \jpc{by} replacing 
Clause 1) in \defref{d:hbw} with the following clause:\\[1mm]
whenever 
$\horel{\Gamma}{\Delta_1}{P_1}{\hby{\news{\tilde{m_1}} \bactout{n}{V_1}}}{\Delta_1'}{P_2}$ with $\Gamma; \es; \Delta \proves V_1 \hastype U$,  
there exits 
$\horel{\Gamma}{\Delta_2}{Q_1}{\Hby{\news{\tilde{m_2}}\bactout{n}{V_2}}}{\Delta_2'}{Q_2}$ with $\Gamma; \es; \Delta' \proves V_2 \hastype U$,  
such that, for fresh $t$, \\[1mm]
$\begin{array}{lrlll}
\Gamma; \Delta''_1  \proves  {\newsp{\tilde{m_1}}{P_2 \Par 
\ftrigger{t}{V_1}{U_1}}}
\ \Re 
\ \Delta''_2 \proves {\newsp{\tilde{m_2}}{Q_2 \Par \ftrigger{t}{V_2}{U_2}}}
\end{array}
$
\end{definition}

\smallskip 

\noi Below we state the main theorem.

\smallskip 

\begin{theorem}[Coincidence]\rm
	\label{the:coincidence}
$\cong$, $\wbc$, $\hwb$ and $\fwb$ coincide in $\CAL\in \{\HOp, \HO\}$
and 
$\cong$, $\wbc$ and $\fwb$ coincide in $\CAL\in \{\HOp, \HO, \sessp\}$. 
\end{theorem}

\smallskip 

\noi The above theorem shows that using $\hwb$ is the most tractable 
in the higher-order setting, but if the calculus is limited into 
%the $\sessp$-calculus, 
\jpc{\sessp}
we can still use $\fwb$. 


\smallskip  

\noi Processes that do not use shared names, are inherently deterministic. 
The following \jpc{determinacy property will be} useful 
\jpc{in formalizing our expressiveness results (\S\,\ref{sec:positive})}:
%for both positive and negative results. 

%\smallskip 

\begin{lemma}[$\tau$-Inertness]\rm
	\label{lem:tau_inert}
	\begin{enumerate}[1)]
		\item (deterministic transitions) 
		Transition $\horel{\Gamma}{\Delta}{P}{\hby{\tau}}{\Delta'}{P'}$ is called
		{\em deterministic} if it is derived by $\ltsrule{App}$ or 
		$\ltsrule{Tau}$ where $\subj{\ell_1}$ and $\subj{\ell_2}$ in the premise 
		are dual session names. Suppose $\Delta$ is balanced. Then 
		$\Gamma; \Delta \proves P \cong \Delta'\proves P'$ 
		with $\Delta \red^\ast \Delta'$ balanced. 
		\item 
		Let $P$ is the $\HOp^{-\mathsf{sh}}$-calculus. 
		Assume $\Gamma; \emptyset; \Delta \proves P \hastype \Proc$. Then 
		$P \red^\ast P'$ implies $\Gamma; \Delta \proves 
		P \cong \Delta'\proves P'$ with $\Delta \red^\ast \Delta'$. 
	\end{enumerate}
\end{lemma}


%\smallskip 


%\begin{IEEEproof}
%	The full details of the proof are in Appendix~\ref{app:sub_coinc}.
%	The theorem is split into a hierarchy of Lemmas. Specifically
%	Lemma~\ref{lem:wb_eq_wbf} proves 
%	$\wb$ coincides with $\fwb$; 
%	Lemma~\ref{lem:wb_is_wbc} exploits the process substitution result
%	(Lemma~\ref{lem:proc_subst}) to prove that $\hwb \subseteq \wbc$.
%	Lemma~\ref{lem:wbc_is_cong} shows that $\wbc$ is a congruence
%	which implies $\wbc \subseteq \cong$.
%	The final result comes from Lemma~\ref{lem:cong_is_wb} where
%	we use label testing to show that $\cong \subseteq \fwb$ using
%	the technique in developed in~\cite{Hennessy07}. The formulation of input
%	triggers in the bisimulation relation allows us to prove
%	the latter result without using a matching operator.
%\end{IEEEproof}

%\smallskip 

%\noi Processes that do not use shared names, are inherently $\tau$-inert.

%\smallskip 

%\begin{lemma}[$\tau$-inertness]\rm
%	\label{lem:tau_inert}
%	Let $P$ is the $\HOp^{-\mathsf{sh}}$-calculus. 
%Assume $\Gamma; \emptyset; \Delta \proves P \hastype \Proc$. Then 
%$P \red^\ast P'$ implies $\Gamma; \Delta \proves 
%P \cong \Delta'\proves P'$ with $\Delta \red^\ast \Delta'$. 
%\end{lemma}


%\begin{IEEEproof}
%	The proof is relied on the fact that processes of the
%	form $\Gamma; \es; \Delta \proves_s \bout{s}{V} P_1 \Par \binp{k}{x} P_2$
%	cannot have any typed transition observables and the fact
%	that bisimulation is a congruence.
%	See details in Appendix~\ref{app:sub_tau_inert}.
%	\qed
%\end{IEEEproof}






\section{Related Work}
\label{sec:relwork}
% !TEX root = main.tex

 Since types can limit
contexts (environments) where processes can interact, typed equivalences
usually offer {\em coarser} semantics than in untyped semantics. 
Pierce and Sangiorgi~\cite{PiSa96b} demonstrated that IO-subtyping can equate 
the optimal encoding of the $\lambda$-calculus by Milner which was not 
in the untyped polyadic $\pi$-calculus \cite{MilnerR:funp}. 
After~\cite{PiSa96b}, many works on typed $\pi$-calculi 
have investigated correctness of encodings of known concurrent and
sequential calculi in order to examine semantic
effects of proposed typing systems. 

A  type discipline closely related
to session types is a family of linear typing systems. Kobayashi, Pierce, and Turner~\cite{LinearPi} first proposed a linearly typed reduction-closed, barbed congruence and 
reasoned a tail-call optimisation of higher-order functions which are
encoded 
as processes. 
Yoshida~\cite{Yoshida96} 
used a bisimulation of graph-based types to prove the full abstraction
of encodings of the polyadic synchronous $\pi$-calculus into the
monadic synchronous $\pi$-calculus. 
Later typed equivalences of a
family of linear and affine calculi \cite{BHY,DBLP:journals/iandc/YoshidaBH04,BergerHY05} 
were used to encode 
PCF \cite{Plotkin1977223,Milner19771}, the simply typed $\lambda$-calculi with sums and products, and System F \cite{GirardJY:protyp}
fully abstractly (a fully abstract encoding of the $\lambda$-calculi 
was an open problem in \cite{MilnerR:funp}).  
Yoshida, Honda, and Berger~\cite{YHB02} proposed a new bisimilarity
method associated with linear type structure and strong
normalisation. It presented applications to reason secrecy in
programming languages. A subsequent work~\cite{HY02} adapted these results
to a practical direction, proposing new typing
systems for secure higher-order and multi-threaded programming 
languages. 
In these works, typed properties, linearity and liveness, 
play a fundamental role in the analysis. In general, linear types 
are suitable to encode ``sequentiality'' in the sense of 
\cite{HylandJME:fulapi,AbramskyS:fulap}.

 
 {Our work follows 
the 
%principles for
%session type 
behavioural semantics in 
\cite{KYHH2015,KY2015,DBLP:journals/iandc/PerezCPT14}
where a bisimulation is defined on an LTS 
that assumes a session typed
observer.
%The bisimilarity is characterised by the corresponding
%reduction-closed, barbed congruence using techniques derived from~\cite{Hennessy07}.
Our theory for higher-order sessions 
differentiates from 
the work in~\cite{KYHH2015} and \cite{KY2015}, which 
considers  (first-order)
binary and multiparty session types, respectively.
P\'{e}rez et al~\cite{DBLP:journals/iandc/PerezCPT14} studied typed equivalences
for a 
theory of binary sessions based on linear logic,
without shared names.}
%Determinacy properties (confluence, $\tau$-inertness) are proven.



%The theory for higher-order session type quivalences is more challenging than
%their corresponding first-order bisimulation theory.
Our approach %for the higher-order 
to typed equivalences
builds upon techniques developed by Sangiorgi~\cite{SangiorgiD:expmpa,San96H}
and Jeffrey and Rathke~\cite{JeffreyR05}.
%The work %Sangiorgi as part of his Ph.D.~research
%%\cite{San96H,SangiorgiD:expmpa}
%\cite{SangiorgiD:expmpa}
%introduced the first fully-abstract encoding from the higher-order 
%$\pi$-calculus into the $\pi$-calculus. 
%Sangiorgi's encoding is based on the idea of a replicated input-guarded process 
%(a trigger process). 
%%We use a similar  replicated triggered process to encode \HOp into \sessp (\defref{d:enc:hopitopi}).
% Operational correspondence for
%the triggered encoding is shown using a context bisimulation
%with first-order labels.
As we have discussed, although contextual bisimilarity has a satisfactory discriminative power,
its use is hindered by the universal quantification on output.
To deal with this, 
Sangiorgi proposes \emph{normal bisimilarity}, 
a tractable  equivalence without universal quantification. 
To prove that context and normal bisimilarities coincide,~\cite{SangiorgiD:expmpa} uses 
triggered processes.
%The encoding also motivates the definition of a form of
Triggered bisimulation is also defined on first-order labels
where the context bisimulation is restricted to arbitrary
trigger substitution. %rather than arbitrary process substitutions.
This
characterisation of context bisimilarity  was refined in~\cite{JeffreyR05} for
calculi with recursive types, not addressed in~\cite{San96H,SangiorgiD:expmpa} and
quite relevant in %our work (cf. \defref{d:enc:hopitoho}).
session-based concurrency.
The
bisimulation in~\cite{JeffreyR05}
is based on an LTS  extended with trigger meta-notation.
%for a full higher-order $\pi$-calculus that allows
%higher-order applications.
As in~\cite{San96H,SangiorgiD:expmpa}, 
the LTS in~\cite{JeffreyR05}
observes first-order triggered values instead of
higher-order values, offering a more direct characterisation of contextual equivalence
and lifting the restriction to finite types.
\emph{Environmental bisimulations}~\cite{DBLP:conf/lics/SangiorgiKS07} 
%which 
%Sangiorgi et al.~\cite{DBLP:conf/lics/SangiorgiKS07}, 
use a higher-order LTS 
to define a bisimulation that stores the observer's knowledge; hence, observed actions are based on this knowledge
at any given time. This approach is enhanced in~\cite{DBLP:journals/cl/KoutavasH12}
with a mapping from constants to higher-order values. This 
allows to observe first-order values instead
of higher-order values. It differs from~\cite{San96H,JeffreyR05} in that 
the mapping between higher- and first-order values is no longer implicit.

\paragraph{Comparison with respect to~\cite{JeffreyR05}.} 
We briefly contrast 
the approach in~\cite{JeffreyR05} and ours based on 
%\dk{higher-order ($\hwb$) and} 
characteristic  bisimilarity ($\fwb$):
\begin{enumerate}[$\bullet$]
%\begin{enumerate}[i.]
\item 
The LTS in~\cite{JeffreyR05} is enriched with extra labels for triggers;
an output action transition emits a trigger and introduces a parallel replicated trigger.
Our 
approach retains usual labels/transitions; in  case of output,
%our bisimilarities 
%$\hwb$ and 
$\fwb$
introduces a parallel
\emph{non-replicated} trigger.

\item Higher-order input in~\cite{JeffreyR05} involves 
the input of a trigger which reduces after substitution.
Rather than a trigger name, %our bisimulations  
%$\hwb$ and 
$\fwb$
decrees the input of a trigger value $\abs{z}\binp{t}{x} (\appl{x}{z})$.

\item Unlike~\cite{JeffreyR05}, 
%our 
$\fwb$ treats  
first- and higher-order values uniformly. %In the latter case, 
%Since the 
As the typed LTS distinguishes linear and shared values,
replicated closures are used only for shared values.

\item In~\cite{JeffreyR05}   name matching   is
crucial to prove completeness of bisimilarity.
In our case, \HOp lacks name matching and 
%Contrarily 
%\jpc{In contrast,} 
we use session types: a characteristic value inhabiting a type enables the simplest form of interactions with the environment.

%We use the characteristic process interaction with the environment, exploiting session types.
%, i.e., instead of matching a name, we embed it into a process and then observe its behaviour.

%In~\cite{JeffreyR05}  a matching construct 
%is crucial to prove completeness of bisimilarity.
%Since our language lacks matching,
%we use session type information to obtain the simplest value that 
%enables interaction with the environment.
\end{enumerate}
%\noi 
We compare our approach to that in~\cite{JeffreyR05} 
using a representative example.
%We considered the transitions and resulting processes involved in checking bisimilarity of process 
%$\bout{n}{\abs{x}{\appl{x}(\abs{y}{\bout{y}{m}} \inact)}} \inact$
%with itself.

%\section{Example}


\begin{example}
	Consider typed process
\[
	\Gamma; \es; \Delta \cat n: \btout{U} \tinact \proves \bout{n}{\abs{x}{\appl{x}(\abs{y}{\bout{y}{m}} \inact)}} \inact \hastype \Proc
\]
	with $U = \shot{(\shot{(\shot{(\btout{S} \tinact)})})}$
	We will count the number of steps require to check the bisimilarity
	of the above process with it self. Note that
%
	\begin{eqnarray*}
		\mapchar{\btinp{U} \inact}{s}\\
		= && \binp{s}{x} \mapchar{\shot{(\shot{(\shot{(\btout{S} \tinact)})})}}{x}\\
		= && \binp{s}{x} \appl{x}{\omapchar{\shot{(\shot{(\btout{S} \tinact)})}}}\\
		= && \binp{s}{x} \appl{x}{(\abs{y}{\mapchar{\shot{(\btout{S} \tinact)}}{y}})}\\
		= && \binp{s}{x} \appl{x}{(\abs{y}{\appl{y}{\omapchar{\btout{S} \tinact}})}}\\
		= && \binp{s}{x} \appl{x}{(\abs{y}{\appl{y}{a})}}\\
	\end{eqnarray*}
%
	The characteristic process of
	the type $\shot{(\shot{(\shot{(\btout{S} \tinact)})})}$ is

	\begin{tabular}{rrl}
		(1)& & $\Gamma; \es; \Delta \cat n: \btout{U} \tinact \proves \bout{n}{\abs{x}{\appl{x}(\abs{y}{\bout{y}{m}} \inact)}} \inact$ \\
		&$\by{\bactout{n}{\abs{x}{\appl{x}(\abs{y}{\bout{y}{m}} \inact)}}}$& $\Gamma; \es; \Delta \proves \inact$\\
		(2) & & $\Gamma; \es; \Delta \proves \binp{t}{z} \newsp{s}{\mapchar{\btinp{U} \inact}{s} \Par \bout{\dual{s}}{\abs{x}{\appl{x}(\abs{y}{\bout{y}{m}} \inact)}} \inact}$\\
		&$=$& $\Gamma; \es; \Delta \proves \binp{t}{z} \newsp{s}{\binp{s}{x} \appl{x}{(\abs{y}{\appl{y}{a})}} \Par \bout{\dual{s}}{\abs{x}{\appl{x}(\abs{y}{\bout{y}{m}} \inact)}} \inact}$\\
		(3) &$\by{\bactinp{t}{b}}$& $\Gamma; \es; \Delta \proves \newsp{s}{\binp{s}{x} \appl{x}{(\abs{y}{\appl{y}{a})}} \Par \bout{\dual{s}}{\abs{x}{\appl{x}(\abs{y}{\bout{y}{m}} \inact)}} \inact}$\\
		(4) &$\by{\tau}$& $\Gamma; \es; \Delta \proves \appl{\abs{x}{\appl{x}(\abs{y}{\bout{y}{m} \inact})}}{(\abs{y}{\appl{y}{a})}}$\\
		(5) &$\by{\tau}$& $\Gamma; \es; \Delta \proves \appl{(\abs{y}{\appl{y}{a}})}{(\abs{y}{\bout{y}{m} \inact})} $\\
		(6) &$\by{\tau}$& $\Gamma; \es; \Delta \proves \appl{(\abs{y}{\bout{y}{m} \inact})}{a}$\\
		(7) &$\by{\tau}$& $\Gamma; \es; \Delta \proves \bout{a}{m} \inact$
	\end{tabular}

On Jeffrey and Rathke Semantics:

	\begin{tabular}{rrl}
		(1)& & $\Gamma; \es; \Delta \cat n: \btout{U} \tinact \proves \bout{n}{\abs{x}{\appl{x}(\abs{y}{\bout{y}{m}} \inact)}} \inact$ \\
		&$\by{\bactout{n}{\abs{x}{\appl{x}(\abs{y}{\bout{y}{m}} \inact)}}}$& $\Gamma; \es; \Delta \proves \inact$\\
		(2) & & $\Gamma; \es; \Delta \proves \repl{} \binp{t}{x} \appl{x}(\abs{y}{\bout{y}{m}} \inact) $\\
		(3) &$\by{\bactinp{t}{\tau_l}}$& $\Gamma; \es; \Delta \proves \repl{} \binp{t}{x} \appl{x}(\abs{y}{\bout{y}{m}} \inact) \Par \appl{\abs{x}{\appl{x}(\abs{y}{\bout{y}{m}} \inact)}}{\tau_l}$\\
		(4) &$\by{\tau}$& $\Gamma; \es; \Delta \proves \repl{} \binp{t}{x} \appl{x}(\abs{y}{\bout{y}{m}} \inact) \Par \appl{\tau_l}{(\abs{y}{\bout{y}{m}} \inact)}$\\
		(5) &$\by{\bactout{l}{\tau_k}}$& $\Gamma; \es; \Delta \proves \repl{} \binp{t}{x} \appl{x}(\abs{y}{\bout{y}{m}} \inact) \Par \repl{} \binp{k}{y} \bout{y}{m} \inact $\\
		(6) &$\by{\bactinp{k}{a}}$& $\Gamma; \es; \Delta \proves \repl{} \binp{t}{x} \appl{x}(\abs{y}{\bout{y}{m}} \inact) \Par \repl{} \binp{k}{y} \bout{y}{m} \inact \Par \appl{\abs{y}{\bout{y}{m} \inact}}{a}$\\
		(7) &$\by{\tau}$& $\Gamma; \es; \Delta \proves \repl{} \binp{t}{x} \appl{x}(\abs{y}{\bout{y}{m}} \inact) \Par \repl{} \binp{k}{y} \bout{y}{m} \inact \Par \bout{a}{m} \inact$
	\end{tabular}


	Comparing
	\begin{itemize}
		\item	Same number of transitions.
		\item	J \& R more observable actions.
		\item	J \& R replicated processes.
	\end{itemize}

\end{example}




The previous comparison %, detailed in \appref{app:jandr}, 
reveals that our approach 
%based on %even if both techniques require the same number of transitions, 
%a refined LTS and characteristic bisimilarity 
requires less visible transitions and replicated processes. 
Therefore, linearity information does simplify analyses, 
as it enables simpler witnesses in  coinductive proofs.





%There are similarities and differences between of the characteristic bisimulation
%and the bisimulation as defined by Jeffrey and Rathke
%(below we use the meta-notation adopted in~\cite{JeffreyR05}):
%%
%\begin{enumerate}[i)]
%	\item	The output of a higher-order value $\abs{x}{Q}$ on name
%		$n$ in Jeffrey\&Rathke approach requires the output of
%		a fresh trigger name $t$ (notation $\tau_t$)
%		on name $n$ 
%		and then the introduction of a replicated triggered process
%		(notation $(t \Leftarrow (x) Q)$)
%		in the context of the acting process:
%		%
%		\[
%			P \by{\news{t} \bactout{n}{\tau_{t}}} P' \Par (t \Leftarrow (x) Q) \by{\bactinp{t}{v}} P' \Par \appl{(x) Q}{v} \Par (t \Leftarrow (x) Q) 
%		\]
%		%
%		In the characteristic bisimulation approach we only observe
%		an output of a value that can be either first- or higher-order:
%		%
%		\[
%			P \hby{\bactout{n}{V}} P' 
%		\]
%		%
%		with $V = \abs{x}{Q}$ or $V = m$.
%		A non-replicated triggered process appears in
%		the parallel context of the acting process when
%		we compare two processes for behavioural equality
%		(cf.~characteristic bisimulation \defref{d:fwb}),
%		$P' \Par \htrigger{t}{\abs{x}{Q}}$.
%		In fact using the LTS in
%		\defref{d:tlts} we can get:
%		%
%		\begin{eqnarray*}
%			P' \Par \htrigger{t}{\abs{x}{Q}}
%			&\by{\abs{z}{\binp{z}{y} \repl{} \binp{t}{x} \appl{y}{x}}}&
%			P' \Par \newsp{s}{\binp{s}{y} \repl{} \binp{t}{x} \appl{y}{x} \Par \bout{s}{\abs{x}{Q}} \inact}\\
%			&\by{\tau}&
%			P' \Par \repl{}\binp{t}{y} \appl{\abs{x}{Q}}{y}
%		\end{eqnarray*}
%		%
%		that simulates the Jeffrey\&Rathke approach.
%
%		The characteristic bisimulation differentiates from
%		the Jeffrey\&Rathke approach:
%		\begin{enumerate}[$\bullet$]
%			\item	The typed LTS predicts the case of linear
%				output values and will never allow replication
%				of such a value;
%				if $V$ is linear the input action would have no replication
%				operator, as
%				$\abs{z}{\binp{z}{y} \binp{t}{x} \appl{y}{x}}$.
%
%			\item	The characteristic bisimulation introduces a uniform approach
%				not only for
%				higher-order values but for first-order values
%				as well. A triggered process can accept any
%				process that can substitute a first-order value as well.
%				This is derived from the fact that the $\HOp$
%				calculus makes no use of a matching operator, in contrast
%				to the calculus defined in~\cite{JeffreyR05})
%				where name matching is crucial to prove completness
%				of the bisimilarity relation.
%				We chose not to include the matching operator
%				because of the requirement of a minimal calculus.
%				In the lack of matching we use types to inhabit
%				a value so we can observe its simplest interaction
%				with the process environment.
%
%			\item	The \HOp calculus requires only first-order
%				applications. Higher-order applications,
%				as in the Jeffrey\&Rathke work,
%				are presented as an extension in the \HOpp
%				calculus.
%
%			\item	The trigger process is non-replicated. In fact
%				the trigger process transforms guards the output
%				value with a higher-order input prefix. The
%				functionality of the input is used to
%				simulate the contextual bisimilarity that subsumes
%				the replicated trigger approach.
%				The transformation of an output action as an input
%				action allows for treating an output
%				using the restricted LTS \defref{def:rlts}:
%				%
%				\[
%					P' \Par \htrigger{t}{\abs{x}{Q}} \hby{\bactinp{t}{\abs{x}{\mapchar{U}{x}}}}
%					P' \Par \news{s}{ \appl{\mapchar{U}{x}}{s} \Par \bout{\dual{s}}{\abs{x}{Q}} \inact}
%				\]
%		\end{enumerate}
%		%
%		%In essence we are transforming a replicated trigger into a process
%		%that is input-prefixed on a fresh name that receives a higher-order
%		%value;
%
%	\item	The input of a higher-order value in the Jeffrey\&Rathke approach requires 
%		the input of a fresh trigger name, which is substituted on the application
%		variable, thus having a meta-suntax for triggered application instead
%		of higher-order applications:
%		%
%		\[
%			\binp{n}{x} P \by{\bactinp{n}{\tau_k}} \appl{\abs{x}{P}}{\tau_k} \by{\tau} P \subst{x}{\tau_k} 
%		\]
%		%
%		with every instance of process variable $x$ in $P$ being substituted
%		with trigger value $\tau_k$ to give a process of the form $\appl{\tau_k}{x}$.
%		The approach in the characteristic bisimulation observes the
%		triggered value
%		$\abs{z}\binp{t}{x} \appl{x}{z}$ as an input instead of the
%		trigger name:
%		%
%		\[
%			P \hby{\bactinp{n}{\abs{z}\binp{t}{x} \appl{x}{z}}} P \subst{\abs{z}\binp{t}{x} \appl{x}{z}}{x}
%		\]
%		%
%		with applications being transformed to
%		$\abs{z}{\binp{t}{x} \appl{x}{z}}{v}$
%		Note that in the characteristic bisimulation semantics
%		we can also observe a characteristic process as an input.
%		
%	\item 	Triggered application in the Jeffrey\&Rahtke
%		are observe using an output
%		lead into an output observation of the
%		application value over
%		the fresh trigger name.
%		%
%		\[
%			\appl{\tau_k}{v} \by{\bactout{k}{v}} \inact
%		\]
%		%
%		In the characteristic bisimulation instead of observing an 
%		application and its value as an action we observe:
%		i) the name of the trigger through the trigger value
%		application; and ii) the application
%		value by inhabiting it in the characteristic value
%		and observing the interaction of the corresponding
%		characteristic process with its environment.
%		%
%		\begin{eqnarray*}
%			\appl{\abs{z}{\binp{t}{x} \appl{x}{z}}}{v} &\by{\tau}& \binp{t}{x} \appl{x}{v}
%			\by{\bactinp{t}{\abs{x}{\mapchar{U}{x}}}}
%			\appl{\abs{x}{\mapchar{U}{x}}}{v}
%			\by{\tau} \mapchar{U}{x} \subst{n}{x}
%		\end{eqnarray*}
%		%
%\end{enumerate}

%The main differences of the triggered
%bisimulation approach comparing to our approach are:
%i) We use observe higher-order values on the LTS in contrast to first-order 
%values in~\cite{DBLP:journals/lmcs/JeffreyR05}.
%ii) In our approach we avoid the replicated triggered process,
%by transforming the output process into a higher-order guarded input.
%iii) The triggered bisimulation gives semantics for higher-order application,
%whereas in our approach we give semantics for first-order applications
%and show that higher-order applications are fully encodable.

%Boreale and Sangiorgi, 
%Deng and Hennessy, 
%Jeffrey and Rathke, Hennessy and Koutavas, Schmitt and Lenglet, Pi\E9rard and Sumii.
%Perez et al (bisimilarities for binary sessions), Kouzapas and Yoshida (bisimilarities for binary and multiparty sessions).
%Bisimilarities for HO processes: \cite{Xu07}.












%%%%%%%%%%%%%%%%%%%%%%%%%%%%%%%%%%%%%%%%%%%%%%%%%%%%%%%%%%%%%%%%%%%%%%%%%%%%%
% Bibliography.
%%%%%%%%%%%%%%%%%%%%%%%%%%%%%%%%%%%%%%%%%%%%%%%%%%%%%%%%%%%%%%%%%%%%%%%%%%%%%

%\bibliographystyle{IEEEtran}
%\bibliographystyle{plain}
\bibliographystyle{abbrv}% the recommended bibstyle
{\bibliography{session}}

\newpage
\appendix 
\section{Comparison with Jeffrey and Rathke, By Example}
\label{app:jandr}
%\section{Example}


\begin{example}
	Consider typed process
\[
	\Gamma; \es; \Delta \cat n: \btout{U} \tinact \proves \bout{n}{\abs{x}{\appl{x}(\abs{y}{\bout{y}{m}} \inact)}} \inact \hastype \Proc
\]
	with $U = \shot{(\shot{(\shot{(\btout{S} \tinact)})})}$
	We will count the number of steps require to check the bisimilarity
	of the above process with it self. Note that
%
	\begin{eqnarray*}
		\mapchar{\btinp{U} \inact}{s}\\
		= && \binp{s}{x} \mapchar{\shot{(\shot{(\shot{(\btout{S} \tinact)})})}}{x}\\
		= && \binp{s}{x} \appl{x}{\omapchar{\shot{(\shot{(\btout{S} \tinact)})}}}\\
		= && \binp{s}{x} \appl{x}{(\abs{y}{\mapchar{\shot{(\btout{S} \tinact)}}{y}})}\\
		= && \binp{s}{x} \appl{x}{(\abs{y}{\appl{y}{\omapchar{\btout{S} \tinact}})}}\\
		= && \binp{s}{x} \appl{x}{(\abs{y}{\appl{y}{a})}}\\
	\end{eqnarray*}
%
	The characteristic process of
	the type $\shot{(\shot{(\shot{(\btout{S} \tinact)})})}$ is

	\begin{tabular}{rrl}
		(1)& & $\Gamma; \es; \Delta \cat n: \btout{U} \tinact \proves \bout{n}{\abs{x}{\appl{x}(\abs{y}{\bout{y}{m}} \inact)}} \inact$ \\
		&$\by{\bactout{n}{\abs{x}{\appl{x}(\abs{y}{\bout{y}{m}} \inact)}}}$& $\Gamma; \es; \Delta \proves \inact$\\
		(2) & & $\Gamma; \es; \Delta \proves \binp{t}{z} \newsp{s}{\mapchar{\btinp{U} \inact}{s} \Par \bout{\dual{s}}{\abs{x}{\appl{x}(\abs{y}{\bout{y}{m}} \inact)}} \inact}$\\
		&$=$& $\Gamma; \es; \Delta \proves \binp{t}{z} \newsp{s}{\binp{s}{x} \appl{x}{(\abs{y}{\appl{y}{a})}} \Par \bout{\dual{s}}{\abs{x}{\appl{x}(\abs{y}{\bout{y}{m}} \inact)}} \inact}$\\
		(3) &$\by{\bactinp{t}{b}}$& $\Gamma; \es; \Delta \proves \newsp{s}{\binp{s}{x} \appl{x}{(\abs{y}{\appl{y}{a})}} \Par \bout{\dual{s}}{\abs{x}{\appl{x}(\abs{y}{\bout{y}{m}} \inact)}} \inact}$\\
		(4) &$\by{\tau}$& $\Gamma; \es; \Delta \proves \appl{\abs{x}{\appl{x}(\abs{y}{\bout{y}{m} \inact})}}{(\abs{y}{\appl{y}{a})}}$\\
		(5) &$\by{\tau}$& $\Gamma; \es; \Delta \proves \appl{(\abs{y}{\appl{y}{a}})}{(\abs{y}{\bout{y}{m} \inact})} $\\
		(6) &$\by{\tau}$& $\Gamma; \es; \Delta \proves \appl{(\abs{y}{\bout{y}{m} \inact})}{a}$\\
		(7) &$\by{\tau}$& $\Gamma; \es; \Delta \proves \bout{a}{m} \inact$
	\end{tabular}

On Jeffrey and Rathke Semantics:

	\begin{tabular}{rrl}
		(1)& & $\Gamma; \es; \Delta \cat n: \btout{U} \tinact \proves \bout{n}{\abs{x}{\appl{x}(\abs{y}{\bout{y}{m}} \inact)}} \inact$ \\
		&$\by{\bactout{n}{\abs{x}{\appl{x}(\abs{y}{\bout{y}{m}} \inact)}}}$& $\Gamma; \es; \Delta \proves \inact$\\
		(2) & & $\Gamma; \es; \Delta \proves \repl{} \binp{t}{x} \appl{x}(\abs{y}{\bout{y}{m}} \inact) $\\
		(3) &$\by{\bactinp{t}{\tau_l}}$& $\Gamma; \es; \Delta \proves \repl{} \binp{t}{x} \appl{x}(\abs{y}{\bout{y}{m}} \inact) \Par \appl{\abs{x}{\appl{x}(\abs{y}{\bout{y}{m}} \inact)}}{\tau_l}$\\
		(4) &$\by{\tau}$& $\Gamma; \es; \Delta \proves \repl{} \binp{t}{x} \appl{x}(\abs{y}{\bout{y}{m}} \inact) \Par \appl{\tau_l}{(\abs{y}{\bout{y}{m}} \inact)}$\\
		(5) &$\by{\bactout{l}{\tau_k}}$& $\Gamma; \es; \Delta \proves \repl{} \binp{t}{x} \appl{x}(\abs{y}{\bout{y}{m}} \inact) \Par \repl{} \binp{k}{y} \bout{y}{m} \inact $\\
		(6) &$\by{\bactinp{k}{a}}$& $\Gamma; \es; \Delta \proves \repl{} \binp{t}{x} \appl{x}(\abs{y}{\bout{y}{m}} \inact) \Par \repl{} \binp{k}{y} \bout{y}{m} \inact \Par \appl{\abs{y}{\bout{y}{m} \inact}}{a}$\\
		(7) &$\by{\tau}$& $\Gamma; \es; \Delta \proves \repl{} \binp{t}{x} \appl{x}(\abs{y}{\bout{y}{m}} \inact) \Par \repl{} \binp{k}{y} \bout{y}{m} \inact \Par \bout{a}{m} \inact$
	\end{tabular}


	Comparing
	\begin{itemize}
		\item	Same number of transitions.
		\item	J \& R more observable actions.
		\item	J \& R replicated processes.
	\end{itemize}

\end{example}




\section{The Typing System of \HOp}
\label{app:types}
% !TEX root = main.tex

%In this appendix 
We 
first 
formally define \emph{type equivalence}. % and \emph{duality}. Then, we  
%present and describe our typing rules, given in~\figref{fig:typerulesmy}.
%Finally, we 
Then we 
give details of the proof of Theorem~\ref{t:sr} (page~\pageref{t:sr}).
%\smallskip
%\subsection{Type Equivalence and Duality}
%\begin{definition}[Type Equivalence]
%\label{def:iso}
%Let $\mathsf{ST}$ a set of closed session types. 
%Two types $S$ and $S'$ are said to be {\em isomorphic} if a pair $(S,S')$ is 
%in the largest fixed point of the monotone function
%$F:\mathcal{P}(\mathsf{ST}\times \mathsf{ST}) \to 
%\mathcal{P}(\mathsf{ST}\times \mathsf{ST})$ defined by: \\
%%\hspace{-0.5cm}
%\begin{tabular}{rcl}
%$F(\Re)$ &$\!\!=\!\!$&	$\set{(\tinact, \tinact)}$\\
%         &$\!\!\cup\!\!$&	$\set{(\btout{U_1} S_1, \btout{U_2} S_2)
%\bnfbar (S_1, S_2),(U_1, U_2)\in \Re}$\\ 
%       &$\!\!\cup\!\!$&	$\set{(\btinp{U_1} S_1, \btinp{U_2} S_2)
%\bnfbar(S_1, S_2),(U_1, U_2)\in \Re}$\\ 
%	&$\!\!\cup\!\!$&	$\set{(\btbra{l_i: S_i}_{i \in I} \,,\, \btbra{l_i: S_i'}_{i \in I}) \bnfbar \forall i\in I. (S_i, S_i')\in \Re}$\\
%	&$\!\!\cup\!\!$&	$\set{(\btsel{l_i: S_i}_{i \in I}\,,\, \btsel{l_i: S_i'}_{i \in I}) \bnfbar \forall i\in I. (S_i, S_i')\in \Re}$\\
%	&$\!\!\cup\!\!$&	$\set{(\trec{t}{S}, S')
%\bnfbar (S\subst{\trec{t}{S}}{\vart{t}},S')\in \Re}$\\
%	&$\!\!\cup\!\!$&	$\set{(S,\trec{t}{S'})
%\bnfbar (S,S'\subst{\trec{t}{S'}}{\vart{t}})\in \Re}$
%\end{tabular}
%	
%\noindent
%Standard arguments ensure that $F$ is monotone, thus the greatest fixed point
%of $F$ exists. We write $S_1 \sim S_2$ if  $(S_1,S_2)\in \Re$. 
%\end{definition}
%
%\smallskip 
%
%\begin{definition}[Duality]
%\label{def:dual}
%Let $\mathsf{ST}$ a set of closed session types. 
%Two types $S$ and $S'$ are said to be {\em dual} if a pair $(S,S')$ is 
%in the largest fixed point of the monotone function
%$F:\mathcal{P}(\mathsf{ST}\times \mathsf{ST}) \to 
%\mathcal{P}(\mathsf{ST}\times \mathsf{ST})$ defined by:\\ %[1mm]
%\begin{tabular}{rcl}
%$F(\Re)$ &$\!\!=\!\!$&	$\set{(\tinact, \tinact)}$\\
%         &$\!\!\cup\!\!$&	$\set{(\btout{U_1} S_1, \btinp{U_2} S_2)
%\bnfbar(S_1, S_2)\in \Re, \  U_1 \sim U_2 }$\\ 
%       &$\!\!\cup\!\!$&	$\set{(\btinp{U_1} S_1, \btout{U_2} S_2)
%\bnfbar(S_1, S_2)\in \Re, \ U_1 \sim U_2}$\\ 
%	&$\!\!\cup\!\!$&	$\set{(\btsel{l_i: S_i}_{i \in I} \,,\, \btbra{l_i: S_i'}_{i \in I}) \bnfbar \forall i\in I. (S_i, S_i')\in \Re}$\\
%	&$\!\!\cup\!\!$&	$\set{(\btbra{l_i: S_i}_{i \in I}\,,\, \btsel{l_i: S_i'}_{i \in I}) \bnfbar \forall i\in I. (S_i, S_i')\in \Re}$\\
%	&$\!\!\cup\!\!$&	$\set{(\trec{t}{S}, S')
%\bnfbar (S\subst{\trec{t}{S}}{\vart{t}},S')\in \Re}$\\
%	&$\!\!\cup\!\!$&	$\set{(S,\trec{t}{S'})
%\bnfbar (S,S'\subst{\trec{t}{S'}}{\vart{t}})\in \Re}$\\[1mm]
%\end{tabular}
%
%\noindent
%%where $U_1 \sim U_2$ means $U_1$ is type equivalent to $U_2$ \cite{yoshida.vasconcelos:language-primitives}.
%Standard arguments ensure that $F$ is monotone, thus the greatest fixed point
%of $F$ exists. We write $S_1 \dualof S_2$ if  $(S_1,S_2)\in \Re$. 
%\end{definition}
%
%\smallskip 
%\subsection{Typing Rules}
%% !TEX root = ../journal16kpy.tex


\begin{figure}[t]
\[
	\begin{array}{c}
		\trule{Sess}~~\Gamma; \emptyset; \set{u:S} \proves u \hastype S 
		\qquad
		\trule{Sh}~~\Gamma \cat u : U; \emptyset; \emptyset \proves u \hastype U
		\qquad
		\trule{LVar}~~\Gamma; \set{x: \lhot{C}}; \emptyset \proves x \hastype \lhot{C}
		\\[4mm]

		\trule{Prom}~~\tree{
			\Gamma; \emptyset; \emptyset \proves V \hastype 
                         \lhot{C}
		}{
			\Gamma; \emptyset; \emptyset \proves V \hastype 
                         \shot{C}
		} 
		\qquad
		\trule{EProm}~~\tree{
		\Gamma; \Lambda \cat x : \lhot{C}; \Delta \proves P \hastype \Proc
		}{
			\Gamma \cat x:\shot{C}; \Lambda; \Delta \proves P \hastype \Proc
		}
		\\[4mm]

		\trule{Abs}~~\tree{
			\Gamma; \Lambda; \Delta_1 \proves P \hastype \Proc
			\quad
			\Gamma; \es; \Delta_2 \proves x \hastype C
		}{
			\Gamma\backslash x; \Lambda; \Delta_1 \backslash \Delta_2 \proves \abs{{x}}{P} \hastype \lhot{{C}}
		}
		\\[4mm]

		\trule{App}~~\tree{
			\begin{array}{c}
				U = \lhot{C} \lor \shot{C}
				\quad
				\Gamma; \Lambda; \Delta_1 \proves V \hastype U
				\quad
				\Gamma; \es; \Delta_2 \proves u \hastype C
			\end{array}
		}{
			\Gamma; \Lambda; \Delta_1 \cat \Delta_2 \proves \appl{V}{u} \hastype \Proc
		} 
		\\[4mm]

		\trule{Send}~~\tree{
			\Gamma; \Lambda_1; \Delta_1 \proves P \hastype \Proc
			\quad
			\Gamma; \Lambda_2; \Delta_2 \proves V \hastype U
			\quad
			u:S \in \Delta_1 \cat \Delta_2
		}{
			\Gamma; \Lambda_1 \cat \Lambda_2; ((\Delta_1 \cat \Delta_2) \setminus u:S) \cat u:\btout{U} S \proves \bout{u}{V} P \hastype \Proc
		}
		\\[4mm]

		\trule{Rcv}~~\tree{
			\Gamma; \Lambda_1; \Delta_1 \cat u: S \proves P \hastype \Proc
			\quad
			\Gamma; \Lambda_2; \Delta_2 \proves {x} \hastype {U}
		}{
			\Gamma \backslash x; \Lambda_1\cat \Lambda_2; \Delta_1\backslash \Delta_2 \cat u: \btinp{U} S \vdash \binp{u}{{x}} P \hastype \Proc
		}
		\\[4mm]

		\trule{Req}~~\tree{
			\begin{array}{c}
				\Gamma; \es; \es \proves u \hastype U_1
				\quad
				\Gamma; \Lambda; \Delta_1 \proves P \hastype \Proc
				\quad
				\Gamma; \es; \Delta_2 \proves V \hastype U_2
				\\
				(U_1 = \chtype{S} 
                                \land %\Leftrightarrow 
                                U_2 = S)
				\lor
				 (U_1 = \chtype{L} 
                                \land %\Leftrightarrow 
                                %\Leftrightarrow 
                                 U_2 = L)
			\end{array}
		}{
			\Gamma; \Lambda; \Delta_1 \cat \Delta_2 \proves \bout{u}{V} P \hastype \Proc
		}
		\\[4mm]

		\trule{Acc}~~\tree{
			\begin{array}{c}
				\Gamma; \emptyset; \emptyset \proves u \hastype U_1 
				\quad
				\Gamma; \Lambda_1; \Delta_1 \proves P \hastype \Proc
				\quad
				\Gamma; \Lambda_2; \Delta_2 \proves x \hastype U_2\\
				(U_1 = \chtype{S} 
                                \land %\Leftrightarrow 
                                U_2 = S)
				\lor
				 (U_1 = \chtype{L} 
                                \land %\Leftrightarrow 
                                %\Leftrightarrow 
                                 U_2 = L)
	               \end{array}
		}{
			\Gamma\backslash x; \Lambda_1 \backslash \Lambda_2; \Delta_1 \backslash \Delta_2 \proves \binp{u}{x} P \hastype \Proc
		}
		\\[4mm]

		\trule{Bra}~~\tree{
			 \forall i \in I \quad \Gamma; \Lambda; \Delta \cat u:S_i \proves P_i \hastype \Proc
		}{
			\Gamma; \Lambda; \Delta \cat u: \btbra{l_i:S_i}_{i \in I} \proves \bbra{u}{l_i:P_i}_{i \in I}\hastype \Proc
		}
		\qquad
	 	\trule{Sel}~~\tree{
			\Gamma; \Lambda; \Delta \cat u: S_j  \proves P \hastype \Proc \quad j \in I

		}{
			\Gamma; \Lambda; \Delta \cat u:\btsel{l_i:S_i}_{i \in I} \proves \bsel{u}{l_j} P \hastype \Proc
		}
		\\[4mm]

		\trule{ResS}~~\tree{
			\Gamma; \Lambda; \Delta \cat s:S_1 \cat \dual{s}: S_2 \proves P \hastype \Proc \quad S_1 \dualof S_2
		}{
			\Gamma; \Lambda; \Delta \proves \news{s} P \hastype \Proc
		}
		\qquad
		\trule{Res}~~\tree{
			\Gamma\cat a:\chtype{S} ; \Lambda; \Delta \proves P \hastype \Proc
		}{
			\Gamma; \Lambda; \Delta \proves \news{a} P \hastype \Proc
		}
		\\[4mm]
 
		\trule{Par}~~\tree{
			\Gamma; \Lambda_{i}; \Delta_{i} \proves P_{i} \hastype \Proc \quad i=1,2
		}{
			\Gamma; \Lambda_{1} \cat \Lambda_2; \Delta_{1} \cat \Delta_2 \proves P_1 \Par P_2 \hastype \Proc
		}
		\qquad
		\trule{End}~~\tree{
			\Gamma; \Lambda; \Delta  \proves P \hastype T \quad u \not\in \dom{\Gamma, \Lambda,\Delta}
		}{
			\Gamma; \Lambda; \Delta \cat u: \tinact  \proves P \hastype \Proc
		}
		\\[4mm]

	 	\trule{Rec}~~\tree{
			\Gamma \cat \rvar{X}: \Delta; \emptyset; \Delta  \proves P \hastype \Proc
		}{
			\Gamma ; \emptyset; \Delta  \proves \recp{X}{P} \hastype \Proc
		}
		\qquad
		\trule{RVar}~~\Gamma \cat \rvar{X}: \Delta; \emptyset; \Delta  \proves \rvar{X} \hastype \Proc
		\qquad
		\trule{Nil}~~\Gamma; \emptyset; \emptyset \proves \inact \hastype \Proc
	\end{array}
\]
\caption{Complete Typing Rules for $\HOp$.\label{fig:typerulesmy}}
%\Hline
\end{figure}
%\myparagraph{Typing System of \HOp}




%\subsection{Proof of Theorem~\ref{t:sr}}
%We state type soundness of our system.

\subsection{Type Equivalence}
\begin{definition}[Type Equivalence]
\label{def:iso}
Let $\mathsf{ST}$ a set of closed session types. 
Two types $S$ and $S'$ are said to be {\em isomorphic} if a pair $(S,S')$ is 
in the largest fixed point of the monotone function
$F:\mathcal{P}(\mathsf{ST}\times \mathsf{ST}) \to 
\mathcal{P}(\mathsf{ST}\times \mathsf{ST})$ defined by: \\
%\hspace{-0.5cm}
\begin{tabular}{rcl}
$F(\Re)$ &$\!\!=\!\!$&	$\set{(\tinact, \tinact)}$\\
         &$\!\!\cup\!\!$&	$\set{(\btout{U_1} S_1, \btout{U_2} S_2)
\bnfbar (S_1, S_2),(U_1, U_2)\in \Re}$\\ 
       &$\!\!\cup\!\!$&	$\set{(\btinp{U_1} S_1, \btinp{U_2} S_2)
\bnfbar(S_1, S_2),(U_1, U_2)\in \Re}$\\ 
	&$\!\!\cup\!\!$&	$\set{(\btbra{l_i: S_i}_{i \in I} \,,\, \btbra{l_i: S_i'}_{i \in I}) \bnfbar \forall i\in I. (S_i, S_i')\in \Re}$\\
	&$\!\!\cup\!\!$&	$\set{(\btsel{l_i: S_i}_{i \in I}\,,\, \btsel{l_i: S_i'}_{i \in I}) \bnfbar \forall i\in I. (S_i, S_i')\in \Re}$\\
	&$\!\!\cup\!\!$&	$\set{(\trec{t}{S}, S')
\bnfbar (S\subst{\trec{t}{S}}{\vart{t}},S')\in \Re}$\\
	&$\!\!\cup\!\!$&	$\set{(S,\trec{t}{S'})
\bnfbar (S,S'\subst{\trec{t}{S'}}{\vart{t}})\in \Re}$
\end{tabular}

Standard arguments ensure that $F$ is monotone, thus the greatest fixed point
of $F$ exists. We write $S_1 \sim S_2$ if  $(S_1,S_2)\in \Re$. 
\end{definition}


\subsection{Proof of Theorem~\ref{t:sr} (Type Soundness)}
As our type system is closely related to that considered
by Mostrous and Yoshida~\cite{MostrousY15}, the proof of type soundness requires notions
and properties which are instances of those already shown in~\cite{MostrousY15}.
We first state weakening and strengthening lemmas,
which have standard proofs.

%%% Weakening
\begin{lemma}[Weakening - Lemma C.2 in~\cite{MostrousY15}]\rm
	\label{l:weak}
	\begin{enumerate}[$-$]
		\item	If $\Gamma; \Lambda; \Delta \proves P \hastype \Proc$
			and
			$x \not\in \dom{\Gamma,\Lambda,\Delta}$
			then
			$\Gamma\cat x: \shot{U}; \Lambda; \Delta \proves P \hastype \Proc$ 
	\end{enumerate}
\end{lemma}

\begin{lemma}[Strengthening - Lemmas C.3 and C.4 in~\cite{MostrousY15}]\rm
	\label{l:stren}
	We have:
	\begin{enumerate}[$-$]
		\item	If $\Gamma \cat x: \shot{U}; \Lambda; \Delta \proves P \hastype \Proc$
			and
			$x \not\in \fpv{P}$ then
			$\Gamma; \Lambda; \Delta \proves P \hastype \Proc$

		\item	If $\Gamma; \Lambda; \Delta \cat s: \tinact \proves P \hastype \Proc$
			and
			$s \not\in \fn{P}$
			then
			$\Gamma; \Lambda; \Delta \proves P \hastype \Proc$
	\end{enumerate}
\end{lemma}

\begin{lemma}[Substitution Lemma - Lemma C.10 in~\cite{MostrousY15}]\rm
	\label{l:subst}
	We have:
	\begin{enumerate}[1.]
		\item	Suppose $\Gamma; \Lambda; \Delta \cat x:S  \proves P \hastype \Proc$ and
			$s \not\in \dom{\Gamma, \Lambda, \Delta}$. 
			Then $\Gamma; \Lambda; \Delta \cat s:S  \vdash P\subst{s}{x} \hastype \Proc$.

		\item	Suppose $\Gamma \cat x:\chtype{U}; \Lambda; \Delta \proves P \hastype \Proc$ and
			$a \notin \dom{\Gamma, \Lambda, \Delta}$. 
			Then $\Gamma \cat a:\chtype{U}; \Lambda; \Delta   \vdash P\subst{a}{x} \hastype \Proc$.

		\item	Suppose $\Gamma; \Lambda_1 \cat x:\lhot{U}; \Delta_1  \proves P \hastype \Proc$ 
			and $\Gamma; \Lambda_2; \Delta_2  \proves V \hastype \lhot{U}$ with 
			$\Lambda_1, \Lambda_2$ and $\Delta_1, \Delta_2$ defined.  
			Then $\Gamma; \Lambda_1 \cat \Lambda_2; \Delta_1 \cat \Delta_2  \proves P\subst{V}{x} \hastype \Proc$.

		\item	Suppose $\Gamma \cat x:\shot{U}; \Lambda; \Delta  \proves P \hastype \Proc$ and
			$\Gamma; \emptyset ; \emptyset  \proves V \hastype \shot{U}$.
			Then $\Gamma; \Lambda; \Delta  \proves P\subst{V}{x} \hastype \Proc$.
		\end{enumerate}
\end{lemma}

\begin{proof}
	In all four parts, we proceed by induction on the typing for $P$,
	with a case analysis on the last applied rule. 
%	Parts (1) and (2) are standard and therefore omitted. 
%
%	In Part (3), we content ourselves by detailing only the case in
%	which the last applied rule is \trule{App}. 
%	Then we have $P = \appl{V}{u}$. By inversion on the first assumption 
%	we infer:
%	\[
%	\tree{
%	\tree{}{
%	\Gamma;\, \Lambda_1 \cat x:\lhot{C} ;\, \Delta_{11}   \proves V \hastype \lhot{C}} \quad
%	\tree{}{\Gamma;\,  \emptyset   ;\, \Delta_{12}  \proves u \hastype C}
%	}{
%	\Gamma;\,  \Lambda_1 \cat x:\lhot{C};\, \Delta_1    \proves \appl{V}{u} \hastype \Proc}
%	\]
%	where $\Delta_1 = \Delta_{11} \cat \Delta_{12}$.
%	By inversion on the second assumption we infer that either
%	(i)\,$V = y$ (for some   variable $y$) or 
%	(ii)\,$V = \abs{z}Q$, for some $Q$ such that
%%
%	\begin{equation}
%		\Gamma'; \Lambda_1 \cat x:\lhot{C} ; \Delta_{11} \cat \Delta' \proves Q \hastype \Proc \label{eq:subseq2}\\
%	\end{equation}
%%
%	In possibility\,(i), we have a simple substitution on process variables and the thesis follows easily. 
%	In possibility\,(ii), we observe that $P\subst{V}{X} = \appl{X}{\mytilde{k}}\subst{\abs{\mytilde{z}}Q}{X} = Q\subst{\mytilde{k}}{\mytilde{z}}$.
%	The thesis then follows by using Lemma~\ref{lem:subst}\,(1) with 
%	the second premise of the typing of $\appl{X}{\mytilde{k}}$
%	and \eqref{eq:subseq2} above to infer 
%	\begin{equation*}
%		\Gamma; \Lambda_2 ; \Delta_2 \cat \mytilde{k}:\mytilde{C}  \proves Q \subst{\mytilde{k}}{\mytilde{z}} \hastype \Proc .
%	\end{equation*}
%%
%	The proof of Part (4) follows similar lines as that of Part (3).
	\qed
\end{proof}

%\begin{definition}[Well-typed Session Environment]%\rm
%	Let $\Delta$ be a session environment.
%	We say that $\Delta$ is {\em well-typed} if whenever
%	$s: S_1, \dual{s}: S_2 \in \Delta$ then $S_1 \dualof S_2$.
%\end{definition}
%
%\begin{definition}[Session Environment Reduction]%\rm
%	We define the relation $\red$ on session environments as:
%	\begin{enumerate}[$-$]
%		\item	$\Delta \cat s: \btout{U} S_1 \cat \dual{s}: \btinp{U} S_2 \red \Delta \cat s: S_1 \cat \dual{s}: S_2$
%		\item	$\Delta \cat s: \btsel{l_i: S_i}_{i \in I} \cat \dual{s}: \btbra{l_i: S_i'}_{i \in I} \red \Delta \cat s: S_k \cat \dual{s}: S_k', \quad k \in I$.
%	\end{enumerate}
%\end{definition}

We now state the instance of type soundness that we
can derive from~\cite{MostrousY15}.
It is worth noticing 
the 
definition of structural congruence in~\cite{MostrousY15} is richer. 
Also, their statement for subject reduction relies on an 
ordering on typing associated to queues and other 
runtime elements. % (such extended typings are denoted $\Delta$ in~\cite{MostrousY15}).
Since we are working with synchronous communication we can omit such an ordering.

We now prove the following statement; its second part corresponds to \thmref{t:sr} (Page~\pageref{t:sr}):

\begin{theorem}[Type Soundness]\rm\label{t:srfull}
We have:
	\begin{enumerate}[1.]
		\item	(Subject Congruence) Suppose $\Gamma; \Lambda; \Delta \proves P \hastype \Proc$.
			Then $P \scong P'$ implies $\Gamma; \Lambda; \Delta \proves P' \hastype \Proc$.

		\item	(Subject Reduction) Suppose $\Gamma; \es; \Delta \proves P \hastype \Proc$
			with
			balanced $\Delta$. \\
			Then $P \red P'$ implies $\Gamma; \es; \Delta'  \proves P' \hastype \Proc$
			and $\Delta = \Delta'$ or $\Delta \red \Delta'$.

	\end{enumerate}
\end{theorem}

\begin{proof}
	Part (1) is standard, using weakening and strengthening lemmas. Part (2) proceeds by induction on the last reduction rule used. Below, we give some details:
	\begin{enumerate}[1.]
	   \item
	   Case \orule{App}: Then we have
	   $$
	   P = (\abs{x}{Q}) \, u   \red  Q \subst{u}{x} = P'
	   $$
	   Suppose $\Gamma;\, \emptyset ;\, \Delta \proves (\abs{x}{Q}) \, u \hastype \Proc$. 
	   We examine one possible way in which 
	   this assumption can be derived; other cases are similar or simpler:
	   \[
	   \tree{
	   \tree{\Gamma;\, \emptyset ;\, \Delta \cat \{x:S\} \proves Q  \hastype \Proc \quad 
	   \Gamma';\, \emptyset ;\, \{x:S\} \proves x  \hastype S}
	   {
	   \Gamma;\, \emptyset ;\, \Delta \proves \abs{x}{Q}  \hastype \lhot{S} }
	   \qquad
	   \tree{}{
	   \Gamma;\, \emptyset ;\, \{u:S\} \proves   u \hastype S}
	   }{
	   \Gamma;\, \emptyset ;\, \Delta \cat u:S \proves (\abs{x}{Q}) \, u \hastype \Proc
	   }
	   \]
	  Then, by combining premise
	   $\Gamma;\, \emptyset ;\, \Delta \cat \{x:S\} \proves Q  \hastype \Proc$
	   with
	   the substitution lemma (\lemref{l:subst}(1)),
	   we obtain 
	    $\Gamma;\, \emptyset ;\, \Delta \cat u:S \proves Q\subst{u}{x}  \hastype \Proc$, as desired.
	    
	    \item Case \orule{Pass}: 
	    There are several sub-cases, depending on the type of the communication 
	    subject $n$ \newc{(which could be a shared or a linear name)} and the type of the object $V$
	    \newc{(which could be an abstraction or a shared/linear name)}. We analyse two representative sub-cases:
	    
	    \begin{enumerate}[(a)]
	    \item $n$ is a shared name and $V$ is a name $v$. 
	    Then we have the following reduction: 
	    $$
	    P = \bout{n}{v} Q_1 \Par \binp{n}{x} Q_2  \red  Q_1 \Par Q_2 \subst{v}{x} = P'
	    $$
	    By assumption, we have 
	    the following typing derivation:
	    \[	    \hspace{-12mm}
	    \tree{
%	    \tree{
%	     \Gamma' \cat n:\chtype{S};\, \emptyset ;\, \emptyset  \proves n  \hastype \chtype{S}
%	     \quad
%	      \Gamma;\, \emptyset ;\, \Delta_1    \proves   Q_1  \hastype \Proc
%	      \quad
%	       \Gamma;\, \emptyset ;\, \{v:S\}  \proves v  \hastype S	    
%	    }{
%	    \Gamma;\, \emptyset ;\, \Delta_1 \cat \{v:S\}  \proves \bout{n}{v} Q_1  \hastype \Proc
%	    } 
		\eqref{eq:sound1}
	    \quad 
	    		\eqref{eq:sound2}
%	    	    \tree{
%	    \Gamma' \cat n:\chtype{S};\, \emptyset ;\, \emptyset  \proves n  \hastype \chtype{S}
%	     \quad
%	      \Gamma;\, \emptyset ;\, \Delta_3 \cat x:S    \proves   Q_2  \hastype \Proc
%	    }{
%	    \Gamma;\, \emptyset ;\, \Delta_3 \proves  \binp{n}{x} Q_2 \hastype \Proc
%	   }
	    }{
	    \Gamma;\, \emptyset ;\, \Delta_1 \cat \{v:S\} \cat \Delta_3 \proves \bout{n}{v} Q_1 \Par \binp{n}{x} Q_2 \hastype \Proc
	    }
	    \]
	    
	    where \eqref{eq:sound1} and \eqref{eq:sound2} are as follows:
	    \begin{eqnarray}
	      & \tree{
	     \Gamma' \cat n:\chtype{S};\, \emptyset ;\, \emptyset  \proves n  \hastype \chtype{S}
	     \quad
	      \Gamma;\, \emptyset ;\, \Delta_1    \proves   Q_1  \hastype \Proc
	      \quad
	       \Gamma;\, \emptyset ;\, \{v:S\}  \proves v  \hastype S	    
	    }{
	    \Gamma;\, \emptyset ;\, \Delta_1 \cat \{v:S\}  \proves \bout{n}{v} Q_1  \hastype \Proc
	    } & \label{eq:sound1}
	    \\
	    	    	&     \tree{
	    \Gamma' \cat n:\chtype{S};\, \emptyset ;\, \emptyset  \proves n  \hastype \chtype{S}
	     \quad
	      \Gamma;\, \emptyset ;\, \Delta_3 \cat x:S    \proves   Q_2  \hastype \Proc
	    }{
	    \Gamma;\, \emptyset ;\, \Delta_3 \proves  \binp{n}{x} Q_2 \hastype \Proc
	   } & 
		\label{eq:sound2}
	    \end{eqnarray}
	    
	    Now, by applying \lemref{l:subst}(1) on $\Gamma;\, \emptyset ;\, \Delta_3 \cat x:S    \proves   Q_2  \hastype \Proc$
			we obtain 
	   $$
	   \Gamma;\, \emptyset ;\, \Delta_3 \cat v:S    \proves   Q_2\subst{v}{x}  \hastype \Proc
	   $$
	   
	   			and the case is completed by using rule~\trule{Par} with this judgement:
							\[		~~ 
				\tree{
					\Gamma; \emptyset; \Delta_1    \proves  
 					 Q_1 \hastype \Proc
					 \quad 
					\Gamma;\, \emptyset ;\, \Delta_3 \cat v:S    \proves   Q_2\subst{v}{x}  \hastype \Proc
					}{
					\Gamma; \emptyset; \Delta_1 \cat \Delta_3  \cat v:S \proves  
 					Q_1  \Par  Q_2\subst{v}{x} \hastype \Proc
					} 
			\]
			Observe how in this case the session environment does not reduce.\\
			
			%%%%%%%%%%%%%%%%%%%%%%%%
			
		\item $n$ is a shared name and $V$ is a higher-order value. 
	    Then we have the following reduction: 
	    $$
	    P = \bout{n}{V} Q_1 \Par \binp{n}{x} Q_2  \red  Q_1 \Par Q_2 \subst{V}{x} = P'
	    $$
	    By assumption, we have 
	    the following typing derivation (below, we write 
	    $L$ to stand for $\shot{C}$ and 
	    $\Gamma$ to stand for $ \Gamma' \setminus \{x:L\}$).
	    \[	    \hspace{-12mm}
	    \tree{
	     \eqref{eq:sound3}
%	    \tree{
%	     \Gamma;\, \emptyset ;\, \emptyset  \proves n  \hastype \chtype{L}
%	     \quad
%	      \Gamma;\, \emptyset ;\, \Delta_1    \proves   Q_1  \hastype \Proc
%	      \quad
%	       \Gamma;\, \emptyset ;\, \emptyset  \proves V  \hastype L	    
%	    }{
%	    \Gamma;\, \emptyset ;\, \Delta_1    \proves \bout{n}{V} Q_1  \hastype \Proc
%	    } 
	    \quad 
	    \eqref{eq:sound4}
%	    	    \tree{
%	    \Gamma' ;\, \emptyset ;\, \emptyset  \proves n  \hastype \chtype{L}
%	     \quad
%	      \Gamma';\, \emptyset ;\, \Delta_3    \proves   Q_2  \hastype \Proc
%	      	     \quad
%	      	    \Gamma' ;\, \emptyset ;\, \emptyset  \proves x  \hastype L
%	    }{
%	    \Gamma;\, \emptyset ;\, \Delta_3 \proves  \binp{n}{x} Q_2 \hastype \Proc
%	   }
	    }{
	    \Gamma;\, \emptyset ;\, \Delta_1 \cat \Delta_3 \proves \bout{n}{v} Q_1 \Par \binp{n}{x} Q_2 \hastype \Proc
	    }
	    \]
	    where \eqref{eq:sound3} and \eqref{eq:sound4} are as follows:
	    \begin{eqnarray}
	    & 	    \tree{
	     \Gamma;\, \emptyset ;\, \emptyset  \proves n  \hastype \chtype{L}
	     \quad
	      \Gamma;\, \emptyset ;\, \Delta_1    \proves   Q_1  \hastype \Proc
	      \quad
	       \Gamma;\, \emptyset ;\, \emptyset  \proves V  \hastype L	    
	    }{
	    \Gamma;\, \emptyset ;\, \Delta_1    \proves \bout{n}{V} Q_1  \hastype \Proc
	    } 
 & 	    \label{eq:sound3} \\
	    & 	    	    \tree{
	    \Gamma' ;\, \emptyset ;\, \emptyset  \proves n  \hastype \chtype{L}
	     \quad
	      \Gamma';\, \emptyset ;\, \Delta_3    \proves   Q_2  \hastype \Proc
	      	     \quad
	      	    \Gamma' ;\, \emptyset ;\, \emptyset  \proves x  \hastype L
	    }{
	    \Gamma;\, \emptyset ;\, \Delta_3 \proves  \binp{n}{x} Q_2 \hastype \Proc
	   }
 & 	    \label{eq:sound4}
	    \end{eqnarray}
	    
	    Now, by applying \lemref{l:subst}(4) on 
	    $\Gamma' \setminus \{x:L\};\, \emptyset ;\, \Delta_3    \proves   Q_2  \hastype \Proc$
	    and
	    $\Gamma;\, \emptyset ;\, \emptyset  \proves V  \hastype L$
	    we obtain 
	   $$
	   \Gamma;\, \emptyset ;\, \Delta_3  \proves   Q_2\subst{V}{x}  \hastype \Proc
	   $$
	   
	   and the case is completed by using rule~\trule{Par} with this judgement:
							\[		~~ 
				\tree{
					\Gamma; \emptyset; \Delta_1    \proves  
 					 Q_1 \hastype \Proc
					 \quad 
					\Gamma;\, \emptyset ;\, \Delta_3     \proves   Q_2\subst{V}{x}  \hastype \Proc
					}{
					\Gamma; \emptyset; \Delta_1 \cat \Delta_3   \proves  
 					Q_1  \Par  Q_2\subst{V}{x} \hastype \Proc
					} 
			\]
			Observe how in this case the session environment does not reduce.\\
			

	\end{enumerate}

%		\item	Case \orule{NPass}:
%			Then there are two sub-cases, depending on whether the
%			communication subject is a shared name or a channel. 
%			In the first case, we have 
%			$$P = \bout{\dual{k}}{n} P_1 \Par \binp{k}{x} P_2 \red P_1 \Par P_2\subst{n}{x} = P'$$ 
%			Suppose $\Gamma; \es; \Delta  \proves \bout{\dual{k}}{n} P_1 \Par \binp{k}{x} P_2 \hastype \Proc$. This assumption is derived first from rules~\trule{Req} and \trule{AccS}:
%			\[
%								\tree{
%					\Gamma; \emptyset; \emptyset  \proves  k \hastype \chtype{S} ~~~
%					\Gamma ; \emptyset ; \Delta_1 \proves   P_1 \hastype \Proc ~~~
%					\Gamma ; \emptyset ; \{n:S\} \proves   n \hastype S
%					}{
%					\Gamma; \emptyset; \Delta_1 \cat \{n:S\}    \proves  
% 					\bout{\dual{k}}{n} P_1 \hastype \Proc} 
%			\]		
%			and
%			\[		~~ 
%				\tree{
%					\Gamma; \emptyset; \emptyset  \proves  k \hastype \chtype{S} \quad 
%					\Gamma ; \emptyset ; \Delta_2 \cat \{x:S\}  \proves  P_2 \hastype \Proc \quad
%					\Gamma ; \emptyset ; \{x:S\}  \proves  x \hastype S
%					}{
%					\Gamma; \emptyset; \Delta_2  \proves  
% 					\binp{k}{x} P_2 \hastype \Proc} 
%			\]
%			and then rule~\trule{Par}, we obtain: %letting $\Delta = \Delta_1 \cat \Delta_2  \cat \Delta_3$.
%				\[		~~ 
%				\tree{
%					\Gamma; \emptyset; \Delta_1 \cat \{n:S\}    \proves  
% 					\bout{\dual{k}}{n} P_1 \hastype \Proc\quad 
%					\Gamma; \emptyset;  \Delta_2 \proves  
% 					\binp{k}{x} P_2 \hastype \Proc
%					}{
%					\Gamma; \emptyset; \Delta_1 \cat \{n:S\} \cat \Delta_2 \proves  
% 					\bout{\dual{k}}{n} P_1  \Par \binp{k}{x} P_2 \hastype \Proc
%					} 
%			\]
%			
%			Now, by applying Lemma~\ref{lem:subst}(1) on $\Gamma ; \emptyset ; \Delta_2 \cat \{x:S\}  \proves  P_2 \hastype \Proc$
%			we obtain 
%			$$\Gamma ; \emptyset ; \Delta_2 \cat \{n:S\} \proves  P_2\subst{n}{x} \hastype \Proc$$
%			and the case is completed by using rule~\trule{Par} with this judgment:
%							\[		~~ 
%				\tree{
%					\Gamma; \emptyset; \Delta_1    \proves  
% 					 P_1 \hastype \Proc\quad 
%					\Gamma; \emptyset;  \Delta_2 \cat \{n:S\}  \proves  
% 					 P_2\subst{n}{x} \hastype \Proc
%					}{
%					\Gamma; \emptyset; \Delta_1 \cat \{n:S\} \cat \Delta_2 \proves  
% 					P_1  \Par  P_2\subst{n}{x} \hastype \Proc
%					} 
%			\]
%			Observe how in this case the session environment does not reduce.\\
%			
%			In the second case we have the following reduction, with   $|\mytilde{h}| = |\mytilde{x}|$:
%			$$P = \bout{\dual{k}}{\mytilde{h}} P_1 \Par \binp{k}{\mytilde{x}} P_2 \red P_1 \Par P_2\subst{\mytilde{h}}{\mytilde{x}} = P'$$ 
%			Also in this case the proof is standard, using rules~\trule{RcvS}, \trule{Send}, and \trule{Par} 
%			to type $P$, and using Lemma~\ref{lem:subst}(1) and rule~\trule{Par} to type $P'$. 
%			In this case, the session environment $\Delta$ does reduce.
%
%		\item	Case \orule{APass}:
%		Then we have
%		$$
%		P = \bout{k}{\abs{\mytilde{x}}{Q}} P_1 \Par \binp{\dual{k}}{\X} P_2  \red  P_1 \Par P_2 \subst{\abs{\mytilde{x}}{Q}}{\X} = P'
%		$$
%		and we distinguish two cases, associated to the type of the higher-order value $\abs{\tilde{x}}{Q}$.
%		We describe the proof for the case in which the type is $\lhot{\mytilde{C}}$; the proof when 
%		the type is $\shot{\mytilde{C}}$ is analogous.
%		The typing of $P$ proceeds first by using rule~\trule{Send} on the left-hand side:
%		\[
%								\tree{
%					\Gamma;\, \emptyset;\, \Delta_1 \proves  P_1 \hastype \Proc \quad
%					\Gamma ;\, \emptyset ;\, \Delta_2 \proves   \abs{\mytilde{x}}{Q} \hastype \lhot{\mytilde{C}}					}{
%					\Gamma;\, \emptyset;\, \big((\Delta_1 \cat \Delta_2) \setminus \{k:S\}\big) \cat k:\btout{\lhot{\mytilde{C}}} S     \proves  
% 					\bout{k}{\abs{\mytilde{x}}{Q}} P_1 \hastype \Proc} 
%			\]	
%			Then,
%			thanks to the well-typedness assumption for $\Delta$ (cf. Def.~\ref{d:wtenv}), 
%			 on the right-hand side we have the following typing (using rule~\trule{RcvH}  and assuming $S \dualof T$):
%					\[
%					\tree{
%					\Gamma;\, X:\lhot{\mytilde{C}} ;\, \Delta_3 \cat \dual{k}:T \proves  P_2 \hastype \Proc \quad
%					\Gamma ;\, \{X:\lhot{\mytilde{C}} \} ;\, \Delta_4 \proves   X \hastype \lhot{\mytilde{C}}					}{
%					\Gamma;\, \emptyset;\, \Delta_3 \setminus \Delta_4 \cat \dual{k}:\btinp{\lhot{\mytilde{C}}} T     \proves  
% 					\binp{\dual{k}}{X} P_2 \hastype \Proc} 
%			\]	
%			Finally, we use rule~\trule{Par} to obtain the typing for $P$.
%			The typing of $P'$ is obtained by using the appropriate substitution lemma (Lemma~\ref{lem:subst}(3)) on the typing for $P_2$.
%			
%			When the type of the higher-order value is $\shot{\mytilde{C}}$,
%			the use of rules~\trule{Send} and~\trule{RcvH} for typing $P$ is similar; 
%			 one would use Lemma~\ref{lem:subst}(4) to type $P'$. The session environment reduces.

		\item	Case \orule{Sel}:
			The proof is standard, the session environment reduces.

%		\item	Case \orule{Sess}:
%			The proof is standard, exploiting induction hypothesis.
%			The session environment may remain invariant (channel restriction)  or reduce (name restriction).

		\item	Cases \orule{Par} and \orule{Res}:
			The proof is standard, exploiting induction hypothesis. 

		\item	Case \orule{Cong}:
			follows from \thmref{t:srfull}\,(1).
	\end{enumerate}
%	\qed
\end{proof}



\section{Proof of \propref{p:examp} - Hotel Booking Scenario}
\label{hotel_closure}

We repeat the statement in Page \pageref{p:examp}:
\begin{proposition}
	Let $S' = \btsel{\accept: \btout{\creditc} \tinact, \reject: \tinact}$,
	$S = \btout{\rtype} \btinp{\Quote} S'$
	and $\Delta = s_1: \btout{\lhot{S}} \tinact \cat s_2: \btout{\lhot{S}} \tinact$
	then
	$\horel{\es}{\Delta}{\Client_1}{\wbf}{\Delta}{\Client_2}$
\end{proposition}

\begin{proof}
	\noi We show a case where each typed process simulates the other.

	\noi For fresh sessions $s, k$ we get
	$
		\mapchar{\btinp{\lhot{S}} \tinact}{s} = \binp{s}{x} (\mapchar{\tinact}{s} \Par \mapchar{\lhot{S}}{x})
%		= \binp{s}{x} (\inact \Par \appl{x}{\omapchar{S}})
%		= \binp{s}{x} (\inact \Par \appl{x}{k})
		\scong \binp{s}{x} (\appl{x}{k})
	$
	

	\noi To observe $\Client_1$ assume:
%
	\begin{eqnarray*}
		R' &\scong& \If\ x \leq y\ \Then (\bsel{\dual{h_1}}{\accept} \bsel{\dual{h_2}}{\reject} \inact
		\Else \bsel{\dual{h_1}}{\reject} \bsel{\dual{h_2}}{\accept} \inact)\\
		Q &\scong& \bbra{z}{\accept: \bsel{k_2}{\accept} \bout{k_2}{\creditc} \inact, \reject: \bsel{k_2}{\reject} \inact}
%		Q &\scong& z \triangleleft \left\{
%		\begin{array}{l}
%			\accept: \bsel{k_2}{\accept} \bout{k_2}{\creditc} \inact,\\
%			\reject: \bsel{k_2}{\reject} \inact
%		\end{array}
%		\right\}
	\end{eqnarray*}
%
	\noi We can now observe $\Client_1$ as:
\[
	\begin{array}{ll}
		& \es; \es; \Delta \proves \Client_1
		\\[1mm]

		\by{\bactout{s_1}{\abs{x}{P \subst{h_1}{y}}}}&
		\es; \es; s_2: \btout{\lhot{S}} \tinact \cat k_1: S \proves \\
		& \newsp{h_1, h_2}{\bout{s_2}{\abs{x}{P \subst{h_2}{y}}} \inact
		\Par \binp{\dual{h_1}}{x} \binp{\dual{h_2}}{y} R'\\
		& \Par \ftrigger{t_1}{P \subst{h_1}{y}}{\lhot{S}}}
		%\binp{t_1}{x} \newsp{s}{\mapchar{\btinp{\lhot{S}}}{s} \Par \bout{\dual{s}}{\abs{x}{P \subst{h_1}{y}}} \inact }}
		\\[1mm]

		\by{\bactout{s_2}{\abs{x}{P \subst{h_2}{y}}}}&
		\es; \es; k_1: S \cat k_2: S \proves \newsp{h_1, h_2}{
		\binp{\dual{h_1}}{x} \binp{\dual{h_2}}{y} R'\\
		& \ftrigger{t_1}{P \subst{h_1}{y}}{\lhot{S}} \Par \ftrigger{t_2}{P \subst{h_2}{y}}{\lhot{S}}}
%		\Par \binp{t_1}{x} \newsp{s}{\binp{s}{x} \appl{x}{k_1} \Par \bout{\dual{s}}{\abs{x}{P \subst{h_1}{y}}} \inact }\\
%		& \Par \binp{t_2}{x} \newsp{s}{\mapchar{\btinp{\lhot{S}}}{s} \Par \bout{\dual{s}}{\abs{x}{P \subst{h_2}{y}}} \inact }}
		\\[1mm]

		\by{\bactinp{t_1}{b}} \by{\bactinp{t_2}{b}} \by{\dtau}\by{\dtau}&
		\es; \es; k_1: S \cat k_2: S \proves \newsp{h_1, h_2}{
		\binp{\dual{h_1}}{x} \binp{\dual{h_2}}{y} R'\\
		& \Par P\subst{h_1}{y} \subst{k_1}{x} \Par P\subst{h_1}{y} \subst{k_2}{x}}
		\\[1mm]

		\by{\bactout{k_1}{\rtype}} \by{\bactout{k_2}{\rtype}}\\
		\by{\bactinp{k_1}{\Quote}} \by{\bactinp{k_2}{\Quote}}
		& \es; \es; k_1: S' \cat k_2: S' \proves \newsp{h_1, h_2}{
		\binp{\dual{h_1}}{x} \binp{\dual{h_2}}{y} R'\\
		& \Par \bout{h_1}{\Quote} Q \subst{h_1}{z} \Par \bout{h_2}{\Quote} Q \subst{h_2}{z}}
		\\[1mm]

		\by{\dtau} \by{\dtau} \by{\dtau}&
		\es; \es; k_1: S' \cat k_2: S' \proves \\
		& \newsp{h_1, h_2}{\bsel{\dual{h_1}}{\accept} \bsel{\dual{h_2}}{\reject} \inact
		\Par Q \subst{h_1}{z} \Par Q \subst{h_2}{z}}
		\\[1mm]

		\by{\dtau} \by{\dtau}&
		\es; \es; k_1: S' \cat k_2: S' \proves
		\bsel{k_1}{\accept} \bout{k_1}{\creditc} \inact 
		\Par \bsel{k_2}{\reject} \inact
		\\[1mm]

		\by{\bactsel{k_1}{\accept}} \by{\bactsel{k_2}{\reject}} \by{\bactsel{k_1}{\creditc}}&
		\es; \es; \es \proves \inact
	\end{array}
\]

\noi	We can observe the same sequence of external transitions for $\Client_2$:

\[
	\begin{array}{ll}
		& \es; \es; \Delta \proves \Client_2
\\[1mm]

		\by{\bactout{s_1}{\abs{x}{Q_1 \subst{h}{y}}}}&
		\es; \es; s_2: \btout{\lhot{S}} \tinact \cat k_1: S \proves \newsp{h}{\bout{s_2}{\abs{x}{Q_2 \subst{\dual{h}}{y}}} \inact\\
		& \Par \ftrigger{t_1}{Q_1 \subst{h}{y}}{\lhot{S}}}
		% \binp{t_1}{x} \newsp{s}{\mapchar{\btinp{\lhot{S}}}{s} \Par \bout{\dual{s}}{\abs{x}{Q_1 \subst{h}{y}}} \inact }}
\\[1mm]

		\by{\bactout{s_2}{\abs{x}{Q_2 \subst{\dual{h}}{y}}}}&
		\es; \es; k_1: S \cat k_2: S \proves \newsp{h}{\\
		& \ftrigger{t_1}{Q_1 \subst{h}{y}}{\lhot{S}} \Par \ftrigger{t_2}{Q_2 \subst{\dual{h}}{y}}{\lhot{S}}}
%		\binp{t_1}{x} \newsp{s}{\binp{s}{x} \appl{x}{k_1} \Par \bout{\dual{s}}{\abs{x}{Q_1 \subst{h}{y}}} \inact }\\
%		& \Par \binp{t_2}{x} \newsp{s}{\mapchar{\btinp{\lhot{S}}}{s} \Par \bout{\dual{s}}{\abs{x}{Q_2 \subst{\dual{h}}{y}}} \inact }}
\\[1mm]

		\by{\bactinp{t_1}{b}} \by{\bactinp{t_2}{b}} \by{\dtau}\by{\dtau}&
		\es; \es; k_1: S \cat k_2: S \proves \newsp{h}{
		P\subst{h}{y} \subst{k_1}{x} \Par P\subst{\dual{h}}{y} \subst{k_2}{x}}
\\[1mm]

		\by{\bactout{k_1}{\rtype}} \by{\bactout{k_2}{\rtype}}\\
		\by{\bactinp{k_1}{\Quote}} \by{\bactinp{k_2}{\Quote}}
		& \es; \es; k_1: S' \cat k_2: S' \proves \newsp{h}{
		\bout{h}{\Quote_1} \binp{h}{\Quote_2} R \subst{k_1}{x} \\
		& \Par \binp{\dual{h}}{\Quote_2} \bout{\dual{h}}{\Quote_1} R \subst{k_2}{x}}
\\[1mm]
		\by{\dtau} \by{\dtau}&
		\es; \es; k_1: S' \cat k_2: S' \proves R \subst{k_1}{x} \Par R \subst{k_2}{x}
\\[1mm]
		\by{\dtau} \by{\dtau}&
		\es; \es; k_1: S' \cat k_2: S' \proves
		\bsel{k_1}{\accept} \bout{k_1}{\creditc} \inact 
		\Par \bsel{k_2}{\reject} \inact
\\[1mm]
		\by{\bactsel{k_1}{\accept}} \by{\bactsel{k_2}{\reject}} \by{\bactsel{k_1}{\creditc}}&
		\es; \es; \es \proves \inact
	\end{array}
\]
\end{proof}

\begin{comment}
\begin{proposition}[Hotel Booking Equivalence]
	Let
	$S = \btout{\rtype} \btinp{\Quote} \btsel{\accept: \btout{\creditc} \tinact, \reject: \tinact}$
	and $\Delta = s_1: \btout{\lhot{S}} \tinact \cat s_2: \btout{\lhot{S}} \tinact$
	then
	$ \horel
	{\es}{\Delta}{\Client_1}
	{\wbf}
	{\Delta}{\Client_2}$
\end{proposition}

\begin{proof}[Proof (Sketch)]
	We show a bisimulation closure by following transitions on each $\Client$.
	We show the initial higher order transitions. A more detailed proof sketch
	can be found in \appref{hotel_closure}.

	\noi First we find the characteristic process
	$\mapchar{\btinp{\lhot{S}} \tinact}{s} = \binp{s}{x} (\abs{x}{k})$

\[
	\begin{array}{ll}
		\es; \es; \Delta \proves \Client_1
		&
		\by{\bactout{s_1}{\abs{x}{P \subst{h_1}{y}}}}
		\by{\bactout{s_2}{\abs{x}{P \subst{h_2}{y}}}}
		\\
		\es; \es; k_1: S \cat k_2: S \proves
		&
		\newsp{h_1, h_2}{\binp{\dual{h_1}}{x} \binp{\dual{h_2}}{y}\\
		& \If\ x \leq y\ \Then (\bsel{\dual{h_1}}{\accept} \bsel{\dual{h_2}}{\reject} \inact
		\Else \bsel{\dual{h_1}}{\reject} \bsel{\dual{h_2}}{\accept} \inact)\\
		& \Par \binp{t_1}{x} \newsp{s}{\binp{s}{x} \appl{x}{k_1} \Par \bout{\dual{s}}{\abs{x}{P \subst{h_1}{y}}} \inact }\\
		& \Par \binp{t_2}{x} \newsp{s}{\mapchar{\btinp{\lhot{S}}}{s} \Par \bout{\dual{s}}{\abs{x}{P \subst{h_2}{y}}} \inact }}
		\\
		\\
		\es; \es; s_1: \Delta \proves \Client_2
		&\by{\bactout{s_1}{\abs{x}{Q_1 \subst{h}{y}}}}
		\by{\bactout{s_2}{\abs{x}{Q_2 \subst{\dual{h}}{y}}}}
		\\
		\es; \es; k_1: S \cat k_2: S \proves & \newsp{h}{
		\binp{t_1}{x} \newsp{s}{\binp{s}{x} \appl{x}{k_1} \Par \bout{\dual{s}}{\abs{x}{Q_1 \subst{h}{y}}} \inact }\\
		&\Par \binp{t_2}{x} \newsp{s}{\binp{s}{x} \appl{x}{k_2} \Par \bout{\dual{s}}{\abs{x}{Q_2 \subst{\dual{h}}{y}}} \inact }}
	\end{array}
\]
	\noi We can then show that the resulting typed processes
	are behaviourally equivalent by following simple chasing diagrams.
\end{proof}
\end{comment}


\section{Proofs for Section~\ref{sec:behavioural}}
\label{app:beh}
% !TEX root = main.tex

%%%%%%%%%%%%%%%%%%%%%%%%%%%%%%%%%%%%%%%%%%%%%%%%%%%%%
% Types inhabit their characteristic process
%%%%%%%%%%%%%%%%%%%%%%%%%%%%%%%%%%%%%%%%%%%%%%%%%%%%%

\label{app:inhabit}

We state a more detailed form of \propref{p:inhabit} as given in Page~\pageref{p:inhabit}:

\begin{proposition}[Characteristic Processes/Values Inhabit Their Types]
	\label{app:characteristic_inhabit}
	\begin{enumerate}
		\item	Let type $U$, then
		\begin{enumerate}
			\item	If $U = S$ then for some s, $\es; \es; s: S \proves \omapchar{S} \hastype S$.
	
			\item	If $U = \chtype{S}$ then for some $a$, $a: \chtype{S}; \es; \es \proves \omapchar{\chtype{S}} \hastype \chtype{S}$.
	
			\item	If $U = \chtype{L}$ then for some $a$, $a: \chtype{L}; \es; \es \proves \omapchar{\chtype{L}} \hastype \chtype{L}$.
	
			\item	If $U = \shot{U'}$ then whenever $\exists \Gamma, \Delta$ such that
					$\Gamma; \es; \Delta \proves \mapchar{U'}{x} \hastype \Proc$ then
					$\Gamma \backslash x ; \es; \Delta \backslash x \proves \omapchar{\shot{U'}} \hastype \shot{U'}$.
	
			\item	If $U = \shot{U'}$ then whenever $\exists \Gamma, \Delta$ such that
					$\Gamma; \es; \Delta \proves \mapchar{U'}{x} \hastype \Proc$ then
					$\Gamma \backslash x ; \es; \Delta \backslash x \proves \omapchar{\shot{U'}} \hastype \shot{U'}$.
		\end{enumerate}

		\item	Let type $S$, then
		\begin{enumerate}
			\item	If $S = \btout{U} S'$ then
					whenever $\Gamma; \es; \Delta \proves \omapchar{U} \hastype U$ then
					$\Gamma; \es; \Delta \cat t: \btout{S'} \tinact \cat s: \btout{U} S' \proves \mapchar{\btout{U} S'}{s} \hastype \Proc$.
	
			\item	If $S = \btinp{U} S'$ then whenever $\exists \Gamma, \Delta$ such that
					$\Gamma; \es; \Delta \proves \mapchar{U'}{x}  \hastype \Proc$ then
					$\Gamma \backslash x; \es; (\Delta\backslash x) \cat t: \btinp{S'} \tinact \cat s: \btout{U} S' \proves \mapchar{\btinp{U} S'}{s} \hastype \Proc$.
	
			\item	If $S = \btsel{l: S'}$ then
					$\es; \es; t: \btout{S'} \tinact \cat s: \btsel{l: S'} \proves \mapchar{\btsel{l: S'}}{s} \hastype \Proc$.
	
			\item	If $S = \btbra{l_i: S_i}_{i \in I}$ then
					$\es; \es; \set{t_i: \btout{S_i}}_{i \in I} \tinact \cat s: \btbra{l_i: S_i}_{i \in I} \proves \mapchar{\btbra{l_i: S_i}_{i \in I}}{s} \hastype \Proc $.
	
			\item	If $S = \trec{t}{S'}$ then whenever
					$\exists, \Gamma, \Delta$ such that
					$\Gamma; \es; \Delta \cat t: S_2 \cat s: S' \subst{\tinact}{\vart{t}} \proves \mapchar{S' \subst{\tinact}{\vart{t}}}{s} \hastype \Proc$
					then $\exists S_1$, 
					$\Gamma; \es; \Delta \cat t: \btout{S_1} \tinact \cat s: \trec{t}{S'} \proves \mapchar{\trec{t}{S'}}{s} \hastype \Proc$
					\dk{and $\pi;S_1 = \mathsf{unfold}(\trec{t}{S'})$ where $\pi$ is a session type prefix}
	
			\item	If $S = \tinact$ then $\es; \es; \es; \proves \mapchar{\tinact}{s} \hastype \Proc$.
		\end{enumerate}

		\item	Let type $U$.
		\begin{enumerate}
			\item	If $U = \chtype{S}$ then whenever $\es; \es; \Delta \proves \omapchar{S} \hastype S$ then
					$a: \chtype{S}; \es; \Delta \cat t: \btout{\chtype{S}} \tinact \proves \mapchar{\chtype{S}}{a} \hastype \Proc$.
	
			\item	If $U = \chtype{L}$ then whenever $\Gamma; \es; \Delta \proves \omapchar{L} \hastype L$ then
					$\Gamma \cat a: \chtype{L}; \es; \Delta \cat t: \btout{\chtype{L}} \tinact \proves \mapchar{\chtype{L}}{a} \hastype \Proc$.
	
			\item	If $U = \shot{U'}$ then whenever $\Gamma; \es; \Delta \proves \omapchar{U'} \hastype U'$ then
					$\Gamma \cat x: \shot{U'}; \es;\Delta \proves \mapchar{\shot{U'}}{x} \hastype \Proc$.
	
			\item	If $U = \lhot{U'}$ then whenever $\Gamma; \es; \Delta \proves \omapchar{U'} \hastype U'$ then
					$\Gamma \cat x: \shot{U'}; \es;\Delta \proves \mapchar{\lhot{U'}}{x} \hastype \Proc$.
		\end{enumerate}
	\end{enumerate}
\end{proposition}

\begin{proof}[Sketch]
	The proof of part 2 is done by induction on the the syntax of $S$.
	We give some notable cases.
	\begin{itemize}
		\item	Case $S = \btout{U} S'$ with
				$\mapchar{S}{s} = \bout{s}{\omapchar{U}} \bout{t}{s} \inact$.
				If we type the last process we get the derivation:
				\[
					\tree{
						\begin{array}{l}
							\Gamma; \es; s: S' \cat t: \btout{S'} \tinact \hastype \bout{t}{s} \inact \hastype \Proc \qquad \text{(Induction)}
							\\
							\Gamma; \es; \Delta \proves \omapchar{U} \hastype U
						\end{array}
					}{
						\Gamma; \es; \Delta \cat s: \btout{U} S' \cat t: \btout{S} \tinact \hastype \bout{s}{\omapchar{U}}\bout{t}{s} \inact \hastype \Proc
					}
				\]
		\item	Case $S = \btinp{S_1} S_2$ with
				$\mapchar{S}{s} = \binp{s}{x} (\bout{t}{s} \inact \Par \mapchar{S_1}{x})$.
				If we type the last process we have the following derivation
				\[
					\tree{
						\tree{
							\begin{array}{l}
								\Gamma; \es; \Delta \cat x: S_1 \proves \mapchar{S_1}{x} \hastype \Proc
								\qquad \text{(Induction)}
								\\
								\Gamma; \es; t: \btout{S_2} \tinact \cat s: S_2 \proves \bout{t}{s} \inact \hastype \Proc
							\end{array}
						}{
							\Gamma; \es; \Delta \cat x: S_1 \cat t: \btout{S_2} \tinact \cat s: S_2 \proves
							\bout{t}{s} \inact \Par \mapchar{S_1}{x} \hastype \Proc
						}
					}{
						\Gamma; \es; \Delta \cat t: \btout{S_2} \tinact \cat s: \btinp{U} S_2 \proves \binp{s}{x} (\bout{t}{s} \inact \Par \mapchar{S_1}{x}) \hastype \Proc
					}
				\]
			\item	Case $S = \trec{t}{S'}$.
					\[
						\tree{
							\Gamma; \es; \Delta \cat t: \btout{S_2} \tinact \cat s: S_1 \proves \bout{t}{s} \inact \hastype \Proc
						}{
							\Gamma; \es; \Delta \cat t: \btout{S_1} \tinact \cat s: \trec{t}{S} \hastype \mapchar{\trec{t}{S}}{s} \hastype \Proc
						}
					\]
					\dk{put proof}
%					\[
%						\tree{
%							\Gamma; \es; \Delta \cat t: \btout{S'} \tinact \cat s: S' \proves \bout{t}{s} \hastype \Proc
%						}{
%							\Gamma; \es; \Delta \cat t: \btout{S'} \tinact \cat s: \trec{t}{S} \hastype \mapchar{\trec{t}{S}}{s} \hastype \Proc
%						}
%					\]
			\item	Other cases are similar.
	\end{itemize}
	The proof for parts 2 and 3 is similar.
	\qed
\end{proof}

\dk{
	\begin{example}
		\begin{itemize}
			\item	$\trec{t}{\btout{U} \vart{t}}$
					%
					\[
						\tree{
							\Gamma; \es; \Delta \hastype \omapchar{U} \hastype U\\
							\Gamma; \es; t: \btout{\tinact} \tinact \cat s: \tinact \proves \bout{t}{s} \inact \hastype \Proc
						}{
							\Gamma; \es; \Delta \cat t: \btout{\tinact} \tinact \cat s: \btout{U} \tinact \proves
							\mapchar{\btout{U} \tinact}{s} = \bout{s}{\omapchar{U}} \bout{t}{s} \inact \hastype \Proc
						}
					\]
					and
					\[
						\tree {
							\Gamma; \es; \Delta \cat t: \btout{\trec{t}{\btout{U} \vart{t}}} \tinact \cat s: \trec{t}{\btout{U} \vart{t}} \proves
							\bout{t}{s} \inact \hastype \Proc
						}{
							\Gamma; \es; \Delta \cat t: \btout{\trec{t}{\btout{U} \vart{t}}} \tinact \cat s: \trec{t}{\btout{U} \vart{t}} \proves
							\mapchar{\trec{t}{\btout{U} \vart{t} }}{s} =
							\mapchar{(\btout{U} \vart{t}) \subst{\tinact}{\vart{t}}}{s} =
							\mapchar{\btout{U} \tinact}{s} = \bout{s}{\omapchar{U}} \bout{t}{s} \inact \hastype \Proc
						}
					\]

					\item	$\trec{t}{\btout{U_1} \btinp{U_2} \vart{t}}$
					%
					\[
						\tree{
							\Gamma; \es; \Delta \hastype \omapchar{U_1} \hastype U_2\\
							\Gamma; \es; t: \btout{\btinp{U_2} \tinact} \tinact \cat s: \btinp{U_2} \tinact \proves \bout{t}{s} \inact \hastype \Proc
						}{
							\Gamma; \es; \Delta \cat t: \btout{\btinp{U_2} \tinact} \tinact \cat s: \btout{U_1} \btinp{U_2} \tinact \proves
							\mapchar{\btout{U_1} \btinp{U_2} \tinact}{s} = \bout{s}{\omapchar{U_1}} \bout{t}{s} \inact \hastype \Proc
						}
					\]
					and
					\[
						\tree {
							\Gamma; \es; \Delta \cat t: \btout{\btinp{U_2} \trec{t}{\btout{U_1} \btinp{U_2} \vart{t}}} \tinact
							\cat s: \btinp{U_2} \trec{t}{\btout{U} \vart{t}} \proves
							\bout{t}{s} \inact \hastype \Proc
						}{
							\Gamma; \es; \Delta \cat t: \btout{\btinp{U_2} \trec{t}{\btout{U} \vart{t}}} \tinact \cat s: \trec{t}{\btout{U} \vart{t}} \proves
							\bout{s}{\omapchar{U_1}} \bout{t}{s} \inact \hastype \Proc
						}
					\]
		\end{itemize}
	\end{example}
}

%\section{Behavioural Semantics}

We present the proofs for 
\thmref{the:coincidence} (Page \pageref{the:coincidence}).
We require an auxiliary result on 
deterministic transitions (\lemref{lem:up_to_deterministic_transition}).
Some notions needed to prove this auxiliary result are presented next.
%Then we present the proof of \thmref{the:coincidence}, based on \emph{higher-order bisimilarity}.

%As mentioned in the paper, 
%the proof of \thmref{the:coincidence}
%relies on an auxiliary typed behavioral equivalence, \emph{higher-order bisimilarity}:

%\begin{definition}[Higher-Order Bisimulation]\myrm
%	\label{def:bisim}
%	Typed relation
%	$\Re$ is a {\em higher-order bisimulation} if for all
%	$\horel{\Gamma}{\Delta_1}{P_1}{\ \Re\ }{\Delta_2}{Q_1}$, % implies:
%%
%	\begin{enumerate}[1.]
%		\item	%$\forall \news{\widetilde{m_1}} \bactout{n}{V_1}$ such that
%		   Whenever 
%			$
%				\horel{\Gamma}{\Delta_1}{P_1}{\hby{\news{\widetilde{m_1}} \bactout{n}{V_1}}}{\Delta_1'}{P_2}
%			$
%			there exist $Q_2$, $V_2$, $\Delta_2'$ such that
%			\[
%				\horel{\Gamma}{\Delta_2}{Q_1}{\Hby{\news{\widetilde{m_2}} \bactout{n}{V_2}}}{\Delta_2'}{Q_2}
%			\]
%			and, for a fresh $t$, 
%			$
%				\horel{\Gamma}{\Delta_1''}{\newsp{\widetilde{m_1}}{P_2 \Par \htrigger{t}{V_1}}}
%				{\ \Re\ }
%				{\Delta_2''}{}{\newsp{\widetilde{m_2}}{Q_2 \Par \htrigger{t}{V_2}}}$.
%			
%%
%		\item	For all 
%			$
%				\horel{\Gamma}{\Delta_1}{P_1}{\hby{\ell}}{\Delta_1'}{P_2}
%			$
%			such that $\ell \not= \news{\widetilde{m}} \bactout{n}{V}$, there exist
%			 $\exists Q_2$ and $\Delta_2'$ such that 
%			\[
%				\horel{\Gamma}{\Delta_1}{Q_1}{\Hby{\hat{\ell}}}{\Delta_2'}{Q_2}
%			\]
%			and
%			$\horel{\Gamma}{\Delta_1'}{P_2}{\ \Re\ }{\Delta_2'}{Q_2}$.
%
%		\item	The symmetric cases of 1 and 2.
%	\end{enumerate}
%	The Knaster-Tarski theorem ensures that the largest higher-order bisimulation exists;
%	it is called \emph{higher-order bisimilarity} and is denoted by $\hwb$.
%\end{definition}
In this appendix, we use the polyadic abstractions and name passing 
for shorthand notations. 

%\smallskip


%the theorem in \secref{sec:behavioural}.


%%%%%%%%%%%%%%%%%%%%%%%%%%%%%%%%%%%%%%%%%%%%%%%%%%%%%%%%%%%%%%
% tau - Innertness
%%%%%%%%%%%%%%%%%%%%%%%%%%%%%%%%%%%%%%%%%%%%%%%%%%%%%%%%%%%%%%

\subsection{Deterministic Transitions}
\label{app:sub_tau_inert}


%%%%%%%%%%%%%%%%%%%%%%%%%%%%%%%%%%%%%%%%%%%%%%%%%%%%%%%%%%%%%%%%%%%%%%%%%%%%%%%%
%    TAU - INNERTNESS
%%%%%%%%%%%%%%%%%%%%%%%%%%%%%%%%%%%%%%%%%%%%%%%%%%%%%%%%%%%%%%%%%%%%%%%%%%%%%%%%%

\begin{proposition}[$\tau$-inertness]
	\label{app:lem:tau_inert}
	Let  $\Gamma; \es; \Delta \proves P \hastype \Proc$ be a balanced \HOp process.
	Then
	\begin{enumerate}[1.]
		\item	$\horel{\Gamma}{\Delta}{P}{\hby{\dtau}}{\Delta'}{P'}$ implies
			$\horel{\Gamma}{\Delta}{P}{\hwb}{\Delta'}{P'}$.
		\item	$\horel{\Gamma}{\Delta}{P}{\Hby{\dtau}}{\Delta'}{P'}$ implies
			$\horel{\Gamma}{\Delta}{P}{\hwb}{\Delta'}{P'}$.
	\end{enumerate}
\end{proposition}

%\jp{This proof seems to be by induction on deterministic transition; but then the analysis is on the structure of processes,
%which is confusing. In general: I would have done this proof by coinduction, constructing a closure containing $(P, P')$.}

\begin{proof}
	\noi 
	We prove Part 1 --- the proof for Part 2 follows straightforwardly.
	The proof is by induction on the structure of $\by{\tau}$
	which coincides with the reduction $\red$.

	\noi Basic step:
	\begin{enumerate}
		\item %Case: $P = \appl{(\abs{x}{P})}{n}$:
	%
		\[
			\horel{\Gamma}{\Delta}{\appl{(\abs{x}{P})}{n}}{\hby{\btau}}{\Delta'}{P \subst{n}{x}}
		\]
	%
		\noi Bisimulation requirements hold because there is no other transition to observe than ${\hby{\btau}}$.

		\item %Case: $P = \bout{s}{V} P_1 \Par \binp{\dual{s}}{x} P_2$:
	%
		\[
			\horel{\Gamma}{\Delta}{\bout{s}{V} P_1 \Par \binp{\dual{s}}{x} P_2}{\hby{\stau}}{\Delta'}{P_1 \Par P_2}
		\]
	%
		\noi The proof follows from the fact that we can only observe a $\tau$
		action on typed process
		$\Gamma; \emptyset; \Delta \proves P \hastype \Proc$.
		Actions $\bactout{s}{V}$ and $\bactinp{\dual{s}}{V}$
		are forbidden by the LTS for typed environments;
		\dk{this is because
		$s: \btout{U} S_1 \cat \dual{s}: \btinp{U} S_2 \in \Delta$ and
		rule \eltsrule{SSnd} (resp., \eltsrule{SRv}) cannot be applied
		in order to observe action $(\Gamma; \es; \Delta) \by{\bactout{s}{V}} (\Gamma; \es; \Delta')$
		(resp., action $(\Gamma; \es; \Delta) \by{\bactinp{\dual{s}}{V}} (\Gamma; \es; \Delta'')$)
		because of the requirement $\dual{s} \notin \dom{\Delta}$ (resp., $s \notin \dom{\Delta}$).
		}

		\noi It is easy to conclude then that $\horel{\Gamma}{\Delta}{P}{\hwb}{\Delta'}{P'}$.

		\item %Case:
			\[
				\horel{\Gamma}{\Delta}{\bsel{s}{l_k} P \Par \bbra{\dual{s}}{l_i: P_i}_{i \in I}}{\hby{\stau}}{\Delta'}{P \Par P_k}
			\]

		\noi Similar arguments as the previous case.
	\end{enumerate}
	
	\noi Induction hypothesis:

	\noi If $P_1 \red P_2$ then $\horel{\Gamma_1}{\Delta_1}{P_1}{\hwb}{\Delta_2}{P_2}$.
	\noi Induction Step:
	\begin{enumerate}
		\item %Case: $P = \news{s} P_1$
	%
		\[
			\horel{\Gamma}{\Delta}{\news{s}{P_1}}{\hby{\stau}}{\Delta'}{\news{s} P_2}
		\]
	%
		\noi From the induction hypothesis and the fact that bisimulation is a congruence
		we get that $\horel{\Gamma}{\Delta}{P}{\hwb}{\Delta'}{P'}$.

		\item  %Case: $P = P_1 \Par P_3$
	%
		\[
			\horel{\Gamma}{\Delta}{P_1 \Par P_3}{\hby{\stau}}{\Delta'}{P_2 \Par P_3}
		\]
	%
		\noi From the induction hypothesis and the fact that bisimulation is a congruence
		we get that $\horel{\Gamma}{\Delta}{P}{\hwb}{\Delta'}{P'}$.

		\item   %Case:
			\[
				P \scong P_1 \text{ and }\horel{\Gamma}{\Delta}{P_1}{\hby{\stau}}{\Delta'}{P'}
			\]
%
		From the induction hypothesis and the fact that bisimulation is a congruence \dk{(\thmref{the:coincidence})}
		and structural congruence preserves $\hwb$
		we get that $\horel{\Gamma}{\Delta}{P}{\hwb}{\Delta'}{P'}$.
	\end{enumerate}
%	The proof for part two is an induction on the length of $\red^*$.
%	The basic step is trivial and the inductive step
%	deploys part 1 of this lemma and the fact that bisimulation is
%	transitive to conclude.
%	We can now conclude that
%	$P \wbc P'$ because $P \wbc P''$ and $P'' \wbc P'$.
	\qed
\end{proof}


%\begin{lemma}[Up-to Deterministic Transition]\myrm
%	\label{lem:up_to_deterministic_transition}
%	Let $\horel{\Gamma}{\Delta_1}{P_1}{\ \Re\ }{\Delta_2}{Q_1}$ such
%	that if whenever:
%%
%	\begin{enumerate}
%		\item	$\forall \news{\widetilde{m_1}} \bactout{n}{V_1}$ such that
%			$
%				\horel{\Gamma}{\Delta_1}{P_1}{\hby{\news{\widetilde{m_1}} \bactout{n}{V_1}}}{\Delta_3}{P_3}
%			$
%			implies that $\exists Q_2, V_2$ such that
%			\[
%				\horel{\Gamma}{\Delta_2}{Q_1}{\Hby{\news{\widetilde{m_2}} \bactout{n}{V_2}}}{\Delta_2'}{Q_2}
%			\]
%			and
%			\[
%				\horel{\Gamma}{\Delta_3}{P_3}{\Hby{\dtau}}{\Delta_1'}{P_2}
%			\]
%			and for fresh $t$:
%			\[
%				\horel{\Gamma}{\Delta_1''}{\newsp{\widetilde{m_1}}{P_2 \Par \htrigger{t}{V_1}}}
%				{\ \Re\ }
%				{\Delta_2''}{}{\newsp{\widetilde{m_2}}{Q_2 \Par \htrigger{t}{V_2}}}
%%				\mhorel{\Gamma}{\Delta_1''}{\newsp{\widetilde{m_1}}{P_2 \Par \hotrigger{t}{x}{s}{V_1}}}
%%				{\ \Re\ }
%%				{\Delta_2''}{}{\newsp{\widetilde{m_2}}{Q_2 \Par \hotrigger{t}{x}{s}{V_2}}}
%			\]
%%
%		\item	$\forall \ell \not= \news{\widetilde{m}} \bactout{n}{V}$ such that
%			$
%				\horel{\Gamma}{\Delta_1}{P_1}{\hby{\ell}}{\Delta_3}{P_3}
%			$
%			implies that $\exists Q_2$ such that 
%			\[
%				\horel{\Gamma}{\Delta_1}{Q_1}{\hat{\Hby{\ell}}}{\Delta_2'}{Q_2}
%			\]
%			and
%			\[
%				\horel{\Gamma}{\Delta_3}{P_3}{\Hby{\dtau}}{\Delta_1'}{P_2}
%			\]
%			and
%			$\horel{\Gamma}{\Delta_1'}{P_2}{\ \Re\ }{\Delta_2'}{Q_2}$
%
%		\item	The symmetric cases of 1 and 2.
%	\end{enumerate}
%	Then $\Re\ \subseteq\ \wb$.
%\end{lemma}
%
%
%\begin{proof}
%	The proof is easy by considering the closure
%	\[
%		\Re^{\Hby{\dtau}} = \set{ \horel{\Gamma}{\Delta_1'}{P_2}{,}{\Delta_2'}{Q_1} \setbar \horel{\Gamma}{\Delta_1}{P_1}{\ \Re\ }{\Delta_2'}{Q_1},
%		\horel{\Gamma}{\Delta_1}{P_1}{\Hby{\dtau}}{\Delta_1'}{P_2} }
%	\]
%	We verify that $\Re^{\Hby{\dtau}}$ is a bisimulation with
%	the use of \propref{app:lem:tau_inert}.
%	\qed
%\end{proof}


%%%%%%%%%%%%%%%%%%%%%%%%%%%%%%%%%%%%%%%%%%%%%%%%%%%%%
%          COINCIDENCE
%%%%%%%%%%%%%%%%%%%%%%%%%%%%%%%%%%%%%%%%%%%%%%%%%%%%%

\subsection{Proof of \thmref{the:coincidence}}
\label{app:sub_coinc}


\noi We split the proof of \thmref{the:coincidence} (Page \pageref{the:coincidence}) into 
several lemmas:
\begin{enumerate}[$-$]
\item	\lemref{app:lem:wb_eq_wbf} establishes $\hwb\ =\ \fwb$.
\item	\lemref{app:lem:wb_is_wbc} exploits the process substitution result
		(\lemref{lem:process_subst}) to prove that $\hwb \subseteq \wbc$.
\item	\lemref{app:lem:wbc_is_cong} shows that $\wbc$ is a congruence
		which implies $\wbc \subseteq \cong$.
\item	\lemref{app:lem:cong_is_wb} shows  that $\cong \subseteq \hwb$.
\end{enumerate}

%By the combination of the lemmas, we can obtain the theorem.

\noi
We now proceed to state and prove these lemmas, together with some auxiliary results.
%\thmref{app:thm:coincidence} (Page~\pageref{app:thm:coincidence}) summarises the coincidence result.

The next lemma states a form of equivalence between the characteristic
and higher order trigger processes.

\begin{lemma}[Trigger Process Equivalence]
	\label{lem:trigger_equiv}
	Let $P$ and $Q$ be processes, $t$ be a fresh name, and
	let $\Gamma; \es; \Delta \proves V_i \hastype U, i \in \set{1, 2}$

	\begin{enumerate}[1)]
		\item	If
				\[
					\horel{\Gamma}{\Delta_1}{\newsp{\widetilde{m_1}}{P \Par \htrigger{t}{V_1}}}
					{\hwb}
					{\Delta_2}{\newsp{\widetilde{m_2}}{Q \Par \htrigger{t}{V_2} }}
				\]
				then for $\Delta_1', \Delta_2'$
				\[
					\horel{\Gamma}{\Delta_1'}{\newsp{\widetilde{m_1}}{P \Par \ftrigger{t}{V_1}{U}}}
					{\hwb}
					{\Delta_2'}{\newsp{\widetilde{m_2}}{Q \Par \ftrigger{t}{V_2}{U}}}
				\]

		\item	If
				\[
					\horel{\Gamma}{\Delta_1}{\newsp{\widetilde{m_1}}{P \Par \ftrigger{t}{V_1}{U}}}
					{\fwb}
					{\Delta_2}{\newsp{\widetilde{m_2}}{Q \Par \ftrigger{t}{V_2}{U}}}
				\]
				then for $\Delta_1', \Delta_2'$
				\[
					\horel{\Gamma}{\Delta_1'}{\newsp{\widetilde{m_1}}{P \Par \htrigger{t}{V_1}}}
					{\fwb}
					{\Delta_2'}{\newsp{\widetilde{m_2}}{Q \Par \htrigger{t}{V_2} }}
				\]
	\end{enumerate}
\end{lemma}

\begin{proof}
	\begin{enumerate}
		\item	Part 1

				\noi Consider the typed relation (for readability, we omit type information):
				\begin{eqnarray*}
					\Re	&=&		\set{	(\newsp{\widetilde{m_1}}{P \Par \ftrigger{t}{V_1}{U} },
										\newsp{\widetilde{m_2}}{Q \Par \ftrigger{t}{V_2}{U}})
								\setbar\\
						&&			\horel{\Gamma}{\Delta_1'}{\newsp{\widetilde{m_1}}{P \Par \htrigger{t}{V_1}}}
									{\hwb}
									{\Delta_2'}{\newsp{\widetilde{m_2}}{Q \Par \htrigger{t}{V_2}}}
						\\
						&&		}
				\end{eqnarray*}
				%
				We show that $\Re \subseteq \hwb$.
				We distinguish four cases, depending on the source/kind of visible action: 
				\begin{enumerate}
					\item	Case
						%
						\[
							\horel{\Gamma}{\Delta_1'}{\newsp{\widetilde{m_1}}{P \Par \ftrigger{t}{V_1}{U} }}
							{\hby{\ell_1}}
							{\Delta_3}{}{\newsp{\widetilde{m_1}'}{P' \Par \ftrigger{t}{V_2}{U} }}
						\]
						%
							then following the requirements of $\Re$ and the freshness of $t$
							we can conclude that there exists $\Delta_1''$ such that
						%
						\[
							\horel{\Gamma}{\Delta_1}{\newsp{\widetilde{m_1}}{P \Par \htrigger{t}{V_1}}}
							{\hby{\ell_1}}
							{\Delta_1''}{\newsp{\widetilde{m_1}'}{P' \Par \htrigger{t}{V_2}}}
						\]
						%
							implies from the characteristic bisimilarity requirement of $\Re$ and
							the freshness of $t$ that $\exists Q', \Delta_2''$ such that
						%
						\begin{eqnarray}
							\horel{\Gamma}{\Delta_2}{\newsp{\widetilde{m_2}}{Q \Par \htrigger{t}{V_2}}}
							{\Hby{\ell_2}}
							{\Delta_2''}{\newsp{\widetilde{m_2}'}{Q' \Par \htrigger{t}{V_2}}}
							\label{proof:trig_equiv00}
						\end{eqnarray}
						%
							and
						\begin{eqnarray}
							\horel{\Gamma}{\Delta_1'''}{\newsp{\widetilde{m_1}''}{P' \Par \htrigger{t}{V_1} \Par C_1}}
							{\hwb}
							{\Delta_2'''}{\newsp{\widetilde{m_2}''}{Q' \Par \htrigger{t}{V_2} \Par C_2}}
							\label{proof:trig_equiv11}
						\end{eqnarray}
						%
							with $C_1$ (resp., $C_2$) being the characteristic trigger process
							in the cases where $\ell_1 = \news{\widetilde{m}} \bactout{n}{V_1'}$ (resp., $\ell_2 = \news{\widetilde{m}'} \bactout{n}{V_2'}$)
							and $C_1 = C_2 = \inact$ otherwise.
						%
							From \eqref{proof:trig_equiv00} we can conclude that $\exists \Delta_4$ such that
						\[
							\horel{\Gamma}{\Delta_2'}{\newsp{\widetilde{m_1}}{Q \Par \ftrigger{t}{V_2}{U}}}
							{\Hby{\ell_2}}
							{\Delta_4}{\newsp{\widetilde{m_2}'}{Q' \Par \ftrigger{t}{V_2}{U}}}
						\]
						%
							Equation \eqref{proof:trig_equiv11} then concludes that
						\[
							\horel{\Gamma}{\Delta_3'}{\newsp{\widetilde{m_1}'''}{P' \Par \ftrigger{t}{V_1}{U} \Par C_1}}
							{\Re}
							{\Delta_4'}{\newsp{\widetilde{m_2}'''}{Q' \Par \ftrigger{t}{V_2}{U} \Par C_2}}
						\]
						%
							as required.

					\item	Case
						\[
							\mhorel{\Gamma}{\Delta_1'}{\newsp{\widetilde{m_1}}{P \Par \ftrigger{t}{V_1}{U}}}
%							{\hby{\bactinp{t}{\map{\btinp{U} \tinact}^{x}}}}
							{\hby{\bactinp{t}{m}}}
							{\Delta_3}{}{\newsp{\widetilde{m_1}}{P \Par \newsp{s}{\binp{s}{y} \mapchar{U}{y} \Par \bout{\dual{s}}{V_1} \inact}}}
						\]
						%
							Following requirements of $\Re$ and the freshness of $t$
							we can conclude that there exists $\Delta_1''$ such that
						%
						\[
							\mhorel{\Gamma}{\Delta_1}{\newsp{\widetilde{m_1}}{P \Par \htrigger{t}{V_1}}}
							{\hby{\bactinp{t}{\omapchar{U}}}}
%							{\hby{\bactinp{t}{m}}}
							{\Delta_1''}{}{\newsp{\widetilde{m_1}}{P \Par \newsp{s}{\binp{s}{y} \mapchar{U}{y} \Par \bout{\dual{s}}{V_1} \inact}}}
						\]
						%
							implies from the higher order bisimilarity requirement of the $\Re$ definition and
							the freshness of $t$ that $\exists Q', \Delta_2''$ such that
						%
							\begin{eqnarray}
								\begin{array}{crll}
									& \Gamma; \es; \Delta_2 & \proves &		
									\newsp{\widetilde{m_2}}{Q \Par \htrigger{t}{V_2}}
									\\
									\Hby{} &&&
									\newsp{\widetilde{m_2}}{Q_2 \Par \htrigger{t}{V_2}}
									\\
%									{\hby{\bactinp{t}{m}}}& & &
									{\hby{\bactinp{t}{\omapchar{U}}}} & & &
									\newsp{\widetilde{m_2}}{Q_2 \Par \newsp{s}{\binp{s}{y} \mapchar{U}{y} \Par \bout{\dual{s}}{V_2} \inact}}
									\\
									\Hby{} & \Delta_2'' & \proves & Q'
								\end{array}
								\label{proof:trig_equiv22}
							\end{eqnarray}
						%
							and
						%
							\nhorel{\Gamma}{\Delta_1''}{\newsp{\widetilde{m_1}}{P \Par \newsp{s}{\binp{s}{y} \mapchar{U}{y} \Par \bout{\dual{s}}{V_1} \inact}}}
							{\hwb}
							{\Delta_2''}{\newsp{\widetilde{m_2}}{Q'}}
							{proof:trig_equiv33}
						%
							The freshness of $t$ allows us to mimic the transitions
							in \eqref{proof:trig_equiv22} to get that for $\Delta_4$
						%
							\begin{eqnarray*}
								\begin{array}{crll}
									& \Gamma; \es; \Delta_2' & \proves &		
									\newsp{\widetilde{m_2}}{Q \Par \ftrigger{t}{V_2}{U}}
									\\
									\Hby{} &&&
									\newsp{\widetilde{m_2}}{Q_2 \Par \ftrigger{t}{V_2}{U}}
									\\
%									{\hby{\bactinp{t}{\map{\btinp{U} \tinact}^{x}}}}& & &
									{\hby{\bactinp{t}{m}}}& & &
									\newsp{\widetilde{m_2}}{Q_2 \Par \newsp{s}{\binp{s}{y} \mapchar{U}{y} \Par \bout{\dual{s}}{V_2} \inact}}
									\\
									\Hby{} & \Delta_4 & \proves & Q'
								\end{array}
							\end{eqnarray*}
						%
							The conclusion is immediate from \eqref{proof:trig_equiv33}.


				\item The action comes from the interaction of $P$ and $\htrigger{t}{V_1}$: This case is not possible, due to the freshness of $t$.
				\end{enumerate}


		\item	Part 2 

				\noi Consider the typed relation (for readability, we omit type information):
				\begin{eqnarray*}
					\Re	&=&		\set{	(\newsp{\widetilde{m_1}}{P \Par \htrigger{t}{V_1}},
										\newsp{\widetilde{m_2}}{Q \Par \htrigger{t}{V_2}})
								\setbar\\
						&&			\mhorel{\Gamma}{\Delta_1'}{\newsp{\widetilde{m_1}}{P \Par \ftrigger{t}{V_1}{U}}}
									{\fwb}
									{\Delta_2'}{}{\newsp{\widetilde{m_2}}{Q \Par \ftrigger{t}{V_2}{U}}}
						\\
						&&		}
				\end{eqnarray*}
				%
				We show that $\Re \subseteq \fwb$. We distinguish four cases, depending on the source/kind of visible action: 
				\begin{enumerate}
					\item	Case
						%
						\[
							\horel{\Gamma}{\Delta_1'}{\newsp{\widetilde{m_1}}{P \Par \htrigger{t}{V_1}}}
							{\hby{\ell_1}}
							{\Delta_3}{\newsp{\widetilde{m_1}'}{P' \Par \htrigger{t}{V_1}}}
						\]
						%
							then following the requirements of $\Re$ and the freshness of $t$
							we can conclude that there exists $\Delta_1''$ such that
						%
						\[
							\horel{\Gamma}{\Delta_1}{\newsp{\widetilde{m_1}}{P \Par \ftrigger{t}{V_1}{U}}}
							{\hby{\ell_1}}
							{\Delta_1''}{\newsp{\widetilde{m_1}'}{P' \Par \ftrigger{t}{V_1}{U}}}
						\]
						%
							implies from the characteristic bisimilarity requirement of $\Re$ and
							the freshness of $t$ that $\exists Q', \Delta_2''$ such that
						%
							\begin{eqnarray}
							\horel{\Gamma}{\Delta_2}{\newsp{\widetilde{m_2}}{Q \Par \ftrigger{t}{V_2}{U}}}
							{\Hby{\ell_2}}
							{\Delta_2''}{\newsp{\widetilde{m_2}'}{Q' \Par \ftrigger{t}{V_2}{U}}}
							\label{proof:trig_equiv0}
							\end{eqnarray}
						%
							and
							\nhorel{\Gamma}{\Delta_1'''}{\newsp{\widetilde{m_1}''}{P' \Par \ftrigger{t}{V_1}{U} \Par C_1}}
							{\fwb}
							{\Delta_2'''}{\newsp{\widetilde{m_2}''}{Q' \Par \ftrigger{t}{V_2}{U} \Par C_2}}
							{proof:trig_equiv1}
						%
							with $C_1$ (resp., $C_2$) being the characteristic trigger process
							in the cases where $\ell_1 = \news{\widetilde{m}} \bactout{n}{V_1'}$ (resp., $\ell_2 = \news{\widetilde{m}'} \bactout{n}{V_2'}$)
							and $C_1 = C_2 = \inact$ otherwise.
						%
							From \eqref{proof:trig_equiv0} we can conclude that $\exists \Delta_4$ such that
						\[
							\horel{\Gamma}{\Delta_2'}{\newsp{\widetilde{m_1}}{Q \Par \htrigger{t}{V_2}}}
							{\Hby{\ell_2}}
							{\Delta_4}{\newsp{\widetilde{m_2}'}{Q' \Par \htrigger{t}{V_2}}}
						\]
						%
							Equation \eqref{proof:trig_equiv1} then concludes that
						\[
							\horel{\Gamma}{\Delta_3'}{\newsp{\widetilde{m_1}'''}{P' \Par \htrigger{t}{V_1} \Par C_1}}
							{\Re}
							{\Delta_4'}{}{\newsp{\widetilde{m_2}'''}{Q' \Par \htrigger{t}{V_2} \Par C_2}}
						\]
						%
							as required.

					\item	%\jasks{I would probably put cases 2 and 3 within a single case, which is the case when the right-hand side process moves. }
						Case
						\[
							\horel{\Gamma}{\Delta_1'}{\newsp{\widetilde{m_1}}{P \Par \htrigger{t}{V_1}}}
							{\hby{\bactinp{t}{\omapchar{U}}}}
							{\Delta_3}{\newsp{\widetilde{m_1}}{P \Par \newsp{s}{\binp{s}{y} \mapchar{U}{y} \Par \bout{\dual{s}}{V_1} \inact}}}
						\]
						%
							Following requirements of $\Re$ and the freshness of $t$
							we can conclude that there exists $\Delta_1''$ such that
						%
						\[
							\horel{\Gamma}{\Delta_1}{\newsp{\widetilde{m_1}}{P \Par \ftrigger{t}{V_1}{U}}}
							{\hby{\bactinp{t}{m}}}
							{\Delta_1''}{\newsp{\widetilde{m_1}}{P \Par \newsp{s}{\binp{s}{y} \mapchar{U}{y} \Par \bout{\dual{s}}{V_1} \inact}}}
						\]
						%
							implies from the higher order bisimilarity requirement of the $\Re$ definition and
							the freshness of $t$ that $\exists Q', \Delta_2''$ such that
						%
							\begin{eqnarray}
								\begin{array}{crll}
									& \Gamma; \es; \Delta_2 & \proves &		
									\newsp{\widetilde{m_2}}{Q \Par \ftrigger{t}{V_2}{U}}
									\\
									\Hby{} &&&
									\newsp{\widetilde{m_2}}{Q_2 \Par \ftrigger{t}{V_2}{U}}
									\\
									{\hby{\bactinp{t}{m}}}& & &
									\newsp{\widetilde{m_2}}{Q_2 \Par \newsp{s}{\binp{s}{y} \mapchar{U}{y} \Par \bout{\dual{s}}{V_2} \inact}}
									\\
									\Hby{} & \Delta_2'' & \proves & Q'
								\end{array}
								\label{proof:trig_equiv2}
							\end{eqnarray}
						%
							and
						%
							\begin{eqnarray}
							\horel{\Gamma}{\Delta_1''}{\newsp{\widetilde{m_1}}{P \Par \newsp{s}{\binp{s}{y} \mapchar{U}{y} \Par \bout{\dual{s}}{V_1} \inact}}}
							{\fwb}
							{\Delta_2''}{\newsp{\widetilde{m_2}}{Q'}}
							\label{proof:trig_equiv3}
							\end{eqnarray}
						%
							The freshness of $t$ allows us to mimic the transitions
							in \eqref{proof:trig_equiv2} to get that for $\Delta_4$
						%
							\begin{eqnarray*}
								\begin{array}{crll}
									& \Gamma; \es; \Delta_2' & \proves &		
									\newsp{\widetilde{m_2}}{Q \Par \htrigger{t}{V_2}}
									\\
									\Hby{} &&&
									\newsp{\widetilde{m_2}}{Q_2 \Par \htrigger{t}{V_2}}
									\\
									{\hby{\bactinp{t}{\omapchar{U}}}}& & &
									\newsp{\widetilde{m_2}}{Q_2 \Par \newsp{s}{\binp{s}{y} \mapchar{U}{y} \Par \bout{\dual{s}}{V_2} \inact}}
									\\
									\Hby{} & \Delta_4 & \proves & Q'
								\end{array}
							\end{eqnarray*}
						%
							The conclusion is immediate from \eqref{proof:trig_equiv3}.

					\item	Case
						\[
							\mhorel{\Gamma}{\Delta_1'}{\newsp{\widetilde{m_1}}{P \Par \hotrigger{t}{x}{s}{V_1}}}
							{\hby{\bactinp{t}{ \abs{x}{\binp{t'}{y}{\appl{y}{x}}  }}}}
							{\Delta_3'}{}{\newsp{\widetilde{m_1}}{P \Par \newsp{s}{  \appl{(\abs{x}{\binp{t'}{y}{\appl{y}{x}})}{s}  \Par \bout{\dual{s}}{V_1} \inact }}}}
						\]
						We show that there $\exists, \Delta_4, \newsp{\widetilde{m_1}}{Q \Par \hotrigger{t'}{x}{s}{V_1}}$ such that
						\[
							\mhorel{\Gamma}{\Delta_2'}{\newsp{\widetilde{m_1}}{Q \Par \hotrigger{t}{x}{s}{V_1}}}
							{\Hby{\bactinp{t}{ \abs{x}{\binp{t'}{y}{\appl{y}{x}}  }}}}
							{\Delta_4}{}{\newsp{\widetilde{m_1}}{Q \Par \hotrigger{t'}{x}{s}{V_1}}}
						\]
						and
						\[
							\mhorel{\Gamma}{\Delta_3'}{\newsp{\widetilde{m_1}}{P \Par \newsp{s}{  \appl{(\abs{x}{\binp{t'}{y}{\appl{y}{x}})}{s}  \Par \bout{\dual{s}}{V_1} \inact }}}}
							{\Hby{\tau_d}}
							{\Delta_3}{}{\newsp{\widetilde{m_1}}{P \Par \hotrigger{t'}{x}{s}{V_1}}}
						\]
						The result
						\[
							\mhorel{\Gamma}{\Delta_3}{\newsp{\widetilde{m_1}}{P \Par \hotrigger{t'}{x}{s}{V_1}}}
							{\Re}
							{\Delta_4}{}{\newsp{\widetilde{m_1}}{Q \Par \hotrigger{t'}{x}{s}{V_1}}}
						\]
						is immediate from the definition of $\Re$.

				\item The action comes from the interaction of $P$ and $\htrigger{t}{V_1}$: This case is not possible, due to the freshness of $t$.
				\end{enumerate}
	\end{enumerate}
	\qed
\end{proof}

%%%%%%%%%%%%%%%%%%%%%%%%%%%%%%%%%%%%%%%%%%%%%%%%%%%%%%%%%%%%%%%%%%%%%%%%%%%%%%%%%%%%%%%%%
%    Higher Weak Bisimilarity = Characteristic Weak Bisimilarity  (\hwb = \fwb)
%%%%%%%%%%%%%%%%%%%%%%%%%%%%%%%%%%%%%%%%%%%%%%%%%%%%%%%%%%%%%%%%%%%%%%%%%%%%%%%%%%%%%%%%%


\begin{lemma}
	\label{app:lem:wb_eq_wbf}
	$\hwb = \fwb$.
\end{lemma}

\begin{proof}
	\noi
	We split the proof into the direction
	$\hwb \subseteq \fwb$ and the direction
	$\fwb \subseteq \hwb$.
 Since the two equivalences differ only in the output case, in both cases our analysis focuses on output actions.

	\begin{enumerate}
		\item	Direction $\hwb \subseteq \fwb$.

				\noi Consider the typed relation (for readability, we omit type information):
		%
				\[
					\Re = \set{
								%\horel{\Gamma}{\Delta_1}{P}{\ ,\ }{\Delta_2}{Q} 
								(P, Q) 
								\setbar
								\horel{\Gamma}{\Delta_1}{P}{\hwb}{\Delta_2}{Q}}
				\]
		%
				We show that $\Re$ is a characteristic bisimulation.
				Suppose
				$
						\horel{\Gamma}{\Delta_1}{P}{\hby{ \ell}}{\Delta_1'}{P'}
		%				\label{lem:wb_eq_wbf1}
				$.
				The proof proceeds by a case analysis on the transition label $\ell$.
				As stated earlier, we only detail the case 
				$\ell = \news{\widetilde{m_1}} \bactout{n}{V_1}$, which is the only non-trivial case.

							\smallskip
							
							 From the definition of $\Re$ we have that if:
						%
							\begin{eqnarray}
								\horel{\Gamma}{\Delta_1}{P}{\hby{\news{\widetilde{m_1}} \bactout{n}{V_1}}}{\Delta_1''}{P'}
								\label{lem:wb_eq_wbf1}
							\end{eqnarray}
							then $\exists Q, V_2$ such that:
						%
							\begin{eqnarray}
								\horel{\Gamma}{\Delta_2}{Q}{\Hby{\news{\widetilde{m_2}} \bactout{n}{V_2}}}{\Delta_2''}{Q'}
								\label{lem:wb_eq_wbf2}
							\end{eqnarray}
						%
							and for fresh $t$:
								\nhorel{\Gamma}{\Delta_1'}{\newsp{\widetilde{m_1}}{P' \Par \htrigger{t}{V_1}}}
								{\hwb}
								{\Delta_2'}{\newsp{\widetilde{m_2}}{Q' \Par \htrigger{t}{V_2} }}
								{lem:wb_eq_wbf3}
						%
							\noi 
							To show that $\Re$ is a characteristic bisimulation
							after the fact that transition~\eqref{lem:wb_eq_wbf1} implies transition~\eqref{lem:wb_eq_wbf2},
							we need to show that for fresh $t$:
						%
							\begin{eqnarray}
								\mhorel{\Gamma}{\Delta_3}{\newsp{\widetilde{m_1}}{P' \Par \ftrigger{t}{V_1}{U}}}
								{\Re}
								{\Delta_4}{}{\newsp{\widetilde{m_2}}{Q' \Par \ftrigger{t}{V_2}{U}}}
								\label{lem:wb_eq_wbf4}
							\end{eqnarray}
						%
							\noi which follows from \eqref{lem:wb_eq_wbf3}, \lemref{lem:trigger_equiv}(1),
							and the definition of $\Re$.

					%\item	The other cases are trivial.
				%\end{itemize}

		\item	Direction $\fwb \subseteq \hwb$.
				\noi Consider the typed relation (for readability, we omit type information):
		%
				\[
					\Re = \set{
								%\horel{\Gamma}{\Delta_1}{P}{\ ,\ }{\Delta_2}{Q} 
								(P, Q) 
								\setbar
								\horel{\Gamma}{\Delta_1}{P}{\fwb}{\Delta_2}{Q}}
				\]
		%
				We show that $\Re$ is a higher-order bisimulation.
				Suppose
				$
						\horel{\Gamma}{\Delta_1}{P}{\hby{ \ell}}{\Delta_1'}{P'}
		%				\label{lem:wb_eq_wbf1}
				$.
				The proof does a case analysis on the transition label $\ell$.
				%\begin{itemize}
				%	\item	
				Here again we only detail the case $\ell = \news{\widetilde{m_1}} \bactout{n}{V_1}$.
				
				\smallskip

							\noi From the definition of $\Re$ we have that if:
						%
							\begin{eqnarray}
								\horel{\Gamma}{\Delta_1}{P}{\hby{\news{\widetilde{m_1}} \bactout{n}{V_1}}}{\Delta_1''}{P'}
								\label{lem:fwb_sub_hbf1}
							\end{eqnarray}
							then $\exists Q, V_2$ such that:
						%
							\begin{eqnarray}
								\horel{\Gamma}{\Delta_2}{Q}{\Hby{\news{\widetilde{m_2}} \bactout{n}{V_2}}}{\Delta_2''}{Q'}
								\label{lem:fwb_sub_hbf2}
							\end{eqnarray}
						%
							and for fresh $t$:
								\nhorel{\Gamma}{\Delta_1'}{\newsp{\widetilde{m_1}}{P' \Par \ftrigger{t}{V_1}{U}}}
								{\fwb}
								{\Delta_2'}{\newsp{\widetilde{m_2}}{Q' \Par \ftrigger{t}{V_2}{U} }}
								{lem:fwb_sub_hbf3}
						%
							\noi 
							To show that $\Re$ is a higher-order bisimulation
							after the fact that transition~\eqref{lem:fwb_sub_hbf1} implies transition~\eqref{lem:fwb_sub_hbf2},
							we need to show that for fresh $t$:
						%
							\begin{eqnarray}
								\mhorel{\Gamma}{\Delta_3}{\newsp{\widetilde{m_1}}{P' \Par \htrigger{t}{V_1}}}
								{\Re}
								{\Delta_4}{}{\newsp{\widetilde{m_2}}{Q' \Par \htrigger{t}{V_2}}}
								\label{lem:fwb_sub_hbf4}
							\end{eqnarray}
						%
							which follows from \eqref{lem:fwb_sub_hbf3}, \lemref{lem:trigger_equiv}(2),
							and the definition of $\Re$.

					%\item	The other cases are trivial.
				%\end{itemize}
	\end{enumerate}
	\qed
\end{proof}


%%%%%%%%%%%%%%%%%%%%%%%%%%%%%%%%%%%%%%%%%%%%%%%%%%%%%%%%%%%%
% PROCESS SUBSTITUTION
%%%%%%%%%%%%%%%%%%%%%%%%%%%%%%%%%%%%%%%%%%%%%%%%%%%%%%%%%%%%

We state an auxiliary lemma that captures a property of trigger
processes in terms of process equivalence.
\begin{lemma}[Trigger Process Application]
	\label{lem:trigger_application}
	Let $P$ and $Q$ be processes. Also, let $t$ be a fresh name.
	\begin{enumerate}
		\item
			If $n_1 \not= n_2$ with $\Gamma; \proves; \Delta \proves n_i \hastype U$ with $U \not= \tinact$
			and
			\[
				\horel{\Gamma}{\Delta_1}{\newsp{\widetilde{m_1}}{P \Par \appl{(\trvalx{t})}{n_1} }}
				{\hwb}
				{\Delta_2}{\newsp{\widetilde{m_2}}{Q \Par \appl{(\trvalx{t})}{n_2} }}
			\]
			then $n_1, n_2$ sessions and $\dual{n_1} \in \fn{P}$ and $\dual{n_2} \in \fn{Q}$.

		\item
			If
			\[
				\horel{\Gamma}{\Delta_1}{\newsp{\widetilde{m_1}}{P \Par  \appl{\omapchar{U}}{n_2} }}%\appl{\abs{x}{\mapchar{U}{x}}{n_1}} }}
				{\hwb}
				{\Delta_2}{\newsp{\widetilde{m_2}}{Q \Par \appl{\omapchar{U}}{n_1} }} %\appl{\abs{x}{\mapchar{U}{x}}{n_1}} }}
			\]
			then whenever
			\[
				\horel{\Gamma}{\Delta_1}{\newsp{\widetilde{m_1}}{P \Par \appl{\omapchar{U}}{n_1} }} %\appl{\abs{x}{\mapchar{U}{x}}{n_1}} }}
				{\hby{\ell_1}}
				{\Delta_1'}{\newsp{\widetilde{m_1}'}{P' \Par \appl{(\trvalx{t})}{n_1} }}
			\]
			implies
			\[
				\horel{\Gamma}{\Delta_2}{\newsp{\widetilde{m_1}}{Q \Par \appl{\omapchar{U}}{n_2} }} %\appl{\abs{x}{\mapchar{U}{x}}{n_2}} }}
				{\Hby{\hat{\ell_2}}}
				{\Delta_2'}{\newsp{\widetilde{m_2}'}{Q' \Par \appl{(\trvalx{t})}{n_2} }} % }}
			\]
			
		\item
			If
			\[
				\horel{\Gamma}{\Delta_1}{ \newsp{\widetilde{m_1}}{P \Par \bout{t}{n_1} \inact} }
				{\hwb}
				{\Delta_2}{ \newsp{\widetilde{m_2}}{Q \Par \bout{t}{n_2} \inact} }
			\]
			then
			\[
				\horel{\Gamma}{\Delta_1}{ \newsp{m_1}{P \Par \binp{t}{x} \appl{x}{n_1}} }
				{\hwb}
				{\Delta_2}{ \newsp{m_2}{Q \Par \binp{t}{x} \appl{x}{n_2}} }
			\]

		\item
			If $n$ fresh and
			\[
				\horel{\Gamma}{\Delta_1}{ \newsp{\widetilde{m_1}}{P \subst{n}{x} \Par \bout{t_1}{n_1} \inact }}
				{\hwb}
				{\Delta_2}{ \newsp{\widetilde{m_2}}{Q \subst{n}{x} \Par \bout{t_1}{m_1} \inact }}
			\]
			then
			\[
				\horel{\Gamma}{\Delta_1}{ \newsp{\widetilde{m_1}}{P \subst{n_1}{x} }}
				{\hwb}
				{\Delta_2}{ \newsp{\widetilde{m_2}}{Q \subst{m_1}{x} }}
			\]

%		\item
%			If
%			\[
%				\horel{\Gamma}{\Delta_1}{ \newsp{\widetilde{m_1}}{P \Par \bout{t_1}{n_1} \inact \Par \bout{t_2}{n_2} \inact} }
%				{\hwb}
%				{\Delta_2}{ \newsp{\widetilde{m_2}}{Q \Par \bout{t_1}{m_1} \inact \Par \bout{t_2}{m_2} \inact} }
%			\]
%			then
%			\[
%				\horel{\Gamma}{\Delta_1}{ \newsp{\widetilde{m_1}}{P \Par \bout{n_1}{n_2} \bout{t}{n_1} \inact} }
%				{\hwb}
%				{\Delta_2}{ \newsp{\widetilde{m_2}}{Q \Par \bout{m_1}{m_2} \bout{t}{m_1} \inact} }
%			\]
	\end{enumerate}
\end{lemma}

\begin{proof}
	\begin{enumerate}
		\item	The proof for Part 1 is done by contradiction.
				Assume that $\dual{n_1} \notin \fn{P}$ and $\dual{n_2} \notin \fn{Q}$.
				Then the bisimulation requirement allows us to observe for some $U \not= \tinact$:
				\[
					\horel{\Gamma}{\Delta_1}{\newsp{\widetilde{m_1}}{P \Par \appl{(\trvalx{t})}{n_1} }}
					{ \hby{ \bactinp{t}{\omapchar{U}}  } \hby{\ell_1} }
					{\Delta_1'}{\newsp{\widetilde{m_1}}{P \Par \bout{t'}{n_1} \inact }}
				\]
				with $\subj{\ell_1} = n_1$, then from the freshness of $t$
				\[
					\horel{\Gamma}{\Delta_2}{\newsp{\widetilde{m_2}}{Q \Par \appl{(\trvalx{t})}{n_1} }}
					{ \Hby{ \bactinp{t}{\omapchar{U}}  } \Hby{\ell_2} }
					{\Delta_2'}{\newsp{\widetilde{m_2}}{Q' \Par \appl{\omapchar{U}}{n_2}}}
				\]
				with $\subj{\ell_2} = n_1$.
				But then in the former result we can observe an action on $t'$ and on the latter
				we cannot, thus leading to a contradiction with respect to bisimilarity.

		\item	The proof for Part 2 is also done by contradiction. Assume that
				\[
					\horel{\Gamma}{\Delta_2}{\newsp{\widetilde{m_1}}{Q \Par \appl{\omapchar{U}}{n_2} }} %\appl{\abs{x}{\mapchar{U}{x}}{n_2}} }}
					{\not\Hby{\hat{\ell_2}}}
					{\Delta_2'}{\newsp{\widetilde{m_2}'}{Q' \Par \appl{(\trvalx{t})}{n_2} }} % }}
				\]
				From the bisimilarity requirement we can observe
				\[
					\horel{\Gamma}{\Delta_2}{\newsp{\widetilde{m_1}}{Q \Par \appl{\omapchar{U}}{n_2} }} %\appl{\abs{x}{\mapchar{U}{x}}{n_2}} }}
					{\Hby{\hat{\ell_2}}}
					{\Delta_2'}{\newsp{\widetilde{m_2}'}{Q' \Par \appl{\omapchar{U}}{n_2} }} % }}
				\]
				But then we can observe an action on fresh name $t$ on process
				\[
					\Gamma; \es; \Delta_1' \proves \newsp{\widetilde{m_1}'}{P' \Par \appl{(\trvalx{t})}{n_1} } \hastype \Proc
				\]
				that cannot be observed by process
				\[
					\Gamma; \es; \Delta_2' \proves \newsp{\widetilde{m_2}'}{Q' \Par \appl{\omapchar{U}}{n_2} }
				\]

		\item	For the proof of Part 3 we do a case analysis on the transitions for checking the bisimulation requirements. The interesting case is
				\[
					\Gamma; \Delta_1 \proves \newsp{\widetilde{m_1}}{P \Par \binp{t}{x} \appl{x}{n_1}}
					\hby{\bactinp{t}{\omapchar{U}}} 
					\Gamma; \Delta_1'' \proves \newsp{\widetilde{m_1}''}{P \Par \appl{\omapchar{U}}{n_1}}
				\]
				From the freshness of $t$ we can derive that
				\[
					\Gamma; \Delta_2 \proves \newsp{\widetilde{m_2}}{Q \Par \binp{t}{x} \appl{x}{n_2}}
					\Hby{\bactinp{t}{\omapchar{U}}} 
					\Gamma; \Delta_2'' \proves \newsp{\widetilde{m_2}'}{Q'' \Par \appl{\omapchar{U}}{n_2}}
				\]
				From the bisimulation requirement of the hypothesis that
				\[
					\horel{\Gamma}{\Delta_1}{ \newsp{\widetilde{m_1}}{P \Par \bout{t}{n_1} \inact} }
					{\hby{\bactout{t}{n_1} }}
					{\Delta_1'}{ \newsp{\widetilde{m_1}'}{P} }
				\]
				implies
				\[
					\horel{\Gamma}{\Delta_2}{ \newsp{\widetilde{m_2}}{Q \Par \bout{t}{n_2} \inact} }
					{\Hby{\bactout{t}{n_2} }}
					{\Delta_2'}{ \newsp{\widetilde{m_2}'}{Q'} }
				\]
				and
				\[
					\mhorel{\Gamma}{\Delta_1'}{ \newsp{\widetilde{m_1}'}{P \Par \binp{t}{x} \newsp{s}{ \binp{s}{y} \appl{x}{y} \Par \bout{\dual{s}}{n_1} \inact } } }
					{\hwb}
					{\Delta_2'}{}{ \newsp{\widetilde{m_2}'}{Q' \Par \binp{t}{x} \newsp{s}{ \binp{s}{y} \appl{x}{y} \Par \bout{\dual{s}}{n_2} \inact }} }
				\]
				Whenever
				\[
				\begin{array}{rcl}
					&& \Gamma; \Delta_1' \proves \newsp{\widetilde{m_1}'}{P \Par \binp{t}{x} \newsp{s}{ \binp{s}{y} \appl{x}{y} \Par \bout{\dual{s}}{n_1} \inact}}\\ 
					\hby{\bactinp{t}{\omapchar{U}}}&& 
					\Delta_1'' \proves \newsp{\widetilde{m_1}''}{P \Par \newsp{s}{ \binp{s}{y} \appl{\omapchar{U}}{y} \Par \bout{\dual{s}}{n_1} \inact }}
					\\
					\hby{\dtau} &&
					\Delta_1'' \proves \newsp{\widetilde{m_1}''}{P \Par \appl{\omapchar{U}}{n_1}}
				\end{array}
				\]
				then
				\[
				\begin{array}{rcl}
					&&\Gamma; \Delta_1' \proves \newsp{\widetilde{m_2}'}{Q' \Par \binp{t}{x} \newsp{s}{ \binp{s}{y} \appl{x}{y} \Par \bout{\dual{s}}{n_2} \inact } }
					\\
					\Hby{\bactinp{t}{\omapchar{U}}} &&
					\Delta_2'' \proves \newsp{\widetilde{m_2}''}{Q''' \Par \newsp{s}{ \binp{s}{y} \appl{\omapchar{U}}{y} \Par \bout{\dual{s}}{n_2} \inact }}
					\\
					\Hby{\dtau} &&
					\Delta_2'' \proves \newsp{\widetilde{m_2}''}{Q'' \Par \appl{\omapchar{U}}{n_2}}
				\end{array}
				\]
				which concludes the case.
				
		\item	Part 4. Let typed relation
				\begin{eqnarray*}
					\Re &=& \set{	\horel{\Gamma}{\Delta_1}{ \newsp{\widetilde{m_1}}{P \subst{n_1}{x} }}
									{\hwb}
									{\Delta_2}{ \newsp{\widetilde{m_2}}{Q \subst{m_1}{x} }} \setbar \\
						&&
									\horel{\Gamma}{\Delta_1}{ \newsp{\widetilde{m_1}}{P \subst{n}{x} \Par \bout{t_1}{n_1} \inact }}
									{\hwb}
									{\Delta_2}{ \newsp{\widetilde{m_2}}{Q \subst{n}{x} \Par \bout{t_1}{m_1} \inact }}
					}
				\end{eqnarray*}
				\[
					\horel{\Gamma}{\Delta_1}{\newsp{\widetilde{m_1}}{P \subst{n_1}{x}}}
					{\hby{\ell_1}}
					{\Delta_1'}{\newsp{\widetilde{m_1}'}{P' \subst{n_1}{x}}}
				\]
				We prove that $\Re$ is a bisimulation by a case analysis on the subject of action $\ell_1$.
				\begin{itemize}
					\item	If $\subj{\ell_1} \not= n_1$ then the proof is straightforward from the premise of the proposition.
							First observed that
							\[
								\horel{\Gamma}{\Delta_1}{\newsp{\widetilde{m_1}}{P \subst{n}{x}} \Par \bout{t}{n_1} \inact}
								{\hby{\ell_1}}
								{\Delta_1'}{\newsp{\widetilde{m_1}'}{P' \subst{n}{x}} \Par \bout{t}{n_1} \inact}
							\]
							implies
							\[
								\horel{\Gamma}{\Delta_2}{\newsp{\widetilde{m_2}'}{Q \subst{n}{x}} \Par \bout{t}{n_2} \inact}
								{\Hby{\ell_2}}
								{\Delta_2'}{\newsp{\widetilde{m_2}'}{Q' \subst{n}{x}} \Par \bout{t}{n_2} \inact}
							\]
							and
							\[
								\horel{\Gamma}{\Delta_2}{\newsp{\widetilde{m_1}'}{P' \subst{n}{x}} \Par \bout{t}{n_2} \inact \Par C_1}
								{\hwb}
								{\Delta_2'}{\newsp{\widetilde{m_2}'}{Q' \subst{n}{x}} \Par \bout{t}{n_2} \inact \Par C_2}
							\]
							with $C_1 = \htrigger{t}{n_1}$ and $C_2 = \htrigger{t}{n_2}$ if $\ell_1$ and $\ell_2$ are output actions,
							$C_1 = \inact$ and $C_2 = \inact$ otherwise.
							From here we can imply that 
							\[
								\horel{\Gamma}{\Delta_2}{\newsp{\widetilde{m_1}}{Q \subst{n_2}{x}}}
								{\Hby{\ell_2}}
								{\Delta_2'}{\newsp{\widetilde{m_2}'}{Q' \subst{n_2}{x}}}
							\]
							Furthermore, we can easily see that
							\[
								\horel{\Gamma}{\Delta_2}{\newsp{\widetilde{m_1}'}{P \subst{n_1}{x}} \Par C_1}
								{\ \Re\ }
								{\Delta_2'}{\newsp{\widetilde{m_2}'}{Q' \subst{n_2}{x}} \Par C_2}
							\]

					\item	$\subj{\ell_1} = n_1$. We distinguish two subcases
							\begin{itemize}
								\item	$n_1 = n_2$. The case is similar as the previous case.
								\item	$n_1 \not= n_2$.
										From the premise and Part 1 of this lemma we get
										that $\dual{n_1} \in \fn{P}$ and $\dual{n_2} \in \fn{Q}$.
										The latter implies that this case is not possible, since
										no external action $\ell_1$ would be observed, because
										of the typed transition requirement.
							\end{itemize}

					\item	$\ell_1 = \tau$. This implies the untyped transitions
							\begin{eqnarray}
								\newsp{\widetilde{m_1}}{P \subst{n_1}{x}} &\hby{\ell_{11}'}& \newsp{\widetilde{m_{11}}}{P_1 \subst{n_1}{x}}
								\label{lem:tr_app_41}\\
								\newsp{\widetilde{m_1}}{P \subst{n_1}{x}} &\hby{\ell_{12}'}& \newsp{\widetilde{m_{12}}}{P_2 \subst{n_1}{x}}
								\label{lem:tr_app_42}\\
								\ell_1' &\asymp& \ell_2'
							\end{eqnarray}
							We distinguish two cases:
							\begin{itemize}
								\item	$\subj{\ell_1'} \not= n_1$. This case is similar with case 1 of this proof.
								\item	$\subj{\ell_1'} = n_1$.
										First observe that
										\[
											\horel{\Gamma}{\Delta_1}{\newsp{\widetilde{m_1}}{P \subst{n}{x} \Par \bout{t}{n_1} \inact}}
											{ \hby{\ell_{11}''} }
											{\Delta_1'}{\newsp{\widetilde{m_1}'}{P_1 \subst{n}{x} \Par \bout{t}{n_1} \inact}}
										\]
										with $\ell_{11}'' \subst{n_1}{n}  = \ell_{11}' $.
										which implies
										\[
											\horel{\Gamma}{\Delta_2}{\newsp{\widetilde{m_2}}{Q \subst{n}{x} \Par \bout{t}{n_2} \inact}}
											{ \Hby{\ell_{21}''} }
											{\Delta_2'}{\newsp{\widetilde{m_2}'}{Q_1 \subst{n}{x} \Par \bout{t}{n_2} \inact}}
										\]
										with $\ell_{21}'' \subst{n_2}{n}  = \ell_{21}' $.
										which in turn implies
										\begin{eqnarray}
											\newsp{\widetilde{m_2}}{Q \subst{n_2}{x}} &\hby{\ell_{22}'}& \newsp{\widetilde{m_{21}}}{Q_1 \subst{n_2}{x}}
											\label{lem:tr_app_43}
										\end{eqnarray}
										Also observe that for $U = \Delta(n_1)$
										\[
											\mhorel{\Gamma}{\Delta_1}{\newsp{\widetilde{m_1}}{P \subst{n}{x} \Par \bout{t}{n_1} \inact}}
											{ \hby{\bactout{t}{n_1}} \hby{\bactinp{t}{\omapchar{U}}} \Hby{\dtau} }
											{\Delta_1''}{}{\newsp{\widetilde{m_1}''}{P \subst{n}{x} \Par \omapchar{U} \subst{n_1}{x}}}
										\]
										which implies
										\[
											\mhorel{\Gamma}{\Delta_2}{\newsp{\widetilde{m_2}}{Q \subst{n}{x} \Par \bout{t}{n_2} \inact}}
											{ \Hby{\bactout{t}{n_2}} \Hby{\bactinp{t}{\omapchar{U}}} \Hby{\dtau} }
											{\Delta_2''}{}{\newsp{\widetilde{m_2}''}{Q' \subst{n}{x} \Par \omapchar{U} \subst{n_2}{x}}}
										\]
										with
										\[
											\mhorel{\Gamma}{\Delta_1''}{\newsp{\widetilde{m_1}''}{P \subst{n}{x} \Par \omapchar{U} \subst{n_1}{x}}}
											{\hwb}
											{\Delta_2''}{}{\newsp{\widetilde{m_2}''}{Q' \subst{n}{x} \Par \omapchar{U} \subst{n_2}{x}}}
										\]
										From \eqref{lem:tr_app_42} we can see that
										\[
											\mhorel{\Gamma}{\Delta_1''}{\newsp{\widetilde{m_1}''}{P \subst{n}{x} \Par \omapchar{U} \subst{n_1}{x}}}
											{ \hby{\tau} }
											{\Delta_1'''}{}{\newsp{\widetilde{m_1}''}{P_2 \subst{n}{x} \Par \bout{t'}{n_1} \inact}}
										\]
										which implies from part 2 of this lemma
										\begin{eqnarray}
											\mhorel{\Gamma}{\Delta_2''}{\newsp{\widetilde{m_2}''}{Q' \subst{n}{x} \Par \omapchar{U} \subst{n_2}{x}}}
											{\Hby{\tau}}
											{\Delta_2'''}{}{\newsp{\widetilde{m_2}''}{Q_2 \subst{n}{x} \Par \btout{t'}{n_1} \inact}}
											\label{lem:tr_app_44}
										\end{eqnarray}
										and
										\begin{eqnarray}
											\horel{\Gamma}{\Delta_1'''}{\newsp{\widetilde{m_1}''}{P_2 \subst{n}{x} \Par \bout{t'}{n_1} \inact}}
											{\hwb}
											{\Delta_2'''}{\newsp{\widetilde{m_2}''}{Q_2 \subst{n}{x} \Par \btout{t'}{n_1} \inact}}
											\label{lem:tr_app_45}
										\end{eqnarray}
										where \eqref{lem:tri_app_44} implies the untyped transition
										\[
											\newsp{\widetilde{m_2}''}{Q' \subst{n}{x} \Par \omapchar{U} \subst{n_2}{x}}
											\Hby{ \ell_{22}'' }
											\newsp{\widetilde{m_2}''}{Q_2 \subst{n}{x} \Par \omapchar{U} \subst{n_2}{x}}
										\]
										and furthermore,
										\[
											\newsp{\widetilde{m_2}''}{Q' \subst{n_2}{x}}
											\Hby{ \ell_{22}' }
											\newsp{\widetilde{m_2}''}{Q_2 \subst{n_2}{x}}
										\]
										with $\ell_{22}'' \subst{n_2}{n} = \ell_{22}'$.
										From the last result and \eqref{{lem:tr_app_43}} we get
										\[
											\mhorel{\Gamma}{\Delta_2}{\newsp{\widetilde{m_2}}{ Q \subst{n_2}{x}   }  }
											{\Hby{\tau}}
											{\Delta_2'}{}{\newsp{\widetilde{m_2}''}{ Q'' \subst{n_2}{x}   }  }
										\]
										Furthermore, from \eqref{lem:tr_app_45} we can get that
										\[
											\horel{\Gamma}{\Delta_1'''}{\newsp{\widetilde{m_1}''}{P_2 \subst{n}{x} \Par \bout{t'}{n_1} \inact}}
											{\hby{\ell_{12}''}}
											{\Delta_3}{\newsp{\widetilde{m_1}'''}{P' \subst{n}{x} \Par \bout{t'}{n_1} \inact}}
										\]
										which implies
										\[
											\horel{\Gamma}{\Delta_2'''}{\newsp{\widetilde{m_2}''}{Q_2 \subst{n}{x} \Par \btout{t'}{n_2} \inact}}
											{\hby{\ell_{22}''}}
											\horel{\Delta_4}{\newsp{\widetilde{m_2}'''}{Q'' \subst{n}{x} \Par \bout{t'}{n_2} \inact}}
										\]
										and
										\begin{eqnarray*}
											\horel{\Gamma}{\Delta_3}{\newsp{\widetilde{m_1}'''}{P' \subst{n}{x} \Par \bout{t'}{n_1} \inact}}
											{\hwb}
											{\Delta_4}{\newsp{\widetilde{m_2}'''}{Q'' \subst{n}{x} \Par \btout{t'}{n_1} \inact}}
										\end{eqnarray*}
										which in turn implies
										\begin{eqnarray*}
											\horel{\Gamma}{\Delta_1'}{\newsp{\widetilde{m_1}'''}{P' \subst{n_1}{x}}}
											{\ \Re\ }
											{\Delta_2'}{\newsp{\widetilde{m_2}'''}{Q'' \subst{n_2}{x}}}
										\end{eqnarray*}
										as required.
							\end{itemize}
					\end{itemize}
	\end{enumerate}
	\qed
\end{proof}

A process substitution lemma is useful for showing the
contextual for higher-order and characteristic bisimilarities.
Before we state and prove a process substitution lemma
we give an intermediate result.

\begin{lemma}[Trigger Substitution]
	\label{lem:trigger_subst}
	Let $P$ and $Q$ be processes. Also, let $t$ be a fresh name. If
	\[
		\horel{\Gamma}{\Delta_1}{\newsp{\widetilde{m_1}}{P \Par \prod_{i \in I} \appl{(\trvalx{t_i})}{n_i} }}
		{\hwb}
		{\Delta_2}{\newsp{\widetilde{m_2}}{Q \Par\prod_{i \in I} \appl{(\trvalx{t_i})}{m_i} }}
	\]
	then $\forall \abs{\widetilde{x}}{R}, \exists \Delta_1', \Delta_2'$ such that
	\[
		\horel{\Gamma}{\Delta_1'}{\newsp{\widetilde{m_1}}{P \Par \appl{(\abs{\widetilde{x}}{R})}{\widetilde{n}} }}
		{\hwb}
		{\Delta_2'}{\newsp{\widetilde{m_2}}{Q \Par \appl{(\abs{\widetilde{x}}{R})}{\widetilde{m}} }}
	\]
\end{lemma}

\begin{proof}
	We proof the result up-to the deterministic application
	transition that substitutes names $n_i$ and $m_i$ to
	process $R$, respectively.
	Let relation
	\begin{eqnarray*}
		\Re	&=&	\set{	(\horel{\Gamma}{\Delta_1'}{\newsp{\widetilde{m_1}}{P \Par R \subst{\widetilde{n}}{\widetilde{x}} }}
						{\ ,\ }
						{\Delta_2'}{\newsp{\widetilde{m_2}}{Q \Par R \subst{\widetilde{m}}{\widetilde{x}} }})
					\setbar\\
					&& \forall \abs{\widetilde{x}}{R}, \exists \Delta_1', \Delta_2', \land \\
					&&	\horel{\Gamma}{\Delta_1}{\newsp{\widetilde{m_1}}{P \Par \prod_{i \in I} \appl{(\trvalx{t_i})}{n_i} }}
						{\hwb}
						{\Delta_2}{\newsp{\widetilde{m_2}}{Q \Par \prod_{i \in I} \appl{(\trvalx{t_i})}{m_i}}}\\
				&&}
	\end{eqnarray*}
	We show that $\Re$ is a higher-order bisimilarity.
	The proof is done by a case analysis on the actions that can be observed
	on the pairs of processes, so to check their higher-order bisimulation requirements.

	\begin{enumerate}
		\item	\[
					\horel{\Gamma}{\Delta_1'}{\newsp{\widetilde{m_1}}{P \Par R \subst{\widetilde{n}}{\widetilde{x}} }}
					{\hby{\ell_1}}
					{\Delta_1''}{\newsp{\widetilde{m_1}'}{P' \Par R \subst{\widetilde{n}}{\widetilde{x}} }}
				\]
				implies
				\[
					\horel{\Gamma}{\Delta_3}{\news{\widetilde{m_1}}{P}}
					{\hby{\ell_1}}
					{\Delta_3'}{\news{\widetilde{m_1}'}{P'}}
				\]
				implies
				\[
					\horel{\Gamma}{\Delta_1}{\newsp{\widetilde{m_1}}{P \Par \prod_{i \in I} \appl{(\trvalx{t_i})}{n_i} }}
					{\hby{\ell_1}}
					{\Delta_5}{\newsp{\widetilde{m_1}'}{P' \Par \prod_{i \in I} \appl{(\trvalx{t_i})}{n_i} }}
				\]
				implies from the higher-order bisimilarity requirement of relation $\Re$ and the
				freshness of $t_i, \forall i \in I$
				\[
					\horel{\Gamma}{\Delta_2}{\newsp{\widetilde{m_2}}{Q \Par \prod_{i \in I} \appl{(\trvalx{t_i})}{m_i} }}
					{\Hby{\ell_2}}
					{\Delta_6}{\newsp{\widetilde{m_2}'}{Q' \Par \prod_{i \in I} \appl{(\trvalx{t_i})}{n_i} }}
				\]
				and
				\begin{eqnarray}
					\mhorel{\Gamma}{\Delta_5}{\newsp{\widetilde{m_1}'}{P' \Par \prod_{i \in I} \appl{(\trvalx{t_i})}{n_i} \Par C_1}}
					{\hwb}
					{\Delta_6}{}{\newsp{\widetilde{m_2}'}{Q' \Par \prod_{i \in I} \appl{(\trvalx{t_i})}{m_i} \Par C_2}}
					\label{lem:trig_subst1}
				\end{eqnarray}
				where $C_1, C_2$ are the trigger processes if $\ell_1, \ell_2$ are output actions
				and $C_1 = C_2 = \inact$ otherwise.

				The former transition implies
				\[
					\horel{\Gamma}{\Delta_4}{\newsp{\widetilde{m_2}}{Q }}
					{\Hby{\ell_2}}
					{\Delta_4'}{\news{\widetilde{m_2}'}{Q'}}
				\]
				which implies
				\[
					\horel{\Gamma}{\Delta_2'}{\newsp{\widetilde{m_2}}{Q \Par R \subst{\widetilde{m}}{\widetilde{x}}}}
					{\Hby{\ell_2}}
					{\Delta_2''}{\newsp{\widetilde{m_2}'}{Q' \Par \subst{\widetilde{m}}{\widetilde{x}} }}
				\]
				Equation \eqref{lem:trig_subst1} and the definition of $\Re$ implies
				\begin{eqnarray*}
					\horel{\Gamma}{\Delta_1''}{\newsp{\widetilde{m_1}'}{P' \Par R \subst{\widetilde{n}}{\widetilde{x}} \Par C_1}}
					{\hwb}
					{\Delta_2''}{\newsp{\widetilde{m_2}'}{Q' \Par R \subst{\widetilde{m}}{\widetilde{x}} \Par C_2 }}
				\end{eqnarray*}
				that concludes the case.

		\item	\[
					\horel{\Gamma}{\Delta_1'}{\newsp{\widetilde{m_1}}{P \Par R \subst{\widetilde{n}}{\widetilde{x}}  }}
					{\hby{\ell}}
					{\Delta_1''}{\newsp{\widetilde{m_1}'}{P \Par R' \subst{\widetilde{n}}{\widetilde{x}} }}
				\]
				We identify three sub-cases:
				\begin{enumerate}[i.]
					\item	$\subj{\ell} \not= n_i$. The case is similar as above.

					\item	$\subj{\ell} = n_k$ and $n_k = m_k$.
							From the definition of $\Re$ we get that
							\[
								\mhorel{\Gamma}{\Delta_1}{\newsp{\widetilde{m_1}}{P \Par \prod_{i \in I} \binp{t_i}{x} \appl{x}{n_i}}}
								{\hby{ \bactinp{t_k}{\omapchar{U}}}}
								{\Delta_3}{}{\newsp{\widetilde{m_1}}{P \Par \prod_{i \in I\backslash{\set{k}}} \binp{t_i}{x} \appl{x}{n_i} \Par \appl{\omapchar{U}}{n_k}}}
							\]
							implies
							\[
								\mhorel{\Gamma}{\Delta_2}{\newsp{\widetilde{m_2}}{Q \Par \prod_{i \in I} \binp{t_i}{x} \appl{x}{m_i}}}
								{\Hby{ \bactinp{t_k}{\omapchar{U}}}}
								{\Delta_4}{}{\newsp{\widetilde{m_1}}{Q' \Par \prod_{i \in I \backslash{\set{k}}} \binp{t_i}{x} \appl{x}{m_i} \Par \mapchar{U}{x} \subst{m_k}{x} }}
							\]
							and
							\begin{eqnarray*}
								&& \Gamma; \Delta_3 \proves \newsp{\widetilde{m_1}}{P \Par \prod_{i \in I \backslash{\set{k}}} \binp{t_i}{x} \appl{x}{n_i} \Par \appl{\omapchar{U}}{n_k}}
								\\
								&\hby{\tau_{\beta}}&
									\Delta_3 \proves \newsp{\widetilde{m_1}}{P \Par \prod_{i \in I \backslash{\set{k}}} \binp{t_i}{x} \appl{x}{n_i} \Par \mapchar{U}{x} \subst{n_k}{x}  }
								\\
								&\hwb&
									\Delta_4 \proves \newsp{\widetilde{m_1}}{Q' \Par \prod_{i \in I \backslash{\set{k}}} \binp{t_i}{x} \appl{x}{m_i} \Par \mapchar{U}{x} \subst{m_k}{x}}
							\end{eqnarray*}

							The proof follows Part 2 of \lemref{lem:trigger_application} where
							\[
								\mhorel{\Gamma}{\Delta_3}{\newsp{\widetilde{m_1}}{P \Par \prod_{i \in I \backslash{\set{k}}} \binp{t_i}{x} \appl{x}{n_i} \Par \mapchar{U}{x} \subst{n_k}{x}  }}
								{\hby{\ell} }
								{\Delta_3'}{}{\newsp{\widetilde{m_1}'}{P \Par \prod_{i \in I \backslash{\set{k}}} \binp{t_i}{x} \appl{x}{n_i} \Par \bout{t'}{n_k} \inact }}
							\]
							implies
							\[
								\mhorel{\Gamma}{\Delta_4}{\newsp{\widetilde{m_2}}{Q' \Par \prod_{i \in I \backslash{\set{k}}} \binp{t_i}{x} \appl{x}{m_i} \Par \mapchar{U}{x} \subst{m_k}{x}  }}
								{\Hby{\ell} \dk{\dots}}
								{\Delta_4'}{}{\newsp{\widetilde{m_2}'}{Q'' \Par \prod_{i \in I \backslash{\set{k}}} \binp{t_i}{x} \appl{x}{m_i} \Par \bout{t'}{m_k} \inact }}
							\]
							and furthermore, from Part 3 of \lemref{lem:trigger_application}
							\[
								\mhorel{\Gamma}{\Delta_3'}{\newsp{\widetilde{m_1}'}{P \Par \prod_{i \in I \backslash{\set{k}}} \binp{t_i}{x} \appl{x}{n_i} \Par \binp{t'}{y}{\appl{y}{n_k}}}}
								{\hwb}
								{\Delta_4'}{}{\newsp{\widetilde{m_2}'}{Q'' \Par \prod_{i \in I \backslash{\set{k}}} \binp{t_i}{x} \appl{x}{m_i} \Par \binp{t'}{y}{\appl{y}{m_k}}}}
							\]
							that implies from the definition of $\Re$ that $\forall R$ such that $\widetilde{x} \in \fn{R}$
							\[
								\horel{\Gamma}{\Delta_3'}{\newsp{\widetilde{m_1}'}{P \Par R \subst{\widetilde{n}}{\widetilde{x}} }}
								{\ \Re\ }
								{\Delta_4'}{\newsp{\widetilde{m_2}'}{Q'' \Par R \subst{\widetilde{m}}{\widetilde{x}}}}
							\]
							The case concludes when we verify that 
							\[
								\horel{\Gamma}{\Delta_2'}{\newsp{\widetilde{m_2}}{Q \Par R \subst{n_2}{x} }}
								{\Hby{\ell}}
								{\Delta_2''}{\newsp{\widetilde{m_1}'}{Q'' \Par R' \subst{n_2}{x} }}
							\]
							

					\item	$\subj{\ell} = n_k$ and $n_k \not= m_k$. This case
							is not possible. \lemref{lem:trigger_application} implies
							that $n_k$ is a session and $\dual{n_k} \in \fn{P}$. From the
							definition of typed transition (\defref{d:tlts}) we get that we cannot observe $\ell$
							on $R \subst{\widetilde{n}}{\widetilde{x}}$, because $\dual{n_k} \in \fn{P}$ and $(\Gamma; \es; \Delta) \not\hby{\ell}$.
				\end{enumerate}

		\item	\[
					\horel{\Gamma}{\Delta_1'}{\newsp{\widetilde{m_1}}{P \Par R \subst{\widetilde{n}}{\widetilde{x}} }}
					{\hby{}}
					{\Delta_1''}{\newsp{\widetilde{m_1}'}{P' \Par R' \subst{\widetilde{n}}{\widetilde{x}} }}
				\]
				From the typed reduction definition (\defref{d:tlts}) we get that
				\begin{eqnarray}
					&&	\horel{\Gamma}{\Delta_3}{\news{\widetilde{m_1}}{P}}
						{\hby{\ell_1}}
						{\Delta_R}{\news{\widetilde{m_1}}{P}}
					\\
					\label{lem:trigger_subst_31}
					&&	\horel{\Gamma}{\Delta_1'}{R \subst{\widetilde{n}}{\widetilde{x}} }
						{\hby{\ell_2}}
						{\Delta_R'}{R' \subst{\widetilde{n}}{\widetilde{x}} }
					\label{lem:trigger_subst_32}
					\\
					&&	\ell_1 \comp \ell_2 \nonumber
				\end{eqnarray}

				We distinguish several subcases.
				\begin{enumerate}[i.]
	
					\item	$\ell_1 = \bactinp{n_k}{n}$ and $\ell_2 = \bactout{n_k}{n}$.
							\dk{put proof}

					\item	An important sub-case is the case where
							$\ell_1 = \bactinp{n}{n_k}$ and $\ell_2 = \bactout{n}{n_k}$.
						%
							From the definition of $\Re$ we have that
							\[
								\horel{\Gamma}{\Delta_1}{ \newsp{\widetilde{m_1}}{P \Par \prod_{i \in I} \binp{t_1}{x} \appl{x}{n_i}}}
								{ \hby{ \bactinp{n}{m} } }
								{\Delta_3}{  \newsp{\widetilde{m_1}}{P' \subst{m}{x} \Par \prod_{i \in I} \binp{t_1}{x} \appl{x}{n_i} }   }
							\]
						%
							implies 
							\begin{eqnarray}
								\horel{\Gamma}{\Delta_2}{ \newsp{\widetilde{m_2}}{Q \Par \prod_{i \in I} \binp{t_1}{x} \appl{x}{n_i}}}
								{ \Hby{ \bactinp{n}{m} } }
								{\Delta_4}{  \newsp{\widetilde{m_2}}{Q' \subst{m}{x} \Par \prod_{i \in I} \binp{t_1}{x} \appl{x}{n_i} }   }
								\label{ lem:trigger_subst_311 }
							\end{eqnarray}
							and
							\[
								\horel{\Gamma}{\Delta_3}{  \newsp{\widetilde{m_1}}{P' \subst{m}{x} \prod_{i \in I} \binp{t_1}{x} \appl{x}{n_i} }   }
								{\hwb}
								{\Delta_4}{  \newsp{\widetilde{m_2}}{Q' \subst{m}{x} \Par \prod_{i \in I} \binp{t_1}{x} \appl{x}{n_i} }   }
							\]
							implies from Part 4 of \lemref{lem:trigger_application} that
							\[
								\mhorel{\Gamma}{\Delta_3'}{  \newsp{\widetilde{m_1}}{P' \subst{n_k}{x} \Par \prod_{i \in I\backslash\set{k}} \binp{t_1}{x} \appl{x}{n_i} }   }
								{\hwb}
								{\Delta_4'}{}{  \newsp{\widetilde{m_2}}{Q' \subst{m_k}{x} \Par \prod_{i \in I\backslash\set{k}} \binp{t_1}{x} \appl{x}{n_i} }   }
							\]
							implies from the definition of $\Re$ that
							\[
								\mhorel{\Gamma}{\Delta_1'}{\newsp{\widetilde{m_1}}{P' \subst{n_k}{x} \Par R \subst{\widetilde{m}}{\widetilde{x}}}}
								{\hwb}
								{\Delta_2'}{}{  \newsp{\widetilde{m_2}}{Q' \subst{m_k}{x} \Par R' \subst{\widetilde{m}}{\widetilde{x}} } }
							\]
						%
							From \eqref{ lem:trigger_subst_311 } and \eqref{lem:trigger_subst_32} we get that
							\begin{eqnarray*}
								\horel{\Gamma}{\Delta_2'}{ \newsp{\widetilde{m_2}}{Q \Par R \subst{\widetilde{m}}{\widetilde{x}} }}
								{ \Hby{  } }
								{\Delta_2''}{  \newsp{\widetilde{m_2}}{Q' \subst{m_k}{x} \Par R' \subst{\widetilde{m}}{\widetilde{x}} }   }
							\end{eqnarray*}
						%
							that concludes the case.

					\item	The most demanding sub-case is the case
							where
							$\ell_1 = \bactinp{n_k}{n_l}$ and $\ell_2 = \bactout{n_k}{n_l}$.
							The proof is a consequence of the previews two sub-cases.	


%					\dk{finish proof}							


					\item	The rest of the sub-cases are similar (or easier) to the above cases.

%					\item	$\subj{\ell_1} \not= n_1$. \dk{put proof} %The case is similar with the previous cases.
%					\item	$\subj{\ell_1} = n_1$.  \dk{put proof} %The case is similar with sub-case (ii) of case 2.
%					\item	$\obj{\ell_1} = n_1$. \dk{put proof}
%					
%					\item	 $\ell_1 = \bactinp{n_k}{n_l}$ and $\ell_2 = \bactout{n_k}{n_l}$.
				\end{enumerate}
	\end{enumerate}
	\qed
\end{proof}


%%%%%%%%%%%%%%%%%%%%%%%%%%%%%%%%%%%%%%%%%%%%%%%%%%%%%%%%%%%%
% PROCESS SUBSTITUTION - Second Lemma
%%%%%%%%%%%%%%%%%%%%%%%%%%%%%%%%%%%%%%%%%%%%%%%%%%%%%%%%%%%%

%\jasks{I can't parse this sentence:}

We can now state a process substitution lemma.
Given a higher-order bisimulation under a trigger value
substitution, we can generalise for any value substitution.

%We can now state a process substitution lemma, where given
%a higher order equivalence under a trigger value substitution,
%we can generalise for any process substitution that makes
%the equivalence typable.

\begin{lemma}[Process Substitution]
	\label{lem:process_subst}
	Let $P_1$ and $P_2$ be processes, with $z \in \fn{P_1}, z \in \fn{P_2}$.
	Also, let $t$ be a fresh name. 
%	\jasks{Don't we need to say that $x$ occurs free in in $P_i$? Also, we have $x$ both in $P$ and in the parameter of the abstraction,
%	it is better to have different variables, just to avoid confusion.}
	If
	\[
		\horel{\Gamma}{\Delta_1}{\newsp{\widetilde{m_1}}{P_1 \subst{\trvalx{t}}{z} }}
		{\hwb}
		{\Delta_2}{\newsp{\widetilde{m_2}}{P_2 \subst{\trvalx{t}}{z} }}
	\]
	then $\forall \abs{x}{R}, \exists \Delta_1', \Delta_2'$ such that
	\[
		\horel{\Gamma}{\Delta_1'}{\newsp{\widetilde{m_1}}{P_1 \subst{{\abs{x}{R}}}{z} }}
		{\hwb}
		{\Delta_2'}{\newsp{\widetilde{m_2}}{P_2 \subst{{\abs{x}{R}}}{z} }}
	\]
\end{lemma}


\begin{proof}
	Consider the typed relation (for readability, we omit type information):
	\begin{eqnarray*}
		\Re	&=&	\set{
					(\newsp{\widetilde{m_1}}{P_1 \subst{{\abs{x}{R}}}{z}}, \newsp{\widetilde{m_2}}{P_1 \subst{{\abs{x}{R}}}{z}})
					\setbar\\
			&&		\qquad \horel{\Gamma}{\Delta_1}{\newsp{\widetilde{m_1}}{P_1 \subst{\trvalx{t}}{z} }}
					{\hwb}
					{\Delta_2}{\newsp{\widetilde{m_2}}{P_2 \subst{\trvalx{t}}{z} }}
			\\
			&&		}
	\end{eqnarray*}
	We show that $\Re$ is a higher-order bisimilarity. Suppose that 
	%
	\begin{eqnarray}
		\horel{\Gamma}{\Delta_1'}{\newsp{\widetilde{m_1}}{P_1 \subst{{\abs{x}{R}}}{z} }}
		{\hby{\ell_1}}
		{\Delta_3}{\newsp{\widetilde{m_1}}{P_1' \subst{{\abs{x}{R}}}{z} }}
		\label{lem:proc_subst1}
	\end{eqnarray}

	Our analysis distinguishes two cases, depending on whether the substitution $\subst{{\abs{x}{R}}}{z}$ has an effect on the action denoted by $\ell_1$:
	\begin{enumerate}
		\item	Case $P_1 \not\scong Q \Par \appl{x}{n}$ that is, the substitution does not affect top-level processes. 

				In other words we can imply from the freshness of $t$ that $\subj{\ell_1} \not= t$.
	%			\jasks{(This sentence on freshness seems misplaced - it makes sense only below.)}
				Furthermore, from the requirements of $\Re$
				we get that
				\begin{eqnarray*}
					\horel{\Gamma}{\Delta_1}{\newsp{\widetilde{m_1}}{P_1 \subst{\trvalx{t}}{z} }}
					{\hby{\ell_1}}
					{\Delta_1''}{\newsp{\widetilde{m_1}}{P_1' \subst{\trvalx{t}}{z} }}
				\end{eqnarray*}
				which implies that there exists $\ell_2, \Delta_2'', P_2'$ such that
				\begin{eqnarray}
					\horel{\Gamma}{\Delta_2}{\newsp{\widetilde{m_2}'}{P_2 \subst{\trvalx{t}}{z} }}
					{\Hby{\ell_2}}
					{\Delta_2''}{\newsp{\widetilde{m_2}'}{P_2' \subst{\trvalx{t}}{z} }}
					\label{lem:proc_subst0}
				\end{eqnarray}
				and
				\begin{eqnarray*}
					\horel{\Gamma}{\Delta_1}{\newsp{\widetilde{m_1}''}{P_1' \subst{\trvalx{t}}{z} \Par C_1}}
					{\hwb}
					{\Delta_2}{\newsp{\widetilde{m_2}''}{P_2' \subst{\trvalx{t}}{z} \Par C_2}}
				\end{eqnarray*}
				with $C_1$ (resp., $C_2$) being the higher order trigger process
				in the cases where $\ell_1 = \news{\widetilde{m}} \bactout{n}{V_1}$ (resp., $\ell_2 = \news{\widetilde{m}'} \bactout{n}{V_2}$)
				and $C_1 = C_2 = \inact$ otherwise.
				Because $C_1$ and $C_2$ are closed terms we can rewrite the substitution as:
				\begin{eqnarray*}
					\horel{\Gamma}{\Delta_1}{\newsp{\widetilde{m_1}''}{(P_1'\Par C_1) \subst{\trvalx{t}}{z}}}
					{\hwb}
					{\Delta_2}{\newsp{\widetilde{m_2}''}{(P_2'\Par C_2) \subst{\trvalx{t}}{z}}}
				\end{eqnarray*}
				Since $\ell_1, \ell_2$ do not pertain to the substitution,
				we can consider any $\abs{x}{R}$ instead of $\trvalx{t}$.
				Thus we further imply that
				\begin{eqnarray}
					\horel{\Gamma}{\Delta_3'}{\newsp{\widetilde{m_1}''}{(P_1'\Par C_1) \subst{{\abs{x}{R}}}{z}}}
					{\ \Re\ }
					{\Delta_4'}{\newsp{\widetilde{m_2}''}{(P_2'\Par C_2) \subst{{\abs{x}{R}}}{z}}}
					\label{lem:proc_subst33}
				\end{eqnarray}

				From \eqref{lem:proc_subst2} we can derive the transition
				\begin{eqnarray*}
					\horel{\Gamma}{\Delta_2'}{\newsp{\widetilde{m_2}}{P_2 \subst{{\abs{x}{R}}}{z} }}
					{\Hby{\ell_2}}
					{\Delta_4}{\newsp{\widetilde{m_2}'}{P_2' \subst{{\abs{x}{R}}}{z} }}
				\end{eqnarray*}
				Equation \eqref{lem:proc_subst33} concludes the case.
%				\jasks{There is a problem with equation labels above!}


		\item	Case $P_1 \scong P \Par \prod_{i \in I} \appl{x}{n_i} \Par \appl{x}{n_1}$ and
				$P \not= P' \Par \appl{x}{n'}$. This is the case where $P$ does not include the form
				of an application on name $x$.

				We identify two sub-cases, depending on the source of the action $\ell_1$:
				\begin{itemize}
					\item	Sub-case
							\begin{eqnarray*}
								\mhorel{\Gamma}{\Delta_1'}{\newsp{\widetilde{m_1}}{(P \Par \prod_{i \in I} \appl{x}{n_i} \Par \appl{x}{n_1}) \subst{{\abs{x}{R}}}{z} }}
								{\hby{\ell_1}}
								{\Delta_3}{}{\newsp{\widetilde{m_1}}{(P' \Par \prod_{i \in I} \appl{x}{n_i} \Par \appl{x}{n_1}) \subst{{\abs{x}{R}}}{z} }}
							\end{eqnarray*}
							%
							This sub-case is similar as the previous case.

					\item	Sub-case (Assume $Q = P \Par \prod_{i \in I} \appl{x}{n_i}$)
							\begin{eqnarray}
								\horel{\Gamma}{\Delta_1'}{\newsp{\widetilde{m_1}}{(Q \Par \appl{x}{n_1}) \subst{{\abs{x}{R}}}{z} }}
								{\hby{\tau}}
								{\Delta_3}{\newsp{\widetilde{m_1}}{Q \subst{{\abs{x}{R}}}{z} \Par R \subst{n_1}{z}  }}
								\label{lem:proc_subst1}
							\end{eqnarray}
							Which is the application of name $n_1$ on abstraction $\abs{x}{R}$.
							%
							From the requirements of $\Re$ we imply that
							\begin{eqnarray*}
								\mhorel{\Gamma}{\Delta_1}{\newsp{\widetilde{m_1}}{(Q \Par \appl{x}{n_1}) \subst{\trvalx{t}}{z} }}
								{\hby{\tau}}
								{\Delta_1''}{}{\newsp{\widetilde{m_1}}{Q \subst{\trvalx{t}}{z} \Par \binp{t}{y} (\appl{y}{n_1})}}
							\end{eqnarray*}
							which implies that there $\exists P_2', \Delta_2''$ such that
							%
								\nhorel{\Gamma}{\Delta_2}{\newsp{\widetilde{m_2}}{P_2 \subst{\trvalx{t}}{z} }}
								{\Hby{}}
								{\Delta_2''}{\newsp{\widetilde{m_2}}{P_2' \subst{\trvalx{t}}{z}}}
								{lem:proc_subst2}
							%
							and
							\begin{eqnarray*}
								\mhorel{\Gamma}{\Delta_1''}{\newsp{\widetilde{m_1}}{Q \subst{\trvalx{t}}{z} \Par \binp{t}{y} (\appl{y}{n_1})}}
								{\hwb}
								{\Delta_2''}{}{\newsp{\widetilde{m_2}}{P_2' \subst{\trvalx{t}}{z}}}
							\end{eqnarray*}
							%
							From the last pair we can see that for fresh $t'$ if
							\begin{eqnarray*}
								\mhorel{\Gamma}{\Delta_1''}{\newsp{\widetilde{m_1}}{Q \subst{\trvalx{t}}{z} \Par \binp{t}{y} (\appl{y}{n_1})}}
								{\hby{\bactinp{t}{\trvalx{t'}}}}
								{\Delta_1'''}{}{\newsp{\widetilde{m_2}}{Q \subst{\trvalx{t}}{z} \Par \appl{\trvalx{t'}}{n_1}}}
							\end{eqnarray*}
							that implies that $\exists P_2'', \Delta_2'''$ such that
							\begin{eqnarray}
								\begin{array}{crll}
											& \Gamma; \es; \Delta_2'' &\proves& \newsp{\widetilde{m_2}}{P_2'\subst{\trvalx{t}}{z}}
									\\
									\Hby{}	&	&&	\newsp{m_2}{(P_3 \Par \appl{x}{n_2}) \subst{\trvalx{t}}{z}}
									\\
									{\hby{\bactinp{t}{\trvalx{t'}}}} &
												&&	\newsp{m_2}{P_3 \subst{\trvalx{t}}{z} \Par \appl{\trvalx{t'}}{n_2}}
									\\
									\Hby{}	& \Delta_2''' & \proves & \newsp{\widetilde{m_2}}{P_2'' \subst{\trvalx{t}}{z} \Par \appl{\trvalx{t'}}{n_2}}
								\end{array}
								\label{lem:proc_subst3}
							\end{eqnarray}
							and
							\begin{eqnarray*}
								\mhorel{\Gamma}{\Delta_1'''}{\newsp{\widetilde{m_1}}{Q \subst{\trvalx{t}}{z} \Par \appl{\trvalx{t'}}{n_1}}}
								{\hwb}
								{\Delta_2'''}{}{\newsp{\widetilde{m_2}}{P_2'' \subst{\trvalx{t}}{z} \Par \appl{\trvalx{t'}}{n_2}}}
							\end{eqnarray*}
							%
							From \lemref{lem:trigger_subst} we can deduce that $\forall \abs{x}{R}$, if $\exists \Delta_5, \Delta_6$ then
							\begin{eqnarray*}
								\mhorel{\Gamma}{\Delta_5}{\newsp{\widetilde{m_1}}{Q \subst{\trvalx{t}}{z} \Par \appl{(\abs{x}{R})}{n_1}}}
								{\hwb}
								{\Delta_6}{}{\newsp{\widetilde{m_2}}{P_2'' \subst{\trvalx{t}}{z} \Par \appl{(\abs{x}{R})}{n_2}}}
							\end{eqnarray*}
							from the definition of $\Re$ we have that for all $\forall \abs{x}{R}$, if $\exists \Delta_3, \Delta_4$
							%
								\nhorel{\Gamma}{\Delta_3}{\newsp{\widetilde{m_1}}{Q \subst{(\abs{x}{R})}{z} \Par \appl{(\abs{x}{R})}{n_1}}}
								{\ \Re\ }
								{\Delta_4}{\newsp{\widetilde{m_2}}{P_2'' \subst{(\abs{x}{R})}{z} \Par \appl{(\abs{x}{R})}{n_2}}}
								{lem:proc_subst4}
							%	
							We show that we can mimic first the
							transition in \eqref{lem:proc_subst2} and then the silent part of
							transitions \eqref{lem:proc_subst3} to get:
							\begin{eqnarray}
								\begin{array}{crll}
											& \Gamma; \es; \Delta_2' &\proves& \newsp{\widetilde{m_2}}{P_2 \subst{(\abs{x}{R})}}
									\\
									\Hby{}	&	\Delta_2'			& \proves&	\newsp{\widetilde{m_2}}{P_2' \subst{\trvalx{t}}{z}}
									\\
									\Hby{} &	\Delta_4			& \proves&	\newsp{m_2}{P_2'' \subst{(\abs{x}{R})}{z} \Par \appl{(\abs{x}{R})}{n_2}}
								\end{array}
								\label{lem:proc_subst5}
							\end{eqnarray}
							We showed that if \eqref{lem:proc_subst0} then \eqref{lem:proc_subst5} and \eqref{lem:proc_subst4}
							as required to show that $\Re$ is a higher-order bisimilarity.
							\qed
				\end{itemize}
	\end{enumerate}
\end{proof}



%%%%%%%%%%%%%%%%%%%%%%%%%%%%%%%%%%%%%%%%%%%%%%%%%%%%%%%%%
%  WB IS WBC
%%%%%%%%%%%%%%%%%%%%%%%%%%%%%%%%%%%%%%%%%%%%%%%%%%%%%%%%%

%\jp{Below we should be consistent and describe an $\Re$ that we use as closure.}
\begin{lemma}
	\label{app:lem:wb_is_wbc}
	$\hwb\ \subseteq\ \wbc$.
\end{lemma}

\begin{proof}
	Let $\Re$ be the typed relation (for readability, we omit typing judgements from the definition):
	\[
		\Re = \set{(P_1, Q_1) \setbar \horel{\Gamma}{\Delta_1}{P_1}{\hwb}{\Delta_2}{Q_1}}
	\]
	We show that $\Re$ is a context bisimulation.
	The proof is divided on cases on the label $\ell$ for the transition:
%
	\begin{eqnarray}
		\horel{\Gamma}{\Delta_1}{P_1}{\hby{\ell}}{\Delta_1'}{P_2}
		\label{lem:wb_is_wbc1}
	\end{eqnarray}
%
We distinguish four cases: $\ell$ is not an output or an input action; $\ell$ is an input action;
$\ell$ is an higher-order output; $\ell$ is a first-order output.
	\begin{enumerate}
		\item
				Case $\ell \notin \set{ \news{\widetilde{m_1}} \bactout{n}{\abs{\widetilde{x}}{P}},  \news{\widetilde{m_1}'} \bactout{n}{\widetilde{m_1}}, \bactinp{n}{\abs{\widetilde{x}}{P}} }$:

				\noi For the latter $\ell$ and transition in \eqref{lem:wb_is_wbc1} we impy that:	
			%
				\[
					\horel{\Gamma}{\Delta_2}{Q_1}{\Hby{\ell}}{\Delta_2'}{Q_2}
				\]
			%
				\noi and
			%
				\[
					\horel{\Gamma}{\Delta_1'}{P_2}{\hwb}{\Delta_2'}{Q_2}
				\]
			%
				The above premise and conclusion coincides with defining cases for $\ell$ in $\wbc$.

		\item	Case $\ell = \bactinp{n}{\abs{\widetilde{x}}{P}}$:

				\noi Transition in \eqref{lem:wb_is_wbc1} implies:
			%
				\[
					\begin{array}{l}
%						\horel{\Gamma}{\Delta_1}{P_1}{\hby{\bactinp{n}{\abs{\widetilde{x}}{\mapchar{U}{\widetilde{x}}}}}}{\Delta_1'}{P_2 \subst{\abs{\widetilde{x}}{\mapchar{U}{\widetilde{x}}}}{x}}\\
						\horel{\Gamma}{\Delta_1}{P_1}{\hby{\bactinp{n}{\auxtr{t}}}}{\Delta_1''}{P_2 \subst{\auxtr{t}}{x}}
					\end{array}
				\]
			%
				\noi The last transition implies:
			%
			\[
				\begin{array}{l}
%					\horel{\Gamma}{\Delta_2}{Q_1}{\Hby{\bactinp{n}{\abs{\widetilde{x}}{\mapchar{U}{\widetilde{x}}}}}}{\Delta_2'}{Q_2 \subst{\abs{\widetilde{x}}{\mapchar{U}{\widetilde{x}}}}{x}}\\
					\horel{\Gamma}{\Delta_2}{Q_1}{\Hby{\bactinp{n}{\auxtr{t}}}}{\Delta_2''}{Q_2 \subst{\auxtr{t}}{x}}
				\end{array}
			\]
			%
				\noi and
			%
			\[
				\begin{array}{l}
%					\horel{\Gamma}{\Delta_1'}{P_2 \subst{\abs{\widetilde{x}}{\mapchar{U}{\widetilde{x}}}}{x}}{\hwb}{\Delta_2'}{Q_2 \subst{\abs{\widetilde{x}}{\mapchar{U}{\widetilde{x}}}}{x}}\\
					\horel{\Gamma}{\Delta_1''}{P_2 \subst{\auxtr{t}}{x}}{\hwb}{\Delta_2''}{Q_2 \subst{\auxtr{t}}{x}}
				\end{array}
			\]
			%
				\noi To conclude from (\lemref{lem:process_subst}) that
				$\forall P$ with $\fv{P} = \widetilde{x}$
			%
			\[
				\horel{\Gamma}{\Delta_1'}{P_2 \subst{\abs{\widetilde{x}}{P}}{x}}{\hwb}{\Delta_2'}{Q_2 \subst{\abs{\widetilde{x}}{P}}{x}}
			\]
			%
				\noi as required.

		\item	Case $\ell = \news{\widetilde{m_1}'} \bactout{n}{\widetilde{m_1}}$: 

				\noi From transition \eqref{lem:wb_is_wbc1} we conclude:
			%
			\[
				\horel{\Gamma}{\Delta_2}{Q_1}{\Hby{\news{\widetilde{m_2}'} \bactout{n}{m_2}}}{\Delta_2'}{Q_2}
			\]
			%
				\noi and for fresh $t$
			%
			\[
				\horel	{\Gamma}{\Delta_1'}{\newsp{\widetilde{m_1}'}{P_2 \Par \htrigger{t}{m_1}}}
				{\hwb}
				{\Delta_2'}{\newsp{\widetilde{m_2}'}{Q_2 \Par \htrigger{t}{m_2}}}
			\]
			%
				\noi From the  second case of this proof we can conclude that $\forall R$ with $\fpv{R} = \set{x}$:
			%
			\[
				\begin{array}{rl}
					\Gamma; \es; &\Delta_1' \proves \newsp{\widetilde{m_1}'}{P_2 \Par \htrigger{t}{m_1}} \\
					\by{\bactinp{t}{\abs{z}{\binp{z}{x} R}}}& \newsp{\widetilde{m_1}'}{P_2 \Par \newsp{s}{\binp{s}{x} R \Par \bout{\dual{s}}{m_1} \inact }}\\
					\by{\tau} \quad &\Delta_1'' \proves \newsp{\widetilde{m_1}'}{P_2 \Par  R \subst{m_1}{x}}
				\end{array}
			\]
			%
				\noi and
			%
			\[
				\begin{array}{rl}
					\Gamma; \es; &\Delta_2' \proves \newsp{\widetilde{m_2}'}{Q_2 \Par \htrigger{t}{m_2} } \\
					\by{\bactinp{t}{\abs{z}{\binp{z}{x} R}}} &\newsp{\widetilde{m_2}'}{Q_2 \Par \newsp{s}{\binp{s}{x} R \Par \bout{\dual{s}}{m_2} \inact}}\\
					\by{\tau} &\Delta_2'' \proves \newsp{\widetilde{m_2}'}{Q_2 \Par  R \subst{m_2}{x}}
				\end{array}
			\]
			%
				\noi and furthermore it is easy to see that $\forall R$ with $\fpv{R} = \set{x}$:
			%
			\[
				\horel{\Gamma}{\Delta_1''}{\newsp{\widetilde{m_1}'}{P_2 \Par  R \subst{m_1}{x}}}{\hwb}{\Delta_2}{\newsp{\widetilde{m_2}'}{Q_2 \Par R \subst{m_2}{x}} }
			\]
			%
				\noi as required by the definition of $\wbc$.



		\item	Case $\ell = \news{\widetilde{m_1}'} \bactout{n}{\abs{\widetilde{x}}{P}}$:

				\noi From transition \eqref{lem:wb_is_wbc1} we conclude:
			%
			\[
				\horel{\Gamma}{\Delta_2}{Q_1}{\Hby{\news{\widetilde{m_2}'} \bactout{n}{\abs{\widetilde{x}}{Q}}}}{\Delta_2'}{Q_2}
			\]
			%
				\noi and for fresh $t$
			%
			\[
				\horel	{\Gamma}{\Delta_1'}{\newsp{\widetilde{m_1}'}{P_2 \Par \htrigger{t}{\abs{\widetilde{x}}{P}}}}
				{\hwb}
				{\Delta_2'}{ \newsp{\widetilde{m_2}' }{Q_2 \Par \htrigger{t}{\abs{\widetilde{x}}{Q}}}}
			\]
			%
			which implies
			\[
				\horel{\Gamma}{\Delta_1'}{\newsp{\widetilde{m_1}'}{P_2 \Par \htrigger{t}{\abs{\widetilde{x}}{P}}}}
				{ \hby{ \bactinp{t}{m} } }
				{\Delta_1''}{\newsp{\widetilde{m_1}'}{P_2 \Par \newsp{s}{\binp{s}{y} \appl{y}{m} \Par \bout{\dual{s}}{\abs{\widetilde{x}}{P}} \inact } }}
			\]
			and
			\[
				\horel{\Gamma}{\Delta_2'}{\newsp{\widetilde{m_2}'}{Q_2 \Par \htrigger{t}{\abs{\widetilde{x}}{P}}}}
				{ \Hby{ \bactinp{t}{m} } }
				{\Delta_2''}{\newsp{\widetilde{m_1}'}{Q_2 \Par \newsp{s}{\binp{s}{y} \appl{y}{m} \Par \bout{\dual{s}}{\abs{\widetilde{x}}{Q}} \inact } }}
			\]
			%
				\noi From the  third case of this proof we can conclude that $\forall R$ with $\fpv{R} = \set{x}$:
			%
			\[
				\begin{array}{rl}
					\Gamma; \es; &\Delta_1' \proves \newsp{\widetilde{m_1}'}{P_2 \Par \htrigger{t}{\abs{\widetilde{x}}{P}}}\\
					\hby{ \bactinp{t}{m} } &
					\Delta_1'' \proves \newsp{\widetilde{m_1}'}{P_2 \Par \newsp{s}{\binp{s}{y} R \subst{m}{x} \Par \bout{\dual{s}}{\abs{\widetilde{x}}{P}} \inact } }\\
					\hby{\tau} &
					\Delta_1'' \proves \newsp{\widetilde{m_1}'}{P_2 \Par R \subst{m}{x} \subst{\abs{\widetilde{x}}{P}}{y}} \\
				\end{array}
			\]
			and
			\[
				\begin{array}{rl}
					\Gamma; \es; &\Delta_1' \proves \newsp{\widetilde{m_2}'}{Q_2 \Par \htrigger{t}{\abs{\widetilde{x}}{P}}}\\
					{ \Hby{ \bactinp{t}{m} } } &
					\Delta_1'' \proves \newsp{\widetilde{m_1}'}{Q_2 \Par \newsp{s}{\binp{s}{y} R \subst{m}{x} \Par \bout{\dual{s}}{\abs{\widetilde{x}}{Q}} \inact } }\\
					\hby{\tau} &
					\Delta_1'' \proves \newsp{\widetilde{m_1}'}{Q_2 \Par\binp{s}{y} R \subst{m}{x} \subst{}{y}{\abs{\widetilde{x}}{Q}}}
				\end{array}
			\]
				\noi and furthermore it is easy to see that $\forall R$ with $\fpv{R} = \set{x}$:
			%
			\[
				\horel{\Gamma}{\Delta_1''}{\newsp{\widetilde{m_1}}{P_2 \Par  R \subst{m}{x} \subst{\abs{\widetilde{x}}{P}}{y}}}{\hwb}{\Delta_2}{\newsp{\widetilde{m_2}}{Q_2 \Par R \subst{m}{x} \subst{\abs{\widetilde{x}}{Q}}{y}}}
			\]
			%
				\noi as required by the definition of $\wbc$.

		\item	Case $\ell = \news{\widetilde{m_1}'} \bactout{n}{\widetilde{m_1}}$: 

				This case shares a similar argumentation with the previous case.
	\end{enumerate}
	\qed
\end{proof}


%%%%%%%%%%%%%%%%%%%%%%%%%%%%%%%%%%%%%%%%%%%%%%%%%%%%%%%%%
%  WBC IS CONG
%%%%%%%%%%%%%%%%%%%%%%%%%%%%%%%%%%%%%%%%%%%%%%%%%%%%%%%%%

\begin{lemma}
	\label{app:lem:wbc_is_cong}
	$\wbc \subseteq \cong$.
\end{lemma}


\begin{proof}
	\noi We prove that $\wbc$ satisfies the three defining properties of $\cong$:
	reduction closure, barb preservation, congruence (cf. \defref{def:rc}).
%

\noi	{\bf Reduction Closed:}
	Let
		$\horel{\Gamma}{\Delta_1}{P_1}{\wbc}{\Delta_2}{P_2}$. The reduction

	\[
		\horel{\Gamma}{\Delta_1}{P_1}{\by{}}{\Delta_1'}{P_1'}
	\]
%
	\noi implies that 
	$\exists P_2'$ such that 
%
	\begin{eqnarray*}
		\horel{\Gamma}{\Delta_2}{P_2}{\By{}}{\Delta_2'}{P_2'}\\
		\horel{\Gamma}{\Delta_1}{P_1'}{\wbc}{\Delta_2'}{P_2'}
	\end{eqnarray*}
%
	\noi Same arguments hold for the symmetric case, thus $\wbc$ is reduction closed.

	\noi {\bf Barb Preservation:} We have that
%
	\begin{eqnarray*}
		\Gamma; \emptyset; \Delta_1 \proves P_1 \barb{n}
	\end{eqnarray*}
%
	implies that
	\begin{eqnarray*}
		P &\cong& \newsp{\widetilde{m}}{\bout{n}{V_1} P_3 \Par P_4}\\
		\dual{n} &\notin& \Delta_1
	\end{eqnarray*}
%
	\noi From the definition of $\wbc$ we get that
%
\[
	\horel	{\Gamma}{\Delta_1}{\newsp{\widetilde{m}}{\bout{n}{V_1} P_3 \Par P_4}}
		{\by{\news{s_1} \bactout{n}{V_1}}}
		{\Delta_1'}
		{\newsp{\widetilde{m'}}{P_3 \Par P_4}}
\]
%
	\noi implies
%
	\begin{eqnarray*}
		\horel{\Gamma}{\Delta_2}{P_2}{\By{\news{m_2} \bactout{n}{V_2}}}{\Delta_2'}{P_2'}\\
	\end{eqnarray*}
%
	\noi From the last result we obtain
%
	\begin{eqnarray*}
		\Gamma; \emptyset; \Delta_2 \proves P_2 \Barb{n}
	\end{eqnarray*}
%
	\noi as required.

	\noi {\bf Congruence:}

	\noi The congruence property requires that we check that $\wbc$
	is preserved under any context.
	The most interesting context case is parallel composition.

	\noi We construct a congruence relation. Let
	\[
	\begin{array}{rcl}
		\mathcal{S} &=&	\set{
				(\Gamma; \emptyset; \Delta_1 \cat \Delta_3 \proves \newsp{\widetilde{n_1}}{P_1 \Par R} \hastype \Proc,
				\Gamma; \emptyset; \Delta_2 \cat \Delta_3 \proves \newsp{\widetilde{n_2}}{P_2 \Par R})
				\setbar \\
		& &		\horel{\Gamma}{\Delta_1}{P_1}{\wbc}{\Delta_2}{P_2}, \forall \Gamma; \emptyset; \Delta_3 \proves R \hastype \Proc\\
		& &}
	\end{array}
	\]
	\noi We need to show that 
	%the above congruence is a bisimulation.
	%To show that 
	$\mathcal{S}$ is a bisimulation:  we do a case analysis on the structure
	of the $\by{\ell}$ transition. There are three main cases.


	\begin{enumerate}
	%%%%%%%%%%%%%%%
	% Case 1
	%%%%%%%%%%%%%%%

		\item Suppose
				%
				\[
					\horel{\Gamma}{\Delta_1 \cat \Delta_3}{\newsp{\widetilde{n_1}}{P_1 \Par R}}
					{\by{\ell}}
					{\Delta_1' \cat \Delta_3}{\newsp{\widetilde{n_1'}}{P_1' \Par R}}
				\]
				%
				\noi The case is divided into three subcases:

				\begin{enumerate}[i.]
					\item Sub-case	$\ell \notin \set{\news{\widetilde{m}} \bactout{n}{\abs{\widetilde{x}}{Q}}, \news{\widetilde{mm_1}} \bactout{n}{\widetilde{m_1}}}$:
					
							\noi From the definition of typed transition we get:
					%
							\[
								\horel{\Gamma}{\Delta_1}{P_1}{\by{\ell}}{\Delta_1'}{P_1'}
							\]
							\noi which implies that
							\begin{eqnarray}
								\horel{\Gamma}{\Delta_1}{P_2}{\By{\ell}}{\Delta_2'}{P_2'}
								\label{lem:wbc_is_cong1}\\
								\horel{\Gamma}{\Delta_1'}{P_1'}{\wbc}{\Delta_2''}{P_2'}
								\label{lem:wbc_is_cong2}
							\end{eqnarray}
					%
							\noi From transition in~\eqref{lem:wbc_is_cong1} we conclude that 
							\[
								\horel{\Gamma}{\Delta_2 \cat \Delta_3}{\newsp{\widetilde{n_2}}{P_2 \Par R}}
								{\By{\ell}}
								{\Delta_2' \cat \Delta_3}{\newsp{\widetilde{n_2}'}{P_2' \Par R}}
							\]
					%
							\noi Furthermore, from \eqref{lem:wbc_is_cong2} and the definition of $\mathcal{S}$ we infer that
					%
							\[
								\horel{\Gamma}{\Delta_1' \cat \Delta_3}{\newsp{\widetilde{n_1}'}{P_1' \Par R}}
								{\ \mathcal{S}\ }
								{\Delta_2' \cat \Delta_3}{\newsp{\widetilde{n_2}'}{P_2' \Par R}}
							\]

					\item	Sub-case $\ell = \news{\widetilde{m_1}} \bactout{n}{\abs{\widetilde{x}}{Q_1}}$:

							\noi From the definition of typed transition we get
							\[
								\horel{\Gamma}{\Delta_1}{P_1}
								{\by{\news{\widetilde{m_1}} \bactout{n}{\abs{\widetilde{x}}{Q_1}}}}
								{\Delta_1'}{P_1'}
							\]
							\noi which implies that
					%
							\begin{eqnarray}
								&& \horel{\Gamma}{\Delta_1}{P_2}{\By{\news{\widetilde{m_2}} \bactout{n}{\abs{\widetilde{x}}{Q_2}}}}{\Delta_2'}{P_2'}
								\label{lem:wbc_is_cong3} \\
								&&\forall Q, \set{x} \in \fpv{Q} \nonumber \\
						%		\forall s'
								&& \horel{\Gamma}{\Delta_1''}{\newsp{\widetilde{n_1}''}{P_1' \Par Q \subst{\abs{\widetilde{x}}{Q_1}}{x}}}
								{\ \wbc\ }
								{\Delta_2''}{\newsp{\widetilde{n_2}''}{P_2' \Par Q \subst{\abs{\widetilde{x}}{Q_2}}{x}}}
								\label{lem:wbc_is_cong4}
							\end{eqnarray}
					%
							\noi From transition~\eqref{lem:wbc_is_cong3} conclude that 
							\[
								\horel{\Gamma}{\Delta_2 \cat \Delta_3}{\newsp{\widetilde{n_2}}{P_2 \Par R}}
								{\By{\news{\widetilde{m_2}} \bactout{n}{\abs{\widetilde{x}}{Q_2}}}}
								{\Delta_2' \cat \Delta_3}{\newsp{\widetilde{n_2}'}{P_2' \Par R}}
							\]
					%
							\noi Furthermore, from~\eqref{lem:wbc_is_cong4} we conclude that $\forall Q$ with $\set{x} = \fpv{Q}$
					%
							\[
								\horel{\Gamma}{\Delta_1'' \cat \Delta_3}{\newsp{\widetilde{n_1}''}{P_1' \Par Q \subst{(\widetilde{x}) Q_1}{x} \Par R}}
								{\ \mathcal{S}\ }
								{\Delta_2'' \cat \Delta_3}{\newsp{\widetilde{n_2}''}{P_2' \Par Q \subst{\abs{\widetilde{x}}{Q_2}}{x} \Par R}}
							\]
					%

					\item	Sub-case $\ell = \news{\widetilde{mm_1}} \bactout{n}{\widetilde{m_1}}$:

							\noi From the definition of typed transition we get that
							\[
								\horel{\Gamma}{\Delta_1}{P_1}
								{\by{\news{\widetilde{mm_1}} \bactout{n}{\widetilde{m_1}}}}
								{\Delta_1'}{P_1'}
							\]
							\noi which implies that $\exists P_2', s_2$ such that
					%
							\begin{eqnarray}
								&& \horel{\Gamma}{\Delta_1}{P_2}
								{\By{\news{\widetilde{mm_2}} \bactout{n}{\widetilde{m_2}}}}
								{\Delta_2'}{P_2'}
								\label{lem:wbc_is_cong5}\\
								&&\forall Q, x = \fn{Q}, \nonumber \\%  &&
								&& \horel{\Gamma}{\Delta_1''}{\newsp{\widetilde{n_1}}{P_1' \Par Q \subst{\widetilde{m_1}}{\widetilde{x}}}}
								{\ \wbc\ }
								{\Delta_2''}{\newsp{\widetilde{n_2}}{P_2' \Par Q \subst{\widetilde{m_2}}{\widetilde{x}}}}
								\label{lem:wbc_is_cong6}
							\end{eqnarray}
					%
						\noi From transition~\eqref{lem:wbc_is_cong5} conclude that 
						\[
							\horel{\Gamma}{\Delta_2 \cat \Delta_3}{\newsp{\widetilde{n_2}'}{P_2 \Par R}}
							{\By{\news{\widetilde{mm_2}} \bactout{n}{\widetilde{m_2}}}}
							{\Delta_2' \cat \Delta_3}{\newsp{\widetilde{n_2}'''}{P_2' \Par R}}
						\]
					%
						\noi Furthermore, from~\eqref{lem:wbc_is_cong6} we conclude that $\forall Q, x = \fn{Q}$
					%
						\[
							\horel{\Gamma}{\Delta_1'' \cat \Delta_3}{\newsp{\widetilde{n_1}''}{P_1' \Par Q \subst{\widetilde{m_1}}{\widetilde{x}} \Par R}}
							{\ \mathcal{S}\ }
							{\Delta_2'' \cat \Delta_3}{\newsp{\widetilde{n_2}''}{P_2' \Par Q \subst{\widetilde{m_2}}{\widetilde{x}} \Par R}}
						\]
					%
				\end{enumerate}
	%%%%%%%%%%%%%%%
	% Case 2
	%%%%%%%%%%%%%%%

		\item Suppose
			%
				\[
					\horel{\Gamma}{\Delta_1 \cat \Delta_3}{\newsp{\widetilde{m_1}}{P_1 \Par R}}
					{\by{\ell}}
					{\Delta_1 \cat \Delta_3'}{\newsp{\widetilde{m_1}'}{P_1 \Par R'}}
				\]
				\noi This case is divided into three subcases:

				\begin{enumerate}[i.]
			%
					\item Sub-case 	$\ell \notin \set{\news{\widetilde{m}} \bactout{n}{\abs{\widetilde{x}}{Q}}, \news{\widetilde{mm_1}} \bactout{n}{\widetilde{m_1}}}$:

							\noi From the LTS we get that:
							\[
								\horel{\Gamma}{\Delta_3}{R}{\by{\ell}}{\Delta_3'}{R'}
							\]
						%
							\noi Which in turn implies
							\begin{eqnarray*}
								\horel{\Gamma}{\Delta_2 \cat \Delta_3}{\newsp{\widetilde{m_2}}{P_2 \Par R}}
								{\by{\ell}}
								{\Delta_2 \cat \Delta_3'}{\newsp{\widetilde{m_2}'}{P_2 \Par R'}}
							\end{eqnarray*}
						%
							\noi From the definition of $\mathcal{S}$ we conclude that
							\[
								\horel{\Gamma}{\Delta_1 \cat \Delta_3'}{\newsp{\widetilde{m_1}'}{P_1 \Par R'}}
								{\ \mathcal{S}\ }
								{\Delta_2 \cat \Delta_3''}{\newsp{\widetilde{m_2}'}{P_2 \Par R'}}
							\]
							\noi as required.

				\item	Sub-case $\ell = \news{\widetilde{m_1}} \bactout{n}{\abs{\widetilde{x}}{Q}}$:

						\noi From the LTS we get that:
						\begin{eqnarray}
							& &	\horel{\Gamma}{\Delta_3}{R}{\by{\ell}}{\Delta_3'}{R'}
								\label{lem:wbc_is_cong7}\\
							& & 	\forall R_1, \set{x} = \fpv{R_1},
								\nonumber\\
					%		\forall s'
							& &	\Gamma; \emptyset; \Delta_3'' \proves \newsp{\widetilde{m}'}{R' \Par R_1 \subst{\abs{\widetilde{x}}{Q}}{x}} \hastype \Proc
								\label{lem:wbc_is_cong8}
						\end{eqnarray}
					%
						\noi From~\eqref{lem:wbc_is_cong7} we get that
						\[
							\horel{\Gamma}{\Delta_2 \cat \Delta_3}{\newsp{\widetilde{m_2}'}{P_2 \Par R}}{\by{\ell}}{\Delta_2 \cat \Delta_3'}{\newsp{\widetilde{m_2}}{P_2 \Par R'}}
						\]
						\noi Furthermore, from~\eqref{lem:wbc_is_cong8} and the definition of $\mathcal{S}$ we infer that
						$\forall R_1$ with $\set{x} \in \fpv{R_1}$
						\[
							\mhorel{\Gamma}{\Delta_1 \cat \Delta_3''}{\newsp{\widetilde{m_1}}{P_1 \Par \newsp{\widetilde{m}'}{R' \Par R_1 \subst{\abs{\widetilde{x}}{Q}}{x}}}}
							{\ \mathcal{S}\ }
							{\Delta_2 \cup \Delta_3''}{}{\newsp{\widetilde{m_2}}{P_2 \Par \newsp{\widetilde{m}'}{R' \Par R_1 \subst{\abs{\widetilde{x}}{Q}}{x}}}}
						\]
						\noi as required.

				\item	Sub-case $\ell = \news{\widetilde{mm}} \bactout{n}{\widetilde{m}}$:

					\noi From the typed LTS we get that:
					\begin{eqnarray}
						& &	\horel{\Gamma}{\Delta_3}{R}{\by{\ell}}{\Delta_3'}{R'}
							\label{lem:wbc_is_cong9} \\
						& &	\forall Q, \widetilde{x} = \fn{Q}, \nonumber\\
						& &	\Gamma; \emptyset; \Delta_3'' \proves \newsp{\widetilde{m}'}{R' \Par Q \subst{\widetilde{m}}{\widetilde{x}}} \hastype \Proc
							\label{lem:wbc_is_cong10}
					\end{eqnarray}
				%
					\noi From~\eqref{lem:wbc_is_cong9}, we obtain that
					\[
						\horel{\Gamma}{\Delta_2 \cat \Delta_3}{\newsp{\widetilde{m_2}}{P_2 \Par R}}{\by{\ell}}{\Delta_2 \cat \Delta_3'}{\newsp{\widetilde{m_2}}{P_2 \Par R'}}
					\]
					\noi Furthermore, from~\eqref{lem:wbc_is_cong10} and the definition of $\mathcal{S}$ we infer that
					$\forall Q, \widetilde{x} = \fn{Q}$
					\[
						\mhorel{\Gamma}{\Delta_1 \cat \Delta_3''}{\newsp{\widetilde{m_1}}{P_1 \Par \newsp{\widetilde{m}}{R' \Par Q \subst{\widetilde{m}'}{\widetilde{x}}}}}
						{\ \mathcal{S}\ }
						{\Delta_2 \cat \Delta_3''}{}{\newsp{\widetilde{m_2}}{P_2 \Par \newsp{\widetilde{m}'}{R' \Par Q \subst{\widetilde{m}}{\widetilde{x}}}}}
					\]
					\noi as required.
			\end{enumerate}

	%%%%%%%%%%%%%%%
	% Case 3
	%%%%%%%%%%%%%%%

	\item For the last case, suppose:
			\[
				\horel{\Gamma}{\Delta_1 \cat \Delta_3}{\newsp{\widetilde{m_1}}{P_1 \Par R}}
				{\by{\tau}}
				{\Delta_1' \cat \Delta_3'}{\newsp{\widetilde{m_1}'}{P_1' \Par R'}}
			\]

			\noi This case is divided into three subcases:

			\begin{enumerate}[i.]

				\item	$\horel{\Gamma}{\Delta_1}{P_1}{\by{\ell}}{\Delta_1'}{P_1'}$
						and $\ell \notin \set{\news{\widetilde{m}} \bactout{n}{\abs{\widetilde{x}}{Q}}, \news{\widetilde{mm_1}} \bactout{n}{\widetilde{m_1}}}$
						implies
					%
						\begin{eqnarray}
							\horel{\Gamma}{\Delta_3}{R}{\by{\dual{\ell}}}{\Delta_3}{R'}
							\label{lem:wbc_is_cong11} \\
							\horel{\Gamma}{\Delta_2}{P_2}{\By{\hat{\ell}}}{\Delta_2'}{P_2'}
							\label{lem:wbc_is_cong12}\\
							\horel{\Gamma}{\Delta_1'}{P_1'}{\wbc}{\Delta_2'}{P_2'}
							\label{lem:wbc_is_cong13}
						\end{eqnarray}
					%
						\noi From~\eqref{lem:wbc_is_cong11} and~\eqref{lem:wbc_is_cong12} we get
						\[
							\horel{\Gamma}{\Delta_2 \cat \Delta_3}{\newsp{\widetilde{m_2}}{P_2 \Par R}}
							{\By{}}
							{\Delta_2' \cat \Delta_3'}{\newsp{\widetilde{m_2}'}{P_2' \Par R'}}
						\]
					%
						\noi From~\eqref{lem:wbc_is_cong13} and the definition of ($\mathcal{S}$) we get that
						\[
							\horel{\Gamma}{\Delta_1' \cat \Delta_3'}{\newsp{\widetilde{m_1}'}{P_1' \Par R'}}
							{\ \mathcal{S}\ }
							{\Delta_2' \cat \Delta_3}{\newsp{\widetilde{m_2}'}{P_2' \Par R'}}
						\]
						\noi as required.

				\item
						$\horel{\Gamma}{\Delta_1}{P_1}{\by{\news{\widetilde{m_1}} \bactout{n}{\abs{\widetilde{x}}{Q_1}}}}{\Delta_1'}{P_1'}$
						implies
					%
						\begin{eqnarray}
							& & \horel{\Gamma}{\Delta_3}{R}
							{\by{\bactinp{n}{\abs{\widetilde{x}} {Q_1}}}}{\Delta_3'}
							{R' \subst{\abs{\widetilde{x}}{Q_1}}{x}}
							\label{lem:wbc_is_cong14}\\
							& & \horel{\Gamma}{\Delta_1 \cat \Delta_3}{\newsp{\widetilde{m_1}}{P_1 \Par R}}
							{\by{}}{\Delta_1' \cat \Delta_3'}
							{\newsp{\widetilde{m_1}''}{P_1' \Par R' \subst{\abs{\widetilde{x}}{Q_1}}{x}}}
							\nonumber \\
							& & \horel{\Gamma}{\Delta_2}{P_2}
							{\By{\news{\widetilde{m_2}} \bactout{n}{\abs{\widetilde{x}}{Q_2}}}}
							{\Delta_2'}{P_2'}
							\label{lem:wbc_is_cong15}\\
							& & \forall Q, \set{x} = \fpv{Q}, \nonumber \\
							& & \horel{\Gamma}{\Delta_1''}{\newsp{\widetilde{m_1}'}{P_1' \Par Q \subst{\abs{\widetilde{x}}{Q_1}}{x}}}
							{\ \wbc\ }
							{\Delta_2''}{\newsp{\widetilde{m_2}'}{P_2' \Par Q \subst{\abs{\widetilde{x}}{Q_2}}{x}}}
							\label{lem:wbc_is_cong16}
						\end{eqnarray}
					%
						From~\eqref{lem:wbc_is_cong14} and the Substitution Lemma~(\lemref{l:subst}) we obtain that
						\[
							\horel{\Gamma}{\Delta_3}{R}{\by{\bactinp{n}{\abs{\widetilde{x}} {Q_2}}}}{\Delta_3''}{R' \subst{\abs{\widetilde{x}}{Q_2}}{x}}
						\]
						%\dk{(prove that $\forall V, R \by{\bactinp{s}{V}} R'\subst{V}{x}$)}
						\noi to combine with~\eqref{lem:wbc_is_cong15} and get
						\[
							\horel{\Gamma}{\Delta_2 \cat \Delta_3}{\newsp{\widetilde{m_2}}{P_2 \Par R}}
							{\By{}}
							{\Delta_2' \cat \Delta_3''}{\newsp{\widetilde{m_2}''}{P_2' \Par R' \subst{\abs{\widetilde{x}}{Q_2}}{X}}}
						\]
					%
						\noi In result in~\eqref{lem:wbc_is_cong16}, set $Q$ as $R'$ to obtain:
					%
					%	\noi From~\eqref{lem:wbc_is_cong16} and the definition of $\mathcal{S}$ we get that
						\[
							\horel{\Gamma}{\Delta_1''}{\newsp{\widetilde{m_1}'}{P_1' \Par R' \subst{\abs{\widetilde{x}}{Q_1}}{x}}}
							{\ \mathcal{S}\ }{ \Delta_2''}
							{\newsp{\widetilde{m_2}'}{P_2' \Par R' \subst{\abs{\widetilde{x}}{Q_2}}{x}}}
						\]

				\item
						$\horel{\Gamma}{\Delta_1}{P_1}{\by{\news{\widetilde{mm_1}} \bactout{n}{\widetilde{m_1}}}}{\Delta_1'}{P_1'}$
					%
						\begin{eqnarray}
							& & \horel{\Gamma}{\Delta_3}{R}
							{\by{\bactinp{n}{\widetilde{m_1}}}}
							{\Delta_3'}{R' \subst{\widetilde{m_1}}{\widetilde{x}}}
							\label{lem:wbc_is_cong24}\\
							& & \horel{\Gamma}{\Delta_1 \cup \Delta_3}{\newsp{\widetilde{m_1}}{P_1 \Par R}}
							{\by{}}
							{\Delta_1' \cup \Delta_3'}{\newsp{\widetilde{m_1}''}{P_1' \Par R' \subst{m_1}{x}}}
							\nonumber \\
							& & \horel{\Gamma}{\Delta_2}{P_2}
							{\By{\news{\widetilde{mm_2}} \bactout{n}{\widetilde{m_2}}}}
							{\Delta_2'}{P_2'}
							\label{lem:wbc_is_cong25}\\
							& & \forall Q, \set{x} = \fpv{Q}, \nonumber \\
							& & \horel{\Gamma}{\Delta_1''}{\newsp{\widetilde{m_1}'}{P_1' \Par Q \subst{\widetilde{m_1}}{\widetilde{x}}}}
							{\ \wbc\ }
							{\Delta_2''}{\newsp{\widetilde{m_2}'}{P_2' \Par Q \subst{\widetilde{m_2}}{\widetilde{x}}}}
							\label{lem:wbc_is_cong26}
						\end{eqnarray}
					%
						From~\eqref{lem:wbc_is_cong24} and the Substitution Lemma~(\lemref{l:subst}) we get that
						\[
							\horel{\Gamma}{\Delta_3}{R}{\by{\bactinp{n}{\widetilde{m_2}}}}{\Delta_3''}{R' \subst{\widetilde{m_2}}{\widetilde{x}}}
						\]
						%\dk{(prove that $\forall V, R \by{\bactinp{s}{V}} R'\subst{V}{x}$)}
						\noi to combine with~\eqref{lem:wbc_is_cong25} and get
						\[
							\horel{\Gamma}{\Delta_2 \cat \Delta_3}{\newsp{\widetilde{m_2}}{P_2 \Par R}}
							{\By{}}
							{\Delta_2' \cat \Delta_3''}{\newsp{\widetilde{m_2}''}{P_2' \Par R' \subst{\widetilde{m_2}}{\widetilde{x}}}}
						\]
					%
						\noi Set $Q$ as $R'$ in result in \eqref{lem:wbc_is_cong26} to obtain
					%
					%	\noi From~\eqref{lem:wbc_is_cong16} and the definition of $\mathcal{S}$ we get that
						\[
							\horel{\Gamma}{\Delta_1''}{\newsp{\widetilde{m_1}'}{P_1' \Par R' \subst{\widetilde{m_1}}{\widetilde{x}}}}
							{\ \mathcal{S}\ }
							{\Delta_2''}{\newsp{\widetilde{m_2}'}{P_2' \Par R' \subst{\widetilde{m_2}}{\widetilde{x}}}}
						\]
		\end{enumerate}
	\end{enumerate}
	\qed
\end{proof}

%%%%%%%%%%%%%%%%%%%%%%%%%%%%%%%%%%%%%%%%%%%%%%%%%%%%%%%%%
%  CONG IS WB
%%%%%%%%%%%%%%%%%%%%%%%%%%%%%%%%%%%%%%%%%%%%%%%%%%%%%%%%%

We prove the result $\cong \subseteq \hwb$ following
the technique developed in~\cite{Hennessy07} and
refined for session types in~\cite{KYHH2015,KY2015}.

%\jp{Below I slightly modify the structure of items.}

\begin{definition}[Definability]\myrm
	\label{app:def:definibility}
	Let $\Gamma; \emptyset; \Delta_1 \proves P \hastype \Proc$.
	A visible action $\ell$ is \emph{definable} whenever
	there exists (testing) process
	$\Gamma; \emptyset; \Delta_2 \proves T\lrangle{\ell, \suc} \hastype \Proc$
	with $\suc$ fresh name % and $N$ a set of names.
	such that:
%
	\begin{enumerate}
		\item	Let $\ell \in \set{\bactsel{n}{\ell}, \bactbra{n}{\ell}, \bactinp{n}{\widetilde{m}}, \bactinp{n}{(\widetilde{x}) Q}}$.
		
			\begin{enumerate}[i.]
				\item	If $\horel{\Gamma}{\Delta_1}{P}{\hby{\ell}}{\Delta_1'}{P'}$
				%		and
				%		$\ell \in \set{\bactsel{n}{\ell}, \bactbra{n}{\ell}, \bactinp{n}{\widetilde{m}}, \bactinp{n}{\abs{\widetilde{x}}{Q}}}$
						then:
				%
						\[
							P \Par T\lrangle{\ell, \suc} \red P' \Par \bout{\suc}{\dual{n}} \inact \textrm{ and }
							\Gamma; \emptyset; \Delta_1' \cat \Delta_2' \proves P' \Par \bout{\suc}{\dual{n}} \inact
						\]

				\item
						If $P \Par T\lrangle{\ell, \suc} \red Q$ with			
						$\Gamma; \emptyset; \Delta \proves Q \barb{\suc}$ then \\
						$\horel{\Gamma}{\Delta_1}{P}{\Hby{\ell}}{\Delta_1'}{P'}$
						and $Q \scong P' \Par \bout{\suc}{\dual{n}} \inact$.
			\end{enumerate}
%
 		\item Let	(i) $\ell = \news{\widetilde{m}}\bactout{n}{V}$,
					and (ii) fresh $t$%, and
					%(iii) $\widetilde{m}'$ such that $ \widetilde{m}' \subseteq \widetilde{m}$

			\begin{enumerate}[i.]
				\item	If $\horel{\Gamma}{\Delta_1}{P}{\hby{\news{\widetilde{m}}\bactout{n}{V}}}{\Delta_1'}{P'}$
						then:
%
						\begin{itemize}
							\item $P \Par T\lrangle{\news{\widetilde{m}}\bactout{n}{V}, \suc} \red
							\newsp{\widetilde{m}}{P' \Par \htrigger{t}{V} \Par \bout{\suc}{\dual{n}, V} \inact}$
							\item $\Gamma; \emptyset; \Delta_1' \cat \Delta_2' \proves
							\newsp{\widetilde{m}}{P' \Par \htrigger{t}{V} \Par  \bout{\suc}{\dual{n}, V} \inact} \hastype \Proc$
						\end{itemize}

				\item	If $P \Par T\lrangle{\news{\widetilde{m}}\bactout{n}{V}, \suc} \red Q$
						with $\Gamma; \emptyset; \Delta \proves Q \barb{\suc}$ then 
						\begin{itemize}
							\item $\horel{\Gamma}{\Delta_1}{P}{\Hby{\news{\widetilde{m}}\bactout{n}{V}}}{\Delta_1'}{P'}$
							\item $Q \scong \newsp{\widetilde{m}}{P' \Par \htrigger{t}{V} \Par \bout{\suc}{\dual{n}, V} \inact}$
						\end{itemize}
			\end{enumerate}
	\end{enumerate}	
%
\end{definition}

We first show that every visible action $\ell$ is definable.

\begin{lemma}[Definability]
	\label{lem:definibility}
	Every action $\ell$ is definable.
\end{lemma}

\begin{proof}
	\noi We define $T\lrangle{\ell, \suc}$:
	\begin{eqnarray*}
		T\lrangle{\bactinp{n}{V}, \suc} &=&
		\bout{\dual{n}}{V} \bout{\suc}{\dual{n}} \inact
		\\
		T\lrangle{\bactbra{n}{l}, \suc} &=&
		\bsel{\dual{n}}{l} \bout{\suc}{\dual{n}} \inact
		\\
%		T\lrangle{\news{\widetilde{m}} \bactout{n}{\widetilde{m}}, \suc} &=&
%		\binp{\dual{n}}{\widetilde{y}} (\hotrigger{t}{x}{s}{\widetilde{y}} \Par \bout{\suc}{\dual{n}, \widetilde{y}} \inact)
%		\\
%		T\lrangle{\news{\widetilde{m}} \bactout{n}{\abs{\widetilde{x}}{Q}}, \suc} &=&
%		\binp{\dual{n}}{y} (\hotrigger{t}{x}{s}{\abs{\widetilde{x}}{(\appl{y}{\widetilde{x}}})} \Par \bout{\suc}{\dual{n}, y} \inact)
%		\\
		T\lrangle{\news{\widetilde{m}} \bactout{n}{V}, \suc} &=&
		\binp{\dual{n}}{y} (\htrigger{t}{y} \Par \bout{\suc}{\dual{n}, y} \inact)
		\\
		T\lrangle{\bactsel{n}{l}, \suc} &=&
		\bbra{\dual{n}}{l: \bout{\suc}{\dual{n}} \inact), l_i: \newsp{a}{\binp{a}{y} \bout{\suc}{\dual{n}} \inact}}_{i \in I}
	\end{eqnarray*}
%		
	\noi Let process 
	\[
		\Gamma; \emptyset; \Delta \proves P \hastype \Proc
	\]
	%
	\noi	It is straightforward to do a case analysis
			on all actions $\ell$ such that
			\[
				\Gamma; \emptyset; \Delta \proves P \hby{\ell} \Delta' \proves P'
			\]
			to show that $\ell$ is definable.
%	\noi it is straightforward to verify that $\forall \ell$, $\ell$ is definable.
	\qed
\end{proof}

%\jp{Here again I think that the closure should not mention environments in the  LHS.}

%%%%%%%%%%%%%%%%%%%%%%%%%%%%%%%%%%%%%%%%%%%%%%%%%%%%%%%%%%%%%%
%  EXTRUSION
%%%%%%%%%%%%%%%%%%%%%%%%%%%%%%%%%%%%%%%%%%%%%%%%%%%%%%%%%%%%%%


\begin{lemma}[Extrusion]\rm
	\label{lem:extrusion}
	Let $m_1 = \fn{V_1}$ and $m_2 = \fn{V_2}$ and fresh name $\suc$. 
	If 
	\[
		\horel{\Gamma}{\Delta_1'}{\newsp{\widetilde{m_1}}{P \Par \bout{\suc}{\dual{n}, V_1} \inact}}{\cong}{\Delta_2}{\newsp{\widetilde{m_2}}{Q \Par \bout{\suc}{\dual{n}, V_2} \inact}}
	\]
	then $\exists \Delta_1, \Delta_2$ such that
	\[
		\horel{\Gamma}{\Delta_1}{P}{\cong}{\Delta_2}{Q}.
	\]
\end{lemma}

\begin{proof}
	\noi Let
%
	\begin{eqnarray*}
		\mathcal{S}	&=&
					\set{(\Gamma; \es; \Delta_1 \proves P \hastype \Proc\ ,\ \Gamma; \es; \Delta_2 \proves Q \hastype \Proc) \setbar \\
				& &	\horel{\Gamma}{\Delta_1'}{\newsp{\widetilde{m_1}}{P \Par \bout{\suc}{\dual{n}, V_1} \inact}}
					{\cong}{\Delta_2}{\newsp{\widetilde{m_2}}{Q \Par \bout{\suc}{\dual{n}, V_2} \inact}},\\
				&&   \land m_1 = \fn{V_1} \land m_2 = \fn{V_2} \\
		&&}
	\end{eqnarray*}
%
	\noi We show that $\mathcal{S}$ is a reduction-closed, barbed congruence.


	\begin{itemize}
		\item	{\bf Reduction-closed:}

				$P \red P'$
				implies
				\[
					\newsp{\widetilde{m_1}}{P \Par \bout{\suc}{\dual{n}, V_1} \inact}
					\red
					\newsp{\widetilde{m_1}}{P' \Par \bout{\suc}{\dual{n}, V_1} \inact}
				\]
				which implies from the freshness of $\suc$
				\[
					\newsp{\widetilde{m_1}}{P \Par \bout{\suc}{\dual{n}, V_2} \inact}
					\red^{*}
					\newsp{\widetilde{m_1}}{Q' \Par \bout{\suc}{\dual{n}, V_2} \inact}
				\]
				which in turn implies
				$Q \red^{*} Q'$ as required.

	\item	{\bf Barb Preserving:}

			Let $\Gamma; \es; \Delta_1 \proves P \barb{m}$. We analyse three cases.
			%
		    \begin{itemize}
				\item	Case: $m \not= s$ ($m$ is not a session name)

						$\Gamma; \es; \Delta_1 \proves P \barb{m}$
						implies
					%
						\[
							\Gamma; \es; \Delta_1' \proves
							\newsp{\widetilde{m_1}}{P \Par \bout{\suc}{\dual{n}, V_1} \inact}
							\barb{m}
						\]
					%
						which implies
						\[
							\Gamma; \es; \Delta_2' \proves
							\newsp{\widetilde{m_2}}{Q \Par \bout{\suc}{\dual{n}, V_2} \inact}
							\Barb{m}
						\]
						which implies from the freshness of $\suc$ that
						$\Gamma; \es; \Delta_2 \proves Q \Barb{m}$ as required.

				\item	Case: $m = s$ ($m$ is a session name) and $m \not= n$.
						Similar proof as the previous case.

				\item	Case: $m = s$ ($m$ is a session name) and $m = n$ and
						$\Gamma; \es; \Delta_1 \proves P \barb{n}$
						
						The fact that $n$ is a session 
						implies that $n, \dual{n} \in \dom{\Delta_1'}$
						which implies from the definition
						of barbs (\defref{def:barbs}) that:
						%
						\[
							\Gamma; \es; \Delta_1' \proves
							\newsp{\widetilde{m_1}}{P \Par \bout{\suc}{\dual{n}, V_1} \inact}
							\not\barb{n}
						\]
						%
						This is because both endpoints of the session $n$
						are present in $\Delta_1'$.


						We compose $\Gamma; \es; \Delta_1 \proves P \hastype \Proc$ with
						$\binp{\dual{\suc}}{x, y} T\lrangle{\ell, \suc'}$
						with $\subj{\ell} = x$ and fresh $\suc'$ to get
						%
						\[
							\Gamma; \es; \Delta_1' \proves
							\newsp{\widetilde{m_1}}{P \Par \bout{\suc}{\dual{n}, V_1} \inact} \Par
							\binp{\dual{\suc}}{x, y} T\lrangle{\ell, \suc'}
						\]
						%
						The definition of definibility and the fact that $\Gamma; \es; \Delta_1 \proves P \barb{n}$
						implies that
						%
						\[
							\newsp{\widetilde{m_1}}{P \Par \bout{\suc}{\dual{n}, V_1} \inact} \Par
							\binp{\dual{\suc}}{x, \widetilde{y}} T\lrangle{\ell, \suc'}
							\red^{*} 
							\newsp{\widetilde{m_1}}{P' \Par \bout{\suc'}{\dual{n}, V_1'} \inact}
						\]
						%
						\noi and furthermore
						%
						\[
							\newsp{\widetilde{m_2}}{Q \Par \bout{\suc}{\dual{n}, V_2} \inact} \Par
							\binp{\dual{\suc}}{x, \widetilde{y}} T\lrangle{\ell, \suc'}
							\red^{*} 
							\newsp{\widetilde{m_2}}{Q' \Par \bout{\suc'}{\dual{n}, V_2'} \inact}
						\]
						%
						\noi The last reduction implies that
						$\Gamma; \es; \Delta_2 \proves Q \Barb{n}$ as required.
				\end{itemize}
    
		\item	{\bf Congruence:}

				The key case of congruence is parallel composition.
				The other cases are easier due to the fact that we are
				working with closed process terms (i.e.~input congruence is straightforward
				on closed process terms).
				We define relation $\mathcal{C}$ as
				%
				\begin{eqnarray*}
					\mathcal{C} &=&
					\set{	(\Gamma; \es; \Delta_1 \cat \Delta_3 \proves P \Par R \hastype \Proc,
							\Gamma; \es; \Delta_2 \cat \Delta_3 \proves Q \Par R \hastype \Proc) \setbar
					\\
					& &	\forall R \textrm{ such that } \exists \Delta_3, \Gamma;\es; \Delta_3 \proves R \hastype \Proc \land\\
					& &	\horel{\Gamma}{\Delta_1'}{\newsp{\widetilde{m_1}}{P \Par \bout{\suc}{\dual{n}, V_1} \inact}}
						{\cong}
						{\Delta_2'}{\newsp{\widetilde{m_2}}{Q \Par \bout{\suc}{\dual{n}, V_2} \inact}}}
				\end{eqnarray*}
		%
				We show that $\mathcal{C}$ is a congruence with respect to parallel composition.
				We distinguish two cases:
				\begin{itemize}
					\item	Case: $(\dual{n} \cup \fn{V_1} \cup \fn{V_2}) \cap \fn{R} = \es$

							From the contextual definition of $\cong$ we can deduce that
							$\forall \Gamma; \es; \Delta_3 \proves R \hastype \Proc$:
							%
							\[
								\horel{\Gamma}{\Delta_1' \cat \Delta_3}{\newsp{\widetilde{m_1}}{P \Par \bout{\suc}{\dual{n}, V_1} \inact} \Par R}
								{\cong}
								{\Delta_2' \cat \Delta_3}{\newsp{\widetilde{m_2}}{Q \Par \bout{\suc}{\dual{n}, V_2} \inact} \Par R}
							\]
							%
							Because of the requirement
							$(\dual{n} \cup \fn{V_1} \cup \fn{V_2}) \cap \fn{R} = \es$
							the above is up to structural congruence with
							\[
								\horel{\Gamma}{\Delta_1' \cat \Delta_3}{\newsp{\widetilde{m_1}}{P \Par \bout{\suc}{\dual{n}, V_1} \inact \Par R}}
								{\cong}
								{\Delta_2' \cat \Delta_3}{\newsp{\widetilde{m_2}}{Q \Par \bout{\suc}{\dual{n}, V_2} \inact \Par R}}
							\]
							From the definition of $\mathcal{C}$ the conclusion is trivial.

					\item	Case: $\widetilde{s} = \set{\dual{n}, \widetilde{m_1}} \cap \set{\dual{n}, \widetilde{m_2}} \in \fn{R}$.

							Let $R^{\widetilde{y}}$ such that $R = R^{y}\subst{\widetilde{s}}{\widetilde{\widetilde{y}}}$.


							From the contextual definition of $\cong$ we can deduce that for fresh $\suc'$
							$\forall \Gamma; \es; \Delta_3' \proves \binp{\dual{\suc}}{\widetilde{y}} (R^{\widetilde{y}} \Par \bout{\suc'}{\widetilde{y}} \inact) \hastype \Proc$:
							%
							\[
								\mhorel{\Gamma}
								{\Delta_1''}
									{\newsp{\widetilde{m_1}}{P \Par \bout{\suc}{\dual{n}, V_1} \inact}
									\Par \binp{\dual{\suc}}{\widetilde{y}} (R^{y} \Par \bout{\suc'}{\widetilde{y}} \inact)}
								{\cong}
								{\Delta_2''}{}
									{\newsp{\widetilde{m_2}}{Q \Par \bout{\suc}{\dual{n}, V_2} \inact}
									\Par \binp{\dual{\suc}}{\widetilde{y}} (R^{y} \Par \bout{\suc'}{\widetilde{y}} \inact)}
							\]

%							From the definition of $\mathcal{C}$
%							we can deduce that $\forall R^{y_1}$ such that $R = R^{y_1}\subst{\widetilde{s}}{\widetilde{y_1}}$
%							and $\suc'$ fresh and $\set{\widetilde{y}} = \set{\widetilde{y_1}} \cup \set{\widetilde{y_2}}$:
							%
%							\[
%								\mhorel{\Gamma}{\Delta_1''}{\newsp{\widetilde{m_1}'}{P \Par \bout{\suc}{\dual{n}, \widetilde{m_1}''} \inact} \Par \binp{\dual{\suc}}{\widetilde{y}} (R^{y_1} \Par \bout{\suc'}{\widetilde{y_2}} \inact)}
%								{\cong}
%								{\Delta_2''}{}{\newsp{\widetilde{m_2}'}{Q \Par \bout{\suc}{\dual{n}, \widetilde{m_2}''} \inact} \Par \binp{\dual{\suc}}{\widetilde{y}} (R^{y_1} \Par \bout{\suc'}{\widetilde{y_2}} \inact)}
%							\]
							%
							\noi Applying reduction closeness to the above pair we get:
							%
							\[
								\horel{\Gamma}{\Delta_1''}{\newsp{\widetilde{m_1}}{P \Par R \Par \bout{\suc'}{\dual{n}, V_1} \inact}}{\cong}{\Delta_2''}{\newsp{\widetilde{m_2}}{Q \Par R \Par \bout{\suc'}{\dual{n}, V_2} \inact}}
							\]
						%
						\noi The conclusion then follows from the definition of $\mathcal{C}$.
	    \end{itemize}
	\end{itemize}
	\qed
\end{proof}


\begin{lemma}\rm
	\label{app:lem:cong_is_wb}
	$\cong \subseteq \hwb$.
\end{lemma}

\begin{proof}
	\noi Let $\Re$ be the typed relation (we omit the typing information in the definition):
	\[
		\Re = \set{(P_1, P_2) \setbar \horel{\Gamma}{\Delta_1}{P_1}{\cong}{\Delta_2}{P_2}}
	\]

	To prove that $\Re$ is a higher-order bisimulation
	we do a case analysis on the transition:
	\[
		\horel{\Gamma}{\Delta_1}{P_1}{\by{\ell}}{\Delta_1'}{P_1'}
	\]
	We distinguish two cases: one for the $\tau$ transition and one case for
	the visible transitions $\ell$.

%% Case tau
\begin{enumerate}
	\item Suppose 
			\[
				\horel{\Gamma}{\Delta_1}{P_1}{\by{\tau}}{\Delta_1'}{P_1'}
			\]
			\noi The result follows the reduction closeness property of $\cong$ since
			\[
				\horel{\Gamma}{\Delta_2}{P_2}{\By{\tau}}{\Delta_2'}{P_2'}
			\]
			\noi and
			\[
				\horel{\Gamma}{\Delta_1'}{P_1'}{\cong}{\Delta_2'}{P_2'} \text{ implies } \horel{\Gamma}{\Delta_1'}{P_1'}{\ \Re\ }{\Delta_2'}{P_2'}
			\]

%% Case ell
	\item Suppose
		%
			\begin{eqnarray}
				\horel{\Gamma}{\Delta_1}{P_1}{\by{\ell}}{\Delta_1'}{P_1'}
				\label{lem:cong_is_wb1}
			\end{eqnarray}
		%
			\noi We choose test $T\lrangle{\ell, \suc}$ to get
		%
			\begin{eqnarray}
				\horel{\Gamma}{\Delta_1 \cat \Delta_3}{P_1 \Par T\lrangle{\ell, \suc}}{\cong}{\Delta_2 \cat \Delta_3}{P_2 \Par T\lrangle{\ell, \suc}}
				\label{lem:cong_is_wb2}
			\end{eqnarray}
		%
			\noi From this point we distinguish two subcases:

			\begin{enumerate}[i.]
			%% Subcase i
				\item	Sub-case $\ell \in \set{\bactinp{n}{V_1}, \bactsel{n}{l}, \bactbra{n}{l}}$:

						\noi By reducing~(\ref{lem:cong_is_wb1}), we obtain
					%
						\begin{eqnarray*}
							&& P_1 \Par T\lrangle{\ell, \suc} \red P_1' \Par \bout{\suc}{\dual{n}} \inact \\
							&& \Gamma; \es; \Delta_1' \cat \Delta_3' \proves P_1' \Par \bout{\suc}{\dual{n}} \inact \barb{\suc}
						\end{eqnarray*}
					%
						\noi implies from~(\ref{lem:cong_is_wb2})
					%
						\begin{eqnarray*}
							&& \Gamma; \es; \Delta_2 \cat \Delta_3 \proves P_2 \Par T\lrangle{\ell, \suc} \Barb{\suc}
						\end{eqnarray*}
					%
						\noi implies from Lemma~\ref{lem:definibility},
					%
						\begin{eqnarray*}
							&& \horel{\Gamma}{\Delta_2}{P_2}{\By{\ell}}{\Delta_2'}{P_2'}\\
							&& P_2 \Par T \lrangle{\ell, \suc} \red^{*} P_2' \Par \bout{\suc}{\dual{n}} \inact
						\end{eqnarray*}
					%
						\noi and
					%
						\[
							\horel{\Gamma}{\Delta_1' \cat \Delta_3'}{P_1' \Par \bout{\suc}{\dual{n}}\inact}{\cong}{\Delta_2' \cat \Delta_3'}{P_2' \Par \bout{\suc}{\dual{n}} \inact}
						\]
						We then apply \lemref{lem:extrusion} to get
					%
						\[
							\horel{\Gamma}{\Delta_1'}{P_1'}{\cong}{\Delta_2'}{P_2'} \text{ implies } \horel{\Gamma}{\Delta_1'}{P_1'}{\ \Re\ }{\Delta_2'}{P_2'}
						\]
					%
						\noi as required.

			%% Subcase ii
				\item	Sub-case $\ell = \news{\widetilde{m_1}} \bactout{n}{V_1}$:

						\noi Note that $T\lrangle{\news{\widetilde{m_1}} \bactout{n}{V_1}, \suc} = T\lrangle{\news{\widetilde{m_2}} \bactout{n}{V_2}, \suc}$

						\noi Transition~in (\ref{lem:cong_is_wb1}) becomes
					%
						\begin{eqnarray}
							\horel{\Gamma}{\Delta_1}{P_1}{\by{\news{\widetilde{m_1}} \bactout{n}{V_1}}}{\Delta_1'}{P_1'}
							\label{lem:cong_is_wb3}
						\end{eqnarray}
					%
						\noi If we use the test process $T\lrangle{\news{\widetilde{m_1}} \bactout{n}{V_1}, \suc}$ we reduce to:%~\ref{lem:cong_is_wb1} we get
					%
						\begin{eqnarray*}
							&& P_1 \Par T\lrangle{\news{\widetilde{m_1}} \bactout{n}{V_1}, \suc}
							\red
							\newsp{\widetilde{m_1}}{P_1' \Par \htrigger{t}{V_1}} \Par \bout{\suc}{\dual{n}, V_1} \inact \\
							&& \Gamma; \es; \Delta_1' \cat \Delta_3' \proves \newsp{\widetilde{m_1}}{P_1' \Par \htrigger{t}{V_1}} \Par \bout{\suc}{\dual{n}, V_1} \inact \barb{\suc}
						\end{eqnarray*}
					%
						\noi implies from~(\ref{lem:cong_is_wb2})
					%
						\[
							\Gamma; \es; \Delta_2 \cat \Delta_3 \proves P_2 \Par T\lrangle{\news{\widetilde{m_2}} \bactout{n}{V_2}, \suc} \Barb{\suc}
						\]
					%
						\noi implies from \lemref{lem:definibility}
					%
						\begin{eqnarray}
							&& \horel{\Gamma}{\Delta_2}{P_2}{\By{\news{\widetilde{m_2}} \bactout{n}{V_2}}}{\Delta_2'}{P_2'}
							\label{lem:cong_is_wb4}\\
							&& P_2 \Par T \lrangle{\ell, \suc} \red^{*} \newsp{\widetilde{m_2}}{P_2' \Par \htrigger{t}{V_2}} \Par \bout{\suc}{\dual{n}, V_2} \inact \nonumber
						\end{eqnarray}
					%
						\noi and
					%
						\[
							\mhorel{\Gamma}{\Delta_1' \cat \Delta_3'}{\newsp{\widetilde{m_1}}{P_1' \Par \htrigger{t}{\abs{\widetilde{x}}{Q_1}}} \Par \bout{\suc}{\dual{n}, V_1} \inact}
							{\cong}
							{\Delta_2' \cat \Delta_3'}{}{\newsp{\widetilde{m_2}}{P_2' \Par \htrigger{t}{\abs{\widetilde{x}}{Q_2}}} \Par \bout{\suc}{\dual{n}, V_2} \inact}
						\]
					%
						\noi We then apply \lemref{lem:extrusion} to get:
					%
						\[
							\mhorel{\Gamma}{\Delta_1'}{\newsp{\widetilde{m_1}}{P_1' \Par \htrigger{t}{V_1}}}
							{\cong}
							{\Delta_2'}{}{\newsp{\widetilde{m_2}}{P_2' \Par \htrigger{t}{V_2}}}
						\]
					%
						\noi From the last result and definition of $\Re$ we get:
						\[
							\mhorel{\Gamma}{\Delta_1'}{\newsp{\widetilde{m_1}}{P_1' \Par \htrigger{t}{V_1}}}
							{\ \Re\ }
							{\Delta_2'}{}{\newsp{\widetilde{m_2}}{P_2' \Par \htrigger{t}{V_2}}}
						\]
						\noi as required.

%				\item	Sub-case $\ell = \news{\widetilde{s}} \bactout{n}{\widetilde{m}}$:
%
%						\noi Follows similar arguments as the previous case.
			\end{enumerate}
\end{enumerate}
	\qed
\end{proof}

%%%%%%%%%%%%%%%%%%%%%%%%%%%%%%%%%%%%%%%%%%%%%%%%%%%%%%%%%%%%%%
% Proof of the main theorem
%%%%%%%%%%%%%%%%%%%%%%%%%%%%%%%%%%%%%%%%%%%%%%%%%%%%%%%%%%%%%%

%\begin{theorem}[Concidence]\label{app:thm:coincidence} We have:
%	\begin{enumerate}
%		\item	$\wbc\ =\ \hwb$.
%		\item	$\wbc\ =\ \cong$.
%	\end{enumerate}
%\end{theorem}
%
%\begin{proof}
%	\noi	\lemref{app:lem:wb_eq_wbf} proves $\hwb\ =\ \fwb$.
%			\lemref{app:lem:cong_is_wb} proves $\cong\ \subseteq\ \hwb$.
%			\lemref{app:lem:wb_is_wbc} proves $\hwb\ \subseteq\ \wbc$.
%			\lemref{app:lem:wbc_is_cong} proves $\wbc\ \subseteq\ \cong$.
%			From the above results, we conclude $\cong\ \subseteq\ \hwb\ =\ \fwb\ \subseteq\ \wbc\ \subseteq\ \cong$. 
%			\qed
%\end{proof}










\end{document}

\end{document}
