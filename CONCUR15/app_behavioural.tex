%\section{Behavioural Semantics}

We present the proofs for 
Theorem~\ref{the:coincidence} (Page \pageref{the:coincidence}).
We require an auxiliary result on 
deterministic transitions (\lemref{lem:up_to_deterministic_transition}).
Some notions needed to prove this auxiliary result are presented next.
Then we present the proof of \thmref{the:coincidence}, based on \emph{higher-order bisimilarity}.


%the theorem in \secref{sec:behavioural}.


%%%%%%%%%%%%%%%%%%%%%%%%%%%%%%%%%%%%%%%%%%%%%%%%%%%%%%%%%%%%%%
% tau - Innertness
%%%%%%%%%%%%%%%%%%%%%%%%%%%%%%%%%%%%%%%%%%%%%%%%%%%%%%%%%%%%%%

\subsection{Deterministic Transitions}
\label{app:sub_tau_inert}

We now define internal deterministic transitions as those associated to session synchronizations or to 
$\beta$-reductions: 
		
\begin{definition}[Deterministic Transition]\myrm
\label{def:dettrans}
	Let  $\Gamma; \es; \Delta \proves P \hastype \Proc$ be a balanced \HOp process. 
	Transition $\horel{\Gamma}{\Delta}{P}{\hby{\tau}}{\Delta'}{P'}$ is called:
	\begin{enumerate}[$-$]
		\item {\em Session transition} whenever the untyped transition $P \by{\tau} P'$ 
		is derived using 
			rule~$\ltsrule{Tau}$ 
		(where $\subj{\ell_1}$ and $\subj{\ell_2}$ in the premise 
		are dual endpoints), 
		possibly followed by uses of  $\ltsrule{Alpha}$, $\ltsrule{Res}$, $\ltsrule{Rec}$, or $\ltsrule{Par${}_L$}/
		\ltsrule{Par${}_R$}$.
		
		\item	{\em \betatran}	whenever the untyped transition $P \by{\tau} P'$
			is derived using rule $\ltsrule{App}$,
			possibly followed by uses of  $\ltsrule{Alpha}$, $\ltsrule{Res}$, $\ltsrule{Rec}$, or $\ltsrule{Par${}_L$}/
		\ltsrule{Par${}_R$}$.
	\end{enumerate}
%
We write
$\horel{\Gamma}{\Delta}{P}{\hby{\stau}}{\Delta'}{P'}$
and 
$\horel{\Gamma}{\Delta}{P}{\hby{\btau}}{\Delta'}{P'}$
to denote session and $\beta$-transitions, resp. Also, 
	 $\horel{\Gamma}{\Delta}{P}{\hby{\dtau}}{\Delta'}{P'}$ denotes
	either a session transition or a \betatran.
\end{definition}

Deterministic transitions imply the $\tau$-inertness property, which
is a property that ensures behavioural invariance on deterministic
transitions.


\begin{proposition}[$\tau$-inertness]\myrm
	\label{lem:tau_inert}
	Let  $\Gamma; \es; \Delta \proves P \hastype \Proc$ be a balanced \HOp process.
	Then
	\begin{enumerate}[1.]
		\item	$\horel{\Gamma}{\Delta}{P}{\hby{\dtau}}{\Delta'}{P'}$ implies
			$\horel{\Gamma}{\Delta}{P}{\wb}{\Delta'}{P'}$.
		\item	$\horel{\Gamma}{\Delta}{P}{\Hby{\dtau}}{\Delta'}{P'}$ implies
			$\horel{\Gamma}{\Delta}{P}{\wb}{\Delta'}{P'}$.
	\end{enumerate}
\end{proposition}

%\begin{proposition}[$\tau$-inertness]\rm
%	Let balanced \HOp process $\Gamma; \es; \Delta \proves P \hastype \Proc$.
%	$\horel{\Gamma}{\Delta}{P}{\hby{\dtau}}{\Delta'}{P'}$ implies
%	$\horel{\Gamma}{\Delta}{P}{\wb}{\Delta'}{P'}$.
%\end{proposition}

\begin{proof}
	\noi 
	We prove Part 1 --- the proof for Part 2 is straightforward from Part 1.
	The proof is done by induction on the structure of $\by{\tau}$
	which coincides the reduction $\red$.

	\noi Basic step:

	\noi - Case: $P = \appl{(\abs{x}{P})}{n}$:
%
	\[
		\horel{\Gamma}{\Delta}{\appl{(\abs{x}{P})}{n}}{\hby{\btau}}{\Delta'}{P \subst{n}{x}}
	\]
%
	\noi Bisimulation requirements hold since, there is no other transition to observe than ${\hby{\btau}}$.

	\noi - Case: $P = \bout{s}{V} P_1 \Par \binp{\dual{s}}{x} P_2$:
%
	\[
		\horel{\Gamma}{\Delta}{\bout{s}{V} P_1 \Par \binp{\dual{s}}{x} P_2}{\hby{\stau}}{\Delta'}{P_1 \Par P_2}
	\]
%
	\noi The proof follows from the fact that we can only observe a $\tau$
	action on typed process
	$\Gamma; \emptyset; \Delta \proves P \hastype \Proc$.
	Actions $\bactout{s}{V}$ and $\bactinp{\dual{s}}{V}$
	are forbiden by the LTS for typed environments.

	\noi It is easy to conclude then that $\horel{\Gamma}{\Delta}{P}{\wb}{\Delta'}{P'}$.

	\noi - Case: $P = \bsel{s}{l} P_1 \Par \bbra{\dual{s}}{l_i: P_i}_{i \in I}$

	\noi Similar arguments as the previous case.

	\noi Induction hypothesis:

	\noi If $P_1 \red P_2$ then $\horel{\Gamma_1}{\Delta_1}{P_1}{\wb}{\Delta_2}{P_2}$.

	\noi Induction Step:

	\noi - Case: $P = \news{s} P_1$
%
	\[
		\horel{\Gamma}{\Delta}{\news{s}{P_1}}{\hby{\stau}}{\Delta'}{\news{s} P_2}
	\]
%
	\noi From the induction hypothesis and the fact that bisimulation is a congruence
	we get that $\horel{\Gamma}{\Delta}{P}{\wb}{\Delta'}{P'}$.

	\noi - Case: $P = P_1 \Par P_3$
%
	\[
		\horel{\Gamma}{\Delta}{P_1 \Par P_3}{\hby{\stau}}{\Delta'}{P_2 \Par P_3}
	\]
%
	\noi From the induction hypothesis and the fact that bisimulation is a congruence
	we get that $\horel{\Gamma}{\Delta}{P}{\wb}{\Delta'}{P'}$.

	\noi - Case: $P \scong P_1$

	From the induction hypothesis and the fact that bisimulation is a congruence
	and structural congruence preserves $\wb$
	we get that $\horel{\Gamma}{\Delta}{P}{\wb}{\Delta'}{P'}$.


%	The proof for part two is an induction on the length of $\red^*$.
%	The basic step is trivial and the inductive step
%	deploys part 1 of this lemma and the fact that bisimulation is
%	transitive to conclude.
%	We can now conclude that
%	$P \wbc P'$ because $P \wbc P''$ and $P'' \wbc P'$.
%	\qed
\end{proof}


\begin{lemma}[Up-to Deterministic Transition]\myrm
	\label{lem:up_to_deterministic_transition}
	Let $\horel{\Gamma}{\Delta_1}{P_1}{\ \Re\ }{\Delta_2}{Q_1}$ such
	that if whenever:
%
	\begin{enumerate}
		\item	$\forall \news{\widetilde{m_1}} \bactout{n}{V_1}$ such that
			$
				\horel{\Gamma}{\Delta_1}{P_1}{\hby{\news{\widetilde{m_1}} \bactout{n}{V_1}}}{\Delta_3}{P_3}
			$
			implies that $\exists Q_2, V_2$ such that
			\[
				\horel{\Gamma}{\Delta_2}{Q_1}{\Hby{\news{\widetilde{m_2}} \bactout{n}{V_2}}}{\Delta_2'}{Q_2}
			\]
			and
			\[
				\horel{\Gamma}{\Delta_3}{P_3}{\Hby{\dtau}}{\Delta_1'}{P_2}
			\]
			and for fresh $t$:
			\[
				\horel{\Gamma}{\Delta_1''}{\newsp{\widetilde{m_1}}{P_2 \Par \htrigger{t}{V_1}}}
				{\ \Re\ }
				{\Delta_2''}{}{\newsp{\widetilde{m_2}}{Q_2 \Par \htrigger{t}{V_2}}}
%				\mhorel{\Gamma}{\Delta_1''}{\newsp{\widetilde{m_1}}{P_2 \Par \hotrigger{t}{x}{s}{V_1}}}
%				{\ \Re\ }
%				{\Delta_2''}{}{\newsp{\widetilde{m_2}}{Q_2 \Par \hotrigger{t}{x}{s}{V_2}}}
			\]
%
		\item	$\forall \ell \not= \news{\widetilde{m}} \bactout{n}{V}$ such that
			$
				\horel{\Gamma}{\Delta_1}{P_1}{\hby{\ell}}{\Delta_3}{P_3}
			$
			implies that $\exists Q_2$ such that 
			\[
				\horel{\Gamma}{\Delta_1}{Q_1}{\hat{\Hby{\ell}}}{\Delta_2'}{Q_2}
			\]
			and
			\[
				\horel{\Gamma}{\Delta_3}{P_3}{\Hby{\dtau}}{\Delta_1'}{P_2}
			\]
			and
			$\horel{\Gamma}{\Delta_1'}{P_2}{\ \Re\ }{\Delta_2'}{Q_2}$

		\item	The symmetric cases of 1 and 2.
	\end{enumerate}
	Then $\Re\ \subseteq\ \wb$.
\end{lemma}


\begin{proof}
	The proof is easy by considering the closure
	\[
		\Re^{\Hby{\dtau}} = \set{ \horel{\Gamma}{\Delta_1'}{P_2}{,}{\Delta_2'}{Q_1} \setbar \horel{\Gamma}{\Delta_1}{P_1}{\ \Re\ }{\Delta_2'}{Q_1},
		\horel{\Gamma}{\Delta_1}{P_1}{\Hby{\dtau}}{\Delta_1'}{P_2} }
	\]
	We verify that $\Re^{\Hby{\dtau}}$ is a bisimulation with
	the use of \propref{lem:tau_inert}.
	%\qed
\end{proof}

\subsection{Proof of \thmref{the:coincidence}}
\label{app:sub_coinc}

As mentioned in the paper, 
the proof of \thmref{the:coincidence}
relies on an auxiliary typed behavioral equivalence, \emph{higher-order bisimilarity}:

\begin{definition}[Higher-Order Bisimulation]\myrm
	\label{def:bisim}
	Typed relation
	$\Re$ is a {\em higher-order bisimulation} if for all
	$\horel{\Gamma}{\Delta_1}{P_1}{\ \Re\ }{\Delta_2}{Q_1}$, % implies:
%
	\begin{enumerate}[1.]
		\item	%$\forall \news{\widetilde{m_1}} \bactout{n}{V_1}$ such that
		   Whenever 
			$
				\horel{\Gamma}{\Delta_1}{P_1}{\hby{\news{\widetilde{m_1}} \bactout{n}{V_1}}}{\Delta_1'}{P_2}
			$
			there exist $Q_2$, $V_2$, $\Delta_2'$ such that
			\[
				\horel{\Gamma}{\Delta_2}{Q_1}{\Hby{\news{\widetilde{m_2}} \bactout{n}{V_2}}}{\Delta_2'}{Q_2}
			\]
			and, for a fresh $t$, 
			$
				\horel{\Gamma}{\Delta_1''}{\newsp{\widetilde{m_1}}{P_2 \Par \htrigger{t}{V_1}}}
				{\ \Re\ }
				{\Delta_2''}{}{\newsp{\widetilde{m_2}}{Q_2 \Par \htrigger{t}{V_2}}}$.
			
%
		\item	For all 
			$
				\horel{\Gamma}{\Delta_1}{P_1}{\hby{\ell}}{\Delta_1'}{P_2}
			$
			such that $\ell \not= \news{\widetilde{m}} \bactout{n}{V}$, there exist
			 $\exists Q_2$ and $\Delta_2'$ such that 
			\[
				\horel{\Gamma}{\Delta_1}{Q_1}{\Hby{\hat{\ell}}}{\Delta_2'}{Q_2}
			\]
			and
			$\horel{\Gamma}{\Delta_1'}{P_2}{\ \Re\ }{\Delta_2'}{Q_2}$.

		\item	The symmetric cases of 1 and 2.
	\end{enumerate}
	The Knaster-Tarski theorem ensures that the largest higher-order bisimulation exists;
	it is called \emph{higher-order bisimilarity} and is denoted by $\hwb$.
\end{definition}

\smallskip

\noi We split the proof of \thmref{the:coincidence} (Page \pageref{the:coincidence}) into 
several lemmas:
\begin{enumerate}[$-$]
\item	\lemref{lem:wb_eq_wbf} establishes $\wb\ =\ \wbf$.
\item	\lemref{lem:wb_is_wbc} exploits the process substitution result
	(\lemref{app:lem:proc_subst}) to prove that $\wb \subseteq \wbc$.
\item	\lemref{lem:wbc_is_cong} shows that $\wbc$ is a congruence
	which implies $\wbc \subseteq \cong$.
\item	\lemref{lem:cong_is_wb} shows  that $\cong \subseteq \wb$.
\end{enumerate}

%By the combination of the lemmas, we can obtain the theorem.

\noi
We now proceed to state and proof these lemmas, together with some auxiliary results.
\thmref{app:thm:coincidence} (Page~\pageref{app:thm:coincidence}) summarizes the coincidence result.

%%%%%%%%%%%%%%%%%%%%%%%%%%%%%%%%%%%%%%%%%%%%%%%%%%%%%%%%%
%  WB = WBF
%%%%%%%%%%%%%%%%%%%%%%%%%%%%%%%%%%%%%%%%%%%%%%%%%%%%%%%%%


\begin{lemma}\rm
	\label{lem:wb_eq_wbf}
	$\wb = \wbf$.
\end{lemma}

\begin{proof}
	\noi We only prove the direction $\wb \subseteq \wbf$. The
	direction $\wbf \subseteq \wb$ is similar.
Consider:
%
	\[
		\Re = \set{\horel{\Gamma}{\Delta_1}{P}{\ ,\ }{\Delta_2}{Q} \setbar \horel{\Gamma}{\Delta_1}{P}{\wb}{\Delta_2}{Q}}
	\]
%
	We show that $\Re$ is a characteristic bisimulation.
	The proof does a case analysis on the transition label $\ell$.

	\noi - Case $\ell = \news{\widetilde{m_1}} \bactout{n}{V_1}$ is the non-trivial case.

	\noi If
%
	\begin{eqnarray}
		\horel{\Gamma}{\Delta_1}{P}{\hby{\news{\widetilde{m_1}} \bactout{n}{V_1}}}{\Delta_1'}{P'}
		\label{lem:wb_eq_wbf1}
	\end{eqnarray}
	then $\exists Q, V_2$ such that
%
	\begin{eqnarray}
		\horel{\Gamma}{\Delta_2}{Q}{\Hby{\news{\widetilde{m_2}} \bactout{n}{V_2}}}{\Delta_2'}{Q'}
		\label{lem:wb_eq_wbf2}
	\end{eqnarray}
%
	and for fresh $t$:
	\[
		\mhorel{\Gamma}{\Delta_1'}{\newsp{\widetilde{m_1}}{P' \Par \hotrigger{t}{x}{s}{V_1}}}
		{\wb}
		{\Delta_2}{}{\newsp{\widetilde{m_2}}{Q' \Par \hotrigger{t}{x}{s}{V_2} }}
	\]
%
	From the last typed pair we can derive that for $\Gamma; \es; \Delta \proves V_1 \hastype U$:
%
	\[
		\mhorel{\Gamma}{\Delta_1'}{\newsp{\widetilde{m_1}}{P' \Par \hotrigger{t}{x}{s}{V_1}}}
		{\hby{\bactinp{t}{\map{\btinp{U} \tinact}^{x}}}}
		{\Delta_1''}{}{\newsp{\widetilde{m_1}}{P' \Par \newsp{s}{\map{\btinp{U} \tinact}^{s} \Par \bout{\dual{s}}{V_1} \inact}}}
	\]
%
	\noi implies
%
	\[
		\mhorel{\Gamma}{\Delta_2'}{\newsp{\widetilde{m_2}}{Q' \Par \hotrigger{t}{x}{s}{V_2}}}
		{\hby{\bactinp{t}{\map{\btinp{U} \tinact}^{x}}}}
		{\Delta_2''}{}{\newsp{\widetilde{m_2}}{Q' \Par \newsp{s}{\map{\btinp{U} \tinact}^{s} \Par \bout{\dual{s}}{V_2} \inact}}}
	\]
%
	\noi and $\Gamma; \es; \Delta' \proves V_2 \hastype U$.

	\noi Transition~(\ref{lem:wb_eq_wbf1}) implies transition~(\ref{lem:wb_eq_wbf2}). It remains to
	show that for fresh $t$:
%
	\[
		\mhorel{\Gamma}{\Delta_1'}{\newsp{\widetilde{m_1}}{P' \Par \fotrigger{t}{x}{s}{\btinp{U} \tinact}{V_1}}}
		{\wb}
		{\Delta_2}{}{\newsp{\widetilde{m_2}}{Q' \Par \fotrigger{t}{x}{s}{\btinp{U} \tinact}{V_2}}}
	\]
%
	The freshness of $t$ implies that
	\[
		\mhorel{\Gamma}{\Delta_1'}{\newsp{\widetilde{m_1}}{P' \Par \fotrigger{t}{x}{s}{\btinp{U} \tinact}{V_1}}}
		{\hby{\bactinp{t}{m'}}}
		{\Delta_1''}{}{\newsp{\widetilde{m_1}}{P' \Par \newsp{s}{\map{\btinp{U} \tinact}^{s} \Par \bout{\dual{s}}{V_1} \inact}}}
	\]
	\noi and
%
	\[
		\mhorel{\Gamma}{\Delta_2'}{\newsp{\widetilde{m_2}}{Q' \Par \fotrigger{t}{x}{s}{\btinp{U} \tinact}{V_2}}}
		{\hby{\bactinp{t}{m'}}}
		{\Delta_2''}{}{\newsp{\widetilde{m_2}}{Q' \Par \newsp{s}{\map{\btinp{U} \tinact}^{s} \Par \bout{\dual{s}}{V_2} \inact}}}
	\]
%
	\noi which coincides with the transitions for $\wb$.

	\noi - The rest of the cases are trivial.

	\noi The direction $\wbf \subseteq \wb$ is very similar to the
	direction $\wb \subseteq \wbf$: it requires a case analysis
	on the transition label $\ell$. Again the non-trivial case is
	$\ell = \news{\widetilde{m_1}} \bactout{n}{V_1}$.
	%\qed
\end{proof}


%%%%%%%%%%%%%%%%%%%%%%%%%%%%%%%%%%%%%%%%%%%%%%%%%%%%%%%%%
%  LINEAR SUBSTITUTION
%%%%%%%%%%%%%%%%%%%%%%%%%%%%%%%%%%%%%%%%%%%%%%%%%%%%%%%%%

The next lemma implies a process substitution lemma as a corollary.
Given two processes that are bisimilar under trigerred substitution
and characteristic process substitution, we can prove that they are
bisimilar under every process substitution. This result is
the key result for proving the soundness of the bisimulation. 
%We prove for the case of the general polyadic abstractions. 

%We also use one of the equalities in the substitution lemmas. 
%However all results hold for other equivalences. 

\newcommand{\auxtr}[1]{\abs{\widetilde{x}}{\binp{#1}{y} (\appl{y}{\widetilde{x}})}}

\begin{lemma}[Linear Process Substitution]\rm
	\label{lem:subst_equiv}
	If 
%
	\begin{enumerate}
		\item	$\fpv{P_2} = \fpv{Q_2} = \set{x}$.
		\item	$\Gamma; x: U; \Delta_1''' \proves P_2 \hastype \Proc$ and $\Gamma; x: U; \Delta_2''' \proves Q_2 \hastype \Proc$.
		\item	$\horel{\Gamma}{\Delta_1'}{\newsp{\widetilde{m_1}}{P_1 \Par P_2 \subst{\auxtr{t}}{x}}}
			{\wb}
			{\Delta_2'}{\newsp{\widetilde{m_2}}{Q_1 \Par Q_2 \subst{\auxtr{t}}{x}}}$, 
			for some fresh $t$.

		\item	$\horel{\Gamma}{\Delta_1''}{\newsp{\widetilde{m_1}}{P_1 \Par P_2 \subst{\omapchar{U}}{x}}}
			{\wb}{\Delta_2''}{\newsp{\widetilde{m_2}}{Q_1 \Par Q_2 \subst{\omapchar{U}}{x}}}$,
			for some $U$.
	\end{enumerate}
%
	then $\forall R$ such that $\fv{R} = \widetilde{x}$
\[
	\horel{\Gamma}{\Delta_1}{\newsp{\widetilde{m_1}}{P_1 \Par P_2 \subst{\abs{\widetilde{x}}{R}}{x}}}
	{\wb}
	{\Delta_2}{\newsp{\widetilde{m_2}}{Q_1 \Par Q_2 \subst{\abs{\widetilde{x}}{R}}{x}}}
\]
\end{lemma}

\begin{proof}
	We create a bisimulation closure:
%
	\begin{eqnarray*}
		\Re &=&
			\set{\horel{\Gamma}{\Delta_1}{\newsp{\widetilde{m_1}}{P_1 \Par P_2 \subst{\abs{\widetilde{x}}{R}}{x}}}{,}
			{\Delta_2}{\newsp{\widetilde{m_2}}{Q_1 \Par Q_2 \subst{\abs{\widetilde{x}}{R}}{x}}} \setbar \\
			&& \quad \forall R \textrm{ such that } \fv{R} = \widetilde{x}, \fpv{P_2} = \fpv{Q_2} = \set{x}\\
			&& \quad \Gamma; x: U; \Delta_1''' \proves P_2 \hastype \Proc, \Gamma; x: U; \Delta_2''' \proves Q_2 \hastype \Proc\\
			&& \quad \textrm{for fresh } t,\\
			&& \quad \horel{\Gamma}{\Delta_1'}{\newsp{\widetilde{m_1}}{P_1 \Par P_2 \subst{\auxtr{t}}{x}}}{\wb}{\Delta_2}{\newsp{\widetilde{m_2}}{Q_1 \Par Q_2 \subst{\auxtr{t}}{x}}},\\
			&& \quad \horel{\Gamma}{\Delta_1''}{\newsp{\widetilde{m_1}}{P_1 \Par P_2 \subst{\omapchar{U}}{x}}}{\wb}{\Delta_2''}{\newsp{\widetilde{m_2}}{Q_1 \Par Q_2 \subst{\omapchar{U}}{x}} \textrm{ for some } U}\\
			&&}
	\end{eqnarray*}
%
	\noi  We show that $\Re$ is a bisimulation up-to \betatran (\lemref{lem:up_to_deterministic_transition}).

	\noi We do a case analysis on the transition:
%
	\[
		\horel{\Gamma}{\Delta_1}{\newsp{\widetilde{m_1}}{P_1 \subst{\abs{\widetilde{x}}{R}}{x} \Par P_2\subst{\abs{\widetilde{x}}{R}}{x} }}{\by{\ell_1}}{\Delta_1'}{P_1'}
	\]
%
	%%%%%%%%%%%%%%%%%%%%%%%%%%%%%%
	%          Case
	%%%%%%%%%%%%%%%%%%%%%%%%%%%%%%

	\noi - Case: $P_2 \not= \appl{x}{\widetilde{n}}$ for some $\widetilde{n}$.
%
	\begin{eqnarray*}
		&&	\horel{\Gamma}{\Delta_1}
	{\newsp{\widetilde{m_1}}{P_1 \Par P_2 \subst{\abs{\widetilde{x}}{R}}{x} }}
			{\hby{\ell_1}}
{\Delta_1'}{\newsp{\widetilde{m_1'}}{P_1 \Par P_2' \subst{\abs{\widetilde{x}}{R}}{x}}}
	\end{eqnarray*}

	\noi From the latter transition we obtain that
%
\[
		\mhorel{\Gamma}{\Delta_1}{\newsp{\widetilde{m_1}}{P_1 \Par P_2 \subst{\auxtr{t}}{x}}}
		{\hby{\ell_1}}{\Delta_1'}{P' \scong}{\newsp{\widetilde{m_1}}{P_1' \Par P_2' \subst{\auxtr{t}}{x}}}
\]
%
	\noi which implies
%
	\begin{eqnarray}
		\Gamma; \es; &\Delta_2& \proves \newsp{\widetilde{m_2}}{Q_1 \Par Q_2 \subst{\auxtr{t}}{x}} \nonumber \\
		\Hby{\ell_2}&
		\Delta_2'& \proves Q' \scong \newsp{\widetilde{m_2}}{Q_1' \Par Q_2' \subst{\auxtr{t}}{x}}
		\label{lem:subst_equiv1}
		\\
		&&\horel{\Gamma}{\Delta_1'}{P' \Par C_1}{\wb}{\Delta_2'}{Q' \Par C_2} \label{lem:subst_equiv2}
	\end{eqnarray}
%
	\noi Furthermore, we have:
%
\[
	\horel{\Gamma}{\Delta_1}{\newsp{\widetilde{m_1}}{P_1 \Par P_2 \subst{\omapchar{U}}{x}}}{\hby{\ell_1}}
	{\Delta_1'}{P'' \scong \newsp{\widetilde{m_1'}}{P_1' \Par P_2' \subst{\omapchar{U}}{x}}}
\]
%
	\noi which implies
%
	\begin{eqnarray}
		\Gamma; \es; &\Delta_2& \proves \newsp{\widetilde{m_2}}{Q_1 \Par Q_2 \subst{\omapchar{U}}{x}} \nonumber \\
		\Hby{\ell_2} &\Delta_2'& \proves  Q'' \scong \newsp{\widetilde{m_2}'}{Q_1' \Par Q_2' \subst{\omapchar{U}}{x}}
		\label{lem:subst_equiv3}
		\\
		&&\horel{\Gamma}{\Delta_1'}{P'' \Par C_1}{\wb}{\Delta_2'}{Q'' \Par C_2} \label{lem:subst_equiv4}
	\end{eqnarray}
%
	\noi From~(\ref{lem:subst_equiv1}) and~(\ref{lem:subst_equiv3}) we obtain that $\forall R$ with $\fv{R} = \widetilde{x}$:
%
	\[
		\horel{\Gamma}{\Delta_2}{\newsp{\widetilde{m_2}}{Q_1 \Par Q_2 \subst{\abs{\widetilde{x}}{R}}{x}}}
		{\Hby{\ell_2}}
		{\Delta_2'}{\newsp{\widetilde{m_2}'}{Q_1' \Par Q_2' \subst{\abs{\widetilde{x}}{R}}{x}}}
	\]
%
	\noi The case concludes if we combine~(\ref{lem:subst_equiv2}) and~(\ref{lem:subst_equiv4}), to obtain that $\forall R$ with $\fv{R} = \widetilde{x}$
%
	\[
		\horel{\Gamma'}{\Delta_1''}{\newsp{\widetilde{m_1}'}{P_1' \Par P_2' \subst{\abs{\widetilde{x}}{R}}{x}} \Par C_1}
		{\ \Re\ }
		{\Delta_2''}{\newsp{\widetilde{m_2}'}{Q_1 \Par Q_2' \subst{\abs{\widetilde{x}}{R}}{x}} \Par C_2}
	\]

	%%%%%%%%%%%%%%%%%%%%%%%%%%%%%%
	%          Case
	%%%%%%%%%%%%%%%%%%%%%%%%%%%%%%

	\noi - Case: $P_2 = \appl{x}{\widetilde{n}}$ for some $\widetilde{n}$.

%	\noi We do a case analysis on action $\ell$.
%	\noi - Subcase: $\ell \not= \tau$.

	\noi $\forall R$ with $\fv{R} = \widetilde{x}$
%
	\[
		\mhorel{\Gamma}{\Delta_1}{\newsp{\widetilde{m_1}}{P_1 \Par (\appl{x}{\widetilde{n}}) \subst{\abs{\widetilde{x}}{R}}{x}}}
		{\hby{\btau}}
		{\Delta_1'}{}{\newsp{\widetilde{m_1'}}{P_1 \Par  R \subst{\widetilde{n}}{\widetilde{x}}}}
	\]
%
	\noi From the latter transition we get that:
%
	\nhorel{\Gamma}{\Delta_1}{\newsp{\widetilde{m_1}}{P_1 \Par \appl{x}{\widetilde{n}} \subst{\auxtr{t}}{x}}}
	{\hby{\btau} \hby{\bactinp{t}{\auxtr{t'}}}}
	{\Delta_1'}{\newsp{\widetilde{m_1'}}{P_1 \Par \appl{x}{\widetilde{n}} \subst{\auxtr{t'}}{x}}}
	{lem:subst_equiv5}
%
%	\begin{eqnarray}
%		\Gamma; \es; \Delta_1 &\hby{\bactinp{t}{\auxtr{t'}}}& \Delta_1' \proves
%		\newsp{\widetilde{m_1}}{P_1 \Par \appl{X}{\widetilde{n}} \subst{\auxtr{t}}{X}} \nonumber \\
%		&\hby{\bactinp{t}{\auxtr{t'}}}& 
%		\newsp{\widetilde{m_1'}}{P_1 \Par \appl{X}{\widetilde{n}} \subst{\auxtr{t'}}{X}}
%		\label{lem:subst_equiv5}
%	\end{eqnarray}
%
	\noi and $t'$ a fresh name. From the freshness of $t$,
	the determinacy of the application transition
	and the fact that $x$ is linear in $Q_2$
	it has to be the case that:
%
	\[
		\begin{array}{crll}
			&\Gamma; \es; \Delta_2'& \proves &
			\newsp{\widetilde{m_2'}}{Q_1 \Par Q_2 \subst{\auxtr{t}}{x}}\\
			\Hby{} & & &
			\newsp{\widetilde{m_2'}}{Q_1'' \Par Q_3 \Par \appl{x}{\widetilde{m}} \subst{\auxtr{t}}{x}} \\
			\hby{\btau} \hby{\bactinp{t}{\auxtr{t'}}}
			& \Delta_2''& \proves& \newsp{\widetilde{m_2'}}{Q_1' \Par \appl{x}{\widetilde{m}} \subst{\auxtr{t'}}{x}}
		\end{array}
	\]
%
%
%	\[
%		\begin{array}{rcll}
%			\Gamma; \es; \Delta_2' &\Hby{\bactinp{t}{\auxtr{t'}}}& \Delta_2'' \proves &
%			\newsp{\widetilde{m_2'}}{Q_1 \Par Q_2 \subst{\auxtr{t}}{X}}\\
%			&\Hby{}& &
%			\newsp{\widetilde{m_2'}}{Q_1'' \Par Q_3 \Par \appl{X}{\widetilde{m}} \subst{\auxtr{t}}{X}} \\
%			&\hby{\bactinp{t}{\auxtr{t'}}}& &
%			\newsp{\widetilde{m_2'}}{Q_1' \Par \appl{X}{\widetilde{m}} \subst{\auxtr{t'}}{X}} \\
%		\end{array}
%	\]
%
	\noi and
%
	\nhorel{\Gamma}{\Delta_1'}{\newsp{\widetilde{m_1'}}{P_1 \Par \appl{x}{\widetilde{n}} \subst{\auxtr{t'}}{x}}}
	{\wb}
	{\Delta_2'}{\newsp{\widetilde{m_2}'}{Q_1' \Par \appl{x}{\widetilde{m}} \subst{\auxtr{t'}}{x}}}
	{lem:subst_equiv6}
%
%
%	\begin{eqnarray}
%		\Gamma; \es; \Delta_1' &\wb& \Delta_2' \proves 
%		\newsp{\widetilde{m_1'}}{P_1 \Par \appl{X}{\widetilde{n}} \subst{\auxtr{t'}}{X}} \nonumber \\
%		& \wb &
%		\newsp{\widetilde{m_2}'}{Q_1' \Par \appl{X}{\widetilde{m}} \subst{\auxtr{t'}}{X}}
%		\label{lem:subst_equiv6}
%	\end{eqnarray} 
%
	\noi From the latter transition we can conclude that $\forall R$ with $\fv{R} = \set{x}$:
%
	\[
		\begin{array}{crll}
			&\Gamma; \es; \Delta_2'& \proves & 
			\newsp{\widetilde{m_2'}}{Q_1 \Par Q_2 \subst{\abs{\widetilde{x}}{R}}{x}}\\
			\Hby{} & & &
			\newsp{\widetilde{m_2'}}{Q_1' \Par \appl{x}{\widetilde{m}} \subst{\abs{\widetilde{x}}{R}}{x}} \\
			\hby{\btau}
			&\Delta_2''& \proves & \newsp{\widetilde{m_2'}}{Q_1' \Par R \subst{\widetilde{m}}{\widetilde{x}}}%\appl{x}{\widetilde{m}} \subst{\abs{\widetilde{x}}{R'}}{x}}
		\end{array}
	\]
%
%	\[
%		\begin{array}{rcll}
%			\Gamma; \es; \Delta_2' &\Hby{\ell}& \Delta_2'' \proves &
%			\newsp{\widetilde{m_2'}}{Q_1 \Par Q_2 \subst{\abs{\widetilde{x}}{R}}{X}}\\
%			&\Hby{}& &
%			\newsp{\widetilde{m_2'}}{Q_1' \Par \appl{X}{\widetilde{m}} \subst{\abs{\widetilde{x}}{R}}{X}} \\
%			&\hby{\ell}& &
%			\newsp{\widetilde{m_2'}}{Q_1' \Par \appl{X}{\widetilde{m}} \subst{\abs{\widetilde{x}}{R'}}{X}} \\
%		\end{array}
%	\]
%
	\noi From the definition of $S$ and~(\ref{lem:subst_equiv6}),
	we also conclude that
	\begin{eqnarray*}
		&& \horel{\Gamma}
		{\Delta_1'}{\newsp{\widetilde{m_1'}}{P_1 \Par R 
\subst{\widetilde{n}}{\widetilde{x}}}}
		{\hby{\btau}\ \Re\ \stackrel{\btau}{\longleftarrow}}
		{\Delta_2'}{\newsp{\widetilde{m_2}'}{Q_1' \Par R \subst{\widetilde{m}}{\widetilde{x}}}}
	\end{eqnarray*}
%
%	\noi the latter substitution can be rewritten as
%
%	\begin{eqnarray*}
%		&&\horel{\Gamma}
%		{\Delta_1'}{\newsp{\widetilde{m_1'}}{P_1 \Par \appl{X}{s} \subst{\abs{\widetilde{x}}{R'}}{X} \Par C}}
%		{\ \mathcal{S}\ }
%		{\Delta_2'}{\newsp{\widetilde{m_2}'}{Q_1' \Par \appl{X}{s'} \subst{\abs{\widetilde{x}}{R'}}{X} \Par C}}
%	\end{eqnarray*}
%
%	\noi with process $C$ being the process derived from action $\ell$
%	to complete the bisimulation closure.
	%\qed
\begin{comment}
	\noi - Subcase: $\ell = \tau$.

	\noi $\forall R$ with $\fv{R} = \widetilde{x}$
%
	\[
		\mhorel{\Gamma}{\Delta_1}{\newsp{\widetilde{m_1}}{P_1 \Par \appl{X}{\widetilde{n}} \subst{\abs{\widetilde{x}}{R}}{x}}}
		{\hby{\tau}}
		{\Delta_1'}{}{\newsp{\widetilde{m_1'}}{P_1' \Par \appl{X}{\widetilde{n}} \subst{\abs{\widetilde{x}}{R'}}{x}}}
	\]
%
	\noi The last transition implies
	\[
		\mhorel{\Gamma}{\Delta_1}{\newsp{\widetilde{m_1}}{P_1 \Par \appl{X}{\widetilde{n}} \subst{\abs{\widetilde{x}}{R}}{x}}}
		{\hby{\ell_1}}
		{\Delta_1'}{}{\newsp{\widetilde{m_1'}}{P_1 \Par \appl{X}{\widetilde{n}} \subst{\abs{\widetilde{x}}{R'}}{x}}}
	\]
%
	\noi and
%
	\[
		\mhorel{\Gamma}{\Delta_1}{\newsp{\widetilde{m_1}}{P_1 \Par \appl{X}{\widetilde{n}} \subst{\abs{\widetilde{x}}{R}}{x}}}
		{\hby{\ell_2}}
		{\Delta_1'}{}{\newsp{\widetilde{m_1'}}{P_1' \Par \appl{X}{\widetilde{n}} \subst{\abs{\widetilde{x}}{R}}{x}}}
	\]
%
	\noi We also get that:
%
	\[
		\begin{array}{crll}
			&\Gamma; \es; \Delta_1 & \proves &
<<<<<<< HEAD
			\newsp{\widetilde{m_1}}{P_1 \Par \appl{X}{\widetilde{n}} \subst{\auxtr{t}}{X}} \\
			\hby{\ell_1} & \Delta_1'' & \proves & \newsp{\widetilde{m_1}}{P_1' \Par \appl{X}{\widetilde{n}} \subst{\auxtr{t}}{X}} \\
			\hby{\bactinp{t}{\auxtr{t'}}} 
			&\Delta_1'& \proves & \newsp{\widetilde{m_1'}}{P_1' \Par \appl{X}{\widetilde{n}} \subst{\auxtr{t'}}{X}}
=======
			\newsp{\widetilde{m_1}}{P_1 \Par \appl{X}{\widetilde{n}} \subst{\auxtr{t}}{x}} \\
			\hby{\ell_1} & \Delta_1'' & \proves & \newsp{\widetilde{m_1}}{P_1' \Par \appl{x}{\widetilde{n}} \subst{\auxtr{t}}{x}} \\
			\hby{\bactinp{t}{\abs{\widetilde{x}}{\binp{t'}{y} \appl{y}{\widetilde{x}}}}} 
			&\Delta_1'& \proves & \newsp{\widetilde{m_1'}}{P_1' \Par \appl{x}{\widetilde{n}} \subst{\abs{\widetilde{x}}{\binp{t'}{y} \appl{y}{\widetilde{x}}}}{x}}
>>>>>>> 439e2867a26627c1ed6fa910f20f06f9788eeabe
		\end{array}
	\]
%
%	\begin{eqnarray*}
%		\Gamma; \es; \Delta_1 &\hby{\bactinp{t}{\auxtr{t'}}}& \Delta_1' \proves
%		\newsp{\widetilde{m_1}}{P_1 \Par \appl{X}{\widetilde{n}} \subst{\auxtr{t}}{X}} \\
%		&\hby{\ell_1}& \newsp{\widetilde{m_1}}{P_1' \Par \appl{X}{\widetilde{n}} \subst{\auxtr{t}}{X}} \\
%		&\hby{\bactinp{t}{\auxtr{t'}}}& 
%		\newsp{\widetilde{m_1'}}{P_1' \Par \appl{X}{\widetilde{n}} \subst{\auxtr{t'}}{X}}
%		\label{lem:subst_equiv5}
%	\end{eqnarray*}
%
	\noi and $t'$ a fresh name. From the freshness of $t$ and the fact that $X$ is linear in $Q_2$
	it has to be the case that:
%
	\[
		\begin{array}{crll}
			& \Gamma; \es; \Delta_2'&  \proves &
			\newsp{\widetilde{m_2'}}{Q_1 \Par Q_2 \subst{\auxtr{t}}{x}}\\
			\Hby{\ell_2}& & &
<<<<<<< HEAD
			\newsp{\widetilde{m_2'}}{Q_1'' \Par Q_3 \Par \appl{X}{\widetilde{m}} \subst{\auxtr{t}}{X}} \\
			\hby{\bactinp{t}{\auxtr{t'}}}
			&\Delta_2''& \proves& \newsp{\widetilde{m_2'}}{Q_1' \Par \appl{X}{\widetilde{m}} \subst{\auxtr{t'}}{X}} \\
=======
			\newsp{\widetilde{m_2'}}{Q_1'' \Par Q_3 \Par \appl{x}{\widetilde{m}} \subst{\auxtr{t}}{x}} \\
			\hby{\bactinp{t}{\abs{\widetilde{x}}{\binp{t'}{y} \appl{y}{\widetilde{x}}}}}
			&\Delta_2''& \proves& \newsp{\widetilde{m_2'}}{Q_1' \Par \appl{x}{\widetilde{m}} \subst{\abs{\widetilde{x}}{\binp{t'}{y} \appl{y}{\widetilde{x}}}}{x}} \\
>>>>>>> 439e2867a26627c1ed6fa910f20f06f9788eeabe
		\end{array}
	\]
%
%	\[
%		\begin{array}{rcll}
%			\Gamma; \es; \Delta_2' &\Hby{\bactinp{t}{\auxtr{t'}}}& \Delta_2'' \proves &
%			\newsp{\widetilde{m_2'}}{Q_1 \Par Q_2 \subst{\auxtr{t}}{X}}\\
%			&\Hby{\ell_2}& &
%			\newsp{\widetilde{m_2'}}{Q_1'' \Par Q_3 \Par \appl{X}{\widetilde{m}} \subst{\auxtr{t}}{X}} \\
%			&\hby{\bactinp{t}{\auxtr{t'}}}& &
%			\newsp{\widetilde{m_2'}}{Q_1' \Par \appl{X}{\widetilde{m}} \subst{\auxtr{t'}}{X}} \\
%		\end{array}
%	\]
%
	\noi From here the proof is similar with the previous case.
	%\qed
\end{comment}
\end{proof}

\noindent
We can generalise the result of the linear process substitution lemma
to prove process substitution (\lemref{app:lem:proc_subst}).
Intuitively, we can subsequently apply linear process substitution
to achieve process substitution.


%%%%%%%%%%%%%%%%%%%%%%%%%%%%%%%%%%%%%%
%	Process Substitution
%%%%%%%%%%%%%%%%%%%%%%%%%%%%%%%%%%%%%%

\begin{lemma}[Process Substitution]\rm
	\label{app:lem:proc_subst}
	If 
%
	\begin{enumerate}
		\item	$\horel{\Gamma}{\Delta_1'}{P \subst{\auxtr{t}}{x}}{\wb}{\Delta_2}{Q \subst{\auxtr{t}}{x}}$
			for some fresh $t$.

		\item	$\horel{\Gamma}{\Delta_1''}{P \subst{\omapchar{U}}{x}}{\wb}{\Delta_2''}{Q \subst{\omapchar{U}}{x}}$
			for some $U$.
	\end{enumerate}
%
	then $\forall R$ such that $\fv{R} = \widetilde{x}$
\[
	\horel{\Gamma}{\Delta_1}{P \subst{\abs{\widetilde{x}}{R}}{x}}{\wb}{\Delta_2}{Q \subst{\abs{\widetilde{x}}{R}}{x}}
\]
\end{lemma}


\begin{proof}
	\noi We define a closure $\Re$ using the normal form of $P$ and $Q$

	\[
		\begin{array}{rcll}
			\Re &=& \set{\horel{\Gamma}{\Delta_1}{\newsp{\widetilde{m_1}}{P_1 \subst{\abs{\widetilde{x}}{R}}{x} \Par P_2 \subst{\abs{\widetilde{x}}{R}}{x}}}{,}{\Delta_2}{\newsp{\widetilde{m_2}}{Q_1 \subst{\abs{\widetilde{x}}{R}}{x} \Par Q_2 \subst{\abs{\widetilde{x}}{R}}{x}}} \setbar \\
%			\mathcal{S} &=& \set{\horel{\Gamma}{\Delta_1}{\newsp{\widetilde{m_1}}{P_1 \subst{\abs{\widetilde{x}}{R}}{x} \Par P_2 \subst{\abs{\widetilde{x}}{R}}{x}}}{,}{\Delta_2}{\newsp{\widetilde{m_2}}{Q_1 \subst{\abs{\widetilde{x}}{R}}{x} \Par Q_2 \subst{\abs{\widetilde{x}}{R}}{x}}} \setbar \\
			&& \qquad \forall R \textrm{ such that } \fv{R} = \widetilde{x},\\
			&& \qquad \textrm{ for fresh } t,
			\mhorel{\Gamma}{\Delta_1'}
			{\newsp{\widetilde{m_1}}{P_1 \subst{\auxtr{t}}{x} \Par P_2 \subst{\auxtr{t}}{x}}}
			{\wb}
			{\Delta_2'}{}{\newsp{\widetilde{m_2}}{Q_1 \subst{\auxtr{t}}{x} \Par Q_2 \subst{\auxtr{t}}{x}}}\\
			&& \qquad \textrm{ for some } U, 
			\mhorel{\Gamma}{\Delta_1''}
			{\newsp{\widetilde{m_1}}{P_1 \subst{\omapchar{U}}{x} \Par P_2 \subst{\omapchar{U}}{x}}}
			{\wb}
			{\Delta_2''}{}{\newsp{\widetilde{m_2}}{Q_1 \subst{\omapchar{U}}{x} \Par Q_2 \subst{\omapchar{U}}{x}}} \\
			&&}
		\end{array}
	\]
%
%	\begin{eqnarray*}
%		\mathcal{S} &=& \set{\horel{\Gamma}{\Delta_1}{\newsp{\widetilde{m_1}}{P_1 \subst{\abs{\widetilde{x}}{R}}{X} \Par P_2 \subst{\abs{\widetilde{x}}{R}}{X}}}{,}{\Delta_2}{\newsp{\widetilde{m_2}}{Q_1 \subst{\abs{\widetilde{x}}{R}}{X} \Par Q_2 \subst{\abs{\widetilde{x}}{R}}{X}}} \setbar \\
%		&& \quad \forall R \textrm{ such that } \fv{R} = \widetilde{x},\\
%		&& \quad \textrm{ for fresh } t, \Gamma; \es; \Delta_1' \wb \Delta_2' \proves \\
%		&& \quad \newsp{\widetilde{m_1}}{P_1 \subst{\auxtr{t}}{X} \Par P_2 \subst{\auxtr{t}}{X}} \wb \newsp{\widetilde{m_2}}{Q_1 \subst{\auxtr{t}}{X} \Par Q_2 \subst{\auxtr{t}}{X}}, \\
%		&& \quad \textrm{ for some } U, \Gamma; \es; \Delta_1'' \wb \Delta_2'' \proves \\
%		&& \quad \newsp{\widetilde{m_1}}{P_1 \subst{\abs{\widetilde{x}}{\map{U}^x}}{X} \Par P_2 \subst{\abs{\widetilde{x}}{\map{U}^x}}{X}} \wb \newsp{\widetilde{m_2}}{Q_1 \subst{\abs{\widetilde{x}}{\map{U}^x}}{X} \Par Q_2 \subst{\abs{\widetilde{x}}{\map{U}^x}}{X}} \\
%		&&}
%	\end{eqnarray*}
%
	\noi We show that $\Re$ is a bisimulation up to \betatran (\lemref{lem:tau_inert}).

	\noi - Case: $P_2 \not= \appl{x}{\widetilde{n}}$ for some $\widetilde{n}$.
%
	\nhorel	{\Gamma}{\Delta_1}{\newsp{\widetilde{m_1}}{P_1 \subst{\abs{\widetilde{x}}{R}}{x} \Par P_2 \subst{\abs{\widetilde{x}}{R}}{x}}}
		{\hby{\ell_1}}
		{\Delta_1'}{\newsp{\widetilde{m_1'}}{P_1 \subst{\abs{\widetilde{x}}{R}}{x} \Par P_2' \subst{\abs{\widetilde{x}}{R}}{x}}}
		{lem:subst_equiv11}
%
	\noi The case is similar to the first case of \lemref{lem:subst_equiv}.

	\noi - Case: $P_2 = \appl{x}{\widetilde{n}}$ for some $\widetilde{n}$.
%
\[
	\mhorel	{\Gamma}{\Delta_1}{\newsp{\widetilde{m_1}}{P_1 \subst{\abs{\widetilde{x}}{R}}{x} \Par \appl{x}{\widetilde{n}} \subst{\abs{\widetilde{x}}{R}}{x}}}
		{\hby{\btau}}{\Delta_1'}{}{\newsp{\widetilde{m_1'}}{P_1 \subst{\abs{\widetilde{x}}{R}}{x} \Par R \subst{\widetilde{n}}{\widetilde{x}}}} %\appl{x}{\widetilde{n}} \subst{\abs{\widetilde{x}}{R'}}{x}}}
%		{\hby{\ell_1}}{\Delta_1'}{}{\newsp{\widetilde{m_1'}}{P_1 \subst{\abs{\widetilde{x}}{R}}{x} \Par \appl{x}{\widetilde{n}} \subst{\abs{\widetilde{x}}{R'}}{x}}}
\]
%
	\noi From the latter transition we get that:
%
	\nhorel{\Gamma}{\Delta_1}
	{\newsp{\widetilde{m_1}}{P_1 \subst{\auxtr{t}}{x} \Par \appl{x}{\widetilde{n}} \subst{\auxtr{t}}{x}}}
	{\hby{\btau} \hby{\bactinp{t}{\auxtr{t'}}}}
	{\Delta_1'}{\newsp{\widetilde{m_1}'}{P_1 \subst{\auxtr{t}}{x} \Par \appl{y}{\widetilde{n}} \subst{\auxtr{t'}}{y}}}
%	{\hby{\bactinp{t}{\auxtr{t'}}}}
%	{\Delta_1'}{\newsp{\widetilde{m_1}'}{P_1 \subst{\auxtr{t}}{x} \Par \appl{y}{\widetilde{n}} \subst{\abs{\widetilde{x}}{\binp{t'}{y} \appl{y}{\widetilde{x}}}}{y}}}
	{cor:subst_equiv5}
%
%	\begin{eqnarray}
%		\Gamma; \es; \Delta_1 &\hby{\bactinp{t}{\auxtr{t'}}}& \Delta_1' \proves
%		\newsp{\widetilde{m_1}}{P_1 \subst{\auxtr{t}}{X} \Par \appl{X}{\widetilde{n}} \subst{\auxtr{t}}{X}}
%		\nonumber \\
%		&\hby{\bactinp{t}{\auxtr{t'}}}& 
%		\newsp{\widetilde{m_1}'}{P_1 \subst{\auxtr{t}}{X} \Par \appl{Y}{\widetilde{n}} \subst{\auxtr{t'}}{Y}}
%		\label{cor:subst_equiv5}
%	\end{eqnarray}
%
	\noi and $t'$ a fresh name. From the freshness of $t$
	and the determinacy of the application transition
	it has to be the case that:
%
	\[
		\begin{array}{crll}
			& \Gamma; \es; \Delta_2'& \proves &
			\newsp{\widetilde{m_2}'}{Q_1 \subst{\auxtr{t}}{x} \Par Q_2 \subst{\auxtr{t}}{x}}\\
			\Hby{} && &
			\newsp{\widetilde{m_2}'}{Q_1' \subst{\auxtr{t}}{x} \Par Q_2' \subst{\auxtr{t}}{x} \Par \\
			&&& \qquad \qquad \appl{x}{\widetilde{m}} \subst{\auxtr{t}}{x}} \\
			\hby{\btau} \hby{\bactinp{t}{\auxtr{t'}}}
			& \Delta_2''& \proves& \newsp{\widetilde{m_2}'}{(Q_1' \Par Q_2') \subst{\auxtr{t}}{x} \Par \appl{y}{\widetilde{m}} \subst{\auxtr{t'}}{y}}
%			\hby{\bactinp{t}{\abs{\widetilde{x}}{\binp{t'}{y} \appl{y}{\widetilde{x}}}}}
%			& \Delta_2''& \proves& \newsp{\widetilde{m_2}'}{(Q_1' \Par Q_2') \subst{\auxtr{t}}{x} \Par \appl{y}{\widetilde{m}} \subst{\abs{\widetilde{x}}{\binp{t'}{y} \appl{y}{\widetilde{x}}}}{y}}
		\end{array}
	\]
%
%	\[
%		\begin{array}{rcll}
%			\Gamma; \es; \Delta_2' &\Hby{\bactinp{t}{\auxtr{t'}}}& \Delta_2'' \proves &
%			\newsp{\widetilde{m_2}'}{Q_1 \subst{\auxtr{t}}{X} \Par Q_2 \subst{\auxtr{t}}{X}}\\
%			&\Hby{}& &
%			\newsp{\widetilde{m_2}'}{Q_1' \subst{\auxtr{t}}{X} \Par Q_2' \subst{\auxtr{t}}{X} \Par \appl{X}{\widetilde{m}} \subst{\auxtr{t}}{X}} \\
%			&\hby{\bactinp{t}{\auxtr{t'}}}& &
%			\newsp{\widetilde{m_2}'}{(Q_1' \Par Q_2') \subst{\auxtr{t}}{X} \Par \appl{Y}{\widetilde{m}} \subst{\auxtr{t'}}{Y}}
%		\end{array}
%	\]
%
	Let $Q_3$ such that
	\[
		\mhorel{\Gamma}{\Delta}{\newsp{\widetilde{m_2}'}{Q_1 \Par Q_3} \subst{\auxtr{t}}{x} \subst{\auxtr{t'}}{y}}
		{\Hby{}}
		{\Delta'}{}{\newsp{\widetilde{m_2}'}{(Q_1' \Par Q_2') \subst{\auxtr{t}}{x} \Par \appl{y}{\widetilde{m}} \subst{\auxtr{t'}}{y}}}
%		\mhorel{\Gamma}{\Delta}{\newsp{\widetilde{m_2}'}{Q_1 \Par Q_3} \subst{\auxtr{t}}{x} \subst{\abs{\widetilde{x}}{\binp{t'}{y} \appl{y}{\widetilde{x}}}}{y}}
%		{\Hby{}}
%		{\Delta'}{}{\newsp{\widetilde{m_2}'}{(Q_1' \Par Q_2') \subst{\auxtr{t}}{x} \Par \appl{y}{\widetilde{m}} \subst{\abs{\widetilde{x}}{\binp{t'}{y} \appl{y}{\widetilde{x}}}}{y}}}
	\]
%
	\noi From \lemref{lem:subst_equiv} we get that $\forall R$ with $\fv{R} = \widetilde{x}$
%
	\begin{eqnarray*}
		\mhorel{\Gamma}{\Delta_1'''}{\newsp{\widetilde{m_1}'}{P_1 \subst{\auxtr{t}}{x} \Par \appl{y}{\widetilde{n}} \subst{\abs{\widetilde{x}}{R}}{y}}}
		{\wb}
		{\Delta'}{}{\newsp{\widetilde{m_2}'}{(Q_1 \Par Q_3) \subst{\auxtr{t}}{x} \subst{\abs{\widetilde{x}}{R}}{y}}}
%		\mhorel{\Gamma}{\Delta_1'''}{\newsp{\widetilde{m_1}'}{P_1 \subst{\auxtr{t}}{X} \Par \appl{y}{\widetilde{n}} \subst{\abs{\widetilde{x}}{R}}{y}}}
%		{\wb}
%		{\Delta'}{}{\newsp{\widetilde{m_2}'}{(Q_1 \Par Q_3) \subst{\auxtr{t}}{X} \subst{\abs{\widetilde{x}}{R}}{y}}}
	\end{eqnarray*}
%
	\noi From~(\ref{lem:subst_equiv11}) we get that
\[
	\mhorel{\Gamma}{\Delta'}{\newsp{\widetilde{m_1}'}{(Q_1 \Par Q_3) \subst{\auxtr{t}}{x} \subst{\abs{\widetilde{x}}{R}}{y}}}
	{\Hby{} \hby{\btau}}
	{\Delta''}{}{\newsp{\widetilde{m_2}'}{(Q_1' \Par Q_2') \subst{\auxtr{t}}{x} \Par R \subst{\widetilde{m}}{\widetilde{x}}}}%\appl{Y}{\widetilde{m}} \subst{\auxtr{t'}}{Y}}}
%	{\Hby{\ell_2}}
%	{\Delta''}{}{\newsp{\widetilde{m_2}'}{(Q_1' \Par Q_2') \subst{\auxtr{t}}{x} \Par R'}}%\appl{Y}{\widetilde{m}} \subst{\abs{\widetilde{x}}{\binp{t'}{Y} \appl{Y}{\widetilde{x}}}}{Y}}}
\]
	\noi and from the definition of $\Re$
%	\noi From Lemma~\ref{lem:subst_equiv} we get that $\forall R$ with $\fv{R} = x$
%
	\[
		\mhorel{\Gamma}{\Delta_1''}{\newsp{\widetilde{m_1}'}{P_1 \subst{\abs{\widetilde{x}}{R}}{x} \Par \appl{y}{\widetilde{n}} \subst{\abs{\widetilde{x}}{R}}{y}}}
		{\hby{\btau}\ \Re\ \stackrel{\btau}{\longleftarrow}}
		{\Delta_2''}{}{\newsp{\widetilde{m_2}'}{(Q_1' \Par Q_2') \subst{\abs{\widetilde{x}}{R}}{x} \Par \appl{y}{\widetilde{m}} \subst{\abs{\widetilde{x}}{R}}{y}}}
%		\mhorel{\Gamma}{\Delta_1''}{\newsp{\widetilde{m_1}'}{P_1 \subst{\abs{\widetilde{x}}{R}}{x} \Par \appl{Y}{\widetilde{n}} \subst{\abs{\widetilde{x}}{R'}}{y}}}
%		{\ \mathcal{S}\ }
%		{\Delta_2''}{}{\newsp{\widetilde{m_2}'}{(Q_1' \Par Q_2') \subst{\abs{\widetilde{x}}{R}}{x} \Par \appl{y}{\widetilde{m}} \subst{\abs{\widetilde{x}}{R'}}{y}}}
	\]
	\noi as required.
%	\noi From here we apply Lemma~\ref{lem:subst_equiv} for each substituting instance of
%	abstraction $\abs{\widetilde{x}}{R}$ to complete the proof.
	%\qed
\end{proof}

%%%%%%%%%%%%%%%%%%%%%%%%%%%%%%%%%%%%%%%%%%%%%%%%%%%%%%%%%
%  WB IS WBC
%%%%%%%%%%%%%%%%%%%%%%%%%%%%%%%%%%%%%%%%%%%%%%%%%%%%%%%%%

\begin{lemma}\rm
	\label{lem:wb_is_wbc}
	$\wb\ \subseteq\ \wbc$.
\end{lemma}

\begin{proof}
	Let
	\[
		\horel{\Gamma}{\Delta_1}{P_1}{\wb}{\Delta_2}{Q_1}
	\]
	The proof is divided on cases on the label $\ell$ for the transition:
%
	\begin{eqnarray}
		\horel{\Gamma}{\Delta_1}{P_1}{\hby{\ell}}{\Delta_1'}{P_2}
		\label{lem:wb_is_wbc1}
	\end{eqnarray}
%
	\noi - Case: $\ell \notin \set{ \news{\widetilde{m_1}} \bactout{n}{\abs{\widetilde{x}}{P}},  \news{\widetilde{m_1}'} \bactout{n}{\widetilde{m_1}}, \bactinp{n}{\abs{\widetilde{x}}{P}} }$

	\noi For the latter $\ell$ and transition~in (\ref{lem:wb_is_wbc1}) we conclude that:	
%
	\[
		\horel{\Gamma}{\Delta_2}{Q_1}{\Hby{\ell}}{\Delta_2'}{Q_2}
	\]
%
	\noi and
%
	\[
		\horel{\Gamma}{\Delta_1'}{P_2}{\wb}{\Delta_2'}{Q_2}
	\]
%
	The above premise and conclusion coincides with defining cases for $\ell$ in $\wbc$.

	\noi - Case: $\ell = \bactinp{n}{\abs{\widetilde{x}}{P}}$

	\noi Transition in~(\ref{lem:wb_is_wbc1}) concludes:
%
\[
	\begin{array}{l}
		\horel{\Gamma}{\Delta_1}{P_1}{\hby{\bactinp{n}{\abs{\widetilde{x}}{\mapchar{U}{\widetilde{x}}}}}}{\Delta_1'}{P_2 \subst{\abs{\widetilde{x}}{\mapchar{U}{\widetilde{x}}}}{x}}\\
		\horel{\Gamma}{\Delta_1}{P_1}{\hby{\bactinp{n}{\auxtr{t}}}}{\Delta_1''}{P_2 \subst{\auxtr{t}}{x}}
	\end{array}
\]
%
	\noi The last two transitions imply:
%
\[
	\begin{array}{l}
		\horel{\Gamma}{\Delta_2}{Q_1}{\Hby{\bactinp{n}{\abs{\widetilde{x}}{\mapchar{U}{\widetilde{x}}}}}}{\Delta_2'}{Q_2 \subst{\abs{\widetilde{x}}{\mapchar{U}{\widetilde{x}}}}{x}}\\
		\horel{\Gamma}{\Delta_2}{Q_1}{\Hby{\bactinp{n}{\auxtr{t}}}}{\Delta_2''}{Q_2 \subst{\auxtr{t}}{x}}
	\end{array}
\]
%
	\noi and
%
\[
	\begin{array}{l}
		\horel{\Gamma}{\Delta_1'}{P_2 \subst{\abs{\widetilde{x}}{\mapchar{U}{\widetilde{x}}}}{x}}{\wb}{\Delta_2'}{Q_2 \subst{\abs{\widetilde{x}}{\mapchar{U}{\widetilde{x}}}}{x}}\\
		\horel{\Gamma}{\Delta_1''}{P_2 \subst{\auxtr{t}}{x}}{\wb}{\Delta_2''}{Q_2 \subst{\auxtr{t}}{x}}
	\end{array}
\]
%
	\noi To conclude from (\ref{app:lem:proc_subst}) that
	$\forall R$ with $\fv{R} = \widetilde{x}$
%
\[
	\horel{\Gamma}{\Delta_1'}{P_2 \subst{\abs{\widetilde{x}}{R}}{x}}{\wb}{\Delta_2'}{Q_2 \subst{\abs{\widetilde{x}}{R}}{x}}
\]
%
	\noi as required.

	\noi - Case: $\ell = \news{\widetilde{m_1}} \bactout{n}{\abs{\widetilde{x}}{P}}$

	\noi From transition~(\ref{lem:wb_is_wbc1}) we conclude:
%
\[
	\horel{\Gamma}{\Delta_2}{Q_1}{\Hby{\news{\widetilde{m_2}} \bactout{n}{\abs{\widetilde{x}}{Q}}}}{\Delta_2'}{Q_2}
\]
%
	\noi and for fresh $t$
%
\[
	\mhorel	{\Gamma}{\Delta_1'}{\newsp{\widetilde{m_1}}{P_2 \Par \binp{t}{x} \newsp{s}{\appl{x}{s} \Par \bout{\dual{s}}{\abs{\widetilde{x}}{P}} \inact}}}
		{\wb}
		{\Delta_2'}{}{\newsp{\widetilde{m_2}}{Q_2 \Par \binp{t}{x} \newsp{s}{\appl{x}{s} \Par \bout{\dual{s}}{\abs{\widetilde{x}}{Q}} \inact}}}
\]
%
	\noi From the  previous case we can conclude that $\forall R$ with $\fpv{R} = \set{x}$:
%
\[
	\begin{array}{rl}
		\Gamma; \es; &\Delta_1' \proves \newsp{\widetilde{m_1}}{P_2 \Par \binp{t}{x} \newsp{s}{\appl{x}{s} \Par \bout{\dual{s}}{\abs{\widetilde{x}}{P}} \inact}} \\
		\by{\bactinp{t}{\abs{z}{\binp{z}{x} R}}}& \newsp{\widetilde{m_1}}{P_2 \Par \newsp{s}{\binp{s}{x} R \Par \bout{\dual{s}}{\abs{\widetilde{x}}{P}} \inact}}\\
		\by{\tau} \quad &\Delta_1'' \proves \newsp{\widetilde{m_1}}{P_2 \Par  R \subst{\abs{\widetilde{x}}{P}}{x}}
	\end{array}
\]
%
	\noi and
%
\[
	\begin{array}{rl}
		\Gamma; \es; &\Delta_2' \proves \newsp{\widetilde{m_2}}{Q_2 \Par \binp{t}{x} \newsp{s}{\appl{x}{s} \Par \bout{\dual{s}}{\abs{\widetilde{x}}{Q}} \inact}} \\
		\by{\bactinp{t}{\abs{z}{\binp{z}{x} R}}} &\newsp{\widetilde{m_2}}{Q_2 \Par \newsp{s}{\binp{s}{x} R \Par \bout{\dual{s}}{{\widetilde{x}}{Q}} \inact}}\\
		\by{\tau} &\Delta_2'' \proves \newsp{\widetilde{m_2}}{Q_2 \Par  R \subst{\abs{\widetilde{x}}{Q}}{x}}
	\end{array}
\]
%
	\noi and furthermore it is easy to see that $\forall R$ with $\fpv{R} = X$:
%
\[
	\horel{\Gamma}{\Delta_1''}{\newsp{\widetilde{m_1}}{P_2 \Par  R \subst{\abs{\widetilde{x}}{P}}{x}}}{\wb}{\Delta_2}{\newsp{\widetilde{m_2}}{Q_2 \Par R \subst{\abs{\widetilde{x}}{Q}}{x}}}
\]
%
	\noi as required by the definition of $\wbc$.

	\noi - Case: $\ell = \news{\widetilde{m_1}'} \bactout{n}{\widetilde{m_1}}$

	The last case shares a similar argumentation with the previous case.
	%\qed
\end{proof}


%%%%%%%%%%%%%%%%%%%%%%%%%%%%%%%%%%%%%%%%%%%%%%%%%%%%%%%%%
%  WB IS CONG
%%%%%%%%%%%%%%%%%%%%%%%%%%%%%%%%%%%%%%%%%%%%%%%%%%%%%%%%%

\begin{lemma}
	\label{lem:wbc_is_cong}
	$\wbc \subseteq \cong$.
\end{lemma}


\begin{proof}
	\noi We prove that $\wbc$ satisfies the defining properties of $\cong$. Let
%
	\[
		\horel{\Gamma}{\Delta_1}{P_1}{\wbc}{\Delta_2}{P_2}
	\]
%
	{\bf Reduction Closed:}
%
	\[
		\horel{\Gamma}{\Delta_1}{P_1}{\by{}}{\Delta_1'}{P_1'}
	\]
%
	\noi implies that 
	$\exists P_2'$ such that 
%
	\begin{eqnarray*}
		\horel{\Gamma}{\Delta_2}{P_2}{\By{}}{\Delta_2'}{P_2'}\\
		\horel{\Gamma}{\Delta_1}{P_1'}{\wbc}{\Delta_2'}{P_2'}
	\end{eqnarray*}
%
	\noi Same argument hold for the symmetric case, thus $\wbc$ is reduction closed.

	\noi {\bf Barb Preservation:}
%
	\begin{eqnarray*}
		\Gamma; \emptyset; \Delta_1 \proves P_1 \hastype \Proc \barb{n}
	\end{eqnarray*}
%
	implies that
	\begin{eqnarray*}
		P &\cong& \newsp{\widetilde{m}}{\bout{n}{V_1} P_3 \Par P_4}\\
		\dual{n} &\notin& \Delta_1
	\end{eqnarray*}
%
	\noi From the definition of $\wbc$ we get that
%
\[
	\horel	{\Gamma}{\Delta_1}{\newsp{\widetilde{m}}{\bout{n}{V_1} P_3 \Par P_4}}
		{\by{\news{s_1} \bactout{m}{V_1}}}
		{\Delta_1'}
		{\newsp{\widetilde{m'}}{P_3 \Par P_4}}
\]
%
	\noi implies
%
	\begin{eqnarray*}
		\horel{\Gamma}{\Delta_2}{P_2}{\By{\news{m_2} \bactout{n}{V_2}}}{\Delta_2'}{P_2'}\\
	\end{eqnarray*}
%
	\noi From the last result we get that
%
	\begin{eqnarray*}
		\Gamma; \emptyset; \Delta_2 \proves P_2 \hastype \Proc \Barb{n}
	\end{eqnarray*}
%
	\noi as required.

	\noi {\bf Congruence:}

	\noi The congruence property requires that we check that $\wbc$
	is preserved under any context.
	The most interesting context case is parallel composition.

	\noi We construct a congruence relation. Let
	\[
	\begin{array}{rcl}
		\mathcal{S} &=&	\set{
				(\Gamma; \emptyset; \Delta_1 \cat \Delta_3 \proves \newsp{\widetilde{n_1}}{P_1 \Par R} \hastype \Proc,
				\Gamma; \emptyset; \Delta_2 \cat \Delta_3 \proves \newsp{\widetilde{n_2}}{P_2 \Par R})
				\setbar \\
		& &		\horel{\Gamma}{\Delta_1}{P_1}{\wbc}{\Delta_2}{P_2}, \forall \Gamma; \emptyset; \Delta_3 \proves R \hastype \Proc\\
		& &}
	\end{array}
	\]
	\noi We need to show that the above congruence is a bisimulation.
	To show that $\mathcal{S}$ is a bisimulation we do a case analysis on the structure
	of the $\by{\ell}$ transition.

	%%%%%%%%%%%%%%%
	% Case 1
	%%%%%%%%%%%%%%%

	\noi - Case: 
	\[
		\horel{\Gamma}{\Delta_1 \cat \Delta_3}{\newsp{\widetilde{n_1}}{P_1 \Par R}}{\by{\ell}}{\Delta_1' \cat \Delta_3}{\newsp{\widetilde{n_1'}}{P_1' \Par R}}
	\]

	\noi The case is divided into three subcases:

	\noi Subcase i: $\ell \notin \set{\news{\widetilde{m}} \bactout{n}{\abs{\widetilde{x}}{Q}}, \news{\widetilde{mm_1}} \bactout{n}{\widetilde{m_1}}}$

	\noi From the definition of typed transition we get:
	\[
		\horel{\Gamma}{\Delta_1}{P_1}{\by{\ell}}{\Delta_1'}{P_1'}
	\]
	\noi which implies that
%
	\begin{eqnarray}
		\horel{\Gamma}{\Delta_1}{P_2}{\By{\ell}}{\Delta_2'}{P_2'}
		\label{lem:wbc_is_cong1}\\
		\horel{\Gamma}{\Delta_1'}{P_1'}{\wbc}{\Delta_2''}{P_2'}
		\label{lem:wbc_is_cong2}
	\end{eqnarray}
%
	\noi From transition in~(\ref{lem:wbc_is_cong1}) we conclude that 
	\[
		\horel{\Gamma}{\Delta_2 \cat \Delta_3}{\newsp{\widetilde{n_2}}{P_2 \Par R}}{\By{\ell}}{\Delta_2' \cat \Delta_3}{\newsp{\widetilde{n_2}'}{P_2' \Par R}}
	\]
%
	\noi Furthermore from~(\ref{lem:wbc_is_cong2}) and the definition of $\mathcal{S}$ we conlude that
	\[
		\horel{\Gamma}{\Delta_1' \cat \Delta_3}{\newsp{\widetilde{n_1}'}{P_1' \Par R}}{\ \mathcal{S}\ }{\Delta_2' \cat \Delta_3}{\newsp{\widetilde{n_2}'}{P_2' \Par R}}
	\]

	\noi Subcase ii: $\ell = \news{\widetilde{m_1}} \bactout{n}{\abs{\widetilde{x}}{Q_1}}$

	\noi From the definition of typed transition we get
	\[
		\horel{\Gamma}{\Delta_1}{P_1}{\by{\news{\widetilde{m_1}} \bactout{n}{\abs{\widetilde{x}}{Q_1}}}}{\Delta_1'}{P_1'}
	\]
	\noi which implies that
%
	\begin{eqnarray}
		&& \horel{\Gamma}{\Delta_1}{P_2}{\By{\news{\widetilde{m_2}} \bactout{n}{\abs{\widetilde{x}}{Q_2}}}}{\Delta_2'}{P_2'}
		\label{lem:wbc_is_cong3} \\
		&&\forall Q, \set{x} \in \fpv{Q} \nonumber \\
%		\forall s'
		&& \horel{\Gamma}{\Delta_1''}{\newsp{\widetilde{n_1}''}{P_1' \Par Q \subst{\abs{\widetilde{x}}{Q_1}}{x}}}
		{\ \wbc\ }
		{\Delta_2''}{\newsp{\widetilde{n_2}''}{P_2' \Par Q \subst{\abs{\widetilde{x}}{Q_2}}{x}}}
		\label{lem:wbc_is_cong4}
	\end{eqnarray}
%
	\noi From transition~(\ref{lem:wbc_is_cong3}) conclude that 
	\[
		\horel{\Gamma}{\Delta_2 \cat \Delta_3}{\newsp{\widetilde{n_2}}{P_2 \Par R}}{\By{\news{\widetilde{m_2}} \bactout{n}{\abs{\widetilde{x}}{Q_2}}}}{\Delta_2' \cat \Delta_3}{\newsp{\widetilde{n_2}'}{P_2' \Par R}}
	\]
%
	\noi Furthermore from~(\ref{lem:wbc_is_cong4}) we conlude that $\forall Q$ with $\set{x} = \fpv{Q}$
%
	\[
		\horel{\Gamma}{\Delta_1'' \cat \Delta_3}{\newsp{\widetilde{n_1}''}{P_1' \Par Q \subst{(\widetilde{x}) Q_1}{x} \Par R}}{\ \mathcal{S}\ }{\Delta_2'' \cat \Delta_3}{\newsp{\widetilde{n_2}''}{P_2' \Par Q \subst{\abs{\widetilde{x}}{Q_2}}{x} \Par R}}
	\]
%
	- Subcase iii: $\ell = \news{\widetilde{mm_1}} \bactout{n}{\widetilde{m_1}}$

	\noi From the definition of typed transition we get that
	\[
		\horel{\Gamma}{\Delta_1}{P_1}{\by{\news{\widetilde{mm_1}} \bactout{n}{\widetilde{m_1}}}}{\Delta_1'}{P_1'}
	\]
	\noi which implies that $\exists P_2', s_2$ such that
%
	\begin{eqnarray}
		&& \horel{\Gamma}{\Delta_1}{P_2}{\By{\news{\widetilde{mm_2}} \bactout{n}{\widetilde{m_2}}}}{\Delta_2'}{P_2'}
		\label{lem:wbc_is_cong5}\\
		&&\forall Q, x = \fn{Q}, \nonumber \\%  &&
		&& \horel{\Gamma}{\Delta_1''}{\newsp{\widetilde{n_1}}{P_1' \Par Q \subst{\widetilde{m_1}}{\widetilde{x}}}}{\ \wbc\ }{\Delta_2''}{\newsp{\widetilde{n_2}}{P_2' \Par Q \subst{\widetilde{m_2}}{\widetilde{x}}}}
		\label{lem:wbc_is_cong6}
	\end{eqnarray}
%
	\noi From transition~(\ref{lem:wbc_is_cong5}) conclude that 
	\[
		\horel{\Gamma}{\Delta_2 \cat \Delta_3}{\newsp{\widetilde{n_2}'}{P_2 \Par R}}{\By{\news{\widetilde{mm_2}} \bactout{n}{\widetilde{m_2}}}}{\Delta_2' \cat \Delta_3}{\newsp{\widetilde{n_2}'''}{P_2' \Par R}}
	\]
%
	\noi Furthermore from~(\ref{lem:wbc_is_cong6}) we conlude that $\forall Q, x = \fn{Q}$
%
	\[
		\horel{\Gamma}{\Delta_1'' \cat \Delta_3}{\newsp{\widetilde{n_1}''}{P_1' \Par Q \subst{\widetilde{m_1}}{\widetilde{x}} \Par R}}{\ \mathcal{S}\ }{\Delta_2'' \cat \Delta_3}{\newsp{\widetilde{n_2}''}{P_2' \Par Q \subst{\widetilde{m_2}}{\widetilde{x}} \Par R}}
	\]
%
	%%%%%%%%%%%%%%%
	% Case 2
	%%%%%%%%%%%%%%%

	\noi - Case:
%
	\[
		\horel{\Gamma}{\Delta_1 \cat \Delta_3}{\newsp{\widetilde{m_1}}{P_1 \Par R}}{\by{\ell}}{\Delta_1 \cat \Delta_3'}{\newsp{\widetilde{m_1}'}{P_1 \Par R'}}
	\]
%
	\noi This case is divided into three subcases:

	\noi Subcase i: $\ell \notin \set{\news{\widetilde{m}} \bactout{n}{\abs{\widetilde{x}}{Q}}, \news{\widetilde{mm_1}} \bactout{n}{\widetilde{m_1}}}$

	\noi From the LTS we get that:
	\[
		\horel{\Gamma}{\Delta_3}{R}{\by{\ell}}{\Delta_3'}{R'}
	\]
%
	\noi Which in turn implies
	\begin{eqnarray*}
		\horel{\Gamma}{\Delta_2 \cat \Delta_3}{\newsp{\widetilde{m_2}}{P_2 \Par R}}{\by{\ell}}{\Delta_2 \cat \Delta_3'}{\newsp{\widetilde{m_2}'}{P_2 \Par R'}}
	\end{eqnarray*}
%
	\noi From the definition of $\mathcal{S}$ we conclude that
	\[
		\horel{\Gamma}{\Delta_1 \cat \Delta_3'}{\newsp{\widetilde{m_1}'}{P_1 \Par R'}}{\ \mathcal{S}\ }{\Delta_2 \cat \Delta_3''}{\newsp{\widetilde{m_2}'}{P_2 \Par R'}}
	\]
	\noi as required.

	\noi Subcase ii: $\ell = \news{\widetilde{m_1}} \bactout{n}{\abs{\widetilde{x}}{Q}}$

	\noi From the LTS we get that:
	\begin{eqnarray}
		& &	\horel{\Gamma}{\Delta_3}{R}{\by{\ell}}{\Delta_3'}{R'}
			\label{lem:wbc_is_cong7}\\
		& & 	\forall R_1, \set{x} = \fpv{R_1},
			\nonumber\\
%		\forall s'
		& &	\Gamma; \emptyset; \Delta_3'' \proves \newsp{\widetilde{m}'}{R' \Par R_1 \subst{\abs{\widetilde{x}}{Q}}{x}} \hastype \Proc
			\label{lem:wbc_is_cong8}
	\end{eqnarray}
%
	\noi From~(\ref{lem:wbc_is_cong7}) we get that
	\[
		\horel{\Gamma}{\Delta_2 \cat \Delta_3}{\newsp{\widetilde{m_2}'}{P_2 \Par R}}{\by{\ell}}{\Delta_2 \cat \Delta_3'}{\newsp{\widetilde{m_2}}{P_2 \Par R'}}
	\]
	\noi Furthermore from~(\ref{lem:wbc_is_cong8}) and the definition of $\mathcal{S}$ we conclude that
	$\forall R_1$ with $\set{x} \in \fpv{R_1}$
	\[
		\horel{\Gamma}{\Delta_1 \cat \Delta_3''}{\newsp{\widetilde{m_1}}{P_1 \Par \newsp{\widetilde{m}'}{R' \Par R_1 \subst{\abs{\widetilde{x}}{Q}}{x}}}}
		{\ \mathcal{S}\ }
		{\Delta_2 \cup \Delta_3''}{\newsp{\widetilde{m_2}}{P_2 \Par \newsp{\widetilde{m}'}{R' \Par R_1 \subst{\abs{\widetilde{x}}{Q}}{x}}}}
	\]
	\noi as required.

	\noi Subcase iii: $\ell = \news{\widetilde{mm}} \bactout{n}{\widetilde{m}}$

	\noi From the typed LTS we get that:
	\begin{eqnarray}
		& &	\horel{\Gamma}{\Delta_3}{R}{\by{\ell}}{\Delta_3'}{R'}
			\label{lem:wbc_is_cong9} \\
		& &	\forall Q, \widetilde{x} = \fn{Q}, \nonumber\\
		& &	\Gamma; \emptyset; \Delta_3'' \proves \newsp{\widetilde{m}'}{R' \Par Q \subst{\widetilde{m}}{\widetilde{x}}} \hastype \Proc
			\label{lem:wbc_is_cong10}
	\end{eqnarray}
%
	\noi From~(\ref{lem:wbc_is_cong9}), we obtain that
	\[
		\horel{\Gamma}{\Delta_2 \cat \Delta_3}{\newsp{\widetilde{m_2}}{P_2 \Par R}}{\by{\ell}}{\Delta_2 \cat \Delta_3'}{\newsp{\widetilde{m_2}}{P_2 \Par R'}}
	\]
	\noi Furthermore from~(\ref{lem:wbc_is_cong10}) and the definition of $\mathcal{S}$ we conclude that
	$\forall Q, \widetilde{x} = \fn{Q}$
	\[
		\horel{\Gamma}{\Delta_1 \cat \Delta_3''}{\newsp{\widetilde{m_1}}{P_1 \Par \newsp{\widetilde{m}}{R' \Par Q \subst{\widetilde{m}'}{\widetilde{x}}}}}
		{\ \mathcal{S}\ }
		{\Delta_2 \cat \Delta_3''}{\newsp{\widetilde{m_2}}{P_2 \Par \newsp{\widetilde{m}'}{R' \Par Q \subst{\widetilde{m}}{\widetilde{x}}}}}
	\]
	\noi as required.


	%%%%%%%%%%%%%%%
	% Case 3
	%%%%%%%%%%%%%%%

	\noi - Case:
	\[
		\horel{\Gamma}{\Delta_1 \cat \Delta_3}{\newsp{\widetilde{m_1}}{P_1 \Par R}}
		{\by{}}
		{\Delta_1' \cat \Delta_3'}{\newsp{\widetilde{m_1}'}{P_1' \Par R'}}
	\]

	\noi This case is divided into three subcases:

	\noi Subcase i: $\horel{\Gamma}{\Delta_1}{P_1}{\by{\ell}}{\Delta_1'}{P_1'}$
	and $\ell \notin \set{\news{\widetilde{m}} \bactout{n}{\abs{\widetilde{x}}{Q}}, \news{\widetilde{mm_1}} \bactout{n}{\widetilde{m_1}}}$ implies
%
	\begin{eqnarray}
		\horel{\Gamma}{\Delta_3}{R}{\by{\dual{\ell}}}{\Delta_3}{R'}
		\label{lem:wbc_is_cong11} \\
		\horel{\Gamma}{\Delta_2}{P_2}{\By{\hat{\ell}}}{\Delta_2'}{P_2'}
		\label{lem:wbc_is_cong12}\\
		\horel{\Gamma}{\Delta_1'}{P_1'}{\wbc}{\Delta_2'}{P_2'}
		\label{lem:wbc_is_cong13}
	\end{eqnarray}
%
	\noi From~(\ref{lem:wbc_is_cong11}) and~(\ref{lem:wbc_is_cong12}) we get
	\[
		\horel{\Gamma}{\Delta_2 \cat \Delta_3}{\newsp{\widetilde{m_2}}{P_2 \Par R}}{\By{}}{\Delta_2' \cat \Delta_3'}{\newsp{\widetilde{m_2}'}{P_2' \Par R'}}
	\]
%
	\noi From~(\ref{lem:wbc_is_cong13}) and the definition of ($\mathcal{S}$) we get that
	\[
		\horel{\Gamma}{\Delta_1' \cat \Delta_3'}{\newsp{\widetilde{m_1}'}{P_1' \Par R'}}{\ \mathcal{S}\ }{\Delta_2' \cat \Delta_3}{\newsp{\widetilde{m_2}'}{P_2' \Par R'}}
	\]
	\noi as required.

	\noi Subcase ii:
	$\horel{\Gamma}{\Delta_1}{P_1}{\by{\news{\widetilde{m_1}} \bactout{n}{\abs{\widetilde{x}}{Q_1}}}}{\Delta_1'}{P_1'}$
	implies
%
	\begin{eqnarray}
		& & \horel{\Gamma}{\Delta_3}{R}{\by{\bactinp{n}{\abs{\widetilde{x}} {Q_1}}}}{\Delta_3'}{R' \subst{\abs{\widetilde{x}}{Q_1}}{x}}
		\label{lem:wbc_is_cong14}\\
		& & \horel{\Gamma}{\Delta_1 \cat \Delta_3}{\newsp{\widetilde{m_1}}{P_1 \Par R}}{\by{}}{\Delta_1' \cat \Delta_3'}{\newsp{\widetilde{m_1}''}{P_1' \Par R' \subst{\abs{\widetilde{x}}{Q_1}}{x}}}
		\nonumber \\
		& & \horel{\Gamma}{\Delta_2}{P_2}{\By{\news{\widetilde{m_2}} \bactout{n}{\abs{\widetilde{x}}{Q_2}}}}{\Delta_2'}{P_2'}
		\label{lem:wbc_is_cong15}\\
		& & \forall Q, \set{x} = \fpv{Q}, \nonumber \\
		& & \horel{\Gamma}{\Delta_1''}{\newsp{\widetilde{m_1}'}{P_1' \Par Q \subst{\abs{\widetilde{x}}{Q_1}}{x}}}{\ \wbc\ }{\Delta_2''}{\newsp{\widetilde{m_2}'}{P_2' \Par Q \subst{\abs{\widetilde{x}}{Q_2}}{x}}}
		\label{lem:wbc_is_cong16}
	\end{eqnarray}
%
	From~(\ref{lem:wbc_is_cong14}) and the Substitution Lemma~(\lemref{l:subst}) we obtain that
	\[
		\horel{\Gamma}{\Delta_3}{R}{\by{\bactinp{n}{\abs{\widetilde{x}} {Q_2}}}}{\Delta_3''}{R' \subst{\abs{\widetilde{x}}{Q_2}}{x}}
	\]
	%\dk{(prove that $\forall V, R \by{\bactinp{s}{V}} R'\subst{V}{x}$)}
	\noi to combine with~(\ref{lem:wbc_is_cong15}) and get
	\[
		\horel{\Gamma}{\Delta_2 \cat \Delta_3}{\newsp{\widetilde{m_2}}{P_2 \Par R}}{\By{}}{\Delta_2' \cat \Delta_3''}{\newsp{\widetilde{m_2}''}{P_2' \Par R' \subst{\abs{\widetilde{x}}{Q_2}}{X}}}
	\]
%
	\noi In result in~(\ref{lem:wbc_is_cong16}), set $Q$ as $R'$ to obtain:
%
%	\noi From~\ref{lem:wbc_is_cong16} and the definition of $\mathcal{S}$ we get that
	\[
		\horel{\Gamma}{\Delta_1''}{\newsp{\widetilde{m_1}'}{P_1' \Par R' \subst{\abs{\widetilde{x}}{Q_1}}{x}}}
		{\ \mathcal{S}\ \Delta_2''}
		{\newsp{\widetilde{m_2}'}{P_2' \Par R' \subst{\abs{\widetilde{x}}{Q_2}}{x}}}
	\]

	\noi Subcase iii:
	$\horel{\Gamma}{\Delta_1}{P_1}{\by{\news{\widetilde{mm_1}} \bactout{n}{\widetilde{m_1}}}}{\Delta_1'}{P_1'}$
%
	\begin{eqnarray}
		& & \horel{\Gamma}{\Delta_3}{R}{\by{\bactinp{n}{\widetilde{m_1}}}}{\Delta_3'}{R' \subst{\widetilde{m_1}}{\widetilde{x}}}
		\label{lem:wbc_is_cong24}\\
		& & \horel{\Gamma}{\Delta_1 \cup \Delta_3}{\newsp{\widetilde{m_1}}{P_1 \Par R}}{\by{}}{\Delta_1' \cup \Delta_3'}{\newsp{\widetilde{m_1}''}{P_1' \Par R' \subst{s_1}{x}}}
		\nonumber \\
		& & \horel{\Gamma}{\Delta_2}{P_2}{\By{\news{\widetilde{mm_2}} \bactout{n}{\widetilde{m_2}}}}{\Delta_2'}{P_2'}
		\label{lem:wbc_is_cong25}\\
		& & \forall Q, \set{x} = \fpv{Q}, \nonumber \\
		& & \horel{\Gamma}{\Delta_1''}{\newsp{\widetilde{m_1}'}{P_1' \Par Q \subst{\widetilde{m_1}}{\widetilde{x}}}}
		{\ \wbc\ }
		{\Delta_2''}{\newsp{\widetilde{m_2}'}{P_2' \Par Q \subst{\widetilde{m_2}}{\widetilde{x}}}}
		\label{lem:wbc_is_cong26}
	\end{eqnarray}
%
	From~(\ref{lem:wbc_is_cong24}) and the Substitution Lemma~(\lemref{l:subst}) we get that
	\[
		\horel{\Gamma}{\Delta_3}{R}{\by{\bactinp{n}{\widetilde{m_2}}}}{\Delta_3''}{R' \subst{\widetilde{m_2}}{\widetilde{x}}}
	\]
	%\dk{(prove that $\forall V, R \by{\bactinp{s}{V}} R'\subst{V}{x}$)}
	\noi to combine with~(\ref{lem:wbc_is_cong25}) and get
	\[
		\horel{\Gamma}{\Delta_2 \cat \Delta_3}{\newsp{\widetilde{m_2}}{P_2 \Par R}}
		{\By{}}
		{\Delta_2' \cat \Delta_3''}{\newsp{\widetilde{m_2}''}{P_2' \Par R' \subst{\widetilde{m_2}}{\widetilde{x}}}}
	\]
%
	\noi Set $Q$ as $R'$ in result in (\ref{lem:wbc_is_cong26}) to obtain
%
%	\noi From~\ref{lem:wbc_is_cong16} and the definition of $\mathcal{S}$ we get that
	\[
		\horel{\Gamma}{\Delta_1''}{\newsp{\widetilde{m_1}'}{P_1' \Par R' \subst{\widetilde{m_1}}{\widetilde{x}}}}
		{\ \mathcal{S}\ }
		{\Delta_2''}{\newsp{\widetilde{m_2}'}{P_2' \Par R' \subst{\widetilde{m_2}}{\widetilde{x}}}}
	\]
	%\qed
\end{proof}

%%%%%%%%%%%%%%%%%%%%%%%%%%%%%%%%%%%%%%%%%%%%%%%%%%%%%%%%%
%  CONG IS WB
%%%%%%%%%%%%%%%%%%%%%%%%%%%%%%%%%%%%%%%%%%%%%%%%%%%%%%%%%

We prove the result $\cong \subseteq \wb$ following
the technique developed in~\cite{Hennessy07} and
refined for session types in~\cite{KYHH2015,KY2015}.

\begin{definition}[Definibility]\myrm
	Let $\Gamma; \emptyset; \Delta_1 \proves P \hastype \Proc$.
	A visible action $\ell$ is \emph{definable} whenever
	there exists (testing) process
	$\Gamma; \emptyset; \Delta_2 \proves T\lrangle{\ell, \suc} \hastype \Proc$
	with $\suc$ fresh name % and $N$ a set of names.
	such that:
%
	\begin{itemize}
		\item	If $\horel{\Gamma}{\Delta_1}{P}{\by{\ell}}{\Delta_1'}{P'}$ and
			$\ell \in \set{\bactsel{n}{\ell}, \bactbra{n}{\ell}, \bactinp{n}{\widetilde{m}}, \bactinp{n}{\abs{\widetilde{x}}{Q}}}$
			then:
%
		\[
			P \Par T\lrangle{\ell, \suc} \red P' \Par \bout{\suc}{\dual{m}} \inact \textrm{ and }
			\Gamma; \emptyset; \Delta_1' \cat \Delta_2' \proves P' \Par \bout{\suc}{\dual{m}} \inact
		\]
%
 		\item	If $\horel{\Gamma}{\Delta_1}{P}{\by{\news{\widetilde{m}}\bactout{n}{V}}}{\Delta_1'}{P'}$,
			$t$ fresh
			and $\widetilde{m}' \subseteq \widetilde{m}$
			then:
%
			\begin{eqnarray*}
				& & P \Par T\lrangle{\news{\widetilde{m}}\bactout{n}{V}, \suc} \red
				\newsp{\widetilde{m}}{P' \Par \hotrigger{t}{x}{s}{V} \Par \bout{\suc}{\dual{n}, \widetilde{m}'} \inact}\\
				& & \Gamma; \emptyset; \Delta_1' \cat \Delta_2' \proves
				\newsp{\widetilde{m}}{P' \Par \hotrigger{t}{x}{s}{V} \Par  \bout{\suc}{\dual{n}, \widetilde{m}'} \inact} \hastype \Proc\\
			\end{eqnarray*}

		\item	Let $\ell \in \set{\bactsel{n}{\ell}, \bactbra{n}{\ell}, \bactinp{n}{\widetilde{m}}, \bactinp{n}{(\widetilde{x}) Q}}$.
			If $P \Par T\lrangle{\ell, \suc} \red Q$ with			
			$\Gamma; \emptyset; \Delta \proves Q \hastype \Proc \barb{\suc}$ then 
			$\horel{\Gamma}{\Delta_1}{P}{\By{\ell}}{\Delta_1'}{P'}$
			and $Q \scong P' \Par \bout{\suc}{\dual{n}} \inact$.

		\item	If $P \Par T\lrangle{\news{\widetilde{m}}\bactout{n}{V}, \suc} \red Q$
			with $\Gamma; \emptyset; \Delta \proves Q \hastype \Proc \barb{\suc}$ then
			$\horel{\Gamma}{\Delta_1}{P}{\By{\news{\widetilde{m}}\bactout{n}{V}}}{\Delta_1'}{P'}$
			and $Q \scong \newsp{\widetilde{m}}{P' \Par \hotrigger{t}{x}{s}{V} \Par \bout{\suc}{\dual{n}, \widetilde{m}'} \inact}$
			with $t$ fresh and $\widetilde{m}' \subseteq \widetilde{m}$.
	\end{itemize}	
%
\end{definition}

We first show that every visible action $\ell$ is {\em definable}.

\begin{lemma}[Definibility]
	\label{lem:definibility}
	Every action $\ell$ is definable.
\end{lemma}

\begin{proof}
	\noi We define $T\lrangle{\ell, \suc}$:
%
	\begin{itemize}
		\item	$T\lrangle{\bactinp{n}{V}, \suc} = \bout{\dual{n}}{V} \bout{\suc}{\dual{n}} \inact$.

		\item	$T\lrangle{\bactbra{n}{l}, \suc} = \bsel{\dual{n}}{l} \bout{\suc}{\dual{n}} \inact$.

%		\item	$T\lrangle{\bactinp{n}{\abs{\widetilde{x}} Q}, \suc} = \bout{\dual{n}}{\abs{\widetilde{x}}{Q}} \bout{\suc}{\dual{n}} \inact$.

		\item	$T\lrangle{\news{\widetilde{m}'} \bactout{n}{\widetilde{m}}, \suc} = \binp{\dual{n}}{\widetilde{x}} (\hotrigger{t}{x}{s}{\widetilde{x}} \Par \bout{\suc}{\dual{n}, \widetilde{m}''} \inact)$
			with $\widetilde{m}'' \subseteq \widetilde{m}'$.

		\item	$T\lrangle{\news{\widetilde{m}} \bactout{n}{\abs{\widetilde{x}}{Q}}, \suc} = \binp{\dual{n}}{y} (\hotrigger{t}{x}{s}{\abs{\widetilde{x}}{(\appl{y}{\widetilde{x}}})} \Par \bout{\suc}{\dual{n}, \widetilde{m}'} \inact)$ with $\widetilde{m}' \subseteq \widetilde{m}$.

		\item	$T\lrangle{\bactsel{n}{l}, \suc} = \bbra{\dual{n}}{l: \bout{\suc}{\dual{n}} \inact), l_i: \newsp{a}{\binp{a}{y} \bout{\suc}{\dual{n}} \inact}}_{i \in I}$.
	\end{itemize}

	\noi Assuming a process 
	\[
		\Gamma; \emptyset; \Delta \proves P \hastype \Proc
	\] 
	\noi it is straightforward to verify that $\forall \ell$, $\ell$ is definable.
	%\qed
\end{proof}

\begin{lemma}[Extrusion]\rm
	\label{lem:extrusion}
	If 
	\[
		\horel{\Gamma}{\Delta_1'}{\newsp{\widetilde{m_1}'}{P \Par \bout{\suc}{\dual{n}, \widetilde{m_1}''} \inact}}{\cong}{\Delta_2}{\newsp{\widetilde{m_2}'}{Q \Par \bout{\suc}{\dual{n}, \widetilde{m_2}''} \inact}}
	\]
	then
	\[
		\horel{\Gamma}{\Delta_1}{P}{\cong}{\Delta_2}{Q}
	\]
\end{lemma}

\begin{proof}
	\noi Let
%
	\begin{eqnarray*}
		\mathcal{S}	&=&
					\set{\Gamma; \es; \Delta_1 \proves P \hastype \Proc, \Gamma; \es; \Delta_2 \proves Q \hastype \Proc \setbar \\
				& &	\horel{\Gamma}{\Delta_1'}{\newsp{\widetilde{m_1}'}{P \Par \bout{\suc}{\dual{n}, \widetilde{m_1}''} \inact}}
					{\cong}{\Delta_2}{\newsp{\widetilde{m_2}'}{Q \Par \bout{\suc}{\dual{n}, \widetilde{m_2}''} \inact}} \\
		&&}
	\end{eqnarray*}
%
	\noi We show that $\mathcal{S}$ is a congruence.

	\noi {\bf Reduction closed:}

	\noi $P \red P'$
	implies
	$\newsp{\widetilde{m_1}'}{P \Par \bout{\suc}{\dual{n}, \widetilde{m_1}''} \inact} \red \newsp{\widetilde{m_1}'}{P' \Par \bout{\suc}{\dual{n}, \widetilde{m_1}''} \inact}$
	implies from the freshness of $\suc$
	$\newsp{\widetilde{m_1}'}{P \Par \bout{\suc}{\dual{n}, \widetilde{m_1}''} \inact} \Red \newsp{\widetilde{m_1}'}{Q' \Par \bout{\suc}{\dual{n}, \widetilde{m_2}''} \inact}$.
	which implies
	$Q \Red Q'$ as required.

	\noi {\bf Barb Preserving:}

	\noi Let $\Gamma; \es; \Delta_1 \proves P \barb{s}$. We analyse two cases.

	\noi - Case: $s \not= n$.

	\noi $\Gamma; \es; \Delta_1 \proves P \barb{s}$
	implies
%
	\[
		\Gamma; \es; \Delta_1' \proves \newsp{\widetilde{m_1}'}{P \Par \bout{\suc}{\dual{n}, \widetilde{m_1}''} \inact} \barb{s}
	\]
%
	\noi implies
	$\Gamma; \es; \Delta_2' \proves \newsp{\widetilde{m_2}'}{Q \Par \bout{\suc}{\dual{n}, \widetilde{m_2}''} \inact} \Barb{s}$
	implies from the freshness of $\suc$ that
	$\Gamma; \es; \Delta_2 \proves Q \Barb{s}$ as required.

	\noi - Case: $s = n$ and $\Gamma; \es; \Delta_1 \proves P \barb{n}$

	\noi We compose with $\binp{\dual{\suc}}{x, \widetilde{y}} T\lrangle{\ell, \suc'}$ with $\subj{\ell} = x$ to get
%
	\[
		\Gamma; \es; \Delta_1' \proves \newsp{\widetilde{m_1}'}{P \Par \bout{\suc}{\dual{n}, \widetilde{m_1}''} \inact} \Par \binp{\dual{\suc}}{x, \widetilde{y}} T\lrangle{\ell, \suc'}
	\]
%
	\noi Which implies from the fact that $\Gamma; \es; \Delta_1 \proves P \barb{n}$ that
%
	\[
		\newsp{\widetilde{m_1}'}{P \Par \bout{\suc}{\dual{n}, \widetilde{m_1}''} \inact} \Par \binp{\dual{\suc}}{x, \widetilde{y}} T\lrangle{\ell, \suc'} \Red 
		\newsp{\widetilde{m_1}'}{P' \Par \bout{\suc'}{\dual{n}, \widetilde{m_1}''} \inact}
	\]
%
	\noi and furthermore
%
	\[
		\newsp{\widetilde{m_2}'}{Q \Par \bout{\suc}{\dual{n}, \widetilde{m_2}''} \inact} \Par \binp{\dual{\suc}}{x, \widetilde{y}} T\lrangle{\ell, \suc'} \Red 
		\newsp{\widetilde{m_2}'}{Q' \Par \bout{\suc'}{\dual{n}, \widetilde{m_2}''} \inact}
	\]
%
	\noi The last reduction implies that
	$\Gamma; \es; \Delta_2 \proves Q \Barb{n}$ as required.

	\noi {\bf Congruence:}
	The key case of congruence is parallel composition.
	We define relation $\mathcal{C}$ as
%
	\begin{eqnarray*}
		\mathcal{C} &=&	\set{ \Gamma; \es; \Delta_1 \cat \Delta_3 \proves P \Par R \hastype \Proc,  \Gamma; \es; \Delta_2 \cat \Delta_3 \proves Q \Par R \hastype \Proc \setbar \\
		& &	\forall R,\\
		& &	\horel{\Gamma}{\Delta_1'}{\newsp{\widetilde{m_1}'}{P \Par \bout{\suc}{\dual{n}, \widetilde{m_1}''} \inact}}{\cong}{\Delta_2'}{\newsp{\widetilde{m_2}'}{Q \Par \bout{\suc}{\dual{n}, \widetilde{m_2}''} \inact}}}
	\end{eqnarray*}
%
	\noi We show that $\mathcal{C}$ is a congruence.

	\noi We distinguish two cases:

	\noi - Case: $\dual{n}, \widetilde{m_1}'', \widetilde{m_2}'' \notin \fn{R}$	

	\noi From the definition of $\mathcal{C}$ we can deduce that $\forall R$:
%
	\[
		\horel{\Gamma}{\Delta_1''}{\newsp{\widetilde{m_1}'}{P \Par \bout{\suc}{\dual{n}, \widetilde{m_1}''} \inact} \Par R}{\cong}{\Delta_2''}{\newsp{\widetilde{m_2}'}{Q \Par \bout{\suc}{\dual{n}, \widetilde{m_2}''} \inact} \Par R}
	\]
	\noi The conclusion is then trivial.

	\noi - Case: $\widetilde{s} = \set{\dual{n}, \widetilde{m_1}''} \cap \set{\dual{n}, \widetilde{m_2}''} \in \fn{R}$

	\noi From the definition of $\mathcal{C}$ we can deduce that $\forall R^{y_1}$ such that $R = R^{y_1}\subst{\widetilde{s}}{\widetilde{y_1}}$
	and $\suc'$ fresh and $\set{\widetilde{y}} = \set{\widetilde{y_1}} \cup \set{\widetilde{y_2}}$:
%
	\[
		\mhorel{\Gamma}{\Delta_1''}{\newsp{\widetilde{m_1}'}{P \Par \bout{\suc}{\dual{n}, \widetilde{m_1}''} \inact} \Par \binp{\dual{\suc}}{\widetilde{y}} (R^{y_1} \Par \bout{\suc'}{\widetilde{y_2}} \inact)}
		{\cong}
		{\Delta_2''}{}{\newsp{\widetilde{m_2}'}{Q \Par \bout{\suc}{\dual{n}, \widetilde{m_2}''} \inact} \Par \binp{\dual{\suc}}{\widetilde{y}} (R^{y_1} \Par \bout{\suc'}{\widetilde{y_2}} \inact)}
	\]
%
	\noi Applying reduction closeness to the above pair we get:
%
	\[
		\horel{\Gamma}{\Delta_1''}{\newsp{\widetilde{m_1}'}{P \Par R \Par \bout{\suc'}{\widetilde{s_2}} \inact}}{\cong}{\Delta_2''}{\newsp{\widetilde{m_2}'}{Q \Par R \Par \bout{\suc'}{\widetilde{s_2}} \inact}}
	\]
%
	\noi The conclusion then follows.
	%\qed
\end{proof}


\begin{lemma}\rm
	\label{lem:cong_is_wb}
	$\cong \subseteq \wb$.
\end{lemma}

\begin{proof}
	\noi Let
		$\horel{\Gamma}{\Delta_1}{P_1}{\cong}{\Delta_2}{P_2}$.
	We distinguish two cases:

%% Case tau
	\noi - Case:
	\[
		\horel{\Gamma}{\Delta_1}{P_1}{\by{\tau}}{\Delta_1'}{P_1'}
	\]
	\noi The result follows the reduction closeness property of $\cong$ since
	\[
		\horel{\Gamma}{\Delta_2}{P_2}{\By{\tau}}{\Delta_2'}{P_2'}
	\]
	\noi and
	\[
		\horel{\Gamma}{\Delta_1'}{P_1'}{\cong}{\Delta_2'}{P_2'}
	\]

%% Case ell
	\noi - Case:
%
	\begin{eqnarray}
		\horel{\Gamma}{\Delta_1}{P_1}{\by{\ell}}{\Delta_1'}{P_1'}
		\label{lem:cong_is_wb1}
	\end{eqnarray}
%
	\noi We choose test $T\lrangle{\ell, \suc}$ to get
%
	\begin{eqnarray}
		\horel{\Gamma}{\Delta_1 \cat \Delta_3}{P_1 \Par T\lrangle{\ell, \suc}}{\cong}{\Delta_2 \cat \Delta_3}{P_2 \Par T\lrangle{\ell, \suc}}
		\label{lem:cong_is_wb2}
	\end{eqnarray}
%
	\noi From this point we distinguish three subcases:

%% Subcase i
	\noi Subcase i: $\ell \in \set{\bactinp{n}{\widetilde{m}}, \bactinp{n}{\abs{\widetilde{x}}{Q}}, \bactsel{n}{l}, \bactbra{n}{l}}$

	\noi By reducing~(\ref{lem:cong_is_wb1}), we obtain
%
	\begin{eqnarray*}
		&& P_1 \Par T\lrangle{\ell, \suc} \red P_1' \Par \bout{\suc}{\dual{n}} \inact \\
		&& \Gamma; \es; \Delta_1' \cat \Delta_3' \proves P_1' \Par \bout{\suc}{\dual{n}} \inact \barb{\suc}
	\end{eqnarray*}
%
	\noi implies from~(\ref{lem:cong_is_wb2})
%
	\begin{eqnarray*}
		&& \Gamma; \es; \Delta_2 \cat \Delta_3 \proves P_2 \Par T\lrangle{\ell, \suc} \Barb{\suc}
	\end{eqnarray*}
%
	\noi implies from Lemma~\ref{lem:definibility},
%
	\begin{eqnarray*}
		&& \horel{\Gamma}{\Delta_2}{P_2}{\By{\ell}}{\Delta_2'}{P_2'}\\
		&& P_2 \Par T \lrangle{\ell, \suc} \Red P_2' \Par \bout{\suc}{\dual{n}} \inact
	\end{eqnarray*}
%
	\noi and
%
	\[
		\horel{\Gamma}{\Delta_1' \cat \Delta_3'}{P_1' \Par \bout{\suc}{\dual{n}}}{\cong}{\Delta_2' \cat \Delta_3'}{P_2' \Par \bout{\suc}{\dual{n}} \inact}
	\]
	We then apply \lemref{lem:extrusion} to get
%
	\[
		\horel{\Gamma}{\Delta_1'}{P_1'}{\cong}{\Delta_2'}{P_2'}
	\]
%
	\noi as required.

%% Subcase ii
	\noi Subcase ii: $\ell = \news{\widetilde{m_1}} \bactout{n}{\abs{\widetilde{x}}{Q_1}}$

	\noi Note that $T\lrangle{\news{\widetilde{m_1}} \bactout{n}{(\widetilde{x}) Q_1}, \suc} = T\lrangle{\news{\widetilde{m_2}} \bactout{n}{\abs{\widetilde{x}}{Q_2}}, \suc}$

	\noi Transition~in (\ref{lem:cong_is_wb1}) becomes
%
	\begin{eqnarray}
		\horel{\Gamma}{\Delta_1}{P_1}{\by{\news{\widetilde{m_1}} \bactout{n}{\abs{\widetilde{x}}{Q_1}}}}{\Delta_1'}{P_1'}
		\label{lem:cong_is_wb3}
	\end{eqnarray}
%
	\noi If we use the test process $T\lrangle{\news{\widetilde{m_1}} \bactout{n}{(\widetilde{x}) Q_1}, \suc}$ we reduce to:%~\ref{lem:cong_is_wb1} we get
%
	\begin{eqnarray*}
		&& P_1 \Par T\lrangle{\news{\widetilde{m_1}} \bactout{n}{\abs{\widetilde{x}}{Q_1}}, \suc}
		\red
		\newsp{m_1}{P_1' \Par \hotrigger{t}{x}{s}{\abs{\widetilde{x}}{Q_1}}} \Par \bout{\suc}{\dual{n}, \widetilde{m_1}'} \inact \\
		&& \Gamma; \es; \Delta_1' \cat \Delta_3' \proves \newsp{m_1}{P_1' \Par \hotrigger{t}{x}{s}{\abs{\widetilde{x}}{Q_1}}} \Par \bout{\suc}{\dual{n}, \widetilde{m_1}'} \inact \barb{\suc}
	\end{eqnarray*}
%
	\noi implies from~(\ref{lem:cong_is_wb2})
%
	\[
		\Gamma; \es; \Delta_2 \cat \Delta_3 \proves P_2 \Par T\lrangle{\news{\widetilde{m_2}} \bactout{n}{\abs{\widetilde{x}}{Q_2}}, \suc} \Barb{\suc}
	\]
%
	\noi implies from \lemref{lem:definibility}
%
	\begin{eqnarray}
		&& \horel{\Gamma}{\Delta_2}{P_2}{\By{\news{\widetilde{m_2}} \bactout{n}{\abs{\widetilde{x}}{Q_2}}}}{\Delta_2'}{P_2'}
		\label{lem:cong_is_wb4}\\
		&& P_2 \Par T \lrangle{\ell, \suc} \Red \newsp{m_2}{P_2' \Par \hotrigger{t}{x}{s}{\abs{\widetilde{x}}{Q_2}}} \Par \bout{\suc}{\dual{n}, \widetilde{m_2}'} \inact \nonumber
	\end{eqnarray}
%
	\noi and
%
	\[
		\mhorel{\Gamma}{\Delta_1' \cat \Delta_3'}{\newsp{m_1}{P_1' \Par \hotrigger{t}{x}{s}{\abs{\widetilde{x}}{Q_1}}} \Par \bout{\suc}{\dual{n}, \widetilde{m_1}'} \inact}
		{\cong}
		{\Delta_2' \cat \Delta_3'}{}{\newsp{m_2}{P_2' \Par \hotrigger{t}{x}{s}{\abs{\widetilde{x}}{Q_2}}} \Par \bout{\suc}{\dual{n}, \widetilde{m_2}'} \inact}
	\]
%
	\noi We then apply \lemref{lem:extrusion} to get
%
	\[
		\mhorel{\Gamma}{\Delta_1'}{\newsp{m_1}{P_1' \Par \hotrigger{t}{x}{s}{\abs{\widetilde{x}}{Q_1}}}}
		{\cong}
		{\Delta_2'}{}{\newsp{m_2}{P_2' \Par \hotrigger{t}{x}{s}{\abs{\widetilde{x}}{Q_2}}}}
	\]
%
	\noi as required.

	\noi -Case: $\ell = \news{\widetilde{s}} \bactout{n}{\widetilde{m}}$

	\noi Follows similar arguments as the previous case.
	%\qed
\end{proof}

%%%%%%%%%%%%%%%%%%%%%%%%%%%%%%%%%%%%%%%%%%%%%%%%%%%%%%%%%%%%%%
% Proof of the main theorem
%%%%%%%%%%%%%%%%%%%%%%%%%%%%%%%%%%%%%%%%%%%%%%%%%%%%%%%%%%%%%%

\begin{theorem}[Concidence]\label{app:thm:coincidence} We have:
	\begin{enumerate}
		\item	$\wbc\ =\ \wb$.
		\item	$\wbc\ =\ \cong$.
	\end{enumerate}
\end{theorem}

\begin{proof}
	\noi	\lemref{lem:wb_eq_wbf} proves $\wb\ =\ \wbf$.
		\lemref{lem:cong_is_wb} proves $\cong\ \subseteq\ \wb$.
		\lemref{lem:wb_is_wbc} proves $\wb\ \subseteq\ \wbc$.
		\lemref{lem:wbc_is_cong} proves $\wbc\ \subseteq\ \cong$.
 From the above results, we conclude $\cong\ \subseteq\ \wb\ =\ \wbf\ \subseteq\ \wbc\ \subseteq\ \cong$. 
	%\qed
\end{proof}



