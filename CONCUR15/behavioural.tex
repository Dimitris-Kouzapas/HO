% !TEX root = main.tex
\noi We develop a theory for observational equivalences over
session typed \HOp processes. The theory follows the principles
laid in our previous works~\cite{KYHH2015,KY2015}.
We introduce 
%three different bisimulations 
characteristic bisimulation
and prove
\jpc{that}
%all of them coincide 
it coincides
with reduction-closed,
barbed congruence (\thmref{the:coincidence}).


\subsection{Labelled Transition System for Processes}\label{ss:lts}
\myparagraph{Labels.}
\noi We define an (early) labelled transition system (LTS) over
untyped processes. 
Later on, using the \emph{environmental} transition semantics, 
we can define a typed transition relation to formalise 
how a typed process interacts with a typed observer. 
%process in its environment. 
The interaction is defined on action $\ell$:
\begin{center}
\begin{tabular}{l}
	$\ell	\bnfis   \tau 
		\bnfbar	\bactinp{n}{V} 
		\bnfbar	\news{\widetilde{m}} \bactout{n}{V}
		\bnfbar	\bactsel{n}{l} 
		\bnfbar	\bactbra{n}{l} $
\end{tabular}
\end{center}
\noi 
Label $\tau$ defines internal actions.
Action $\news{\widetilde{m}} \bactout{n}{V}$ denotes the sending of value $V$
over channel $n$ with
a possible empty set of names $\widetilde{m}$ being restricted 
(we may write $\bactout{n}{V}$ when $\widetilde{m}$ is empty).
%and 
%$\news{\widetilde{m}} \bactout{n}{\AT{V}{U}}$
%when the type of $V$ is~$U$.
Dually, the action for value reception is 
$\bactinp{n}{V}$.
Actions for select
and branch on
a label~$l$ are denoted $\bactsel{n}{l}$ and $\bactbra{n}{l}$, resp.
We write $\fn{\ell}$ and $\bn{\ell}$ to denote the
 sets of free/bound names in $\ell$, resp.
%and set $\mathsf{n}(\ell)=\bn{\ell}\cup \fn{\ell}$. 
Given $\ell \neq \tau$, we write 
$\subj{\ell}$
to denote the \emph{subject} of $\ell$.


\emph{Dual actions} %, defined below, 
occur on subjects that are dual between them and carry the same
object; thus, output is dual to input and 
selection is dual to branching.
Formally, duality 
\jpc{on actions}
is the symmetric relation $\asymp$ that satisfies:
\jpc{(i)~$\bactsel{n}{l} \asymp \bactbra{\dual{n}}{l}$ 
and (ii)~$\news{\widetilde{m}} \bactout{n}{V} \asymp \bactinp{\dual{n}}{V}$}.
%
%\begin{tabular}{c}
%	$\bactsel{n}{l} \asymp \bactbra{\dual{n}}{l}
%	\qquad \qquad \qquad
%	\news{\widetilde{m}} \bactout{n}{V} \asymp \bactinp{\dual{n}}{V}$s
%
%\end{tabular}
\smallskip



%%%%%%%%%%%%%%%%%%%% LTS Figure %%%%%%%%%%%%%%%%%%%%%%%%%%%%%%%%%%%%%
\begin{figure}[t]
\[
	\begin{array}{c}
%	\inferrule[ ]{ }{b}
	\inferrule*[left=\ltsrule{App}]{ }{(\abs{x}{P}) \, V   \by{\tau} P \subst{V}{x}}
%		\ltsrule{App} \ 		(\abs{x}{P}) \, V   \by{\tau} P \subst{V}{x}
		\qquad
		\inferrule*[left=\ltsrule{Snd}]{ }{\bout{n}{V} P \by{\bactout{n}{V}} P}
%		\ltsrule{Snd}\	\bout{n}{V} P \by{\bactout{n}{V}} P 
		\qquad
		\inferrule*[left=\ltsrule{Rv}]{ }{\binp{n}{x} P \by{\bactinp{n}{V}} P\subst{V}{x} }
%		\ltsrule{Rv}\	\binp{n}{x} P \by{\bactinp{n}{V}} P\subst{V}{x} 
		\\[2mm]
		\inferrule*[left=\ltsrule{Sel}]{ }{\bsel{s}{l}{P} \by{\bactsel{s}{l}} P}
		%\ltsrule{Sel}\ \bsel{s}{l}{P} \by{\bactsel{s}{l}} P
		\qquad \quad
		\inferrule*[left=\ltsrule{Bra}]{ }{\bbra{s}{l_i:P_i}_{i \in I} \by{\bactbra{s}{l_j}} P_j \ (j\in I)}
		%\ltsrule{Bra}\ \bbra{s}{l_i:P_i}_{i \in I} \by{\bactbra{s}{l_j}} P_j \ (j\in I)
		\\[2mm]
		\inferrule[\ltsrule{Alpha}]{P \scong_\alpha Q \quad Q\by{\ell} P'}{P \by{\ell} P'}
%		\ltsrule{Alpha}
%		\tree{
%			P \scong_\alpha Q \quad Q\by{\ell} P'
%		}{
%			P \by{\ell} P'
%		}
		\qquad \quad
		\inferrule[\ltsrule{Res}]{P \by{\ell} P' \quad n \notin \fn{\ell}}{\news{n} P \by{\ell} \news{n} P'}
%		\ltsrule{Res}
%		\tree{
%			P \by{\ell} P' \quad n \notin \fn{\ell}
%		}{
%			\news{n} P \by{\ell} \news{n} P' 
%		}
		%\\[5mm]
		\qquad \quad
		\inferrule[\ltsrule{New}]{P \by{\news{\widetilde{m}} \bactout{n}{V}} P' \quad m \in \fn{V}}{\news{m} P \by{\news{m\cat\widetilde{m}'} \bactout{n}{V}} P'}
%		\ltsrule{New}
%		\tree{
%			P \by{\news{\widetilde{m}} \bactout{n}{V}} P' \quad m \in \fn{V}
%		}{
%			\news{m} P \by{\news{m\cat\widetilde{m}'} \bactout{n}{V}} P'
%		}
		\\[4mm]
		\inferrule[\ltsrule{Par${}_L$}]{P \by{\ell} P' \quad \bn{\ell} \cap \fn{Q} = \es}{P \Par Q \by{\ell} P' \Par Q}
%		\ltsrule{Par${}_L$}
%		\tree{
%
%			P \by{\ell} P' \quad \bn{\ell} \cap \fn{Q} = \es
%		}{
%			P \Par Q \by{\ell} P' \Par Q
%		}
%		\\[5mm]
\quad ~~
		\inferrule[\ltsrule{Tau}]{P \by{\ell_1} P' \qquad Q \by{\ell_2} Q' \qquad \ell_1 \asymp \ell_2}{P \Par Q \by{\tau} \newsp{\bn{\ell_1} \cup \bn{\ell_2}}{P' \Par Q'}}
%		\ltsrule{Tau}
%		\tree{
%			P \by{\ell_1} P' \qquad Q \by{\ell_2} Q' \qquad \ell_1 \asymp \ell_2
%		}{
%			P \Par Q \by{\tau} \newsp{\bn{\ell_1} \cup \bn{\ell_2}}{P' \Par Q'}
%		} 
		\quad ~~
		\inferrule[\ltsrule{Rec}]{P\subst{\recp{X}{P}}{\rvar{X}} \by{\ell} P'}{\recp{X}{P}  \by{\ell} P'}
%		\ltsrule{Rec}
%		\tree{
%			P\subst{\recp{X}{P}}{\rvar{X}} \by{\ell} P' 
%		}{
%			\recp{X}{P}  \by{\ell} P'
%		}
	\end{array}
\]
	\caption{The Untyped LTS for \HOp processes. We omit rule $\ltsrule{Par${}_R$}$.  \label{fig:untyped_LTS}}
\end{figure}
%%%%%%%%%%%%%%%%%%%% End LTS Figure %%%%%%%%%%%%%%%%%%%%%%%%%%%%%%%%%
\myparagraph{LTS over Untyped Processes.}
The labelled transition system (LTS) over untyped processes
is given in
\figref{fig:untyped_LTS}. 
We write $P_1 \by{\ell} P_2$ with the usual meaning.
The rules are standard~\cite{KYHH2015,KY2015}.
A process with an output prefix can
interact with the environment with an output action that carries a value
$V$ (rule~$\ltsrule{Snd}$).  Dually, in rule $\ltsrule{Rv}$ a
receiver process can observe an input of an arbitrary value $V$.
Select and branch processes observe the select and branch
actions in rules $\ltsrule{Sel}$ and $\ltsrule{Bra}$, resp.
Rule $\ltsrule{Res}$ closes the LTS under restriction 
if the restricted name does not occur free in the
observable action. 
%If a restricted name occurs free,  
If a restricted name occurs free in
the carried value of an output action,
the process performs scope opening (rule~$\ltsrule{New}$).  
Rule~$\ltsrule{Rec}$ handles recursion unfolding.
Rule~$\ltsrule{Tau}$ 
states that two parallel processes which perform
dual actions can synchronise by an internal transition.
%states that if two parallel processes can perform
%dual actions then the two actions can synchronise by 
%an internal transition. 
Rules $\ltsrule{Par${}_L$}$/$\ltsrule{Par${}_R$}$ 
and $\ltsrule{Alpha}$ close the LTS
under parallel composition and $\alpha$-renaming. 
%provided that the observable
%action does not share any bound names with the parallel processes.

\subsection{Environmental Labelled Transition System}
\label{ss:elts}
\noi 
\figref{fig:envLTS}
defines a labelled transition relation between 
a triple of environments, 
denoted
$(\Gamma_1, \Lambda_1, \Delta_1) \by{\ell} (\Gamma_2, \Lambda_2, \Delta_2)$.
It extends the LTSs
in \cite{KYHH2015,KY2015} 
to higher-order sessions. 

\myparagraph{Input Actions} 
are defined by 
rules~$\eltsrule{SRv}$ and $\eltsrule{ShRv}$.
%describe the input action
In rule~$\eltsrule{SRv}$
%($n$ session or shared channel respectively $\bactinp{n}{V}$). 
the type of value $V$
and the type of the object associated to the session type on $s$ 
should coincide. 
Moreover, 
the resulting type tuple must contain the environments 
associated to $V$. 
The %condition $\dual{s} \notin \dom{\Delta}$
rule requires that 
the dual endpoint $\dual{s}$ is not 
present in the session environment, since if it were present
the only communication that could take place is the interaction
between the two endpoints (using rule~$\eltsrule{Tau}$ below).
Rule~$\eltsrule{ShRv}$ is for shared names and follows similar principles.

\myparagraph{Output Actions} are defined by rules~$\eltsrule{SSnd}$
and $\eltsrule{ShSnd}$.  
Rule $\eltsrule{SSnd}$ states the conditions for observing action
$\news{\widetilde{m}} \bactout{s}{V}$ on a type tuple 
$(\Gamma, \Lambda, \Delta\cdot \AT{s}{S})$. 
The session environment $\Delta$ with $\AT{s}{S}$ 
should include the session environment of the sent value $V$, 
{\em excluding} the session environments of names $m_j$ 
in $\widetilde{m}$ which restrict the scope of value $V$. 
Similarly, the linear variable environment 
$\Lambda'$ of $V$ should be included in $\Lambda$. 
Scope extrusion of session names in $\widetilde{m}$ requires
that the dual endpoints of $\widetilde{m}$ should appear in
the resulting session environment. Similarly for shared 
names in $\widetilde{m}$ that are extruded.  
All free values used for typing $V$ are subtracted from the
resulting type tuple. The prefix of session $s$ is consumed
by the action.
Rule $\eltsrule{ShSnd}$ is for output actions on shared names:
the name must be typed with $\chtype{U}$; conditions on $V$ are identical to those
% the requirements 
on rule~$\eltsrule{SSnd}$.
%\NY{Given a $V$ of type $U$, we sometimes annotate the output action 
%$\news{\widetilde{m}} \bactout{n}{V}$
%%with the type of $V$ 
%as $\news{\widetilde{m}} \bactout{n}{\AT{V}{U}}$.}

\myparagraph{Other Actions}
Rules $\eltsrule{Sel}$ and $\eltsrule{Bra}$ describe actions for
select and branch.
%Both
%rules require the absence of the dual endpoint from the session
%environment.%, and the presence of the action labels in the type.
Hidden transitions defined by rule $\eltsrule{Tau}$ 
do not change the session environment or they follow the reduction on session
environments (\defref{def:ses_red}).

%A second environment LTS, denoted $\hby{\ell}$,
%is defined in the lower part of \figref{fig:envLTS}.
%The definition substitutes rules
%$\eltsrule{SRecv}$ and $\eltsrule{ShRecv}$
%of relation $\by{\ell}$ with rule $\eltsrule{RRcv}$.
%% the corresponding input cases
%%of $\by{\ell}$ with the definitions of $\hby{\ell}$.
%All other cases remain the same as the cases for
%relation $\by{\ell}$.
%Rule $\eltsrule{RRcv}$ restricts the higher-order input
%in relation $\hby{\ell}$;
%only characteristic processes and trigger processes
%are allowed to be received on a higher-order input.
%Names can still be received as in the definition of
%the $\by{\ell}$ relation.
%The conditions for input follow the conditions
%for the $\by{\ell}$ definition.

%%%%%%%%%%%%%%%%%%%% Environment LTS Figure %%%%%%%%%%%%%%%%%%%%%%%%%
\begin{figure}[t]
\[
\begin{array}{c}
\inferrule[\eltsrule{SRv}]{
		\dual{s} \notin \dom{\Delta}
		\quad
		\Gamma; \Lambda'; \Delta' \proves V \hastype U
	}{
		(\Gamma; \Lambda; \Delta \cat s: \btinp{U} S) \by{\bactinp{s}{V}} (\Gamma; \Lambda\cat\Lambda'; \Delta\cat\Delta' \cat s: S)
	}
	%\\[7mm]
	\quad
	\inferrule[\eltsrule{ShRv}]{
		\Gamma; \es; \es \proves a \hastype \chtype{U}
		\quad
		\Gamma; \Lambda'; \Delta' \proves V \hastype U
	}{
		(\Gamma; \Lambda; \Delta) \by{\bactinp{a}{{V}}} (\Gamma; \Lambda\cat\Lambda'; \Delta\cat\Delta')
	}
	\\[6mm]
%%	\eltsrule{SSnd} &
	\inferrule*[left=\eltsrule{SSnd}]{
		\begin{array}{lll}
			\Gamma \cat \Gamma'; \Lambda'; \Delta' \proves V \hastype U
			&				
			\Gamma'; \es; \Delta_j \proves m_j  \hastype U_j
			& 
			\dual{s} \notin \dom{\Delta}
			\\
			\Delta'\backslash \cup_j \Delta_j \subseteq (\Delta \cat s: S)
			& 
			\Gamma'; \es; \Delta_j' \proves \dual{m}_j  \hastype U_j'
			& 
			\Lambda' \subseteq \Lambda
		\end{array}
	}{
		(\Gamma; \Lambda; \Delta \cat s: \btout{U} S)
		\by{\news{\widetilde{m}} \bactout{s}{V}}
		(\Gamma \cat \Gamma'; \Lambda\backslash\Lambda'; (\Delta \cat s: S \cat \cup_j \Delta_j') \backslash \Delta')
	}
	\\[2mm]
	\inferrule*[left=\eltsrule{ShSnd}]{
		\begin{array}{lll}
			\Gamma \cat \Gamma' ; \Lambda'; \Delta' \proves V \hastype U
			&  
			\Gamma'; \es; \Delta_j \proves m_j \hastype U_j
			&
			\Gamma ; \es ; \es \proves a \hastype \chtype{U}
			\\
			\Delta'\backslash \cup_j \Delta_j \subseteq \Delta
			&
			\Gamma'; \es; \Delta_j' \proves \dual{m}_j\hastype U_j'
			& 
			\Lambda' \subseteq \Lambda
		\end{array}
	}{
		(\Gamma ; \Lambda; \Delta) \by{\news{\widetilde{m}}
		\bactout{a}{V}}
		(\Gamma \cat \Gamma'; \Lambda\backslash\Lambda'; (\Delta \cat \cup_j \Delta_j') \backslash \Delta')
	}
	\\[2mm]
	\inferrule*[left=\eltsrule{Sel}]{\dual{s} \notin \dom{\Delta} \quad j \in I}{(\Gamma; \Lambda; \Delta \cat s: \btsel{l_i: S_i}_{i \in I}) \by{\bactsel{s}{l_j}} (\Gamma; \Lambda; \Delta \cat s:S_j)}
%	\eltsrule{Sel}~~
%	\tree{
%		\dual{s} \notin \dom{\Delta} \quad j \in I
%	}{
%		(\Gamma; \Lambda; \Delta \cat s: \btsel{l_i: S_i}_{i \in I}) \by{\bactsel{s}{l_j}} (\Gamma; \Lambda; \Delta \cat s:S_j)
%	}
	\\[2mm]
	\inferrule*[left=\eltsrule{Bra}]{\dual{s} \notin \dom{\Delta} \quad j \in I}{(\Gamma; \Lambda; \Delta \cat s: \btbra{l_i: T_i}_{i \in I}) \by{\bactbra{s}{l_j}} (\Gamma; \Lambda; \Delta \cat s:S_j)}
%	\eltsrule{Bra}~~
%	\tree{
%		\dual{s} \notin \dom{\Delta} \quad j \in I
%	}{
%		(\Gamma; \Lambda; \Delta \cat s: \btbra{l_i: T_i}_{i \in I}) \by{\bactbra{s}{l_j}} (\Gamma; \Lambda; \Delta \cat s:S_j)
%	}
%	\\[7mm]
%	\eltsrule{Tau}~~
%	\tree{
%		\Delta_1 \red \Delta_2 \vee \Delta_1 = \Delta_2
%	}{
%		(\Gamma; \Lambda; \Delta_1) \by{\tau} (\Gamma; \Lambda; \Delta_2)
%	}
\qquad
	\inferrule*[left=\eltsrule{Tau}]{
		\Delta_1 \red \Delta_2 \vee \Delta_1 = \Delta_2
	}{
		(\Gamma; \Lambda; \Delta_1) \by{\tau} (\Gamma; \Lambda; \Delta_2)
	}

\end{array}
\]
\caption{Labelled Transition System for Typed Environments. 
\label{fig:envLTS}}
\Hlinefig
\end{figure}
%%%%%%%%%%%%%%%%%%%% End Environment LTS Figure %%%%%%%%%%%%%%%%%%%%%%
\noi
The weakening property on shared environments ensures that
$(\Gamma_2, \Lambda_1, \Delta_1) \hby{\ell} (\Gamma_2, \Lambda_2, \Delta_2)$
if
$(\Gamma_1, \Lambda_1, \Delta_1) \hby{\ell} (\Gamma_2, \Lambda_2, \Delta_2)$.
Our typed LTS  combines
the LTSs in \figref{fig:untyped_LTS}
and \figref{fig:envLTS}. 

\begin{example}
	Consider environment tuple
	$
		(\Gamma; \es; s: \btout{\lhot{\btout{S} \tinact}} \tinact \cat s': S)
	$
	and typed value
	\[
		\Gamma; \es; s': S \cat m: \btinp{\tinact} \tinact \proves \abs{x} \bout{x}{s'} \binp{m}{z} \inact \hastype \lhot{\btout{\tinact} \tinact}
	\]
	We can then observe that the premises of environment rule $\eltsrule{SSnd}$
	are satisfied to derive that
	\[
		(\Gamma; \es; s: \btout{\lhot{\btout{S} \tinact}} \tinact \cat s': S) \by{\news{m} \bactout{s}{V}} (\Gamma; \es; s: \tinact)
	\]
\end{example}


\smallskip

\begin{definition}[Typed Transition System]\label{d:tlts}\rm
A {\em typed transition relation} is a typed relation
%\begin{enumerate}
%\item 
$\horel{\Gamma}{\Delta_1}{P_1}{\by{\ell}}{\Delta_2}{P_2}$
%$\Gamma; \emptyset; \Delta_1 \proves P_1 \hastype \Proc \by{\ell} \Gamma; \emptyset; \Delta_2 \proves P_2 \hastype \Proc$
	where:
%
(1) $P_1 \by{\ell} P_2$ and (2) 
$(\Gamma, \emptyset, \Delta_1) \by{\ell} (\Gamma, \emptyset, \Delta_2)$ 
with $\Gamma; \emptyset; \Delta_i \proves P_i \hastype \Proc$ 
($i=1,2$).
%\dk{We sometimes annotated the output action with
%the type of value $V$ as in $\widetilde{m} \bactout{n}{V: U}$.}
%
% Efficient 
%\item 
%$\horel{\Gamma}{\Delta_1}{P_1}{\hby{\ell}}{\Delta_2}{P_2}$
%whenever: 
%$P_1 \by{\ell} P_2$, 
%$(\Gamma, \emptyset, \Delta_1) \hby{\ell} (\Gamma, \emptyset, \Delta_2)$, 
%and $\Gamma; \emptyset; \Delta_i \proves P_i \hastype \Proc$ 
%($i=1,2$)
%\end{enumerate}
%
We extend to $\By{}$ 
%(resp.\ $\Hby{}$) and  
and $\By{\hat{\ell}}$ 
%(resp.\ $\Hby{\hat{\ell}}$) 
where we write 
$\By{}$ for the reflexive and
transitive closure of $\by{}$, $\By{\ell}$ for the transitions
$\By{}\by{\ell}\By{}$, and $\By{\hat{\ell}}$ for $\By{\ell}$ if
$\ell\not = \tau$ otherwise $\By{}$. 
%We extend to $\By{}$ 
%(resp.\ $\Hby{}$) and  and 
%$\By{\hat{\ell}}$ 
%(resp.\ $\Hby{\hat{\ell}}$) 
%in the standard way.
\end{definition}

\subsection{Reduction-Closed, Barbed Congruence}
\label{subsec:rc}
\noi We define \emph{typed relations} and \emph{contextual equivalence}.  
%\begin{definition}[Session Environment Confluence]\rm
We first define \emph{confluence}
over session environments $\Delta$:
we denote $\Delta_1 \bistyp \Delta_2$ if there exists $\Delta$ such that
	$\Delta_1 \red^\ast \Delta$ and $\Delta_2 \red^\ast \Delta$
	\jpc{(here we write $\red^\ast$ for the multi-step reduction in \defref{def:ses_red})}.
%\end{definition}

\smallskip 

\begin{definition}\rm %[Typed Relation]\rm
	We say that
	$\Gamma; \emptyset; \Delta_1 \proves P_1 \hastype \Proc\ \Re \ \Gamma; \emptyset; \Delta_2 \proves P_2 \hastype \Proc$
	is a {\em typed relation} whenever $P_1$ and $P_2$ are closed;
		$\Delta_1$ and $\Delta_2$ are balanced; and 
		$\Delta_1 \bistyp \Delta_2$.
We write
$\horel{\Gamma}{\Delta_1}{P_1}{\ \Re \ }{\Delta_2}{P_2}$
for the typed relation $\Gamma; \emptyset; \Delta_1 \proves P_1 \hastype \Proc\ \Re \ \Gamma; \emptyset; \Delta_2 \proves P_2 \hastype \Proc$.
\end{definition}

\smallskip 

\noi Typed relations relate only closed terms whose
session environments %and the two session environments
are balanced  and confluent.
Next we define  {\em barbs}~\cite{MiSa92}
with respect to types. 

\smallskip 

\begin{definition}[Barbs]\rm
Let $P$ be a closed process. We define:
\begin{enumerate}
		\item	$P \barb{n}$ if $P \scong \newsp{\widetilde{m}}{\bout{n}{V} P_2 \Par P_3}, n \notin \widetilde{m}$. %; $P \Barb{n}$ if $P \red^* \barb{n}$.

		\item	$\Gamma; \Delta \proves P \barb{n}$ if
			$\Gamma; \emptyset; \Delta \proves P \hastype \Proc$ with $P \barb{n}$ and $\dual{n} \notin \dom{\Delta}$.

	$\Gamma; \Delta \proves P \Barb{n}$ if $P \red^* P'$ and
			$\Gamma; \Delta' \proves P' \barb{n}$.			
	\end{enumerate}
\end{definition}

\smallskip 

\noi A barb $\barb{n}$ is an observable on an output prefix with subject $n$.
Similarly a weak barb $\Barb{n}$ is a barb after a number of reduction steps.
Typed barbs $\barb{n}$ (resp.\ $\Barb{n}$)
occur on typed processes $\Gamma; \emptyset; \Delta \proves P \hastype \Proc$.
When $n$ is a session name we require that its dual endpoint $\dual{n}$ is not present
in %the session environment 
$\Delta$.

To define a congruence relation, we introduce the family $\C$ of contexts:\\  

%\begin{definition}[Context]\rm
%	A context $\C$ is defined as:
\noi 
\begin{tabular}{rl}
	$\C::=$ & $\hole \bnfbar \bout{u}{V} \C \bnfbar \binp{u}{x} \C \bnfbar \bout{u}{\lambda x.\C} P \bnfbar \news{n} \C$
	$(\lambda x.\C)u \bnfbar \recp{X}{\C}$\\ 
	$\bnfbar$ & $\C \Par P \bnfbar P \Par \C \bnfbar \bsel{u}{l} \C \bnfbar \bbra{u}{l_1:P_1,\cdots,l_i:\C,\cdots,l_n:P_n}$\\
	\end{tabular}
\smallskip 

\noi 
Notation $\context{\C}{P}$ replaces 
\jpc{the hole}
$\hole$ in $\C$ with $P$.
%\end{definition}

\smallskip 

\noi The first behavioural relation we define is reduction-closed, barbed congruence \cite{HondaKYoshida95}. 

\smallskip 

\begin{definition}[Reduction-Closed, Barbed Congruence]\rm
\label{def:rc}
	Typed relation
	$\horel{\Gamma}{\Delta_1}{P_1}{\ \Re\ }{\Delta_2}{P_2}$
	is a {\em reduction-closed, barbed congruence} whenever:
	\begin{enumerate}[1)]
		\item	If $P_1 \red P_1'$ then there exist $P_2', \Delta_2'$ such that $P_2 \red^* P_2'$ and
			$\horel{\Gamma}{\Delta_1'}{P_1'}{\ \Re\ }{\Delta_2'}{P_2'}$; 
			%and its symmetric case;
%		\item	If $P_2 \red P_2'$ then $\exists P_1', P_1 \red^* P_1'$ and
%		$\horel{\Gamma}{\Delta_1'}{P_1'}{\ \Re\ }{\Delta_2'}{P_2'}$
%		\end{itemize}

%		\item
%		\begin{itemize}
			\item	If $\Gamma;\Delta_1 \proves P_1 \barb{n}$ then $\Gamma;\Delta_2 \proves P_2 \Barb{n}$; %and its symmetric case; 

%			\item	If $\Gamma;\emptyset;\Delta \proves P_2 \barb{s}$ then $\Gamma;\emptyset;\Delta \proves P_1 \Barb{s}$.
%		\end{itemize}

		\item	For all $\C$, there exist $\Delta_1'',\Delta_2''$: $\horel{\Gamma}{\Delta_1''}{\context{\C}{P_1}}{\ \Re\ }{\Delta_2''}{\context{\C}{P_2}}$;
		                      \item	The symmetric cases of 1 and 2.                
	\end{enumerate}
	The largest such relation is denoted with $\cong$.
\end{definition}


\subsection{Context Bisimulation ($\wbc$)}
\label{subsec:bisimulation}
\noi The first bisimulation which we define 
is the standard (weak) context bisimulation~\cite{San96H}. 
%
\begin{definition}[Context Bisimulation]\rm
\label{def:wbc}
A typed relation $\Re$ is {\em a context bisimulation} if
for all $\horel{\Gamma}{\Delta_1}{P_1}{\ \Re \ }{\Delta_2}{Q_1}$, 
	\begin{enumerate}[1)] 
	\item Whenever 
$\horel{\Gamma}{\Delta_1}{P_1}
        {\by{\news{\widetilde{m_1}} \bactout{n}{V_1}}}{\Delta_1'}{P_2}$,
there exist 
$Q_2$, $V_2$, $\Delta'_2$
such that 
$\horel{\Gamma}{\Delta_2}{Q_1}{\By{\news{\widetilde{m_2}} \bactout{n}{V_2}}}{\Delta_2'}{Q_2}$ and 
for all $R$ with $\fv{R}=x$:
\[\horel{\Gamma}{\Delta_1''}{\newsp{\widetilde{m_1}}{P_2 \Par R\subst{V_1}{x}}}
				{\ \Re\ }
				{\Delta_2''}{\newsp{\widetilde{m_2}}{Q_2 \Par R\subst{V_2}{x}}};\]  
%\item	$\forall \news{\widetilde{m_1}'} \bactout{n}{\widetilde{m_1}}$ such that
%			\[
%				\horel{\Gamma}{\Delta_1}{P_1}{\by{\news{\widetilde{m_1}'} \bactout{n}{\widetilde{m_1}}}}{\Delta_1'}{P_2}
%			\]
%			implies that $\exists Q_2, \widetilde{m_2}$ such that
%			\[
%				\horel{\Gamma}{\Delta_2}{Q_1}{\By{\news{\widetilde{m_2}'} \bactout{n}{\widetilde{m_2}}}}{\Delta_2'}{Q_2}
%			\]
%			and $\forall R$ with $\widetilde{x} = \fn{R}$, 
%			then
%			\[
%				\horel{\Gamma}{\Delta_1''}{\newsp{\widetilde{m_1}'}{P_2 \Par R \subst{\widetilde{m_1}}{\widetilde{x}}}}
%				{\ \Re \ }
%				{\Delta_2''}{\newsp{\widetilde{m_2}'}{Q_2 \Par R \subst{\widetilde{m_2}}{\widetilde{x}}}}
%			\]
		\item	
For all $\horel{\Gamma}{\Delta_1}{P_1}{\by{\ell}}{\Delta_1'}{P_2}$ such that 
$\ell$ is not an output, 
 there exist $Q_2$, $\Delta'_2$ such that 
$\horel{\Gamma}{\Delta_2}{Q_1}{\By{\hat{\ell}}}{\Delta_2'}{Q_2}$
			and
			$\horel{\Gamma}{\Delta_1'}{P_2}{\ \Re \ }{\Delta_2'}{Q_2}$; and  

                      \item	The symmetric cases of 1 and 2.                
	\end{enumerate}
	The largest such bisimulation is called context bisimilarity \jpc{and} denoted by $\wbc$.
\end{definition}

\smallskip 

\noi As hinted at in %\secref{subsec:intro:bisimulation}, 
\secref{sec:overview},
in the general case,
context bisimulation 
is hard to compute. Below we introduce 
\emph{characteristic bisimulations}.
%$\hwb$ and  $\fwb$.
%due to: (1) the universal
%quantification over contexts in the output case;
%and (2) a higher-order input prefix which can observe
%infinitely many different input actions (since
%infinitely many different processes can match
%the session type of an input prefix).

\subsection{%Higher-Order  and  
Characteristic  Bisimulations ($\fwb$)}\label{ss:hwb}
\noi 
We formalise the novelties motivated in \secref{sec:overview}.
Our main result is \thmref{the:coincidence}.
We define characteristic processes/values:

\begin{definition}[Characteristic Process and Values]\rm
\label{def:char}
%	Let names $\widetilde{k}$ and type $\widetilde{C}$; then we define a {\em characteristic process}:
%	$\map{\widetilde{C}}^{\widetilde{k}}$:
%	\[
%		\map{C_1, \cdots, C_n}^{k_1 \cdots k_n} = \map{C_1}^{k_1} \Par \dots \Par \map{C_n}^{k_n}		
%	\]
%	with 
	Let $u$ and $U$ be a name and a type, respectively.
	\figref{fig:char} defines the {\em characteristic process} 
	$\mapchar{U}{u}$ and the {\em characteristic value} $\omapchar{U}$.
\end{definition}

\smallskip

%%%%%%%%%%%%%%%%%%%%%%%% Characteristic Process Figure %%%%%%%%%%%%%%%%%%%%%%%%%
\begin{figure}[t]
\[
	\begin{array}{rclcrcl}
		\mapchar{\btinp{U} S}{u}
		&\defeq&
		\binp{u}{x} (\mapchar{S}{u} \Par \mapchar{U}{x})
		&&
		\mapchar{\btout{U} S}{u}
		&\defeq&
		\bout{u}{\omapchar{U}} \mapchar{S}{u} %& & n \textrm{ fresh}
		\\

		\mapchar{\btsel{l : S}}{u}
		& \defeq &
		\bsel{u}{l} \mapchar{S}{u}
		&&
		\mapchar{\btbra{l_i: S_i}_{i \in I}}{u}
		& \defeq &
		\bbra{u}{l_i: \mapchar{S_i}{u}}_{i \in I}
		\\

		\mapchar{\tvar{t}}{u}
		&\defeq&
		\varp{X}_{\vart{t}}
		& & 
		\mapchar{\trec{t}{S}}{u}
		&\defeq&
		\recp{X_{\vart{t}}}{\mapchar{S}{u}}
		\\

		\mapchar{\tinact}{u}
		& \defeq &
		\inact
		& & 
		\mapchar{\chtype{S}}{u} 
		&\defeq&
		\bout{u}{\omapchar{S}} \inact
		\\

		\mapchar{\chtype{L}}{u}
		&\defeq&
		\bout{u}{\omapchar{L}} \inact
		&&
		\mapchar{\shot{U}}{u}
		&\defeq &
		\mapchar{\lhot{U}}{u}
		\, \defeq \,
		\appl{u}{\omapchar{U}}
		\end{array}
		\]
		\[
		\begin{array}{c}
		\omapchar{S}  \defeq  s ~~ (s \textrm{ fresh})
		\qquad
		\omapchar{\chtype{S}} \defeq \omapchar{\chtype{L}} \defeq a ~~ (a \textrm{ fresh})
		\qquad
		\omapchar{\shot{U}} \defeq \omapchar{\lhot{U}} \,\defeq\, \abs{x}{\mapchar{U}{x}}
	\end{array}
\]
\caption{Characteristic Processes \jpc{(top)} and Values \jpc{(bottom)}.\label{fig:char}}
%\Hlinefig
\end{figure}
%%%%%%%%%%%%%%%%%%%%%%%% End Characteristic Process Figure %%%%%%%%%%%%%%%%%%%%%

%	\[
%	\begin{array}{rcllrcll}
%		\map{\btinp{C} S}^{u} &=& \binp{u}{x} (\map{S}^{u} \Par 
%\map{{C}}^{x})\\
%		\map{\btout{C} S}^{u} &=& \bout{u}{n} \map{S}^{u} & n\textrm{ fresh}
%		\\
%		\map{\btsel{l : S}}^{u} &=& \bsel{u}{l} \map{S}^{u}
%\\
%		\map{\btbra{l_i: S_i}_{i \in I}}^{u} &=& \bbra{k}{l_i: \map{S_i}^{u}}_{i \in I}
%		\\

%		\map{\tvar{t}}^{u} &=& \rvar{X}_{\tvar{t}}
%\\
%		\map{\trec{t}{S}}^{u} &=& \mu \rvar{X}_{\tvar{t}}.\map{S}^{u}
%		\\

%		\map{\btout{\lhot{C}} S}^{u} &=& \bout{u}{\abs{x}{\map{C}^{x}}} \map{S}^u
%\\
%		\map{\btinp{\lhot{C}} S}^{u} &=& \binp{u}{x} (\map{S}^u \Par \appl{x}{n}) & {n}\textrm{ fresh}
%		\\
%		\map{\btout{\shot{C}} S}^{u} &=& \bout{u}{\abs{x}{\map{C}^{x}}}
%\map{S}^u \\
%		\map{\btinp{\shot{C}} S}^{u} &=& \binp{u}{x} (\map{S}^u \Par \appl{x}{n}) & n\textrm{ fresh}
%		\\

%%		\map{\btinp{\chtype{S}} S}^{k} &=& \binp{k}{x} (\map{S}^k \Par \map{\chtype{S}}^y)
%%		&&
%%		\map{\btout{\chtype{S}} S}^{k} &=& \bout{k}{a} \map{S}^k  & a\textrm{ fresh}
%%		\\
%		\map{\tinact}^{u} &=& \inact
%\\
%		\map{\chtype{S}}^{u} &=& \bout{u}{s} \inact &s\textrm{ fresh}
%\\
%		\map{\chtype{\lhot{C}}}^{u} &=& \bout{u}{\abs{x} \map{C}^{x}} \inact
%		\\
%		\map{\chtype{\shot{C}}}^{u} &=& \bout{u}{\abs{x} \map{C}^{x}} \inact
%	\end{array}
%	\]
%\end{definition}


%\noi Characteristic processes are inhabitants of their associated type:

%\begin{proposition}[Characteristic Processes]\rm
%\label{pro:characteristic}
%\begin{enumerate}
%\item $\Gamma; \emptyset; \Delta \cat u:S \proves \mapchar{S}{u} \hastype \Proc$; and $\Gamma \cat u:\chtype{S}; \emptyset; \Delta \proves \mapchar{\chtype{S}}{u} \hastype \Proc$; and 
%\item  	If $\Gamma; \emptyset; \Delta \proves \mapchar{C}{u} \hastype \Proc$
%	then
%	$\Gamma; \es; \Delta \proves u \hastype C$.
%\end{enumerate}
%\end{proposition}
%%\begin{IEEEproof}
%%	By induction on $\mapchar{S}{u}$, $\mapchar{\chtype{S}}{u}$
%%and $\mapchar{C}{u}$. 
%%\end{IEEEproof}

\noi We can easily verify that characteristic processes/values  
do inhabit  their associated type. 
The following example motivates the refined 
LTS explained in \secref{sec:overview}.

\begin{proposition}
	\begin{itemize}
		\item	Let session type $S$, then $\Gamma; \es; \Delta \cat s: S \proves \mapchar{S}{s} \hastype \Proc$.
		\item	Let session type $\chtype{U}$, then $\Gamma \cat a: \chtype{U}; \es; \Delta \proves \mapchar{\chtype{U}}{a} \hastype \Proc$.
	\end{itemize}
\end{proposition}


\begin{proof}
	By induction on the definition of $\mapchar{U}{n}$.
\end{proof}

\begin{example}
	Consider for example the type $U = \shot{(\shot{(\shot{(\btout{S} \tinact)})})}$.
	The characteristic process of the above type is:
%
	\begin{eqnarray*}
		\mapchar{\btinp{U} \inact}{s} = && \binp{s}{x} \mapchar{\shot{(\shot{(\shot{(\btout{S} \tinact)})})}}{x}\\
		= && \binp{s}{x} \appl{x}{\omapchar{\shot{(\shot{(\btout{S} \tinact)})}}}\\
		= && \binp{s}{x} \appl{x}{(\abs{y}{\mapchar{\shot{(\btout{S} \tinact)}}{y}})}\\
		= && \binp{s}{x} \appl{x}{(\abs{y}{\appl{y}{\omapchar{\btout{S} \tinact}})}}\\
		= && \binp{s}{x} \appl{x}{(\abs{y}{\appl{y}{n})}}\\
	\end{eqnarray*}
%
\end{example}


\smallskip  

\begin{example}[The Need for Refined Typed LTS]
\label{ex:motivation}
First we demonstrate that observing a characteristic value
input alone is not sufficient
\dk{to define a sound bisimulation closure}.
Consider typed processes $P_1, P_2$:
%
\begin{eqnarray}
	P_1 = \binp{s}{x} (\appl{x}{s_1} \Par \appl{x}{s_2}) 
	& & 
	P_2 = \binp{s}{x} (\appl{x}{s_1} \Par \binp{s_2}{y} \inact) 
	\label{equ:6}
\end{eqnarray}
%
%We can show that 
where
$\Gamma; \es; \Delta \cat s: \btinp{\shot{(\btinp{C} \tinact)}} \tinact \proves P_i \hastype \Proc$ ($i \in \{1,2\}$).
If $P_1$ and $P_2$ input and substitute over $x$
the characteristic value $\dk{\omapchar{\shot{(\btinp{C} \tinact)}} =} \abs{x}{\binp{x}{y} \inact}$, 
then they evolve into:%(\ref{eq:5}) and (\ref{eq:6}) in become:
\begin{center}
\begin{tabular}{c}
	$\Gamma; \es; \Delta \proves \binp{s_1}{y} \inact \Par \binp{s_2}{y} \inact \hastype \Proc$
\end{tabular}
\end{center}
\noi therefore becoming 
context bisimilar.
%after the input of $\abs{x}{\binp{x}{y}} \inact$.
However, the processes in (\ref{equ:6}) 
are clearly {\em not} context bisimilar: there are many input actions
which may be used to distinguish them.
For example, if 
$P_1$ and $P_2$ input 
$\abs{x} \newsp{s}{\bout{a}{s} \binp{x}{y} \inact}$ with
$\Gamma; \es; \Delta \proves s \hastype \tinact$,
then their derivatives are not bisimilar. 

Observing only the characteristic value 
results in an under-discriminating bisimulation.
However, if a trigger value
$\abs{{x}}{\binp{t}{y} (\appl{y}{{x}})}$ 
is received on $s$, 
then we can distinguish $P_1$ and $P_2$ 
%processes 
in \eqref{equ:6}:  
%
\small
\begin{eqnarray*}
%	\Gamma; \es; \Delta &\proves& 
	P_1 \By{\ell} \binp{t}{x} (\appl{x}{s_1}) \Par 
\binp{t}{x} (\appl{x}{s_2})
%\hastype \Proc
	\mbox{~~and~~}
%	\Gamma; \es; \Delta &\proves& 
	P_2 \By{\ell} \binp{t}{x} (\appl{x}{s_1}) \Par \binp{s_2}{y} \inact 
%\hastype \Proc
\end{eqnarray*}
\normalsize
%\noi resulting two distinct processes.  
%
\noi where 
$\ell = s?\ENCan{\abs{{x}}{\binp{t}{y} (\appl{y}{{x}})}}$.
One question that arises here is whether the trigger value is enough
to distinguish two processes, hence no need of 
characteristic values as the input. 
This is not the case since the trigger value
alone also results in an under-discriminating bisimulation relation.
In fact the  trigger value can be observed on any input prefix
of {\em any type}. For example, consider the following processes:
%
\begin{eqnarray}
	\Gamma; \es; \Delta \proves \newsp{s}{\binp{n}{x} (\appl{x}{s}) \Par \bout{\dual{s}}{\abs{x} P} \inact} \hastype \Proc\label{equ:7}
	\\
	\Gamma; \es; \Delta \proves \newsp{s}{\binp{n}{x} (\appl{x}{s}) \Par \bout{\dual{s}}{\abs{x} Q} \inact} \hastype \Proc\label{equ:8}
\end{eqnarray}
%
\noi if processes in \eqref{equ:7}/\eqref{equ:8}
input the trigger value, we obtain: % they evolved to 
\begin{eqnarray*}
%\Gamma; \es; \Delta \proves 
	\newsp{s}{\binp{t}{x} (\appl{x}{s}) \Par \bout{\dual{s}}{\abs{x} P} \inact} 
%\hastype \Proc
	\mbox{ and }
%      \\
%\Gamma; \es; \Delta \proves 
	\newsp{s}{\binp{t}{x} (\appl{x}{s}) \Par \bout{\dual{s}}{\abs{x} Q} \inact}
%\hastype \Proc
\end{eqnarray*}

\noi thus we can easily derive a bisimulation closure if we 
assume a bisimulation definition that allows only trigger value input.
%
%\noi It is easy to obtain a closure if allow only the
%trigger value as the input value. 
But if processes in \eqref{equ:7}/\eqref{equ:8}
input the characteristic value $\abs{z}{\binp{z}{x} (\appl{x}{m})}$,  
then they would become:
%
\begin{eqnarray*}
	\Gamma; \es; \Delta \proves \newsp{s}{\binp{s}{x} (\appl{x}{m}) \Par \bout{\dual{s}}{\abs{x} P} \inact} \wbc \Delta \proves P \subst{m}{x}
	\\
	\Gamma; \es; \Delta \proves \newsp{s}{\binp{s}{x} (\appl{x}{m}) \Par \bout{\dual{s}}{\abs{x} Q} \inact} \wbc \Delta \proves Q \subst{m}{x}
\end{eqnarray*}
\noi which are not bisimilar if $P \subst{m}{x} \not\wb Q \subst{m}{x}$.
%\qed
%In conclusion, these examples explain a need of both 
%trigger and characteristic values 
%as an input observation in the input transition relation (\eltsrule{RRcv})
%which will be defined in Definition~\ref{def:rlts}.  
\end{example}

\smallskip 
%\noi We define the \emph{refined} typed LTS. 
\noi As explained in \secref{sec:overview}, we define the
\emph{refined} typed LTS
by considering a transition rule for input in which admitted values are
trigger or characteristic values or names:

%\noi We define the \emph{refined} typed LTS. 
%As explained in \secref{subsec:intro:bisimulation}, this new LTS is defined 
%by considering a transition rule for input in which admitted values are
%trigger or characteristic values:
%\dk{(assume extension of the structural
%congruence to acommodate values: i) $\abs{x}{P} \scong \abs{x}{Q}$ if
%$P \scong Q$) and ii) $n \scong m$ if $n = n$)}: 

\smallskip 

\begin{definition}[Refined Typed Labelled Transition Relation]
	\label{def:rlts}
	We define the environment transition rule for input actions 
	%restricted environment transition relation using the
	%following rule %using the environment transition relation defined in 
	using the input rules in \figref{fig:envLTS}:
	\[ \!\!\!
	\begin{array}{l}
%			\eltsrule{RRcv}&\tree {
%	\begin{array}{c}
%	(\Gamma_1; \Lambda_1; \Delta_1) \by{\bactinp{n}{V}} (\Gamma_2; \Lambda_2; \Delta_2)
%	\\
%				\begin{array}{lll}
%					V = m \vee V  \scong
%					\abs{{x}}{\binp{t}{y} (\appl{y}{{x}})}
%					\vee  V \scong \omapchar{U}%\abs{{x}}{\map{U}^{{x}}}
%					\textrm{ with } t \textrm{ fresh} 
%				\end{array}
%				\end{array}
%			}{
%				(\Gamma_1; \Lambda_1; \Delta_1) \hby{\bactinp{n}{V}} (\Gamma_2; \Lambda_2; \Delta_2)
%			}
			
						\eltsrule{RRcv}\,\tree {
	%\begin{array}{c}
	(\Gamma_1; \Lambda_1; \Delta_1) \by{\bactinp{n}{V}} (\Gamma_2; \Lambda_2; \Delta_2)
	\quad
					V = m 
					\vee  V \scong \omapchar{U}%\abs{{x}}{\map{U}^{{x}}}
										\vee V  \scong \abs{{x}}{\binp{t}{y} (\appl{y}{{x}})}
					\text{ {\small with $t$ fresh}} 
				%\end{array}
			}{
				(\Gamma_1; \Lambda_1; \Delta_1) \hby{\bactinp{n}{V}} (\Gamma_2; \Lambda_2; \Delta_2)
			}
			
			
	\end{array}
	\]
	\noi Rule $\eltsrule{RRcv}$ is defined on top
	of rules $\eltsrule{SRv}$ and $\eltsrule{ShRv}$
	in \figref{fig:envLTS}.
%	uses the environment transition
%	$(\Gamma, \Lambda_1, \Delta_1) \hby{\ell} (\Gamma, \Lambda_2, \Delta_2)$
%	in \figref{fig:envLTS}. 
\dk{	We  use the non-receiving rules in \figref{fig:envLTS}
	together with rule $\eltsrule{RRcv}$
	to define 
	$\horel{\Gamma}{\Delta_1}{P_1}{\hby{\ell}}{\Delta_2}{P_2}$
	as in \defref{d:tlts}.}
%	by replacing $\by{\ell}$ by $\hby{\ell}$ in \defref{d:tlts}. 
\end{definition}

\smallskip 

\noi Note 
$\horel{\Gamma}{\Delta_1}{P_1}{\hby{\ell}}{\Delta_2}{P_2}$ implies  
$\horel{\Gamma}{\Delta_1}{P_1}{\by{\ell}}{\Delta_2}{P_2}$.
Below we sometimes write  
$\hby{\news{\widetilde{m}} \bactout{n}{\AT{V}{U}}}$
when the type of $V$ is~$U$.

%See \exref{ex:motivation} for the reason why {\em both} 
%the trigger values ($\lambda x.\binp{t}{y} (\appl{y}{{x}})$) 
%and characteristic values ($\lambda x.\map{U}^{{x}}$) are required 
%to define the following two bisimulations. 

\smallskip 

\myparagraph{Characteristic Bisimulations.} We define 
%\emph{higher-order} and
\emph{characteristic
bisimulations}, 
%two tractable bisimulations for $\HO$ and $\HOp$.
a tractable bisimulations for $\HOp$.
As explained in \secref{sec:overview},
%the two bisimulations 
characteristic bisimulations
use trigger processes:
%the key difference between them is in the trigger processes they use:
\begin{eqnarray*}
%\htrigger{t}{V_1}  & \defeq &  \hotrigger{t}{V} \label{eqb:0} \\
	\ftrigger{t}{V}{U} & \defeq &  \fotrigger{t}{x}{s}{\btinp{U} \tinact}{V} 	\label{eqb:4}
\end{eqnarray*}
%\noi
%Notice that while 
%in~\eqref{eqb:0} there is a higher-order input on $t$, 
%in~\eqref{eqb:4} variable $x$ does not play any role.
% \dk{The two bisimulations differ on the fact that
%they use the different 
%trigger processes: $\htrigger{t}{V}$ and $\ftrigger{t}{V}{U}$.}

\smallskip 

%\begin{definition}[Higher-Order Bisimilarity]\rm
%	\label{d:hbw}
%A typed relation $\Re$ is a {\em  HO bisimulation} if 
%for all $\horel{\Gamma}{\Delta_1}{P_1}{\ \Re \ }{\Delta_2}{Q_1}$ 
%\begin{enumerate}[1)]
%\item 
%Whenever 
%$\horel{\Gamma}{\Delta_1}{P_1}{\hby{\news{\widetilde{m_1}} \bactout{n}{V_1}}}{\Delta_1'}{P_2}$, there exist 
%$Q_2$, $V_2$, $\Delta'_2$ such that 
%$\horel{\Gamma}{\Delta_2}{Q_1}{\Hby{\news{\widetilde{m_2}} \bactout{n}{V_2}}}{\Delta_2'}{Q_2}$ and, for fresh $t$, 
%\[
%\begin{array}{lrlll}
%\Gamma; \Delta''_1  \proves  {\newsp{\widetilde{m_1}}{P_2 \Par 
%\htrigger{t}{V_1}}}
%\ \Re 
%\ \Delta''_2 \proves {\newsp{\widetilde{m_2}}{Q_2 \Par \htrigger{t}{V_2}}}
%\end{array}
%\]
%		\item	
%For all $\horel{\Gamma}{\Delta_1}{P_1}{\hby{\ell}}{\Delta_1'}{P_2}$ such that 
%$\ell$ is not an output, 
% there exist $Q_2$, $\Delta'_2$ such that 
%$\horel{\Gamma}{\Delta_2}{Q_1}{\Hby{\hat{\ell}}}{\Delta_2'}{Q_2}$
%			and
%			$\horel{\Gamma}{\Delta_1'}{P_2}{\ \Re \ }{\Delta_2'}{Q_2}$; and 
%
%                      \item	The symmetric cases of 1 and 2.                
%	\end{enumerate}
%	The largest such bisimulation
%	is called \emph{higher-order bisimilarity} \jpc{and} denoted by $\hwb$.
%\end{definition}

 
We define characteristic bisimilarity:
% is given using characteristic trigger processes. 

\begin{definition}[Characteristic Bisimilarity]\rm
	\label{d:fwb}
A typed relation $\Re$ is a {\em  characteristic bisimulation} if 
for all $\horel{\Gamma}{\Delta_1}{P_1}{\ \Re \ }{\Delta_2}{Q_1}$ 
\begin{enumerate}[1)]
\item 
	Whenever 
	$\horel{\Gamma}{\Delta_1}{P_1}{\hby{\news{\widetilde{m_1}} \bactout{n}{\dk{V_1: U}}}}{\Delta_1'}{P_2}$ %with $\Gamma; \es; \Delta \proves V_1 \hastype U$,  
	then there exist 
	$Q_2$, $V_2$, $\Delta'_2$ such that 
	$\horel{\Gamma}{\Delta_2}{Q_1}{\Hby{\news{\widetilde{m_2}}\bactout{n}{\dk{V_2: U}}}}{\Delta_2'}{Q_2}$ %with $\Gamma; \es; \Delta' \proves V_2 \hastype U$,  
	and, for fresh $t$, \\ 
	$%\begin{array}{lrlll}
	\Gamma; \Delta''_1  \proves  {\newsp{\widetilde{m_1}}{P_2 \Par 
	\ftrigger{t}{V_1}{U_1}}}
	  \,\Re\,
	 \Delta''_2 \proves {\newsp{\widetilde{m_2}}{Q_2 \Par \ftrigger{t}{V_2}{U_2}}}
%\end{array}
$
		
\item	
For all $\horel{\Gamma}{\Delta_1}{P_1}{\hby{\ell}}{\Delta_1'}{P_2}$ such that 
$\ell$ is not an output, 
 there exist $Q_2$, $\Delta'_2$ such that 
$\horel{\Gamma}{\Delta_2}{Q_1}{\Hby{\hat{\ell}}}{\Delta_2'}{Q_2}$
			and
			$\horel{\Gamma}{\Delta_1'}{P_2}{\ \Re \ }{\Delta_2'}{Q_2}$; and 

                      \item	The symmetric cases of 1 and 2.                
	\end{enumerate}
	The largest such bisimulation
	is called \emph{characteristic bisimilarity} \jpc{and} denoted by $\fwb$.
\end{definition}

\smallskip 

%\begin{definition}[Characteristic Bisimilarity]\rm
%	\label{d:fwb}
%	Characteristic bisimilarity, denoted by $\fwb$, is defined \jpc{by} replacing 
%	Clause 1) in \defref{d:hbw} with the following clause:\\[1mm]
%	Whenever 
%	$\horel{\Gamma}{\Delta_1}{P_1}{\hby{\news{\widetilde{m_1}} \bactout{n}{\dk{V_1: U}}}}{\Delta_1'}{P_2}$ %with $\Gamma; \es; \Delta \proves V_1 \hastype U$,  
%	then there exist 
%	$Q_2$, $V_2$, $\Delta'_2$ such that 
%	$\horel{\Gamma}{\Delta_2}{Q_1}{\Hby{\news{\widetilde{m_2}}\bactout{n}{\dk{V_2: U}}}}{\Delta_2'}{Q_2}$ %with $\Gamma; \es; \Delta' \proves V_2 \hastype U$,  
%	and, for fresh $t$, \\[1mm]
%	$\begin{array}{lrlll}
%	\!\!\Gamma; \Delta''_1  \proves  {\newsp{\widetilde{m_1}}{P_2 \Par 
%	\ftrigger{t}{V_1}{U_1}}}
%	\ \!\!\Re\!\!
%	\ \Delta''_2 \proves {\newsp{\widetilde{m_2}}{Q_2 \Par \ftrigger{t}{V_2}{U_2}}}
%\end{array}
%$
%\end{definition}
%
%\smallskip 

\noi We now state our main theorem: %typed bisimilarities collapse. 

\smallskip 

\begin{theorem}[Coincidence]\rm
%	\label{the:coincidence}
%$\cong$, $\wbc$, $\hwb$ and $\fwb$ coincide in $\CAL\in \{\HOp, \HO\}$
%and 
%$\cong$, $\wbc$ and $\fwb$ coincide in $\CAL\in \{\HOp, \HO, \sessp\}$. 
	\label{the:coincidence}
$\cong$, $\wbc$,  and $\fwb$ coincide in $\HOp$. 
\end{theorem}

\begin{proof}[Proof (Sketch)]
DIMITRIS, PLEASE ADD SKETCH.
\end{proof}
%\smallskip 

\noi 
%Thus, we may use $\hwb$ for tractable reasoning %is the most tractable 
%in the higher-order setting;  
%%the calculus is limited  
%%the $\sessp$-calculus, 
%%into~\jpc{\sessp}
%in the first-order setting of $\sessp$
%we can still use~$\fwb$. 

\noi Thus, we may use $\fwb$ for tractable reasoning %is the most tractable 
in the higher-order setting.

%PERHAPS REVISIT EXAMPLE \ref{ex:motivation}??


%\smallskip  
%
%\noi Processes that do not use shared names are inherently deterministic. 
%The following \jpc{determinacy property will be} useful 
%\dk{for proving our expressiveness results (\secref{sec:positive})}.
%%for both positive and negative results. 
%We require an auxiliary definition. 
%A transition $\horel{\Gamma}{\Delta}{P}{\hby{\tau}}{\Delta'}{P'}$ is said
%		{\em deterministic} if it is derived using~$\ltsrule{App}$ or~$\ltsrule{Tau}$ 
%		(where $\subj{\ell_1}$ and $\subj{\ell_2}$ in the premise 
%		are dual endpoints), 
%		possibly followed by uses of  $\ltsrule{Alpha}$, $\ltsrule{Res}$, $\ltsrule{Rec}$, or $\ltsrule{Par${}_L$}/\ltsrule{Par${}_R$}$.
%
%
%%\smallskip 
%
%\begin{lemma}[$\tau$-Inertness]\rm
%	\label{lem:tau_inert}
%	\begin{enumerate}[1)]
%		\item %(deterministic transitions) 
%		Let $\horel{\Gamma}{\Delta}{P}{\hby{\tau}}{\Delta'}{P'}$ be a deterministic transition,
%		with balanced $\Delta$. Then 
%		$\Gamma; \Delta \proves P \cong \Delta'\proves P'$ 
%		with $\Delta \red^\ast \Delta'$ balanced.
%%		Transition $\horel{\Gamma}{\Delta}{P}{\hby{\tau}}{\Delta'}{P'}$ is called
%%		{\em deterministic} if it is derived by $\ltsrule{App}$ or 
%%		$\ltsrule{Tau}$ where $\subj{\ell_1}$ and $\subj{\ell_2}$ in the premise 
%%		are dual session names. Suppose $\Delta$ is balanced. Then 
%%		$\Gamma; \Delta \proves P \cong \Delta'\proves P'$ 
%%		with $\Delta \red^\ast \Delta'$ balanced. 
%		\item 
%		%Let $P$ is the $\HOp^{-\mathsf{sh}}$-calculus. 
%		Let $P$ be an $\HOp^{-\mathsf{sh}}$ process. 
%		Assume $\Gamma; \emptyset; \Delta \proves P \hastype \Proc$. Then 
%		$P \red^\ast P'$ implies $\Gamma; \Delta \proves 
%		P \cong \Delta'\proves P'$ with $\Delta \red^\ast \Delta'$. 
%	\end{enumerate}
%\end{lemma}


%\smallskip 


%\begin{IEEEproof}
%	The full details of the proof are in Appendix~\ref{app:sub_coinc}.
%	The theorem is split into a hierarchy of Lemmas. Specifically
%	Lemma~\ref{lem:wb_eq_wbf} proves 
%	$\wb$ coincides with $\fwb$; 
%	Lemma~\ref{lem:wb_is_wbc} exploits the process substitution result
%	(Lemma~\ref{lem:proc_subst}) to prove that $\hwb \subseteq \wbc$.
%	Lemma~\ref{lem:wbc_is_cong} shows that $\wbc$ is a congruence
%	which implies $\wbc \subseteq \cong$.
%	The final result comes from Lemma~\ref{lem:cong_is_wb} where
%	we use label testing to show that $\cong \subseteq \fwb$ using
%	the technique in developed in~\cite{Hennessy07}. The formulation of input
%	triggers in the bisimulation relation allows us to prove
%	the latter result without using a matching operator.
%\end{IEEEproof}

%\smallskip 

%\noi Processes that do not use shared names, are inherently $\tau$-inert.

%\smallskip 

%\begin{lemma}[$\tau$-inertness]\rm
%	\label{lem:tau_inert}
%	Let $P$ is the $\HOp^{-\mathsf{sh}}$-calculus. 
%Assume $\Gamma; \emptyset; \Delta \proves P \hastype \Proc$. Then 
%$P \red^\ast P'$ implies $\Gamma; \Delta \proves 
%P \cong \Delta'\proves P'$ with $\Delta \red^\ast \Delta'$. 
%\end{lemma}


%\begin{IEEEproof}
%	The proof is relied on the fact that processes of the
%	form $\Gamma; \es; \Delta \proves_s \bout{s}{V} P_1 \Par \binp{k}{x} P_2$
%	cannot have any typed transition observables and the fact
%	that bisimulation is a congruence.
%	See details in Appendix~\ref{app:sub_tau_inert}.
%	\qed
%\end{IEEEproof}

