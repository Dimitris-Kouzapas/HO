% !TEX root = main.tex
\noi %Here we define 
We define a session typing system for \HOp and state its main properties. 
Our system distills the key features of~\cite{tlca07,MostrousY15}.
We give selected definitions; 
see \appref{app:types} for a full description. 

%thus it is simpler.
%Our system is simpler than that in~\cite{tlca07,MostrousY15}, thus distilling the key
%features of higher-order sessions. %communications. %in a session-typed setting.

\smallskip 

%\subsection{Types}
%\label{subsec:types}
\myparagraph{Types.}
The syntax of types of \HOp is given below:
\[
	\begin{array}{cc}
		\begin{array}{lcl}
			\text{(value)} & U \bnfis & C \bnfbar L
			\\[1mm]
			\text{(name)} & C \bnfis & S \bnfbar \chtype{S} \bnfbar \chtype{L}
			\\[1mm]
			\text{(abstr)} & L \bnfis & \shot{U} \bnfbar \lhot{U}
		\end{array}
		&
		\begin{array}{lcl}
			\text{(session)} & S \bnfis &  \btout{U} S \bnfbar \btinp{U} S \bnfbar \tinact
			\\[1mm]
			&\bnfbar& \btsel{l_i:S_i}_{i \in I} \bnfbar \trec{t}{S} \bnfbar \vart{t}
			\\[1mm]
			&\bnfbar& \btbra{l_i:S_i}_{i \in I}
		\end{array}
	\end{array}
	\]
Value type $U$ includes
the first-order types $C$ and the higher-order
types $L$. Session types are denoted with $S$ and
shared types with $\chtype{S}$ and $\chtype{L}$.
%Note that we dissallow type $\chtype{U}$, thus
%in the type discipline shared names cannot carry shared names.
%In name types, $\chtype{U}$ is shared name types 
%which are sent via shared names. 
Types $\shot{U}$ and $\lhot{U}$ denote
{\em shared} and {\em linear} higher-order 
%\jpc{functional}
types, respectively.
%,
%used to type abstraction values.
%$\lhot{C}$ \cite{tlca07,mostrous_phd} ensures values which contain free 
%session names used once. 
As for session types,  
%We write $S$ to denote %binary 
%session types. 
the {\em output type}
$\btout{U} S$ %is assigned to a name that 
first sends a value of
type $U$ and then follows the type described by $S$.  Dually,
$\btinp{U} S$ denotes an {\em input type}. The {\em branching type}
$\btbra{l_i:S_i}_{i \in I}$ and the {\em selection type}
$\btsel{l_i:S_i}_{i \in I}$ define the labelled choice. 
We assume the {\em recursive type} $\trec{t}{S}$ is guarded,
i.e.,  $\trec{t}{\vart{t}}$ is not allowed. 
%We stress that carried type $U$ in $\btout{U} S$ and
%$\btinp{U} S$ can contain free type variables, which is crucial
%to encode $\HOp$ into $\HO$.
Type $\tinact$ is the termination type. 

%Types of \HO exclude $\nonhosyntax{C}$ from 
%value types of \HOp; the types of \sessp exclude $L$. 
%From each $\CAL \in \{\HOp, \HO, \pi \}$, $\CAL^{-\mathsf{sh}}$ 
%excludes shared name types ($\chtype{S}$ and $\chtype{L}$), 
%from name type $C$.

%Following \cite{TGC14},
We write $S_1 \dualof S_2$ if 
$S_1$ is the \emph{dual} of $S_2$.   
See the Appendix (\defref{def:dual}) for a definition; 
intuitively, 
duality
converts $!$ into $?$ and $\oplus$ into $\&$ (and viceversa). 
%incorporating also the fixed point construction.
%(see in the Appendix). 

\smallskip 

%\subsection{Typing System of \HOp}
%\label{subsec:typing}
\myparagraph{Typing Environments and Judgements.}
\noi Typing \emph{environments} are defined below:
%\[
\begin{eqnarray*}
	\Gamma  & \bnfis  &\emptyset \bnfbar \Gamma \cat \varp{x}: \shot{U} \bnfbar \Gamma \cat u: \chtype{S} \bnfbar \Gamma \cat u: \chtype{L} 
        \bnfbar \Gamma \cat \rvar{X}: \Delta \\
	\Lambda &\bnfis & \emptyset \bnfbar \Lambda \cat \AT{x}{\lhot{U}}
	\qquad\qquad
	\Delta  \bnfis  \emptyset \bnfbar \Delta \cat \AT{u}{S}
\end{eqnarray*}
%]
\noi 
$\Gamma$ maps variables and shared names to value types, and recursive 
variables to session environments;  
it admits weakening, contraction, and exchange principles.
$\Lambda$ is a mapping from variables to 
%the
 linear %functional 
higher-order
types; and $\Delta$ is a mapping from 
session names to session types. 
Both $\Lambda$ and $\Delta$ %behave linearly: they 
are
only subject to exchange.  
%We require that t
The domains of $\Gamma,
\Lambda$ and $\Delta$ are assumed pairwise distinct. 
$\Delta_1\cdot \Delta_2$ means 
a disjoint union of $\Delta_1$ and $\Delta_2$.  
We define \emph{typing judgements} for values $V$
and processes $P$:
%\begin{center}
%\begin{tabular}{c}
	$$\Gamma; \Lambda; \Delta \proves V \hastype U \qquad \qquad \qquad \qquad \qquad \Gamma; \Lambda; \Delta \proves P \hastype \Proc$$
%\end{tabular}
%\end{center}
\noi The first judgement
says that under environments $\Gamma; \Lambda; \Delta$ value $V$
has type $U$; the second judgement says that under
environments $\Gamma; \Lambda; \Delta$ process $P$ has the process type~$\Proc$.
The type soundness result for \HOp (Thm.~\ref{t:sr})
relies on two auxiliary notions on session environments: 
%that contain dual endpoints typed with dual types.
%The following definition ensures two session endpoints 
%are dual each other. 

%\smallskip

\begin{definition}[Session Environments: Balanced/Reduction]\label{d:wtenvred}%\rm
	Let $\Delta$ be a session environment.
	\begin{enumerate}[$\bullet$]
	\item A session environment $\Delta$ is {\em balanced} if whenever
	$s: S_1, \dual{s}: S_2 \in \Delta$ then $S_1 \dualof S_2$.
	\item We define the reduction relation $\red$ on session environments as: %\\ %[-2mm]
\begin{eqnarray*}
	\Delta \cat s: \btout{U} S_1 \cat \dual{s}: \btinp{U} S_2  & \red & 
	\Delta \cat s: S_1 \cat \dual{s}: S_2  \\
	\Delta \cat s: \btsel{l_i: S_i}_{i \in I} \cat \dual{s}: \btbra{l_i: S_i'}_{i \in I} &\red& \Delta \cat s: S_k \cat \dual{s}: S_k' \ (k \in I)
\end{eqnarray*}
\end{enumerate}
\end{definition}

\noi We rely on a typing system is very similar to the one developed in~\cite{tlca07,MostrousY15}. 
We state the type soundness result for \HOp processes;
see \appref{app:types} for details of the associated proofs.

%\smallskip

\begin{theorem}[Type Soundness]\label{t:sr}%\rm
			Suppose $\Gamma; \es; \Delta \proves P \hastype \Proc$
			with
			$\Delta$ balanced. 
			Then $P \red P'$ implies $\Gamma; \es; \Delta'  \proves P' \hastype \Proc$
			and $\Delta = \Delta'$ or $\Delta \red \Delta'$
			with $\Delta'$ balanced. 
\end{theorem}


\begin{example}[Hotel Booking Revisited]\label{exam:type}
We give types to some of the processes of Example~\ref{exam:proc}.
Let us assume $S = \btout{\Quote} \btbra{\accept: \tinact, \reject: \tinact}$ and
$U = \btout{\rtype} \btinp{\Quote} \btsel{\accept: \btout{\creditc} \tinact, \reject: \tinact }$.
The type for $\abs{x}{P}$ is $\es; \es; y: S \proves \abs{x}{P} \hastype \lhot{U}$
and the type for $\Client_1$ is
$~~
	\es; \es; s_1: \btout{\lhot{U}} \tinact \cat s_2: \btout{\lhot{U}} \tinact \proves \Client_1 \hastype \Proc
$.
%
%
%The types for $Q_1$ and $Q_2$ are
%%
%%\begin{eqnarray*}
%$	\es; \es; y: \btout{\Quote} \btinp{\Quote} \tinact \proves \abs{x}{Q_i} \hastype \lhot{U}
%$ ($i=1,2$)
%%	\es; \es; y: \btinp{\Quote} \btout{\Quote} \tinact \proves \abs{x}{Q_2} \hastype \lhot{U}
%%\end{eqnarray*}
%%
%and the type for $\Client_2$ is
%$~~
%	\es; \es; s_1: \btout{\lhot{U}} \tinact \cat s_2: \btout{\lhot{U}} \tinact \proves \Client_2 \hastype \Proc
%$.
%%
\end{example}

%%\begin{definition}[Balanced]\label{d:wtenv}%\rm
%%	%Let $\Delta$ be a session environment.
%%	We say that %a session environment 
%%	$\Delta$ is {\em balanced} if whenever
%%	$s: S_1, \dual{s}: S_2 \in \Delta$ then $S_1 \dualof S_2$.
%%\end{definition}
%%
%%\smallskip
%%
%%\myparagraph{The Typing System of \HOp.}
%Since the typing system is similar to~\cite{tlca07,MostrousY15}, 
%we fully describe it  in \appref{app:types}.  The %rest of the 
%paper can
%be read without knowing the details of the typing system. 
%\jpc{Type soundness relies on the following auxiliary notion}.
%%We list the key properties.
%
%\smallskip
%
%\begin{definition}[Reduction of Session Environment]%\rm
%\label{def:ses_red}
%We define the relation $\red$ on session environments as:\\[-2mm]
%\begin{center}
%\begin{tabular}{l}
%	$\Delta \cat s: \btout{U} S_1 \cat \dual{s}: \btinp{U} S_2 \red
%	\Delta \cat s: S_1 \cat \dual{s}: S_2$\\[1mm]
%	$\Delta \cat s: \btsel{l_i: S_i}_{i \in I} \cat \dual{s}: \btbra{l_i: S_i'}_{i \in I} \red \Delta \cat s: S_k \cat \dual{s}: S_k' \ (k \in I)$
%\end{tabular}
%\end{center}
%%\begin{tabular}{rcl}
%%	\setlength{\tabcolsep}{0pt}
%%	$\Delta \cat s: \btout{U} S_1 \cat \dual{s}: \btinp{U} S_2$ & $\red$ & 
%%	$\Delta \cat s: S_1 \cat \dual{s}: S_2$\\[1mm]
%%	$\Delta \cat s: \btsel{l_i: S_i}_{i \in I} \cat \dual{s}: \btbra{l_i: S_i'}_{i \in I}$ & $\red$ & $\Delta \cat s: S_k \cat \dual{s}: S_k' \ (k \in I)$
%%\end{tabular}
%%\[
%%\begin{array}{rcl}
%%\Delta \cat s: \btout{U} S_1 \cat \dual{s}: \btinp{U} S_2 & \red & 
%%\Delta \cat s: S_1 \cat \dual{s}: S_2\\[1mm]
%%\Delta \cat s: \btsel{l_i: S_i}_{i \in I} \cat \dual{s}: \btbra{l_i: S_i'}_{i \in I} & \red & \Delta \cat s: S_k \cat \dual{s}: S_k' \ (k \in I)
%%\end{array}
%%\]
%\end{definition}
%
%\smallskip
%
%%The following result %Theorem 7.3 in M\&Y
%\noi We state the type soundness results of \HOp: %; it implies 
%%the type soundness of the sub-calculi \HO, \sessp, and $\CAL^{-\mathsf{sh}}$. 
%
%\smallskip
%
%\begin{theorem}[Type Soundness]\label{t:sr}%\rm
%%	\begin{enumerate}[1.]
%%		\item	(Subject Congruence) Suppose $\Gamma; \es; \Delta \proves P \hastype \Proc$.
%%			Then $P \scong P'$ implies $\Gamma; \es; \Delta \proves P' \hastype \Proc$.
%%
%%		\item
%%			(Subject Reduction)
%			Suppose $\Gamma; \es; \Delta \proves P \hastype \Proc$
%			with
%			$\Delta$ balanced. 
%			Then $P \red P'$ implies $\Gamma; \es; \Delta'  \proves P' \hastype \Proc$
%			and $\Delta = \Delta'$ or $\Delta \red \Delta'$
%			with $\Delta'$ balanced. 
%%	\end{enumerate}
%\end{theorem}
