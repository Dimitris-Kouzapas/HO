% !TEX root = main.tex
%\section{Example}


\begin{example}
	Consider process
	$\Gamma; \es; \Delta \cat n: \btout{U} \tinact \proves \bout{n}{\abs{x}{\appl{x}(\abs{y}{\bout{y}{m}} \inact)}} \inact \hastype \Proc$
	with $U = \shot{(\shot{(\shot{(\btout{S} \tinact)})})}$. 
	We will compare with~\cite{JeffreyR05} by counting the steps required to check the bisimilarity
	of this process with itself. Note that
	$\mapchar{\btinp{U} \inact}{s} = \binp{s}{x} \appl{x}{(\abs{y}{\appl{y}{a})}}$, for some fresh $a$.
%
%	\begin{eqnarray*}
%		\mapchar{\btinp{U} \inact}{s}\\
%		= && \binp{s}{x} \mapchar{\shot{(\shot{(\shot{(\btout{S} \tinact)})})}}{x}\\
%		= && \binp{s}{x} \appl{x}{\omapchar{\shot{(\shot{(\btout{S} \tinact)})}}}\\
%		= && \binp{s}{x} \appl{x}{(\abs{y}{\mapchar{\shot{(\btout{S} \tinact)}}{y}})}\\
%		= && \binp{s}{x} \appl{x}{(\abs{y}{\appl{y}{\omapchar{\btout{S} \tinact}})}}\\
%		= && \binp{s}{x} \appl{x}{(\abs{y}{\appl{y}{a})}}\\
%	\end{eqnarray*}
%
%	The characteristic process of
%	the type $\shot{(\shot{(\shot{(\btout{S} \tinact)})})}$ is

\noi In our approach, we would have the following typed transitions:

	\begin{tabular}{rrl}
		(1)& & $\Gamma; \es; \Delta \cat n: \btout{U} \tinact \proves \bout{n}{\abs{x}{\appl{x}(\abs{y}{\bout{y}{m}} \inact)}} \inact$ \\
		&$\by{\bactout{n}{\abs{x}{\appl{x}(\abs{y}{\bout{y}{m}} \inact)}}}$& $\Gamma; \es; \Delta \proves \inact$\\
		(2) & & $\Gamma; \es; \Delta \proves \binp{t}{z} \newsp{s}{\mapchar{\btinp{U} \inact}{s} \Par \bout{\dual{s}}{\abs{x}{\appl{x}(\abs{y}{\bout{y}{m}} \inact)}} \inact}$\\
		&$=$& $\Gamma; \es; \Delta \proves \binp{t}{z} \newsp{s}{\binp{s}{x} \appl{x}{(\abs{y}{\appl{y}{a})}} \Par \bout{\dual{s}}{\abs{x}{\appl{x}(\abs{y}{\bout{y}{m}} \inact)}} \inact}$\\
		(3) &$\by{\bactinp{t}{b}}$& $\Gamma; \es; \Delta \proves \newsp{s}{\binp{s}{x} \appl{x}{(\abs{y}{\appl{y}{a})}} \Par \bout{\dual{s}}{\abs{x}{\appl{x}(\abs{y}{\bout{y}{m}} \inact)}} \inact}$\\
		(4) &$\by{\tau}$& $\Gamma; \es; \Delta \proves \appl{\abs{x}{\appl{x}(\abs{y}{\bout{y}{m} \inact})}}{(\abs{y}{\appl{y}{a})}}$\\
		(5) &$\by{\tau}$& $\Gamma; \es; \Delta \proves \appl{(\abs{y}{\appl{y}{a}})}{(\abs{y}{\bout{y}{m} \inact})} $\\
		(6) &$\by{\tau}$& $\Gamma; \es; \Delta \proves \appl{(\abs{y}{\bout{y}{m} \inact})}{a}$\\
		(7) &$\by{\tau}$& $\Gamma; \es; \Delta \proves \bout{a}{m} \inact$
	\end{tabular}

\noi In the approach defined in~\cite{JeffreyR05} we would have:

	\begin{tabular}{rrl}
		(1)& & $\Gamma; \es; \Delta \cat n: \btout{U} \tinact \proves \bout{n}{\abs{x}{\appl{x}(\abs{y}{\bout{y}{m}} \inact)}} \inact$ \\
		&$\by{\bactout{n}{\abs{x}{\appl{x}(\abs{y}{\bout{y}{m}} \inact)}}}$& $\Gamma; \es; \Delta \proves \inact$\\
		(2) & & $\Gamma; \es; \Delta \proves \repl{} \binp{t}{x} \appl{x}(\abs{y}{\bout{y}{m}} \inact) $\\
		(3) &$\by{\bactinp{t}{\tau_l}}$& $\Gamma; \es; \Delta \proves \repl{} \binp{t}{x} \appl{x}(\abs{y}{\bout{y}{m}} \inact) \Par \appl{\abs{x}{\appl{x}(\abs{y}{\bout{y}{m}} \inact)}}{\tau_l}$\\
		(4) &$\by{\tau}$& $\Gamma; \es; \Delta \proves \repl{} \binp{t}{x} \appl{x}(\abs{y}{\bout{y}{m}} \inact) \Par \appl{\tau_l}{(\abs{y}{\bout{y}{m}} \inact)}$\\
		(5) &$\by{\bactout{l}{\tau_k}}$& $\Gamma; \es; \Delta \proves \repl{} \binp{t}{x} \appl{x}(\abs{y}{\bout{y}{m}} \inact) \Par \repl{} \binp{k}{y} \bout{y}{m} \inact $\\
		(6) &$\by{\bactinp{k}{a}}$& $\Gamma; \es; \Delta \proves \repl{} \binp{t}{x} \appl{x}(\abs{y}{\bout{y}{m}} \inact) \Par \repl{} \binp{k}{y} \bout{y}{m} \inact \Par \appl{\abs{y}{\bout{y}{m} \inact}}{a}$\\
		(7) &$\by{\tau}$& $\Gamma; \es; \Delta \proves \repl{} \binp{t}{x} \appl{x}(\abs{y}{\bout{y}{m}} \inact) \Par \repl{} \binp{k}{y} \bout{y}{m} \inact \Par \bout{a}{m} \inact$
	\end{tabular}
	
\noi This simple example shows how both approaches feature the same number of (typed) transitions.
It is interesting to see how our approach based on refined LTS and characteristic bisimilarity requires less observable actions than that in~\cite{JeffreyR05}.
Also, as we are able to distinguish between linear and shared names, we require less replicated processes than in~\cite{JeffreyR05}. 

%	Comparing
%	\begin{itemize}
%		\item	Same number of transitions.
%		\item	J \& R more observable actions.
%		\item	J \& R replicated processes.
%	\end{itemize}

\end{example}


