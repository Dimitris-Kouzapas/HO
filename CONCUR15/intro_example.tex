\newcommand{\rtype}{\mathsf{roomtype}}
\newcommand{\Quote}{\mathsf{quote}}
\newcommand{\accept}{\mathsf{accept}}
\newcommand{\reject}{\mathsf{reject}}
\newcommand{\creditc}{\mathsf{credit}}

\newcommand{\Client}{\mathsf{Client}}

\begin{example}[Program Equivalence]
\label{ex:hotelbooking}
We introduce the usecase scenario where a Client wants to book
a hotel for her holidays in a remote island
(cf.~\cite{MostrousY15} for a similar usecase).
The Client narrows down her choice to two hotels.
It then requires a quote from the two hotels in order to
make her choice. The Round Trip Time required for
taking quotes from the two hotels in not optimal (cf.~\cite{MostrousY15}),
so she decides to send remote codes to the hotels
to automatically negotiate and book the hotel for
her:
%
\[
	\begin{array}{rcl}
		P &\defeq& \bout{x}{\rtype} \binp{x}{\Quote} \bout{y}{\Quote}
		y \triangleleft \left\{
				\begin{array}{l}
					\accept: \bsel{x}{\accept} \bout{x}{\creditc} \inact,\\
					\reject: \bsel{x}{\reject} \inact
				\end{array}
				\right\}
		\\[6mm]
		\Client_1 &\defeq& \newsp{h_1, h_2}{\bout{s_1}{\abs{x}{P \subst{h_1}{y}}} \bout{s_2}{\abs{x}{P \subst{h_2}{y}}} \inact \Par\\
			&&
			\begin{array}{lll}
				\binp{\dual{h_1}}{x} \binp{\dual{h_2}}{y} & \If\ x \leq y\ \Then & \bsel{\dual{h_1}}{\accept} \bsel{\dual{h_2}}{\reject} \inact\\
				& \Else& \bsel{\dual{h_1}}{\reject} \bsel{\dual{h_2}}{\accept} \inact
			\end{array}
		}
	\end{array}
\]
%
\begin{itemize}
	\item	Process $P$ is the remote code responsible for negotiation with a hotel.
		Channel $x$ is intended to be instatiated by the hotel as the negotiating
		endpoint. Channel $y$ is used to interact with $\Client_1$.

	\item	Process $P$
		i) sends the room requirements to the hotel;
		ii) receives a quote from the hotel;
		iii) sends the quote to the $\Client_1$;
		iv) expects a choice from the $\Client_1$ whether to accept or reject the offer and;
		v) If the choice is accept it informs the hotel and performs the booking,
		if the choice is reject it informs the hotel and ends the session.

	\item	$\Client_1$ instantiates two copies of process $P$ as abstractions
		on session $x$. It further uses two
		fresh endpoints $h_1, h_2$ to substitute channels $y$, respectively,
		in order for the two instances of $P$ to be able to interact
		with $\Client_1$.
	
	\item	$\Client_1$ then sends the two abstractions instances of $P$
		to the two hotels via sessions $s_1$ and $s_2$, respectively.

	\item	$\Client_1$ uses the dual endpoints $\dual{h_1}$ and $\dual{h_2}$
		to receive the negotiation
		result from the two remote instances of $P$ and then inform the two
		processes for the final booking decision.
\end{itemize}

The above scenario does not add a significant gain
to the time needed for the entire protocol to take
place, since the two remote processes are required
to send and receive data to $\Client_1$.

As an alternative we can propose a different implementation
of the same scenario that requires from the two
remote processes to interact with each other,
instead of $\Client_1$, to reach to a consensus:
%
\[
	\begin{array}{rcl}
		P_1 &\defeq&	\bout{x}{\rtype} \binp{x}{\Quote_1} \bout{y}{\Quote_1} \binp{y}{\Quote_2}\\
			&&
				\begin{array}{ll}
					\If\ \Quote_1 \leq \Quote_2\ \Then & \bsel{x}{\accept} \bout{x}{\creditc} \inact\\
					\Else & \bsel{x}{\reject} \inact
				\end{array}
		\\
		P_2 &\defeq&	\bout{x}{\rtype} \binp{x}{\Quote_2} \bout{y}{\Quote_1} \bout{y}{\Quote_1}\\
			&&
				\begin{array}{ll}
					\If\ \Quote_1 \leq \Quote_2\ \Then & \bsel{x}{\accept} \bout{x}{\creditc} \inact\\
					\Else & \bsel{x}{\reject} \inact
				\end{array}
		\\
		\Client_2 &\defeq& \newsp{h}{\bout{s_1}{\abs{x}{P_1 \subst{h}{y}}} \bout{s_2}{\abs{x}{P_2 \subst{\dual{h}}{y}}} \inact}
	\end{array}
\]
\end{example}

\begin{itemize}
	\item	Processes $P_1$ and $P_2$ are responsible for negotiating a quote from the
		hotel in the same fashion as process $P$ in the previous implementation.

	\item	The difference with process $P$ is that the channel $y$ is used for
		interaction between process $P_1$ and $P_2$. Both processes send
		there quotes to each other and then internally follow the same
		logic to reach to a decision.

	\item	The role of $\Client_2$ is to instantiate $P_1$ and $P_2$ as abstractions
		on name $x$. It further substitutes
		the two endpoints of a fresh channel $h$ to channels $y$ respectively,
		in order for the two instances to be able to communicate with each other.

	\item	Process $\Client_2$ then uses sessions $s_1$ and $s_2$ to send the two
		instances of $P_1$ and $P_2$ to the two hotels.
\end{itemize}

We can show that process $\Client_1$ and $\Client_2$
are interchangeable by showing that they are bisimilar.
We use characteristic bisimulation to give a bisimulation
closure.
%
%\begin{eqnarray*}
%	S &=& \set{
%		\Client_1, \Client_2\\
%	&&	\newsp{h_1, h_2}{\bout{s_1}{\abs{x}{P \subst{h_1}{y}}} \bout{s_2}{\abs{x}{P \subst{h_2}{y}}} \inact \Par\\
%	&&
%		\begin{array}{lll}
%			\binp{\dual{h_1}}{x} \binp{\dual{h_2}}{y} & \If\ x \leq y\ \Then & \bsel{\dual{h_1}}{\accept} \bsel{\dual{h_2}}{\reject} \inact\\
%			& \Else& \bsel{\dual{h_1}}{\reject} \bsel{\dual{h_2}}{\accept} \inact
%		\end{array}}
%		}
%\end{eqnarray*}

\subsection{Types}

Assuming $S = \btout{\Quote} \btbra{\accept: \tinact, \reject: \tinact}$ and
$U = \btout{\rtype} \btinp{\Quote} \btsel{\accept: \btout{\creditc} \tinact, \reject: \tinact }$.
The type for $\abs{x}{P}$ is
%
\[
	\es; \es; y: S \proves \abs{x}{P} \hastype \lhot{U}
\]
%
\noi and the type for $\Client_1$ is
%
\[
	\es; \es; s_1: \btout{\lhot{U}} \tinact \cat s_2: \btout{\lhot{U}} \tinact \proves \Client_1 \hastype \Proc
\]
%
The types for $P_1$ and $P_2$ are
%
\begin{eqnarray*}
	\es; \es; y: \btout{\Quote} \btinp{\Quote} \tinact \proves \abs{x}{P_1} \hastype \lhot{U}\\
	\es; \es; y: \btinp{\Quote} \btout{\Quote} \tinact \proves \abs{x}{P_2} \hastype \lhot{U}
\end{eqnarray*}
%
\noi ant the type for $\Client_2$ is
%
\[
	\es; \es; s_1: \btout{\lhot{U}} \tinact \cat s_2: \btout{\lhot{U}} \tinact \proves \Client_2 \hastype \Proc
\]
%
