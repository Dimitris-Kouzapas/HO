% !TEX root = main.tex
\noindent 
We introduce the syntax and semantics of the 
\emph{Higher-Order Session $\pi$-Calculus} (\HOp).
\HOp includes both name- and abstraction-passing, shared and session communication,  
as well as recursion; it is 
essentially 
the process language
proposed 
in~\cite{tlca07} (where tractable bisimilarities are not studied). 
\smallskip

%Following the literature~\cite{JeffreyR05},
%for simplicity of the presentation
%we concentrate on the second-order call-by-value \HOp.  
%(In \secref{sec:extension} we consider extensions of 
%\HOp with higher-order abstractions 
%and polyadicity in name-passing/abstractions.)
%We also introduce two subcalculi of \HOp. In particular, we define the 
%core higher-order session
%calculus (\HO), which 
%%. The \HO calculus is  minimal: it 
%includes constructs for shared name synchronisation and 
%%constructs for session establish\-ment/communication and 
%(monadic) name-abstraction, but lacks name-passing and recursion.

%Although minimal, in \secref{s:expr}
%the abstraction-passing capabilities of \HOp will prove 
%expressive enough to capture key features of session communication, 
%such as delegation and recursion.

%\subsection{Syntax}
%\label{subsec:syntax}
%%%%%%%%%%%%%%%%%%%%%% HOp Syntax Figure %%%%%%%%%%%%%%%%%%%%%%%%%%%%%%%% 
%\begin{figure}[t]
%	\[
%		\begin{array}{rcl}
%			u,w &\bnfis& n \bnfbar x,y,z \qquad
%			n \bnfis a,b  \bnfbar s, \dual{s} \qquad 
%			V,W  \bnfis u \bnfbar \abs{x}{P} \\[1mm]
%			P,Q & \bnfis & \bout{u}{V}{P}  \bnfbar  \binp{u}{x}{P} \bnfbar
%			\bsel{u}{l} P \bnfbar \bbra{u}{l_i:P_i}_{i \in I}   \\[1mm]
%			& \bnfbar & \rvar{X} \bnfbar \recp{X}{P} \bnfbar \appl{V}{W} \bnfbar P\Par Q \bnfbar \news{n} P \bnfbar \inact
%		\end{array}
%	\]
%	\caption{Syntax of \HOp.}
%	\label{fig:syntax}
%%\Hlinefig
%\end{figure}
%%%%%%%%%%%%%%%%%%%%%% End HOp Syntax Figure %%%%%%%%%%%%%%%%%%%%%%%%%%%%


	\begin{figure}
	\[
		\begin{array}{rcl}
			u,w &\bnfis& n \bnfbar x,y,z \qquad
			n \bnfis a,b  \bnfbar s, \dual{s} \qquad 
			V,W  \bnfis u \bnfbar \abs{x}{P} \\[1mm]
			P,Q & \bnfis & \bout{u}{V}{P}  \bnfbar  \binp{u}{x}{P} \bnfbar
			\bsel{u}{l} P \bnfbar \bbra{u}{l_i:P_i}_{i \in I}   \\[1mm]
			& \bnfbar & \rvar{X} \bnfbar \recp{X}{P} \bnfbar \appl{V}{W} \bnfbar P\Par Q \bnfbar \news{n} P \bnfbar \inact
		\end{array}
	\]
	\[
	\begin{array}{c}
		P \Par \inact \scong P
		\quad
		P_1 \Par P_2 \scong P_2 \Par P_1
		\quad
		P_1 \Par (P_2 \Par P_3) \scong (P_1 \Par P_2) \Par P_3
		\quad 
		\recp{X}{P} \scong P\subst{\recp{X}{P}}{\rvar{X}}
		\\%[1mm]

		\news{n} \inact \scong \inact
		\qquad 
		P \Par \news{n} Q \scong \news{n}(P \Par Q)
		\	(n \notin \fn{P})
		\qquad
		P \scong Q \textrm{ if } P \scong_\alpha Q
	\end{array}
\]
	\[
		\!\!\!\begin{array}{lllcrll}
			\orule{App} & (\abs{x}{P}) \, V   \red    P \subst{V}{x}
			& 
			  

			\orule{Pass} & \bout{n}{V} P \Par \binp{\dual{n}}{x} Q   \red   P \Par Q \subst{V}{x} 
			
			\\[1mm]

			 \orule{Res} & P \red P'  \Rightarrow  \news{n} P  \red  \news{n} P' 

			&  
			\orule{Sel}
			&  \!\!\! \bsel{n}{l_j} Q \Par \bbra{\dual{n}}{l_i : P_i}_{i \in I}  \red   Q \Par P_j ~~(j \in I)
			
			\\[1mm]
			\orule{Par} & P \red P'   \Rightarrow    P \Par Q  \red   P' \Par Q  
			&  
			\orule{Cong} & P \scong Q \red Q' \scong P'   \Rightarrow  P  \red  P' 
	\end{array}
	\]
	\vspace{-3mm}
\caption{$\HOp$: Syntax and Operational Semantics (Structural Congruence and Reduction Relation).
\label{fig:redsem}}
%\Hlinefig
\end{figure}

\noindent\myparagraph{Syntax.} 
The syntax of \HOp is defined in \figref{fig:redsem} (upper part).
We use $a,b,c, \dots$ (resp.~$s, \dual{s}, \dots$) 
to range over shared (resp. session) names. 
We use $m, n, t, \dots$ for session or shared names. 
We define the dual operation over names $n$ as $\dual{n}$ with
$\dual{\dual{s}} = s$ and $\dual{a} = a$.
Intuitively, names $s$ and $\dual{s}$ are dual (two) \emph{endpoints} while 
shared names represent non-deterministic points. 
Variables are denoted with $x, y, z, \dots$, 
and recursive variables are denoted with $\varp{X}, \varp{Y} \dots$.
An abstraction %(or higher-order value) 
$\abs{x}{P}$ is a process $P$ with name parameter $x$.
%Symbols $u, v, \dots$ range over identifiers; and  $V, W, \dots$ to denote values. 
Values $V,W$ include 
identifiers $u, v, \ldots$ %(first-order values) 
and 
abstractions $\abs{x}{P}$ (first- and higher-order values, resp.). 
{Terms} 
include $\pi$-calculus constructs for sending/receiving values $V$.
Process $\bout{u}{V} P$ denotes the output of   $V$
over name $u$, with continuation $P$;
process $\binp{u}{x} P$ denotes the input prefix on name $u$ of a value
that 
will substitute variable $x$ in continuation $P$. 
Recursion is expressed by $\recp{X}{P}$,
which binds the recursive variable $\varp{X}$ in process $P$.
Process 
%ny
%$\appl{x}{u}$ 
$\appl{V}{W}$ 
is the application
which substitutes values $W$ on the abstraction~$V$. 
\dk{Typing  ensures \jpc{that} $V$ is not a name.}
Processes $\bbra{u}{l_i: P_i}_{i \in I}$  and $\bsel{u}{l} P$ define labeled choice:
given a finite index set $I$, process $\bbra{u}{l_i: P_i}_{i \in I}$ offers a choice 
among processes with pairwise distinct labels;
%on labels $l_i$ with continuation $P_i$, given that $i \in I$.
 process $\bsel{u}{l} P$ selects label $l$ on name $u$ and then behaves as $P$.
%Given $i \in I$ 
%Others are standard c
Constructs for 
inaction $\inact$,  parallel composition $P_1 \Par P_2$, and 
name restriction $\news{n} P$ are standard.
Session name restriction $\news{s} P$ simultaneously binds endpoints $s$ and $\dual{s}$ in $P$.
%A well-formed process relies on assumptions for
%guarded recursive processes.
We use $\fv{P}$ and $\fn{P}$ to denote a set of free 
%\jpc{recursion}
variables and names; 
and assume $V$ in $\bout{u}{V}{P}$ does not include free recursive 
variables $\rvar{X}$. 
If $\fv{P} = \emptyset$, we call $P$ {\em closed}.
%; and closed $P$ without 
%free session names a {\em program}. 

%\subsection{Subcalculi of \HOp}
%\label{subsec:subcalculi}
%\noi
%We define two subcalculi of \HOp. 
%%We now define several sub-calculi of \HOp. 
%The first is the 
%{\em core higher-order session calculus} (denoted \HO),
%which lacks recursion and name passing; its 
%formal syntax is obtained from \figref{fig:syntax} by excluding 
%constructs in \nonhosyntax{\text{grey}}.
%The second subcalculus is 
%the {\em session $\pi$-calculus} 
%(denoted $\sessp$), which 
%lacks  the
%higher-order constructs
%(i.e., abstraction passing and application), but includes recursion.
%Let $\CAL \in \{\HOp, \HO, \sessp\}$. We write 
%$\CAL^{-\mathsf{sh}}$ for $\CAL$ without shared names
%(we delete $a,b$ from $n$). 
%We shall demonstrate that 
%$\HOp$, $\HO$, and $\sessp$ have the same expressivity.


%	\begin{figure}
%	\[
%		\begin{array}{rcl}
%			u,w &\bnfis& n \bnfbar x,y,z \qquad
%			n \bnfis a,b  \bnfbar s, \dual{s} \qquad 
%			V,W  \bnfis u \bnfbar \abs{x}{P} \\[1mm]
%			P,Q & \bnfis & \bout{u}{V}{P}  \bnfbar  \binp{u}{x}{P} \bnfbar
%			\bsel{u}{l} P \bnfbar \bbra{u}{l_i:P_i}_{i \in I}   \\[1mm]
%			& \bnfbar & \rvar{X} \bnfbar \recp{X}{P} \bnfbar \appl{V}{W} \bnfbar P\Par Q \bnfbar \news{n} P \bnfbar \inact
%		\end{array}
%	\]
%	\[
%
%		\!\!\!\begin{array}{lllcrll}
%			\orule{App} & (\abs{x}{P}) \, V   \red    P \subst{V}{x}
%			& 
%			  
%
%			\orule{Pass} & \bout{n}{V} P \Par \binp{\dual{n}}{x} Q   \red   P \Par Q \subst{V}{x} 
%			
%			\\[1mm]
%
%			 \orule{Res} & P \red P'  \Rightarrow  \news{n} P  \red  \news{n} P' 
%
%			&  
%			\orule{Sel}
%			&  \!\!\! \bsel{n}{l_j} Q \Par \bbra{\dual{n}}{l_i : P_i}_{i \in I}  \red   Q \Par P_j ~~(j \in I)
%			
%			\\[1mm]
%			\orule{Par} & P \red P'   \Rightarrow    P \Par Q  \red   P' \Par Q  
%			&  
%			\orule{Cong} & P \scong Q \red Q' \scong P'   \Rightarrow  P  \red  P' 
%	\end{array}
%	\]
%\[
%	\begin{array}{c}
%		P \Par \inact \scong P
%		\quad
%		P_1 \Par P_2 \scong P_2 \Par P_1
%		\quad
%		P_1 \Par (P_2 \Par P_3) \scong (P_1 \Par P_2) \Par P_3
%		\quad 
%		\recp{X}{P} \scong P\subst{\recp{X}{P}}{\rvar{X}}
%		\\%[1mm]
%
%		\news{n} \inact \scong \inact
%		\qquad 
%		P \Par \news{n} Q \scong \news{n}(P \Par Q)
%		\	(n \notin \fn{P})
%		\qquad
%		P \scong Q \textrm{ if } P \scong_\alpha Q
%
%%		\qquad
%%		\dk{V \scong W \textrm{ if } V \scong_\alpha W
%%		\quad \abs{x}{P} \scong \abs{x}{Q} \textrm{ if } P \scong Q}
%	\end{array}
%\]
%\caption{Operational Semantics of $\HOp$. 
%\label{fig:reduction}}
%\Hlinefig
%\end{figure}

\smallskip
\noindent \myparagraph{Operational Semantics.}
%	\label{subsec:semantics}
	\figref{fig:redsem} (lower part) defines the operational semantics 
of \HOp.
We define a reduction relation which relies 
on 
a \emph{structural congruence} $\scong$, in the standard way. % are defined in \figref{fig:reduction} (bottom). 
\jpc{We assume the expected extension of $\scong$ to values $V$.}
Reduction is denoted $\red$; some intuitions on the rules in \figref{fig:redsem} (lower part) follow.
Rule~$\orule{App}$ is a value application; 
rule~$\orule{Pass}$ defines a shared interaction at $n$ 
(\jpc{with} $\dual{n}=n$) or a session interaction;  
rule~$\orule{Sel}$ is the standard rule for labelled choice/selection:
given an index set $I$, 
a process selects label $l_j$ on name $n$ over a set of
labels $\set{l_i}_{i \in I}$ offered by a branching 
on the dual endpoint $\dual{n}$; and other rules are standard.
We write $\red^\ast$ for a multi-step reduction. 


%\begin{example}
\begin{example}[Hotel Booking Scenario]\label{exam:proc}
To illustrate \HOp and its expressive power, 
we consider a usecase scenario that adapts the example given by Mostrous and Yoshida~\cite{MostrousY15}.
The scenario involves a $\Client$ process that wants to book
a hotel room. % for her holidays. % in a remote island
%The Client 
$\Client$
narrows  her choice down to two hotels, and requires 
 a quote from the two in order to
decide. The round-trip time (RTT) required for
taking quotes from the two hotels in not optimal, % (cf.~\cite{MostrousY15}),
so she  sends mobile codes to both hotels
to automatically negotiate and book a room. 
We now present two \HOp implementations of this scenario.
For convenience, we write $\If e\ \Then (P\ \Else \ Q)$ 
to denote a conditional process (encodable using labeled choice)
that executes $P$ or $Q$ depending on boolean expression $e$.
The first implementation is  as follows:
%
%\[
	\begin{eqnarray*}%{rcl}
		 P  \!\!\! & \defeq &  \!\!\! \bout{x}{\rtype} \binp{x}{\Quote} \bout{y}{\Quote}
		y \triangleleft \left\{
				\begin{array}{l}
					\accept: \bsel{x}{\accept} \bout{x}{\creditc} \inact,\\
					\reject: \bsel{x}{\reject} \inact
				\end{array}
				\right\}
		\\ %[3mm]
		 \Client_1 \!\!\! & \defeq  &  \!\!\! \newsp{h_1, h_2}{\bout{s_1}{\abs{x}{P \subst{h_1}{y}}} \bout{s_2}{\abs{x}{P \subst{h_2}{y}}} \inact \Par  \\
		& & 
		\!\!\! \binp{\dual{h_1}}{x} \binp{\dual{h_2}}{y}  \If\ x \leq y\   \Then (\bsel{\dual{h_1}}{\accept} \bsel{\dual{h_2}}{\reject} \inact \ \Else \ \bsel{\dual{h_1}}{\reject} \bsel{\dual{h_2}}{\accept} \inact )
		}
	\end{eqnarray*}
%\]
%
Process $\Client_1$ sends two abstractions with body $P$, one to each hotel, 
		using sessions $s_1$ and $s_2$.
	Process $P$ is the mobile code responsible for negotiating with a hotel.
		Name $x$ in $P$ is meant to be instantiated by the hotel as the negotiating
		endpoint. Channel $y$ is used to interact with $\Client_1$.	
		We have that $P$: (i)  sends the room requirements to the hotel;
		(ii) receives a quote from the hotel;
		(iii) sends the quote to  $\Client_1$;
		(iv) expects a choice from   $\Client_1$ whether to accept or reject the offer;
		(v) if the choice is accept then it informs the hotel and performs the booking;
		otherwise, if the choice is reject then it informs the hotel and ends the session.
				$\Client_1$ instantiates two copies of  $P$ as abstractions
		on session $x$. It uses two
		fresh endpoints $h_1, h_2$ to substitute channel $y$
		in $P$. This enables communication with the mobile code(s).
		In fact, 
		$\Client_1$ uses the dual endpoints $\dual{h_1}$ and $\dual{h_2}$
		to receive the negotiation
		result from the two remote instances of $P$ and then inform the two
		processes for the final booking decision.

		Notice that the above implementation does not add a significant gain
to the time needed for the entire protocol to take
place, since the two remote processes are required
to send and receive data to $\Client_1$.
We present now an alternative implementation
of the same scenario, in which two mobile processes are meant 
to interact with each other (rather than with $\Client_1$) to reach to a consensus:
%
\[
	\begin{array}{rcl}
	    R & \defeq & \If\ \Quote_1 \leq \Quote_2 \, \Then  (\bsel{x}{\accept} \bout{x}{\creditc} \inact \  \Else \ \bsel{x}{\reject} \inact) \\
		Q_1 &\defeq&	\bout{x}{\rtype} \binp{x}{\Quote_1} \bout{y}{\Quote_1} \binp{y}{\Quote_2} R \\
		Q_2 &\defeq&	\bout{x}{\rtype} \binp{x}{\Quote_1} \binp{y}{\Quote_2} \bout{y}{\Quote_1} R \\
%			&&
%				\begin{array}{ll}
%					\If\ \Quote_1 \leq \Quote_2 &\Then  \bsel{x}{\accept} \bout{x}{\creditc} \inact \  \Else \ \bsel{x}{\reject} \inact %\\
%				%	 & \Else \bsel{x}{\reject} \inact
%				\end{array}
%		\\
%		Q_2 &\defeq&	\bout{x}{\rtype} \binp{x}{\Quote_1} \binp{y}{\Quote_2} \bout{y}{\Quote_1}\\
%			&&
%				\begin{array}{ll}
%					\If\ \Quote_1 \leq \Quote_2  & \Then \bsel{x}{\accept} \bout{x}{\creditc} \inact\\
%					 & \Else \bsel{x}{\reject} \inact
%				\end{array}
%		\\
		\Client_2 &\defeq& \newsp{h}{\bout{s_1}{\abs{x}{Q_1 \subst{h}{y}}} \bout{s_2}{\abs{x}{Q_2 \subst{\dual{h}}{y}}} \inact}
	\end{array}
\]
%\end{example}
Processes $Q_1$ and $Q_2$  negotiate a quote from the
		hotel in the same fashion as process $P$ in $\Client_1$.
%		Notice that $Q_2$ is defined exactly as $Q_1$ except for the sequence of messages on~$y$:
%		rather than 
%		sending $\Quote_1$ first and receiving $\Quote_2$ later, 
%		process $Q_2$ receives $\Quote_2$ first and sends $\Quote_1$ later.
		The key difference with respect to $P$ is that $y$ is used for
		interaction between process $Q_1$ and $Q_2$. Both processes send
		their quotes to each other and then internally follow the same
		logic to reach to a decision.
		Process  $\Client_2$ then uses sessions $s_1$ and $s_2$ to send the two
		instances of $Q_1$ and $Q_2$ to the two hotels, using them 
	 as abstractions
		on name $x$. It further substitutes
		the two endpoints of a fresh channel $h$ to channels $y$ respectively,
		in order for the two instances to communicate with each other.



%\begin{itemize}
%	\item	Processes $P_1$ and $P_2$ are responsible for negotiating a quote from the
%		hotel in the same fashion as process $P$ in the previous implementation.
%
%	\item	The difference with process $P$ is that the channel $y$ is used for
%		interaction between process $P_1$ and $P_2$. Both processes send
%		there quotes to each other and then internally follow the same
%		logic to reach to a decision.
%
%	\item	The role of $\Client_2$ is to instantiate $P_1$ and $P_2$ as abstractions
%		on name $x$. It further substitutes
%		the two endpoints of a fresh channel $h$ to channels $y$ respectively,
%		in order for the two instances to be able to communicate with each other.
%
%	\item	Process $\Client_2$ then uses sessions $s_1$ and $s_2$ to send the two
%		instances of $P_1$ and $P_2$ to the two hotels.
%\end{itemize}

Differences between $\Client_1$ and $\Client_2$ are more easily seen in the sequence diagrams of \figref{fig:exam}. 
We will assign session types to these client processes in Example \ref{exam:type}.
Later on, we will show that they are behaviorally equivalent using characteristic bisimilarity.
\begin{figure}

\newcommand{\Hotel}{\mathsf{Hotel}}
\newcommand{\Code}{\mathsf{Code}}

%\makeatletter
%\newcommand{\gettikzxy}[3]{%
%  \tikz@scan@one@point\pgfutil@firstofone#1\relax
%  \edef#2{\the\pgf@x}%
%  \edef#3{\the\pgf@y}%
%}
%\makeatother
%	\gettikzxy{(Client1.south)}{\ax}{\ay}
%	\draw[dashed]			(Client1.south) -- (\ax, 0);


%\begin{center}

\begin{tikzpicture}
%	\draw[help lines]		(0, 0) grid (13, 10);

	%%%%%%%%%%%%%%%%%%%%% Scenario 1

	%%%% Nodes
	\node	(Client1)	at	(0, 10) {\footnotesize $\Client_1$};
	\node	(Hotel1)	at	(2.5, 10) {\footnotesize $\Hotel_1$};
	\node	(Hotel2)	at	(5, 10) {\footnotesize $\Hotel_2$};

	\node	(Code1)		at	(1.25, 8.8) {\scriptsize $\Code_1$};
	\node	(Code2)		at	(3.75, 8.8) {\scriptsize $\Code_2$};

	%%%% Lines for Nodes
%	\draw[dashed]		(Client1.south west) -- (Client1.south east);
	\draw
		let
			\p1 = (Client1.south),
			\p2 = (Hotel1.south),
			\p3 = (Hotel2.south),
			\p4 = (Code1.south),
			\p5 = (Code2.south)
		in
			(\x1, \y1) -- (\x1, 4.8)
			(\x2, \y2) -- (\x2, 4.8)
			(\x3, \y3) -- (\x3, 4.8)
			(\x4, \y4) -- (\x4, 4.8)
			(\x5, \y5) -- (\x5, 4.8);

	%%%% Arrows
	\draw[->]
		let
			\p1 = (Client1),
			\p2 = (Hotel1)
		in
			(\x1, 9.6) to node[above] {\scriptsize $\abs{x}{P_{xy}}$} (\x2, 9.6);

	\draw[->]
		let
			\p1 = (Client1),
			\p2 = (Hotel2)
		in
			(\x1, 9.2) to node[above] {\qquad \qquad \scriptsize $\abs{x}{P_{xy}}$} (\x2, 9.2);

	\draw[->]
		let
			\p1 = (Hotel1)
		in
			(\x1, 9.6) -- (Code1.north);

	\draw[->]
		let
			\p1 = (Hotel2)
		in
			(\x1, 9.2) -- (Code2.north);


	\draw[->]
		let
			\p1 = (Code1),
			\p2 = (Hotel1)
		in
			(\x1, 8.4) to node[above] {\scriptsize $\rtype$} (\x2, 8.4);

	\draw[->]
		let
			\p1 = (Code1),
			\p2 = (Hotel1)
		in
			(\x2, 8) to node[above] {\scriptsize $\Quote$} (\x1, 8);

	\draw[->]
		let
			\p1 = (Code2),
			\p2 = (Hotel2)
		in
			(\x1, 8.4) to node[above] {\scriptsize $\rtype$} (\x2, 8.4);
	\draw[->]
		let
			\p1 = (Code2),
			\p2 = (Hotel2)
		in
			(\x2, 8) to node[above] {\scriptsize $\Quote$} (\x1, 8);

	\draw[->]
		let
			\p1 = (Code1),
			\p2 = (Client1)
		in
			(\x1, 7.6) to node[above] {\scriptsize $\Quote$} (\x2, 7.6);

	\draw[->]
		let
			\p1 = (Code2),
			\p2 = (Client1)
		in
			(\x1, 7.2) to node[above] {\scriptsize $\Quote$} (\x2, 7.2);


	%%%% Choice 
	%% Client1 --> Code1
	\draw[dashed]
		let
			\p1 = (Client1),
			\p2 = (Code1)
		in
			(-0.4, 6.8) to node {$\oplus$} (-0.4, 5.4);
%			(\x1, 4.8) -- (\x2, 4.8)
%			(\x1, 2.8) -- (\x2, 2.8);

	\draw[dashed, ->]
		let
			\p1 = (Client1),
			\p2 = (Code1)
		in
			(-0.4, 6.8) to node[above] {\scriptsize $\accept$} (\x2, 6.8);

%	\draw[dashed]
%		let
%			\p1 = (Code1),
%			\p2 = (Hotel1)
%		in
%			(\x1, 6.4) -- (\x2, 6.4)
%			(\x1, 5.2) -- (\x2, 5.2)
%			(\x1, 4) -- (\x2, 4)
%			(\x1, 3.2) -- (\x2, 3.2);

	\draw[->,dashed]
		let
			\p1 = (Code1),
			\p2 = (Hotel1)
		in
			(\x1, 6.2) to node[above] {\scriptsize $\accept$} (\x2, 6.2);

	\draw[->]
		let
			\p1 = (Code1),
			\p2 = (Hotel1)
		in
			(\x1, 5.8) to node[above] {\scriptsize $\creditc$} (\x2, 5.8);

	\draw[dashed, ->]
		let
			\p1 = (Client1),
			\p2 = (Code1)
		in
			(-0.4, 5.4) to node[above] {\scriptsize $\reject$} (\x2, 5.4);

	\draw[dashed, ->]
		let
			\p1 = (Code1),
			\p2 = (Hotel1)
		in
			(\x1, 4.8) to node[above] {\scriptsize $\reject$} (\x2, 4.8);


	%%%% Choice 
	%% Client1 --> Code2
	\draw[dashed]
		let
			\p1 = (Client1),
			\p2 = (Code2)
		in
			(-0.2, 6.6) to node {$\oplus$} (-0.2, 5.2);
%			(\x1, 4.8) -- (\x2, 4.8)
%			(\x1, 2.8) -- (\x2, 2.8);

	\draw[dashed, ->]
		let
			\p1 = (Client1),
			\p2 = (Code2)
		in
			(-0.2, 6.6) to node[above] {\scriptsize $\accept$} (\x2, 6.6);

%	\draw[dashed]
%		let
%			\p1 = (Code1),
%			\p2 = (Hotel1)
%		in
%			(\x1, 6.4) -- (\x2, 6.4)
%			(\x1, 5.2) -- (\x2, 5.2)
%			(\x1, 4) -- (\x2, 4)
%			(\x1, 3.2) -- (\x2, 3.2);

	\draw[->,dashed]
		let
			\p1 = (Code2),
			\p2 = (Hotel2)
		in
			(\x1, 6.2) to node[above] {\scriptsize $\accept$} (\x2, 6.2);

	\draw[->]
		let
			\p1 = (Code2),
			\p2 = (Hotel2)
		in
			(\x1, 5.8) to node[above] {\scriptsize $\creditc$} (\x2, 5.8);

	\draw[dashed, ->]
		let
			\p1 = (Client1),
			\p2 = (Code2)
		in
			(-0.2, 5.2) to node[above] {\scriptsize $\reject$} (\x2, 5.2);

	\draw[dashed, ->]
		let
			\p1 = (Code2),
			\p2 = (Hotel2)
		in
			(\x1, 4.8) to node[above] {\scriptsize $\reject$} (\x2, 4.8);


	%%%%%%%%%%%%%%%%%%%%% Scenario 2

	%%%% Nodes
	\node	(Client1)	at	(6.5, 10) {\footnotesize $\Client_2$};
	\node	(Hotel1)	at	(9, 10) {\footnotesize $\Hotel_1$};
	\node	(Hotel2)	at	(11.5, 10) {\footnotesize $\Hotel_2$};

	\node	(Code1)		at	(7.75, 8.8) {\scriptsize $\Code_1$};
	\node	(Code2)		at	(10.25, 8.8) {\scriptsize $\Code_2$};
%	\node	(Client1)	at	(7.5, 10) {\footnotesize $\Client_2$};
%	\node	(Hotel1)	at	(10, 10) {\footnotesize $\Hotel_1$};
%	\node	(Hotel2)	at	(12.5, 10) {\footnotesize $\Hotel_2$};
%
%	\node	(Code1)		at	(8.75, 8.8) {\scriptsize $\Code_1$};
%	\node	(Code2)		at	(11.25, 8.8) {\scriptsize $\Code_2$};


	\draw
		let
			\p1 = (Client1.south),
			\p2 = (Hotel1.south),
			\p3 = (Hotel2.south),
			\p4 = (Code1.south),
			\p5 = (Code2.south)
		in
			(\x1, \y1) -- (\x1, 4.8)
			(\x2, \y2) -- (\x2, 4.8)
			(\x3, \y3) -- (\x3, 4.8)
			(\x4, \y4) -- (\x4, 4.8)
			(\x5, \y5) -- (\x5, 4.8);

	%%%% Arrows
	\draw[->]
		let
			\p1 = (Client1),
			\p2 = (Hotel1)
		in
			(\x1, 9.6) to node[above] {\scriptsize $\abs{x}{Q_1}$} (\x2, 9.6);

	\draw[->]
		let
			\p1 = (Client1),
			\p2 = (Hotel2)
		in
			(\x1, 9.2) to node[above] {\qquad \qquad \scriptsize $\abs{x}{Q_2}$} (\x2, 9.2);

	\draw[->]
		let
			\p1 = (Hotel1)
		in
			(\x1, 9.6) -- (Code1.north);

	\draw[->]
		let
			\p1 = (Hotel2)
		in
			(\x1, 9.2) -- (Code2.north);


	\draw[->]
		let
			\p1 = (Code1),
			\p2 = (Hotel1)
		in
			(\x1, 8.4) to node[above] {\scriptsize $\rtype$} (\x2, 8.4);

	\draw[->]
		let
			\p1 = (Code1),
			\p2 = (Hotel1)
		in
			(\x2, 8) to node[above] {\scriptsize $\Quote$} (\x1, 8);

	\draw[->]
		let
			\p1 = (Code2),
			\p2 = (Hotel2)
		in
			(\x1, 8.4) to node[above] {\scriptsize $\rtype$} (\x2, 8.4);
	\draw[->]
		let
			\p1 = (Code2),
			\p2 = (Hotel2)
		in
			(\x2, 8) to node[above] {\scriptsize $\Quote$} (\x1, 8);

	\draw[->]
		let
			\p1 = (Code1),
			\p2 = (Code2)
		in
			(\x1, 7.6) to node[above] {\qquad \qquad \scriptsize $\Quote$} (\x2, 7.6);

	\draw[->]
		let
			\p1 = (Code2),
			\p2 = (Code1)
		in
			(\x1, 7.2) to node[above] {\scriptsize $\Quote$ \qquad \qquad \qquad} (\x2, 7.2);


	%%%% Choice
	% Client1 --> Hotel1
	\draw[dashed]
		let
			\p1 = (Code1),
			\p2 = (Hotel1)
		in
			(7.35, 6.8) to node {$\oplus$} (7.35, 6);
%			(\x1, 5.6) -- (\x2, 5.6)
%			(\x1, 4.8) -- (\x2, 4.8);

	\draw[dashed, ->]
		let
			\p1 = (Code1),
			\p2 = (Hotel1)
		in
			(7.35, 6.8) to node[above] {\scriptsize  $\accept$} (\x2, 6.8);

	\draw[->]
		let
			\p1 = (Code1),
			\p2 = (Hotel1)
		in
			(\x1, 6.4) to node[above] {\scriptsize $\creditc$} (\x2, 6.4);

	\draw[dashed, ->]
		let
			\p1 = (Code1),
			\p2 = (Hotel1)
		in
			(7.35, 6) to node[above] {\scriptsize $\reject$} (\x2, 6);

	% Client2 --> Hotel2
	\draw[dashed]
		let
			\p1 = (Code2),
			\p2 = (Hotel2)
		in
			(9.85, 6.8) to node {$\oplus$} (9.85, 6);
%			(\x1, 5.6) -- (\x2, 5.6)
%			(\x1, 4.8) -- (\x2, 4.8);

	\draw[dashed, ->]
		let
			\p1 = (Code2),
			\p2 = (Hotel2)
		in
			(9.85, 6.8) to node[above] {\scriptsize $\accept$} (\x2, 6.8);

	\draw[->]
		let
			\p1 = (Code2),
			\p2 = (Hotel2)
		in
			(\x1, 6.4) to node[above] {\scriptsize $\creditc$} (\x2, 6.4);

	\draw[dashed, ->]
		let
			\p1 = (Code2),
			\p2 = (Hotel2)
		in
			(9.85, 6) to node[above] {\scriptsize $\reject$} (\x2, 6);

\end{tikzpicture}
%\end{center}

\caption{Sequence diagram for $\Client_1$ and $\Client_2$ as in Example~\ref{exam:proc}\label{fig:exam}.}
\end{figure}
\end{example}
