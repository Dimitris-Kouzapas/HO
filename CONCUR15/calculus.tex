% !TEX root = main.tex
\noindent 
We introduce the syntax and semantics of the 
\emph{Higher-Order Session $\pi$-Calculus} (\HOp).
\HOp includes both name- and abstraction-passing, shared and session communication,  
as well as recursion; it is 
essentially 
the process language
proposed 
in~\cite{tlca07} (where tractable bisimilarities are not studied). 
\smallskip

%Following the literature~\cite{JeffreyR05},
%for simplicity of the presentation
%we concentrate on the second-order call-by-value \HOp.  
%(In \secref{sec:extension} we consider extensions of 
%\HOp with higher-order abstractions 
%and polyadicity in name-passing/abstractions.)
%We also introduce two subcalculi of \HOp. In particular, we define the 
%core higher-order session
%calculus (\HO), which 
%%. The \HO calculus is  minimal: it 
%includes constructs for shared name synchronisation and 
%%constructs for session establish\-ment/communication and 
%(monadic) name-abstraction, but lacks name-passing and recursion.

%Although minimal, in \secref{s:expr}
%the abstraction-passing capabilities of \HOp will prove 
%expressive enough to capture key features of session communication, 
%such as delegation and recursion.

%\subsection{Syntax}
%\label{subsec:syntax}
%%%%%%%%%%%%%%%%%%%%%% HOp Syntax Figure %%%%%%%%%%%%%%%%%%%%%%%%%%%%%%%% 
%\begin{figure}[t]
%	\[
%		\begin{array}{rcl}
%			u,w &\bnfis& n \bnfbar x,y,z \qquad
%			n \bnfis a,b  \bnfbar s, \dual{s} \qquad 
%			V,W  \bnfis u \bnfbar \abs{x}{P} \\[1mm]
%			P,Q & \bnfis & \bout{u}{V}{P}  \bnfbar  \binp{u}{x}{P} \bnfbar
%			\bsel{u}{l} P \bnfbar \bbra{u}{l_i:P_i}_{i \in I}   \\[1mm]
%			& \bnfbar & \rvar{X} \bnfbar \recp{X}{P} \bnfbar \appl{V}{W} \bnfbar P\Par Q \bnfbar \news{n} P \bnfbar \inact
%		\end{array}
%	\]
%	\caption{Syntax of \HOp.}
%	\label{fig:syntax}
%%\Hlinefig
%\end{figure}
%%%%%%%%%%%%%%%%%%%%%% End HOp Syntax Figure %%%%%%%%%%%%%%%%%%%%%%%%%%%%


	\begin{figure}
	\[
		\begin{array}{rcl}
			u,w &\bnfis& n \bnfbar x,y,z \qquad
			n \bnfis a,b  \bnfbar s, \dual{s} \qquad 
			V,W  \bnfis u \bnfbar \abs{x}{P} \\[1mm]
			P,Q & \bnfis & \bout{u}{V}{P}  \bnfbar  \binp{u}{x}{P} \bnfbar
			\bsel{u}{l} P \bnfbar \bbra{u}{l_i:P_i}_{i \in I}   \\[1mm]
			& \bnfbar & \rvar{X} \bnfbar \recp{X}{P} \bnfbar \appl{V}{W} \bnfbar P\Par Q \bnfbar \news{n} P \bnfbar \inact
		\end{array}
	\]
	\[
	\begin{array}{c}
		P \Par \inact \scong P
		\quad
		P_1 \Par P_2 \scong P_2 \Par P_1
		\quad
		P_1 \Par (P_2 \Par P_3) \scong (P_1 \Par P_2) \Par P_3
		\quad 
		\recp{X}{P} \scong P\subst{\recp{X}{P}}{\rvar{X}}
		\\%[1mm]

		\news{n} \inact \scong \inact
		\qquad 
		P \Par \news{n} Q \scong \news{n}(P \Par Q)
		\	(n \notin \fn{P})
		\qquad
		P \scong Q \textrm{ if } P \scong_\alpha Q
	\end{array}
\]
	\[
		\!\!\!\begin{array}{lllcrll}
			\orule{App} & (\abs{x}{P}) \, V   \red    P \subst{V}{x}
			& 
			  

			\orule{Pass} & \bout{n}{V} P \Par \binp{\dual{n}}{x} Q   \red   P \Par Q \subst{V}{x} 
			
			\\[1mm]

			 \orule{Res} & P \red P'  \Rightarrow  \news{n} P  \red  \news{n} P' 

			&  
			\orule{Sel}
			&  \!\!\! \bsel{n}{l_j} Q \Par \bbra{\dual{n}}{l_i : P_i}_{i \in I}  \red   Q \Par P_j ~~(j \in I)
			
			\\[1mm]
			\orule{Par} & P \red P'   \Rightarrow    P \Par Q  \red   P' \Par Q  
			&  
			\orule{Cong} & P \scong Q \red Q' \scong P'   \Rightarrow  P  \red  P' 
	\end{array}
	\]
	\vspace{-3mm}
\caption{$\HOp$: Syntax and Operational Semantics (Structural Congruence and Reduction Relation).
\label{fig:redsem}}
%\Hlinefig
\end{figure}

\noindent\myparagraph{Syntax.} 
The syntax of \HOp is defined in \figref{fig:redsem} (upper part).
We use $a,b,c, \dots$ (resp.~$s, \dual{s}, \dots$) 
to range over shared (resp. session) names. 
We use $m, n, t, \dots$ for session or shared names. 
We define the dual operation over names $n$ as $\dual{n}$ with
$\dual{\dual{s}} = s$ and $\dual{a} = a$.
Intuitively, names $s$ and $\dual{s}$ are dual (two) \emph{endpoints} while 
shared names represent non-deterministic points. 
Variables are denoted with $x, y, z, \dots$, 
and recursive variables are denoted with $\varp{X}, \varp{Y} \dots$.
An abstraction %(or higher-order value) 
$\abs{x}{P}$ is a process $P$ with name parameter $x$.
%Symbols $u, v, \dots$ range over identifiers; and  $V, W, \dots$ to denote values. 
Values $V,W$ include 
identifiers $u, v, \ldots$ %(first-order values) 
and 
abstractions $\abs{x}{P}$ (first- and higher-order values, resp.). 
{Terms} 
include $\pi$-calculus constructs for sending/receiving values $V$.
Process $\bout{u}{V} P$ denotes the output of   $V$
over name $u$, with continuation $P$;
process $\binp{u}{x} P$ denotes the input prefix on name $u$ of a value
that 
will substitute variable $x$ in continuation $P$. 
Recursion is expressed by $\recp{X}{P}$,
which binds the recursive variable $\varp{X}$ in process $P$.
Process 
%ny
%$\appl{x}{u}$ 
$\appl{V}{W}$ 
is the application
which substitutes values $W$ on the abstraction~$V$. 
\dk{Typing  ensures \jpc{that} $V$ is not a name.}
Processes $\bbra{u}{l_i: P_i}_{i \in I}$  and $\bsel{u}{l} P$ define labeled choice:
given a finite index set $I$, process $\bbra{u}{l_i: P_i}_{i \in I}$ offers a choice 
among processes with pairwise distinct labels;
%on labels $l_i$ with continuation $P_i$, given that $i \in I$.
 process $\bsel{u}{l} P$ selects label $l$ on name $u$ and then behaves as $P$.
%Given $i \in I$ 
%Others are standard c
Constructs for 
inaction $\inact$,  parallel composition $P_1 \Par P_2$, and 
name restriction $\news{n} P$ are standard.
Session name restriction $\news{s} P$ simultaneously binds endpoints $s$ and $\dual{s}$ in $P$.
%A well-formed process relies on assumptions for
%guarded recursive processes.
We use $\fv{P}$ and $\fn{P}$ to denote a set of free 
%\jpc{recursion}
variables and names; 
and assume $V$ in $\bout{u}{V}{P}$ does not include free recursive 
variables $\rvar{X}$. 
If $\fv{P} = \emptyset$, we call $P$ {\em closed}.
%; and closed $P$ without 
%free session names a {\em program}. 

%\subsection{Subcalculi of \HOp}
%\label{subsec:subcalculi}
%\noi
%We define two subcalculi of \HOp. 
%%We now define several sub-calculi of \HOp. 
%The first is the 
%{\em core higher-order session calculus} (denoted \HO),
%which lacks recursion and name passing; its 
%formal syntax is obtained from \figref{fig:syntax} by excluding 
%constructs in \nonhosyntax{\text{grey}}.
%The second subcalculus is 
%the {\em session $\pi$-calculus} 
%(denoted $\sessp$), which 
%lacks  the
%higher-order constructs
%(i.e., abstraction passing and application), but includes recursion.
%Let $\CAL \in \{\HOp, \HO, \sessp\}$. We write 
%$\CAL^{-\mathsf{sh}}$ for $\CAL$ without shared names
%(we delete $a,b$ from $n$). 
%We shall demonstrate that 
%$\HOp$, $\HO$, and $\sessp$ have the same expressivity.


%	\begin{figure}
%	\[
%		\begin{array}{rcl}
%			u,w &\bnfis& n \bnfbar x,y,z \qquad
%			n \bnfis a,b  \bnfbar s, \dual{s} \qquad 
%			V,W  \bnfis u \bnfbar \abs{x}{P} \\[1mm]
%			P,Q & \bnfis & \bout{u}{V}{P}  \bnfbar  \binp{u}{x}{P} \bnfbar
%			\bsel{u}{l} P \bnfbar \bbra{u}{l_i:P_i}_{i \in I}   \\[1mm]
%			& \bnfbar & \rvar{X} \bnfbar \recp{X}{P} \bnfbar \appl{V}{W} \bnfbar P\Par Q \bnfbar \news{n} P \bnfbar \inact
%		\end{array}
%	\]
%	\[
%
%		\!\!\!\begin{array}{lllcrll}
%			\orule{App} & (\abs{x}{P}) \, V   \red    P \subst{V}{x}
%			& 
%			  
%
%			\orule{Pass} & \bout{n}{V} P \Par \binp{\dual{n}}{x} Q   \red   P \Par Q \subst{V}{x} 
%			
%			\\[1mm]
%
%			 \orule{Res} & P \red P'  \Rightarrow  \news{n} P  \red  \news{n} P' 
%
%			&  
%			\orule{Sel}
%			&  \!\!\! \bsel{n}{l_j} Q \Par \bbra{\dual{n}}{l_i : P_i}_{i \in I}  \red   Q \Par P_j ~~(j \in I)
%			
%			\\[1mm]
%			\orule{Par} & P \red P'   \Rightarrow    P \Par Q  \red   P' \Par Q  
%			&  
%			\orule{Cong} & P \scong Q \red Q' \scong P'   \Rightarrow  P  \red  P' 
%	\end{array}
%	\]
%\[
%	\begin{array}{c}
%		P \Par \inact \scong P
%		\quad
%		P_1 \Par P_2 \scong P_2 \Par P_1
%		\quad
%		P_1 \Par (P_2 \Par P_3) \scong (P_1 \Par P_2) \Par P_3
%		\quad 
%		\recp{X}{P} \scong P\subst{\recp{X}{P}}{\rvar{X}}
%		\\%[1mm]
%
%		\news{n} \inact \scong \inact
%		\qquad 
%		P \Par \news{n} Q \scong \news{n}(P \Par Q)
%		\	(n \notin \fn{P})
%		\qquad
%		P \scong Q \textrm{ if } P \scong_\alpha Q
%
%%		\qquad
%%		\dk{V \scong W \textrm{ if } V \scong_\alpha W
%%		\quad \abs{x}{P} \scong \abs{x}{Q} \textrm{ if } P \scong Q}
%	\end{array}
%\]
%\caption{Operational Semantics of $\HOp$. 
%\label{fig:reduction}}
%\Hlinefig
%\end{figure}

\smallskip
\noindent \myparagraph{Operational Semantics.}
%	\label{subsec:semantics}
	\figref{fig:redsem} (lower part) defines the operational semantics 
of \HOp.
We define a reduction relation which relies 
on 
a \emph{structural congruence} $\scong$, in the standard way. % are defined in \figref{fig:reduction} (bottom). 
\jpc{We assume the expected extension of $\scong$ to values $V$.}
Reduction is denoted $\red$; some intuitions on the rules in \figref{fig:redsem} (lower part) follow.
Rule~$\orule{App}$ is a value application; 
rule~$\orule{Pass}$ defines a shared interaction at $n$ 
(\jpc{with} $\dual{n}=n$) or a session interaction;  
rule~$\orule{Sel}$ is the standard rule for labelled choice/selection:
given an index set $I$, 
a process selects label $l_j$ on name $n$ over a set of
labels $\set{l_i}_{i \in I}$ offered by a branching 
on the dual endpoint $\dual{n}$; and other rules are standard.
We write $\red^\ast$ for a multi-step reduction. 


%\begin{example}
\begin{example}[Hotel Booking Scenario]\label{exam:proc}
To illustrate \HOp and its expressive power, 
we consider a usecase scenario that adapts the example given by Mostrous and Yoshida~\cite{MostrousY15}.
We introduce a usecase scenario where a $\Client$ process wants to book
a hotel. % for her holidays. % in a remote island
%The Client 
$\Client$
narrows  her choice down to two hotels, and requires 
 a quote from the two in order to
make her choice. The round-trip time (RTT) required for
taking quotes from the two hotels in not optimal, % (cf.~\cite{MostrousY15}),
so she decides to send remote codes to both hotels
to automatically negotiate and book the hotel for
her. 

We present two implementations; the first one is given as follows:
%
\[
	\begin{array}{rcl}
		P &\defeq& \bout{x}{\rtype} \binp{x}{\Quote} \bout{y}{\Quote}
		y \triangleleft \left\{
				\begin{array}{l}
					\accept: \bsel{x}{\accept} \bout{x}{\creditc} \inact,\\
					\reject: \bsel{x}{\reject} \inact
				\end{array}
				\right\}
		\\[6mm]
		\Client_1 &\defeq& \newsp{h_1, h_2}{\bout{s_1}{\abs{x}{P \subst{h_1}{y}}} \bout{s_2}{\abs{x}{P \subst{h_2}{y}}} \inact \Par\\
			&&
			\begin{array}{lll}
				\binp{\dual{h_1}}{x} \binp{\dual{h_2}}{y} & \If\ x \leq y\ \Then & \bsel{\dual{h_1}}{\accept} \bsel{\dual{h_2}}{\reject} \inact\\
				& \Else& \bsel{\dual{h_1}}{\reject} \bsel{\dual{h_2}}{\accept} \inact
			\end{array}
		}
	\end{array}
\]
%
\begin{itemize}
	\item	Process $P$ is the remote code responsible for negotiation with a hotel.
		Channel $x$ is intended to be instatiated by the hotel as the negotiating
		endpoint. Channel $y$ is used to interact with $\Client_1$.

	\item	Process $P$
		i) sends the room requirements to the hotel;
		ii) receives a quote from the hotel;
		iii) sends the quote to the $\Client_1$;
		iv) expects a choice from the $\Client_1$ whether to accept or reject the offer and;
		v) If the choice is accept it informs the hotel and performs the booking,
		if the choice is reject it informs the hotel and ends the session.

	\item	$\Client_1$ instantiates two copies of process $P$ as abstractions
		on session $x$. It further uses two
		fresh endpoints $h_1, h_2$ to substitute channels $y$, respectively,
		in order for the two instances of $P$ to be able to interact
		with $\Client_1$.
	
	\item	$\Client_1$ then sends the two abstractions instances of $P$
		to the two hotels via sessions $s_1$ and $s_2$, respectively.

	\item	$\Client_1$ uses the dual endpoints $\dual{h_1}$ and $\dual{h_2}$
		to receive the negotiation
		result from the two remote instances of $P$ and then inform the two
		processes for the final booking decision.
\end{itemize}

The above scenario does not add a significant gain
to the time needed for the entire protocol to take
place, since the two remote processes are required
to send and receive data to $\Client_1$.

As an alternative we can propose a different implementation
of the same scenario that requires from the two
remote processes to interact with each other,
instead of $\Client_1$, to reach to a consensus:
%
\[
	\begin{array}{rcl}
		P_1 &\defeq&	\bout{x}{\rtype} \binp{x}{\Quote_1} \bout{y}{\Quote_1} \binp{y}{\Quote_2}\\
			&&
				\begin{array}{ll}
					\If\ \Quote_1 \leq \Quote_2\ \Then & \bsel{x}{\accept} \bout{x}{\creditc} \inact\\
					\Else & \bsel{x}{\reject} \inact
				\end{array}
		\\
		P_2 &\defeq&	\bout{x}{\rtype} \binp{x}{\Quote_1} \binp{y}{\Quote_2} \bout{y}{\Quote_1}\\
			&&
				\begin{array}{ll}
					\If\ \Quote_1 \leq \Quote_2\ \Then & \bsel{x}{\accept} \bout{x}{\creditc} \inact\\
					\Else & \bsel{x}{\reject} \inact
				\end{array}
		\\
		\Client_2 &\defeq& \newsp{h}{\bout{s_1}{\abs{x}{P_1 \subst{h}{y}}} \bout{s_2}{\abs{x}{P_2 \subst{\dual{h}}{y}}} \inact}
	\end{array}
\]
%\end{example}

\begin{itemize}
	\item	Processes $P_1$ and $P_2$ are responsible for negotiating a quote from the
		hotel in the same fashion as process $P$ in the previous implementation.

	\item	The difference with process $P$ is that the channel $y$ is used for
		interaction between process $P_1$ and $P_2$. Both processes send
		there quotes to each other and then internally follow the same
		logic to reach to a decision.

	\item	The role of $\Client_2$ is to instantiate $P_1$ and $P_2$ as abstractions
		on name $x$. It further substitutes
		the two endpoints of a fresh channel $h$ to channels $y$ respectively,
		in order for the two instances to be able to communicate with each other.

	\item	Process $\Client_2$ then uses sessions $s_1$ and $s_2$ to send the two
		instances of $P_1$ and $P_2$ to the two hotels.
\end{itemize}

We can show that process $\Client_1$ and $\Client_2$
are interchangeable by showing that they are bisimilar.
\end{example}
