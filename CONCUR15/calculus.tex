% !TEX root = main.tex
\noindent 
We introduce the 
\emph{Higher-Order Session $\pi$-Calculus} (\HOp).
\HOp includes both name- and abstraction-passing 
as well as recursion; it is a subcalculus 
of the language
studied 
in~\cite{tlca07}. 
Following the literature~\cite{JeffreyR05},
for simplicity of the presentation
we concentrate on the second-order call-by-value \HOp.  
%(In \secref{sec:extension} we consider extensions of 
%\HOp with higher-order abstractions 
%and polyadicity in name-passing/abstractions.)
%We also introduce two subcalculi of \HOp. In particular, we define the 
%core higher-order session
%calculus (\HO), which 
%%. The \HO calculus is  minimal: it 
%includes constructs for shared name synchronisation and 
%%constructs for session establish\-ment/communication and 
%(monadic) name-abstraction, but lacks name-passing and recursion.

%Although minimal, in \secref{s:expr}
%the abstraction-passing capabilities of \HOp will prove 
%expressive enough to capture key features of session communication, 
%such as delegation and recursion.

\subsection{Syntax of \HOp}
\label{subsec:syntax}
\noindent\myparagraph{Values} The syntax of \HOp is defined in \figref{fig:syntax}

%%%%%%%%%%%%%%%%%%%%%% HOp Syntax Figure %%%%%%%%%%%%%%%%%%%%%%%%%%%%%%%% 
\begin{figure}[t]
	\[
		\begin{array}{rcl}
			u,w &\bnfis& n \bnfbar x,y,z \qquad
			n \bnfis a,b  \bnfbar s, \dual{s} \qquad 
			V,W  \bnfis u \bnfbar \abs{x}{P} \\[1mm]
			P,Q & \bnfis & \bout{u}{V}{P}  \bnfbar  \binp{u}{x}{P} \bnfbar
			\bsel{u}{l} P \bnfbar \bbra{u}{l_i:P_i}_{i \in I}   \\[1mm]
			& \bnfbar & \rvar{X} \bnfbar \recp{X}{P} \bnfbar \appl{V}{W} \bnfbar P\Par Q \bnfbar \news{n} P \bnfbar \inact
		\end{array}
	\]
	\caption{Syntax of \HOp.}
	\label{fig:syntax}
%\Hlinefig
\end{figure}
%%%%%%%%%%%%%%%%%%%%%% End HOp Syntax Figure %%%%%%%%%%%%%%%%%%%%%%%%%%%%

We use $a,b,c, \dots$ (resp.~$s, \dual{s}, \dots$) 
to range over shared (resp. session) names. 
We use $m, n, t, \dots$ for session or shared names. 
We define the dual operation over names $n$ as $\dual{n}$ with
$\dual{\dual{s}} = s$ and $\dual{a} = a$.
Intuitively, names $s$ and $\dual{s}$ are dual (two) \emph{endpoints} while 
shared names represent shared (non-deterministic) points. 
Variables are denoted with $x, y, z, \dots$, 
and recursive variables are denoted with $\varp{X}, \varp{Y} \dots$.
An abstraction %(or higher-order value) 
$\abs{x}{P}$ is a process $P$ with name parameter $x$.
%Symbols $u, v, \dots$ range over identifiers; and  $V, W, \dots$ to denote values. 
Values $V,W$ include 
identifiers $u, v, \ldots$ %(first-order values) 
and 
abstractions $\abs{x}{P}$ (first- and higher-order values, resp.). 

\myparagraph{Terms} 
include the
$\pi$-calculus prefixes for sending and receiving values $V$.
Process $\bout{u}{V} P$ denotes the output of value $V$
over name $u$, with continuation $P$;
process $\binp{u}{x} P$ denotes the input prefix on name $u$ of a value
that 
will substitute variable $x$ in continuation $P$. 
Recursion is expressed by $\recp{X}{P}$,
which binds the recursive variable $\varp{X}$ in process $P$.
Process 
%ny
%$\appl{x}{u}$ 
$\appl{V}{W}$ 
is the application
which substitutes values $W$ on the abstraction~$V$. 
\dk{Typing  ensures \jpc{that} $V$ is not a name.}
Prefix $\bsel{u}{l} P$ selects label $l$ on name $u$ and then behaves as $P$.
%Given $i \in I$ 
Process $\bbra{u}{l_i: P_i}_{i \in I}$ offers a choice on labels $l_i$ with
continuation $P_i$, given that $i \in I$.
%Others are standard c
Constructs for 
inaction $\inact$,  parallel composition $P_1 \Par P_2$, and 
name restriction $\news{n} P$ are standard.
Session name restriction $\news{s} P$ simultaneously binds endpoints $s$ and $\dual{s}$ in $P$.
%A well-formed process relies on assumptions for
%guarded recursive processes.
We use $\fv{P}$ and $\fn{P}$ to denote a set of free 
%\jpc{recursion}
variables and names; 
and assume $V$ in $\bout{u}{V}{P}$ does not include free recursive 
variables $\rvar{X}$. 
If $\fv{P} = \emptyset$, we call $P$ {\em closed}.
%; and closed $P$ without 
%free session names a {\em program}. 

\subsection{Subcalculi of \HOp}
\label{subsec:subcalculi}
\noi
We define two subcalculi of \HOp. 
%We now define several sub-calculi of \HOp. 
The first is the 
{\em core higher-order session calculus} (denoted \HO),
which lacks recursion and name passing; its 
formal syntax is obtained from \figref{fig:syntax} by excluding 
constructs in \nonhosyntax{\text{grey}}.
The second subcalculus is 
the {\em session $\pi$-calculus} 
(denoted $\sessp$), which 
lacks  the
higher-order constructs
(i.e., abstraction passing and application), but includes recursion.
Let $\CAL \in \{\HOp, \HO, \sessp\}$. We write 
$\CAL^{-\mathsf{sh}}$ for $\CAL$ without shared names
(we delete $a,b$ from $n$). 
We shall demonstrate that 
$\HOp$, $\HO$, and $\sessp$ have the same expressivity.

\subsection{Operational Semantics}
	\label{subsec:semantics}
	\begin{figure}
	\[
		\begin{array}{rclrcrclr}
			(\abs{x}{P}) \, V  & \red & P \subst{V}{x}
			& \orule{App}
			\\[1mm]

			\bout{n}{V} P \Par \binp{\dual{n}}{x} Q & \red & P \Par Q \subst{V}{x} 
			& \orule{Pass}
			\\[1mm]

			\bsel{n}{l_j} Q \Par \bbra{\dual{n}}{l_i : P_i}_{i \in I} & \red & Q \Par P_j ~~(j \in I)~~ 
			& \orule{Sel}
			\\[1mm]

			P \red P' & \Rightarrow & \news{n} P  \red  \news{n} P'  & \orule{Res}
			\\[1mm]
			P \red P' & \Rightarrow  &  P \Par Q  \red   P' \Par Q  & \orule{Par}
			\\[1mm]
			P \scong Q \red Q' \scong P' & \Rightarrow & P  \red  P' & \orule{Cong}
	\end{array}
	\]
\[
	\begin{array}{c}
		P \Par \inact \scong P
		\quad
		P_1 \Par P_2 \scong P_2 \Par P_1
		\quad
		P_1 \Par (P_2 \Par P_3) \scong (P_1 \Par P_2) \Par P_3
		\quad 
		\recp{X}{P} \scong P\subst{\recp{X}{P}}{\rvar{X}}
		\\[1mm]

		\news{n} \inact \scong \inact
		\qquad 
		P \Par \news{n} Q \scong \news{n}(P \Par Q)
		\	(n \notin \fn{P})
		\qquad
		P \scong Q \textrm{ if } P \scong_\alpha Q

%		\qquad
%		\dk{V \scong W \textrm{ if } V \scong_\alpha W
%		\quad \abs{x}{P} \scong \abs{x}{Q} \textrm{ if } P \scong Q}
	\end{array}
\]
\caption{Operational Semantics of $\HOp$. 
\label{fig:reduction}}
\Hlinefig
\end{figure}
\noindent \figref{fig:reduction} defines the operational semantics 
of \HOp.
$\orule{App}$ is a value application; 
$\orule{Pass}$ defines a shared interaction at $n$ 
(\jpc{with} $\dual{n}=n$) or a session interaction;  
$\orule{Sel}$ is the standard rule for labelled choice/selection:
given an index set $I$, 
a process selects label $l_j$ on name $n$ over a set of
labels $\set{l_i}_{i \in I}$ offered by a branching 
on the dual endpoint $\dual{n}$; and other rules are standard.
Rules for \emph{structural congruence} are defined in \figref{fig:reduction} (bottom). 
\jpc{We assume the expected extension of $\scong$ to values $V$.}
We write $\red^\ast$ for a multi-step reduction. 
