\documentclass{beamer}

\usepackage{amsmath}
\usepackage{stmaryrd}
\usepackage{xspace}
%\usepackage{color}


\usepackage{comment}
\usepackage{multirow}
\usepackage{tikz}

\usetikzlibrary{calc}

\input{macros.tex}

\newcommand{\prcolor}[1]{{\color{blue} #1}}

%%%%%%%%%%%%%%%%%%% Title Page Info%%%%%%%%%%%%%%%%%%%%%%%%%%

\title{Characteristic Bisimulations for Higher-Order Session Types}

\author{{\bf Dimitrios Kouzapas$^{1,3}$}, Jorge A. P\'{e}rez$^{2}$, and Nobuko Yoshida$^1$}

\institute{Imperial College London$^1$, University of Groningen$^2$, University of Glasgow$^3$}

\date

\begin{document}
	\begin{frame}
		\titlepage

		{ \tiny %This work has been partially 
		Sponsored by the The Doctoral Prize Fellowship, EPSRC EP/K011715/1,
		EPSRC EP/K034413/1, and EPSRC EP/L00058X/1,
		EU project FP7-612985 UpScale, and EU COST Action IC1201 BETTY.  
		P\'{e}rez is  also affiliated to
		%the NOVA Laboratory for Computer Science and Informatics (NOVA LINCS),  
		Universidade Nova de Lisboa.%, Portugal
		}
	\end{frame}

	\begin{frame}{Motivation}
		\begin{itemize}
			\item	Higher-order session calculus

				\begin{itemize}
					\item	Values in messages may be processes
					\item	Bridge between process calculi and the $\lambda$ calculus
					\item	Higher-order session types
				\end{itemize}

%			\item	Equivalence theory for session typed programs

			\item	Equivalence theory for higher-order sessions is
				\begin{itemize}
					\item	Interesting - real applications, relation between the $\lambda$ calculus and process calculi
					\item	Challenging - difficult to compute
				\end{itemize}
%				interesting and challenging

			\item	This work offers the first {\em tractable} theory based on labelled bisimilarities
				informed by session types % The solution is:
				\begin{itemize}
					\item	Natural - follows the behaviour of session types
					\item	Economical - exploits the linearity of session types
					\item	Original - with respect to bibliography
				\end{itemize}
		\end{itemize}
	\end{frame}

	\begin{frame}{Motivation: Hotel Booking Scenario}
		\begin{itemize}
			\item	Hotel booking scenario
				
\newcommand{\Hotel}{\mathsf{Hotel}}
\newcommand{\Code}{\mathsf{Code}}

%\makeatletter
%\newcommand{\gettikzxy}[3]{%
%  \tikz@scan@one@point\pgfutil@firstofone#1\relax
%  \edef#2{\the\pgf@x}%
%  \edef#3{\the\pgf@y}%
%}
%\makeatother
%	\gettikzxy{(Client1.south)}{\ax}{\ay}
%	\draw[dashed]			(Client1.south) -- (\ax, 0);


%\begin{center}

\begin{tikzpicture}
%	\draw[help lines]		(0, 0) grid (13, 10);

	%%%%%%%%%%%%%%%%%%%%% Scenario 1

	%%%% Nodes
	\node	(Client1)	at	(0, 10) {\footnotesize $\Client_1$};
	\node	(Hotel1)	at	(2.5, 10) {\footnotesize $\Hotel_1$};
	\node	(Hotel2)	at	(5, 10) {\footnotesize $\Hotel_2$};

	\node	(Code1)		at	(1.25, 8.8) {\scriptsize $\Code_1$};
	\node	(Code2)		at	(3.75, 8.8) {\scriptsize $\Code_2$};

	%%%% Lines for Nodes
%	\draw[dashed]		(Client1.south west) -- (Client1.south east);
	\draw
		let
			\p1 = (Client1.south),
			\p2 = (Hotel1.south),
			\p3 = (Hotel2.south),
			\p4 = (Code1.south),
			\p5 = (Code2.south)
		in
			(\x1, \y1) -- (\x1, 4.8)
			(\x2, \y2) -- (\x2, 4.8)
			(\x3, \y3) -- (\x3, 4.8)
			(\x4, \y4) -- (\x4, 4.8)
			(\x5, \y5) -- (\x5, 4.8);

	%%%% Arrows
	\draw[->]
		let
			\p1 = (Client1),
			\p2 = (Hotel1)
		in
			(\x1, 9.6) to node[above] {\scriptsize $\abs{x}{P_{xy}}$} (\x2, 9.6);

	\draw[->]
		let
			\p1 = (Client1),
			\p2 = (Hotel2)
		in
			(\x1, 9.2) to node[above] {\qquad \qquad \scriptsize $\abs{x}{P_{xy}}$} (\x2, 9.2);

	\draw[->]
		let
			\p1 = (Hotel1)
		in
			(\x1, 9.6) -- (Code1.north);

	\draw[->]
		let
			\p1 = (Hotel2)
		in
			(\x1, 9.2) -- (Code2.north);


	\draw[->]
		let
			\p1 = (Code1),
			\p2 = (Hotel1)
		in
			(\x1, 8.4) to node[above] {\scriptsize $\rtype$} (\x2, 8.4);

	\draw[->]
		let
			\p1 = (Code1),
			\p2 = (Hotel1)
		in
			(\x2, 8) to node[above] {\scriptsize $\Quote$} (\x1, 8);

	\draw[->]
		let
			\p1 = (Code2),
			\p2 = (Hotel2)
		in
			(\x1, 8.4) to node[above] {\scriptsize $\rtype$} (\x2, 8.4);
	\draw[->]
		let
			\p1 = (Code2),
			\p2 = (Hotel2)
		in
			(\x2, 8) to node[above] {\scriptsize $\Quote$} (\x1, 8);

	\draw[->]
		let
			\p1 = (Code1),
			\p2 = (Client1)
		in
			(\x1, 7.6) to node[above] {\scriptsize $\Quote$} (\x2, 7.6);

	\draw[->]
		let
			\p1 = (Code2),
			\p2 = (Client1)
		in
			(\x1, 7.2) to node[above] {\scriptsize $\Quote$} (\x2, 7.2);


	%%%% Choice 
	%% Client1 --> Code1
	\draw[dashed]
		let
			\p1 = (Client1),
			\p2 = (Code1)
		in
			(-0.4, 6.8) to node {$\oplus$} (-0.4, 5.4);
%			(\x1, 4.8) -- (\x2, 4.8)
%			(\x1, 2.8) -- (\x2, 2.8);

	\draw[dashed, ->]
		let
			\p1 = (Client1),
			\p2 = (Code1)
		in
			(-0.4, 6.8) to node[above] {\scriptsize $\accept$} (\x2, 6.8);

%	\draw[dashed]
%		let
%			\p1 = (Code1),
%			\p2 = (Hotel1)
%		in
%			(\x1, 6.4) -- (\x2, 6.4)
%			(\x1, 5.2) -- (\x2, 5.2)
%			(\x1, 4) -- (\x2, 4)
%			(\x1, 3.2) -- (\x2, 3.2);

	\draw[->,dashed]
		let
			\p1 = (Code1),
			\p2 = (Hotel1)
		in
			(\x1, 6.2) to node[above] {\scriptsize $\accept$} (\x2, 6.2);

	\draw[->]
		let
			\p1 = (Code1),
			\p2 = (Hotel1)
		in
			(\x1, 5.8) to node[above] {\scriptsize $\creditc$} (\x2, 5.8);

	\draw[dashed, ->]
		let
			\p1 = (Client1),
			\p2 = (Code1)
		in
			(-0.4, 5.4) to node[above] {\scriptsize $\reject$} (\x2, 5.4);

	\draw[dashed, ->]
		let
			\p1 = (Code1),
			\p2 = (Hotel1)
		in
			(\x1, 4.8) to node[above] {\scriptsize $\reject$} (\x2, 4.8);


	%%%% Choice 
	%% Client1 --> Code2
	\draw[dashed]
		let
			\p1 = (Client1),
			\p2 = (Code2)
		in
			(-0.2, 6.6) to node {$\oplus$} (-0.2, 5.2);
%			(\x1, 4.8) -- (\x2, 4.8)
%			(\x1, 2.8) -- (\x2, 2.8);

	\draw[dashed, ->]
		let
			\p1 = (Client1),
			\p2 = (Code2)
		in
			(-0.2, 6.6) to node[above] {\scriptsize $\accept$} (\x2, 6.6);

%	\draw[dashed]
%		let
%			\p1 = (Code1),
%			\p2 = (Hotel1)
%		in
%			(\x1, 6.4) -- (\x2, 6.4)
%			(\x1, 5.2) -- (\x2, 5.2)
%			(\x1, 4) -- (\x2, 4)
%			(\x1, 3.2) -- (\x2, 3.2);

	\draw[->,dashed]
		let
			\p1 = (Code2),
			\p2 = (Hotel2)
		in
			(\x1, 6.2) to node[above] {\scriptsize $\accept$} (\x2, 6.2);

	\draw[->]
		let
			\p1 = (Code2),
			\p2 = (Hotel2)
		in
			(\x1, 5.8) to node[above] {\scriptsize $\creditc$} (\x2, 5.8);

	\draw[dashed, ->]
		let
			\p1 = (Client1),
			\p2 = (Code2)
		in
			(-0.2, 5.2) to node[above] {\scriptsize $\reject$} (\x2, 5.2);

	\draw[dashed, ->]
		let
			\p1 = (Code2),
			\p2 = (Hotel2)
		in
			(\x1, 4.8) to node[above] {\scriptsize $\reject$} (\x2, 4.8);


	%%%%%%%%%%%%%%%%%%%%% Scenario 2

	%%%% Nodes
	\node	(Client1)	at	(6.5, 10) {\footnotesize $\Client_2$};
	\node	(Hotel1)	at	(9, 10) {\footnotesize $\Hotel_1$};
	\node	(Hotel2)	at	(11.5, 10) {\footnotesize $\Hotel_2$};

	\node	(Code1)		at	(7.75, 8.8) {\scriptsize $\Code_1$};
	\node	(Code2)		at	(10.25, 8.8) {\scriptsize $\Code_2$};
%	\node	(Client1)	at	(7.5, 10) {\footnotesize $\Client_2$};
%	\node	(Hotel1)	at	(10, 10) {\footnotesize $\Hotel_1$};
%	\node	(Hotel2)	at	(12.5, 10) {\footnotesize $\Hotel_2$};
%
%	\node	(Code1)		at	(8.75, 8.8) {\scriptsize $\Code_1$};
%	\node	(Code2)		at	(11.25, 8.8) {\scriptsize $\Code_2$};


	\draw
		let
			\p1 = (Client1.south),
			\p2 = (Hotel1.south),
			\p3 = (Hotel2.south),
			\p4 = (Code1.south),
			\p5 = (Code2.south)
		in
			(\x1, \y1) -- (\x1, 4.8)
			(\x2, \y2) -- (\x2, 4.8)
			(\x3, \y3) -- (\x3, 4.8)
			(\x4, \y4) -- (\x4, 4.8)
			(\x5, \y5) -- (\x5, 4.8);

	%%%% Arrows
	\draw[->]
		let
			\p1 = (Client1),
			\p2 = (Hotel1)
		in
			(\x1, 9.6) to node[above] {\scriptsize $\abs{x}{Q_1}$} (\x2, 9.6);

	\draw[->]
		let
			\p1 = (Client1),
			\p2 = (Hotel2)
		in
			(\x1, 9.2) to node[above] {\qquad \qquad \scriptsize $\abs{x}{Q_2}$} (\x2, 9.2);

	\draw[->]
		let
			\p1 = (Hotel1)
		in
			(\x1, 9.6) -- (Code1.north);

	\draw[->]
		let
			\p1 = (Hotel2)
		in
			(\x1, 9.2) -- (Code2.north);


	\draw[->]
		let
			\p1 = (Code1),
			\p2 = (Hotel1)
		in
			(\x1, 8.4) to node[above] {\scriptsize $\rtype$} (\x2, 8.4);

	\draw[->]
		let
			\p1 = (Code1),
			\p2 = (Hotel1)
		in
			(\x2, 8) to node[above] {\scriptsize $\Quote$} (\x1, 8);

	\draw[->]
		let
			\p1 = (Code2),
			\p2 = (Hotel2)
		in
			(\x1, 8.4) to node[above] {\scriptsize $\rtype$} (\x2, 8.4);
	\draw[->]
		let
			\p1 = (Code2),
			\p2 = (Hotel2)
		in
			(\x2, 8) to node[above] {\scriptsize $\Quote$} (\x1, 8);

	\draw[->]
		let
			\p1 = (Code1),
			\p2 = (Code2)
		in
			(\x1, 7.6) to node[above] {\qquad \qquad \scriptsize $\Quote$} (\x2, 7.6);

	\draw[->]
		let
			\p1 = (Code2),
			\p2 = (Code1)
		in
			(\x1, 7.2) to node[above] {\scriptsize $\Quote$ \qquad \qquad \qquad} (\x2, 7.2);


	%%%% Choice
	% Client1 --> Hotel1
	\draw[dashed]
		let
			\p1 = (Code1),
			\p2 = (Hotel1)
		in
			(7.35, 6.8) to node {$\oplus$} (7.35, 6);
%			(\x1, 5.6) -- (\x2, 5.6)
%			(\x1, 4.8) -- (\x2, 4.8);

	\draw[dashed, ->]
		let
			\p1 = (Code1),
			\p2 = (Hotel1)
		in
			(7.35, 6.8) to node[above] {\scriptsize  $\accept$} (\x2, 6.8);

	\draw[->]
		let
			\p1 = (Code1),
			\p2 = (Hotel1)
		in
			(\x1, 6.4) to node[above] {\scriptsize $\creditc$} (\x2, 6.4);

	\draw[dashed, ->]
		let
			\p1 = (Code1),
			\p2 = (Hotel1)
		in
			(7.35, 6) to node[above] {\scriptsize $\reject$} (\x2, 6);

	% Client2 --> Hotel2
	\draw[dashed]
		let
			\p1 = (Code2),
			\p2 = (Hotel2)
		in
			(9.85, 6.8) to node {$\oplus$} (9.85, 6);
%			(\x1, 5.6) -- (\x2, 5.6)
%			(\x1, 4.8) -- (\x2, 4.8);

	\draw[dashed, ->]
		let
			\p1 = (Code2),
			\p2 = (Hotel2)
		in
			(9.85, 6.8) to node[above] {\scriptsize $\accept$} (\x2, 6.8);

	\draw[->]
		let
			\p1 = (Code2),
			\p2 = (Hotel2)
		in
			(\x1, 6.4) to node[above] {\scriptsize $\creditc$} (\x2, 6.4);

	\draw[dashed, ->]
		let
			\p1 = (Code2),
			\p2 = (Hotel2)
		in
			(9.85, 6) to node[above] {\scriptsize $\reject$} (\x2, 6);

\end{tikzpicture}
%\end{center}


%		\vspace{5mm}
			\item	$\Client_1$ has a different implementation than $\Client_2$
			\item	$\Client_1$ should have equivalent behaviour with $\Client_2$%: $\Client_1 \wbc \Client_2$
		\end{itemize}
	\end{frame}

	\begin{frame}{Equivalence Theory: Reduction-closed, Barbed Congruence}
				\begin{definition}[Reduction-closed, barbed congruence (untyped)]
				A symmetric relation $\Re$ is  a reduction-closed, barbed congruence
				if $P\ \Re\ Q$ implies
				\begin{itemize}
					\item	$P \red P'$ if $Q \Red Q'$ and $P'\ \Re\ Q'$
					\item	$P \barb{n}$ if $Q \Barb{n}$
					\item	$\forall C, \context{C}{P}\ \Re\ \context{C}{Q}$
				\end{itemize}
				\end{definition}

		\begin{itemize}
			\item	Natural equivalence relation
			\item	Processes not distinguished under any context (observer)
			\item	Heavy universal quantification
		\end{itemize}
	\end{frame}


	\begin{frame}{Equivalence Theory for Higher-Order Calculi (Untyped Setting)}
		\begin{itemize}
			\item	Context bisimulation
				\begin{itemize}
					\item	Well known behavioural equivalence for higher-order calculi
					\item	Natural characterisation of reduction-closed, barbed congruence
					\item	Universal quantifications on output and input clauses
				\end{itemize}

%			\item	context bisimulation is based on heavy quantifications on output and input clauses

			\item	Alternative behavioural characterisations that do not suffer from universal quantification clauses
				\begin{itemize}
					\item	Triggered bisimulation (Sangiorgi) - no recursion primitives
					\item	Higher-order bisimulation (Jeffrey and Rathke)- no linearity
				\end{itemize}
%			\item	Sangiorgi and, subsequently Jeffrey and Rathke offered alernative behavioural characterisations
%%				in the untyped setting
%				that do not suffer from universal quantification clauses
		\end{itemize}
	\end{frame}

	\begin{frame}{This Work: Equivalence Theory for Higher-Order Sessions}
		\begin{itemize}
			\item	A session based higher-order calculus - typed setting
%				We work on the typed setting using as a basis a session based higher-order calculus

			\item	Sessions describe a structured behaviour on the communication semantics of processes

			\item	New solution for the higher-order equivelence problem based
				on session types principles

			\item	Define a refined labelled transition system
				that restricts process actions according to session behaviour.

				%that are used for the first time and are derived out of the behaviour of session types
%		\end{itemize}
%	\end{frame}


%	\begin{frame}{Equivalence Theory for Higher-Order Sessions}
%		\begin{itemize}
			\item	The nature of session types allows to define
				a new behavioural characterisation %for typed higher-order calculi
				called {\bf Characteristic Bisimulation}

%			\item	Characteristic bisimulation is defined on a refined labelled transition system
%				that restricts process actions following the session behaviour.

			\item	A further result is that, due to types, the proofs do
				not require a name matching construct.
		\end{itemize}
	\end{frame}

	\begin{frame}{Higher-Order Session Calculus}
		\begin{itemize}
			\item	Higher-Order Session $\pi$-calculus (\HOp)
				\[
					\begin{array}{rcl}
						V, W &\bnfis& u \bnfbar \abs{x} P \qquad u\ \bnfis\ n \bnfbar x \\[2mm]
						P, Q &\bnfis& \bout{u}{V} P \bnfbar \binp{u}{x} P \bnfbar \appl{V}{W} \bnfbar P \Par Q \\
						&\bnfbar& \bsel{u}{l} P \bnfbar \bbra{u}{l_i: P_i}_{i \in I} \bnfbar \news{u} P \\
						&\bnfbar& \inact \bnfbar \varp{X} \bnfbar \recp{X}{P}
					\end{array}
				\]

			\item	\HOp may communicate and apply abstractions of processes% as values
				\[
					\begin{array}{rcl}
						\bout{u}{\abs{y}{R}} P \Par \binp{\dual{u}}{x} Q &\red& P \Par Q \subst{\abs{y}{R}}{x}\\
						\appl{(\abs{x}{P})}{V} &\red& P \subst{V}{x}
					\end{array}
				\]
		\end{itemize}
	\end{frame}

	\begin{frame}{Higher-Order Session Calculus}
		\begin{itemize}
			\item	Session types syntax
				\[
					\arraycolsep=0pt
					\begin{array}{rcl}
						U &\bnfis & C \bnfbar L
						\\
						C &\bnfis & S \bnfbar \chtype{S} \bnfbar \chtype{L}
						\qquad
						L \bnfis \shot{U} \bnfbar \lhot{U}
						\\
						S &\bnfis & \btout{U} S \bnfbar \btinp{U} S \bnfbar \tinact \bnfbar \trec{t}{S} \bnfbar \vart{t}
						\\
						&\bnfbar& \btsel{l_i:S_i}_{i \in I} \bnfbar \btbra{l_i:S_i}_{i \in I}
					\end{array}
				\]
%				\[
%					\begin{array}{cc}
%						\begin{array}{rcl}
%							U &\bnfis & C \bnfbar L
%							\\
%							C &\bnfis & S \bnfbar \chtype{S} \bnfbar \chtype{L}
%							\\
%							L &\bnfis & \shot{U} \bnfbar \lhot{U}
%						\end{array}
%						&
%						\begin{array}{lcl}
%							S &\bnfis &  \btout{U} S \bnfbar \btinp{U} S \bnfbar \tinact
%							\\
%							&\bnfbar& \btsel{l_i:S_i}_{i \in I} \bnfbar \trec{t}{S} \bnfbar \vart{t}
%							\\
%							&\bnfbar& \btbra{l_i:S_i}_{i \in I}
%						\end{array}
%					\end{array}
%				\]

%			\item	\HOp is a session based higher-order calculus
%				\[
%					\Gamma; \Lambda; \Delta \proves P \hastype \Proc \qquad \qquad \qquad \Gamma; \Lambda; \Delta \proves V \hastype \U
%				\]
			\item	Example: higher-order output typing
				\[
					\tree {
						\Gamma; \Lambda_1; \Delta_1 \cat n: S_2 \proves P \hastype \Proc
						\qquad
						\tree {
							\Gamma; \Lambda_2; \Delta_2 \cat x: S_1 \proves Q \hastype \Proc
						}{
							\Gamma; \Lambda_2; \Delta_2 \proves \abs{x}{Q} \hastype \shot{S_1}
						}
					}{
						\Gamma; \Lambda_1 \cat \Lambda_2; \Delta_1 \cat \Delta_2 \cat n: \btout{\shot{S_1}} S_2 \proves \bout{n}{\abs{x}{Q}} P \hastype \Proc
					}
				\]

			\item	Labelled transition system
				\[
					\tree {
						(\Gamma; \es; \Delta) \by{\ell} (\Gamma; \es; \Delta') \qquad P \by{\ell} P'
					}{
						\Gamma; \Delta \proves P \by{\ell} \Delta' \proves P'
					}
				\]
		\end{itemize}
	\end{frame}

	\begin{frame}{Equivalence Relations}
		\begin{itemize}

			\item	Typed relation
				\[
					\Gamma; \Delta_1 \proves P\ \Re\ \Delta_2 \proves Q 
				\]
			\item	Typed labelled transition semantics is used to define typed relations

			\begin{itemize}
				\item	$\cong$\ \ : Reduction-closed, barbed, congruence\\
				\item	$\wbc$\ \ : Context bisimilarity\\
				\item	$\fwb$ : Characteristic bisimilarity
			\end{itemize}

			\item	This work shows their coincidence

%			\begin{tabular}{lcl}
%				$\cong$ &:& Reduction-closed, Barb-preserving, Congruence\\
%				$\wbc$ &:& Context Bisimilarity\\
%				$\fwb$ &:& Characteristic Bisimilarity
%			\end{tabular}

		\end{itemize}
	\end{frame}

%	\begin{frame}{Labelled Transition System}
%	\end{frame}

	\begin{frame}{Higher-Order Context Bisimulation: Output}
		Suppose $\Gamma; \Delta_1 \proves P\ \Re\ \Delta_2 \proves Q$, for some symmetric $\Re$. Relation $\Re$ is
		a context bisimulation whenever
		\begin{enumerate}[$(\star)$]
			\item	For all $\news{\widetilde{m_1}} \bactout{n}{V}$ such that
				\[
					\Gamma; \Delta_1 \proves P \by{\news{\widetilde{m_1}} \bactout{n}{V}} \Delta_1' \proves P'
				\]
				there exist $Q'$ and $W$ such that 
				\[
					\Gamma; \Delta_2 \proves Q \By{\news{\widetilde{m_2}} \bactout{n}{W}} \Delta_2' \proves Q'
				\]
				and, \fbox{\emph{\textbf{for all} $R$}}  with $\fv{R}=x$, 
				\[
					\Gamma; \Delta_1'' \proves \newsp{\widetilde{m_1}}{P' \Par R\subst{V}{x}}\ \Re\ \Delta_2'' \proves \newsp{\widetilde{m_2}}{Q' \Par R\subst{W}{x}}
				\]
		\end{enumerate}
	\end{frame}

	\begin{frame}{Higher-Order Context Bisimulation: Output}
		Equivalently rewrite the conclusion of the \textcolor{blue}{($\star$)} clause:
		\\[2mm]

		Suppose $\Gamma; \Delta_1 \proves P\ \Re\ \Delta_2 \proves Q$, for some symmetric $\Re$. Relation $\Re$ is
		a context bisimulation whenever
		\begin{enumerate}[$(\star)$]
			\item	\dots\\
%				\[
%					P \by{\news{\widetilde{m_1}} \bactout{n}{V}} P'
%				\]
%				there exist $Q'$ and $W$ such that 
%				\[
%					Q \By{\news{\widetilde{m_2}} \bactout{n}{W}} Q'
%				\]
%				and,
				\fbox{\emph{\textbf{for all} $R$}} with $\fv{R}=x$, 
				\[
					\begin{array}{rcc}
						\Gamma; \Delta_1'' &\proves& \newsp{\widetilde{m_1}}{P' \Par \prcolor{\newsp{s}{\binp{s}{x} R \Par  \bout{\dual{s}}{V} \inact}}}
						\\
						&&\Re
						\\
						\Delta_2'' &\proves&  \newsp{\widetilde{m_2}}{Q' \Par \prcolor{\newsp{s}{\binp{s}{x} R \Par \bout{\dual{s}}{W} \inact}}}
					\end{array}
				\]
		\end{enumerate}
		This is because:
		\[
			\begin{array}{c}
				\newsp{s}{\binp{s}{x} R \Par \bout{\dual{s}}{V} \inact}
				\by{\tau}
				R \subst{V}{x}
				\\
				\newsp{s}{\binp{s}{x} R \Par \bout{\dual{s}}{W} \inact}
				\by{\tau}
				R \subst{W}{x}
			\end{array}
		\]
	\end{frame}

	\begin{frame}{Higher-Order Context Bisimulation: Output}
		Further rewrite the \textcolor{blue}{($\star$)} clause as a \underline{non universally quantified} input triggered context:
		\\[2mm]

		Suppose $\Gamma; \Delta_1 \proves P\ \Re\ \Delta_2 \proves Q$, for some symmetric $\Re$. Relation $\Re$ is
		a context bisimulation whenever
		\begin{enumerate}[$(\star)$]
			\item
				\dots\\
%				\[
%					P \by{\news{\widetilde{m_1}} \bactout{n}{V}} P'
%				\]
%				there exist $Q'$ and $W$ such that 
%				\[
%					Q \By{\news{\widetilde{m_2}} \bactout{n}{W}} Q'
%				\]
%				and,
				%\dots \emph{\textbf{for all} $R$} with $\fv{R}=x$,
				and
				\[
					\begin{array}{rcc}
						\Gamma; \Delta_1'' &\proves& \newsp{\widetilde{m_1}}{P' \Par \prcolor{\binp{t}{y}\newsp{s}{\binp{s}{z} (\appl{y}{z}) \Par \bout{\dual{s}}{V} \inact}}}
						\\
						&& \Re
						\\
						\Delta_1'' &\proves& \newsp{\widetilde{m_2}}{Q' \Par \prcolor{\binp{t}{y}\newsp{s}{\binp{s}{z} (\appl{y}{z}) \Par \bout{\dual{s}}{W} \inact}}}
					\end{array}
				\]
		\end{enumerate}

		This is because \fbox{\emph{\textbf{for all} $R$}} with $\fv{R}=x$, 
		\[
			\begin{array}{c}
				\binp{t}{y}\newsp{s}{\binp{s}{z} (\appl{y}{z}) \Par \bout{\dual{s}}{V} \inact}
				\by{\bactinp{t}{\abs{x}{R}}} \by{\tau} \by{\tau}
				R \subst{V}{x}
				\\
				\binp{t}{y}\newsp{s}{\binp{s}{z} (\appl{y}{z}) \Par \bout{\dual{s}}{W} \inact}
				\by{\bactinp{t}{\abs{x}{R}}} \by{\tau} \by{\tau}
				R \subst{W}{x}
			\end{array}
		\]
	\end{frame}

	\begin{frame}{Higher-Order Context Bisimulation: Input}
		Suppose $\Gamma; \Delta_1 \proves P\ \Re\ \Delta_2 \proves Q$, for some symmetric $\Re$. Relation $\Re$ is
		a context bisimulation whenever
		\begin{enumerate}[$(\bullet)$]
			\item	\fbox{\emph{\textbf{For all} $V$}} such that:
				\[
					\Gamma; \Delta_1 \proves P \by{\bactinp{n}{V}} \Delta_1' \proves P'
				\]
				then there exists $Q'$ such that
				\[
					\Gamma; \Delta_2 \proves Q \By{\bactinp{n}{V}} \Delta_2' \proves Q'
				\]
				and
				\[
					\Gamma; \Delta_1' \proves P'\ \Re\ \Delta_2' \proves Q'
				\]
		\end{enumerate}
	\end{frame}

	\begin{frame}{Higher-Order Context Bisimulation: Input}
		\begin{itemize}
			\item	$\Gamma; \Delta_1$ informs that $V$ comes from the class of processes with type $U$
		\end{itemize}
		\[
			\Gamma; \es; \Delta \proves V \hastype U
		\]
		\begin{itemize}
			\item	Choose the {\em simplest} process $V$ that inhabits $U$\\
				(to improve the \textcolor{blue}{$(\bullet)$} clause)
		\end{itemize}
	\end{frame}

	\begin{frame}{Characteristic Bisimulation: Characteristic Process and Value}
%		We can take advantage of session types to improve the
%		\textcolor{blue}{$(\bullet)$} clause.
%		\vspace{3mm}

%		\begin{notation}[

		\begin{itemize}
			\item	Characteristic Process and Value
				\begin{itemize}
					\item	Type $U$ is inhabited by a simple
						{\bf characteristic process}, $\mapchar{U}{u}$,
						assuming $u$ is a fresh name

					\item	Type $U$ is inhabited by simple
						{\bf characteristic value}, $\omapchar{U}$
				\end{itemize}
%			where $u$ is a name.
%		\end{notation}
		%\vspace{1mm}

			\item	For example, session
				\[
					S = \btinp{\shot{S_1}} \btout{S_2} \tinact
				\]
				is inhabited by {\em characteristic process}
				\[
					\mapchar{S}{u} = \binp{u}{x} (\appl{x}{s_1} \Par \bout{u}{s_2} \inact)
				\]
				for a fresh name $u$.
		\end{itemize}
%		\vspace{3mm}
%
%		Characteristic types for values are inhabited as:
%		\[
%			\omapchar{S} = s \qquad \omapchar{\shot{S}} = \abs{x}{\mapchar{S}{x}}
%		\]
	\end{frame}

	\begin{frame}{Characteristic Bisimulation: Labelled Transition System}
		\begin{itemize}
			\item	Define a coarser labelled transition system
			\[
				\tree {
					\begin{array}{c}
						%\Gamma; \Delta \proves P \by{\bactinp{n}{V}} \Delta' \proves P' \\
						(\Gamma_1; \Lambda_1; \Delta_1) \by{\bactinp{n}{V}} (\Gamma_1; \Lambda_2; \Delta_2)\\
						\begin{array}{rrcl}
						& V &=& m \\
						\vee& V &\scong& \omapchar{U} \\ 
						\vee& V &\scong& \abs{x}{\binp{t}{y} (\appl{y}{x})} \text{ with $t$ fresh}
						\end{array}
					\end{array}
				}{
					(\Gamma_1; \Lambda_1; \Delta_1) \hby{\bactinp{n}{V}} (\Gamma_1; \Lambda_2; \Delta_2)
	%				\Gamma; \Delta \proves P \hby{\bactinp{n}{V}} \Delta' \proves P'
				}
			\]

			\item	Define $\Gamma; \Delta \proves P \hby{\ell} \Delta' \proves P'$
			\item	Value $\abs{x}{\binp{t}{y} (\appl{y}{x})}$ is called {\bf trigger value}

			\item	The trigger value is important to achieve the appropriate equivalence discrimination.
%			\item
				Lack of it results
				in an under-discriminating equivalence 
				(see example~12 in the paper)
		\end{itemize}
	\end{frame}

	\begin{frame}{Characteristic Bisimulation: Trigger Value}
		\begin{itemize}
			\item	The trigger value is a universal higher-order input
				\[
				\begin{array}{l}
					\Gamma; \Delta \cat n: \btinp{\shot{U}} S \proves \binp{n}{x} P
					\by{\bactinp{n}{\abs{x}{\binp{t}{y} (\appl{y}{x})}}}\\
					\Gamma \cat t: \chtype{\shot{U}}; \Delta \cat n: S %\cat t: \btinp{\shot{U}} \tinact
					\proves P \subst{\abs{x}{\binp{t}{y} (\appl{y}{x})}}{x} 
				\end{array}
				\]
			\item	This is because
				\[
					\forall \shot{U}, \quad \Gamma \cat t: \chtype{\shot{U}}; \es %\cat t: \btinp{\shot{U}} \tinact
					\proves \abs{x}{\binp{t}{y} (\appl{y}{x})} \hastype \shot{U}
				\]

			\item	If an instance of the trigger value inputs a trigger value
				\begin{eqnarray*}
					\Gamma \cat t_1: \chtype{\shot{U}}; \Delta
					%\cat t_1: \btinp{\shot{U}} \tinact
					\proves \binp{t_1}{y} (\appl{y}{n}) & \by{\bactinp{t_1}{\abs{x}{\binp{t_2}{y} (\appl{y}{x})}}} \by{\tau}\\
					\Gamma \cat t_2: \chtype{\shot{U}}; \Delta
					%\cat t_2: \btinp{\shot{U}} \tinact
					\proves \binp{t_2}{y} (\appl{y}{n}) %\appl{\abs{x}{\binp{t'}{y} (\appl{y}{x})}}{n} 
				\end{eqnarray*}
%			\item	Lack of the trigger value in the above definition would result
%				in an under-discriminating equivalence relation (example~12 in the paper)
		\end{itemize}
	\end{frame}

	\begin{frame}{Characteristic Bisimulation: Trigger Process}
		\begin{itemize}
			\item	In the light of the coarser semantics that
				relation $\hby{\ell}$ offers, process
				\[
					\binp{t}{y}\newsp{s}{\prcolor{\binp{s}{z} (\appl{y}{z})} \Par \bout{s}{V} \inact}
				\]
				is equivalent with {\bf trigger process}
				\[
					\binp{t}{y}\newsp{s}{\prcolor{\mapchar{\btinp{U} \tinact}{s}} \Par \bout{s}{V} \inact}
				\]
				assuming that
				\[
					\Gamma; \es; \Delta \proves V \hastype U
				\]
			\item	This is because
				\[
					\begin{array}{rl}
						\binp{t}{y} \newsp{s}{\binp{s}{z} (\appl{y}{z}) \Par \bout{s}{V} \inact}
						&\by{\bactinp{t}{\omapchar{U}}} \wbc \\
						\newsp{s}{\mapchar{\btinp{U_1} \tinact}{s} \Par \bout{s}{V} \inact} %= \binp{s}{z} \appl{y}{z}
					\end{array}
				\]
		\end{itemize}
	\end{frame}

	\begin{frame}{Characteristic Bisimulation}

%		Suppose $\Gamma; \Delta_1 \proves P\ \Re\ \Delta_2 \proves Q$ for some {\em characteristic} bisimulation~$\Re$. Then
		Suppose $\Gamma; \Delta_1 \proves P\ \Re\ \Delta_2 \proves Q$, for some symmetric $\Re$. Relation $\Re$ is a
		{\bf characteristic} bisimulation whenever
		\begin{enumerate}[$(\star)$]
			\item	Whenever
				\[
					\Gamma; \Delta_1 \proves P \hby{\news{\widetilde{m_1}} \Delta_1' \proves \bactout{n}{V: U_1}} P'
				\]
				there exist $Q'$ and $W$ such that 
				\[
					\Gamma; \Delta_2 \proves Q \Hby{\news{\widetilde{m_2}} \bactout{n}{W: U_2}} \Delta_2' \proves Q'
				\]
				and
				\[
					\begin{array}{c}
						\Gamma; \Delta_1'' \proves \newsp{\widetilde{m_1}}{P' \Par \binp{t}{y}\newsp{s}{\mapchar{\btinp{U_1} \tinact}{s} \Par \bout{s}{V} \inact}}
						\\
						\Re
						\\
						\Delta_2'' \proves \newsp{\widetilde{m_2}}{Q' \Par \binp{t}{y}\newsp{s}{\mapchar{\btinp{U_2} \tinact}{s} \Par \bout{s}{W} \inact}}
					\end{array}
				\]
		\end{enumerate}
	\end{frame}

	\begin{frame}{Characteristic Bisimulation}

		\begin{enumerate}[$(\bullet)$]
			\item	Whenever
				\[
					\Gamma; \Delta_1 \proves P \hby{\bactinp{n}{V}} \Delta_1' \proves P'
				\]
				there exist $Q'$ such that 
				\[
					\Gamma; \Delta_2 \proves Q \Hby{\bactinp{n}{V}} \Delta_2' \proves Q'
				\]
				and
				\[
					\Gamma; \Delta_1' \proves P'\, \Re\, \Delta_2' \proves Q'
				\]
		\end{enumerate}
	\end{frame}

	\begin{frame}{Soundness and Completeness}
%		\begin{tabular}{lcc}
%			Reduction-closed, Barb-preserving, Congruence&: & $\cong$\\
%			Context Bisimilarity&: & $\wbc$\\
%			Characteristic Bisimilarity&: & $\fwb$
%		\end{tabular}
%		\vspace{5mm}
		\begin{itemize}
			\item	The behavioural equivalences coincide

				\begin{theorem}
					\begin{center}
					$\cong \qquad =\qquad \wbc \qquad =\qquad \fwb$
					\end{center}
				\end{theorem}

			\item	No matching construct is required for proofs - types contain all the information needed

		\end{itemize}
	\end{frame}

	\begin{frame}{No matching: Example}
			\begin{itemize}
				\item	Assume
					\begin{eqnarray*}
						&&\es; \es; n: \btout{S} \tinact, m_1 : S \proves P = \bout{n}{m_1} \inact \\
						&&\es; \es; n: \btout{S} \tinact, m_2 : S \proves Q = \bout{n}{m_2} \inact
					\end{eqnarray*}
				\item	Observe actions $\bactout{n}{m_1}$ and $\bactout{n}{m_2}$, respectively
					\begin{eqnarray*}
						&& t: \chtype{\tinact}; \es; m_1 : S \proves \binp{t}{y} \newsp{s}{\mapchar{\btinp{S} \tinact}{s}\Par \bout{\dual{s}}{m_1}\inact} \\
						&& t: \chtype{\tinact}; \es; m_2 : S \proves \binp{t}{y} \newsp{s}{\mapchar{\btinp{S} \tinact}{s}\Par \bout{\dual{s}}{m_2}\inact} 
					\end{eqnarray*}


				\item	No need for the observer to match $m_1$ and $m_2$
				\begin{itemize}
					\item	If $S = \tinact$ then $P$ and $Q$ are bisimilar because there is no further observation on $m_1$ and $m_2$
					\item	Otherwise $P$ and $Q$ can be distinguished because by further observing interaction on $m_1$ and $m_2$,
						respectively.
				\end{itemize}
			\end{itemize}
	\end{frame}

	\begin{frame}{Summary of the Results/Contribution}
		\begin{itemize}
			\item	Session based higher-order $\pi$-calculus - \HOp
				\begin{itemize}
					\item	Syntax and semantics
					\item	Session types
					\item	Session types inform a coarser/simpler labelled transition semantics
				\end{itemize}

			\item	Behavioural equivalence theory
				\begin{itemize}
					\item	Sessions enforce a structured behaviour different than
						the behaviour of untyped processes %typed processes
					\item	Sessions provide additional and crucial information to elliminate
					\begin{itemize}
						\item	Heavy quantification requirements
						\item	The need of a name matching construct
					\end{itemize}
				\end{itemize}
		\end{itemize}

	\end{frame}

	\begin{frame}{Summary of the Results/Contribution}
		\begin{itemize}
			\item	Characteristic Bisimilarity
				\begin{itemize}
					\item	First equivalence definition to exploit session type information
					\item	Follows session behaviour
					\item	We claim that is easier to compute due to simpler semantics and session linearity
				\end{itemize}

			\item	Characterisation
				\begin{itemize}
					\item	Reduction-closed, barbed congruence
					\item	Context bisimularity
					\item	Characteristic bisimilarity
				\end{itemize}

		\end{itemize}
	\end{frame}
%	\begin{frame}{Summary of the Results}
%		\begin{itemize}
%			\item	Characteristic bisimilarity
%				\begin{itemize}
%					\item	Information from session types
%					\item	Principle applied for the first time in process calculi equivalence theory
%					\item	No matching construct is required have a full context characterisation
%				\end{itemize}
%		\end{itemize}
%	\end{frame}

	\begin{frame}{Questions?}
		\begin{center}
			\huge Thank you for your attention
		\end{center}
	\end{frame}

	\begin{frame}{The need for trigger value}
%		\begin{example}[The Need for Refined Typed LTS]
%		\label{ex:motivation}
%		We show that observing a characteristic value
%		input alone is not enough
%		\dk{to define a sound bisimulation closure}.
		Consider processes % $P_1, P_2$:
		%
		\begin{eqnarray*}
			P_1 = \binp{s}{x} (\appl{x}{s_1} \Par \appl{x}{s_2}) 
			& & 
			P_2 = \binp{s}{x} (\appl{x}{s_1} \Par \binp{s_2}{y} \inact) 
%			\label{equ:6}
		\end{eqnarray*}
		%
		%We can show that 
		with
		\[
			\Gamma; \es; \Delta \cat s: \btinp{\shot{(\btinp{C} \tinact)}} \tinact \proves P_i \hastype \Proc (i \in \{1,2\})
		\]

		If $P_1$ and $P_2$ input and substitute over $x$
		the characteristic value
		\[
			\omapchar{\shot{(\btinp{C} \tinact)}} = \abs{x}{\binp{x}{y} \inact}
		\] 
		then they both evolve into %(\ref{eq:5}) and (\ref{eq:6}) in become:
		\begin{center}
		%\begin{tabular}{c}
			$\Gamma; \es; \Delta \proves \binp{s_1}{y} \inact \Par \binp{s_2}{y} \inact \hastype \Proc$
		%\end{tabular}
		\end{center}
		\noi therefore becoming 
		context bisimilar
	\end{frame}

	\begin{frame}{The need for trigger value}
		%after the input of $\abs{x}{\binp{x}{y}} \inact$.
		However, $P_1$ and $P_2$
%		the processes in (\ref{equ:6}) 
		are
%clearly
		{\em not} context bisimilar:
%		: many input actions
%		may be used to distinguish them.
%		For example,
		If 
		$P_1$ and $P_2$ input and substitute over $x$ the value
		\[
			\abs{x} \newsp{s}{\bout{a}{s} \binp{x}{y} \inact}
		\]
		with
		\[
			\Gamma; \es; \Delta \proves s \hastype \tinact
		\]
		then their derivatives
		\begin{eqnarray*}
			P_1' &=& \abs{x}{\newsp{s}{\bout{a}{s} \binp{s_1}{y} \inact}} \Par \abs{x}{\newsp{s}{\bout{a}{s} \binp{s_2}{y} \inact}}\\
			P_2' &=& \abs{x}{\newsp{s}{\bout{a}{s} \binp{s_1}{y} \inact}} \Par \binp{s_2}{y} \inact
		\end{eqnarray*}
		are not bisimilar
	\end{frame}

	\begin{frame}{The need for trigger value}
%		Observing only the characteristic value 
%		results in an under-discriminating bisimulation.
		However, if a trigger value
		\[
			\abs{{x}}{\binp{t}{y} (\appl{y}{{x}})}
		\]
		is received on $s$, we can distinguish $P_1$, $P_2$ %($\ell = \bactinp{s}{\abs{{x}}{\binp{t}{y} (\appl{y}{{x}})}}$):
		%
		\begin{eqnarray*}
			P_1 &\hby{\bactinp{s}{\abs{{x}}{\binp{t}{y} (\appl{y}{{x}})}}} \hby{\tau}& \binp{t}{x} (\appl{x}{s_1}) \Par \binp{t}{x} (\appl{x}{s_2})\\
%			\mbox{~and~}
			P_2 &\hby{\bactinp{s}{\abs{{x}}{\binp{t}{y} (\appl{y}{{x}})}}} \hby{\tau}& \binp{t}{x} (\appl{x}{s_1}) \Par \binp{s_2}{y} \inact
%			\quad 
		\end{eqnarray*}
	\end{frame}

	\begin{frame}{The need for trigger value}
		The trigger value alone is not enough.
		Consider processes
		%
		\begin{eqnarray*}
			&&\newsp{s}{\binp{n}{x} (\appl{x}{s}) \Par \bout{\dual{s}}{\abs{x} R_1} \inact} \\
			&&\newsp{s}{\binp{n}{x} (\appl{x}{s}) \Par \bout{\dual{s}}{\abs{x} R_2} \inact} 
		\end{eqnarray*}
		%
		On a trigger value input, we obtain the derivatives
		\begin{eqnarray*}
			&&\newsp{s}{\binp{t}{x} (\appl{x}{s}) \Par \bout{\dual{s}}{\abs{x} R_1} \inact} \\
			&&\newsp{s}{\binp{t}{x} (\appl{x}{s}) \Par \bout{\dual{s}}{\abs{x} R_2} \inact}
		\end{eqnarray*}

		\noi thus concluding a bisimulation closure

		But
		on a characteristic value ($\abs{z}{\binp{z}{x} (\appl{x}{m})}$) input 
		then they would become
		%
		\begin{eqnarray*}
			\Gamma; \es; \Delta \proves \newsp{s}{\binp{s}{x} (\appl{x}{m}) \Par \bout{\dual{s}}{\abs{x} R_i} \inact} \wbc \Delta \proves R_i \subst{m}{x}
		\quad (i=1,2)
		\end{eqnarray*}
		\noi which are not bisimilar if $R_1 \subst{m}{x} \not\wbc R_2 \subst{m}{x}$.
	\end{frame}

\end{document}

