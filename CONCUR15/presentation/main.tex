\documentclass{beamer}

\usepackage{amsmath}
\usepackage{stmaryrd}
\usepackage{xspace}
%\usepackage{color}


\usepackage{comment}
\usepackage{multirow}
\usepackage{tikz}

\usetikzlibrary{calc}

\input{macros.tex}

%%%%%%%%%%%%%%%%%%% Title Page Info%%%%%%%%%%%%%%%%%%%%%%%%%%

\title{Characteristic Bisimulations for Higher-Order Session Types}

\author{{\bf Dimitrios Kouzapas$^{1,3}$}, Jorge A. P\'{e}rez$^{2}$, and Nobuko Yoshida$^1$}

\institute{Imperial College London$^1$, University of Groningen$^2$, University of Glasgow$^3$}

\date

\begin{document}
	\begin{frame}
		\titlepage

		{ \tiny %This work has been partially 
		Sponsored by the The Doctoral Prize Fellowship, EPSRC EP/K011715/1,
		EPSRC EP/K034413/1, and EPSRC EP/L00058X/1,
		EU project FP7-612985 UpScale, and EU COST Action IC1201 BETTY.  
		P\'{e}rez is  also affiliated to
		%the NOVA Laboratory for Computer Science and Informatics (NOVA LINCS),  
		Universidade Nova de Lisboa.%, Portugal
		}
	\end{frame}

	\begin{frame}{Motivation}
		\begin{itemize}
			\item	Higher-order session calculus

				\begin{itemize}
					\item	Session types
					\item	Values in message may be processes
					\item	Bridge between process calculi and the $\lambda$-calculus
				\end{itemize}

%			\item	Equivalence theory for session typed programs

			\item	Equivalence theory for higher-order sessions is both interesting and challenging

			\item	This paper offers the first {\em tractable} theory, based on labelled bisimilarities
				informed by session types. The solution is:
				\begin{itemize}
					\item	Natural - follows the behaviour of session types
					\item	Economical - exploits the linearity of session types
					\item	Original - with respect to bibliography
				\end{itemize}
		\end{itemize}
	\end{frame}

	\begin{frame}{Motivation: Example}
%		\begin{figure}
		
\newcommand{\Hotel}{\mathsf{Hotel}}
\newcommand{\Code}{\mathsf{Code}}

%\makeatletter
%\newcommand{\gettikzxy}[3]{%
%  \tikz@scan@one@point\pgfutil@firstofone#1\relax
%  \edef#2{\the\pgf@x}%
%  \edef#3{\the\pgf@y}%
%}
%\makeatother
%	\gettikzxy{(Client1.south)}{\ax}{\ay}
%	\draw[dashed]			(Client1.south) -- (\ax, 0);


%\begin{center}

\begin{tikzpicture}
%	\draw[help lines]		(0, 0) grid (13, 10);

	%%%%%%%%%%%%%%%%%%%%% Scenario 1

	%%%% Nodes
	\node	(Client1)	at	(0, 10) {\footnotesize $\Client_1$};
	\node	(Hotel1)	at	(2.5, 10) {\footnotesize $\Hotel_1$};
	\node	(Hotel2)	at	(5, 10) {\footnotesize $\Hotel_2$};

	\node	(Code1)		at	(1.25, 8.8) {\scriptsize $\Code_1$};
	\node	(Code2)		at	(3.75, 8.8) {\scriptsize $\Code_2$};

	%%%% Lines for Nodes
%	\draw[dashed]		(Client1.south west) -- (Client1.south east);
	\draw
		let
			\p1 = (Client1.south),
			\p2 = (Hotel1.south),
			\p3 = (Hotel2.south),
			\p4 = (Code1.south),
			\p5 = (Code2.south)
		in
			(\x1, \y1) -- (\x1, 4.8)
			(\x2, \y2) -- (\x2, 4.8)
			(\x3, \y3) -- (\x3, 4.8)
			(\x4, \y4) -- (\x4, 4.8)
			(\x5, \y5) -- (\x5, 4.8);

	%%%% Arrows
	\draw[->]
		let
			\p1 = (Client1),
			\p2 = (Hotel1)
		in
			(\x1, 9.6) to node[above] {\scriptsize $\abs{x}{P_{xy}}$} (\x2, 9.6);

	\draw[->]
		let
			\p1 = (Client1),
			\p2 = (Hotel2)
		in
			(\x1, 9.2) to node[above] {\qquad \qquad \scriptsize $\abs{x}{P_{xy}}$} (\x2, 9.2);

	\draw[->]
		let
			\p1 = (Hotel1)
		in
			(\x1, 9.6) -- (Code1.north);

	\draw[->]
		let
			\p1 = (Hotel2)
		in
			(\x1, 9.2) -- (Code2.north);


	\draw[->]
		let
			\p1 = (Code1),
			\p2 = (Hotel1)
		in
			(\x1, 8.4) to node[above] {\scriptsize $\rtype$} (\x2, 8.4);

	\draw[->]
		let
			\p1 = (Code1),
			\p2 = (Hotel1)
		in
			(\x2, 8) to node[above] {\scriptsize $\Quote$} (\x1, 8);

	\draw[->]
		let
			\p1 = (Code2),
			\p2 = (Hotel2)
		in
			(\x1, 8.4) to node[above] {\scriptsize $\rtype$} (\x2, 8.4);
	\draw[->]
		let
			\p1 = (Code2),
			\p2 = (Hotel2)
		in
			(\x2, 8) to node[above] {\scriptsize $\Quote$} (\x1, 8);

	\draw[->]
		let
			\p1 = (Code1),
			\p2 = (Client1)
		in
			(\x1, 7.6) to node[above] {\scriptsize $\Quote$} (\x2, 7.6);

	\draw[->]
		let
			\p1 = (Code2),
			\p2 = (Client1)
		in
			(\x1, 7.2) to node[above] {\scriptsize $\Quote$} (\x2, 7.2);


	%%%% Choice 
	%% Client1 --> Code1
	\draw[dashed]
		let
			\p1 = (Client1),
			\p2 = (Code1)
		in
			(-0.4, 6.8) to node {$\oplus$} (-0.4, 5.4);
%			(\x1, 4.8) -- (\x2, 4.8)
%			(\x1, 2.8) -- (\x2, 2.8);

	\draw[dashed, ->]
		let
			\p1 = (Client1),
			\p2 = (Code1)
		in
			(-0.4, 6.8) to node[above] {\scriptsize $\accept$} (\x2, 6.8);

%	\draw[dashed]
%		let
%			\p1 = (Code1),
%			\p2 = (Hotel1)
%		in
%			(\x1, 6.4) -- (\x2, 6.4)
%			(\x1, 5.2) -- (\x2, 5.2)
%			(\x1, 4) -- (\x2, 4)
%			(\x1, 3.2) -- (\x2, 3.2);

	\draw[->,dashed]
		let
			\p1 = (Code1),
			\p2 = (Hotel1)
		in
			(\x1, 6.2) to node[above] {\scriptsize $\accept$} (\x2, 6.2);

	\draw[->]
		let
			\p1 = (Code1),
			\p2 = (Hotel1)
		in
			(\x1, 5.8) to node[above] {\scriptsize $\creditc$} (\x2, 5.8);

	\draw[dashed, ->]
		let
			\p1 = (Client1),
			\p2 = (Code1)
		in
			(-0.4, 5.4) to node[above] {\scriptsize $\reject$} (\x2, 5.4);

	\draw[dashed, ->]
		let
			\p1 = (Code1),
			\p2 = (Hotel1)
		in
			(\x1, 4.8) to node[above] {\scriptsize $\reject$} (\x2, 4.8);


	%%%% Choice 
	%% Client1 --> Code2
	\draw[dashed]
		let
			\p1 = (Client1),
			\p2 = (Code2)
		in
			(-0.2, 6.6) to node {$\oplus$} (-0.2, 5.2);
%			(\x1, 4.8) -- (\x2, 4.8)
%			(\x1, 2.8) -- (\x2, 2.8);

	\draw[dashed, ->]
		let
			\p1 = (Client1),
			\p2 = (Code2)
		in
			(-0.2, 6.6) to node[above] {\scriptsize $\accept$} (\x2, 6.6);

%	\draw[dashed]
%		let
%			\p1 = (Code1),
%			\p2 = (Hotel1)
%		in
%			(\x1, 6.4) -- (\x2, 6.4)
%			(\x1, 5.2) -- (\x2, 5.2)
%			(\x1, 4) -- (\x2, 4)
%			(\x1, 3.2) -- (\x2, 3.2);

	\draw[->,dashed]
		let
			\p1 = (Code2),
			\p2 = (Hotel2)
		in
			(\x1, 6.2) to node[above] {\scriptsize $\accept$} (\x2, 6.2);

	\draw[->]
		let
			\p1 = (Code2),
			\p2 = (Hotel2)
		in
			(\x1, 5.8) to node[above] {\scriptsize $\creditc$} (\x2, 5.8);

	\draw[dashed, ->]
		let
			\p1 = (Client1),
			\p2 = (Code2)
		in
			(-0.2, 5.2) to node[above] {\scriptsize $\reject$} (\x2, 5.2);

	\draw[dashed, ->]
		let
			\p1 = (Code2),
			\p2 = (Hotel2)
		in
			(\x1, 4.8) to node[above] {\scriptsize $\reject$} (\x2, 4.8);


	%%%%%%%%%%%%%%%%%%%%% Scenario 2

	%%%% Nodes
	\node	(Client1)	at	(6.5, 10) {\footnotesize $\Client_2$};
	\node	(Hotel1)	at	(9, 10) {\footnotesize $\Hotel_1$};
	\node	(Hotel2)	at	(11.5, 10) {\footnotesize $\Hotel_2$};

	\node	(Code1)		at	(7.75, 8.8) {\scriptsize $\Code_1$};
	\node	(Code2)		at	(10.25, 8.8) {\scriptsize $\Code_2$};
%	\node	(Client1)	at	(7.5, 10) {\footnotesize $\Client_2$};
%	\node	(Hotel1)	at	(10, 10) {\footnotesize $\Hotel_1$};
%	\node	(Hotel2)	at	(12.5, 10) {\footnotesize $\Hotel_2$};
%
%	\node	(Code1)		at	(8.75, 8.8) {\scriptsize $\Code_1$};
%	\node	(Code2)		at	(11.25, 8.8) {\scriptsize $\Code_2$};


	\draw
		let
			\p1 = (Client1.south),
			\p2 = (Hotel1.south),
			\p3 = (Hotel2.south),
			\p4 = (Code1.south),
			\p5 = (Code2.south)
		in
			(\x1, \y1) -- (\x1, 4.8)
			(\x2, \y2) -- (\x2, 4.8)
			(\x3, \y3) -- (\x3, 4.8)
			(\x4, \y4) -- (\x4, 4.8)
			(\x5, \y5) -- (\x5, 4.8);

	%%%% Arrows
	\draw[->]
		let
			\p1 = (Client1),
			\p2 = (Hotel1)
		in
			(\x1, 9.6) to node[above] {\scriptsize $\abs{x}{Q_1}$} (\x2, 9.6);

	\draw[->]
		let
			\p1 = (Client1),
			\p2 = (Hotel2)
		in
			(\x1, 9.2) to node[above] {\qquad \qquad \scriptsize $\abs{x}{Q_2}$} (\x2, 9.2);

	\draw[->]
		let
			\p1 = (Hotel1)
		in
			(\x1, 9.6) -- (Code1.north);

	\draw[->]
		let
			\p1 = (Hotel2)
		in
			(\x1, 9.2) -- (Code2.north);


	\draw[->]
		let
			\p1 = (Code1),
			\p2 = (Hotel1)
		in
			(\x1, 8.4) to node[above] {\scriptsize $\rtype$} (\x2, 8.4);

	\draw[->]
		let
			\p1 = (Code1),
			\p2 = (Hotel1)
		in
			(\x2, 8) to node[above] {\scriptsize $\Quote$} (\x1, 8);

	\draw[->]
		let
			\p1 = (Code2),
			\p2 = (Hotel2)
		in
			(\x1, 8.4) to node[above] {\scriptsize $\rtype$} (\x2, 8.4);
	\draw[->]
		let
			\p1 = (Code2),
			\p2 = (Hotel2)
		in
			(\x2, 8) to node[above] {\scriptsize $\Quote$} (\x1, 8);

	\draw[->]
		let
			\p1 = (Code1),
			\p2 = (Code2)
		in
			(\x1, 7.6) to node[above] {\qquad \qquad \scriptsize $\Quote$} (\x2, 7.6);

	\draw[->]
		let
			\p1 = (Code2),
			\p2 = (Code1)
		in
			(\x1, 7.2) to node[above] {\scriptsize $\Quote$ \qquad \qquad \qquad} (\x2, 7.2);


	%%%% Choice
	% Client1 --> Hotel1
	\draw[dashed]
		let
			\p1 = (Code1),
			\p2 = (Hotel1)
		in
			(7.35, 6.8) to node {$\oplus$} (7.35, 6);
%			(\x1, 5.6) -- (\x2, 5.6)
%			(\x1, 4.8) -- (\x2, 4.8);

	\draw[dashed, ->]
		let
			\p1 = (Code1),
			\p2 = (Hotel1)
		in
			(7.35, 6.8) to node[above] {\scriptsize  $\accept$} (\x2, 6.8);

	\draw[->]
		let
			\p1 = (Code1),
			\p2 = (Hotel1)
		in
			(\x1, 6.4) to node[above] {\scriptsize $\creditc$} (\x2, 6.4);

	\draw[dashed, ->]
		let
			\p1 = (Code1),
			\p2 = (Hotel1)
		in
			(7.35, 6) to node[above] {\scriptsize $\reject$} (\x2, 6);

	% Client2 --> Hotel2
	\draw[dashed]
		let
			\p1 = (Code2),
			\p2 = (Hotel2)
		in
			(9.85, 6.8) to node {$\oplus$} (9.85, 6);
%			(\x1, 5.6) -- (\x2, 5.6)
%			(\x1, 4.8) -- (\x2, 4.8);

	\draw[dashed, ->]
		let
			\p1 = (Code2),
			\p2 = (Hotel2)
		in
			(9.85, 6.8) to node[above] {\scriptsize $\accept$} (\x2, 6.8);

	\draw[->]
		let
			\p1 = (Code2),
			\p2 = (Hotel2)
		in
			(\x1, 6.4) to node[above] {\scriptsize $\creditc$} (\x2, 6.4);

	\draw[dashed, ->]
		let
			\p1 = (Code2),
			\p2 = (Hotel2)
		in
			(9.85, 6) to node[above] {\scriptsize $\reject$} (\x2, 6);

\end{tikzpicture}
%\end{center}


		\[
			\Client_1 \wbc \Client_2
		\]
%		\caption{Sequence diagrams for $\Client_1$ and $\Client_2$ as in Example~\ref{exam:proc}\label{fig:exam}.}
%		\vspace{-2mm}
%		\end{figure}
	\end{frame}

	\begin{frame}{Equivalence Theory: Reduction-closed, Barbed-preserving Congruence}
		\begin{itemize}
			\item Suppose a symmetric relation $P\ \Re\ Q$. $\Re$ is a reduction-closed, barbed-preserving congruence whenever:
			\begin{itemize}
				\item	$P \red P'$ if $Q \Red Q'$ and $P'\ \Re\ Q'$
				\item	$P \barb{n}$ if $Q \Barb{n}$
				\item	$\forall C, \context{C}{P}\ \Re\ \context{C}{Q}$
			\end{itemize}

			\item	Natural equivalence relation
			\item	Processes not distinguished under any context (observer)
			\item	Untyped Setting
		\end{itemize}
	\end{frame}


	\begin{frame}{Equivalence Theory for Higher-Order Calculi (Untyped Setting)}
		\begin{itemize}
			\item	Contextual bisimulation
				\begin{itemize}
					\item	Well known behavioural equivalence for higher-order calculi
					\item	Natural characterisation of reduction-closed, barb-preserving congruence
					\item	Universal quantifications on output and input clauses
				\end{itemize}

%			\item	Contextual bisimulation is based on heavy quantifications on output and input clauses

			\item	Sangiorgi and, subsequently Jeffrey and Rathke offered alernative behavioural characterisations
%				in the untyped setting
				that do not suffer from universal quantification clauses
		\end{itemize}
	\end{frame}

	\begin{frame}{Equivalence Theory for Higher-Order Sessions}
		\begin{itemize}
			\item	A session based higher-order calculus - typed setting
%				We work on the typed setting using as a basis a session based higher-order calculus

			\item	A new solution on the higher-order equivelence problem based
				on principles that are
				used for the first time and are derived out of the behaviour of session types

			\item	The linear nature of session types allows to define
				%yet another
				a behavioural characterisation for typed Higher-order calculi
				which we call {\em Characteristic Bisimulation}

			\item	Characteristic bisimulation is defined on a refined labelled transition system
				that restricts process actions following the session behaviour.

			\item	A further interesting result is that, due to types, the proofs do
				not require the presence of a name matching construct.
		\end{itemize}
	\end{frame}

	\begin{frame}{Higher-Order Calculi}
		\begin{itemize}
			\item	Higher-Order Session $\pi$-calculus (\HOp).
		\end{itemize}

		\[
		\begin{array}{rcl}
			V, W &\bnfis& u \bnfbar \abs{x} P \qquad u\ \bnfis\ n \bnfbar x \\[2mm]
			P, Q &\bnfis& \bout{u}{V} P \bnfbar \binp{u}{x} P \bnfbar \appl{V}{W} \bnfbar P \Par Q \\
			&\bnfbar& \bsel{u}{l} P \bnfbar \bbra{u}{l_i: P_i}_{i \in I} \bnfbar \news{u} P \\
			&\bnfbar& \inact \bnfbar \varp{X} \bnfbar \recp{X}{P}
		\end{array}
		\]

		\begin{itemize}
			\item	Higher-order calculi may communicate processes as values
		\end{itemize}

		\[
			\begin{array}{rcl}
				\bout{u}{\abs{y}{R}} P \Par \binp{\dual{u}}{x} Q &\red& P \Par Q \subst{\abs{y}{R}}{x}\\
				(\appl{\abs{x}{P}}{V}) &\red& P \subst{V}{x}
			\end{array}
		\]
	\end{frame}

	\begin{frame}{Higher-Order Sessions}
		\begin{itemize}
			\item	\HOp is a higher-order calculus that specifies session protocols
		\end{itemize}

		\[
			\tree {
				\Gamma; \Lambda_1; \Delta_1 \cat n: S_2 \proves P \hastype \Proc
				\qquad
				\tree {
					\Gamma; \Lambda_2; \Delta_2 \cat x: S_1 \proves Q \hastype \Proc
				}{
					\Gamma; \Lambda_2; \Delta_2 \proves \abs{x}{Q} \hastype \shot{S_1}
				}
			}{
				\Gamma; \Lambda_1 \cat \Lambda_2; \Delta_1 \cat \Delta_2 \cat n: \btout{\shot{S_1}} S_2 \proves \bout{n}{\abs{x}{Q}} P \hastype \Proc
			}
		\]

		\begin{itemize}
			\item	Labelled transition system
		\end{itemize}

		\[
			\tree {
				(\Gamma; \es; \Delta) \by{\ell} (\Gamma; \es; \Delta') \qquad P \by{\ell} P'
			}{
				\Gamma; \Delta \proves P \by{\ell} \Delta' \proves P'
			}
		\]
	\end{frame}

	\begin{frame}{Equivalences}
		\begin{itemize}
			\item	Typed labelled transition semantics is used to define

			\begin{tabular}{lcl}
				$\cong$ &:& Reduction-closed, Barb-preserving, Congruence\\
				$\wbc$ &:& Context Bisimilarity\\
				$\fwb$ &:& Characteristic Bisimilarity
			\end{tabular}

		\end{itemize}
	\end{frame}

%	\begin{frame}{Labelled Transition System}
%	\end{frame}

	\begin{frame}{Higher-Order Contextual Bisimulation: Output}
		Suppose $P \,\Re\, Q$, for some context bisimulation~$\Re$. Then:
		\begin{enumerate}[$(\star)$]
			\item	Whenever
				\[
					\Gamma; \Delta_1 \proves P \by{\news{\widetilde{m_1}} \bactout{n}{V}} \Delta_1' \proves P'
				\]
				there exist $Q'$ and $W$ such that 
				\[
					\Gamma; \Delta_2 \proves Q \By{\news{\widetilde{m_2}} \bactout{n}{W}} \Delta_2' \proves Q'
				\]
				and, \fbox{\emph{\textbf{for all} $R$}}  with $\fv{R}=x$, 
				\[
					\Gamma; \Delta_1'' \proves \newsp{\widetilde{m_1}}{P' \Par R\subst{V}{x}}\ \Re\ \Delta_2'' \proves \newsp{\widetilde{m_2}}{Q' \Par R\subst{W}{x}}
				\]
		\end{enumerate}
	\end{frame}

	\begin{frame}{Higher-Order Contextual Bisimulation: Output}
		Equivalently rewrite the conclusion of the \textcolor{blue}{($\star$)} clause:
		\\[2mm]

		Suppose $P \,\Re\, Q$, for some context bisimulation~$\Re$. Then:
		\begin{enumerate}[$(\star)$]
			\item	Whenever\\
				\dots\\
%				\[
%					P \by{\news{\widetilde{m_1}} \bactout{n}{V}} P'
%				\]
%				there exist $Q'$ and $W$ such that 
%				\[
%					Q \By{\news{\widetilde{m_2}} \bactout{n}{W}} Q'
%				\]
%				and,
				\dots \emph{\textbf{for all} $R$}  with $\fv{R}=x$, 
				\[
					\begin{array}{rcc}
						\Gamma; \Delta_1'' &\proves& \newsp{\widetilde{m_1}}{P' \Par \newsp{s}{\binp{s}{x} R \Par  \bout{s}{V} \inact}}
						\\
						&&\Re
						\\
						\Delta_2'' &\proves&  \newsp{\widetilde{m_2}}{Q' \Par \newsp{s}{\binp{s}{x} R \Par \bout{s}{W} \inact}}
					\end{array}
				\]
				This is because:
				\[
					\begin{array}{c}
						\newsp{s}{\binp{s}{x} R \Par \bout{s}{V} \inact}
						\by{\tau}
						R \subst{V}{x}
						\\
						\newsp{s}{\binp{s}{x} R \Par \bout{s}{W} \inact}
						\by{\tau}
						R \subst{W}{x}
					\end{array}
				\]
		\end{enumerate}
	\end{frame}

	\begin{frame}{Higher-Order Contextual Bisimulation: Output}
		We can further rewrite the \textcolor{blue}{($\star$)} clause as a \underline{non universally quantified} input triggered context:
		\\[2mm]

		Suppose $P \,\Re\, Q$, for some context bisimulation~$\Re$. Then:
		\begin{enumerate}[$(\star)$]
			\item	Whenever\\
				\dots
%				\[
%					P \by{\news{\widetilde{m_1}} \bactout{n}{V}} P'
%				\]
%				there exist $Q'$ and $W$ such that 
%				\[
%					Q \By{\news{\widetilde{m_2}} \bactout{n}{W}} Q'
%				\]
%				and,
				%\dots \emph{\textbf{for all} $R$} with $\fv{R}=x$, 
				\[
					\begin{array}{rcc}
						\Gamma; \Delta_1'' &\proves& \newsp{\widetilde{m_1}}{P' \Par \binp{t}{y}\newsp{s}{\binp{s}{z} \appl{y}{z} \Par \bout{s}{V} \inact}}
						\\
						&& \Re
						\\
						\Delta_1'' &\proves& \newsp{\widetilde{m_2}}{Q' \Par \binp{t}{y}\newsp{s}{\binp{s}{z} \appl{y}{z} \Par \bout{s}{W} \inact}}
					\end{array}
				\]
		\end{enumerate}

		This is because \emph{\textbf{for all} $R$} with $\fv{R}=x$, 
		\[
			\begin{array}{c}
			\binp{t}{y}\newsp{s}{\binp{s}{z} \appl{y}{z} \Par \bout{s}{V} \inact}
			\by{\bactinp{t}{\abs{x}{R}}} \by{\tau}
			R \subst{V}{x}
			\\
			\binp{t}{y}\newsp{s}{\binp{s}{z} \appl{y}{z} \Par \bout{s}{W} \inact}
			\by{\bactinp{t}{\abs{x}{R}}} \by{\tau}
			R \subst{W}{x}
			\end{array}
		\]
	\end{frame}

	\begin{frame}{Higher-Order Contextual Bisimulation: Input}
		Suppose $P \,\Re\, Q$, for some context bisimulation~$\Re$. Then:
		\begin{enumerate}[$(\bullet)$]
			\item	\fbox{\emph{\textbf{For all} $V$}} such that:
				\[
					\Gamma; \Delta_1 \proves P \by{\bactinp{n}{V}} \Delta_1' \proves P'
				\]
				then there exists $Q'$ such that
				\[
					\Gamma; \Delta_2 \proves Q \By{\bactinp{b}{V}} \Delta_2' \proves Q'
				\]
				and
				\[
					\Gamma; \Delta_1' \proves P'\ \Re\ \Delta_2' \proves Q'
				\]
		\end{enumerate}
	\end{frame}

	\begin{frame}{Higher-Order Contextual Bisimulation: Input}
		We can take advantage of session types to improve the
		\textcolor{blue}{$(\bullet)$} clause.
		\vspace{3mm}

		\begin{definition}[Characteristic Process and Value]
			\begin{itemize}
				\item	Type $U$ is inhabited by a simple
					{\em characteristic} process, $\mapchar{U}{u}$.

				\item	Type $U$ is inhabited by simple
					{\em characteristic} value, $\omapchar{U}$.
			\end{itemize}
%			where $u$ is a name.
		\end{definition}
		%\vspace{1mm}

		For example session:
		\[
			S = \btinp{\shot{S_1}} \btout{S_2} \tinact
		\]
		is inhabited by {\em characteristic process}:
		\[
			\mapchar{S}{u} = \binp{u}{x} (\bout{u}{s_2} \inact \Par \appl{x}{s_1})
		\]
		for a fresh name $u$.
%		\vspace{3mm}
%
%		Characteristic types for values are inhabited as:
%		\[
%			\omapchar{S} = s \qquad \omapchar{\shot{S}} = \abs{x}{\mapchar{S}{x}}
%		\]
	\end{frame}

	\begin{frame}{Higher-Order Contextual Bisimulation: Input}
		We define a coarser labelled transition system:

		\[
			\tree {
				\begin{array}{c}
					%\Gamma; \Delta \proves P \by{\bactinp{n}{V}} \Delta' \proves P' \\
					(\Gamma_1; \Lambda_1; \Delta_1) \by{\bactinp{n}{V}} (\Gamma_1; \Lambda_2; \Delta_2)\\
					V = m \vee V \scong \omapchar{U} \vee V \scong \abs{x}{\binp{t}{y} (\appl{y}{x})} \textrm{ with $t$ fresh}
				\end{array}
			}{
				(\Gamma_1; \Lambda_1; \Delta_1) \hby{\bactinp{n}{V}} (\Gamma_1; \Lambda_2; \Delta_2)
%				\Gamma; \Delta \proves P \hby{\bactinp{n}{V}} \Delta' \proves P'
			}
		\]

		\begin{itemize}
			\item	Define $\Gamma; \Delta \proves P \hby{\ell} \Delta' \proves P'$
			\item	Value $\abs{x}{\binp{t}{y} (\appl{y}{x})}$ is called {\em trigger value}
			\item	Lack of the trigger value in the above definition would result
				in an under-discriminating equivalence relation (example~12 in the paper)
		\end{itemize}
	\end{frame}

	\begin{frame}{Characteristic Bisimulation: Trigger Process}
		In the light of the coarser semantics that
		relation $\hby{\ell}$ offers, process:
		\[
			\newsp{\widetilde{m_1}}{P' \Par \binp{t}{y}\newsp{s}{\binp{s}{z} \appl{y}{z} \Par \bout{s}{V} \inact}}
		\]
		is equivalent with process:
		\[
			\newsp{\widetilde{m_1}}{P' \Par \binp{t}{y}\newsp{s}{\mapchar{\btinp{U_1} \tinact}{s} \Par \bout{s}{V} \inact}}
		\]
	\end{frame}

	\begin{frame}{Characteristic Bisimulation}

		Suppose $P\, \Re\, Q$ for some {\em characteristic} bisimulation~$\Re$. Then
		\begin{enumerate}[$(\star)$]
			\item	Whenever
				\[
					\Gamma; \Delta_1 \proves P \hby{\news{\widetilde{m_1}} \Delta_1' \proves \bactout{n}{V: U_1}} P'
				\]
				there exist $Q'$ and $W$ such that 
				\[
					\Gamma; \Delta_2 \proves Q \Hby{\news{\widetilde{m_2}} \bactout{n}{W: U_2}} \Delta_2' \proves Q'
				\]
				and
				\[
					\begin{array}{c}
						\Gamma; \Delta_1'' \proves \newsp{\widetilde{m_1}}{P' \Par \binp{t}{y}\newsp{s}{\mapchar{\btinp{U_1} \tinact}{s} \Par \bout{s}{V} \inact}}
						\\
						\Re
						\\
						\Delta_2'' \proves \newsp{\widetilde{m_2}}{Q' \Par \binp{t}{y}\newsp{s}{\mapchar{\btinp{U_2} \tinact}{s} \Par \bout{s}{W} \inact}}
					\end{array}
				\]
		\end{enumerate}
	\end{frame}

	\begin{frame}{Characteristic Bisimulation}

		\begin{enumerate}[$(\bullet)$]
			\item	Whenever
				\[
					\Gamma; \Delta_1 \proves P \hby{\bactinp{n}{V}} \Delta_1' \proves P'
				\]
				there exist $Q'$ and $W$ such that 
				\[
					\Gamma; \Delta_2 \proves Q \Hby{\bactout{n}{V}} \Delta_2' \proves Q'
				\]
				and
				\[
					\Gamma; \Delta_1' \proves P'\, \Re\, \Delta_2' \proves Q'
				\]
		\end{enumerate}
	\end{frame}

	\begin{frame}{Soundness and Completeness}
%		\begin{tabular}{lcc}
%			Reduction-closed, Barb-preserving, Congruence&: & $\cong$\\
%			Context Bisimilarity&: & $\wbc$\\
%			Characteristic Bisimilarity&: & $\fwb$
%		\end{tabular}
%		\vspace{5mm}
		\begin{itemize}
			\item	The behavioural equivalences coincide

				\begin{theorem}
					\begin{center}
					$\cong \qquad =\qquad \wbc \qquad =\qquad \fwb$
					\end{center}
				\end{theorem}

			\item	No matching construct is required for proofs - types contain all the information needed

		\end{itemize}
	\end{frame}

	\begin{frame}{No matching: Example}
			\begin{eqnarray*}
				&&\es; \es; n: \btout{S} \tinact, m_1 : S \proves P = \bout{n}{m_1} \inact \\
				&&\es; \es; n: \btout{S} \tinact, m_2 : S \proves Q = \bout{n}{m_2} \inact
			\end{eqnarray*}
			Observe actions $\bactout{n}{m_1}$ and $\bactout{n}{m_2}$, respectively.
			\begin{eqnarray*}
				&&\es; \es; m_1 : S \proves P' = \btinp{t}{y} \newsp{s}{\mapchar{\btinp{S} \tinact}{s}\Par \bout{\dual{s}}{m_1}\inact} \\
				&&\es; \es; m_2 : S \proves Q' = \btinp{t}{y} \newsp{s}{\mapchar{\btinp{S} \tinact}{s}\Par \bout{\dual{s}}{m_2}\inact} 
			\end{eqnarray*}

			\begin{itemize}
				\item	If $S = \tinact$ then $P$ and $Q$ are bisimilar because there is no further observation on $m_1$ and $m_2$
				\item	Otherwise $P'$ and $Q'$ can be distinguished because we observe interaction on $m_1$ and $m_2$,
					respectively.
			\end{itemize}
	\end{frame}

	\begin{frame}{Summary of the Results}
		\begin{itemize}
			\item	Session based higher-order $\pi$-calculus
				\begin{itemize}
					\item	Syntax and semantics
					\item	Typed labelled transition semantics
				\end{itemize}
			\item	Theory on behavioural equivalence
				\begin{itemize}
					\item	Reduction-closed, barb-preserving congruence
					\item	Context bisimularity
					\item	Characteristic bisimilarity
				\end{itemize}

			\item	Characteristic bisimilarity
				\begin{itemize}
					\item	Information from session types
					\item	Principle applied for the first time in process calculi equivalence theory
					\item	No matching construct is required have a full contextual characterisation
				\end{itemize}
		\end{itemize}
	\end{frame}


	\begin{frame}{Questions?}
		\begin{center}
			\huge Thank you for your attention
		\end{center}
	\end{frame}

	\begin{frame}{The need for trigger value}
%		\begin{example}[The Need for Refined Typed LTS]
%		\label{ex:motivation}
%		We show that observing a characteristic value
%		input alone is not enough
%		\dk{to define a sound bisimulation closure}.
		Consider   processes: % $P_1, P_2$:
		%
		\begin{eqnarray*}
			P_1 = \binp{s}{x} (\appl{x}{s_1} \Par \appl{x}{s_2}) 
			& & 
			P_2 = \binp{s}{x} (\appl{x}{s_1} \Par \binp{s_2}{y} \inact) 
%			\label{equ:6}
		\end{eqnarray*}
		%
		%We can show that 
		with
		\[
			\Gamma; \es; \Delta \cat s: \btinp{\shot{(\btinp{C} \tinact)}} \tinact \proves P_i \hastype \Proc (i \in \{1,2\})
		\]

		If $P_1$ and $P_2$ input and substitute over $x$
		the characteristic value:
		\[
			\omapchar{\shot{(\btinp{C} \tinact)}} = \abs{x}{\binp{x}{y} \inact}
		\] 
		then they evolve into:%(\ref{eq:5}) and (\ref{eq:6}) in become:
		\begin{center}
		%\begin{tabular}{c}
			$\Gamma; \es; \Delta \proves \binp{s_1}{y} \inact \Par \binp{s_2}{y} \inact \hastype \Proc$
		%\end{tabular}
		\end{center}
		\noi therefore becoming 
		context bisimilar.
	\end{frame}

	\begin{frame}{The need for trigger value}
		%after the input of $\abs{x}{\binp{x}{y}} \inact$.
		However, $P_1$ and $P_2$
%		the processes in (\ref{equ:6}) 
		are
%clearly
		{\em not} context bisimilar:
%		: many input actions
%		may be used to distinguish them.
%		For example,
		If 
		$P_1$ and $P_2$ input 
		\[
			\abs{x} \newsp{s}{\bout{a}{s} \binp{x}{y} \inact}
		\]
		with
		\[
			\Gamma; \es; \Delta \proves s \hastype \tinact
		\]
		then the derivatives
		\begin{eqnarray*}
			P_1' &=& \appl{\abs{x}{\newsp{s}{\bout{a}{s} \binp{x}{y} \inact}}}{s_1} \Par \appl{\abs{x}{\newsp{s}{\bout{a}{s} \binp{x}{y} \inact}}}{s_2}\\
			P_2' &=& \appl{\abs{x}{\newsp{s}{\bout{a}{s} \binp{x}{y} \inact}}}{s_2} \Par \binp{s_2}{y} \inact
		\end{eqnarray*}
		are not bisimilar.
	\end{frame}

	\begin{frame}{The need for trigger value}
%		Observing only the characteristic value 
%		results in an under-discriminating bisimulation.
		However, if a trigger value
		\[
			\abs{{x}}{\binp{t}{y} (\appl{y}{{x}})}
		\]
		is received on $s$, we can distinguish $P_1$, $P_2$ ($\ell = \bactinp{s}{\abs{{x}}{\binp{t}{y} (\appl{y}{{x}})}}$):
		%
		\begin{eqnarray*}
			P_1 &\By{\ell}& \binp{t}{x} (\appl{x}{s_1}) \Par \binp{t}{x} (\appl{x}{s_2})\\
%			\mbox{~and~}
			P_2 &\By{\ell}& \binp{t}{x} (\appl{x}{s_1}) \Par \binp{s_2}{y} \inact
%			\quad 
		\end{eqnarray*}
	\end{frame}

	\begin{frame}{The need for trigger value}
		The trigger value alone is not enough.
		Consider processes:
		%
		\begin{eqnarray*}
			&&\newsp{s}{\binp{n}{x} (\appl{x}{s}) \Par \bout{\dual{s}}{\abs{x} R_1} \inact} \\
			&&\newsp{s}{\binp{n}{x} (\appl{x}{s}) \Par \bout{\dual{s}}{\abs{x} R_2} \inact} 
		\end{eqnarray*}
		%
		On a trigger value input, we obtain:
		\begin{eqnarray*}
			&&\newsp{s}{\binp{t}{x} (\appl{x}{s}) \Par \bout{\dual{s}}{\abs{x} R_1} \inact} \\
			&&\newsp{s}{\binp{t}{x} (\appl{x}{s}) \Par \bout{\dual{s}}{\abs{x} R_2} \inact}
		\end{eqnarray*}

		\noi thus conclude a bisimulation closure.

		But
		on a characteristic value ($\abs{z}{\binp{z}{x} (\appl{x}{m})}$) input 
		then they would become:
		%
		\begin{eqnarray*}
			\Gamma; \es; \Delta \proves \newsp{s}{\binp{s}{x} (\appl{x}{m}) \Par \bout{\dual{s}}{\abs{x} R_i} \inact} \wbc \Delta \proves R_i \subst{m}{x}
		\quad (i=1,2)
		\end{eqnarray*}
		\noi which are not bisimilar if $R_1 \subst{m}{x} \not\wbc R_2 \subst{m}{x}$.
	\end{frame}

\end{document}

