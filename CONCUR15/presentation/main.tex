\documentclass{beamer}

\usepackage{amsmath}
\usepackage{stmaryrd}
\usepackage{xspace}
%\usepackage{color}


\usepackage{comment}
\usepackage{multirow}
\usepackage{tikz}

\usetikzlibrary{calc}

%%%%%%%%%%%%%%%%%%%%%%%%%%%%%%%%%%%%%%%%%%%%%%%%%%%%%%%%%%%%%%%%%%%%%%%%%%%%%%%%%%%%%%%%%%%%%%%%%%%%
% Contents
% --------

% 1.  Formating
% 2.  Maths - Theorems
% 3.  The pi Calculus
% 4.  Session Syntax
% 5.  Subject Reduction
% 6.  Global Session Types
% 7.  Global Session Types Equivalence
% 8.  Projection
% 9.  Local Session Types
% 10. Behavioural Theory
% 11. Typed Transitions - Reductions
% 12. Typed Relations
% 13. Confluence Determinacy
% 14. Mapping
% 15. pi Constructs
% 16. LN Transform
% 17. General Types Processes Names Sessions ETC
% 18. newtheorem - newenvironment
% 19. Misc
%%%%%%%%%%%%%%%%%%%%%%%%%%%%%%%%%%%%%%%%%%%%%%%%%%%%%%%%%%%%%%%%%%%%%%%%%%%%%%%%%%%%%%%%%%%%%%%%%%%%


%%%%%%%%%%%%%%%%%%%%%%%%%%%%%%%%%%%%%%%%%%%%%%%%%%%%%%%%%%%%%%%%%%%%%%%%%%%%%%%%%%%%%%%%%%%%%%%%%%%%
%                                       FORMATING
%%%%%%%%%%%%%%%%%%%%%%%%%%%%%%%%%%%%%%%%%%%%%%%%%%%%%%%%%%%%%%%%%%%%%%%%%%%%%%%%%%%%%%%%%%%%%%%%%%%%

% Symbols
\newcommand{\semicolon}{:}
%\newcommand{\colon}{;}
\newcommand{\lrangle}[1]{\langle #1 \rangle}
\newcommand{\blrangle}[1]{\big\langle #1 \big\rangle}

%Tags
\newcommand{\parenthtext}[1]{(\textrm{\small #1})}
\newcommand{\brtext}[1]{[\textrm{\small #1}]}
\newcommand{\textinmath}[1]{\textrm{#1}}
\newcommand{\srule}[1]{\parenthtext{#1}}
\newcommand{\strule}[1]{\textrm{#1}}
\newcommand{\stypes}[1]{{\footnotesize \parenthtext{#1}}}
\newcommand{\ltsrule}[1]{{\footnotesize \lrangle{\textrm{#1}}}}
\newcommand{\eltsrule}[1]{{\footnotesize [\textrm{#1}]}}
\newcommand{\trule}[1]{{\footnotesize\brtext{#1}}}
\newcommand{\orule}[1]{{\scriptsize{\brtext{#1}}}}
\newcommand{\mrule}[1]{{\footnotesize{\parenthtext{#1}}}}

\newcommand{\iftag}{{\textrm{if }}}

% General
\newcommand{\noi}{\noindent}
\newcommand{\Hline}{\rule{\linewidth}{.5pt}}
\newcommand{\Hlinefig}{\rule{\linewidth}{.5pt}\vspace{-4mm}}
\newcommand{\myparagraph}[1]{\noindent{\textbf{#1}\ }}
\newcommand{\jparagraph}[1]{\paragraph{\textbf{#1}}}

%%%%%%%%%%%%%%%%%%%%%%%%%%%%%%%%%%%%%%%%%%%%%%%%%%%%%%%%%%%%%%%%%%%%%%%%%%%%%%%%%%%%%%%%%%%%%%%%%%%%
%                                       MATHS - THEOREMS
%%%%%%%%%%%%%%%%%%%%%%%%%%%%%%%%%%%%%%%%%%%%%%%%%%%%%%%%%%%%%%%%%%%%%%%%%%%%%%%%%%%%%%%%%%%%%%%%%%%%

%\newtheorem{notation}[definition]{Notation}

% BNF form
\newcommand{\bnfis}{\;\;::=\;\;}
\newcommand{\bnfbar}{\;\;\;|\;\;\;}
\newcommand{\sbnfbar}{\;\;|\;\;}

% Proof
\newcommand{\Case}[1]{\noi {\bf Case: }#1\\}
\newcommand{\proofend}{\qed}
%\newcommand{\proofend}{}
\newcommand{\Proof}{\noi {\bf Proof: }}

% Logic
\newcommand{\LogAnd}{\texttt{ and }}
\newcommand{\LogOr}{\texttt{ or }}

% Induction

\newcommand{\basic}{\noi {\bf Basic Step:}\\}
\newcommand{\inductive}{\noi {\bf Inductive Hypothesis:}\\}
\newcommand{\induction}{\noi {\bf Induction Step:}\\}

% Tree

\newcommand{\tree}[2]{
\ensuremath{\displaystyle
		\frac
		{
			%%\raisebox{0.0mm}{$\displaystyle{#1}$}
			#1
			%\vspace{0mm}
		}{
			%\vspace{2mm}
			#2
			%\raisebox{-0.4mm}{$\displaystyle{#2}$}
		}
	}
}


%\newcommand{\tree}[2]{
%\begin{prooftree}
%	#1
%	\justifies
%	#2
%\end{prooftree}
%}

\newcommand{\treeusing}[3]{
\begin{prooftree}
	#1
	\justifies
	#2
	\using
	#3
\end{prooftree}}

% Vectors
\newcommand{\vect}[1]{\tilde{#1}}
\newcommand{\mytilde}[1]{\widetilde{#1}}

% Functions - Set theory
\newcommand{\set}[1]{\{#1\}}
\newcommand{\es}{\emptyset}
\newcommand{\maxset}[1]{\max(#1)}
\newcommand{\setbar}{\ \ |\ \ }
\newcommand{\tuple}[2]{(#1, #2)}
\newcommand{\suchthat}{\cdot}
\newcommand{\powerset}[1]{\mathcal{P}(#1)}
\newcommand{\product}{\times}

\newcommand{\eval}{\downarrow}

\newcommand{\setsubtr}[2]{#1 \backslash #2}

\newcommand{\func}[2]{#1(#2)}
\newcommand{\dom}[1]{\mathtt{dom}(#1)}
\newcommand{\codom}[1]{\mathtt{codom}(#1)}

\newcommand{\funcbr}[2]{#1\lrangle{#2}}

\newcommand{\entails}{\text{implies}}


%%%%%%%%%%%%%%%%%%%%%%%%%%%%%%%%%%%%%%%%%%%%%%%%%%%%%%%%%%%%%%%%%%%%%%%%%%%%%%%%%%%%%%%%%%%%%%%%%%%%
%                                        pi - CALCULUS
%%%%%%%%%%%%%%%%%%%%%%%%%%%%%%%%%%%%%%%%%%%%%%%%%%%%%%%%%%%%%%%%%%%%%%%%%%%%%%%%%%%%%%%%%%%%%%%%%%%%

% Free-Bound notation
\newcommand{\freev}[1]{\lrangle{#1}}
\newcommand{\boundv}[1]{(#1)}

% General pi calculus Syntax
\newcommand{\send}[1]{\overline{#1}}
\newcommand{\ol}[1]{\overline{#1}}
\newcommand{\receive}[1]{#1.}
\newcommand{\inact}{\mathbf{0}}
\newcommand{\If}{\sessionfont{if}\ }
\newcommand{\Then}{\sessionfont{then}\ }
\newcommand{\Else}{\sessionfont{else}\ }
\newcommand{\ifthen}[2]{\If #1\ \Then #2\ }
\newcommand{\ifthenelse}[3]{\ifthen{#1}{#2} \Else #3}
\newcommand{\Par}{\;|\;}
\newcommand{\news}[1]{(\nu\, #1)}
\newcommand{\newsp}[2]{(\nu\, #1)(#2)}
\newcommand{\varp}[1]{#1}
%\newcommand{\rvar}[1]{\mathcal{#1}}
\newcommand{\rvar}[1]{#1}
%\newcommand{\rec}[2]{\mu #1. #2}
\newcommand{\recp}[2]{\mu \rvar{#1}. #2}

\newcommand{\Def}{\sessionfont{def}\ }

\newcommand{\defeq}{\stackrel{\Def}{=}}

\newcommand{\repl}{\ast\,}
\newcommand{\parcomp}[2]{\prod_{#1}{#2}}

% Free-Bound-Names sets
\newcommand{\bn}[1]{\mathtt{bn}(#1)}
\newcommand{\fn}[1]{\mathtt{fn}(#1)}
\newcommand{\ofn}[1]{\mathsf{ofn}(#1)}
\newcommand{\fv}[1]{\mathtt{fv}(#1)}
\newcommand{\bv}[1]{\mathtt{bv}(#1)}
\newcommand{\fs}[1]{\mathtt{fs}(#1)}
\newcommand{\fpv}[1]{\mathtt{fpv}(#1)}
\newcommand{\nam}[1]{\mathtt{n}(#1)}

%Subject - Object
\newcommand{\subj}[1]{\mathtt{subj}(#1)}
\newcommand{\obj}[1]{\mathtt{obj}(#1)}

% Relations
\newcommand{\relfont}[1]{\mathcal{#1}}
\newcommand{\rel}[3]{#1\ \relfont{#2}\ #3}

\newcommand{\scong}{\equiv}
\newcommand{\acong}{\scong_{\alpha}}
\newcommand{\wb}{\approx}
\newcommand{\fwb}{\approx^C}
\newcommand{\hwb}{\approx^H}
\newcommand{\swb}{\approx^{s}}
\newcommand{\wbc}{\approx}
\newcommand{\WB}{\approx}

\newcommand{\red}{\longrightarrow}
\newcommand{\Red}{\rightarrow\!\!\!\!\!\rightarrow}
\newcommand{\Redleft}{\leftarrow\!\!\!\!\!\leftarrow}

%\newcommand{\subst}[2]{\set{#1/#2 }}
\def\subst#1#2{\{\raisebox{.5ex}{\small$#1$}\! / \mbox{\small$#2$}\}}

% Context
\newcommand{\hole}{-}
\newcommand{\context}[2]{#1[#2]}
\newcommand{\Ccontext}[1]{\C[#1]}

% Expression Context
\newcommand{\Econtext}[1]{\E[#1]}

% Barbs
\newcommand{\barb}[1]{\downarrow_{#1}}
\newcommand{\Barb}[1]{\Downarrow_{#1}}
\newcommand{\nbarb}[1]{\not\downarrow_{#1}}
\newcommand{\nBarb}[1]{\not\Downarrow_{#1}}

% General
%\newcommand{\ESP}{\ensuremath{\mathbf{ESP}}}
\newcommand{\ESP}{\text{ESP}}
\newcommand{\ESPsel}{\ESP^+}

%%%%%%%%%%%%%%%%%%%%%%%%%%%%%%%%%%%%%%%%%%%%%%%%%%%%%%%%%%%%%%%%%%%%%%%%%%%%%%%%%%%%%%%%%%%%%%%%%%%%
%                                        SESSION SYNTAX
%%%%%%%%%%%%%%%%%%%%%%%%%%%%%%%%%%%%%%%%%%%%%%%%%%%%%%%%%%%%%%%%%%%%%%%%%%%%%%%%%%%%%%%%%%%%%%%%%%%%

% Session font
\newcommand{\sessionfont}[1]{\mathtt{#1}}
\newcommand{\vart}[1]{\mathsf{#1}}

% General Session symbols
\newcommand{\ssep}{;}
\newcommand{\shsep}{.}
\newcommand{\outses}{!}
\newcommand{\inpses}{?}
\newcommand{\selses}{\triangleleft}
\newcommand{\brases}{\triangleright}
\newcommand{\dual}[1]{\overline{#1}}
\newcommand{\cat}{\cdot}

\newcommand{\allstypes}{\mathcal{S}}

% Binary Session Syntax


\newcommand{\bacc}[2]{#1 \boundv{#2} \shsep}
\newcommand{\breq}[2]{\send{#1} \freev{#2} \shsep}
\newcommand{\bareq}[2]{\send{#1} \freev{#2}}

\newcommand{\breqt}[3]{\send{#1} \boundv{#2:#3} \shsep}
\newcommand{\bacct}[3]{#1 \boundv{#2:#3} \shsep}

\newcommand{\bout}[2]{#1 \outses \freev{#2} \shsep}
\newcommand{\bbout}[2]{#1 \outses \blrangle{#2} \shsep}
\newcommand{\binp}[2]{#1 \inpses \boundv{#2} \shsep}
\newcommand{\bsel}[2]{#1 \selses #2 \shsep}

%\newcommand{\bout}[2]{#1 \outses \freev{#2} \ssep}
%\newcommand{\bbout}[2]{#1 \outses \blrangle{#2} \ssep}
%\newcommand{\binp}[2]{#1 \inpses \boundv{#2} \ssep}
%\newcommand{\bsel}[2]{#1 \selses #2 \ssep}
\newcommand{\bbra}[2]{#1 \brases \set{#2}}
\newcommand{\bbras}[2]{#1 \brases #2}
\newcommand{\bbraP}[1]{#1 \brases \lPi}

% Multiparty Session syntax

\newcommand{\role}[1]{[#1]}

\newcommand{\srole}[2]{#1\role{#2}}
\newcommand{\sqrole}[2]{#1^{[]}\role{#2}}

\newcommand{\fromto}[2]{\role{#1} \role{#2}}
\newcommand{\sfromto}[3]{#1\fromto{#2}{#3}}

\newcommand{\sout}[3]{\srole{#1}{#2} \outses \freev{#3} \ssep}
\newcommand{\sinp}[3]{\srole{#1}{#2} \inpses \boundv{#3} \ssep}
\newcommand{\sdel}[4]{\srole{#1}{#2} \outses \freev{\srole{#3}{#4}} \ssep}
\newcommand{\ssel}[3]{\srole{#1}{#2} \selses #3 \ssep}
\newcommand{\sbra}[3]{\srole{#1}{#2} \brases \set{#3}}
\newcommand{\sbras}[3]{\srole{#1}{#2} \brases #3}
\newcommand{\sbraP}[2]{\srole{#1}{#2} \brases \lPi}

\newcommand{\acc}[3]{#1 \role{#2} \boundv{#3} \shsep}
\newcommand{\req}[3]{\send{#1} \role{#2} \boundv{#3} \shsep}
\newcommand{\areq}[3]{\send{#1} \role{#2} \freev{#3}}

\newcommand{\out}[4]{\sfromto{#1}{#2}{#3} \outses \freev{#4} \ssep}
\newcommand{\inp}[4]{\sfromto{#1}{#2}{#3} \inpses \boundv{#4} \ssep}
\newcommand{\del}[5]{\sfromto{#1}{#2}{#3} \outses \freev{\srole{#4}{#5}} \ssep}
\newcommand{\sel}[4]{\sfromto{#1}{#2}{#3} \selses #4 \ssep}
\newcommand{\bra}[4]{\sfromto{#1}{#2}{#3} \brases \set{#4}}
\newcommand{\bras}[4]{\sfromto{#1}{#2}{#3} \brases #4}
\newcommand{\braP}[3]{\sfromto{#1}{#2}{#3} \brases \lPi}

% Arrive construct
\newcommand{\arrivetext}{\mathtt{arrive}}
\newcommand{\arrive}[1]{\arrivetext\ #1}
\newcommand{\arrivem}[2]{\arrivetext\ #1\ #2}

% Typecase construct
\newcommand{\typecasetext}{\mathtt{typecase}}
\newcommand{\oftext}{\mathtt{of}}
\newcommand{\typecase}[2]{\typecasetext\ #1\ \oftext\ \set{#2}}

% IO symbols
\newcommand{\inputsym}{\mathtt{i}}
\newcommand{\outputsym}{\mathtt{o}}

%%%%%%%%%%%%%%%%%%%%%%%%%%%%%%%%%%%%%%%%%%%%%%%%%%%%%%%%%%%%%%%%%%%%%%%%%%%%%%%%%%%%%%%%%%%%%%%%%%%%
%                                      SUBJECT REDUCTION
%%%%%%%%%%%%%%%%%%%%%%%%%%%%%%%%%%%%%%%%%%%%%%%%%%%%%%%%%%%%%%%%%%%%%%%%%%%%%%%%%%%%%%%%%%%%%%%%%%%%

% typing reduction
\newcommand{\typingred}{\red}
\newcommand{\typingRed}{\Red}

\newcommand{\wellconf}[1]{\mathtt{wc}(#1)}
\newcommand{\cohses}[2]{\mathtt{co}(#1(#2))}
\newcommand{\coherent}[1]{\mathtt{co}(#1)}
\newcommand{\fcoherent}[1]{\mathtt{fco}(#1)}

%%%%%%%%%%%%%%%%%%%%%%%%%%%%%%%%%%%%%%%%%%%%%%%%%%%%%%%%%%%%%%%%%%%%%%%%%%%%%%%%%%%%%%%%%%%%%%%%%%%%
%                                      SESSION ENDPOINTS
%%%%%%%%%%%%%%%%%%%%%%%%%%%%%%%%%%%%%%%%%%%%%%%%%%%%%%%%%%%%%%%%%%%%%%%%%%%%%%%%%%%%%%%%%%%%%%%%%%%%

% Asynchronous syntax
%\newcommand{\mareq}[4]{\newsp{\srole{#2}{#3}, \dots, \srole{#2}{#4}}{\send{#1}[#3] \freev{#2} \Par \dots \Par \send{#1}[#4]\freev{#2}}}

%\newcommand{\areqs}[3]{\send{#1}[\set{#2}] \freev{#3}}

% Queues
\newcommand{\emp}{\epsilon}
\newcommand{\squeue}[3]{\srole{#1}{#2}:#3}
\newcommand{\srqueue}[4]{\srole{#1}{#2}[\inputsym: #3, \outputsym: #4]}
\newcommand{\srqueuei}[3]{\srole{#1}{#2}[\inputsym: #3]}
\newcommand{\srqueueo}[3]{\srole{#1}{#2}[\outputsym: #3]}
\newcommand{\srqueueio}[4]{\srole{#1}{#2}[\inputsym: #3, \outputsym: #4]}
\newcommand{\sgqueue}[2]{\srole{#1}:#2}

% Shared Names Queues
\newcommand{\shqueue}[2]{#1[#2]}
%\newcommand{\shqueuet}[3]{#1[#2, #3]}

% IO Queues
\newcommand{\squeueio}[3]{#1[\inputsym: #2, \outputsym: #3]}
\newcommand{\squeuei}[2]{#1[\inputsym: #2]}
\newcommand{\squeueo}[2]{#1[\outputsym: #2]}

\newcommand{\squeuetio}[4]{#1[#2, \inputsym: #3, \outputsym: #4]}
\newcommand{\squeueto}[3]{#1[#2, \outputsym: #3]}
\newcommand{\squeueti}[3]{#1[#2, \inputsym: #3]}
\newcommand{\squeuet}[2]{#1[#2]}

% Queue Values

\newcommand{\queuev}[2]{\role{#1}(#2)}
\newcommand{\queuel}[2]{\role{#1} #2}
\newcommand{\queues}[3]{\role{#1}(\srole{#2}{#3})}

%%%%%%%%%%%%%%%%%%%%%%%%%%%%%%%%%%%%%%%%%%%%%%%%%%%%%%%%%%%%%%%%%%%%%%%%%%%%%%%%%%%%%%%%%%%%%%%%%%%%
%                                        GLOBAL SESSION TYPES
%%%%%%%%%%%%%%%%%%%%%%%%%%%%%%%%%%%%%%%%%%%%%%%%%%%%%%%%%%%%%%%%%%%%%%%%%%%%%%%%%%%%%%%%%%%%%%%%%%%%

\newcommand{\gtfont}[1]{\mathtt{#1}}
\newcommand{\gsep}{.}

\newcommand{\globaltype}[1]{\lrangle{#1}}
\newcommand{\parties}[1]{\mathtt{\p}(#1)}
\newcommand{\roles}[1]{\mathtt{roles}(#1)}

\newcommand{\fromtogt}[2]{#1 \rightarrow #2 \semicolon}

\newcommand{\valuegt}[3]{\fromtogt{#1}{#2} \lrangle{#3} \gsep}
\newcommand{\selgt}[3]{\fromtogt{#1}{#2} \set{#3}}
\newcommand{\selgtG}[2]{\fromtogt{#1}{#2} \lGi}
\newcommand{\recgt}[2]{\mu \vart{#1}. #2}
\newcommand{\vargt}[1]{\vart{#1}}
\newcommand{\inactgt}{\gtfont{end}}

%%%%%%%%%%%%%%%%%%%%%%%%%%%%%%%%%%%%%%%%%%%%%%%%%%%%%%%%%%%%%%%%%%%%%%%%%%%%%%%%%%%%%%%%%%%%%%%%%%%%
%                              GLOBAL SESSION TYPES EQUIVALENCE
%%%%%%%%%%%%%%%%%%%%%%%%%%%%%%%%%%%%%%%%%%%%%%%%%%%%%%%%%%%%%%%%%%%%%%%%%%%%%%%%%%%%%%%%%%%%%%%%%%%%

\newcommand{\projset}[1]{\mathtt{proj}(#1)}
\newcommand{\aprojset}[1]{\mathtt{aproj}\ #1 }
\newcommand{\gcong}{\equiv}
\newcommand{\govcong}{\cong_g}
\newcommand{\gperm}{\simeq}

%%%%%%%%%%%%%%%%%%%%%%%%%%%%%%%%%%%%%%%%%%%%%%%%%%%%%%%%%%%%%%%%%%%%%%%%%%%%%%%%%%%%%%%%%%%%%%%%%%%%
%                                        PROJECTION
%%%%%%%%%%%%%%%%%%%%%%%%%%%%%%%%%%%%%%%%%%%%%%%%%%%%%%%%%%%%%%%%%%%%%%%%%%%%%%%%%%%%%%%%%%%%%%%%%%%%

\newcommand{\projsymb}{\lceil}
\newcommand{\proj}[2]{#1 \projsymb #2}

%%%%%%%%%%%%%%%%%%%%%%%%%%%%%%%%%%%%%%%%%%%%%%%%%%%%%%%%%%%%%%%%%%%%%%%%%%%%%%%%%%%%%%%%%%%%%%%%%%%%
%                                        LOCAL SESSION TYPES
%%%%%%%%%%%%%%%%%%%%%%%%%%%%%%%%%%%%%%%%%%%%%%%%%%%%%%%%%%%%%%%%%%%%%%%%%%%%%%%%%%%%%%%%%%%%%%%%%%%%

\newcommand{\tfont}[1]{\mathtt{#1}}
\newcommand{\tsep}{;}

\newcommand{\chtype}[1]{\lrangle{#1}}
\newcommand{\chtypei}[1]{\inputsym \lrangle{#1}}
\newcommand{\chtypeo}[1]{\outputsym \lrangle{#1}}
\newcommand{\chtypeio}[1]{\inputsym \outputsym \lrangle{#1}}

\newcommand{\outtype}{\outses}
\newcommand{\inptype}{\inpses}
\newcommand{\seltype}{\selses}
\newcommand{\bratype}{\brases}

\newcommand{\trec}[2]{\mu\vart{#1}.#2}
\newcommand{\tvar}[1]{\vart{#1}}
%\newcommand{\settype}[1]{\set{#1}}
\newcommand{\tset}[1]{\set{#1}}
\newcommand{\tinact}{\tfont{end}}

%\newcommand{\sminus}[1]{#1^-}
\newcommand{\sminus}[1]{#1^{\text{--}}}

\newcommand{\subt}{\leq}
\newcommand{\supt}{\geq}

% Multiparty Local Session Types
\newcommand{\tout}[2]{\role{#1} \outtype \lrangle{#2} \tsep}
\newcommand{\tinp}[2]{\role{#1} \inptype (#2) \tsep}
\newcommand{\tsel}[2]{\role{#1} \seltype \set{#2}}
\newcommand{\tsels}[2]{\role{#1} \seltype #2}
\newcommand{\tselT}[1]{\role{#1} \seltype \lTi}
\newcommand{\tbra}[2]{\role{#1} \bratype \set{#2}}
\newcommand{\tbras}[2]{\role{#1} \bratype #2}
\newcommand{\tbraT}[1]{\role{#1} \bratype \lTi}

% Binary Session Types
\newcommand{\btout}[1]{\outtype \lrangle{#1} \tsep}
\newcommand{\bbtout}[1]{\outtype \big\langle{#1}\big\rangle \tsep}
\newcommand{\btinp}[1]{\inptype (#1) \tsep}
\newcommand{\bbtinp}[1]{\inptype \big({#1}\big) \tsep}
\newcommand{\btsel}[1]{\oplus \set{#1}}
\newcommand{\btselS}{\oplus \lSi}
\newcommand{\btbra}[1]{\& \set{#1}}
\newcommand{\btbraS}{\& \lSi}

% Queue Typing

\newcommand{\mtout}[2]{\role{#1} \outtype \lrangle{#2}}
\newcommand{\mtinp}[2]{\role{#1} \inptype (#2)}
\newcommand{\mtsel}[2]{\role{#1} \seltype #2}
\newcommand{\mtbra}[2]{\role{#1} \bratype #2}


% Binary Queue Typing

\newcommand{\bmtout}[1]{\outtype \lrangle{#1}}
\newcommand{\bmtinp}[1]{\inptype (#1)}
\newcommand{\bmtsel}[1]{\seltype #1}
\newcommand{\bmtbra}[1]{\bratype #1}

% Message concatanation
\newcommand{\mcat}{\;*\;}
\newcommand{\icat}{\;\circ\;}


%%%%%%%%%%%%%%%%%%%%%%%%%%%%%%%%%%%%%%%%%%%%%%%%%%%%%%%%%%%%%%%%%%%%%%%%%%%%%%%%%%%%%%%%%%%%%%%%%%%%
%                                        TYPED PROCESSES
%%%%%%%%%%%%%%%%%%%%%%%%%%%%%%%%%%%%%%%%%%%%%%%%%%%%%%%%%%%%%%%%%%%%%%%%%%%%%%%%%%%%%%%%%%%%%%%%%%%%

\newcommand{\Ga}{\Gamma}
\newcommand{\De}{\Delta}
\newcommand{\proves}{\vdash}
\newcommand{\hastype}{\triangleright}

\newcommand{\Decat}[1]{\De \cat #1}
\newcommand{\Gacat}[1]{\Ga \cat #1}

\newcommand{\tcat}{\circ}

\newcommand{\typed}[1]{#1:}
\newcommand{\typedrole}[2]{\typed{\srole{#1}{#2}}}
%\newcommand{\typedqrole}[2]{\typed{\srole{#1^{[]}}{#2}}}

\newcommand{\typedprocess}[3]{#1 \proves #2 \hastype #3}

\newcommand{\Eproves}[3]{#1 \proves \typed{#2} #3}
\newcommand{\Gproves}[2]{\Eproves{\Ga}{#1}{#2}}

\newcommand{\tprocess}[3]{#1 \proves #2 \hastype #3}
\newcommand{\Gtprocess}[2]{\tprocess{\Ga}{#1}{#2}}
\newcommand{\Gptprocess}[2]{\tprocess{\Ga'}{#1}{#2}}

\newcommand{\noGtprocess}[2]{#1 \hastype #2}

%%%%%%%%%%%%%%%%%%%%%%%%%%%%%%%%%%%%%%%%%%%%%%%%%%%%%%%%%%%%%%%%%%%%%%%%%%%%%%%%%%%%%%%%%%%%%%%%%%%%
%                                    MULTIPARTY TYPED THEORY
%%%%%%%%%%%%%%%%%%%%%%%%%%%%%%%%%%%%%%%%%%%%%%%%%%%%%%%%%%%%%%%%%%%%%%%%%%%%%%%%%%%%%%%%%%%%%%%%%%%%

\newcommand{\globalenv}[1]{\set{#1}}
%\newcommand{\globalenvI}{\set{\typed{s_i} \G_i}_{i \in I}}
\newcommand{\globalenvI}{E}
\newcommand{\globalenvJ}{\set{\typed{s_j} \G_j}_{j \in J}}

\newcommand{\Gltprocess}[4]{\tprocess{#1}{#2}{#3, #4}}

\newcommand{\geI}{\globalenvI}
\newcommand{\geJ}{\globalenvJ}

\newcommand{\Stprocess}[3]{\tprocess{\geI, #1}{#2}{#3}}
\newcommand{\SGtprocess}[2]{\Gtprocess{#1}{#2, \globalenvI}}
\newcommand{\SJGtprocess}[2]{\globalenvJ, \Gtprocess{#1}{#2}}

\newcommand{\Observer}[2]{\mathsf{Observer}(#1, #2)}
\newcommand{\ObserverG}[1]{\mathsf{Observer}(\globalenvI, #1)}

\newcommand{\Obs}{\ensuremath{\mathsf{Obs}}}

%%%%%%%%%%%%%%%%%%%%%%%%%%%%%%%%%%%%%%%%%%%%%%%%%%%%%%%%%%%%%%%%%%%%%%%%%%%%%%%%%%%%%%%%%%%%%%%%%%%%
%                                        BEHAVIOURAL THEORY
%%%%%%%%%%%%%%%%%%%%%%%%%%%%%%%%%%%%%%%%%%%%%%%%%%%%%%%%%%%%%%%%%%%%%%%%%%%%%%%%%%%%%%%%%%%%%%%%%%%%

\newcommand{\fromtolts}[2]{\fromto{#1}{#2}}

\newcommand{\outlts}{\outses}
\newcommand{\inplts}{\inpses}
\newcommand{\sellts}{\oplus}
\newcommand{\bralts}{\&}

% Multiparty Labels
\newcommand{\actreq}[3]{\send{#1} \role{#2} \boundv{#3}}
%\newcommand{\actbreq}[3]{\send{#1} \role{#2} \boundv{#3}}

\newcommand{\actreqs}[3]{\send{#1} \role{\set{#2}} \boundv{#3}}
%\newcommand{\actbreqs}[3]{\send{#1} \role{\set{#2}} \boundv{#3}}

\newcommand{\actacc}[3]{#1 \role{#2} \boundv{#3}}
\newcommand{\actaccs}[3]{#1 \role{\set{#2}} \boundv{#3}}

\newcommand{\actout}[4]{#1 \fromtolts{#2}{#3} \outlts \freev{#4}}
\newcommand{\actqout}[4]{#1^{[]} \fromtolts{#2}{#3} \outlts \freev{#4}}
\newcommand{\actbout}[4]{#1 \fromtolts{#2}{#3} \outlts \boundv{#4}}

\newcommand{\actdel}[5]{#1 \fromtolts{#2}{#3} \outlts \freev{\srole{#4}{#5}}}
\newcommand{\actbdel}[5]{#1 \fromtolts{#2}{#3} \outlts \boundv{\srole{#4}{#5}}}
\newcommand{\actqdel}[5]{#1^{[]} \fromtolts{#2}{#3} \outlts \boundv{\srole{#4}{#5}}}

\newcommand{\actinp}[4]{#1 \fromtolts{#2}{#3} \inplts \freev{#4}}
\newcommand{\actqinp}[4]{#1^{[]} \fromtolts{#2}{#3} \inplts \freev{#4}}

\newcommand{\actsel}[4]{#1 \fromtolts{#2}{#3} \sellts #4}
\newcommand{\actqsel}[4]{#1^{[]} \fromtolts{#2}{#3} \sellts #4}

\newcommand{\actbra}[4]{#1 \fromtolts{#2}{#3} \bralts #4}
\newcommand{\actqbra}[4]{#1^{[]} \fromtolts{#2}{#3} \bralts #4}

\newcommand{\actval}[4]{#1: #2 \rightarrow #3:#4}
\newcommand{\actgsel}[4]{#1: #2 \rightarrow #3:#4}

% Binary Labels
\newcommand{\bactreq}[2]{\send{#1} \freev{#2}}
\newcommand{\bactbreq}[2]{\send{#1} \boundv{#2}}
\newcommand{\bactacc}[2]{#1 \freev{#2}}

\newcommand{\bactout}[2]{#1 \outlts \freev{#2}}
\newcommand{\bactbout}[2]{#1\outlts \boundv{#2}}
\newcommand{\bactinp}[2]{#1 \inplts \freev{#2}}
\newcommand{\bactsel}[2]{#1 \sellts #2}
\newcommand{\bactbra}[2]{#1 \bralts #2}

% Labelled transition relations
\newcommand{\by}[1]{\stackrel{#1}{\longrightarrow}}
\newcommand{\By}[1]{\stackrel{#1}{\Longrightarrow}}

\newcommand{\hby}[1]{\stackrel{#1}{\longmapsto}}
\newcommand{\Hby}[1]{\stackrel{#1}{\Longmapsto}}


% Session barbs
\newcommand{\barbreq}[1]{\barb{#1}}
%\newcommand{\barbacc}[2]{\barb{#1\role{\set{#2}}}}
\newcommand{\barbout}[3]{\barb{\sfromto{#1}{#2}{#3}}}
%\newcommand{\barbinp}[3]{\barb{#1\fromto{#2}{#3}\inpses}}

\newcommand{\Barbreq}[2]{\Barb{\send{#1}\role{#2}}}
%\newcommand{\Barbacc}[2]{\Barb{#1\role{\set{#2}}}}
%\newcommand{\Barbout}[3]{\Barb{\sfromto{#1}{#2}{#3}\outses}}
\newcommand{\Barbout}[3]{\Barb{\sfromto{#1}{#2}{#3}}}
%\newcommand{\Barbinp}[3]{\Barb{#1\fromto{#2}{#3}\inpses}}

% Binary Session barbs
\newcommand{\bbarbreq}[1]{\barb{#1}}
%\newcommand{\bbarbacc}[1]{\barb{#1}}
\newcommand{\bbarbout}[1]{\barb{#1}}
%\newcommand{\bbarbinp}[1]{\barb{#1\inpses}}

\newcommand{\bBarbreq}[1]{\Barb{\send{#1}}}
%\newcommand{\bBarbacc}[2]{\Barb{#1}}
\newcommand{\bBarbout}[1]{\Barb{#1\outses}}
%\newcommand{\bBarbinp}[1]{\Barb{#1\inpses}}

\newcommand{\comp}{\asymp}
\newcommand{\coh}{\asymp}
\newcommand{\bistyp}{\rightleftharpoons}

\newcommand{\typingbeh}{\leftrightarrow}

\newcommand{\bufrel}{\succ}

\newcommand{\ordercup}{\bowtie}

%%%%%%%%%%%%%%%%%%%%%%%%%%%%%%%%%%%%%%%%%%%%%%%%%%%%%%%%%%%%%%%%%%%%%%%%%%%%%%%%%%%%%%%%%%%%%%%%%%%%
%                                TYPED TRANSITIONS - REDUCTIONS
%%%%%%%%%%%%%%%%%%%%%%%%%%%%%%%%%%%%%%%%%%%%%%%%%%%%%%%%%%%%%%%%%%%%%%%%%%%%%%%%%%%%%%%%%%%%%%%%%%%%

% Environment Transitions
\newcommand{\envtrans}[1]{\by{#1}}
\newcommand{\envTrans}[1]{\by{#1}}
\newcommand{\typedtrans}[1]{\by{#1}}
\newcommand{\typedTrans}[1]{\By{#1}}
\newcommand{\typedred}{\red}
\newcommand{\typedRed}{\Red}

% Typed Environment Transitions - Binary case
\newcommand{\benv}[2]{(#1, #2)}
\newcommand{\bGenv}[1]{\benv{\Ga}{#1}}
\newcommand{\bGDenv}{\envtyp{\Ga}{\De}}

\newcommand{\envby}[5]{\benv{#1}{#2} \envtrans{#3} \benv{#4}{#5}}
\newcommand{\Genvby}[3]{\envby{\Ga}{#1}{#2}{\Ga}{#3}}

% Typed Environment Transitions - Multiparty case

\newcommand{\env}[3]{(#1, #2, #3)}
\newcommand{\Genv}[2]{\env{\globalenvI}{#1}{#2}}
\newcommand{\GGenv}[1]{\env{\globalenvI}{\Ga}{#1}}
\newcommand{\GGDenv}{\env{\globalenvI}{\Ga}{\De}}

% Typed Process Transitions - Binary case

\newcommand{\ftby}[7]{\tprocess{#1}{#2}{#3} \typedtrans{#4} \tprocess{#5}{#6}{#7}}
\newcommand{\ftBy}[7]{\tprocess{#1}{#2}{#3} \typedTrans{#4} \tprocess{#5}{#6}{#7}}

\newcommand{\tpby}[6]{\tprocess{#1}{#2}{#3} \typedtrans{#4} \noGtprocess{#5}{#6}}
\newcommand{\tpBy}[6]{\tprocess{#1}{#2}{#3} \typedTrans{#4} \noGtprocess{#5}{#6}}
\newcommand{\Gtpby}[5]{\Gtprocess{#1}{#2} \typedtrans{#3} \noGtprocess{#4}{#5}}
\newcommand{\GtpBy}[5]{\Gtprocess{#1}{#2} \typedTrans{#3} \noGtprocess{#4}{#5}}

\newcommand{\GGtpby}[5]{\tprocess{\globalenvI, \Ga}{#1}{#2} \typedtrans{#3} \noGtprocess{#4}{#5}}
\newcommand{\GGtpBy}[5]{\tprocess{\globalenvI, \Ga}{#1}{#2} \typedTrans{#3} \noGtprocess{#4}{#5}}

% Typed Reductions - Binary case

\newcommand{\ftpred}[6]{\tprocess{#1}{#2}{#3} \typedred \tprocess{#4}{#5}{#6}}
\newcommand{\ftpRed}[6]{\tprocess{#1}{#2}{#3} \typedRed \tprocess{#4}{#5}{#6}}

\newcommand{\tpred}[5]{\tprocess{#1}{#2}{#3} \typedred \noGtprocess{#4}{#5}}
\newcommand{\tpRed}[5]{\tprocess{#1}{#2}{#3} \typedRed \noGtprocess{#4}{#5}}
\newcommand{\Gtpred}[4]{\Gtprocess{#1}{#2} \typedred \noGtprocess{#3}{#4}}
\newcommand{\GtpRed}[4]{\Gtprocess{#1}{#2} \typedRed \noGtprocess{#3}{#4}}

% Observer Reductions

\newcommand{\obsred}{\red_{obs}}
\newcommand{\obsRed}{\Red_{obs}}


%%%%%%%%%%%%%%%%%%%%%%%%%%%%%%%%%%%%%%%%%%%%%%%%%%%%%%%%%%%%%%%%%%%%%%%%%%%%%%%%%%%%%%%%%%%%%%%%%%%%
%                                    TYPED RELATIONS
%%%%%%%%%%%%%%%%%%%%%%%%%%%%%%%%%%%%%%%%%%%%%%%%%%%%%%%%%%%%%%%%%%%%%%%%%%%%%%%%%%%%%%%%%%%%%%%%%%%%

% Typed Relations
\newcommand{\fulltrel}[7]{\rel{\typedprocess{#1}{#2}{#3}}{#4}{\typedprocess{#5}{#6}{#7}}}
\newcommand{\treld}[6]{\rel{\typedprocess{#1}{#2}{#3}}{#4}{\noGtypedprocess{#5}{#6}}}
\newcommand{\trel}[5]{\rel{#1 \proves #2}{#3}{\noGtypedprocess{#4}{#5}}}

\newcommand{\tcong}{\cong}
\newcommand{\twb}{\approx}
\newcommand{\govwb}{\approx_g}
\newcommand{\tequiv}{\approx}

%%%%%%%%%%%%%%%%%%%%%%%%%%%%%%%%%%%%%%%%%%%%%%%%%%%%%%%%%%%%%%%%%%%%%%%%%%%%%%%%%%%%%%%%%%%%%%%%%%%%
%                                    CONFIGURATION THEORY
%%%%%%%%%%%%%%%%%%%%%%%%%%%%%%%%%%%%%%%%%%%%%%%%%%%%%%%%%%%%%%%%%%%%%%%%%%%%%%%%%%%%%%%%%%%%%%%%%%%%

\newcommand{\confpair}[2]{(#1, #2)}
\newcommand{\uptoconfpair}[2]{[#1, #2]}


%%%%%%%%%%%%%%%%%%%%%%%%%%%%%%%%%%%%%%%%%%%%%%%%%%%%%%%%%%%%%%%%%%%%%%%%%%%%%%%%%%%%%%%%%%%%%%%%%%%%
%                                   CONFLUENCE DETERMINACY
%%%%%%%%%%%%%%%%%%%%%%%%%%%%%%%%%%%%%%%%%%%%%%%%%%%%%%%%%%%%%%%%%%%%%%%%%%%%%%%%%%%%%%%%%%%%%%%%%%%%

\newcommand{\sesstrans}[1]{\stackrel{#1}{\longrightarrow_{s}}}
\newcommand{\sessTrans}[1]{\stackrel{#1}{\Longrightarrow_{s}}}

\newcommand{\fulltypedsesstrans}[7]{\typedprocess{#1}{#2}{#3} \sesstrans{#4} \typedprocess{#5}{#6}{#7}}
\newcommand{\fulltypedsessTrans}[7]{\typedprocess{#1}{#2}{#3} \sessTrans{#4} \typedprocess{#5}{#6}{#7}}

\newcommand{\typedsesstrans}[6]{\typedprocess{#1}{#2}{#3} \sesstrans{#4} \noGtypedprocess{#5}{#6}}
\newcommand{\typedsessTrans}[6]{\typedprocess{#1}{#2}{#3} \sessTrans{#4} \noGtypedprocess{#5}{#6}}
\newcommand{\Gtypedsesstrans}[5]{\Gtypedprocess{#1}{#2} \sesstrans{#3} \noGtypedprocess{#4}{#5}}
\newcommand{\GtypedsessTrans}[5]{\Gtypedprocess{#1}{#2} \sessTrans{#3} \noGtypedprocess{#4}{#5}}

% Actions
\newcommand{\confact}[2]{#1 \lfloor #2}

%%%%%%%%%%%%%%%%%%%%%%%%%%%%%%%%%%%%%%%%%%%%%%%%%%%%%%%%%%%%%%%%%%%%%%%%%%%%%%%%%%%%%%%%%%%%%%%%%%%%
%                                    MAPPING AND ENCODINGS
%%%%%%%%%%%%%%%%%%%%%%%%%%%%%%%%%%%%%%%%%%%%%%%%%%%%%%%%%%%%%%%%%%%%%%%%%%%%%%%%%%%%%%%%%%%%%%%%%%%%

\newcommand{\map}[1]{[\!\![#1]\!\!]}
\newcommand{\umap}[1]{[\!\![#1]\!\!]^u}
\newcommand{\pmap}[2]{\ensuremath{[\!\![#1]\!\!]^#2}}
\newcommand{\pmapp}[3]{\ensuremath{[\!\![#1]\!\!]^#2_#3}}
\newcommand{\auxmap}[2]{\ensuremath{\{\!\{#1\}\!\}^#2}}
\newcommand{\tauxmap}[2]{\ensuremath{\{\!|#1|\!\}^#2}}
\newcommand{\auxmapp}[3]{\ensuremath{\big\lfloor\!\!\big\lfloor#1\big\rfloor\!\!\big\rfloor^#2_#3}}
\newcommand{\tmap}[2]{\ensuremath{(\!\!\langle#1\rangle\!\!)^{#2}}}
\newcommand{\vtmap}[2]{{\ensuremath{\big\lfloor #1\big\rfloor^{#2}}}}
\newcommand{\mapt}[1]{\ensuremath{(\!\!\langle#1\rangle\!\!)}}
\newcommand{\mapa}[1]{\ensuremath{\{\!\!\{#1\}\!\!\}}}
\newcommand{\namemap}[2]{#1\map{#2}}

\newcommand{\enc}[2]{\big\langle\map{#1}, \mapt{#2}\big\rangle}
\newcommand{\enco}[1]{\big\langle #1\big\rangle}
\newcommand{\encod}[3]{\lrangle{\map{#1}^{#3}, \mapt{#2}^{#3}}}
\newcommand{\fencod}[4]{\lrangle{\map{#1}^{#3}_{#4} \, , \, \mapt{#2}^{#3}}}

\newcommand{\calc}[5]{\lrangle{#1, #2, #3, #4, #5}}
\newcommand{\tyl}[1]{\ensuremath{\mathcal{#1}}}

%%%%%%%%%%%%%%%%%%%%%%%%%%%%%%%%%%%%%%%%%%%%%%%%%%%%%%%%%%%%%%%%%%%%%%%%%%%%%%%%%%%%%%%%%%%%%%%%%%%%
%                                    PI CONSTRUCTS
%%%%%%%%%%%%%%%%%%%%%%%%%%%%%%%%%%%%%%%%%%%%%%%%%%%%%%%%%%%%%%%%%%%%%%%%%%%%%%%%%%%%%%%%%%%%%%%%%%%%

\newcommand{\constrtype}[1]{\mathtt{#1}}


\newcommand{\Let}{\constrtype{let}\ }
\newcommand{\In}{\constrtype{in}\ }
\newcommand{\To}{\constrtype{to}\ }
\newcommand{\new}{\constrtype{new}\ }
\newcommand{\from}{\constrtype{from}\ }
\newcommand{\select}{\constrtype{select}\ }
\newcommand{\register}{\constrtype{register}\ }
\newcommand{\Update}{\constrtype{update}\ }

\newcommand{\selectfrom}[2]{\select #1\ \from #2\ \In}
\newcommand{\registerto}[2]{\register #1\ \To #2\ \In}

\newcommand{\newselector}[1]{\new \constrtype{sel}\ #1\ \In}
\newcommand{\newselectorT}[2]{\new \constrtype{sel}\lrangle{#2}\ #1\ \In}
\newcommand{\selecttype}[1]{\dual{\constrtype{sel}}\lrangle{#1}}
\newcommand{\sselecttype}[1]{\constrtype{sel}\lrangle{#1}}

\newcommand{\update}[3]{\Update(#1, #2, #3)\ \In}

\newcommand{\newenv}[1]{\new \mathtt{env}\ #1\ \In\ }
\newcommand{\Letin}[2]{\Let #1 = #2\ \In}

\newcommand{\selqueue}[2]{#1\lrangle{#2}}

%%%%%%%%%%%%%%%%%%%%%%%%%%%%%%%%%%%%%%%%%%%%%%%%%%%%%%%%%%%%%%%%%%%%%%%%%%%%%%%%%%%%%%%%%%%%%%%%%%%%
%                                    DUALITY
%%%%%%%%%%%%%%%%%%%%%%%%%%%%%%%%%%%%%%%%%%%%%%%%%%%%%%%%%%%%%%%%%%%%%%%%%%%%%%%%%%%%%%%%%%%%%%%%%%%%
\newcommand{\dualof}{\ \mathsf{dual}\ }


%%%%%%%%%%%%%%%%%%%%%%%%%%%%%%%%%%%%%%%%%%%%%%%%%%%%%%%%%%%%%%%%%%%%%%%%%%%%%%%%%%%%%%%%%%%%%%%%%%%%
%                                        lambda - CALCULUS
%%%%%%%%%%%%%%%%%%%%%%%%%%%%%%%%%%%%%%%%%%%%%%%%%%%%%%%%%%%%%%%%%%%%%%%%%%%%%%%%%%%%%%%%%%%%%%%%%%%%

\newcommand{\labs}[2]{\lambda #1. #2}

%%%%%%%%%%%%%%%%%%%%%%%%%%%%%%%%%%%%%%%%%%%%%%%%%%%%%%%%%%%%%%%%%%%%%%%%%%%%%%%%%%%%%%%%%%%%%%%%%%%%
%                                    HIGHER ORDER SESSION PI
%%%%%%%%%%%%%%%%%%%%%%%%%%%%%%%%%%%%%%%%%%%%%%%%%%%%%%%%%%%%%%%%%%%%%%%%%%%%%%%%%%%%%%%%%%%%%%%%%%%%
%\newcommand{\pHOp}{\ensuremath{\mathsf{HO}\pi_{\mathsf{p}}}\xspace}
%\newcommand{\pHOpnr}{\ensuremath{\mathsf{HO}\pi^{-\mu}_{\mathsf{p}}}\xspace}
\newcommand{\HOp}{\ensuremath{\mathsf{HO}\pi}\xspace}
%\newcommand{\sessp}{\ensuremath{\mathtt{SE}\pi}\xspace}
\newcommand{\sessp}{\ensuremath{\pi}\xspace}
\newcommand{\haskp}{\ensuremath{\pi^{\lambda}}\xspace}
\newcommand{\pHOp}{\ensuremath{\mathsf{HO}\tilde{\pi}}\xspace}
%\newcommand{\psesp}{\ensuremath{\mathtt{sess}\pi_{\mathsf{p}}}\xspace}
%\newcommand{\psespnr}{\ensuremath{\mathtt{sess}\pi^{-\mu}_{\mathsf{p}}}\xspace}
%\newcommand{\sespnr}{\ensuremath{\mathtt{sess}\pi^{-\mu}}\xspace}
\newcommand{\HO}{\ensuremath{\mathsf{HO}}\xspace}
\newcommand{\HOpp}{\ensuremath{\mathsf{HO\pi^{+}}}\xspace}
\newcommand{\PHOp}{\ensuremath{\mathsf{HO}\,{\widetilde{\pi}}}\xspace}
\newcommand{\PHOpp}{\ensuremath{\mathsf{HO}\,{\widetilde{\pi}}^{\,+}}\xspace}
\newcommand{\PHO}{\ensuremath{\vec{\mathsf{HO}}}\xspace}
\newcommand{\Psessp}{\ensuremath{\vec{\pi}}\xspace}


\newcommand{\CAL}{\ensuremath{\mathsf{C}}\xspace}

\newcommand{\pol}{\mathsf{p}}


\newcommand{\ST}{\mathsf{ST}}


%\newcommand{\pHO}{\mathsf{pure\ HO}}

%\newcommand{\HOp}{\HO^+}
%\newcommand{\pHOp}{\pHO^+}
%\newcommand{\ppi}{\mathsf{pure\ session\ }\pi}
%\newcommand{\spi}{\mathsf{session\ }\pi}

\newcommand{\Proc}{\ensuremath{\diamond}}


%\newcommand{\appl}[2]{#1\lrangle{#2}}
\newcommand{\appl}[2]{#1\, {#2}}
%\newcommand{\abs}[2]{(#1)#2}
\newcommand{\abs}[2]{\lambda #1.\,#2}

\newcommand{\lollipop}{\multimap}
\newcommand{\sharedop}{\rightarrow}
\newcommand{\logicop}{\multimapdot}

\newcommand{\lhot}[1]{#1\!\! \lollipop\!\! \diamond}
\newcommand{\shot}[1]{#1\!\! \sharedop\!\! \diamond}
\newcommand{\hot}[1]{#1 \logicop \diamond}

%\newcommand{\absmap}[1]{}
\newcommand{\vmap}[1]{|\!|#1|\!|}
\newcommand{\smap}[1]{(\!|\!|#1|\!|\!)^s}
\newcommand{\svmap}[1]{(\!|\!|#1|\!|\!)^{s\rightarrow v}}
\newcommand{\amap}[1]{\mathcal{A}\map{#1}}
\newcommand{\absmap}[2]{\mathcal{A}\map{#1}^{#2}}



%%%%% triggers

\newcommand{\hotrigger}[2]{\binp{#1}{x} \newsp{s}{\appl{x}{s} \Par \bout{\dual{s}}{#2} \inact}}
\newcommand{\fotrigger}[5]{\binp{#1}{#2} \newsp{#3}{\map{#4}^{#3} \Par \bout{\dual{#3}}{#5} \inact}}
%\newcommand{\fotrigger}[2]{\binp{#1}{X} \appl{X}{#2}}

%%%%%% Typed relations

\newcommand{\horel}[6]{#1; #2 \proves #3 #4 #5 \proves #6}
%\newcommand{\horel}[6]{#1; \es; #2 \proves #3 #4 #5 \proves #6}

\newcommand{\mhorel}[7]{
	\begin{array}{rcll}
		#1; \es; #2 &#4& #5 \proves& #3\\
			&#4& #6 & #7
	\end{array}
}

%%%%%%%%%%%%%%%%%%%%%%%%%%%%%%%%%%%%%%%%%%%%%%%%%%%%%%%%%%%%%%%%%%%%%%%%%%%%%%%%%%%%%%%%%%%%%%%%%%%%
%                                    LN TRANSFORM
%%%%%%%%%%%%%%%%%%%%%%%%%%%%%%%%%%%%%%%%%%%%%%%%%%%%%%%%%%%%%%%%%%%%%%%%%%%%%%%%%%%%%%%%%%%%%%%%%%%%

\newcommand{\Loop}{\mathsf{Loop}}
\newcommand{\CodeBlocks}{\mathsf{CodeBlocks}}

\newcommand{\lnmap}[1]{\namemap{LN}{#1}}
\newcommand{\lnrmap}[1]{\namemap{LNR}{#1}}
\newcommand{\lnblockmap}[1]{\namemap{\mathcal{B}}{#1}}
\newcommand{\lnnonblockmap}[2]{\map{#1, #2}}

\newcommand{\mapenv}[2]{\map{#1}_{#2}}

%%%%%%%%%%%%%%%%%%%%%%%%%%%%%%%%%%%%%%%%%%%%%%%%%%%%%%%%%%%%%%%%%%%%%%%%%%%%%%%%%%%%%%%%%%%%%%%%%%%%
%                                        GENERAL TYPES
%%%%%%%%%%%%%%%%%%%%%%%%%%%%%%%%%%%%%%%%%%%%%%%%%%%%%%%%%%%%%%%%%%%%%%%%%%%%%%%%%%%%%%%%%%%%%%%%%%%%

% Values
\newcommand{\true}{\sessionfont{tt}}
\newcommand{\false}{\sessionfont{ff}}

% Typed
\newcommand{\bool}{\sessionfont{bool}}
\newcommand{\nat}{\sessionfont{nat}}



%%%%%%%%%%%%%%%%%%%%%%%%%%%%%%%%%%%%%%%%%%%%%%%%%%%%%%%%%%%%%%%%%%%%%%%%%%%%%%%%%%%%%%%%%%%%%%%%%%%%
%                                        PROCESSES NAMES SESSIONS ETC
%%%%%%%%%%%%%%%%%%%%%%%%%%%%%%%%%%%%%%%%%%%%%%%%%%%%%%%%%%%%%%%%%%%%%%%%%%%%%%%%%%%%%%%%%%%%%%%%%%%%

% Processes
\newcommand{\PP}{\ensuremath{P}}
\newcommand{\Q}{\ensuremath{Q}}
\newcommand{\R}{\ensuremath{R}}
\newcommand{\OP}{\ensuremath{\mathsf{O}}}

% Global environments
\newcommand{\En}{\ensuremath{En}}

% Session channels
\newcommand{\s}{\ensuremath{s}}
\newcommand{\ds}{\ensuremath{\dual{s}}}
%\newcommand{\Ms}[2]{\ensuremath{s}\role{#1}\role{#2}}

%Dummy channels
\newcommand{\sd}{\mathtt{sd}}
\newcommand{\shd}{\mathtt{shd}}

% Names
\newcommand{\Ia}{\ensuremath{a}}
\newcommand{\Iu}{\ensuremath{u}}

% Variables, values, expressions
\newcommand{\x}{\ensuremath{x}}
\newcommand{\y}{\ensuremath{y}}
\newcommand{\ks}{\ensuremath{k}}
\newcommand{\cc}{\ensuremath{c}}
\newcommand{\va}{\ensuremath{v}}
\newcommand{\e}{\ensuremath{e}}
\newcommand{\n}{\ensuremath{n}}

% Process Variables

\newcommand{\X}{\varp{X}}
\newcommand{\Y}{\varp{Y}}

% Roles
\newcommand{\p}{\ensuremath{\mathtt{p}}}
\newcommand{\q}{\ensuremath{\mathtt{q}}}
\newcommand{\A}{\ensuremath{A}}

% Types
\newcommand{\G}{\ensuremath{G}}
\newcommand{\gG}{\globaltype{\G}}
\newcommand{\U}{\ensuremath{U}}
\newcommand{\So}{\ensuremath{S}}
\newcommand{\T}{\ensuremath{T}}

% Queue Types
\newcommand{\M}{\ensuremath{M}}
\newcommand{\I}{\ensuremath{M_\inputsym}}
\newcommand{\Om}{\ensuremath{M_\outputsym}}
\newcommand{\Typ}{\ensuremath{\mathsf{T}}}

% Queues values
\newcommand{\h}{\ensuremath{h}}

%barbs
\newcommand{\m}{\ensuremath{\mu}}

% Contexts
\newcommand{\C}{\ensuremath{{\Bbb C}}}
\newcommand{\E}{\ensuremath{E}}

% Congruence completness - Definibility
\newcommand{\TT}{\ensuremath{T}}
\newcommand{\suc}{\textrm{succ}}
\newcommand{\fail}{\textrm{fail}}


% Set selection labels
\newcommand{\lPi}{\set{l_i:\PP_i}_{i \in I}}
\newcommand{\lGi}{\set{l_i:\G_i}_{i \in I}}
\newcommand{\lTi}{\set{l_i:\T_i}_{i \in I}}
\newcommand{\lSi}{\set{l_i:\So_i}_{i \in I}}

% Selector proof

\newcommand{\SEL}{P_\mathit{Sel}}
\newcommand{\DSEL}{P_\mathit{DSel}}
\newcommand{\Sel}{\mathsf{Sel}}
\newcommand{\IfSel}{\mathsf{IfSel}}
\newcommand{\DSel}{\mathsf{DSel}}
\newcommand{\PSel}{\mathsf{PermSel}}
\newcommand{\PIfSel}{\mathsf{PermIfSel}}
\newcommand{\PDSel}{\mathsf{PermDSel}}

%%%%%%%%%%%%%%%%%%%%%%%%%%%%%%%%%%%%%%%%%%%%%%%%%%%%%%%%%%%%%%%%%%%%%%%%%%%%%%%%%%%%%%%%%%%%%%%%%%%%
%                                        ENVIRONMENTS
%%%%%%%%%%%%%%%%%%%%%%%%%%%%%%%%%%%%%%%%%%%%%%%%%%%%%%%%%%%%%%%%%%%%%%%%%%%%%%%%%%%%%%%%%%%%%%%%%%%%
\newtheorem{fact}{Fact}[section]
\newtheorem{notation}{Notation}

%\newenvironment{notation}{\paragraph{{\bf Notation}}}{}

%\newtheorem{proposition}[fact]{{\bf\em Proposition}}
%\newtheorem{example}[fact]{{\bf\em Example}}
%\newtheorem{lemma}[fact]{{\bf\em Lemma}}
%\newtheorem{corollary}[fact]{{\bf\em Corollary}}
%\newtheorem{definition}[fact]{{\bf\em Definition}}
%\newtheorem{theorem}[fact]{{\bf\em Theorem}}
%\newtheorem{remark}[fact]{{\bf\em Remark}}

\newcommand{\nonhosyntax}[1]{\colorbox{lightgray}{\ensuremath{#1}}}

\newenvironment{mytheorem}{%\vspace{-3pt}
	\begin{theorem}
}{%\vspace{-4pt}
	\end{theorem}
}

\newenvironment{myproposition}{
	\begin{proposition}%\vspace{-3pt}
}{%\vspace{-4pt}
	\end{proposition}
}

\newenvironment{mycorollary}{
	\begin{corollary}%\vspace{-3pt}
}{%\vspace{-4pt}
	\end{corollary}
}

\newenvironment{mylemma}{
	\begin{lemma}%\vspace{-3pt}
}{%\vspace{-4pt}
	\end{lemma}
}


\newenvironment{mydefinition}{%\vspace{-3pt}
	\begin{definition}
}{%\vspace{-3pt}
	\end{definition}
}

%\newenvironment{proof}{
%	{\em Proof.}
%}{}


%\newcommand{\qed}{\ensuremath{\square}}





%%%%%%%%%%%%%%%%%%%%%%%%%%%%%%%%%%%%%%%%%%%%%%%%%%%%%%%%%%%%%%%%%%%%%%%%%%%%%%%%%%%%%%%%%%%%%%%%%%%%
%                                        MISC
%%%%%%%%%%%%%%%%%%%%%%%%%%%%%%%%%%%%%%%%%%%%%%%%%%%%%%%%%%%%%%%%%%%%%%%%%%%%%%%%%%%%%%%%%%%%%%%%%%%%
\newcommand{\Appendix}[1]{Appendix \ref{#1}}

\newcommand{\dimcom}[1]{{\bf Comment: #1 \\}}

\newcommand{\hintcom}[1]{{\bf Hint: #1 \\}}

\newif\ifny\nyfalse
%\nytrue
\newcommand{\NY}[1]
{\ifny{\color{purple}{#1}}\else{#1}\fi}

\newcommand{\KH}[1]
{\ifny{\color{brown}{#1}}\else{#1}\fi}

\newif\ifdm\dmtrue
%\dmfalse
\newcommand{\dk}[1]
{\ifdm{\color{blue}{#1}}\else{#1}\fi}

\newif\ifrhu\rhutrue
%\rhufalse
\newcommand{\rh}[1]
{\ifdm{\color{red}{#1}}\else{#1}\fi}

\newif\ifjp\jptrue
%\jpfalse
\newcommand{\jp}[1]
{\ifjp{\color{red}{#1}}\else{#1}\fi}

\newif\ifjp\jptrue
%\jpfalse
\newcommand{\jpc}[1]
{\ifjp{\color{red}{#1}}\else{#1}\fi}

\newcommand{\ENCan}[1]{\langle #1 \rangle}
\newcommand{\NI}{\noindent}


\newcommand{\syntaxvspace}{\\[1mm]}

\newcommand{\TO}[2]{#1\to #2}
\newcommand{\GS}[3]{\TO{#1}{#2}\colon \!\ENCan{#3}}

\newcommand{\ASET}[1]{\{#1\}}
\newcommand{\participant}[1]{\mathtt{#1}}
\newcommand{\CODE}[1]{{\tt #1}}

\newcommand{\AT}[2]{#1 \! : \! #2}


\newcommand{\myrm}{}


\newcommand{\secref}[1]{\S\,\ref{#1}}
\newcommand{\defref}[1]{Def.~\ref{#1}}
\newcommand{\notref}[1]{Not.~\ref{#1}}
\newcommand{\defsref}[1]{Defs.~\ref{#1}}
\newcommand{\figref}[1]{Fig.~\ref{#1}}
\newcommand{\thmref}[1]{Thm.~\ref{#1}}
\newcommand{\thmsref}[1]{Thms.~\ref{#1}}
\newcommand{\exref}[1]{Ex.~\ref{#1}}
\newcommand{\propref}[1]{Prop.~\ref{#1}}
\newcommand{\propsref}[1]{Props.~\ref{#1}}
\newcommand{\appref}[1]{App.~\ref{#1}}
\newcommand{\lemref}[1]{Lem.~\ref{#1}}



\newcommand{\stytra}[6]{\ensuremath{#1; #3 \proves #4 \hby{#2} #5 \proves #6 }}
\newcommand{\stytraarg}[7]{\ensuremath{#1; #3 \proves_{#7} #4 \hby{#2} #5 \proves_{#7} #6 }}
\newcommand{\stytraargi}[8]{\ensuremath{#1; #3 \proves_{#7} #4 \hby{#2}_{#8} #5 \proves_{#7} #6 }}
\newcommand{\wtytra}[6]{\ensuremath{#1; #3 \proves #4 \Hby{#2}  #5 \proves #6}}
\newcommand{\wtytraarg}[7]{\ensuremath{#1; #3 \proves_{#7} #4 \Hby{#2}  #5 \proves_{#7} #6 }}
\newcommand{\wtytraargi}[8]{\ensuremath{#1; #3 \proves_{#7} #4 \Hby{#2}_{#8}  #5 \proves_{#7} #6 }}
\newcommand{\wbb}[6]{\ensuremath{#1; #3 \proves #4 \wb #5 \proves #6 }}
\newcommand{\wbbarg}[7]{\ensuremath{#1; #3 \proves_{#7} #4 \wb_{#7} #5 \proves_{#7} #6 }}

\newcommand{\minussh}{\ensuremath{\mathsf{-sh}}\xspace}

\definecolor{lightgray}{gray}{0.75}

\newcommand\greybox[1]{%
  \vskip\baselineskip%
  \par\noindent\colorbox{lightgray}{%
    \begin{minipage}{\textwidth}#1\end{minipage}%
  }%
  \vskip\baselineskip%
}

%\newcommand{\myparagraph}[1]{\paragraph{\bf #1}}

\newcommand{\mapchar}[2]{\ensuremath{[\!\!(#1)\!\!]^{#2}}}
\newcommand{\omapchar}[1]{\ensuremath{[\!\!(#1)\!\!]_{\mathsf{c}}}}

\newcommand{\trigger}[3]{#1 \leftarrow\!\!\!\!\!\!\!\leftarrow #2:#3 }
\newcommand{\htrigger}[2]{#1 \Leftarrow #2}
\newcommand{\ftrigger}[3]{#1 \Leftarrow \AT{#2}{#3}}

\newcommand{\btau}{\tau_{\beta}}
\newcommand{\stau}{\tau_{s}}
\newcommand{\dtau}{\tau_{d}}



%\newcommand{\HOpp}{\ensuremath{\mathsf{HO\pi^{+}}}\xspace}
%\newcommand{\PHOp}{\ensuremath{\mathsf{HO}{\vec{\pi}}}\xspace}
%\newcommand{\PHOpp}{\ensuremath{\mathsf{HO}{\vec{\pi}}^{+}}\xspace}



\newcommand{\prcolor}[1]{{\color{blue} #1}}

%%%%%%%%%%%%%%%%%%% Title Page Info%%%%%%%%%%%%%%%%%%%%%%%%%%

\title{Characteristic Bisimulations for Higher-Order Session Types}

\author{{\bf Dimitrios Kouzapas$^{1,3}$}, Jorge A. P\'{e}rez$^{2}$, and Nobuko Yoshida$^1$}

\institute{Imperial College London$^1$, University of Groningen$^2$, University of Glasgow$^3$}

\date

\begin{document}
	\begin{frame}
		\titlepage

		{ \tiny %This work has been partially 
		Sponsored by the The Doctoral Prize Fellowship, EPSRC EP/K011715/1,
		EPSRC EP/K034413/1, and EPSRC EP/L00058X/1,
		EU project FP7-612985 UpScale, and EU COST Action IC1201 BETTY.  
		P\'{e}rez is  also affiliated to
		%the NOVA Laboratory for Computer Science and Informatics (NOVA LINCS),  
		Universidade Nova de Lisboa.%, Portugal
		}
	\end{frame}

	\begin{frame}{Motivation}
		\begin{itemize}
			\item	Higher-order session calculus

				\begin{itemize}
					\item	Values in messages may be processes
					\item	Higher-order session types
					\item	Bridge between process calculi and the $\lambda$ calculus
				\end{itemize}

%			\item	Equivalence theory for session typed programs

			\item	Equivalence theory for higher-order sessions is
				\begin{itemize}
					\item	Interesting - real applications, relation between the $\lambda$ calculus and process calculi
					\item	Challenging - difficult to compute
				\end{itemize}
%				interesting and challenging

			\item	This paper offers the first {\em tractable} theory based on labelled bisimilarities
				informed by session types % The solution is:
				\begin{itemize}
					\item	Natural - follows the behaviour of session types
					\item	Economical - exploits the linearity of session types
					\item	Original - with respect to bibliography
				\end{itemize}
		\end{itemize}
	\end{frame}

	\begin{frame}{Motivation: Example}
		
\newcommand{\Hotel}{\mathsf{Hotel}}
\newcommand{\Code}{\mathsf{Code}}

%\makeatletter
%\newcommand{\gettikzxy}[3]{%
%  \tikz@scan@one@point\pgfutil@firstofone#1\relax
%  \edef#2{\the\pgf@x}%
%  \edef#3{\the\pgf@y}%
%}
%\makeatother
%	\gettikzxy{(Client1.south)}{\ax}{\ay}
%	\draw[dashed]			(Client1.south) -- (\ax, 0);


%\begin{center}

\begin{tikzpicture}
%	\draw[help lines]		(0, 0) grid (13, 10);

	%%%%%%%%%%%%%%%%%%%%% Scenario 1

	%%%% Nodes
	\node	(Client1)	at	(0, 10) {\footnotesize $\Client_1$};
	\node	(Hotel1)	at	(2.5, 10) {\footnotesize $\Hotel_1$};
	\node	(Hotel2)	at	(5, 10) {\footnotesize $\Hotel_2$};

	\node	(Code1)		at	(1.25, 8.8) {\footnotesize $\Code_1$};
	\node	(Code2)		at	(3.75, 8.8) {\footnotesize $\Code_2$};

	%%%% Lines for Nodes
%	\draw[dashed]		(Client1.south west) -- (Client1.south east);
	\draw
		let	\p1 = (Client1.south)
		in
			(\x1, \y1) -- (\x1, 2);

%	\draw[dashed]		(Hotel1.south west) -- (Hotel1.south east);
	\draw
		let
			\p1 = (Hotel1.south)
		in
			(\x1, \y1) -- (\x1, 2);


%	\draw[dashed]		(Hotel2.south west) -- (Hotel2.south east);
	\draw
		let
			\p1 = (Hotel2.south)
		in
			(\x1, \y1) -- (\x1, 2);


%	\draw[dashed]		(Code1.south west) -- (Code1.south east);
	\draw
		let
			\p1 = (Code1.south)
		in
			(\x1, \y1) -- (\x1, 2);

%	\draw[dashed]		(Code2.south west) -- (Code2.south east);
	\draw
		let
			\p1 = (Code2.south)
		in
			(\x1, \y1) -- (\x1, 2);


	%%%% Arrows
	\draw[->]
		let
			\p1 = (Client1),
			\p2 = (Hotel1)
		in
			(\x1, 9.6) to node[above] {\scriptsize $\mathsf{Code1}$} (\x2, 9.6);

	\draw[->]
		let
			\p1 = (Client1),
			\p2 = (Hotel2)
		in
			(\x1, 9.2) to node[above] {\scriptsize $\mathsf{Code1}$} (\x2, 9.2);

	\draw[->]
		let
			\p1 = (Hotel1)
		in
			(\x1, 9.6) -- (Code1.north);

	\draw[->]
		let
			\p1 = (Hotel2)
		in
			(\x1, 9.2) -- (Code2.north);


	\draw[->]
		let
			\p1 = (Code1),
			\p2 = (Hotel1)
		in
			(\x1, 8.4) to node[above] {\scriptsize $\rtype$} (\x2, 8.4);

	\draw[->]
		let
			\p1 = (Code1),
			\p2 = (Hotel1)
		in
			(\x2, 8) to node[above] {\scriptsize $\Quote$} (\x1, 8);

	\draw[->]
		let
			\p1 = (Code2),
			\p2 = (Hotel2)
		in
			(\x1, 8.4) to node[above] {\scriptsize $\rtype$} (\x2, 8.4);
	\draw[->]
		let
			\p1 = (Code2),
			\p2 = (Hotel2)
		in
			(\x2, 8) to node[above] {\scriptsize $\Quote$} (\x1, 8);

	\draw[->]
		let
			\p1 = (Code1),
			\p2 = (Client1)
		in
			(\x1, 7.6) to node[above] {\scriptsize $\Quote$} (\x2, 7.6);

	\draw[->]
		let
			\p1 = (Code2),
			\p2 = (Client1)
		in
			(\x1, 7.2) to node[above] {\scriptsize $\Quote$} (\x2, 7.2);


	%%%% Choice 
	%% Client1 --> Code1
	\draw[dashed]
		let
			\p1 = (Client1),
			\p2 = (Code1)
		in
			(\x1, 6.8) -- (\x2, 6.8)
			(\x1, 4.4) -- (\x2, 4.4)
			(\x1, 2.4) -- (\x2, 2.4);

	\draw[->]
		let
			\p1 = (Client1),
			\p2 = (Code1)
		in
			(\x1, 6.4) to node[above] {\scriptsize $\accept$} (\x2, 6.4);

	\draw[dashed]
		let
			\p1 = (Code1),
			\p2 = (Hotel1)
		in
			(\x1, 6) -- (\x2, 6)
			(\x1, 4.8) -- (\x2, 4.8)
			(\x1, 3.6) -- (\x2, 3.6)
			(\x1, 2.8) -- (\x2, 2.8);

	\draw[->]
		let
			\p1 = (Code1),
			\p2 = (Hotel1)
		in
			(\x1, 5.6) to node[above] {\scriptsize $\accept$} (\x2, 5.6);

	\draw[->]
		let
			\p1 = (Code1),
			\p2 = (Hotel1)
		in
			(\x1, 5.2) to node[above] {\scriptsize $\creditc$} (\x2, 5.2);

	\draw[->]
		let
			\p1 = (Client1),
			\p2 = (Code1)
		in
			(\x1, 4) to node[above] {\scriptsize $\reject$} (\x2, 4);

	\draw[->]
		let
			\p1 = (Code1),
			\p2 = (Hotel1)
		in
			(\x1, 3.2) to node[above] {\scriptsize $\reject$} (\x2, 3.2);


	%%%% Choice 
	%% Client1 --> Code1
	\draw[dashed]
		let
			\p1 = (Client1),
			\p2 = (Code2)
		in
			(\x1, 6.8) -- (\x2, 6.8)
			(\x1, 4.4) -- (\x2, 4.4)
			(\x1, 2.4) -- (\x2, 2.4);

	\draw[->]
		let
			\p1 = (Client1),
			\p2 = (Code2)
		in
			(\x1, 6.2) to node[above] {\scriptsize $\accept$} (\x2, 6.2);

	\draw[dashed]
		let
			\p1 = (Code2),
			\p2 = (Hotel2)
		in
			(\x1, 6) -- (\x2, 6)
			(\x1, 4.8) -- (\x2, 4.8)
			(\x1, 3.6) -- (\x2, 3.6)
			(\x1, 2.8) -- (\x2, 2.8);

	\draw[->]
		let
			\p1 = (Code2),
			\p2 = (Hotel2)
		in
			(\x1, 5.6) to node[above] {\scriptsize $\accept$} (\x2, 5.6);

	\draw[->]
		let
			\p1 = (Code2),
			\p2 = (Hotel2)
		in
			(\x1, 5.2) to node[above] {\scriptsize $\creditc$} (\x2, 5.2);

	\draw[->]
		let
			\p1 = (Client1),
			\p2 = (Code2)
		in
			(\x1, 3.8) to node[above] {\scriptsize $\reject$} (\x2, 3.8);

	\draw[->]
		let
			\p1 = (Code2),
			\p2 = (Hotel2)
		in
			(\x1, 3.2) to node[above] {\scriptsize $\reject$} (\x2, 3.2);


	%%%%%%%%%%%%%%%%%%%%% Scenario 2

	%%%% Nodes
	\node	(Client1)	at	(7.5, 10) {\footnotesize $\Client_1$};
	\node	(Hotel1)	at	(10, 10) {\footnotesize $\Hotel_1$};
	\node	(Hotel2)	at	(12.5, 10) {\footnotesize $\Hotel_2$};

	\node	(Code1)		at	(8.75, 8.8) {\footnotesize $\Code_1$};
	\node	(Code2)		at	(11.25, 8.8) {\footnotesize $\Code_2$};

	%%%% Lines for Nodes
%	\draw[dashed]		(Client1.south west) -- (Client1.south east);
	\draw
		let	\p1 = (Client1.south)
		in
			(\x1, \y1) -- (\x1, 4);

%	\draw[dashed]		(Hotel1.south west) -- (Hotel1.south east);
	\draw
		let
			\p1 = (Hotel1.south)
		in
			(\x1, \y1) -- (\x1, 4);


%	\draw[dashed]		(Hotel2.south west) -- (Hotel2.south east);
	\draw
		let
			\p1 = (Hotel2.south)
		in
			(\x1, \y1) -- (\x1, 4);


%	\draw[dashed]		(Code1.south west) -- (Code1.south east);
	\draw
		let
			\p1 = (Code1.south)
		in
			(\x1, \y1) -- (\x1, 4);

%	\draw[dashed]		(Code2.south west) -- (Code2.south east);
	\draw
		let
			\p1 = (Code2.south)
		in
			(\x1, \y1) -- (\x1, 4);


	%%%% Arrows
	\draw[->]
		let
			\p1 = (Client1),
			\p2 = (Hotel1)
		in
			(\x1, 9.6) to node[above] {\scriptsize $\mathsf{Code1}$} (\x2, 9.6);

	\draw[->]
		let
			\p1 = (Client1),
			\p2 = (Hotel2)
		in
			(\x1, 9.2) to node[above] {\scriptsize $\mathsf{Code1}$} (\x2, 9.2);

	\draw[->]
		let
			\p1 = (Hotel1)
		in
			(\x1, 9.6) -- (Code1.north);

	\draw[->]
		let
			\p1 = (Hotel2)
		in
			(\x1, 9.2) -- (Code2.north);


	\draw[->]
		let
			\p1 = (Code1),
			\p2 = (Hotel1)
		in
			(\x1, 8.4) to node[above] {\scriptsize $\rtype$} (\x2, 8.4);

	\draw[->]
		let
			\p1 = (Code1),
			\p2 = (Hotel1)
		in
			(\x2, 8) to node[above] {\scriptsize $\Quote$} (\x1, 8);

	\draw[->]
		let
			\p1 = (Code2),
			\p2 = (Hotel2)
		in
			(\x1, 8.4) to node[above] {\scriptsize $\rtype$} (\x2, 8.4);
	\draw[->]
		let
			\p1 = (Code2),
			\p2 = (Hotel2)
		in
			(\x2, 8) to node[above] {\scriptsize $\Quote$} (\x1, 8);

	\draw[->]
		let
			\p1 = (Code1),
			\p2 = (Code2)
		in
			(\x1, 7.6) to node[above] {\scriptsize $\Quote$} (\x2, 7.6);

	\draw[->]
		let
			\p1 = (Code2),
			\p2 = (Code1)
		in
			(\x1, 7.2) to node[above] {\scriptsize $\Quote$} (\x2, 7.2);


	%%%% Choice
	% Client1 --> Hotel1
	\draw[dashed]
		let
			\p1 = (Code1),
			\p2 = (Hotel1)
		in
			(\x1, 6.8) -- (\x2, 6.8)
			(\x1, 5.6) -- (\x2, 5.6)
			(\x1, 4.8) -- (\x2, 4.8);

	\draw[->]
		let
			\p1 = (Code1),
			\p2 = (Hotel1)
		in
			(\x1, 6.4) to node[above] {\scriptsize $\accept$} (\x2, 6.4);

	\draw[->]
		let
			\p1 = (Code1),
			\p2 = (Hotel1)
		in
			(\x1, 6) to node[above] {\scriptsize $\creditc$} (\x2, 6);

	\draw[->]
		let
			\p1 = (Code1),
			\p2 = (Hotel1)
		in
			(\x1, 5.2) to node[above] {\scriptsize $\reject$} (\x2, 5.2);

	% Client2 --> Hotel2
	\draw[dashed]
		let
			\p1 = (Code2),
			\p2 = (Hotel2)
		in
			(\x1, 6.8) -- (\x2, 6.8)
			(\x1, 5.6) -- (\x2, 5.6)
			(\x1, 4.8) -- (\x2, 4.8);

	\draw[->]
		let
			\p1 = (Code2),
			\p2 = (Hotel2)
		in
			(\x1, 6.4) to node[above] {\scriptsize $\accept$} (\x2, 6.4);

	\draw[->]
		let
			\p1 = (Code2),
			\p2 = (Hotel2)
		in
			(\x1, 6) to node[above] {\scriptsize $\creditc$} (\x2, 6);

	\draw[->]
		let
			\p1 = (Code2),
			\p2 = (Hotel2)
		in
			(\x1, 5.2) to node[above] {\scriptsize $\reject$} (\x2, 5.2);

\end{tikzpicture}
%\end{center}


		\vspace{5mm}
		$\Client_1$ should be equivalent with $\Client_2$: $\Client_1 \wbc \Client_2$
	\end{frame}

	\begin{frame}{Equivalence Theory: Reduction-closed, Barbed-preserving Congruence}
		\begin{itemize}
			\item	A symmetric relation $\Re$ is  a reduction-closed, barbed-preserving congruence
				if $P\ \Re\ Q$ implies
				\begin{itemize}
					\item	$P \red P'$ if $Q \Red Q'$ and $P'\ \Re\ Q'$
					\item	$P \barb{n}$ if $Q \Barb{n}$
					\item	$\forall C, \context{C}{P}\ \Re\ \context{C}{Q}$
				\end{itemize}

			\item	Natural equivalence relation
			\item	Processes not distinguished under any context (observer)
			\item	Untyped Setting
		\end{itemize}
	\end{frame}


	\begin{frame}{Equivalence Theory for Higher-Order Calculi (Untyped Setting)}
		\begin{itemize}
			\item	Contextual bisimulation
				\begin{itemize}
					\item	Well known behavioural equivalence for higher-order calculi
					\item	Natural characterisation of reduction-closed, barb-preserving congruence
					\item	Universal quantifications on output and input clauses
				\end{itemize}

%			\item	Contextual bisimulation is based on heavy quantifications on output and input clauses

			\item	Alternative behavioural characterisations that do not suffer from universal quantification clauses
				\begin{itemize}
					\item	Triggered bisimulation (Sangiorgi) - no recussion primitives
					\item	Higher-order bisimulation (Jeffrey and Rathke)- no linearity
				\end{itemize}
%			\item	Sangiorgi and, subsequently Jeffrey and Rathke offered alernative behavioural characterisations
%%				in the untyped setting
%				that do not suffer from universal quantification clauses
		\end{itemize}
	\end{frame}

	\begin{frame}{Equivalence Theory for Higher-Order Sessions}
		\begin{itemize}
			\item	A session based higher-order calculus - typed setting
%				We work on the typed setting using as a basis a session based higher-order calculus

			\item	Sessions describe a structured behaviour on the communication semantics of processes

			\item	A new solution on the higher-order equivelence problem based
				on principles that are
				used for the first time and are derived out of the behaviour of session types
		\end{itemize}
	\end{frame}


	\begin{frame}{Equivalence Theory for Higher-Order Sessions}
		\begin{itemize}
			\item	The linear nature of session types allows to define
				a behavioural characterisation for typed higher-order calculi
				which we call {\bf Characteristic Bisimulation}

			\item	Characteristic bisimulation is defined on a refined labelled transition system
				that restricts process actions following the session behaviour.

			\item	A further interesting result is that, due to types, the proofs do
				not require the presence of a name matching construct.
		\end{itemize}
	\end{frame}

	\begin{frame}{Higher-Order Session Calculus}
		\begin{itemize}
			\item	Higher-Order Session $\pi$-calculus (\HOp)
				\[
					\begin{array}{rcl}
						V, W &\bnfis& u \bnfbar \abs{x} P \qquad u\ \bnfis\ n \bnfbar x \\[2mm]
						P, Q &\bnfis& \bout{u}{V} P \bnfbar \binp{u}{x} P \bnfbar \appl{V}{W} \bnfbar P \Par Q \\
						&\bnfbar& \bsel{u}{l} P \bnfbar \bbra{u}{l_i: P_i}_{i \in I} \bnfbar \news{u} P \\
						&\bnfbar& \inact \bnfbar \varp{X} \bnfbar \recp{X}{P}
					\end{array}
				\]

			\item	\HOp may communicate and apply abstractions of processes% as values
				\[
					\begin{array}{rcl}
						\bout{u}{\abs{y}{R}} P \Par \binp{\dual{u}}{x} Q &\red& P \Par Q \subst{\abs{y}{R}}{x}\\
						\appl{(\abs{x}{P})}{V} &\red& P \subst{V}{x}
					\end{array}
				\]
		\end{itemize}
	\end{frame}

	\begin{frame}{Higher-Order Session Calculus}
		\begin{itemize}
			\item	\HOp is a session based higher-order calculus
				\[
					\Gamma; \Lambda; \Delta \proves P \hastype \Proc \qquad \qquad \qquad \Gamma; \Lambda; \Delta \proves V \hastype \U
				\]
			\item	Example: higher-order output typing
				\[
					\tree {
						\Gamma; \Lambda_1; \Delta_1 \cat n: S_2 \proves P \hastype \Proc
						\qquad
						\tree {
							\Gamma; \Lambda_2; \Delta_2 \cat x: S_1 \proves Q \hastype \Proc
						}{
							\Gamma; \Lambda_2; \Delta_2 \proves \abs{x}{Q} \hastype \shot{S_1}
						}
					}{
						\Gamma; \Lambda_1 \cat \Lambda_2; \Delta_1 \cat \Delta_2 \cat n: \btout{\shot{S_1}} S_2 \proves \bout{n}{\abs{x}{Q}} P \hastype \Proc
					}
				\]

			\item	Labelled transition system
				\[
					\tree {
						(\Gamma; \es; \Delta) \by{\ell} (\Gamma; \es; \Delta') \qquad P \by{\ell} P'
					}{
						\Gamma; \Delta \proves P \by{\ell} \Delta' \proves P'
					}
				\]
		\end{itemize}
	\end{frame}

	\begin{frame}{Equivalence Relations}
		\begin{itemize}
			\item	Typed labelled transition semantics is used to define

			\begin{itemize}
				\item	$\cong$\ \ : Reduction-closed, Barb-preserving, Congruence\\
				\item	$\wbc$\ \ : Context Bisimilarity\\
				\item	$\fwb$ : Characteristic Bisimilarity
			\end{itemize}

%			\begin{tabular}{lcl}
%				$\cong$ &:& Reduction-closed, Barb-preserving, Congruence\\
%				$\wbc$ &:& Context Bisimilarity\\
%				$\fwb$ &:& Characteristic Bisimilarity
%			\end{tabular}

		\end{itemize}
	\end{frame}

%	\begin{frame}{Labelled Transition System}
%	\end{frame}

	\begin{frame}{Higher-Order Contextual Bisimulation: Output}
		Suppose $\Gamma; \Delta_1 \proves P\ \Re\ \Delta_2 \proves Q$, for some symmetric $\Re$. Relation $\Re$ is
		a context bisimulation whenever
		\begin{enumerate}[$(\star)$]
			\item	For all $\news{\widetilde{m_1}} \bactout{n}{V}$ such that
				\[
					\Gamma; \Delta_1 \proves P \by{\news{\widetilde{m_1}} \bactout{n}{V}} \Delta_1' \proves P'
				\]
				there exist $Q'$ and $W$ such that 
				\[
					\Gamma; \Delta_2 \proves Q \By{\news{\widetilde{m_2}} \bactout{n}{W}} \Delta_2' \proves Q'
				\]
				and, \fbox{\emph{\textbf{for all} $R$}}  with $\fv{R}=x$, 
				\[
					\Gamma; \Delta_1'' \proves \newsp{\widetilde{m_1}}{P' \Par R\subst{V}{x}}\ \Re\ \Delta_2'' \proves \newsp{\widetilde{m_2}}{Q' \Par R\subst{W}{x}}
				\]
		\end{enumerate}
	\end{frame}

	\begin{frame}{Higher-Order Contextual Bisimulation: Output}
		Equivalently rewrite the conclusion of the \textcolor{blue}{($\star$)} clause:
		\\[2mm]

		Suppose $\Gamma; \Delta_1 \proves P\ \Re\ \Delta_2 \proves Q$, for some symmetric $\Re$. Relation $\Re$ is
		a context bisimulation whenever
		\begin{enumerate}[$(\star)$]
			\item	\dots\\
%				\[
%					P \by{\news{\widetilde{m_1}} \bactout{n}{V}} P'
%				\]
%				there exist $Q'$ and $W$ such that 
%				\[
%					Q \By{\news{\widetilde{m_2}} \bactout{n}{W}} Q'
%				\]
%				and,
				\emph{\textbf{for all} $R$}  with $\fv{R}=x$, 
				\[
					\begin{array}{rcc}
						\Gamma; \Delta_1'' &\proves& \newsp{\widetilde{m_1}}{P' \Par \prcolor{\newsp{s}{\binp{s}{x} R \Par  \bout{\dual{s}}{V} \inact}}}
						\\
						&&\Re
						\\
						\Delta_2'' &\proves&  \newsp{\widetilde{m_2}}{Q' \Par \prcolor{\newsp{s}{\binp{s}{x} R \Par \bout{\dual{s}}{W} \inact}}}
					\end{array}
				\]
		\end{enumerate}
		This is because:
		\[
			\begin{array}{c}
				\newsp{s}{\binp{s}{x} R \Par \bout{\dual{s}}{V} \inact}
				\by{\tau}
				R \subst{V}{x}
				\\
				\newsp{s}{\binp{s}{x} R \Par \bout{\dual{s}}{W} \inact}
				\by{\tau}
				R \subst{W}{x}
			\end{array}
		\]
	\end{frame}

	\begin{frame}{Higher-Order Contextual Bisimulation: Output}
		We can further rewrite the \textcolor{blue}{($\star$)} clause as a \underline{non universally quantified} input triggered context:
		\\[2mm]

		Suppose $\Gamma; \Delta_1 \proves P\ \Re\ \Delta_2 \proves Q$, for some symmetric $\Re$. Relation $\Re$ is
		a context bisimulation whenever
		\begin{enumerate}[$(\star)$]
			\item
				\dots\\
%				\[
%					P \by{\news{\widetilde{m_1}} \bactout{n}{V}} P'
%				\]
%				there exist $Q'$ and $W$ such that 
%				\[
%					Q \By{\news{\widetilde{m_2}} \bactout{n}{W}} Q'
%				\]
%				and,
				%\dots \emph{\textbf{for all} $R$} with $\fv{R}=x$,
				and
				\[
					\begin{array}{rcc}
						\Gamma; \Delta_1'' &\proves& \newsp{\widetilde{m_1}}{P' \Par \prcolor{\binp{t}{y}\newsp{s}{\binp{s}{z} (\appl{y}{z}) \Par \bout{\dual{s}}{V} \inact}}}
						\\
						&& \Re
						\\
						\Delta_1'' &\proves& \newsp{\widetilde{m_2}}{Q' \Par \prcolor{\binp{t}{y}\newsp{s}{\binp{s}{z} (\appl{y}{z}) \Par \bout{\dual{s}}{W} \inact}}}
					\end{array}
				\]
		\end{enumerate}

		This is because \emph{\textbf{for all} $R$} with $\fv{R}=x$, 
		\[
			\begin{array}{c}
				\binp{t}{y}\newsp{s}{\binp{s}{z} (\appl{y}{z}) \Par \bout{\dual{s}}{V} \inact}
				\by{\bactinp{t}{\abs{x}{R}}} \by{\tau} \by{\tau}
				R \subst{V}{x}
				\\
				\binp{t}{y}\newsp{s}{\binp{s}{z} (\appl{y}{z}) \Par \bout{\dual{s}}{W} \inact}
				\by{\bactinp{t}{\abs{x}{R}}} \by{\tau} \by{\tau}
				R \subst{W}{x}
			\end{array}
		\]
	\end{frame}

	\begin{frame}{Higher-Order Contextual Bisimulation: Input}
		Suppose $\Gamma; \Delta_1 \proves P\ \Re\ \Delta_2 \proves Q$, for some symmetric $\Re$. Relation $\Re$ is
		a context bisimulation whenever
		\begin{enumerate}[$(\bullet)$]
			\item	\fbox{\emph{\textbf{For all} $V$}} such that:
				\[
					\Gamma; \Delta_1 \proves P \by{\bactinp{n}{V}} \Delta_1' \proves P'
				\]
				then there exists $Q'$ such that
				\[
					\Gamma; \Delta_2 \proves Q \By{\bactinp{n}{V}} \Delta_2' \proves Q'
				\]
				and
				\[
					\Gamma; \Delta_1' \proves P'\ \Re\ \Delta_2' \proves Q'
				\]
		\end{enumerate}
	\end{frame}

	\begin{frame}{Higher-Order Contextual Bisimulation: Input}
		\begin{itemize}
			\item	$\Gamma; \Delta_1$ informs that $V$ comes from the class of processes with type $U$
		\end{itemize}
		\[
			\Gamma; \es; \Delta \proves V \hastype U
		\]
		\begin{itemize}
			\item	Choose the {\em simplest} process $V$ that inhabits $U$\\
				(to improve the \textcolor{blue}{$(\bullet)$} clause)
		\end{itemize}
	\end{frame}

	\begin{frame}{Characteristic Bisimulation: Characteristic Process and Value}
%		We can take advantage of session types to improve the
%		\textcolor{blue}{$(\bullet)$} clause.
%		\vspace{3mm}

		\begin{definition}[Characteristic Process and Value]
			\begin{itemize}
				\item	Type $U$ is inhabited by a simple
					{\bf characteristic process}, $\mapchar{U}{u}$,
					assuming $u$ is a fresh name

				\item	Type $U$ is inhabited by simple
					{\bf characteristic value}, $\omapchar{U}$
			\end{itemize}
%			where $u$ is a name.
		\end{definition}
		%\vspace{1mm}

		\begin{itemize}
			\item	For example session
				\[
					S = \btinp{\shot{S_1}} \btout{S_2} \tinact
				\]
				is inhabited by {\em characteristic process}
				\[
					\mapchar{S}{u} = \binp{u}{x} (\appl{x}{s_1} \Par \bout{u}{s_2} \inact)
				\]
				for a fresh name $u$.
		\end{itemize}
%		\vspace{3mm}
%
%		Characteristic types for values are inhabited as:
%		\[
%			\omapchar{S} = s \qquad \omapchar{\shot{S}} = \abs{x}{\mapchar{S}{x}}
%		\]
	\end{frame}

	\begin{frame}{Characteristic Bisimulation: Labelled Transition System}
		\begin{itemize}
			\item	Define a coarser labelled transition system
			\[
				\tree {
					\begin{array}{c}
						%\Gamma; \Delta \proves P \by{\bactinp{n}{V}} \Delta' \proves P' \\
						(\Gamma_1; \Lambda_1; \Delta_1) \by{\bactinp{n}{V}} (\Gamma_1; \Lambda_2; \Delta_2)\\
						\begin{array}{rrcl}
						& V &=& m \\
						\vee& V &\scong& \omapchar{U} \\ 
						\vee& V &\scong& \abs{x}{\binp{t}{y} (\appl{y}{x})} \text{ with $t$ fresh}
						\end{array}
					\end{array}
				}{
					(\Gamma_1; \Lambda_1; \Delta_1) \hby{\bactinp{n}{V}} (\Gamma_1; \Lambda_2; \Delta_2)
	%				\Gamma; \Delta \proves P \hby{\bactinp{n}{V}} \Delta' \proves P'
				}
			\]

			\item	Define $\Gamma; \Delta \proves P \hby{\ell} \Delta' \proves P'$
			\item	Value $\abs{x}{\binp{t}{y} (\appl{y}{x})}$ is called {\bf trigger value}
			\item	Lack of the trigger value in the above definition would result
				in an under-discriminating equivalence relation\\
				(see example~12 in the paper)
		\end{itemize}
	\end{frame}

	\begin{frame}{Characteristic Bisimulation: Trigger Value}
		\begin{itemize}
			\item	The trigger value is a universal higher-order input
				\[
				\begin{array}{l}
					\Gamma; \Delta \cat n: \btinp{\shot{U}} S \proves \binp{n}{x} P
					\by{\bactinp{n}{\abs{x}{\binp{t}{y} (\appl{y}{x})}}}\\
					\Gamma \cat t: \chtype{\shot{U}}; \Delta \cat n: S %\cat t: \btinp{\shot{U}} \tinact
					\proves P \subst{\abs{x}{\binp{t}{y} (\appl{y}{x})}}{x} 
				\end{array}
				\]
			\item	This is because
				\[
					\forall \shot{U}, \quad \Gamma \cat t: \chtype{\shot{U}}; \es %\cat t: \btinp{\shot{U}} \tinact
					\proves \abs{x}{\binp{t}{y} (\appl{y}{x})} \hastype \shot{U}
				\]

			\item	If an instance of the trigger value inputs a trigger value
				\begin{eqnarray*}
					\Gamma \cat t_1: \chtype{\shot{U}}; \Delta
					%\cat t_1: \btinp{\shot{U}} \tinact
					\proves \binp{t_1}{y} (\appl{y}{n}) & \by{\bactinp{t_1}{\abs{x}{\binp{t_2}{y} (\appl{y}{x})}}} \by{\tau}\\
					\Gamma \cat t_2: \chtype{\shot{U}}; \Delta
					%\cat t_2: \btinp{\shot{U}} \tinact
					\proves \binp{t_2}{y} (\appl{y}{n}) %\appl{\abs{x}{\binp{t'}{y} (\appl{y}{x})}}{n} 
				\end{eqnarray*}
%			\item	Lack of the trigger value in the above definition would result
%				in an under-discriminating equivalence relation (example~12 in the paper)
		\end{itemize}
	\end{frame}

	\begin{frame}{Characteristic Bisimulation: Trigger Process}
		\begin{itemize}
			\item	In the light of the coarser semantics that
				relation $\hby{\ell}$ offers, process
				\[
					\newsp{\widetilde{m_1}}{P' \Par \binp{t}{y}\newsp{s}{\prcolor{\binp{s}{z} (\appl{y}{z})} \Par \bout{s}{V} \inact}}
				\]
				is equivalent with {\bf trigger process}
				\[
					\newsp{\widetilde{m_1}}{P' \Par \binp{t}{y}\newsp{s}{\prcolor{\mapchar{\btinp{U_1} \tinact}{s}} \Par \bout{s}{V} \inact}}
				\]
			\item	This is because
				\[
					\begin{array}{rl}
						\binp{t}{y} \newsp{s}{\binp{s}{z} (\appl{y}{z}) \Par \bout{s}{V} \inact}
						&\by{\bactinp{t}{\omapchar{U}}}\\
						\newsp{s}{\mapchar{\btinp{U_1} \tinact}{s} \Par \bout{s}{V} \inact} %= \binp{s}{z} \appl{y}{z}
					\end{array}
				\]
		\end{itemize}
	\end{frame}

	\begin{frame}{Characteristic Bisimulation}

%		Suppose $\Gamma; \Delta_1 \proves P\ \Re\ \Delta_2 \proves Q$ for some {\em characteristic} bisimulation~$\Re$. Then
		Suppose $\Gamma; \Delta_1 \proves P\ \Re\ \Delta_2 \proves Q$, for some symmetric $\Re$. Relation $\Re$ is a
		{\bf characteristic} bisimulation whenever
		\begin{enumerate}[$(\star)$]
			\item	Whenever
				\[
					\Gamma; \Delta_1 \proves P \hby{\news{\widetilde{m_1}} \Delta_1' \proves \bactout{n}{V: U_1}} P'
				\]
				there exist $Q'$ and $W$ such that 
				\[
					\Gamma; \Delta_2 \proves Q \Hby{\news{\widetilde{m_2}} \bactout{n}{W: U_2}} \Delta_2' \proves Q'
				\]
				and
				\[
					\begin{array}{c}
						\Gamma; \Delta_1'' \proves \newsp{\widetilde{m_1}}{P' \Par \binp{t}{y}\newsp{s}{\mapchar{\btinp{U_1} \tinact}{s} \Par \bout{s}{V} \inact}}
						\\
						\Re
						\\
						\Delta_2'' \proves \newsp{\widetilde{m_2}}{Q' \Par \binp{t}{y}\newsp{s}{\mapchar{\btinp{U_2} \tinact}{s} \Par \bout{s}{W} \inact}}
					\end{array}
				\]
		\end{enumerate}
	\end{frame}

	\begin{frame}{Characteristic Bisimulation}

		\begin{enumerate}[$(\bullet)$]
			\item	Whenever
				\[
					\Gamma; \Delta_1 \proves P \hby{\bactinp{n}{V}} \Delta_1' \proves P'
				\]
				there exist $Q'$ such that 
				\[
					\Gamma; \Delta_2 \proves Q \Hby{\bactinp{n}{V}} \Delta_2' \proves Q'
				\]
				and
				\[
					\Gamma; \Delta_1' \proves P'\, \Re\, \Delta_2' \proves Q'
				\]
		\end{enumerate}
	\end{frame}

	\begin{frame}{Soundness and Completeness}
%		\begin{tabular}{lcc}
%			Reduction-closed, Barb-preserving, Congruence&: & $\cong$\\
%			Context Bisimilarity&: & $\wbc$\\
%			Characteristic Bisimilarity&: & $\fwb$
%		\end{tabular}
%		\vspace{5mm}
		\begin{itemize}
			\item	The behavioural equivalences coincide

				\begin{theorem}
					\begin{center}
					$\cong \qquad =\qquad \wbc \qquad =\qquad \fwb$
					\end{center}
				\end{theorem}

			\item	No matching construct is required for proofs - types contain all the information needed

		\end{itemize}
	\end{frame}

	\begin{frame}{No matching: Example}
			\begin{itemize}
				\item	Assume
					\begin{eqnarray*}
						&&\es; \es; n: \btout{S} \tinact, m_1 : S \proves \bout{n}{m_1} \inact \\
						&&\es; \es; n: \btout{S} \tinact, m_2 : S \proves \bout{n}{m_2} \inact
					\end{eqnarray*}
				\item	Observe actions $\bactout{n}{m_1}$ and $\bactout{n}{m_2}$, respectively
					\begin{eqnarray*}
						&& t: \chtype{\tinact}; \es; m_1 : S \proves \binp{t}{y} \newsp{s}{\mapchar{\btinp{S} \tinact}{s}\Par \bout{\dual{s}}{m_1}\inact} \\
						&& t: \chtype{\tinact}; \es; m_2 : S \proves \binp{t}{y} \newsp{s}{\mapchar{\btinp{S} \tinact}{s}\Par \bout{\dual{s}}{m_2}\inact} 
					\end{eqnarray*}


				\item	No need for the observer to match $m_1$ and $m_2$
				\begin{itemize}
					\item	If $S = \tinact$ then $P$ and $Q$ are bisimilar because there is no further observation on $m_1$ and $m_2$
					\item	Otherwise $P'$ and $Q'$ can be distinguished because we observe interaction on $m_1$ and $m_2$,
						respectively.
				\end{itemize}
			\end{itemize}
	\end{frame}

	\begin{frame}{Summary of the Results/Contribution}
		\begin{itemize}
			\item	Session based higher-order $\pi$-calculus - \HOp
				\begin{itemize}
					\item	Syntax and semantics
					\item	Session types
					\item	Session types inform a coarser/simpler labelled transition semantics
				\end{itemize}

			\item	Behavioural equivalence theory
				\begin{itemize}
					\item	Sessions enforce a structured behaviour different than
						the behaviour of untyped processes %typed processes
					\item	Sessions provide additional and crucial information to elliminate
					\begin{itemize}
						\item	Heavy quantification requirements
						\item	The need of a name matching construct
					\end{itemize}
				\end{itemize}
		\end{itemize}

	\end{frame}

	\begin{frame}{Summary of the Results/Contribution}
		\begin{itemize}
			\item	Characteristic Bisimilarity
				\begin{itemize}
					\item	First equivalence definition to exploit session type information
					\item	Follows session behaviour
					\item	We claim that is easier to compute due to session linearity
				\end{itemize}

			\item	Characterisation
				\begin{itemize}
					\item	Reduction-closed, barb-preserving congruence
					\item	Context bisimularity
					\item	Characteristic bisimilarity
				\end{itemize}

		\end{itemize}
	\end{frame}
%	\begin{frame}{Summary of the Results}
%		\begin{itemize}
%			\item	Characteristic bisimilarity
%				\begin{itemize}
%					\item	Information from session types
%					\item	Principle applied for the first time in process calculi equivalence theory
%					\item	No matching construct is required have a full contextual characterisation
%				\end{itemize}
%		\end{itemize}
%	\end{frame}

	\begin{frame}{Questions?}
		\begin{center}
			\huge Thank you for your attention
		\end{center}
	\end{frame}

	\begin{frame}{The need for trigger value}
%		\begin{example}[The Need for Refined Typed LTS]
%		\label{ex:motivation}
%		We show that observing a characteristic value
%		input alone is not enough
%		\dk{to define a sound bisimulation closure}.
		Consider processes % $P_1, P_2$:
		%
		\begin{eqnarray*}
			P_1 = \binp{s}{x} (\appl{x}{s_1} \Par \appl{x}{s_2}) 
			& & 
			P_2 = \binp{s}{x} (\appl{x}{s_1} \Par \binp{s_2}{y} \inact) 
%			\label{equ:6}
		\end{eqnarray*}
		%
		%We can show that 
		with
		\[
			\Gamma; \es; \Delta \cat s: \btinp{\shot{(\btinp{C} \tinact)}} \tinact \proves P_i \hastype \Proc (i \in \{1,2\})
		\]

		If $P_1$ and $P_2$ input and substitute over $x$
		the characteristic value
		\[
			\omapchar{\shot{(\btinp{C} \tinact)}} = \abs{x}{\binp{x}{y} \inact}
		\] 
		then they both evolve into %(\ref{eq:5}) and (\ref{eq:6}) in become:
		\begin{center}
		%\begin{tabular}{c}
			$\Gamma; \es; \Delta \proves \binp{s_1}{y} \inact \Par \binp{s_2}{y} \inact \hastype \Proc$
		%\end{tabular}
		\end{center}
		\noi therefore becoming 
		context bisimilar
	\end{frame}

	\begin{frame}{The need for trigger value}
		%after the input of $\abs{x}{\binp{x}{y}} \inact$.
		However, $P_1$ and $P_2$
%		the processes in (\ref{equ:6}) 
		are
%clearly
		{\em not} context bisimilar:
%		: many input actions
%		may be used to distinguish them.
%		For example,
		If 
		$P_1$ and $P_2$ input and substitute over $x$ the value
		\[
			\abs{x} \newsp{s}{\bout{a}{s} \binp{x}{y} \inact}
		\]
		with
		\[
			\Gamma; \es; \Delta \proves s \hastype \tinact
		\]
		then their derivatives
		\begin{eqnarray*}
			P_1' &=& \abs{x}{\newsp{s}{\bout{a}{s} \binp{s_1}{y} \inact}} \Par \abs{x}{\newsp{s}{\bout{a}{s} \binp{s_2}{y} \inact}}\\
			P_2' &=& \abs{x}{\newsp{s}{\bout{a}{s} \binp{s_1}{y} \inact}} \Par \binp{s_2}{y} \inact
		\end{eqnarray*}
		are not bisimilar
	\end{frame}

	\begin{frame}{The need for trigger value}
%		Observing only the characteristic value 
%		results in an under-discriminating bisimulation.
		However, if a trigger value
		\[
			\abs{{x}}{\binp{t}{y} (\appl{y}{{x}})}
		\]
		is received on $s$, we can distinguish $P_1$, $P_2$ %($\ell = \bactinp{s}{\abs{{x}}{\binp{t}{y} (\appl{y}{{x}})}}$):
		%
		\begin{eqnarray*}
			P_1 &\hby{\bactinp{s}{\abs{{x}}{\binp{t}{y} (\appl{y}{{x}})}}} \hby{\tau}& \binp{t}{x} (\appl{x}{s_1}) \Par \binp{t}{x} (\appl{x}{s_2})\\
%			\mbox{~and~}
			P_2 &\hby{\bactinp{s}{\abs{{x}}{\binp{t}{y} (\appl{y}{{x}})}}} \hby{\tau}& \binp{t}{x} (\appl{x}{s_1}) \Par \binp{s_2}{y} \inact
%			\quad 
		\end{eqnarray*}
	\end{frame}

	\begin{frame}{The need for trigger value}
		The trigger value alone is not enough.
		Consider processes
		%
		\begin{eqnarray*}
			&&\newsp{s}{\binp{n}{x} (\appl{x}{s}) \Par \bout{\dual{s}}{\abs{x} R_1} \inact} \\
			&&\newsp{s}{\binp{n}{x} (\appl{x}{s}) \Par \bout{\dual{s}}{\abs{x} R_2} \inact} 
		\end{eqnarray*}
		%
		On a trigger value input, we obtain the derivatives
		\begin{eqnarray*}
			&&\newsp{s}{\binp{t}{x} (\appl{x}{s}) \Par \bout{\dual{s}}{\abs{x} R_1} \inact} \\
			&&\newsp{s}{\binp{t}{x} (\appl{x}{s}) \Par \bout{\dual{s}}{\abs{x} R_2} \inact}
		\end{eqnarray*}

		\noi thus concluding a bisimulation closure

		But
		on a characteristic value ($\abs{z}{\binp{z}{x} (\appl{x}{m})}$) input 
		then they would become
		%
		\begin{eqnarray*}
			\Gamma; \es; \Delta \proves \newsp{s}{\binp{s}{x} (\appl{x}{m}) \Par \bout{\dual{s}}{\abs{x} R_i} \inact} \wbc \Delta \proves R_i \subst{m}{x}
		\quad (i=1,2)
		\end{eqnarray*}
		\noi which are not bisimilar if $R_1 \subst{m}{x} \not\wbc R_2 \subst{m}{x}$.
	\end{frame}

\end{document}

