% !TEX root = main.tex
%\myparagraph{Key points}
%\begin{enumerate}[1.]
%%	\item	Session $\pi$ calculus with process passing. DONE
%%	\item	Identify session $\pi$ and process passing subcalculi and their polyadic variants. DONE
%%	\item	Bisimulation theory for higher-order session semantics. DONE
%%	\item	New triggered bisimulation, related to J\&R's. DONE
%%	\item   Elementary values key to characterizations of behavioural equivalence. DONE
%	\item	Types provide techniques to prove completeness without matching. \jp{TBD}
%	\item	We are interested in encodings with properties a la Gorla. 
%                We extended them to typed setting. \jp{TBD}
%%	\item	Encode name-passing to pure process abstraction calculus, with name abstractions. DONE
%%	\item	Type of the recursion encoding uses non tail recursive type $\trec{t}{\btinp{t} \tinact}$. DONE
%%	\item	Encode higher-order semantics to first order semantics. DONE
%%	\item	Negative result. Cannot encode shared names using only shared names.
%%	\item   Extensions with higher-order abstractions and polyadicity also explored. DONE
%\end{enumerate}

%\smallskip 
%
%\myparagraph{Important things to explain}
%Explain our \HO is very small without containg name passing 
%\[ 
%\abs{x}.P \quad \appl{x}{u}
%\]

%Explain we input only characteristic processes.  
%
%\[
%\lambda x.\mapchar{S}{x}
%\]

%\subsection{Higher-Order Session Calculi}
\noindent
%\myparagraph{Context.}
By combining features from the $\lambda$-calculus and the $\pi$-calculus, 
in \emph{higher-order process calculi} exchanged values may contain  processes. 
In this paper we consider \HOp, a higher-order calculus with \emph{session primitives}:
in addition to 
functional
abstractions and applications, \HOp 
contains constructs for 
synchronisation on shared names, 
  session communication on linear names (value passing, 
labelled choice), and recursion.
Thus, \HOp processes may specify reciprocal exchanges (protocols) 
for higher-order mobile processes that
 can be verified via type-checking using \emph{session types}~\cite{honda.vasconcelos.kubo:language-primitives}.
%These calculi allow us to specify   
%session protocols in which higher-order values 
%(mobile code) can be exchanged in a type-safe manner. 
%; 
%governed by session types, 
%such protocols cleanly distinguish between 
%linear and unrestricted behaviors in 
%%directed %point-to-point 
%communications.
The study of higher-order concurrency has received significant attention, 
from untyped and typed perspectives (see, e.g.,~\cite{ThomsenB:plachoasgcfhop,SangiorgiD:expmpa,JeffreyR05,MostrousY15,DBLP:journals/iandc/LanesePSS11,DBLP:conf/icalp/LanesePSS10,DBLP:conf/esop/KoutavasH11,XuActa2012}).
%in particular via  comparisons with the first-order mobility of the $\pi$-calculus~\cite{MilnerR:calmp1}. 
Although models of session-typed 
communication with features of higher-order concurrency exist~\cite{tlca07,DBLP:journals/jfp/GayV10},
their  \emph{tractable behavioural equivalences} 
%and \emph{relative expressiveness}
remain little understood. 
%for higher-order session calculi. 
%these two issues 
%have been throughly studied
%%are well-understood 
%for higher-order languages without sessions \cite{},
%but not for higher-order process calculi with sessions.
%This is unfortunate, given the wide applicability of session-based concurrency; indeed,
%session types are expressive enough to describe complex 
%communication structures found in practical protocols,  expressible, e.g., via recursive session types.
%Clarifying the status of typed equivalences and relative expressiveness for session languages
Clarifying their status is essential to, e.g., 
justify non-trivial protocol optimisations involving both name- and process-passing mobility.
%but also for transferring key reasoning techniques between (higher-order) session calculi. 
Our discovery is that \emph{linearity} of session types plays a vital role to 
offer equalities and characterisations
% and fully abstract encodability, 
which to our knowledge have not been proposed before.   

%In this paper we study
%%address  behavioural equivalences for 
%\HOp, 
%%study behavioral equalities for \HOp, 
%an extension of the higher-order $\pi$-calculus~\cite{SangiorgiD:expmpa} with session primitives:
%\HOp contains constructs for 
%%session establishment
%synchronisation on shared names, 
%recursion, 
% (linear) session communication (value passing and
%labelled choice),
%abstractions and applications. 
%Abstractions are functions from values to processes, 
%\jpc{denoted}
%$\lambda x.P$; applications are 
%denoted $(\lambda x.P)V$, where the value $V$ is either a name or an abstraction.
%We study two significant subcalculi of \HOp, 
%\jpc{which}
%distil higher- and first-order mobility:
%the \HO-calculus, which is \HOp without recursion and name passing, and 
%the session \sessp-calculus \jpc{(here denoted~\sessp)}, which is \HOp without abstractions and applications.  
%While \sessp is, 
%in essence, the calculus in~\cite{honda.vasconcelos.kubo:language-primitives}, 
%this paper shows that \HO  is a new core calculus 
%for higher-order session concurrency.

A well-studied behavioural equality in the higher-order setting 
is \emph{context bisimilarity}~\cite{San96H}. This is 
a labelled characterisation of reduction-closed, barbed congruence 
that offers an adequate distinguishing power at the price of heavy universal quantifications in output clauses.
Obtaining alternative characterisations of context bisimilarity
is thus a recurring issue 
in the study of higher-order calculi---see, e.g.,~\cite{SangiorgiD:expmpa,San96H,JeffreyR05,DBLP:journals/cl/KoutavasH12}. 
Our approach 
to this long-standing problem is to 
rely on protocol specifications---as given by session types---to limit 
the behaviour of higher-order session processes. 
This limitation is formally enforced by 
a refined labelled transition system (LTS)
which effectively 
narrows down the spectrum of allowed process behaviours, 
exploiting elementary processes inhabiting session types.
%thus enabling tractable reasoning techniques. 
We propose \emph{characteristic bisimilarity}: this is 
a new notion of typed bisimilarity that is 
\emph{tractable}, in that 
it relies on the refined LTS for input actions and, more importantly, 
does not appeal to universal quantifications on output actions. 
%shown to coincide with context bisimilarity.
Our main result is that characteristic  %tractable
bisimilarity coincides with context bisimilarity:
in addition to its intrinsic value as an useful reasoning technique, this result is remarkable 
also from a technical perspective, in that associated 
completeness proofs do not require 
operators for 
name-matching in the process language, in contrast to known methods for untyped higher-order processes
with recursion (cf.~\cite{JeffreyR05}).
%Remarkably session type structures enable to provide 
%a coincidence without name-matching operators in the calculi.


%\myparagraph{Outline / Contributions.} 

This paper 
is structured as follows.
%\begin{enumerate}[$\bullet$]
%\item 
Next, \secref{sec:overview} overviews key ideas of characteristic bisimilarity, 
our 
tractable characterisation of contextual equivalence.
%\item 
Then, \secref{sec:calculus} presents the higher-order session calculus \HOp. 
\secref{sec:types} gives the session type system
and states type soundness for \HOp and its variants.
%\item 
\secref{sec:behavioural} 
develops %\emph{higher-order} and 
\emph{characteristic} bisimilarity,  which 
alleviates the issues of context bisimilarity~\cite{San96H} and is shown  
to coincide for well-typed \HOp processes (\thmref{the:coincidence}).
\secref{sec:relwork} concludes with related works. The appendix summarises the typing system. 
%\end{enumerate}
%\noi
%The paper is self-contained. 
\textbf{Omitted definitions and proofs and additional related work/examples  
%can be found
are 
in~\cite{KouzapasPY15}.} 

