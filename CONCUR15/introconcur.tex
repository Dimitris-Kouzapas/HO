% !TEX root = main.tex
%\myparagraph{Key points}
%\begin{enumerate}[1.]
%%	\item	Session $\pi$ calculus with process passing. DONE
%%	\item	Identify session $\pi$ and process passing subcalculi and their polyadic variants. DONE
%%	\item	Bisimulation theory for higher-order session semantics. DONE
%%	\item	New triggered bisimulation, related to J\&R's. DONE
%%	\item   Elementary values key to characterizations of behavioural equivalence. DONE
%	\item	Types provide techniques to prove completeness without matching. \jp{TBD}
%	\item	We are interested in encodings with properties a la Gorla. 
%                We extended them to typed setting. \jp{TBD}
%%	\item	Encode name-passing to pure process abstraction calculus, with name abstractions. DONE
%%	\item	Type of the recursion encoding uses non tail recursive type $\trec{t}{\btinp{t} \tinact}$. DONE
%%	\item	Encode higher-order semantics to first order semantics. DONE
%%	\item	Negative result. Cannot encode shared names using only shared names.
%%	\item   Extensions with higher-order abstractions and polyadicity also explored. DONE
%\end{enumerate}

%\smallskip 
%
%\myparagraph{Important things to explain}
%Explain our \HO is very small without containg name passing 
%\[ 
%\abs{x}.P \quad \appl{x}{u}
%\]

%Explain we input only characteristic processes.  
%
%\[
%\lambda x.\mapchar{S}{x}
%\]

%\subsection{Higher-Order Session Calculi}
\noindent
By combining features from the $\lambda$-calculus and the $\pi$-calculus, 
in \emph{higher-order process calculi} exchanged values may contain  processes. 
In this paper, we consider higher-order calculi with \emph{session primitives},
thus enabling the specification of reciprocal exchanges (protocols) 
for higher-order mobile processes, 
which can be verified via type-checking using \emph{session types}~\cite{honda.vasconcelos.kubo:language-primitives}.
%These calculi allow us to specify   
%session protocols in which higher-order values 
%(mobile code) can be exchanged in a type-safe manner. 
%; 
%governed by session types, 
%such protocols cleanly distinguish between 
%linear and unrestricted behaviors in 
%%directed %point-to-point 
%communications.
The study of higher-order concurrency has received significant attention, 
from untyped and typed perspectives (see, e.g.,~\cite{ThomsenB:plachoasgcfhop,SangiorgiD:expmpa,San96int,JeffreyR05,MostrousY15,DBLP:journals/iandc/LanesePSS11,DBLP:conf/icalp/LanesePSS10,DBLP:conf/esop/KoutavasH11,XuActa2012}).
%in particular via  comparisons with the first-order mobility of the $\pi$-calculus~\cite{MilnerR:calmp1}. 
Although models of session-typed 
communication with features of higher-order concurrency exist~\cite{tlca07,DBLP:journals/jfp/GayV10},
their  \emph{tractable behavioural equivalences} and \emph{relative expressiveness}
remain little understood. 
%for higher-order session calculi. 
%these two issues 
%have been throughly studied
%%are well-understood 
%for higher-order languages without sessions \cite{},
%but not for higher-order process calculi with sessions.
%This is unfortunate, given the wide applicability of session-based concurrency; indeed,
%session types are expressive enough to describe complex 
%communication structures found in practical protocols,  expressible, e.g., via recursive session types.
%Clarifying the status of typed equivalences and relative expressiveness for session languages
Clarifying their status is not only useful for, 
e.g.,~justifying non-trivial mobile protocol
optimisations, but also for transferring key reasoning techniques
between (higher-order) session calculi. Our discovery 
is that \emph{linearity} of session types plays a vital role to 
offer new equalities and fully abstract encodability, 
which to our best knowledge have not been proposed before.   

The main higher-order language in our work, denoted \HOp,
extends the higher-order $\pi$-calculus~\cite{SangiorgiD:expmpa} with session primitives:
it contains constructs for 
%session establishment
synchronisation on shared names, 
recursion, 
name abstractions (i.e., functions from name identifiers  to processes, 
\jpc{denoted}
$\lambda x.P$) and applications 
\jpc{(denoted $(\lambda x.P)a$)};
and session communication (value passing and
labelled choice using linear names). 
We study two significant subcalculi of \HOp, 
\jpc{which}
distil higher- and first-order mobility:
the \HO-calculus, which is \HOp without recursion and name passing, and 
the session \sessp-calculus \jpc{(here denoted~\sessp)}, which is \HOp without abstractions and applications.  
While \sessp is, 
in essence, the calculus in~\cite{honda.vasconcelos.kubo:language-primitives}, 
this paper shows that \HO  is a new core calculus 
for higher-order session concurrency.

In this paper, we address tractable behavioural equivalences
for \HOp.
A well-studied behavioural equivalence in the higher-order setting 
is \emph{context bisimilarity}~\cite{San96H},
a labelled characterisation of reduction-closed, barbed congruence, 
which offers an appropriate discriminative power at the price of heavy universal quantifications in output clauses.
Obtaining alternative characterisations 
is thus a recurring issue 
in the study of higher-order calculi. 
Our approach 
shows that protocol specifications given by session types are 
essential to  limit 
the behaviour of higher-order session processes. 
Exploiting elementary processes inhabiting session types, 
this limitation is formally enforced by 
a refined (typed) labelled transition system (LTS)
that narrows down the spectrum of allowed process behaviours, 
thus enabling tractable reasoning techniques. 
Two tractable characterisations of bisimilarity 
are then introduced. 
%shown to coincide with context bisimilarity.
Remarkably, using session types we prove that these %tractable
bisimilarities coincide with context bisimilarity, without using
operators for 
name-matching.
%Remarkably session type structures enable to provide 
%a coincidence without name-matching operators in the calculi.


\myparagraph{Outline / Contributions.} This paper 
is structured as follows:
\begin{enumerate}[$\bullet$]
\item \secref{sec:overview} overviews key ideas of our tractable bisimulations.
\item \secref{sec:calculus} presents the higher-order session calculus \HOp and its 
subcalculi \HO and \sessp.  Then, \secref{sec:types} gives the session type system
and states type soundness for \HOp and its variants.
\item \secref{sec:behavioural} 
develops \emph{higher-order} and \emph{characteristic} bisimilarities, our two
tractable characterisations of contextual equivalence which 
alleviate the issues of context bisimilarity~\cite{San96H}. These 
relations are shown to coincide in \HOp (\thmref{the:coincidence}).

\item \secref{sec:relwork} concludes with related works. The appendix summarises the typing system. 
\end{enumerate}
\noi
The paper is self-contained. 
{\bf\em Additional related work, more examples, omitted definitions, and  proofs 
%can be found
are 
in~\cite{KouzapasPY15}.} 

