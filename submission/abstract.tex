% !TEX root = main.tex
\begin{abstract}
In calculi for \emph{higher-order} concurrency, values exchanged as communication objects may include processes. Based on process passing, this paradigm is in contrast with the \emph{first-order} (or name passing) concurrency of the $\pi$-calculus. %based on name passing. 
%The higher-order paradigm is sometimes consideredBy combining mechanisms for functional and concurrent computation, these languages offer a unified account of different forms of interaction. 
Although previous works have related calculi for higher-order and first-order concurrency, little is known about the nature of this relationship when interactions are disciplined by \emph{types for structured communications}. Here we tackle this challenge from the perspective of \emph{session types}, focusing on \emph{typed behavioral equivalences} and issues of \emph{relative expressiveness}. First, we identify a core higher-order, session-typed calculus and develop for it  characterizations of typed contextual equivalences as labeled bisimilarities. 
Second, we formalize (non) encodability results between session calculi with either name-passing or process-passing mechanisms. 
Our results clarify the relationship between name- and process-passing in terms of high-level communication structures based on types. 
\end{abstract}