\section{Behavioural Semantics}
\label{sec:beh_sem}

In this section we define a theory for observational equivalence over
session typed $\HOp$ processes. The theory follows the principles
laid by the previous work of the authors
\cite{DBLP:conf/forte/KouzapasYH11,KY13,dkphdthesis}.
We require a bisimulation relation over typed processes that
is also characterised by the corresponding typed, reduction-closed,
barbed congruence relation.

\dk{(Jorge, I think you have a paper we can cite over session typed bisimulations)}

\subsection{Labelled Transition Semantics}

We define a relation $(P_1, \lambda, P_2) \in R$ over
(untyped) processes, that allows us to follow how a process may
interact with a process in its enviroment. The interaction
is defined on action $\lambda$:

\begin{tabular}{rcl}
		$\lambda$ &$\bnfis$& $\tau \bnfbar \bactout{s}{s'} \bnfbar \bactout{s}{\abs{x} P} \bnfbar\bactinp{s}{s} \bnfbar \bactinp{s}{\abs{x} P}$ \\
		&	$\bnfbar$ & $\bactsel{s}{l} \bnfbar \bactbra{s}{l} \bnfbar \news{s'} \bactout{s}{s'} \bnfbar \news{\tilde{s}} \bactout{s}{\abs{x} P}$
%		\\
%		&$\bnfbar$ &	\dk{$\bactout{s}{\tilde{s}} \bnfbar \bactout{s}{\abs{\tilde{x}} P} \bactinp{s}{\tilde{s'}} \bnfbar \bactinp{s}{\abs{\tilde{x}} P}$}
\end{tabular}

The internal action is defined on label $\tau$.
Action $\bactout{s}{s'}$ denotes the sending of name $s'$ over channel $s$.
Similarly action $\bactout{s}{\abs{x}{P}}$ is the sending of abstraction $\abs{x}{P}$
over channel $s$. Dually actions for the reception of names and abstractions are
$\bactinp{s}{s'}$ and $\bactinp{s}{\abs{x}{P}}$ respectively. We also defined
actions for selecting a label $l$, $\bactsel{s}{l}$ and branching on a label
$l$, $\bactbra{s}{l}$. When output actions carry name restrictions a scope
opening is implied.

We define the notion of dual actions as the symmetric relation $\asymp$, that satisfies the rules:

\begin{tabular}{c}
	$\bactsel{s}{l} \asymp \bactbra{\dual{s}}{l}
	\qquad \news{\tilde{s}} \bactout{s}{\abs{x} P} \asymp \bactinp{\dual{s}}{\abs{x} P}
	\qquad \bactout{s}{s'} \asymp \bactinp{\dual{s}}{s'}
	\qquad \news{s'} \bactout{s}{s'} \asymp \bactinp{\dual{s}}{s'}$
\end{tabular}

Dual actions happen on subjects that are dual between them; they carry the same
object; and furthermore output action is dual with input action and 
select action is dual with branch action.

\paragraph{Untyped Labelled Transition System}

\begin{figure}[t]
	\[
	\begin{array}{c}
		\appl{\abs{x}{P}}{u} \by{\tau} P \subst{u}{x}\ \ltsrule{App}
		\qquad
		\bout{n}{V} P \by{\bactout{n}{V}} P\ \ltsrule{Out}
		\qquad
		\binp{n}{x} P \by{\bactinp{n}{V}} P\subst{V}{x}\ \ltsrule{In}
%		\qquad
%		\bout{n}{\abs{x}{Q}} P \by{\bactout{n}{\abs{x}{Q}}} P\ \ltsrule{OutA}
		\\[4mm]

%		\binp{n}{X} P \by{\bactinp{n}{\abs{x}{Q}}} P\subst{\abs{x}Q}{X}\ \ltsrule{InA}
%		\qquad
		\bsel{s}{l}{P} \by{\bactsel{s}{l}} P\ \ltsrule{Sel}
		\qquad
		\tree{
			j \in I
		}
		{
			\bbra{s}{l_i:P_i}_{i \in I} \by{\bactbra{s}{l_j}} P_j
		}\ \ltsrule{Bra}
		\\[6mm]

		\tree{
			P \by{\ell} P' \quad n \notin \fn{\ell}
		}{
			\news{n} P \by{\ell} \news{n} P' 
		}\ \ltsrule{Res}
		\qquad
		\tree{
			P \scong_\alpha P'' \quad P'' \by{\ell} P'
		}{
			P \by{\ell} P'
		}\ \ltsrule{Alpha}
		\qquad
		\tree{
			P \subst{\recp{X}{P}}{\varp{X}} \by{\ell} P'% \qquad P \scong_\alpha P'' \quad P'' \by{\ell} P'
		}{
			\recp{X}{P} \by{\ell} P'
		}\ \ltsrule{Rec}
		\\[6mm]


		\tree{
			P \by{\news{\tilde{m}} \bactout{n}{V}} P' \quad m \in \fn{V}
		}{
			\news{m} P \by{\news{m\cat\tilde{m}} \bactout{n}{V}} P'
		}\ \ltsrule{Scope}
		\qquad
%		\tree{
%			P \by{\news{\tilde{m}} \bactout{n}{\abs{x} Q}} P' \quad m' \in \fn{\abs{x} Q}
%		}{
%			\news{m'} P \by{\news{m'\cat\tilde{m}} \bactout{n}{\abs{x} Q}} P'
%		}\ \ltsrule{ScopeA}
%		\qquad
		\tree{
			P \by{\ell_1} P' \qquad Q \by{\ell_2} Q' \qquad \ell_1 \asymp \ell_2
		}{
			P \Par Q \by{\tau} \newsp{\bn{\ell_1} \cup \bn{\ell_2}}{P' \Par Q'}
		}\ \ltsrule{Tau}
		\\[6mm]

		\tree{

			P \by{\ell} P' \quad \bn{\ell} \cap \fn{Q} = \es
		}{
			P \Par Q \by{\ell} P' \Par Q
		}\ \ltsrule{LPar}
		\qquad
		\tree{
			Q \by{\ell} Q' \quad \bn{\ell} \cap \fn{P} = \es
		}{
			P \Par Q \by{\ell} P \Par Q'
		}\ \ltsrule{RPar}
%		\\[6mm]
	\end{array}
	\]
	\caption{The Untyped (Early) Labelled Transition System \label{fig:untyped_LTS}}
\end{figure}


The labelled transition system, LTS, is defined in Figure~\ref{fig:untyped_LTS}.
A process with a send prefix can interact with the environment with a send
action that carries a name $s'$ or an abstraction $\abs{x}{Q}$. Dually
on a received prefixed process we can observe a receive action of a name or
an abstraction. Select and branch prefixed processes can trigger select
and branch actions respectively. The LTS is closed under the name creation
operator provided that the restricted name does not occur free in the LTS action.
If the restricted name occurs free in the LTS action then we observe a bound name
action and the continuation process performs scope opening. Similarly the LTS 
is closed on the parallel operator provided that the LTS action does not shared
any bound names with parallel processes. If two parallale processes can perform
dual actions then the two actions can synchronise to observe an internal transition.
Finally the LTS is closed under structural congruence.


\paragraph{Labeled Transition System for Typed Environments}

We define a relation
$((\Gamma, \Lambda_1, \Sigma_1), \lambda, (\Gamma, \Lambda_2, \Sigma_2)) \in R$
over type tuples, that allows us to follow the progress of types over actions $\lambda$.

\begin{figure}
	\[
	\begin{array}{c}
		\tree{
			\dual{s} \notin \dom{\Sigma} \quad \Gamma; \Lambda_1; \Sigma_1 \proves V \hastype U \quad \Sigma_1 \subseteq \Sigma \quad \Lambda_1 \subseteq \Lambda
		}{
			(\Gamma; \Lambda; \Sigma \cat s: \btout{U} S) \by{\bactout{s}{V}} (\Gamma; \Lambda\backslash\Lambda_1; \Sigma\backslash\Sigma_1 \cat s: S)
		}
		\\[6mm]
		\tree{
			\dual{s} \notin \dom{\Sigma} \quad  \Gamma; \Lambda_1; \Sigma_1 \proves V \hastype U
		}{
			(\Gamma; \Lambda; \Sigma \cat s: \btinp{U} S) \by{\bactinp{s}{U}} (\Gamma; \Lambda \cup \Lambda_1; \Sigma \cup \Sigma_1 \cat s: S)
		}
		\\[6mm]
		\tree{
			\dual{s} \notin \dom{\Sigma} \quad k \in I
		}{
			(\Gamma; \Lambda; \Sigma \cat s: \btsel{l_i: S_i}_{i \in I}) \by{\bactsel{s}{l_k}} (\Gamma; \Lambda; \Sigma \cat s:S_k)
		}
		\qquad
		\tree{
			\dual{s} \notin \dom{\Sigma} \quad k \in I
		}{
			(\Gamma; \Lambda; \Sigma \cat s: \btbra{l_i: T_i}_{i \in I}) \by{\bactbra{s}{l_k}} (\Gamma; \Lambda; \Sigma \cat s:S_k)
		}
		\\[6mm]

		\tree{
			(\Gamma; \Lambda_1; \Sigma_1) \by{\news{\tilde{s}} \bactout{s}{V}} (\Gamma; \Lambda_2; \Sigma_2)
	%		\quad S_1 \dualof S_2
		}{
			(\Gamma; \Lambda_1; \Sigma_1) \by{\news{s' \cat \tilde{s'}} \bactout{s}{V}} (\Gamma; \Lambda_2; \Sigma_2 \cat \dual{s'}: S)
		}
		\qquad
		\tree{
			\Sigma_1 \red \Sigma_2
		}{
			(\Gamma; \Lambda; \Sigma_1) \by{\tau} (\Gamma; \Lambda; \Sigma_2)
		}
	\end{array}
	\]
	\caption{Labelled Transition Semantics for Typed Enviroments \label{fig:envLTS}}
\end{figure}


\dk{describe env LTS}

\paragraph{Typed Transition System}

The transition system over processes is defined as a combination
of the untyped LTS and the LTS for typed environments:

\begin{definition}[Typed Transition System]\rm
	We write $\Gamma; \emptyset; \Sigma_1 \proves P_1 \hastype \Proc \by{\lambda} \Gamma; \emptyset; \Sigma_1 \proves P_2 \hastype \Proc$
	whenever:
	\begin{itemize}
		\item	$P_1 \by{\lambda} P_2$.
		\item	$(\Gamma, \emptyset, \Sigma_1) \by{\lambda} (\Gamma, \emptyset, \Sigma_2)$.
	\end{itemize}
\end{definition}

For notational convenience we can write
$\Gamma; \emptyset; \Sigma_1 \by{\lambda} \Sigma_2 \proves P_1 \by{\lambda} P_2$,
instead of $\Gamma; \emptyset; \Sigma_1 \proves P_1 \hastype \Proc \by{\lambda} \Gamma; \emptyset; \Sigma_1 \proves P_2 \hastype \Proc$.
We extend to $\By{}$ and $\By{\hat{\lambda}}$ in the \dk{standard way}.

The next invariant clarifies the soundness of the
typed transition system.

\begin{lemma}[Invariant]
	\begin{itemize}
		\item	If $\Gamma; \emptyset; \Sigma_1 \proves P_1 \hastype \Proc$ and
			$P_1 \by{\lambda} P_2$ and $(\Gamma; \emptyset; \Sigma_1) \by{\lambda} (\Gamma; \emptyset; \Sigma_2)$
			then $\Gamma; \emptyset; \Sigma_2 \proves P_2 \hastype \Proc$.
	\end{itemize}
\end{lemma}

\begin{proof}
	\dk{TODO}
\end{proof}

\subsection{Behavioural Semantics}

We use the typed labelled transition semantics to define
a set of relations over typed processes that allow us to compare
typed processes over a notion of observational equivalence.


We begin with a definition of a notion of confluence
over session environments $\Sigma$:
\begin{definition}[Session Environment Confluence]\rm
	We denote $\Sigma_1 \bistyp \Sigma_2$ whenever $\exists \Sigma$ such that
	$\Sigma_1 \red^* \Sigma$ and $\Sigma_2 \red^* \Sigma$.
\end{definition}

%\jp{The following definition is a bit too "loose". Need to add conditions on $\Sigma_1,\Sigma_2$, and a better notation not involving the empty $\Lambda$.}

A typed relation is a relation over typed programs:

\begin{definition}[Typed Relation]\rm
	We say that
	$\Gamma; \emptyset; \Sigma_1 \proves P_1 \hastype \Proc\ R\ \Gamma; \emptyset; \Sigma_2 \proves P_2 \hastype \Proc$
	is a typed relation whenever
	\begin{itemize}
		\item	$P_1$ and $P_2$ are programs.
		\item	$\Sigma_1$ and $\Sigma_2$ are well typed.
		\item	$\Sigma_1 \bistyp \Sigma_2$.
	\end{itemize}
\end{definition}

We require that relate only programs (i.e.\ processes with no free variables) with
well typed session environments and furthermore we require that two related processes
have confluent session environments.

For notational convenience we write $\Gamma; \emptyset; \Sigma_1\ R\ \Sigma_2 \proves P_1\ R\ P_2$
for expressing the typed relation $\Gamma; \emptyset; \Sigma_1 \proves P_1 \hastype \Proc\ R\ \Gamma; \emptyset; \Sigma_2 \proves P_2 \hastype \Proc$.

We define the notions of barb and typed barb.

\begin{definition}[Barbs]\rm
	Let program $P$.
	\begin{enumerate}
%		\item	We write $P \barb{s}$ if $P \scong \newsp{\tilde{s}}{\bout{s}{\abs{x} P_1} P_2 \Par P_3}, s \notin \tilde{s}$.
%			We write $P \Barb{s}$ if $P \red^* \barb{s}$.

		\item	We write $P \barb{s}$ if $P \scong \newsp{\tilde{s}}{\bout{s}{V} P_2 \Par P_3}, s \notin \tilde{s}$.
			We write $P \Barb{s}$ if $P \red^* \barb{s}$.

		\item	We write $\Gamma; \emptyset; \Sigma \proves P \barb{s}$ if
			$\Gamma; \emptyset; \Sigma \proves P \hastype \Proc$ with $P \barb{s}$ and $\dual{s} \notin \Sigma$.
			We write $\Gamma; \emptyset; \Sigma \proves P \Barb{s}$ if $P \red^* P'$ and
			$\Gamma; \emptyset; \Sigma' \proves P' \barb{s}$.			
	\end{enumerate}
\end{definition}

A barb $\barb{s}$ is an observable on an output prefix with subject $s$.
Similarly a weak barb $\Barb{s}$ is a barb after a number of reduction steps.
Typed barbs $\barb{s}$ (resp.\ $\Barb{s}$)
happen on typed processes $\Gamma; \emptyset; \Sigma \proves P \hastype \Proc$
where we require that the corresponding dual endpoint $\dual{s}$ is not present
in the session type $\Sigma$.

We define the notion of the context:

\begin{definition}[Context]\rm
	$C$ is a context defined on the grammar:

	\begin{tabular}{rcl}
		$C$ &$=$& $\hole \bnfbar P \bnfbar \bout{k}{V} C \bnfbar \binp{k}{X} C \bnfbar \binp{k}{x} C \bnfbar \news{s} C \bnfbar C \Par C \bnfbar \bsel{k}{l} C \bnfbar \bbra{k}{l_i:C_i}_{i \in I}$
	\end{tabular}
	Notation $\context{C}{P}$ replaces every $\hole$ in $C$ with $P$.
\end{definition}

A context is a function that takes a process and returns a new process
according to the above syntax.

%We extend the notion of context to the notion of typed context:
%\begin{definition}[Typed Context]
%	Let program $\Gamma; \emptyset; \Sigma \proves P \hastype \Proc$ then	
%\end{definition}

The first equivalence relation we define is reduction-closed, barbed congruence:
\begin{definition}[Reduction-closed, Barbed Congruence]\rm
	Typed relation $\Gamma; \emptyset; \Sigma_1\ R\ \Sigma \proves P_1 \ R\ P_2$ is a barbed congruence
	whenever:
	\begin{enumerate}
		\item
		\begin{itemize}
			\item	If $P_1 \red P_1'$ then $\exists P_2', P_2 \red^* P_1'$ and $\Gamma; \emptyset; \Sigma_1' \proves P_1'\ R\ \Gamma; \emptyset; \Sigma_2' \proves P_2' \hastype \Proc$.
			\item	If $P_2 \red P_2'$ then $\exists P_1', P_1 \red^* P_1'$ and $\Gamma; \emptyset; \Sigma_1' \proves P_1'\ R\ \Gamma; \emptyset; \Sigma_2' \proves P_2' \hastype \Proc$.
		\end{itemize}
		\item
		\begin{itemize}
			\item	If $\Gamma;\emptyset;\Sigma \proves P_1 \barb{s}$ then $\Gamma;\emptyset;\Sigma \proves P_2 \Barb{s}$.
			\item	If $\Gamma;\emptyset;\Sigma \proves P_2 \barb{s}$ then $\Gamma;\emptyset;\Sigma \proves P_1 \Barb{s}$.
		\end{itemize}
		\item	$\forall C, \Gamma; \emptyset; \Sigma_1'\ R\ \Sigma_2' \proves \context{C}{P_1}\ R\ \context{C}{P_2}$.
	\end{enumerate}
	The largest such congruence is denoted with $\cong$.
\end{definition}

Reduction-closed, barbed congruence has closed reduction semantics and 
preserves barbs under any context. In a sense no barb observer can distinguish
between two related processes.

We can use a session type to define the simplest process that is typed
under the given session type.
\begin{definition}[Simple Process]\rm
	Let type $U = S$ or $U = \chtype{S}$ and name $k$; then we define a process $\map{U}^{k}$:
	\[
	\begin{array}{rcl}
		\map{\tinact}^{k} &=& \inact \\
		\map{\btinp{S'} S}^{k} &=& \binp{k}{x} (\map{S}^{k} \Par \map{S'}^{x}) \\
		\map{\btout{U} S}^{k} &=& \binp{k}{s} \map{S}^{k} \quad s\textrm{ fresh}\\
		\map{\btsel{l : S}}^{k} &=& \bsel{k}{l} \map{S}^{k} \\
		\map{\btbra{l_i: S_i}_{i \in I}}^{k} &=& \bbra{k}{l_i: \map{S_i}^{k}}_{i \in I}\\
		\map{\tvar{t}}^{k} &=& \rvar{T} \\
		\map{\trec{t}{S}}^{k} &=& \recp{T}{\map{S}^{k}}\\
		\map{\btout{\lhot{S'}} S}^{k} &=& \bout{k}{\abs{x} \map{S'}^x} \map{S}^k\\
		\map{\btinp{\lhot{S'}} S}^{k} &=& \binp{k}{X} (\map{S}^k \Par \appl{X}{s}) \quad s\textrm{ fresh}\\
		\map{\btout{\chtype{S}} S}^{k} &=& \bout{k}{a} \map{S}^k  \quad a\textrm{ fresh}\\
		\map{\btinp{\chtype{S}} S}^{k} &=& \binp{k}{x} (\map{S}^k \Par \map{\chtype{S}}^y)\\
		\map{\chtype{S}}^{k} &=& \binp{k}{x} \map{S}^x
	\end{array}
	\] 
\end{definition}

\begin{proposition}\rm
	\begin{itemize}
		\item	$\Gamma; \emptyset; \Sigma \cat k:S \proves \map{S}^k \hastype \Proc$
		\item	$\Gamma \cat k:\chtype{S}; \emptyset; \Sigma \proves \map{\chtype{S}}^k \hastype \Proc$
	\end{itemize}
\end{proposition}

\begin{proof}
	By induction on the definition $\map{S}^k$ and $\map{\chtype{S}}^k$.
\end{proof}

The second equivalent relation is a bisimulation relation called
contextual bisimulation:
\begin{definition}[Contextual Bisimulation]\rm
	Let typed relation $\mathcal{R}$ such that $\Gamma; \emptyset; \Sigma_1\ \mathcal{R}\ \Sigma_2 \proves P_1\ \mathcal{R}\ Q_1$.
	$\mathcal{R}$ is a {\em contextual bisimulation} whenever:
	\begin{enumerate}
		\item	$\forall \news{\tilde{s}} \bactout{s}{\abs{x} P}$ such that
			\[
				\Gamma; \emptyset; \Sigma_1 \by{\news{\tilde{s}} \bactout{s}{\abs{x} P}} \Sigma_1' \proves P_1 \by{\news{\tilde{s}} \bactout{s}{\abs{x} P}} P_2
			\]
			$\exists Q_2, \abs{x}{Q}$ such that
			\[
				\Gamma; \emptyset; \Sigma_2 \By{\news{\tilde{s'}} \bactout{s}{\abs{x} Q}} \Sigma_2' \proves Q_1 \By{\news{\tilde{s'}} \bactout{s}{\abs{x} Q}} Q_2
			\]
			and $\forall C, s'$, %such that
%			\begin{eqnarray*}
%				\Gamma; \emptyset; \Sigma_1'' \proves \newsp{\tilde{s}}{P_2 \Par \context{C}{P \subst{s'}{x}}} \hastype \Proc \\
%				\Gamma; \emptyset; \Sigma_2'' \proves \newsp{\tilde{s}}{Q_2 \Par \context{C}{Q \subst{s'}{x}}} \hastype \Proc
%			\end{eqnarray*}
			then
			\[
				\Gamma; \emptyset; \Sigma_1''\ \mathcal{R}\ \Sigma_2'' \proves \newsp{\tilde{s}}{P_2 \Par \context{C}{P \subst{s'}{x}}}\ \mathcal{R}\ 
				\newsp{\tilde{s'}}{Q_2 \Par \context{C}{Q \subst{s'}{x}}}
			\]
		\item	$\forall \news{\tilde{s}} \bactout{s}{s_1}$ such that
			\[
				\Gamma; \emptyset; \Sigma_1 \by{\news{\tilde{s}} \bactout{s}{s_1}} \Sigma_1' \proves P_1 \by{\news{\tilde{s}} \bactout{s}{s_1}} P_2
			\]
			$\exists Q_2, s_2$ such that
			\[
				\Gamma; \emptyset; \Sigma_2 \By{\news{\tilde{s'}} \bactout{s}{s_2}} \Sigma_2' \proves Q_1 \By{\news{\tilde{s'}} \bactout{s}{s_2}} Q_2
			\]
			and $\forall R$ with $\set{x} = \fn{R}$, %such that
%			\begin{eqnarray*}
%				\Gamma; \emptyset; \Sigma_1'' \proves \newsp{\tilde{s}}{P_2 \Par \context{C}{P \subst{s'}{x}}} \hastype \Proc \\
%				\Gamma; \emptyset; \Sigma_2'' \proves \newsp{\tilde{s}}{Q_2 \Par \context{C}{Q \subst{s'}{x}}} \hastype \Proc
%			\end{eqnarray*}
			then
			\[
				\Gamma; \emptyset; \Sigma_1''\ \mathcal{R}\ \Sigma_2'' \proves \newsp{\tilde{s}}{P_2 \Par R \subst{s_1}{x}}\ \mathcal{R}\ 
				\newsp{\tilde{s'}}{Q_2 \Par R \subst{s_2}{x}}
			\]

		\item	$\forall \lambda \notin \set{\news{\tilde{s}} \bactout{s}{s'}, \news{\tilde{s}} \bactout{s}{\abs{x} P}}s$ such that
			\[
				\Gamma; \emptyset; \Sigma_1 \by{\lambda} \Sigma_1' \proves P_1 \by{\lambda} P_2
			\]
			$\exists Q_2$ such that 
			\[
				\Gamma; \emptyset; \Sigma_2 \by{\lambda} \Sigma_2' \proves Q_1 \By{\hat{\lambda}} Q_2
			\]
			and
			$\Gamma; \emptyset; \Sigma_1\ \mathcal{R}\ \Sigma_2 \proves P_2\ \mathcal{R}\ Q_2$.

		\item	The symmetric cases of 1, 2 and 3.
	\end{enumerate}
	The Knaster Tarski theorem ensures that the largest contextual bisimulation exists, it is called contextual bisimilarity and is denoted by $\wb^c$.
\end{definition}

The contextual bisimulation in the general case is a hard relation to be computed
since it is quantified over every context and every process in the case of 
$\bactout{s}{\abs{x} P}$ and $\bactout{s}{s'}$ respectively. The next definition
of a bisimulation relation avoids these two universal quantifiers.


\begin{definition}[Bisimulation]\rm
	Let typed relation $\mathcal{R}$ such that $\Gamma; \emptyset; \Sigma_1\ \mathcal{R}\ \Sigma_2 \proves P_1\ \mathcal{R}\ Q_1$.
	$\mathcal{R}$ is a {\em bisimulation} whenever:
	\begin{enumerate}
		\item	$\forall \news{\tilde{s}} \bactout{s}{\abs{x} P}$ such that
			\[
				\Gamma; \emptyset; \Sigma_1 \by{\news{\tilde{s}} \bactout{s}{\abs{x} P}} \Sigma_1' \proves P_1 \by{\news{\tilde{s}} \bactout{s}{\abs{x} P}} P_2
			\]
			$\exists Q_2, \abs{x}{Q}$ such that
			\[
				\Gamma; \emptyset; \Sigma_2 \By{\news{\tilde{s'}} \bactout{s}{\abs{x} Q}} \Sigma_2' \proves Q_1 \By{\news{\tilde{s'}} \bactout{s}{\abs{x} Q}} Q_2
			\]
			and $\forall s'$, %such that
%			\begin{eqnarray*}
%				\Gamma; \emptyset; \Sigma_1'' \proves \newsp{\tilde{s}}{P_2 \Par \context{C}{P \subst{s'}{x}}} \hastype \Proc \\
%				\Gamma; \emptyset; \Sigma_2'' \proves \newsp{\tilde{s}}{Q_2 \Par \context{C}{Q \subst{s'}{x}}} \hastype \Proc
%			\end{eqnarray*}
			then
			\[
				\Gamma; \emptyset; \Sigma_1''\ \mathcal{R}\ \Sigma_2'' \proves \newsp{\tilde{s}}{P_2 \Par P \subst{s'}{x}}\ \mathcal{R}\ 
				\newsp{\tilde{s'}}{Q_2 \Par Q \subst{s'}{x}}
			\]
		\item	$\forall \news{\tilde{s}} \bactout{s}{s_1}$ such that
			\[
				\Gamma; \emptyset; \Sigma_1 \by{\news{\tilde{s}} \bactout{s}{s_1}} \Sigma_1' \proves P_1 \by{\news{\tilde{s}} \bactout{s}{s_1}} P_2
			\]
			with $s_1: S \in \Sigma_1 \vee (\dual{s_2}: S' \in \Sigma_2 \wedge S \dualof S')$
			then $\exists Q_2, s_2$ such that
			\[
				\Gamma; \emptyset; \Sigma_2 \By{\news{\tilde{s'}} \bactout{s}{s_2}} \Sigma_2' \proves Q_1 \By{\news{\tilde{s'}} \bactout{s}{s_2}} Q_2
			\]
			%such that
%			\begin{eqnarray*}
%				\Gamma; \emptyset; \Sigma_1'' \proves \newsp{\tilde{s}}{P_2 \Par \context{C}{P \subst{s'}{x}}} \hastype \Proc \\
%				\Gamma; \emptyset; \Sigma_2'' \proves \newsp{\tilde{s}}{Q_2 \Par \context{C}{Q \subst{s'}{x}}} \hastype \Proc
%			\end{eqnarray*}
			and
			\[
				\Gamma; \emptyset; \Sigma_1''\ \mathcal{R}\ \Sigma_2'' \proves \newsp{\tilde{s}}{P_2 \Par \map{S}^{s_1}}\ \mathcal{R}\ 
				\newsp{\tilde{s'}}{Q_2 \Par \map{S}^{s_2}}
			\]

		\item	$\forall \lambda \notin \set{\news{\tilde{s}} \bactout{s}{s'}, \news{\tilde{s}} \bactout{s}{\abs{x} P}}s$ such that
			\[
				\Gamma; \emptyset; \Sigma_1 \by{\lambda} \Sigma_1' \proves P_1 \by{\lambda} P_2
			\]
			$\exists Q_2$ such that 
			\[
				\Gamma; \emptyset; \Sigma_2 \by{\lambda} \Sigma_2' \proves Q_1 \By{\hat{\lambda}} Q_2
			\]
			and
			$\Gamma; \emptyset; \Sigma_1\ \mathcal{R}\ \Sigma_2 \proves P_2\ \mathcal{R}\ Q_2$.

		\item	The symmetric cases of 1, 2 and 3.
	\end{enumerate}
	The Knaster Tarski theorem ensures that the largest bisimulation exists, it is called bisimilarity and is denoted by $\wb$.
\end{definition}

The next theorem justifies our definition choices
for both of the bisimulation relations, since
they coincide between them and they also
coincide with reduction closed, barbed congruence.

\begin{theorem}[Coincidence]\rm
	\label{the:coincidence}
	$\wb\ =\ \wbc$ and $\cong\ =\ \wbc$.
\end{theorem}

\begin{proof}
	\dk{add sketch}
\end{proof}

Session types are inherintly $\tau$-inert.

\begin{lemma}[$\tau$-inertness]
	\label{lem:tau_inert}
	Let $P$ an $\HOp$ process
	and $\Gamma; \emptyset; \Sigma \proves P \hastype \Proc$
	\begin{enumerate}
		\item	If $P \red P'$ then $\Gamma; \empty; \Sigma \wb \Sigma' \proves P \wb P'$.
		\item	If $P \red^* P'$ then $\Gamma; \empty; \Sigma \wb \Sigma' \proves P \wb P'$.
	\end{enumerate}
\end{lemma}

\begin{proof}
	The proof for part 1 is done by induction on the structure of $\red$.


%	The proof is a double induction first on the length and then on the structure of $\red$.
%	Induction on the length of $\red$.

%	The basic step is trivial
%	Induction hypothesis:
%	If $P \red^* P''$ then
%	$\Gamma; \Lambda; \Sigma \proves P \wbc \Gamma''; \Lambda''; \Sigma'' \proves P'' \hastype \Proc$.
%
%	Induction step:
%	Let $P'' \red P'$. We do an induction on the structure of $\red$
%	to show that $\Gamma'; \Lambda'; \Sigma' \proves P'' \wbc \Gamma'; \Lambda'; \Sigma' \proves P' \hastype \Proc$.
%
	Basic step:

	\Case{$P = \bout{s}{V} P_1 \Par \binp{\dual{s}}{X} P_2$}
	\[
		\Gamma; \emptyset; \Sigma \red \Sigma' \proves \bout{s}{V} P_1 \Par \binp{\dual{s}}{X} P_2 \red P_1 \Par P_2
	\]
	The proof follows from the fact that we can only observe a $\tau$
	action on typed process
	$\Gamma; \emptyset; \Sigma \proves P \hastype \Proc$.
	Actions $\bactout{s}{V}$ and $\bactinp{\dual{s}}{V}$
	are forbiden by the LTS for typed environments.

	It is easy to conclude then that $\Gamma; \emptyset; \Sigma \wb \Sigma' \proves P \wb P'$.

	\Case{$P = \bsel{s}{l} P_1 \Par \bbra{\dual{s}}{l_i: P_i}_{i \in I}$}

	Similar arguments as the previous case.

	Induction hypothesis:
	If $P_1 \red P_2$ then
	$\Gamma_1; \emptyset; \Sigma_1 \wb \Sigma_2 \proves P_1 \wb P_2$.

	Induction Step:

	\Case{$P = \news{s} P_1$}
	\[
		\Gamma; \emptyset; \Sigma \red \Sigma' \proves \news{s}{P_1} \red \news{s} P_2
	\]
	From the induction hypothesis and the fact that bisimulation is a congruence
	we get that $\Gamma; \emptyset; \Sigma \wb \Sigma' \proves P \wb P'$.

	\Case{$P = P_1 \Par P_3$}

	\[
		\Gamma; \emptyset; \Sigma \red \Sigma' \proves P_1 \Par P_3 \red P_2 \Par P_3
	\]
	From the induction hypothesis and the fact that bisimulation is a congruence
	we get that $\Gamma; \emptyset; \Sigma \wb \Sigma' \proves P \wb P'$.

	\Case{$P \scong P_1$}

	From the induction hypothesis and the facts that bisimulation is a congruence
	and structural congruence preserves $\wb$
	we get that $\Gamma; \emptyset; \Sigma \wb \Sigma' \proves P \wb P'$.


	The proof for part two is an induction on the length of $\red^*$.
	The basic step is trivial and the inductive step
	deploys part 1 of this lemma and the fact that bisimulation is
	transitive to conclude.
%	We can now conclude that
%	$P \wbc P'$ because $P \wbc P''$ and $P'' \wbc P'$.
\end{proof}


