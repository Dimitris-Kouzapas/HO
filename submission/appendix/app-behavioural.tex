\section{Behavioural Semantics}

We present the proofs for the theorems in
Section~\ref{sec:beh_sem}.

\subsection{Proof for Theorem~\ref{the:coincidence}}

We split Theorem~\ref{the:coincidence} into 
Lemmas which we prove independently.
The combination of the lemmas is the proof for the parts
of the theorem.

The proof for Part 1 for the theorem is based on the
stratified definition of the bisimulation relations
$\wbc$ and $\wb$. The Knaster-Tarski theorem ensures
that both definitions are equivalent.
We give the stratified  definitions:

\begin{definition}[Stratified Contextual Bisimulation]\rm
	We define a set of relations $\mathcal{R}^c_n$
	on the following conditions:
%
	\begin{itemize}
		\item	$\mathcal{R}^c_0 = \bigcup_{\forall R} R, R$ is a typed relation.
		\item	$\Gamma; \emptyset; \Sigma_1\ \mathcal{R}^c_n\ \Sigma_2 \proves P_1\ \mathcal{R}^c_n\ P_2$
			whenever
			\begin{enumerate}
				\item	$\forall \news{\tilde{s}} \bactout{s}{\abs{x} P}$ such that
					\[
						\Gamma; \emptyset; \Sigma_1 \by{\news{\tilde{s}} \bactout{s}{\abs{x} P}} \Sigma_1' \proves P_1 \by{\news{\tilde{s}} \bactout{s}{\abs{x} P}} P_2
					\]
					$\exists Q_2, \abs{x}{Q}$ such that
					\[
						\Gamma; \emptyset; \Sigma_2 \By{\news{\tilde{s'}} \bactout{s}{\abs{x} Q}} \Sigma_2' \proves Q_1 \By{\news{\tilde{s}} \bactout{s}{\abs{x} Q}} Q_2
					\]
					and $\forall C, s'$
%					such that
%					\begin{eqnarray*}
%						\Gamma; \emptyset; \Sigma_1'' \proves \newsp{\tilde{s}}{P_2 \Par P \subst{s'}{x}} \hastype \Proc \\
%						\Gamma; \emptyset; \Sigma_2'' \proves \newsp{\tilde{s}}{Q_2 \Par Q \subst{s'}{x}} \hastype \Proc
%					\end{eqnarray*}
%					then
					\[
						\Gamma; \emptyset; \Sigma_1''\ \mathcal{R}^c_{n-1}\ \Sigma_2'' \proves
						\newsp{\tilde{s}}{P_2 \Par \context{C}{P \subst{s'}{x}}}\ \mathcal{R}^c_{n-1}\  \newsp{\tilde{s}}{Q_2 \Par \context{C}{Q \subst{s'}{x}}}
					\]

				\item	$\forall \news{\tilde{s}} \bactout{s}{s_1}$ such that
					\[
						\Gamma; \emptyset; \Sigma_1 \by{\news{\tilde{s}} \bactout{s}{s_1}} \Sigma_1' \proves P_1 \by{\news{\tilde{s}} \bactout{s}{s_1}} P_2
					\]
					%with $s_1: S \in \Sigma_1 \vee (\dual{s_2}: S' \in \Sigma_2 \wedge S \dualof S')$
					then $\exists Q_2, s_2$ such that
					\[
						\Gamma; \emptyset; \Sigma_2 \By{\news{\tilde{s'}} \bactout{s}{s_2}} \Sigma_2' \proves Q_1 \By{\news{\tilde{s'}} \bactout{s}{s_2}} Q_2
					\]
					%such that
		%			\begin{eqnarray*}
		%				\Gamma; \emptyset; \Sigma_1'' \proves \newsp{\tilde{s}}{P_2 \Par \context{C}{P \subst{s'}{x}}} \hastype \Proc \\
		%				\Gamma; \emptyset; \Sigma_2'' \proves \newsp{\tilde{s}}{Q_2 \Par \context{C}{Q \subst{s'}{x}}} \hastype \Proc
		%			\end{eqnarray*}
					and $\forall R, \set{x} = \fn{R}$
					\[
						\Gamma; \emptyset; \Sigma_1''\ \mathcal{R}^c_{n-1}\ \Sigma_2'' \proves \newsp{\tilde{s}}{P_2 \Par R\subst{s_1}{x}}\ \mathcal{R}^c_{n-1}\ 
						\newsp{\tilde{s'}}{Q_2 \Par R\subst{s_2}{x}}
					\]


				\item	$\forall \lambda \not= \news{\tilde{s}} \bactout{s}{\abs{x} P}$ such that
					\[
						\Gamma; \emptyset; \Sigma_1 \by{\lambda} \Sigma_1' \proves P_1 \by{\lambda} P_2
					\]
					$\exists Q_2$ such that 
					\[
						\Gamma; \emptyset; \Sigma_2 \by{\hat{\lambda}} \Sigma_2' \proves Q_1 \By{\hat{\lambda}} Q_2
					\]
					and
					$\Gamma; \emptyset; \Sigma_1'\ \mathcal{R}^c_{n-1}\ \Sigma_2' \proves P_2\ \mathcal{R}^c_{n-1}\ Q_2$.

				\item	The symmetric cases of 1, 2, 3.
			\end{enumerate}
	\end{itemize}
	\noi The above function is monotone and the Knaster-Tarski theorem ensures that a lattice is define
	with the largest $\mathcal{R}^c_n$ to be denote as $\wbc_n$ and the largest fix-point is equal to the
	contextual bisimilarity relation $\wbc = \bigcap_{i \geq 0} \wbc_i$.
\end{definition}


\begin{definition}[Stratified Bisimulation]\rm
	We define a set of relations $\mathcal{R}_n$
	on the following conditions:
%
	\begin{itemize}
		\item	$\mathcal{R}_0 = \bigcup_{\forall R} R, R$ is a typed relation.
		\item	$\Gamma; \emptyset; \Sigma_1\ \mathcal{R}_n\ \Sigma_2 \proves P_1\ \mathcal{R}_n\ P_2$
			whenever
			\begin{enumerate}
				\item	$\forall \news{\tilde{s}} \bactout{s}{\abs{x} P}$ such that
					\[
						\Gamma; \emptyset; \Sigma_1 \by{\news{\tilde{s}} \bactout{s}{\abs{x} P}} \Sigma_1' \proves P_1 \by{\news{\tilde{s}} \bactout{s}{\abs{x} P}} P_2
					\]
					$\exists Q_2, \abs{x}{Q}$ such that
					\[
						\Gamma; \emptyset; \Sigma_2 \By{\news{\tilde{s'}} \bactout{s}{\abs{x} Q}} \Sigma_2' \proves Q_1 \By{\news{\tilde{s}} \bactout{s}{\abs{x} Q}} Q_2
					\]
					and $\forall s'$
%					such that
%					\begin{eqnarray*}
%						\Gamma; \emptyset; \Sigma_1'' \proves \newsp{\tilde{s}}{P_2 \Par P \subst{s'}{x}} \hastype \Proc \\
%						\Gamma; \emptyset; \Sigma_2'' \proves \newsp{\tilde{s}}{Q_2 \Par Q \subst{s'}{x}} \hastype \Proc
%					\end{eqnarray*}
%					then
					\[
						\Gamma; \emptyset; \Sigma_1''\ \mathcal{R}_{n-1}\ \Sigma_2'' \proves
						\newsp{\tilde{s}}{P_2 \Par P \subst{s'}{x}}\ \mathcal{R}_{n-1}\  \newsp{\tilde{s}}{Q_2 \Par Q \subst{s'}{x}}
					\]

				\item	$\forall \news{\tilde{s}} \bactout{s}{s_1}$ such that
					\[
						\Gamma; \emptyset; \Sigma_1 \by{\news{\tilde{s}} \bactout{s}{s_1}} \Sigma_1' \proves P_1 \by{\news{\tilde{s}} \bactout{s}{s_1}} P_2
					\]
					with $s_1: S \in \Sigma_1 \vee (\dual{s_2}: S' \in \Sigma_2 \wedge S \dualof S')$
					then $\exists Q_2, s_2$ such that
					\[
						\Gamma; \emptyset; \Sigma_2 \By{\news{\tilde{s'}} \bactout{s}{s_2}} \Sigma_2' \proves Q_1 \By{\news{\tilde{s'}} \bactout{s}{s_2}} Q_2
					\]
					%such that
		%			\begin{eqnarray*}
		%				\Gamma; \emptyset; \Sigma_1'' \proves \newsp{\tilde{s}}{P_2 \Par \context{C}{P \subst{s'}{x}}} \hastype \Proc \\
		%				\Gamma; \emptyset; \Sigma_2'' \proves \newsp{\tilde{s}}{Q_2 \Par \context{C}{Q \subst{s'}{x}}} \hastype \Proc
		%			\end{eqnarray*}
					and
					\[
						\Gamma; \emptyset; \Sigma_1''\ \mathcal{R}_{n-1}\ \Sigma_2'' \proves \newsp{\tilde{s}}{P_2 \Par \map{S}^{s_1}}\ \mathcal{R}_{n-1}\ 
						\newsp{\tilde{s'}}{Q_2 \Par \map{S}^{s_2}}
					\]


				\item	$\forall \lambda \not= \news{\tilde{s}} \bactout{s}{\abs{x} P}$ such that
					\[
						\Gamma; \emptyset; \Sigma_1 \by{\lambda} \Sigma_1' \proves P_1 \by{\lambda} P_2
					\]
					$\exists Q_2$ such that 
					\[
						\Gamma; \emptyset; \Sigma_2 \by{\hat{\lambda}} \Sigma_2' \proves Q_1 \By{\hat{\lambda}} Q_2
					\]
					and
					$\Gamma; \emptyset; \Sigma_1'\ \mathcal{R}_{n-1}\ \Sigma_2' \proves P_2\ \mathcal{R}_{n-1}\ Q_2$.

				\item	The symmetric cases of 1, 2, 3.
			\end{enumerate}
	\end{itemize}
	\noi The above function is monotone and the Knaster-Tarski theorem ensures that a lattice is define
	with the largest $\mathcal{R}_n$ to be denote as $\wb_n$ and the largest fix-point is equal to the
	bisimilarity relation $\wb = \bigcap_{i \geq 0} \wb_i$.
\end{definition}


\begin{lemma}\rm
	$\wbc\ \subseteq\ \wb$
\end{lemma}

\begin{proof}
	Statement $\wbc \subseteq \wb$
	is equivalent to the statement $\forall n, \wbc_n \subseteq \wb_n$.
	From here the proof is done using induction on the definitions of $\wbc$.

	\noi {\bf Basic step:}
	From the definitions of $\wbc_0$ and $\wb_0$ we get that $\wbc_0 = \wb_0$.

	\noi {\bf Induction hypothesis:}
	$\wbc_n\ \subseteq\ \wb_n$.

	\noi {\bf Inductive step:}
	Let
%
	\begin{eqnarray*}
		\Gamma; \emptyset; \Sigma_1\ \wbc_{n+1}\ \Sigma_2 \proves P_1\ \wbc_{n+1}\ Q_1
	\end{eqnarray*}
%
	\noi We perform a case analysis on transition $\by\lambda$.

	%%%%%%%%%%%%%%%%%%%%%%%%%%%%%%%%%%%%%%%%%%%%%%%

	\noi - Case: $\lambda \notin \set{\news{\tilde{s}} \bactout{s}{\abs{x} P}, \news{\tilde{s}} \bactout{s}{s_1}}$
%
	\begin{eqnarray}
		\Gamma; \emptyset; \Sigma_1 \by{\lambda} \Sigma_1' \proves P_1 \by{\lambda} P_2 \label{lem:wbc_is_wb1}
	\end{eqnarray}
%
	\noi implies that 
	$\exists Q_1$ such that
%
	\begin{eqnarray}
		\Gamma; \emptyset; \Sigma_2 \by{\lambda} \Sigma_2' &\proves& Q_1 \by{\lambda} Q_2 \label{lem:wbc_is_wb2}\\
		\Gamma; \emptyset; \Sigma_1'\ \wbc_n\ \Sigma_2' &\proves& P_1\ \wbc_n\ Q_2
	\end{eqnarray}
%
	We apply the induction hypothesis to the latter judgement:
%
	\begin{eqnarray}
		\Gamma; \emptyset; \Sigma_1' \proves P_2\ \wb_n\ \Gamma; \emptyset; \Sigma_2' \proves Q_2 \hastype \Proc  \label{lem:wbc_is_wb3}
	\end{eqnarray}
%
	Assume $\mathcal{R}_{n+1} = \set{\Gamma; \emptyset; \Sigma_1 \proves P_1 \hastype \Proc, \Gamma; \emptyset; \Sigma_2 \proves Q_1 \hastype \Proc}$.

	\noi $\mathcal{R}_{n+1}$ satisfies the condition for the stratified definition of bisimulation
	because statement~\ref{lem:wbc_is_wb1} implies $\exists Q'$ such that
	statements~\ref{lem:wbc_is_wb2} holds and furthermore statement~\ref{lem:wbc_is_wb3} holds
%	if $\Gamma; \emptyset; \Sigma_1 \proves P \by{\lambda} \Gamma; \emptyset; \Sigma_1' \proves P' \hastype \Proc$ then
%	$\exists Q'$ such that
%	$\Gamma; \emptyset; \Sigma_2 \proves Q \by{\lambda} \Gamma; \emptyset; \Sigma_2' \proves Q' \hastype \Proc$
%	and \ref{pr:biscong_is_bis2}.

	\noi Because $\wb_{n+1}$ is the largest relation we get that $\mathcal{R}_{n+1} \subseteq \wb_{n+1}$ as required.

	%%%%%%%%%%%%%%%%%%%%%%%%%%%%%%%%%%%%%%%%%%%%%%%

	\noi - Case: $\lambda = \news{\tilde{s}} \bactout{s}{\abs{x} P}$
%
	\begin{eqnarray}
		\Gamma; \emptyset; \Sigma_1 \by{\news{\tilde{s}} \bactout{s}{\abs{x} P}} \Sigma_1' \proves P_1 \by{\news{\tilde{s}} \bactout{s}{\abs{x} P}} P_2 \label{lem:wbc_is_wb4}
	\end{eqnarray}
%
	\noi implies that
	$\exists Q_2, \abs{x}{Q}$ such that
	\begin{eqnarray}
		\Gamma; \emptyset; \Sigma_2 \By{\news{\tilde{s'}} \bactout{s}{\abs{x} Q}} \Sigma_2' \proves Q_1 \By{\news{\tilde{s'}} \bactout{s}{\abs{x} Q}} Q_2  \label{lem:wbc_is_wb5}
	\end{eqnarray}
	and $\forall C, s'$
%	such that
%	\begin{eqnarray*}
%		\Gamma; \emptyset; \Sigma_1'' \proves \newsp{\tilde{s}}{\context{C}{P' \Par P \subst{s'}{x}}} \hastype \Proc \\
%		\Gamma; \emptyset; \Sigma_2'' \proves \newsp{\tilde{s}}{\context{C}{Q' \Par Q \subst{s'}{x}}} \hastype \Proc
%	\end{eqnarray*}
%	then
%
	\begin{eqnarray*}
		\Gamma; \emptyset; \Sigma_1''\ \wbc_{n}\ \Sigma_2'' \proves \newsp{\tilde{s}}{\context{C}{P_2 \Par P \subst{s'}{x}}}\ \wbc_{n}\ 
		\newsp{\tilde{s'}}{\context{C}{Q_2 \Par Q \subst{s'}{x}}}
	\end{eqnarray*}
%
	\noi For $C = \hole$ we have that 
%
	\begin{eqnarray*}
		\Gamma; \emptyset; \Sigma_1''\ \wbc_{n}\ \Sigma_2'' \proves \newsp{\tilde{s}}{P_2 \Par P \subst{s'}{x}}\ \wbc_{n}\ 
		\newsp{\tilde{s}}{Q' \Par Q_2 \subst{s'}{x}}
	\end{eqnarray*}
%
	\dk{(prove that it is typable)}

	\noi If we apply the induction hypothesis to the latter statement we get
%
	\begin{eqnarray}
		\Gamma; \emptyset; \Sigma_1''\ \wb_{n}\ \Sigma_2'' \proves \newsp{\tilde{s}}{P_2 \Par P \subst{s'}{x}}\ \wb_{n}\ 
		\newsp{\tilde{s}}{Q_2 \Par Q \subst{s'}{x}}
		\label{lem:wbc_is_wb6}
	\end{eqnarray}
%
	\noi Assume $\mathcal{R}_{n+1} = \set{\Gamma; \emptyset; \Sigma_1 \proves P_1 \hastype \Proc, \Gamma; \emptyset; \Sigma_2 \proves Q_2 \hastype \Proc}$.

	\noi $\mathcal{R}_{n+1}$ satisfies the condition for the stratified definition for bisimulation
	because statement~\ref{lem:wbc_is_wb4} implies that
	$\exists Q', \abs{x}{Q}$ such that
	statement~\ref{lem:wbc_is_wb5} holds and furthemore statement~\ref{lem:wbc_is_wb6} holds.

	\noi Because $\wb_{n+1}$ is the largest relation we get that $\mathcal{R}_{n+1} \subseteq \wb_{n+1}$ as required.

	%%%%%%%%%%%%%%%%%%%%%%%%%%%%%%%%%%%%%%%%%%%%%%%

	\noi - Case: $\lambda = \news{\tilde{s}} \bactout{s}{s_1}$
%
	\begin{eqnarray}
		\Gamma; \emptyset; \Sigma_1 \by{\news{\tilde{s}} \bactout{s}{s_1}} \Sigma_1' \proves P_1 \by{\news{\tilde{s}} \bactout{s}{s_1}} P_2 \label{lem:wbc_is_wb7}
	\end{eqnarray}
%
	\noi \dk{with $s_1: S \in \Sigma_1 \vee (\dual{s_1}: S' \in \Sigma_1' \wedge S \dualof S')$} implies that
	$\exists Q', s_2$ such that
	\begin{eqnarray}
		\Gamma; \emptyset; \Sigma_2 \By{\news{\tilde{s'}} \bactout{s}{s_2}} \Sigma_2' \proves Q_1 \By{\news{\tilde{s'}} \bactout{s}{s_2}} Q_2 \label{lem:wbc_is_wb8}
	\end{eqnarray}
	and $\forall R$ with $\set{x} = \fn{P}$
%	such that
%	\begin{eqnarray*}
%		\Gamma; \emptyset; \Sigma_1'' \proves \newsp{\tilde{s}}{\context{C}{P' \Par P \subst{s'}{x}}} \hastype \Proc \\
%		\Gamma; \emptyset; \Sigma_2'' \proves \newsp{\tilde{s}}{\context{C}{Q' \Par Q \subst{s'}{x}}} \hastype \Proc
%	\end{eqnarray*}
%	then
%
	\begin{eqnarray*}
		\Gamma; \emptyset; \Sigma_1''\ \wbc_{n}\ \Sigma_2'' \proves \newsp{\tilde{s}}{P_2 \Par R \subst{s_1}{x}}\ \wbc_{n}\ 
		\newsp{\tilde{s'}}{Q_2 \Par R \subst{s_2}{x}}
	\end{eqnarray*}
%
	\noi From the latter statement we get
	\begin{eqnarray*}
		\Gamma; \emptyset; \Sigma_1''\ \wbc_{n}\ \Sigma_2'' \proves \newsp{\tilde{s}}{P_2 \Par \map{S}^{x} \subst{s_1}{x}}\ \wbc_{n}\ 
		\newsp{\tilde{s'}}{Q_2 \Par \map{S}^{x} \subst{s_2}{x}} \label{lem:wbc_is_wb9}
	\end{eqnarray*}
%
	\noi Assume $\mathcal{R}_{n+1} = \set{\Gamma; \emptyset; \Sigma_1 \proves P_1 \hastype \Proc, \Gamma; \emptyset; \Sigma_2 \proves Q_1 \hastype \Proc}$.

	\noi $\mathcal{R}_{n+1}$ satisfies the condition for the stratified definition for bisimulation
	because statement~\ref{lem:wbc_is_wb7} implies that
	$\exists Q', s_2$ such that
	statement~\ref{lem:wbc_is_wb8} holds and furthemore statement~\ref{lem:wbc_is_wb9} holds.
\end{proof}

\begin{lemma}\rm
	$\wb\ \subseteq\ \wbc$
\end{lemma}

\begin{proof}
	The statement $\wb \subseteq \wbc$.
	is equivalent to the statement $\forall n, \wb_n \subseteq \wbc_n$.
	The proof is done using induction on the definition $\wb$.

	\noi {\bf Basic step:} From the definitions of $\wbc_0$ and $\wb_0$ we get that $\wbc_0 = \wb_0$.

	\noi {\bf Induction hypothesis:} $\wb_n \subseteq \wbc_n$.

	\noi {\bf Inductive step:}
	Let 
	\[
		\Gamma; \emptyset; \Sigma_1\ \wb_{n+1}\ \Sigma_2 \proves P_1\ \wb_{n+1}\ Q_1
	\]
	We perform a case analysis on transition $\by{\lambda}$.

	\noi - Case: $\lambda \notin \set{\news{\tilde{s}} \bactout{s}{\abs{x} P}, \news{\tilde{s}} \bactout{s}{s_1}}$

	\noi Same arguments with the same case of the direction $\wbc \subseteq \wb$

	\noi - Case: $\lambda = \news{\tilde{s}} \bactout{s}{\abs{x} P}$
%
	\begin{eqnarray*}
		\Gamma; \emptyset; \Sigma_1 \by{\news{\tilde{s}} \bactout{s}{\abs{x} P}} \Sigma_1' \proves P_1 \by{\news{\tilde{s}} \bactout{s}{\abs{x} P}} P_2
	\end{eqnarray*}
%
	implies that
	$\exists Q_2, \abs{x}{Q}$ such that
	$\Gamma; \emptyset; \Sigma_2 \proves Q \by{\news{\tilde{s}} \bactout{s}{\abs{x} Q}} \Gamma; \emptyset; \Sigma_2' \proves Q' \hastype \Proc$
	and $s'$
	such that
	\begin{eqnarray*}
		\Gamma; \emptyset; \Sigma_1'' \proves \newsp{\tilde{s}}{P' \Par P \subst{s'}{x}} \hastype \Proc \\
		\Gamma; \emptyset; \Sigma_2'' \proves \newsp{\tilde{s}}{Q' \Par Q \subst{s'}{x}} \hastype \Proc
	\end{eqnarray*}
	then
	\[
		\Gamma; \emptyset; \Sigma_1'' \proves \newsp{\tilde{s}}{P' \Par P \subst{s'}{x}}\ \wb_{n}\ 
		\Gamma; \emptyset; \Sigma_2'' \proves \newsp{\tilde{s}}{Q' \Par Q \subst{s'}{x}} \hastype \Proc
	\]

	Let
	\[
		\begin{array}{rcl}
			R_{n + 1} &=& \set{	(\Gamma; \emptyset; \Sigma_1 \proves \newsp{\tilde{s_1}}{\bout{s}{\abs{x}{\context{C}{P}}} P''} \hastype \Proc,
						\Gamma; \emptyset; \Sigma_2 \proves \newsp{\tilde{s_2}}{\bout{s}{\abs{x}{\context{C}{P}}} Q''} \hastype \Proc) \setbar\\
				& &		\forall C, P' \scong \news{\tilde{s_1}} P'', Q' \scong \news{\tilde{s_2}} Q''}
		\end{array}
	\]

	\dk{prove that $R_{n+1}$ is typable}

	$R_{n+1}$ satisfies the stratified definition for bisimulation and furthermore we can deduce that
	$\forall C, s'$
	such that
	\begin{eqnarray*}
		\Gamma; \emptyset; \Sigma_1'' \proves \newsp{\tilde{s}}{\context{C}{P' \Par P \subst{s'}{x}}} \hastype \Proc \\
		\Gamma; \emptyset; \Sigma_2'' \proves \newsp{\tilde{s}}{\context{C}{Q' \Par Q \subst{s'}{x}}} \hastype \Proc
	\end{eqnarray*}
	then
	\[
		\Gamma; \emptyset; \Sigma_1'' \proves \newsp{\tilde{s}}{\context{C}{P' \Par P \subst{s'}{x}}}\ \wb_{n}\ 
		\Gamma; \emptyset; \Sigma_2'' \proves \newsp{\tilde{s}}{\context{C}{Q' \Par Q \subst{s'}{x}}} \hastype \Proc
	\]

	From the induction hypothesis we get that
	$\forall C, s'$
	such that
	\begin{eqnarray*}
		\Gamma; \emptyset; \Sigma_1'' \proves \newsp{\tilde{s}}{\context{C}{P' \Par P \subst{s'}{x}}} \hastype \Proc \\
		\Gamma; \emptyset; \Sigma_2'' \proves \newsp{\tilde{s}}{\context{C}{Q' \Par Q \subst{s'}{x}}} \hastype \Proc
	\end{eqnarray*}
	then
	\begin{eqnarray}
		\Gamma; \emptyset; \Sigma_1'' \proves \newsp{\tilde{s}}{\context{C}{P' \Par P \subst{s'}{x}}}\ \wbc_{n}\ 
		\Gamma; \emptyset; \Sigma_2'' \proves \newsp{\tilde{s}}{\context{C}{Q' \Par Q \subst{s'}{x}}} \hastype \Proc
		\label{pr:bis_is_contextbis}
	\end{eqnarray}

%	For $C = \hole$ we have that 
%	\[
%		\Gamma; \emptyset; \Sigma_1'' \proves \newsp{\tilde{s}}{P_2 \Par P \subst{s'}{x}}\ \wbc_{n}\ 
%		\Gamma; \emptyset; \Sigma_2'' \proves \newsp{\tilde{s}}{Q_2 \Par Q \subst{s'}{x}} \hastype \Proc
%	\]

	Assume $R^c_{n+1} = \set{\Gamma; \emptyset; \Sigma_1 \proves P \hastype \Proc, \Gamma; \emptyset; \Sigma_2 \proves Q \hastype \Proc}$.

	$R^c_{n+1}$ satisfies the condition for the stratified definition of the contextual bisimulation because of the fact that
	if $\Gamma; \emptyset; \Sigma_1 \proves P \by{\news{\tilde{s}} \bactout{s}{\abs{x} P}} \Gamma; \emptyset; \Sigma_1' \proves P' \hastype \Proc$ then
	$\exists Q', \abs{x}{Q}$ such that
	$\Gamma; \emptyset; \Sigma_2 \proves Q \by{\news{\tilde{s}} \bactout{s}{\abs{x} Q}} \Gamma; \emptyset; \Sigma_2' \proves Q' \hastype \Proc$
	and \ref{pr:bis_is_contextbis}.

	Because $\wbc_{n+1}$ is the largest relation we get that $R^c_{n+1} \subseteq \wb_{n+1}$ as required.
\end{proof}

