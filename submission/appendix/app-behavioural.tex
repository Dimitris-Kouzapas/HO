\section{Behavioural Semantics}

We present the proofs for the theorems in
Section~\ref{sec:beh_sem}.

\subsection{Proof for Theorem~\ref{the:coincidence}}

We split Theorem~\ref{the:coincidence} into 
Lemmas which we prove independently.
The combination of the lemmas is the proof for the parts
of the theorem.

The proof for Part 1 for the theorem is based on the
stratified definition of the bisimulation relations
$\wbc$ and $\wb$. The Knaster-Tarski theorem ensures
that both definitions are equivalent.
We give the stratified  definitions:

%%%%%%%%%%%%%%%%%%%%%%%%%%%%%%%%%%%%%%%%%%%%%%%%%%%%%%%%%
%  Stratified Contextual Bisimulation
%%%%%%%%%%%%%%%%%%%%%%%%%%%%%%%%%%%%%%%%%%%%%%%%%%%%%%%%%

\begin{definition}[Stratified Contextual Bisimulation]\rm
	We define a set of relations $\mathcal{R}^c_n$
	on the following conditions:
%
	\begin{itemize}
		\item	$\mathcal{R}^c_0 = \bigcup_{\forall R} R, R$ is a typed relation.
		\item	$\Gamma; \emptyset; \Sigma_1\ \mathcal{R}^c_n\ \Sigma_2 \proves P_1\ \mathcal{R}^c_n\ P_2$
			whenever
			\begin{enumerate}
				\item	$\forall \news{\tilde{s}} \bactout{s}{\abs{x} P}$ such that
					\[
						\Gamma; \emptyset; \Sigma_1 \by{\news{\tilde{s}} \bactout{s}{\abs{x} P}} \Sigma_1' \proves P_1 \by{\news{\tilde{s}} \bactout{s}{\abs{x} P}} P_2
					\]
					$\exists Q_2, \abs{x}{Q}$ such that
					\[
						\Gamma; \emptyset; \Sigma_2 \By{\news{\tilde{s'}} \bactout{s}{\abs{x} Q}} \Sigma_2' \proves Q_1 \By{\news{\tilde{s}} \bactout{s}{\abs{x} Q}} Q_2
					\]
					and $\forall R$ with $\set{X} = \fpv{R}$
%					such that
%					\begin{eqnarray*}
%						\Gamma; \emptyset; \Sigma_1'' \proves \newsp{\tilde{s}}{P_2 \Par P \subst{s'}{x}} \hastype \Proc \\
%						\Gamma; \emptyset; \Sigma_2'' \proves \newsp{\tilde{s}}{Q_2 \Par Q \subst{s'}{x}} \hastype \Proc
%					\end{eqnarray*}
%					then
					\[
						\Gamma; \emptyset; \Sigma_1''\ \mathcal{R}^c_{n-1}\ \Sigma_2'' \proves
						\newsp{\tilde{s}}{P_2 \Par R\subst{(x) P}{X}}\ \mathcal{R}^c_{n-1}\  \newsp{\tilde{s}}{Q_2 \Par R\subst{(x) Q}{X}}
					\]

				\item	$\forall \news{\tilde{s}} \bactout{s}{s_1}$ such that
					\[
						\Gamma; \emptyset; \Sigma_1 \by{\news{\tilde{s}} \bactout{s}{s_1}} \Sigma_1' \proves P_1 \by{\news{\tilde{s}} \bactout{s}{s_1}} P_2
					\]
					%with $s_1: S \in \Sigma_1 \vee (\dual{s_2}: S' \in \Sigma_2 \wedge S \dualof S')$
					then $\exists Q_2, s_2$ such that
					\[
						\Gamma; \emptyset; \Sigma_2 \By{\news{\tilde{s'}} \bactout{s}{s_2}} \Sigma_2' \proves Q_1 \By{\news{\tilde{s'}} \bactout{s}{s_2}} Q_2
					\]
					%such that
		%			\begin{eqnarray*}
		%				\Gamma; \emptyset; \Sigma_1'' \proves \newsp{\tilde{s}}{P_2 \Par \context{C}{P \subst{s'}{x}}} \hastype \Proc \\
		%				\Gamma; \emptyset; \Sigma_2'' \proves \newsp{\tilde{s}}{Q_2 \Par \context{C}{Q \subst{s'}{x}}} \hastype \Proc
		%			\end{eqnarray*}
					and $\forall R, \set{x} = \fn{R}$
					\[
						\Gamma; \emptyset; \Sigma_1''\ \mathcal{R}^c_{n-1}\ \Sigma_2'' \proves \newsp{\tilde{s}}{P_2 \Par R\subst{s_1}{x}}\ \mathcal{R}^c_{n-1}\ 
						\newsp{\tilde{s'}}{Q_2 \Par R\subst{s_2}{x}}
					\]


				\item	$\forall \lambda \not= \news{\tilde{s}} \bactout{s}{\abs{x} P}$ such that
					\[
						\Gamma; \emptyset; \Sigma_1 \by{\lambda} \Sigma_1' \proves P_1 \by{\lambda} P_2
					\]
					$\exists Q_2$ such that 
					\[
						\Gamma; \emptyset; \Sigma_2 \by{\hat{\lambda}} \Sigma_2' \proves Q_1 \By{\hat{\lambda}} Q_2
					\]
					and
					$\Gamma; \emptyset; \Sigma_1'\ \mathcal{R}^c_{n-1}\ \Sigma_2' \proves P_2\ \mathcal{R}^c_{n-1}\ Q_2$.

				\item	The symmetric cases of 1, 2, 3.
			\end{enumerate}
	\end{itemize}
	\noi The above function is monotone and the Knaster-Tarski theorem ensures that a lattice is define
	with the largest $\mathcal{R}^c_n$ to be denote as $\wbc_n$ and the largest fix-point is equal to the
	contextual bisimilarity relation $\wbc = \bigcap_{i \geq 0} \wbc_i$.
\end{definition}

%%%%%%%%%%%%%%%%%%%%%%%%%%%%%%%%%%%%%%%%%%%%%%%%%%%%%%%%%
%  Stratified Bisimulation
%%%%%%%%%%%%%%%%%%%%%%%%%%%%%%%%%%%%%%%%%%%%%%%%%%%%%%%%%

\begin{definition}[Stratified Bisimulation]\rm
	We define a set of relations $\mathcal{R}_n$
	on the following conditions:
%
	\begin{itemize}
		\item	$\mathcal{R}_0 = \bigcup_{\forall R} R, R$ is a typed relation.
		\item	$\Gamma; \emptyset; \Sigma_1\ \mathcal{R}_n\ \Sigma_2 \proves P_1\ \mathcal{R}_n\ P_2$
			whenever
			\begin{enumerate}
				\item	$\forall \news{\tilde{s}} \bactout{s}{\abs{x} P}$ such that
					\[
						\Gamma; \emptyset; \Sigma_1 \by{\news{\tilde{s}} \bactout{s}{\abs{x} P}} \Sigma_1' \proves P_1 \by{\news{\tilde{s}} \bactout{s}{\abs{x} P}} P_2
					\]
					$\exists Q_2, \abs{x}{Q}$ such that
					\[
						\Gamma; \emptyset; \Sigma_2 \By{\news{\tilde{s'}} \bactout{s}{\abs{x} Q}} \Sigma_2' \proves Q_1 \By{\news{\tilde{s}} \bactout{s}{\abs{x} Q}} Q_2
					\]
					and $\forall s'$
%					such that
%					\begin{eqnarray*}
%						\Gamma; \emptyset; \Sigma_1'' \proves \newsp{\tilde{s}}{P_2 \Par P \subst{s'}{x}} \hastype \Proc \\
%						\Gamma; \emptyset; \Sigma_2'' \proves \newsp{\tilde{s}}{Q_2 \Par Q \subst{s'}{x}} \hastype \Proc
%					\end{eqnarray*}
%					then
					\[
						\Gamma; \emptyset; \Sigma_1''\ \mathcal{R}_{n-1}\ \Sigma_2'' \proves
						\newsp{\tilde{s}}{P_2 \Par P \subst{s'}{x}}\ \mathcal{R}_{n-1}\  \newsp{\tilde{s}}{Q_2 \Par Q \subst{s'}{x}}
					\]

				\item	$\forall \news{\tilde{s}} \bactout{s}{s_1}$ such that
					\[
						\Gamma; \emptyset; \Sigma_1 \by{\news{\tilde{s}} \bactout{s}{s_1}} \Sigma_1' \proves P_1 \by{\news{\tilde{s}} \bactout{s}{s_1}} P_2
					\]
					with $s_1: S \in \Sigma_1 \vee (\dual{s_2}: S' \in \Sigma_2 \wedge S \dualof S')$
					then $\exists Q_2, s_2$ such that
					\[
						\Gamma; \emptyset; \Sigma_2 \By{\news{\tilde{s'}} \bactout{s}{s_2}} \Sigma_2' \proves Q_1 \By{\news{\tilde{s'}} \bactout{s}{s_2}} Q_2
					\]
					%such that
		%			\begin{eqnarray*}
		%				\Gamma; \emptyset; \Sigma_1'' \proves \newsp{\tilde{s}}{P_2 \Par \context{C}{P \subst{s'}{x}}} \hastype \Proc \\
		%				\Gamma; \emptyset; \Sigma_2'' \proves \newsp{\tilde{s}}{Q_2 \Par \context{C}{Q \subst{s'}{x}}} \hastype \Proc
		%			\end{eqnarray*}
					and
					\[
						\Gamma; \emptyset; \Sigma_1''\ \mathcal{R}_{n-1}\ \Sigma_2'' \proves \newsp{\tilde{s}}{P_2 \Par \map{S}^{s_1}}\ \mathcal{R}_{n-1}\ 
						\newsp{\tilde{s'}}{Q_2 \Par \map{S}^{s_2}}
					\]


				\item	$\forall \lambda \not= \news{\tilde{s}} \bactout{s}{\abs{x} P}$ such that
					\[
						\Gamma; \emptyset; \Sigma_1 \by{\lambda} \Sigma_1' \proves P_1 \by{\lambda} P_2
					\]
					$\exists Q_2$ such that 
					\[
						\Gamma; \emptyset; \Sigma_2 \by{\hat{\lambda}} \Sigma_2' \proves Q_1 \By{\hat{\lambda}} Q_2
					\]
					and
					$\Gamma; \emptyset; \Sigma_1'\ \mathcal{R}_{n-1}\ \Sigma_2' \proves P_2\ \mathcal{R}_{n-1}\ Q_2$.

				\item	The symmetric cases of 1, 2, 3.
			\end{enumerate}
	\end{itemize}
	\noi The above function is monotone and the Knaster-Tarski theorem ensures that a lattice is define
	with the largest $\mathcal{R}_n$ to be denote as $\wb_n$ and the largest fix-point is equal to the
	bisimilarity relation $\wb = \bigcap_{i \geq 0} \wb_i$.
\end{definition}

%%%%%%%%%%%%%%%%%%%%%%%%%%%%%%%%%%%%%%%%%%%%%%%%%%%%%%%%%
%  Auxiliary Lemma
%%%%%%%%%%%%%%%%%%%%%%%%%%%%%%%%%%%%%%%%%%%%%%%%%%%%%%%%%


We develop an auxiliary lemma used to prove
the requirements of Theorem~\ref{the:coincidence}.

\begin{lemma}\rm
	\label{lem:aux}
%	\begin{enumerate}
		\item
%		$\Gamma; \emptyset; \Sigma_1 \by{\bactout{s}{(x) P}} \Sigma_2 \proves P_1 \by{\bactout{s}{(x) P}} P_2$
		implies that $\exists s'$ such that $\Gamma; \emptyset; \Sigma_1 \cat s': S \proves P_2 \Par P \subst{s'}{x}$
%	\end{enumerate}
\end{lemma}

\begin{proof}
%	\noi Proof for part 1.

	\noi We do an induction on the labelled transition system
	to derivation for action $\bactout{s}{(x) P}$.
	{\bf Basic Step: }

	\noi $\bout{s}{(x) P} P_2 \by{\bactout{s}{(x) P}} P_2$ and
	\[
		\Gamma; \emptyset; \Sigma_1 \proves \bout{s}{(x) P} P_2 \hastype \Proc
	\]
	\noi Type rule $\trule{Send}$ implies
	\begin{eqnarray}
		\Gamma; \emptyset; \Sigma_1' &\proves& P_2 \hastype \Proc \label{lem:aux1}\\
		\Gamma; \emptyset; \Sigma_1'' &\proves& (x) P \hastype U \label{lem:aux2} \\
		\Sigma_1 &=& (\Sigma_1' \cup \Sigma_1'')\backslash s \cup \set{s: \btout{U} S}
	\end{eqnarray}
	\noi Rule $\trule{Abs}$ implies that from~\ref{lem:aux2} we get
	\begin{eqnarray*}
		\Gamma; \emptyset; \Sigma_1'' \cat x: S' \proves P \hastype \Proc
	\end{eqnarray*}
	\noi we apply the Substitution Lemma~\ref{aaa} to get that for fresh $s'$
	\begin{eqnarray*}
		\Gamma; \emptyset; \Sigma_1'' \cat s': S' \proves P\subst{s'}{x} \hastype \Proc
	\end{eqnarray*}
	\noi Finally we apply rule $\trule{Par}$ to~\ref{lem:aux1} and the latter judgement to get
	\begin{eqnarray*}
		\Gamma; \emptyset; \Sigma_1' \cup \Sigma_1'' \cat s': S' \proves P_2 \Par P\subst{s'}{x} \hastype \Proc
	\end{eqnarray*}

	\noi {\bf Induction step:}

	\noi - Case: $Q \Par R \by{\bactout{s}{(x) P}} Q' \Par R$ and
	\[
		\Gamma; \emptyset; \Sigma_1 \proves Q \Par R \hastype \Proc
	\]
	\noi Type rule $\trule{Par}$ implies
	\begin{eqnarray}
		\Gamma; \emptyset; \Sigma_1' &\proves& Q \hastype \Proc\\
		\Gamma; \emptyset; \Sigma_1'' &\proves& R \hastype \Proc \label{lem:aux3}
	\end{eqnarray}
	\noi From the induction hypothesis we get that 
	\begin{eqnarray*}
		\Gamma; \emptyset; \Sigma_2' \proves Q' \Par P \hastype \Proc
	\end{eqnarray*}
	we apply rule $\trule{Par}$ to the latter judgement and judgement~\ref{lem:aux3} to get
	\begin{eqnarray*}
		\Gamma; \emptyset; \Sigma_2' \cup \Sigma'' \proves Q' \Par R \Par P \hastype \Proc
	\end{eqnarray*}

	The rest of the cases enjoy similar argumentation.
\end{proof}

%%%%%%%%%%%%%%%%%%%%%%%%%%%%%%%%%%%%%%%%%%%%%%%%%%%%%%%%%
%  WBC IS WB
%%%%%%%%%%%%%%%%%%%%%%%%%%%%%%%%%%%%%%%%%%%%%%%%%%%%%%%%%


\begin{lemma}\rm
	\label{lem:wbc_is_wb}
	$\wbc\ \subseteq\ \wb$
\end{lemma}

\begin{proof}
	Statement $\wbc \subseteq \wb$
	is equivalent to the statement $\forall n, \wbc_n \subseteq \wb_n$.
	From here the proof is done using induction on the definitions of $\wbc$.

	\noi {\bf Basic step:}
	From the definitions of $\wbc_0$ and $\wb_0$ we get that $\wbc_0 = \wb_0$.

	\noi {\bf Induction hypothesis:}
	$\wbc_n\ \subseteq\ \wb_n$.

	\noi {\bf Inductive step:}
	Let
%
	\begin{eqnarray*}
		\Gamma; \emptyset; \Sigma_1\ \wbc_{n+1}\ \Sigma_2 \proves P_1\ \wbc_{n+1}\ Q_1
	\end{eqnarray*}
%
	\noi We perform a case analysis on transition $\by\lambda$.

	%%%%%%%%%%%%%%%%%%%%%%%%%%%%%%%%%%%%%%%%%%%%%%%

	\noi - Case: $\lambda \notin \set{\news{\tilde{s}} \bactout{s}{\abs{x} P}, \news{\tilde{s}} \bactout{s}{s_1}}$
%
	\begin{eqnarray}
		\Gamma; \emptyset; \Sigma_1 \by{\lambda} \Sigma_1' \proves P_1 \by{\lambda} P_2 \label{lem:wbc_is_wb1}
	\end{eqnarray}
%
	\noi implies that 
	$\exists Q_1$ such that
%
	\begin{eqnarray}
		\Gamma; \emptyset; \Sigma_2 \by{\lambda} \Sigma_2' &\proves& Q_1 \by{\lambda} Q_2 \label{lem:wbc_is_wb2}\\
		\Gamma; \emptyset; \Sigma_1'\ \wbc_n\ \Sigma_2' &\proves& P_1\ \wbc_n\ Q_2
	\end{eqnarray}
%
	We apply the induction hypothesis to the latter judgement:
%
	\begin{eqnarray}
		\Gamma; \emptyset; \Sigma_1' \proves P_2\ \wb_n\ \Gamma; \emptyset; \Sigma_2' \proves Q_2 \hastype \Proc  \label{lem:wbc_is_wb3}
	\end{eqnarray}
%
	Assume $\mathcal{R}_{n+1} = \set{\Gamma; \emptyset; \Sigma_1 \proves P_1 \hastype \Proc, \Gamma; \emptyset; \Sigma_2 \proves Q_1 \hastype \Proc}$.

	\noi $\mathcal{R}_{n+1}$ satisfies the condition for the stratified definition of bisimulation
	because statement~\ref{lem:wbc_is_wb1} implies $\exists Q'$ such that
	statements~\ref{lem:wbc_is_wb2} holds and furthermore statement~\ref{lem:wbc_is_wb3} holds
%	if $\Gamma; \emptyset; \Sigma_1 \proves P \by{\lambda} \Gamma; \emptyset; \Sigma_1' \proves P' \hastype \Proc$ then
%	$\exists Q'$ such that
%	$\Gamma; \emptyset; \Sigma_2 \proves Q \by{\lambda} \Gamma; \emptyset; \Sigma_2' \proves Q' \hastype \Proc$
%	and \ref{pr:biscong_is_bis2}.

	\noi Because $\wb_{n+1}$ is the largest relation we get that $\mathcal{R}_{n+1} \subseteq \wb_{n+1}$ as required.

	%%%%%%%%%%%%%%%%%%%%%%%%%%%%%%%%%%%%%%%%%%%%%%%

	\noi - Case: $\lambda = \news{\tilde{s}} \bactout{s}{\abs{x} P}$
%
	\begin{eqnarray}
		\Gamma; \emptyset; \Sigma_1 \by{\news{\tilde{s}} \bactout{s}{\abs{x} P}} \Sigma_1' \proves P_1 \by{\news{\tilde{s}} \bactout{s}{\abs{x} P}} P_2 \label{lem:wbc_is_wb4}
	\end{eqnarray}
%
	\noi implies that
	$\exists Q_2, \abs{x}{Q}$ such that
	\begin{eqnarray}
		\Gamma; \emptyset; \Sigma_2 \By{\news{\tilde{s'}} \bactout{s}{\abs{x} Q}} \Sigma_2' \proves Q_1 \By{\news{\tilde{s'}} \bactout{s}{\abs{x} Q}} Q_2  \label{lem:wbc_is_wb5}
	\end{eqnarray}
	and $\forall R$ with $\set{X} = \fpv{R}$
%	such that
%	\begin{eqnarray*}
%		\Gamma; \emptyset; \Sigma_1'' \proves \newsp{\tilde{s}}{\context{C}{P' \Par P \subst{s'}{x}}} \hastype \Proc \\
%		\Gamma; \emptyset; \Sigma_2'' \proves \newsp{\tilde{s}}{\context{C}{Q' \Par Q \subst{s'}{x}}} \hastype \Proc
%	\end{eqnarray*}
%	then
%
	\begin{eqnarray*}
		\Gamma; \emptyset; \Sigma_1''\ \wbc_{n}\ \Sigma_2'' \proves \newsp{\tilde{s}}{P_2 \Par R \subst{(x) P}{X}}\ \wbc_{n}\ 
		\newsp{\tilde{s'}}{Q_2 \Par R\subst{(x) Q}{X}}
	\end{eqnarray*}
%
	\noi For $R = \appl{X}{s'}, \forall s'$ we have that 
%
	\begin{eqnarray*}
		\Gamma; \emptyset; \Sigma_1''\ \wbc_{n}\ \Sigma_2'' \proves \newsp{\tilde{s}}{P_2 \Par P \subst{s'}{x}}\ \wbc_{n}\ 
		\newsp{\tilde{s}}{Q' \Par Q_2 \subst{s'}{x}}
	\end{eqnarray*}
%
	\noi Lemma~\ref{lem:aux} ensures for the validity of the latter statement.

	\noi If we apply the induction hypothesis to the latter statement we get
%
	\begin{eqnarray}
		\Gamma; \emptyset; \Sigma_1''\ \wb_{n}\ \Sigma_2'' \proves \newsp{\tilde{s}}{P_2 \Par P \subst{s'}{x}}\ \wb_{n}\ 
		\newsp{\tilde{s}}{Q_2 \Par Q \subst{s'}{x}}
		\label{lem:wbc_is_wb6}
	\end{eqnarray}
%
	\noi Assume $\mathcal{R}_{n+1} = \set{\Gamma; \emptyset; \Sigma_1 \proves P_1 \hastype \Proc, \Gamma; \emptyset; \Sigma_2 \proves Q_2 \hastype \Proc}$.

	\noi $\mathcal{R}_{n+1}$ satisfies the condition for the stratified definition for bisimulation
	because statement~\ref{lem:wbc_is_wb4} implies that
	$\exists Q', \abs{x}{Q}$ such that
	statement~\ref{lem:wbc_is_wb5} holds and furthemore statement~\ref{lem:wbc_is_wb6} holds.

	\noi Because $\wb_{n+1}$ is the largest relation we get that $\mathcal{R}_{n+1} \subseteq \wb_{n+1}$ as required.

	%%%%%%%%%%%%%%%%%%%%%%%%%%%%%%%%%%%%%%%%%%%%%%%

	\noi - Case: $\lambda = \news{\tilde{s}} \bactout{s}{s_1}$
%
	\begin{eqnarray}
		\Gamma; \emptyset; \Sigma_1 \by{\news{\tilde{s}} \bactout{s}{s_1}} \Sigma_1' \proves P_1 \by{\news{\tilde{s}} \bactout{s}{s_1}} P_2 \label{lem:wbc_is_wb7}
	\end{eqnarray}
%
	\noi \dk{with $s_1: S \in \Sigma_1 \vee (\dual{s_1}: S' \in \Sigma_1' \wedge S \dualof S')$} implies that
	$\exists Q', s_2$ such that
	\begin{eqnarray}
		\Gamma; \emptyset; \Sigma_2 \By{\news{\tilde{s'}} \bactout{s}{s_2}} \Sigma_2' \proves Q_1 \By{\news{\tilde{s'}} \bactout{s}{s_2}} Q_2 \label{lem:wbc_is_wb8}
	\end{eqnarray}
	and $\forall R$ with $\set{x} = \fn{P}$
%	such that
%	\begin{eqnarray*}
%		\Gamma; \emptyset; \Sigma_1'' \proves \newsp{\tilde{s}}{\context{C}{P' \Par P \subst{s'}{x}}} \hastype \Proc \\
%		\Gamma; \emptyset; \Sigma_2'' \proves \newsp{\tilde{s}}{\context{C}{Q' \Par Q \subst{s'}{x}}} \hastype \Proc
%	\end{eqnarray*}
%	then
%
	\begin{eqnarray*}
		\Gamma; \emptyset; \Sigma_1''\ \wbc_{n}\ \Sigma_2'' \proves \newsp{\tilde{s}}{P_2 \Par R \subst{s_1}{x}}\ \wbc_{n}\ 
		\newsp{\tilde{s'}}{Q_2 \Par R \subst{s_2}{x}}
	\end{eqnarray*}
%
	\noi From the latter statement we get
	\begin{eqnarray*}
		\Gamma; \emptyset; \Sigma_1''\ \wbc_{n}\ \Sigma_2'' \proves \newsp{\tilde{s}}{P_2 \Par \map{S}^{x} \subst{s_1}{x}}\ \wbc_{n}\ 
		\newsp{\tilde{s'}}{Q_2 \Par \map{S}^{x} \subst{s_2}{x}} \label{lem:wbc_is_wb9}
	\end{eqnarray*}
%
	\noi Assume $\mathcal{R}_{n+1} = \set{\Gamma; \emptyset; \Sigma_1 \proves P_1 \hastype \Proc, \Gamma; \emptyset; \Sigma_2 \proves Q_1 \hastype \Proc}$.

	\noi $\mathcal{R}_{n+1}$ satisfies the condition for the stratified definition for bisimulation
	because statement~\ref{lem:wbc_is_wb7} implies that
	$\exists Q', s_2$ such that
	statement~\ref{lem:wbc_is_wb8} holds and furthemore statement~\ref{lem:wbc_is_wb9} holds.
\end{proof}

%%%%%%%%%%%%%%%%%%%%%%%%%%%%%%%%%%%%%%%%%%%%%%%%%%%%%%%%%
%  WB IS WBC
%%%%%%%%%%%%%%%%%%%%%%%%%%%%%%%%%%%%%%%%%%%%%%%%%%%%%%%%%

\begin{lemma}\rm
	\label{lem:wb_is_wbc}
	$\wb\ \subseteq\ \wbc$
\end{lemma}

\begin{proof}
	The statement $\wb \subseteq \wbc$.
	is equivalent to the statement $\forall n, \wb_n \subseteq \wbc_n$.
	The proof is done using induction on the definition $\wb$.

	\noi {\bf Basic step:} From the definitions of $\wbc_0$ and $\wb_0$ we get that $\wbc_0 = \wb_0$.

	\noi {\bf Induction hypothesis:} $\wb_n \subseteq \wbc_n$.

	\noi {\bf Inductive step:}
	Let 
	\[
		\Gamma; \emptyset; \Sigma_1\ \wb_{n+1}\ \Sigma_2 \proves P_1\ \wb_{n+1}\ Q_1
	\]
	We perform a case analysis on transition $\by{\lambda}$.

	\noi - Case: $\lambda \notin \set{\news{\tilde{s}} \bactout{s}{\abs{x} P}, \news{\tilde{s}} \bactout{s}{s_1}}$

	\noi Same arguments with the same case of the direction $\wbc \subseteq \wb$

	%%%%%%%%%%%%%%%%%%%%%%%%%%%%%%%%%%%%%%%%%%%%%%%

	\noi - Case: $\lambda = \news{\tilde{s}} \bactout{s}{\abs{x} P}$
%
	\begin{eqnarray}
		\Gamma; \emptyset; \Sigma_1 \by{\news{\tilde{s}} \bactout{s}{\abs{x} P}} \Sigma_1' \proves P_1 \by{\news{\tilde{s}} \bactout{s}{\abs{x} P}} P_2 \label{lem:wb_is_wbc1}
	\end{eqnarray}
%
	\noi implies that
	$\exists Q_2, \abs{x}{Q}$ such that
	\begin{eqnarray}
		\Gamma; \emptyset; \Sigma_2 \By{\news{\tilde{s'}} \bactout{s}{\abs{x} Q}} \Sigma_2' \proves Q_1 \By{\news{\tilde{s'}} \bactout{s}{\abs{x} Q}} Q_2 \label{lem:wb_is_wbc2}
	\end{eqnarray}
%
	\noi and $\forall s'$
%	such that
%	\begin{eqnarray*}
%		\Gamma; \emptyset; \Sigma_1'' \proves \newsp{\tilde{s}}{P' \Par P \subst{s'}{x}} \hastype \Proc \\
%		\Gamma; \emptyset; \Sigma_2'' \proves \newsp{\tilde{s}}{Q' \Par Q \subst{s'}{x}} \hastype \Proc
%	\end{eqnarray*}
%	then
	\begin{eqnarray*}
		\Gamma; \emptyset; \Sigma_1''\ \wb_{n}\ \Sigma_2'' \proves \newsp{\tilde{s}}{P' \Par P \subst{s'}{x}}\ \wb_{n}\ 
		\newsp{\tilde{s'}}{Q' \Par Q \subst{s'}{x}}
	\end{eqnarray*}
%
	\noi Set
%
	\begin{eqnarray*}
		\mathcal{R}_{n + 1} &=& \set{	(\Gamma; \emptyset; \Sigma_1 \proves \newsp{\tilde{s}}{\bout{s}{\abs{x}{R \subst{(x) P}{X}}} P_2} \hastype \Proc,
						\Gamma; \emptyset; \Sigma_2 \proves \newsp{\tilde{s'}}{\bout{s}{\abs{x}{R \subst{(x) Q}{X}}} Q_2} \hastype \Proc) \setbar
						\forall R, \set{X} = \fpv{R}}
	\end{eqnarray*}
%
	\dk{prove that $R_{n+1}$ is typable}

	\noi $\mathcal{R}_{n+1}$ satisfies the stratified definition for the bisimulation 
	because $\forall \news{\tilde{s}} \bactout{s}{(x) \context{C}{P}} $
%
	\begin{eqnarray*}
		\Gamma; \emptyset; \Sigma_1 \by{\news{\tilde{s}} \bactbout{s}{(x) \context{C}{P}}} \Sigma_2'' \proves \newsp{\tilde{s}}{\bout{s}{\abs{x}{\context{C}{P}}} P_2} \by{\news{\tilde{s}} \bactbout{s}{(x) \context{C}{P}}} P_2
	\end{eqnarray*}
%
	implies that $\exists Q_2, \context{C}{Q}$ such that
%
	\begin{eqnarray*}
		\Gamma; \emptyset; \Sigma_1 \by{\news{\tilde{s'}} \bactbout{s}{(x) \context{C}{Q}}} \Sigma_2'' \proves \newsp{\tilde{s'}}{\bout{s}{\abs{x}{\context{C}{Q}}} Q_2} \by{\news{\tilde{s}} \bactbout{s}{(x) \context{C}{P}}} Q_2
	\end{eqnarray*}
%
	\noi and $\forall s'$
%	$\forall C, s'$
%	such that
%	\begin{eqnarray*}
%		\Gamma; \emptyset; \Sigma_1'' \proves \newsp{\tilde{s}}{\context{C}{P' \Par P \subst{s'}{x}}} \hastype \Proc \\
%		\Gamma; \emptyset; \Sigma_2'' \proves \newsp{\tilde{s}}{\context{C}{Q' \Par Q \subst{s'}{x}}} \hastype \Proc
%	\end{eqnarray*}
%	then
	\begin{eqnarray*}
		\Gamma; \emptyset; \Sigma_1'' \ \wb_{n}\ \Sigma_2'' \proves \newsp{\tilde{s}}{P' \Par \context{C}{P \subst{s'}{x}}}\ \wb_{n}\ 
		\newsp{\tilde{s'}}{Q' \Par \context{C}{Q \subst{s'}{x}}}
	\end{eqnarray*}
	\dk{the latter statement is wrong... Need to prove that it holds through induction on $n$}

%
	\noi We apply the induction hypothesis to get that
	$\forall C, s'$
%	such that
%	\begin{eqnarray*}
%		\Gamma; \emptyset; \Sigma_1'' \proves \newsp{\tilde{s}}{\context{C}{P' \Par P \subst{s'}{x}}} \hastype \Proc \\
%		\Gamma; \emptyset; \Sigma_2'' \proves \newsp{\tilde{s}}{\context{C}{Q' \Par Q \subst{s'}{x}}} \hastype \Proc
%	\end{eqnarray*}
%	then
	\begin{eqnarray}
		\Gamma; \emptyset; \Sigma_1'' \ \wbc_{n}\ \Sigma_2'' \proves \newsp{\tilde{s}}{P' \Par \context{C}{P \subst{s'}{x}}}\ \wbc_{n}\ 
		\newsp{\tilde{s'}}{Q' \Par \context{C}{Q \subst{s'}{x}}} \label{lem:wb_is_wbc3}
	\end{eqnarray}
%
	\noi Set $\mathcal{R}^c_{n+1} = \set{\Gamma; \emptyset; \Sigma_1 \proves P \hastype \Proc, \Gamma; \emptyset; \Sigma_2 \proves Q \hastype \Proc}$.

	\noi $\mathcal{R}^c_{n+1}$ satisfies the condition for the stratified definition of the contextual bisimulation because
	statement~\ref{lem:wb_is_wbc1} implies $\exists Q_2, Q$ such that statement~\ref{lem:wb_is_wbc2} holds
	and furthermore statement~\ref{lem:wb_is_wbc3} holds.
%	if $\Gamma; \emptyset; \Sigma_1 \proves P \by{\news{\tilde{s}} \bactout{s}{\abs{x} P}} \Gamma; \emptyset; \Sigma_1' \proves P' \hastype \Proc$ then
%	$\exists Q', \abs{x}{Q}$ such that
%	$\Gamma; \emptyset; \Sigma_2 \proves Q \by{\news{\tilde{s}} \bactout{s}{\abs{x} Q}} \Gamma; \emptyset; \Sigma_2' \proves Q' \hastype \Proc$
%	and \ref{pr:bis_is_contextbis}.

	\noi Because $\wbc_{n+1}$ is the largest $n$-th contextual bisimulation we get that $R^c_{n+1} \subseteq \wb_{n+1}$ as required.

	%%%%%%%%%%%%%%%%%%%%%%%%%%%%%%%%%%%%%%%%%%%%%%%

	\noi - Case: $\lambda = \news{\tilde{s}} \bactout{s}{s_1}$
%
	\begin{eqnarray}
		\Gamma; \emptyset; \Sigma_1 \by{\news{\tilde{s}} \bactout{s}{s_1}} \Sigma_1' \proves P_1 \by{\news{\tilde{s}} \bactout{s}{s_1}} P_2 \label{lem:wb_is_wbc4}
	\end{eqnarray}
%
	\noi with $s_1: S \in \Sigma_1 \vee (\dual{s_1}: S' \in \Sigma_1' \wedge S \dualof S')$ implies that
	$\exists Q', s_2$ such that
	\begin{eqnarray}
		\Gamma; \emptyset; \Sigma_2 \By{\news{\tilde{s'}} \bactout{s}{s_2}} \Sigma_2' \proves Q_1 \By{\news{\tilde{s'}} \bactout{s}{s_2}} Q_2 \label{lem:wb_is_wbc5}
	\end{eqnarray}
	and
%	such that
%	\begin{eqnarray*}
%		\Gamma; \emptyset; \Sigma_1'' \proves \newsp{\tilde{s}}{\context{C}{P' \Par P \subst{s'}{x}}} \hastype \Proc \\
%		\Gamma; \emptyset; \Sigma_2'' \proves \newsp{\tilde{s}}{\context{C}{Q' \Par Q \subst{s'}{x}}} \hastype \Proc
%	\end{eqnarray*}
%	then
%
	\begin{eqnarray*}
		\Gamma; \emptyset; \Sigma_1''\ \wb_{n}\ \Sigma_2'' \proves \newsp{\tilde{s}}{P_2 \Par \map{S}^{x} \subst{s_1}{x}}\ \wb_{n}\ 
		\newsp{\tilde{s'}}{Q_2 \Par \map{S}^{x} \subst{s_2}{x}}
	\end{eqnarray*}
%
	\noi Set
%
	\begin{eqnarray*}
		\mathcal{R}_{n + 1} &=& \set{	(\Gamma; \emptyset; \Sigma_1 \proves \newsp{\tilde{s}}{\bout{s}{\abs{x}{R}} P_2} \hastype \Proc,
						\Gamma; \emptyset; \Sigma_2 \proves \newsp{\tilde{s'}}{\bout{s}{\abs{x}{R}} Q_2} \hastype \Proc) \setbar \forall R}
	\end{eqnarray*}
%
	\dk{prove that $R_{n+1}$ is typable}

	\noi $\mathcal{R}_{n+1}$ satisfies the stratified definition for the bisimulation 
	because $\forall \news{\tilde{s}} \bactout{s}{(x) R} $
%
	\begin{eqnarray*}
		\Gamma; \emptyset; \Sigma_1 \by{\news{\tilde{s}} \bactbout{s}{(x) R}} \Sigma_2'' \proves \newsp{\tilde{s}}{\bout{s}{\abs{x}{R}} P_2} \by{\news{\tilde{s}} \bactbout{s}{(x) R}} P_2
	\end{eqnarray*}
%
	implies that $\exists Q_2, R$ such that
%
	\begin{eqnarray*}
		\Gamma; \emptyset; \Sigma_1 \by{\news{\tilde{s}} \bactbout{s}{(x) R}} \Sigma_2'' \proves \newsp{\tilde{s}}{\bout{s}{\abs{x}{R}} Q_2} \by{\news{\tilde{s}} \bactbout{s}{(x) R}} Q_2
	\end{eqnarray*}
%
	\noi and $\forall s'$
%
	\begin{eqnarray*}
		\Gamma; \emptyset; \Sigma_1'' \ \wb_{n}\ \Sigma_2'' \proves \newsp{\tilde{s}}{P' \Par R \subst{s'}{x}}\ \wb_{n}\ 
		\newsp{\tilde{s}}{Q' \Par R \subst{s'}{x}}
	\end{eqnarray*}
%
\dk{continue from here}
\begin{comment}
	\noi We apply the induction hypothesis to get that
	$\forall C, s'$
%
	\begin{eqnarray}
		\Gamma; \emptyset; \Sigma_1'' \ \wbc_{n}\ \Sigma_2'' \proves \newsp{\tilde{s}}{\context{C}{P' \Par P \subst{s'}{x}}}\ \wbc_{n}\ 
		\newsp{\tilde{s}}{\context{C}{Q' \Par Q \subst{s'}{x}}} \label{lem:wb_is_wbc3}
	\end{eqnarray}
%%
	\noi Assume $\mathcal{R}_{n+1} = \set{\Gamma; \emptyset; \Sigma_1 \proves P_1 \hastype \Proc, \Gamma; \emptyset; \Sigma_2 \proves Q_1 \hastype \Proc}$.

	\noi $\mathcal{R}_{n+1}$ satisfies the condition for the stratified definition for bisimulation
	because statement~\ref{lem:wbc_is_wb7} implies that
	$\exists Q', s_2$ such that
	statement~\ref{lem:wbc_is_wb8} holds and furthemore statement~\ref{lem:wbc_is_wb9} holds.
\end{comment}
\end{proof}


%%%%%%%%%%%%%%%%%%%%%%%%%%%%%%%%%%%%%%%%%%%%%%%%%%%%%%%%%
%  WB IS CONG
%%%%%%%%%%%%%%%%%%%%%%%%%%%%%%%%%%%%%%%%%%%%%%%%%%%%%%%%%

\begin{lemma}
	\label{lem:wbc_is_cong}
	$\wbc \subseteq \cong$.
\end{lemma}


\begin{proof}
	\noi We prove that $\wbc$ satisfies the defining properties of $\cong$. Let 
%
	\[
		\Gamma; \emptyset; \Sigma_1 \wbc \Sigma_2 \proves P_1 \wbc P_2
	\]
%
	{\bf Reduction Closed:}

	\begin{eqnarray*}
		\Gamma; \emptyset; \Sigma_1 \by{} \Sigma_1' \proves P_1 \by{} P_1'
	\end{eqnarray*}
%
	\noi implies that 
	$\exists P_2'$ such that 
%
	\begin{eqnarray*}
		\Gamma; \emptyset; \Sigma_2 \By{} \Sigma_2' &\proves& P_2 \By{} P_2'\\
		\Gamma; \emptyset; \Sigma_1' \wbc \Sigma_2' &\proves& P_1' \wbc P_2'
	\end{eqnarray*}

	\noi Same argument hold for the symmetric case, thus $\wbc$ is reduction closed.

	\noi {\bf Barb Preservation:}
%
	\begin{eqnarray*}
		\Gamma; \emptyset; \Sigma_1 \proves P_1 \hastype \Proc \barb{s}
	\end{eqnarray*}
%
	implies that
	\begin{eqnarray*}
		P &\cong& \newsp{\tilde{s}}{\bout{s}{V_1} P_3 \Par P_4}\\
		\dual{s} &\notin& \Sigma_1
	\end{eqnarray*}
%
	\noi From the definition of $\wbc$ we get that
%
	\begin{eqnarray*}
		\Gamma; \emptyset; \Sigma_1 \by{\news{s_1} \bactout{s}{V_1}} \Sigma_1' \proves \newsp{\tilde{s}}{\bout{s}{V_1} P_3 \Par P_4} \by{\news{s_1} \bactout{s}{V_1}} \newsp{\tilde{s'}}{P_3 \Par P_4}
	\end{eqnarray*}
%
	\noi implies
%
	\begin{eqnarray*}
		\Gamma; \emptyset; \Sigma_2 \By{\news{s_2} \bactout{s}{V_2}} \Sigma_2' \proves P_2 \By{\news{s_2} \bactout{s}{V_2}} \proves P_2'
	\end{eqnarray*}
%
	\noi From the last result we get that
%
	\begin{eqnarray*}
		\Gamma; \emptyset; \Sigma_2 \proves P_2 \hastype \Proc \Barb{s}
	\end{eqnarray*}
%
	\noi as required.

	\noi {\bf Congruence:}

	\noi The congruence property requires that we check that $\wbc$
	is preserved under any context.
	The most interesting context case is parallel composition.

	\noi We construct a congruence relation. Let
	\[
	\begin{array}{rcl}
		\mathcal{S} &=&	\set{(\Gamma; \emptyset; \Sigma_1 \cup \Sigma_3 \proves \newsp{\tilde{s_1}}{P_1 \Par R} \hastype \Proc, \Gamma; \emptyset; \Sigma_2 \cup \Sigma_3 \proves \newsp{\tilde{s_2}}{P_2 \Par R}) \setbar \\
		& &	\Gamma;\emptyset;\Sigma_1 \wbc \Sigma_2 \proves P_1 \wbc P_2, \forall \Gamma; \emptyset; \Sigma_3 \proves R \hastype \Proc}
	\end{array}
	\]
	\noi We need to show that the above congruence is a bisimulation.
	To show that $\mathcal{S}$ is a bisimulation we do a case analysis on the structure
	of the $\by{\lambda}$ transition.

	%%%%%%%%%%%%%%%
	% Case 1
	%%%%%%%%%%%%%%%

	\noi - Case: 
	\[
		\Gamma; \emptyset; \Sigma_1 \cup \Sigma_3 \by{\lambda} \Sigma_1' \cup \Sigma_3 \proves \newsp{\tilde{s_1}}{P_1 \Par R} \by{\lambda} \newsp{\tilde{s_1'}}{P_1' \Par R}
	\]

	\noi The case is divided into three subcases:

	\noi Subcase i: $\lambda \notin \set{\news{\tilde{s}} \bactout{s}{Q}, \news{\tilde{s}} \bactout{s}{s_1}}$

	\noi From the definition of typed transition we get
	\[
		\Gamma; \emptyset; \Sigma_1 \by{\lambda} \Sigma_1' \proves P_1 \by{\lambda} P_1'
	\]
	\noi which implies that
%
	\begin{eqnarray}
		\Gamma; \emptyset; \Sigma_1 \By{\lambda} \Sigma_2' \proves P_2 \By{\lambda} P_2' \label{lem:wbc_is_cong1}\\
		\Gamma; \emptyset; \Sigma_1' \wbc \Sigma_2'' \proves P_1' \wbc P_2' \label{lem:wbc_is_cong2}
	\end{eqnarray}
%
	\noi From transition~\ref{lem:wbc_is_cong1} we conclude that 
	\[
		\Gamma;\emptyset; \Sigma_2 \cup \Sigma_3 \By{\lambda} \Sigma_2' \cup \Sigma_3 \proves \newsp{\tilde{s_2}}{P_2 \Par R} \By{\lambda} \newsp{\tilde{s_2'}}{P_2' \Par R}
	\]
%
	\noi Furthermore from~\ref{lem:wbc_is_cong2} and the definition of $\mathcal{S}$ we conlude that
	\[
		\Gamma; \emptyset; \Sigma_1' \cup \Sigma_3\ \mathcal{S}\ \Sigma_2' \cup \Sigma_3 \proves \newsp{\tilde{s_1'}}{P_1' \Par R}\ \mathcal{S}\ \newsp{\tilde{s_2'}}{P_2' \Par R}
	\]

	\noi Subcase ii: $\lambda = \news{\tilde{s_3}} \bactout{s}{\abs{x}{Q_1}}$

	\noi From the definition of typed transition we get
	\[
		\Gamma; \emptyset; \Sigma_1 \by{\news{\tilde{s_3}} \bactout{s}{\abs{x}{Q_1}}} \Sigma_1' \proves P_1 \by{\news{\tilde{s_3}} \bactout{s}{\abs{x}{Q_1}}} P_1'
	\]
	\noi which implies that
%
	\begin{eqnarray}
		&&\Gamma; \emptyset; \Sigma_1 \By{\news{\tilde{s_4}} \bactout{s}{\abs{x}{Q_2}}} \Sigma_2' \proves P_2 \By{\news{\tilde{s_4}} \bactout{s}{\abs{x}{Q_2}}} P_2' \label{lem:wbc_is_cong3} \\
		&&\forall Q, \set{X} \in \fpv{Q} \\
%		\forall s'
%		&&\Gamma; \emptyset; \Sigma_1'' \wb \Sigma_2'' \proves \newsp{\tilde{s_1''}}{P_1' \Par Q_1 \subst{s'}{s}}\ \wb\ \newsp{\tilde{s_2''}}{P_2' \Par Q_2 \subst{s'}{x}} \label{lem:wb_is_cong4}
		&&\Gamma; \emptyset; \Sigma_1'' \wbc \Sigma_2'' \proves \newsp{\tilde{s_1''}}{P_1' \Par Q \subst{(x) Q_1}{X}}\ \wbc\ \newsp{\tilde{s_2''}}{P_2' \Par Q \subst{(x) Q_2}{X}} \label{lem:wbc_is_cong4}
	\end{eqnarray}
%
	\noi From transition~\ref{lem:wbc_is_cong3} conclude that 
	\[
		\Gamma;\emptyset; \Sigma_2 \cup \Sigma_3 \By{\news{\tilde{s_4}} \bactout{s}{\abs{x}{Q_2}}} \Sigma_2' \cup \Sigma_3 \proves \newsp{\tilde{s_2}}{P_2 \Par R} \By{\news{\tilde{s_4}} \bactout{s}{\abs{x}{Q_2}}} \proves \newsp{\tilde{s_2'}}{P_2' \Par R}
	\]
%
	\noi Furthermore from~\ref{lem:wbc_is_cong4} we conlude that $\forall Q$ with $\set{X} = \fpv{Q}$
%
	\[
		\Gamma; \emptyset; \Sigma_1'' \cup \Sigma_3\ \mathcal{S}\ \Sigma_2'' \cup \Sigma_3 \proves \newsp{\tilde{s_1''}}{P_1' \Par Q \subst{(x) Q_1}{X} \Par R}\ \mathcal{S}\ \newsp{\tilde{s_2''}}{P_2' \Par Q \subst{(x) Q_2}{X} \Par R}
	\]
%
	- Subcase iii: $\lambda = \news{\tilde{s}} \bactout{s}{s_1}$

	\noi From the definition of typed transition we get that
	\[
		\Gamma; \emptyset; \Sigma_1 \by{\news{\tilde{s}} \bactout{s}{s_1}} \Sigma_1' \proves P_1 \by{\news{\tilde{s}} \bactout{s}{s_1}} P_1'
	\]
	\noi which implies that $\exists P_2', s_2$ such that
%
	\begin{eqnarray}
		&& \Gamma; \emptyset; \Sigma_1 \By{\news{\tilde{s'}} \bactout{s}{s_2}} \Sigma_2' \proves P_2 \By{\news{\tilde{s'}} \bactout{s}{s_2}} P_2' \label{lem:wbc_is_cong5}\\
		\forall Q, x = \fn{Q}%  &&
		&&\Gamma; \emptyset; \Sigma_1''\ \wbc\ \Sigma_2'' \proves \newsp{\tilde{s}}{P_1' \Par Q \subst{s_1}{x}}\ \wbc\ \newsp{\tilde{s'}}{P_2' \Par Q \subst{s_2}{x}} \label{lem:wbc_is_cong6}
	\end{eqnarray}
%
	\noi From transition~\ref{lem:wbc_is_cong5} conclude that 
	\[
		\Gamma;\emptyset; \Sigma_2 \cup \Sigma_3 \By{\news{\tilde{s'}} \bactout{s}{s_2}} \Sigma_2' \cup \Sigma_3 \proves \newsp{\tilde{s_2}}{P_2 \Par R} \By{\news{\tilde{s'}} \bactout{s}{s_2}} \proves \newsp{\tilde{s_2'}}{P_2' \Par R}
	\]
%
	\noi Furthermore from~\ref{lem:wbc_is_cong6} we conlude that $\forall Q, x = \fn{Q}$
%
	\[
		\Gamma; \emptyset; \Sigma_1'' \cup \Sigma_3\ \mathcal{S}\ \Sigma_2'' \cup \Sigma_3 \proves \newsp{\tilde{s_1''}}{P_1' \Par Q \subst{s_1}{x} \Par R}\ \mathcal{S}\ \newsp{\tilde{s_2''}}{P_2' \Par Q \subst{s_2}{x} \Par R}
	\]
%
	%%%%%%%%%%%%%%%
	% Case 2
	%%%%%%%%%%%%%%%

	\noi - Case:
%
	\[
		\Gamma; \emptyset; \Sigma_1 \cup \Sigma_3 \by{\lambda} \Sigma_1 \cup \Sigma_3' \proves \newsp{\tilde{s_1}}{P_1 \Par R} \by{\lambda} \newsp{\tilde{s_1'}}{P_1 \Par R'}
	\]
%
	\noi This case is divided into three subcases:

	\noi Subcase i: $\lambda \notin \set{\news{\tilde{s}} \bactout{s}{\abs{x}{Q}}, \news{\tilde{s}} \bactout{s}{\s_1}}$

	\noi From the LTS we get that:
	\[
		\Gamma; \emptyset; \Sigma_3 \by{\lambda} \Sigma_3' \proves R \by{\lambda} R'
	\]
%
	\noi Which in turn implies
	\begin{eqnarray*}
		\Gamma; \emptyset; \Sigma_2 \cup \Sigma_3 \by{\lambda} \Sigma_2 \cup \Sigma_3' \proves \newsp{\tilde{s_2}}{P_2 \Par R} \by{\lambda} \newsp{\tilde{s_2'}}{P_2 \Par R'}
	\end{eqnarray*}
%
	\noi From the definition of $\mathcal{S}$ we conclude that
	\[
		\Gamma; \emptyset; \Sigma_1 \cup \Sigma_3'\ \mathcal{S}\ \Sigma_2 \cup \Sigma_3'' \proves \newsp{\tilde{s_1'}}{P_1 \Par R'}\ \mathcal{S}\ \newsp{\tilde{s_2'}}{P_2 \Par R'}
	\]
	\noi as required.

	\noi Subcase ii: $\lambda = \news{\tilde{s}} \bactout{s}{\abs{x}{Q}}$

	\noi From the LTS we get that:
	\begin{eqnarray}
		& &	\Gamma; \emptyset; \Sigma_3 \by{\lambda} \Sigma_3' \proves R \by{\lambda} R' \label{lem:wbc_is_cong7}\\
		& & 	\forall R_1, \set{X} = \fpv{R_1}\\
		%\forall s'
		& &	\Gamma; \emptyset; \Sigma_3'' \proves \newsp{\tilde{s'}}{R' \Par R_1 \subst{(x) Q}{X}} \hastype \Proc \label{lem:wbc_is_cong8}
	\end{eqnarray}
%
	\noi From~\ref{lem:wbc_is_cong7} we get that
	\[
		\Gamma; \emptyset; \Sigma_2 \cup \Sigma_3 \by{\lambda} \Sigma_2 \cup \Sigma_3' \proves \newsp{\tilde{s_2}}{P_2 \Par R} \by{\lambda} \newsp{\tilde{s_2}}{P_2 \Par R'}
	\]
	\noi Furthermore from~\ref{lem:wbc_is_cong8} and the definition of $\mathcal{S}$ we conclude that
	$\forall R_1$ with $\set{X} \in \fpv{R_1}$
	\[
		\Gamma; \emptyset; \Sigma_1 \cup \Sigma_3''\ \mathcal{S}\ \Sigma_2 \cup \Sigma_3'' \proves \newsp{\tilde{s_1}}{P_1 \Par \newsp{\tilde{s'}}{R' \Par R_1 \subst{(x) Q}{X}}}\ \mathcal{S}\ \newsp{\tilde{s_2}}{P_2 \Par \newsp{\tilde{s'}}{R' \Par R_1 \subst{(x) Q}{X}}}
	\]
	\noi as required.

	\noi Subcase iii: $\lambda = \news{\tilde{s}} \bactout{s}{s'}$

	\noi From the LTS we get that:
	\begin{eqnarray}
		& &	\Gamma; \emptyset; \Sigma_3 \by{\lambda} \Sigma_3' \proves R \by{\lambda} R' \label{lem:wbc_is_cong9}\\
		\forall Q, x = \fn{Q}
		& & \Gamma; \emptyset; \Sigma_3'' \proves \newsp{\tilde{s'}}{R' \Par Q \subst{s'}{x}} \hastype \Proc \label{lem:wbc_is_cong10}
	\end{eqnarray}
%
	\noi From~\ref{lem:wbc_is_cong9} we get that
	\[
		\Gamma; \emptyset; \Sigma_2 \cup \Sigma_3 \by{\lambda} \Sigma_2 \cup \Sigma_3' \proves \newsp{\tilde{s_2}}{P_2 \Par R} \by{\lambda} \newsp{\tilde{s_2}}{P_2 \Par R'}
	\]
	\noi Furthermore from~\ref{lem:wbc_is_cong10} and the definition of $\mathcal{S}$ we conclude that
	$\forall Q, x = \fn{Q}$
	\[
		\Gamma; \emptyset; \Sigma_1 \cup \Sigma_3''\ \mathcal{S}\ \Sigma_2 \cup \Sigma_3'' \proves \newsp{\tilde{s_1}}{P_1 \Par \newsp{\tilde{s'}}{R' \Par Q \subst{s'}{x}}}\ \mathcal{S}\ \newsp{\tilde{s_2}}{P_2 \Par \newsp{\tilde{s'}}{R' \Par Q \subst{s'}{x}}}
	\]
	\noi as required.


	%%%%%%%%%%%%%%%
	% Case 3
	%%%%%%%%%%%%%%%

	\noi - Case:
	\[
		\Gamma; \emptyset; \Sigma_1 \cup \Sigma_3 \by{} \Sigma_1' \cup \Sigma_3' \proves \newsp{\tilde{s_1}}{P_1 \Par R} \by{} \newsp{\tilde{s_1'}}{P_1' \Par R'}
	\]

	\noi This case is divided into three subcases:

	\noi Subcase i: $\Gamma; \emptyset; \Sigma_1 \by{\lambda} \Sigma_1' \proves P_1 \by{\lambda} P_1'$
	and $\lambda \not= \news{\tilde{s}} \bactout{s}{\abs{x}{Q}}$ implies
%
	\begin{eqnarray}
		\Gamma; \emptyset; \Sigma_3 \by{\dual{\lambda}} \Sigma_3 \proves R \by{\dual{\lambda}} R' \label{lem:wbc_is_cong11} \\
		\Gamma; \emptyset; \Sigma_2 \By{\hat{\lambda}} \Sigma_2' \proves P_2 \By{\hat{\lambda}} P_2' \label{lem:wbc_is_cong12}\\
		\Gamma; \emptyset; \Sigma_1' \wbc \Sigma_2' \proves P_1' \wbc P_2' \label{lem:wbc_is_cong13}
	\end{eqnarray}
%
	\noi From~\ref{lem:wbc_is_cong11} and~\ref{lem:wbc_is_cong12} we get
	\[
		\Gamma; \emptyset; \Sigma_2 \cup \Sigma_3 \By{} \Sigma_2' \cup \Sigma_3' \proves \newsp{\tilde{s_2}}{P_2 \Par R} \By{} \newsp{\tilde{s_2'}}{P_2' \Par R'}
	\]
%
	\noi From~\ref{lem:wbc_is_cong13} and the definition of $\mathcal{S}$ we get that
	\[
		\Gamma; \emptyset; \Sigma_1' \cup \Sigma_3'\ \mathcal{S}\ \Sigma_2' \cup \Sigma_3 \proves \newsp{\tilde{s_1'}}{P_1' \Par R'}\ \mathcal{S}\ \newsp{\tilde{s_2'}}{P_2' \Par R'}
	\]
	\noi as required.

	\noi Subcase ii: $\Gamma; \emptyset; \Sigma_1 \by{\news{\tilde{s_3}} \bactout{s}{\abs{x}{Q_1}}} \Sigma_1' \proves P_1 \by{\news{\tilde{s_3}} \bactout{s}{\abs{x}{Q_1}}} P_1'$ implies
%
	\begin{eqnarray}
		& & \Gamma; \emptyset; \Sigma_3 \by{\bactinp{s}{\abs{x} {Q_1}}} \Sigma_3' \proves R \by{\bactinp{s}{\abs{x} {Q_1}}} R' \subst{\abs{x}{Q_1}}{X} \label{lem:wbc_is_cong14}\\
		& & \Gamma; \emptyset; \Sigma_1 \cup \Sigma_3 \by{} \Sigma_1' \cup \Sigma_3' \proves \newsp{\tilde{s_1}}{P_1 \Par R} \by{} \newsp{\tilde{s_1''}}{P_1' \Par R' \subst{\abs{x}{Q_1}}{X}}\\
		& & \Gamma; \emptyset; \Sigma_2\By{\news{\tilde{s_4}} \bactout{s}{\abs{x}{Q_2}}} \Sigma_2' \proves P_2 \By{\news{\tilde{s_4}} \bactout{s}{\abs{x}{Q_2}}} P_2' \label{lem:wbc_is_cong15}\\
		& & \forall Q, \set{X} = \fpv{Q} \\
		& & \Gamma; \emptyset; \Sigma_1''\ \wbc\ \Sigma_2'' \proves \newsp{\tilde{s_1'}}{P_1' \Par Q \subst{(x) Q_1}{X}}\ \wbc\ \newsp{\tilde{s_2'}}{P_2' \Par Q \subst{(x) Q_2}{X}} \label{lem:wbc_is_cong16}
	\end{eqnarray}
%
	From~\ref{lem:wbc_is_cong14} and the Substitution Lemma~\ref{aaa} we get that
	\[
		\Gamma; \emptyset; \Sigma_3 \by{\bactinp{s}{\abs{x} {Q_2}}} \Sigma_3'' \proves R \by{\bactinp{s}{\abs{x} {Q_2}}} R' \subst{\abs{x}{Q_2}}{X}
	\]
	%\dk{(prove that $\forall V, R \by{\bactinp{s}{V}} R'\subst{V}{X}$)}
	\noi to combine with~\ref{lem:wbc_is_cong15} and get
	\[
		\Gamma; \emptyset; \Sigma_2 \cup \Sigma_3 \By{} \Sigma_2' \cup \Sigma_3'' \proves \newsp{\tilde{s_2}}{P_2 \Par R} \By{} \newsp{\tilde{s_2''}}{P_2' \Par R' \subst{\abs{x}{Q_2}}{X}}
	\]
%
	\noi Set $Q$ as $R'$ and combine with~\ref{lem:wbc_is_cong16} to get that
%
%	\noi From~\ref{lem:wbc_is_cong16} and the definition of $\mathcal{S}$ we get that
	\[
		\Gamma; \emptyset; \Sigma_1''\ \mathcal{S}\ \Sigma_2'' \proves \newsp{\tilde{s_1'}}{P_1' \Par R' \subst{\abs{x}{Q_1}}{X}}\ \mathcal{S}\ \newsp{\tilde{s_2'}}{P_2' \Par R' \subst{\abs{x}{Q_2}}{X}}
	\]

	\noi Subcase iii: $\Gamma; \emptyset; \Sigma_1 \by{\news{\tilde{s}} \bactout{s}{s_1}} \Sigma_1' \proves P_1 \by{\news{\tilde{s}} \bactout{s}{s_1}} P_1'$
%
	\begin{eqnarray}
		& & \Gamma; \emptyset; \Sigma_3 \by{\bactinp{s}{s_1}} \Sigma_3' \proves R \by{\bactinp{s}{s_1}} R' \subst{s_1}{x} \label{lem:wbc_is_cong24}\\
		& & \Gamma; \emptyset; \Sigma_1 \cup \Sigma_3 \by{} \Sigma_1' \cup \Sigma_3' \proves \newsp{\tilde{s_1}}{P_1 \Par R} \by{} \newsp{\tilde{s_1''}}{P_1' \Par R' \subst{s_1}{x}}\\
		& & \Gamma; \emptyset; \Sigma_2\By{\news{\tilde{s_4}} \bactout{s}{s_2}} \Sigma_2' \proves P_2 \By{\news{\tilde{s_4}} \bactout{s}{s_2}} P_2' \label{lem:wbc_is_cong25}\\
		& & \forall Q, \set{x} = \fpv{Q} \\
		& & \Gamma; \emptyset; \Sigma_1''\ \wbc\ \Sigma_2'' \proves \newsp{\tilde{s_1'}}{P_1' \Par Q \subst{s_1}{x}}\ \wbc\ \newsp{\tilde{s_2'}}{P_2' \Par Q \subst{s_2}{x}} \label{lem:wbc_is_cong26}
	\end{eqnarray}
%
	From~\ref{lem:wbc_is_cong24} and the Substitution Lemma~\ref{aaa} we get that
	\[
		\Gamma; \emptyset; \Sigma_3 \by{\bactinp{s}{s_2}} \Sigma_3'' \proves R \by{\bactinp{s}{s_2}} R' \subst{s_2}{x}
	\]
	%\dk{(prove that $\forall V, R \by{\bactinp{s}{V}} R'\subst{V}{X}$)}
	\noi to combine with~\ref{lem:wbc_is_cong25} and get
	\[
		\Gamma; \emptyset; \Sigma_2 \cup \Sigma_3 \By{} \Sigma_2' \cup \Sigma_3'' \proves \newsp{\tilde{s_2}}{P_2 \Par R} \By{} \newsp{\tilde{s_2''}}{P_2' \Par R' \subst{s_2}{x}}
	\]
%
	\noi Set $Q$ as $R'$ to combine with~\ref{lem:wbc_is_cong26} to get that
%
%	\noi From~\ref{lem:wbc_is_cong16} and the definition of $\mathcal{S}$ we get that
	\[
		\Gamma; \emptyset; \Sigma_1''\ \mathcal{S}\ \Sigma_2'' \proves \newsp{\tilde{s_1'}}{P_1' \Par R' \subst{s_1}{x}}\ \mathcal{S}\ \newsp{\tilde{s_2'}}{P_2' \Par R' \subst{s_2}{x}}
	\]
\end{proof}

%%%%%%%%%%%%%%%%%%%%%%%%%%%%%%%%%%%%%%%%%%%%%%%%%%%%%%%%%
%  CONG IS WB
%%%%%%%%%%%%%%%%%%%%%%%%%%%%%%%%%%%%%%%%%%%%%%%%%%%%%%%%%

\begin{definition}[Definability]\rm
	Let $\Gamma; \emptyset; \Sigma_1 \proves P \hastype \Proc$.
	An external action $\lambda$ is definable whenever
	there exists process
	$\Gamma; \emptyset; \Sigma_2 \proves T\lrangle{\lambda \cat \tilde{\lambda}, \suc \cat \tilde{\suc}} \hastype \Proc$
	with $\suc\cat\tilde{\suc}$ fresh names.% and $N$ a set of names.
	such that:
	\begin{itemize}
		\item	If $\Gamma; \emptyset; \Sigma_1 \by{\lambda} \Sigma_1' \proves P \by{\lambda} P'$ and
			$\lambda\notin \set{\news{\tilde{s}}\bactout{s}{\abs{x}{Q}}, \news{\tilde{s}} \bactout{s}{s_1}}$
			then
			\begin{eqnarray*}
				& &P \Par T\lrangle{\lambda \cat \tilde{\lambda}, \suc \cat \tilde{\suc}} \red P' \Par T\lrangle{\tilde{\lambda}, \tilde{\suc}} \Par \bout{\suc}{\ast} \inact\\
				& & \Gamma; \emptyset; \Sigma_1' \cup \Sigma_2' \proves P' \Par T\lrangle{\tilde{\lambda}, \tilde{\suc}} \Par \bout{\suc}{\ast} \inact
			\end{eqnarray*}

		\item	If $\Gamma; \emptyset; \Sigma_1 \proves P \by{\news{\tilde{s}}\bactout{s}{\abs{x}{Q}}} \Sigma_1' \proves P' \hastype \Proc$
			then
			\begin{eqnarray*}
				& & P \Par T\lrangle{\news{\tilde{s}}\bactout{s}{\abs{x}{Q}} \cat \tilde{\lambda}, \suc \cat \tilde{\suc}} \red P' \Par Q\subst{s'}{x} \Par T\lrangle{\tilde{\lambda}, \tilde{\suc}} \Par \bout{\suc}{\ast} \inact \hastype \Proc\\
				& & \Gamma; \emptyset; \Sigma_1' \cup \Sigma_2' \proves P' \Par Q\subst{s'}{x} \Par T\lrangle{\tilde{\lambda}, \tilde{\suc}} \Par \bout{\suc}{\ast} \inact \hastype \Proc
			\end{eqnarray*}

		\item	If $\Gamma; \emptyset; \Sigma_1 \proves P \by{\news{\tilde{s}}\bactout{s}{s_1}} \Sigma_1' \proves P' \hastype \Proc$
			then
			\begin{eqnarray*}
				& & P \Par T\lrangle{\news{\tilde{s}}\bactout{s}{s_1} \cat \tilde{\lambda}, \suc \cat \tilde{\suc}} \red P' \Par \dk{\map{S}^{x} \subst{s_1}{x}} \Par T\lrangle{\tilde{\lambda}, \tilde{\suc}} \Par \bout{\suc}{\ast} \inact \hastype \Proc\\
				& & \Gamma; \emptyset; \Sigma_1' \cup \Sigma_2' \proves P' \Par \dk{\map{S}^{x} \subst{s_1}{x}} \Par T\lrangle{\tilde{\lambda}, \tilde{\suc}} \Par \bout{\suc}{\ast} \inact \hastype \Proc
			\end{eqnarray*}

		\item	If $P \Par T\lrangle{\lambda \cat \tilde{\lambda}, \suc \cat \tilde{\suc}} \red Q$
			and $\lambda\notin \set{\news{\tilde{s}}\bactout{s}{\abs{x}{Q}}, \news{\tilde{s}}\bactout{s}{s_1}}$
			and $\Gamma; \emptyset; \Sigma \proves Q \hastype \Proc \barb{\suc}$ then
			$\Gamma; \emptyset; \Sigma_1 \proves P \By{\lambda} \Sigma_1' \proves P' \hastype \Proc$,
			and $Q \scong P' \Par T\lrangle{\tilde{\lambda}, \tilde{\suc}} \Par \bout{\suc}{\ast} \inact$.

		\item	If $P \Par T\lrangle{\news{\tilde{s}}\bactout{s}{\abs{x}{Q}} \cat \tilde{\lambda}, \suc \cat \tilde{\suc}} \red Q $
			and $\Gamma; \emptyset; \Sigma \proves Q \hastype \Proc \barb{\suc}$ then
			$\Gamma; \emptyset; \Sigma_1 \proves P \By{\news{\tilde{s}}\bactout{s}{\abs{x}{Q}}} \Sigma_1' \proves P' \hastype \Proc$,
			and $Q \scong P' \Par Q \subst{s'}{x} \Par T\lrangle{\tilde{\lambda}, \tilde{\suc}} \Par \bout{\suc}{\ast} \inact$.

		\item	If $P \Par T\lrangle{\news{\tilde{s}}\bactout{s}{s_1} \cat \tilde{\lambda}, \suc \cat \tilde{\suc}} \red Q $
			and $\Gamma; \emptyset; \Sigma \proves Q \hastype \Proc \barb{\suc}$ then
			$\Gamma; \emptyset; \Sigma_1 \proves P \By{\news{\tilde{s}}\bactout{s}{s_1}} \Sigma_1' \proves P' \hastype \Proc$,
			and $Q \scong P' \Par \dk{\map{S}^x \subst{s_1}{x}} \Par T\lrangle{\tilde{\lambda}, \tilde{\suc}} \Par \bout{\suc}{\ast} \inact$.


	\end{itemize}	
\end{definition}

We first show that every external action $\lambda$ is {\em definable}.

\begin{lemma}[Definability]
	Every action $\lambda$ is definable.
\end{lemma}

\begin{proof}
	\noi We define $T\lrangle{\lambda \cat \tilde{\lambda}, \suc \cat \tilde{\suc}}$:

	\begin{itemize}
		\item	$T\lrangle{\emptyset, \emptyset} \scong \newsp{s_1, s_2}{\binp{s_1}{x} \bout{s_2}{\ast} R \Par \binp{s_2}{x} \bout{s_1}{\ast} \inact}$ with
			$\Gamma; \emptyset; \Sigma \proves T\lrangle{\emptyset, \emptyset} \hastype \Proc$.
			\dk{$R$ is used to complete the type of a session}.

		\item	$T\lrangle{\bactinp{s}{s'} \cat \tilde{\lambda}, \suc \cat \tilde{\suc}} = \bout{\dual{s}}{s'} (T\lrangle{\tilde{\lambda}, \tilde{\suc}, N} \Par \bout{\suc}{\ast} \inact)$.

		\item	$T\lrangle{\bactbra{s}{l} \cat \tilde{\lambda}, \suc \cat \tilde{\suc}} = \bsel{\dual{s}}{l} (T\lrangle{\tilde{\lambda}, \tilde{\suc}, N} \Par \bout{\suc}{\ast} \inact)$.

		\item	$T\lrangle{\bactinp{s}{\abs{x} Q} \cat \tilde{\lambda}, \suc \cat \tilde{\suc}} = \bout{\dual{s}}{\abs{x}{Q}} (T\lrangle{\tilde{\lambda}, \tilde{\suc}, N} \Par \bout{\suc}{\ast} \inact)$.

		\item	$T\lrangle{\bactout{s}{s'} \cat \tilde{\lambda}, \suc \cat \tilde{\suc}} = \binp{\dual{s}}{x} (\dk{\map{S}^x} \Par T\lrangle{\tilde{\lambda}, \tilde{\suc}, N} \Par \bout{\suc}{\ast} \inact)$.

		\item	$T\lrangle{\news{\s'} \bactout{s}{s'} \cat \tilde{\lambda}, \suc \cat \tilde{\suc}, N} = \binp{\dual{s}}{x} (\dk{\map{S}^x} \Par T\lrangle{\tilde{\lambda}, \tilde{\suc}, N} \Par \bout{\suc}{\ast} \inact)$.

		\item	$T\lrangle{\bactsel{s}{l} \cat \tilde{\lambda}, \suc \cat \tilde{\suc}} = \bbra{\dual{s}}{l:(T\lrangle{\tilde{\lambda}, \tilde{\suc}, N} \Par \bout{\suc}{\ast} \inact), l_i: R}_{i \in I}$.

		\item	$T\lrangle{\news{\tilde{s}} \bactout{s}{\news{\tilde{s}} \abs{x}{Q}} \cat \tilde{\lambda}, \suc \cat \tilde{\suc}} = \binp{\dual{s}}{X} (\appl{X}{s'} \Par T\lrangle{\tilde{\lambda}, \tilde{\suc}, N} \Par \bout{\suc}{\ast} \inact)$. \dk{$s'$ is such that $T$ is typable.}
	\end{itemize}

	\noi Assuming a process 
	\[
		\Gamma; \emptyset; \Sigma_1 \proves P \hastype \Proc
	\] 
	\noi it is straightforward to verify that $\forall \lambda$ each $\lambda$ is definable.
\end{proof}

\begin{lemma}\rm
	\label{lem:cong_is_wb}
	$\cong \subseteq \wb$
\end{lemma}

\begin{proof}
	\noi We show that
	$\forall \lambda \cat \tilde{\lambda}$
	such that
	\begin{eqnarray}
		\Gamma; \emptyset; \Sigma_1 \by{\lambda} \Sigma_1' \proves P_1 \by{\lambda} \By{\tilde{\lambda}} P_1' \label{lem:cong_is_wb1}
	\end{eqnarray}
	\noi and 
	\[
		\Gamma; \emptyset; \Sigma_1 \cup \Sigma_3 \cong \Sigma_2 \cup \Sigma_3 \proves P_1 \Par T\lrangle{\lambda \cat \tilde{\lambda}, \tilde{\suc}} \cong P_2 \Par T\lrangle{\lambda \cat \tilde{\lambda}, \tilde{\suc}}
	\]
	\noi then $\forall n$
	\[
		\Gamma; \emptyset; \Sigma_1 \wb_n \Sigma_2 \proves P_1 \wb_n P_2
	\]
%
	\noi We use induction on the definition of $\wb_n$.

	\noi {\bf Basic step:} Trivialy $\Gamma; \emptyset; \Sigma_1 \wb_0 \Sigma_2 \proves P_1 \wb_0 P_2$

	\noi {\bf Inductive Hypothesis:} $\Gamma; \emptyset; \Sigma_1 \wb_n \Sigma_2 \proves P_1 \wb_n P_2$

	\noi {\bf Inductive step:}

	\noi We follow three cases:

	\noi - Case: $\lambda = \news{\tilde{s_1}} \bactout{s}{\abs{x}{Q_1}}$.

	\noi Take into account the fact that
%
	\[
		T\lrangle{\news{\tilde{s_1}} \bactout{s}{\abs{x}{Q_1}} \cat \tilde{\lambda}, \tilde{\suc}} = T\lrangle{\news{\tilde{s_2}} \bactout{s}{\abs{x}{Q_2}} \cat \tilde{\lambda}, \tilde{\suc}}
	\]
%
	\noi to assume:
%
	\[
		\Gamma; \emptyset; \Sigma_1 \cup \Sigma_3 \cong \Sigma_2 \cup \Sigma_3  \proves P_1 \Par T\lrangle{\news{\tilde{s_1}} \bactout{s}{\abs{x}{Q_1}} \cat \tilde{\lambda}, \tilde{\suc}} \cong P_2 \Par T\lrangle{\news{\tilde{s_2}} \bactout{s}{\abs{x}{Q_2}} \cat \tilde{\lambda}, \tilde{\suc}}
	\]

	\noi From the above we get that
	\[
		P_1 \Par T\lrangle{\news{\tilde{s_1}} \bactout{s}{\abs{x}{Q_1}} \cat \tilde{\lambda}, \suc \cat \tilde{\suc}} \red P_1' \Par Q_1 \subst{s'}{x} \Par T\lrangle{\tilde{\lambda}, \tilde{\suc}} \Par \bout{\suc}{\ast} \inact
	\]
	\noi with
	\[
		\Gamma; \emptyset; \Sigma_1' \cup \Sigma_3' \proves P_1' \Par Q_1 \subst{s'}{x} \Par T\lrangle{\tilde{\lambda}, \tilde{\suc}} \Par \bout{\suc}{\ast} \inact \hastype \Proc \barb{\suc}
	\]
	\noi which implies
%
	\begin{eqnarray*}
		& & P_2 \Par T\lrangle{\news{\tilde{s_2}} \bactout{s}{\abs{x}{Q_2}} \cat \tilde{\lambda}, \suc \cat \tilde{\suc}} \Red Q\\
		& & \Gamma; \emptyset; \Sigma_2' \cup \Sigma_3 \proves Q \hastype \Proc \Barb{\suc}
	\end{eqnarray*}
%
	\noi which implies (from the definibility of $\news{\tilde{s_1}} \bactout{s}{\abs{x}{Q_1}}$)
%
	\begin{eqnarray}
		& & Q \scong P_2' \Par Q_2 \subst{s'}{x} \Par T\lrangle{\tilde{\lambda}, \tilde{\suc}} \Par \bout{\suc}{\ast}\\
		& & \Gamma; \emptyset; \Sigma_2\By{\news{\tilde{s_1}} \bactout{s}{\abs{x}{Q_1}}} \Sigma_2' \proves P_2 \By{\news{\tilde{s_1}} \bactout{s}{\abs{x}{Q_1}}} P_2' \label{lem:cong_is_wb2}
	\end{eqnarray}
%
	\noi From the induction hypothesis we get that
	\begin{eqnarray}
		\Gamma;\emptyset; \Sigma_1' \wb_n \Sigma_2' \proves P_1' \Par Q_1 \subst{s'}{x} \wb_n P_2' \Par Q_2 \subst{s'}{x} \label{lem:cong_is_wb3}
	\end{eqnarray}
%
	\noi Let
	\[
		\mathcal{R}_{n+1} = \set{(\Gamma; \emptyset; \Sigma_1 \proves P_1 \hastype \Proc, \Gamma; \emptyset; \Sigma_2 \proves P_2 \hastype \Proc)}
	\]
	\noi We check that requirements for $\mathcal{R}_{n+1}$ to be a stratified bisimulation
	hold because transition~\ref{lem:cong_is_wb1} implies transition~\ref{lem:cong_is_wb2}
	and furthermore~\ref{lem:cong_is_wb3} holds.
%	with respect
%	to $\wb_n$
	The last statement implies that $R_{n+1} \subseteq \wb_{n+1}$ as required.

	\noi - Case: $\lambda = \news{\tilde{s_1}} \bactout{s}{s_1}$.

	\noi Take into account the fact that
%
	\[
		T\lrangle{\news{\tilde{s_1}} \bactout{s}{s_1} \cat \tilde{\lambda}, \tilde{\suc}} = T\lrangle{\news{\tilde{s_2}} \bactout{s}{s_2} \cat \tilde{\lambda}, \tilde{\suc}}
	\]
	\dk{for $s_1: S, s_2: S$}
%
	\noi to assume:
%
	\[
		\Gamma; \emptyset; \Sigma_1 \cup \Sigma_3 \cong \Sigma_2 \cup \Sigma_3  \proves P_1 \Par T\lrangle{\news{\tilde{s_1}} \bactout{s}{s_1} \cat \tilde{\lambda}, \tilde{\suc}} \cong P_2 \Par T\lrangle{\news{\tilde{s_2}} \bactout{s}{s_2} \cat \tilde{\lambda}, \tilde{\suc}}
	\]

	\noi From the above we get that
	\[
		P_1 \Par T\lrangle{\news{\tilde{s_1}} \bactout{s}{s_1} \cat \tilde{\lambda}, \suc \cat \tilde{\suc}} \red P_1' \Par \dk{\map{S}^x} \subst{s_1}{x} \Par T\lrangle{\tilde{\lambda}, \tilde{\suc}} \Par \bout{\suc}{\ast} \inact
	\]
	\noi with
	\[
		\Gamma; \emptyset; \Sigma_1' \cup \Sigma_3' \proves P_1' \Par \dk{\map{S}^x} \subst{s_1}{x} \Par T\lrangle{\tilde{\lambda}, \tilde{\suc}} \Par \bout{\suc}{\ast} \inact \hastype \Proc \barb{\suc}
	\]
	\noi which implies
%
	\begin{eqnarray*}
		& & P_2 \Par T\lrangle{\news{\tilde{s_2}} \bactout{s}{\abs{x}{Q_2}} \cat \tilde{\lambda}, \suc \cat \tilde{\suc}} \Red Q\\
		& & \Gamma; \emptyset; \Sigma_2' \cup \Sigma_3 \proves Q \hastype \Proc \Barb{\suc}
	\end{eqnarray*}
%
	\noi which implies (from the definibility of $\news{\tilde{s_1}} \bactout{s}{s_2}$)
%
	\begin{eqnarray}
		& & Q \scong P_2' \Par \dk{\map{S}^x} \subst{s_2}{x} \Par T\lrangle{\tilde{\lambda}, \tilde{\suc}} \Par \bout{\suc}{\ast}\\
		& & \Gamma; \emptyset; \Sigma_2 \By{\news{\tilde{s_1}} \bactout{s}{s_2}} \Sigma_2' \proves P_2 \By{\news{\tilde{s_1}} \bactout{s}{s_2}} P_2' \label{lem:cong_is_wb4}
	\end{eqnarray}
%
	\noi From the induction hypothesis we get that
	\begin{eqnarray}
		\Gamma;\emptyset; \Sigma_1' \wb_n \Sigma_2' \proves P_1' \Par \dk{\map{S}^x} \subst{s_1}{x} \wb_n P_2' \Par \dk{\map{S}} \subst{s_2}{x} \label{lem:cong_is_wb5}
	\end{eqnarray}
%
	\noi Let
	\[
		\mathcal{R}_{n+1} = \set{(\Gamma; \emptyset; \Sigma_1 \proves P_1 \hastype \Proc, \Gamma; \emptyset; \Sigma_2 \proves P_2 \hastype \Proc)}
	\]
	\noi We check that requirements for $\mathcal{R}_{n+1}$ to be a stratified bisimulation
	hold because transition~\ref{lem:cong_is_wb1} implies transition~\ref{lem:cong_is_wb4}
	and furthermore~\ref{lem:cong_is_wb5} holds.
%	with respect
%	to $\wb_n$
	The last statement implies that $R_{n+1} \subseteq \wb_{n+1}$ as required.

	\noi - Case $\lambda \notin \news{\tilde{s_1}} \bactout{s}{\abs{x}{Q_1}}$ is similar and easier than the previous cases.

	\noi We are ready to prove the main result.

	\noi Let 
	\[
		\Gamma; \emptyset; \Sigma_1 \cong \Sigma_2 \proves P_1 \cong P_2
	\]
	\noi For all $\tilde{\lambda}$ such that
	\[
		\Gamma; \emptyset; \Sigma_1 \By{\lambda} \Sigma_1' \proves P_1 \By{\lambda} P_1'
	\]
	\noi choose
	\[
		\Gamma; \emptyset; \Sigma_3 \proves T\lrangle{\tilde{\lambda}, \tilde{\suc}} \hastype \Proc
	\]
	\noi such that
	\[
		\Gamma; \emptyset; \Sigma_1 \cup \Sigma_3 \cong \Sigma_2 \cup \Sigma_3 \proves P_1 \Par T\lrangle{\tilde{\lambda}, \tilde{\suc}} \cong P_2 \Par T\lrangle{\tilde{\lambda}, \tilde{\suc}}
	\]
	\noi to imply that
	\[
		\Gamma; \emptyset; \Sigma_1 \wb \Sigma_2 \proves P_1 \wb P_2
	\]
\end{proof}



%%%%%%%%%%%%%%%%%%%%%%%%%%%%%%%%%%%%%%%%%%%%%%%%%%%%%%%%%%%%%%
% Proof of the main theorem
%%%%%%%%%%%%%%%%%%%%%%%%%%%%%%%%%%%%%%%%%%%%%%%%%%%%%%%%%%%%%%

\begin{theorem}[Concidence]
	\begin{enumerate}
		\item	$\wbc\ =\ \wb$.
		\item	$\wbc\ =\ \cong$.
	\end{enumerate}
\end{theorem}

\begin{proof}
	\noi Proof for $\wbc\ =\ \wb$ is derived from Lemmas~\ref{lem:wbc_is_wb} and~\ref{lem:wb_is_wbc}.

	\noi For Part 2 we get from Lemma~\ref{lem:wbc_is_cong} that $\wbc\ \subseteq \cong$ and
	from Lemma~\ref{lem:cong_is_wb} that $\cong\ \subseteq\ \wb$. Part 1 of this lemma completes the proof.
\end{proof}

%%%%%%%%%%%%%%%%%%%%%%%%%%%%%%%%%%%%%%%%%%%%%%%%%%%%%%%%%%%%%%
% T - Innertness
%%%%%%%%%%%%%%%%%%%%%%%%%%%%%%%%%%%%%%%%%%%%%%%%%%%%%%%%%%%%%%


\subsection{$\tau$-inertness}

Session types are inherintly $\tau$-inert.

\begin{lemma}[$\tau$-inertness]\rm
	\label{lem:tau_inert}
	Let $P$ an $\HOp$ process
	and $\Gamma; \emptyset; \Sigma \proves P \hastype \Proc$
	\begin{enumerate}
		\item	If $P \red P'$ then $\Gamma; \empty; \Sigma \wb \Sigma' \proves P \wb P'$.
		\item	If $P \red^* P'$ then $\Gamma; \empty; \Sigma \wb \Sigma' \proves P \wb P'$.
	\end{enumerate}
\end{lemma}

\begin{proof}
	The proof for part 1 is done by induction on the structure of $\red$.


%	The proof is a double induction first on the length and then on the structure of $\red$.
%	Induction on the length of $\red$.

%	The basic step is trivial
%	Induction hypothesis:
%	If $P \red^* P''$ then
%	$\Gamma; \Lambda; \Sigma \proves P \wbc \Gamma''; \Lambda''; \Sigma'' \proves P'' \hastype \Proc$.
%
%	Induction step:
%	Let $P'' \red P'$. We do an induction on the structure of $\red$
%	to show that $\Gamma'; \Lambda'; \Sigma' \proves P'' \wbc \Gamma'; \Lambda'; \Sigma' \proves P' \hastype \Proc$.
%
	Basic step:

	\Case{$P = \bout{s}{V} P_1 \Par \binp{\dual{s}}{X} P_2$}
	\[
%		\Gamma; \emptyset; \Sigma \red \Sigma' \proves
		\bout{s}{V} P_1 \Par \binp{\dual{s}}{X} P_2 \red P_1 \Par P_2
	\]
	The proof follows from the fact that we can only observe a $\tau$
	action on typed process
	$\Gamma; \emptyset; \Sigma \proves P \hastype \Proc$.
	Actions $\bactout{s}{V}$ and $\bactinp{\dual{s}}{V}$
	are forbiden by the LTS for typed environments.

	It is easy to conclude then that $\Gamma; \emptyset; \Sigma \wb \Sigma' \proves P \wb P'$.

	\Case{$P = \bsel{s}{l} P_1 \Par \bbra{\dual{s}}{l_i: P_i}_{i \in I}$}

	Similar arguments as the previous case.

	Induction hypothesis:
	If $P_1 \red P_2$ then
	$\Gamma_1; \emptyset; \Sigma_1 \wb \Sigma_2 \proves P_1 \wb P_2$.

	Induction Step:

	\Case{$P = \news{s} P_1$}
	\[
%		\Gamma; \emptyset; \Sigma \red \Sigma' \proves
		\news{s}{P_1} \red \news{s} P_2
	\]
	From the induction hypothesis and the fact that bisimulation is a congruence
	we get that $\Gamma; \emptyset; \Sigma \wb \Sigma' \proves P \wb P'$.

	\Case{$P = P_1 \Par P_3$}

	\[
%		\Gamma; \emptyset; \Sigma \red \Sigma' \proves
		P_1 \Par P_3 \red P_2 \Par P_3
	\]
	From the induction hypothesis and the fact that bisimulation is a congruence
	we get that $\Gamma; \emptyset; \Sigma \wb \Sigma' \proves P \wb P'$.

	\Case{$P \scong P_1$}

	From the induction hypothesis and the facts that bisimulation is a congruence
	and structural congruence preserves $\wb$
	we get that $\Gamma; \emptyset; \Sigma \wb \Sigma' \proves P \wb P'$.


	The proof for part two is an induction on the length of $\red^*$.
	The basic step is trivial and the inductive step
	deploys part 1 of this lemma and the fact that bisimulation is
	transitive to conclude.
%	We can now conclude that
%	$P \wbc P'$ because $P \wbc P''$ and $P'' \wbc P'$.
\end{proof}


