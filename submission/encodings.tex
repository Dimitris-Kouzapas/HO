\newcommand{\enc}[2]{\lrangle{\map{#1}, \mapt{#2}}}
\newcommand{\encod}[3]{\lrangle{\map{#1}^{#3}, \mapt{#2}^{#3}}}

\newcommand{\calc}[2]{\lrangle{#1, #2}}

\section{Encodings}

In this section we present a study of the expressiveness 
of the sub-calculi of $\HOp$.

We first define what is an encoding over typed calculus.

\begin{definition}
	Let $\calc{L_1}{T_1}$ be the calculus with process set $L_1$ and
	session type set $T_1$ and
	$\calc{L_2}{T_2}$ the calculus with process set $L_2$ and
	session type set $T_2$.
	Assuming mappings $\map{\cdot}: L_1 \longrightarrow L_2$ and
	$\mapt{\cdot}: T_1 \longrightarrow T_2$
	we write $\enc{\cdot}{\cdot}: \calc{L_1}{T_1} \longrightarrow \calc{L_2}{T_2}$
	for the encoding from
	$\calc{L_1}{T_1}$ to $\calc{L_2}{T_2}$.
\end{definition}

\subsection{Encoding Properties}

We require that a {\em good} encoding should 
transform preserve not only the syntax but
also the operational, typing and behavioural
semantics. Formally:

\begin{definition}[Encoding Properties]
	\label{def:ep}
	Let $\Gamma; \emptyset; \Sigma \proves P \hastype \Proc$ 
	a process from calculus $\calc{L_1}{T_1}$
	and an encoding 
	$\enc{\cdot}{\cdot}: \calc{L_1}{T_1} \longrightarrow \calc{L_2}{T_2}$.
	Then
	\begin{enumerate}
		\item	$\enc{\cdot}{\cdot}$ is type preservering whenever
			$\Gamma; \emptyset; \Sigma \proves P \hastype \Proc$ implies $\mapt{\Gamma}; \emptyset; \mapt{\Sigma} \proves \map{P} \hastype \Proc$

		\item	$\enc{\cdot}{\cdot}$ satisfies operational correspondence whenever
		\begin{itemize}
			\item	If $P \red P'$ then
				$\map{P} \Red \map{P'}$ and
				$\mapt{\Gamma}; \emptyset; \mapt{\Sigma'} \proves \map{P'} \hastype \Proc$.
			\item	If $\map{P} \red \map{Q}$ then
				$\exists P'$ such that $P \red P'$ and 
				$\mapt{\Gamma};\emptyset; \mapt{\Sigma_1} \proves \map{Q} \wb \Sigma_2 \proves P' \hastype \Proc$.
		\end{itemize}
		
		\item	$\enc{\cdot}{\cdot}$ is fully abstract whenever
			$\Gamma; \emptyset; \Sigma_1 \proves P_1 \wb \Sigma_2 \proves P_2 \hastype \Proc$ if and only if
			$\mapt{\Gamma}; \emptyset; \mapt{\Sigma_1} \proves \map{P_1} \wb \mapt{\Sigma_2} \proves \map{P_2} \hastype \Proc$.
	\end{enumerate}
\end{definition}

Encoding composition is closed on the above properties.

\begin{proposition}
	Let $\encod{\cdot}{\cdot}{1}: \calc{L_1}{T_1} \longrightarrow \calc{L_2}{T_2}$ and 
	$\encod{\cdot}{\cdot}{2}: \calc{L_2}{T_2} \longrightarrow \calc{L_3}{T_3}$
	encodings that respect the properties of definition~\ref{def:ep}.
	Then $\encod{\cdot}{\cdot}{1} \cdot \encod{\cdot}{\cdot}{2}$
	also respect the properties of definition~\ref{def:ep}.
\end{proposition}

\begin{proof}
	Straightforward application of the definition of each property.
\end{proof}

\subsection{Encoding from \sesp to $\HOp$}

The semantics of the $\HOp$ are powerfull enough to
express the semantics of the standard \sesp calculus.

\paragraph{Encoding from \sesp without recursion to $\HOp$:}

We first encode the name passing semantics:

\begin{tabular}{ll}
\begin{tabular}{l}
	$\map{\bout{k}{k'} P} \defeq \bout{k}{ \abs{z}{\binp{z}{X} \appl{X}{k'}} } \map{P}$
	\\
	$\map{\binp{k}{x} P} \defeq \binp{k}{X} \newsp{s}{\appl{X}{s} \Par \bout{\dual{s}}{\abs{x} \map{P}} \inact}$
\end{tabular}
&
\begin{tabular}{l}
	 $\mapt{\btout{S_1} S_2} \defeq \btout{\dk{todo}} \mapt{S_2}$\\
	 $\mapt{\btinp{S_1} S_2} \defeq \btinp{\dk{todo}} \mapt{S_2}$
\end{tabular}
\end{tabular}








