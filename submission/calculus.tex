\section{Full Higher Order Session $\pi$}

We present the syntax and operational semantics for the
{\em Full Higher Order Session $\pi$} calculus or $\HOp$.
$\HOp$ is a simpler version of the higher order calculus developed
in \cite{tlca07}, that name and process abstraction application, as
well as implicit application of abstractions. We believe that
the higher order semantics of the calculus are small, yet expressive
enough to capture in a straightforward way the principles of
session types.

\subsection{$\HOp$ Syntax}

The syntax for $\HOp$ is defined in Figure~\ref{fig:syntax}.
We assume a set of names $S$ that range over $s, s_1, \dots$
and a dual set of $S$, $\dual{S}$ that ranges over $\dual{s}, \dual{s_1}, \dots$.
Both sets $S$ and $\dual{S}$ constitude the set of names $N$.
Name variables $x, y, z, \dots$ are taken from set $\mathsf{Var}$ and
process variable $X, Y, Z, \dots$ from set $\mathsf{PVar}$. 
Set $\mathsf{RVar}$ is the set of recursive 
variables that range over $\rvar{X}, \rvar{Y}, \dots$.
Abstactions, written $\abs{x}{P}$, are processes $P$ with a bound parameter $x$.
The set for abstractions is $\mathsf{Abs}$.
We denote either names or name variables with $k, k_1, \dots$.
Names, name variables, process variables, recursive variables or abstractions
are denoted by symbol $V, V_1 \dots$.
Note that set $N$ includes session and shared names.
We sometimes denote shared names with $a,b, c, \dots$, although $a, b, c, \dots \in S$.

Processes include the standard $\pi$-calculus prefixes for sending and receiving names.
Prefix $\bout{k}{k'} P$ denotes the sending of name $k$ over channel $k$ and then
continuing as $P$ whereas prefix $\binp{k}{x} P$ denotes the reception of a value
on channel $k$ over variable $x$ and then continue as $P$. Recursion is expressed
on the primitive recursor $\recp{X}{P}$ that binds the recursive variable/process
$\rvar{X}$ in the structure of $P$.

Higher order syntax is composed by the sending prefix $\bout{k}{\abs{x}{Q}} P$ that
denotes the sending of abstraction $\abs{x}{Q}$ over channel $k$ and then continuing
as $P$. On the receiving side prefix $\binp{k}{X} P$ denotes the reception of an abstraction
on channel $k$ and over the process variable $\X$. Process $\appl{X}{k}$ is the application
process which is used to bind channel $k$ on the process substituting process variable $\X$.

We assume the standard session syntax for selection and branching. Process
$\bsel{k}{l} P$ selects label $l$ on channel $k$ and then behaves as $P$. Process 
$\bbra{k}{l_i:P_i}_{i \in I}$ offers a choice on labels $l_i, i \in I$ with
corresponding continuations the processes $P_i, i \in I$.
The calculus is completed with the standard processes for parallel composition $P_1 \Par P_2$,
name restriction $\news{s} P$ and the inactive process $\inact$.

A well formed process relies on the assumptions for guarded recursive processes.
We say that a process is a \emph{program} if it contain 
no free variables or free process/recursive variables.

\begin{figure}[t!]
%	\begin{tabular}{l|l}
		\begin{tabular}{lrcl}
			1. &	$P,Q$
			 	&$\bnfis$&	$\bout{k}{k'} P \bnfbar \binp{k}{x} P \bnfbar \rvar{X} \bnfbar \recp{X}{P}$ \\
			2. & 	&$\bnfbar$&	$\bbout{k}{\abs{x} Q} P \bnfbar \binp{k}{X} P \bnfbar \appl{X}{k}$ \\ 
			3. & 	&$\bnfbar$&	$\bsel{k}{l} P \bnfbar \bbra{k}{l_i:P_i}_{i \in I} \bnfbar 
						P_1 \Par P_2 \bnfbar \news{s} P \bnfbar \inact$ \\
			4. & 	&$\bnfbar$&	$\bout{k}{\tilde{k'}} P \bnfbar \binp{k}{\tilde{x}} P$\\
			5. & 	&$\bnfbar$&	$\bout{k}{\abs{\tilde{x}} Q} P \bnfbar \appl{X}{\tilde{k}}$\\
		&	Names & : & 	$S = \set{a, b, c, m, n, s, t, \dots} \qquad \dual{S} = \set{\dual{n} \setbar n \in S} \qquad N = S \cup \dual{S}$\\
		&	Variables & : &	$\mathsf{Var} = \set{x,y,z, \dots} \qquad \mathsf{PVar} = \set{\varp{X}, \varp{Y}, \varp{Z}, \dots}
					\qquad \mathsf{RVar} = \set{r, \dots}$\\
		&		& &	$\mathsf{Vars} = \mathsf{Var} \cup \mathsf{PVar} \cup \mathsf{RVar}$\\
		&	Abstractions & : & $\mathsf{Abs} = \set{\abs{x}{P} \setbar P \textrm{ is a process}}$\\
		&		& &	$k \in N \cup \mathsf{Var} \quad \qquad V \in N^n \cup \mathsf{Vars}^n \cup \mathsf{Abs}\ \forall n$
		\end{tabular}
	\caption{Syntax for $\HOp$ \label{fig:syntax}}
\end{figure}



We identify two important sub-calculi of the $\HOp$ that will form
the basis of the study in this paper, i) The calculus defined by the lines 1 and 3 of the
process syntax in Figure~\ref{fig:syntax}, and ii) The calculus defined by the lines 2 and 3
of the process syntax in Figure~\ref{fig:syntax}. The first calculus is the standard
session $\pi$ calculus or \sesp as defined in the bibliography~\cite{}. The second
calculus allows only for abstraction passing and abstraction application as well 
as selecting and branching and it is called Higher Order session calculus or $\HO$.


\subsection{Operational Semantics}

\paragraph{Structural Congruence}

We define structural congruence as the least congruence that satisfies the commutative monoid $(\Par, \inact)$
and the rules:

\begin{tabular}{c}
%	P \Par \inact \scong P \qquad P_1 \Par P_2 \scong P_2 \Par P_1 \qquad P_1 \Par (P_2 \Par P_3)
%	\qquad (P_1 \Par P_2) \Par P_3 \qquad
	$\news{s} \inact \scong \inact \quad
	s \notin \fn{P_1}$ implies $P_1 \Par \news{s} P_2 \scong \news{s}(P_1 \Par P_2)
	\quad \rec{X}{P} \scong P\subst{\rec{X}{P}}{\rvar{X}}$
\end{tabular}


We define the reduction semantics in Figure~\ref{fig:reduction}.
Figure~\ref{fig:reduction} first describes the process variable
substitution through the semantics
of name substitution. Substitution of application process $\appl{X}{k}$
over abstraction $\abs{x} Q$ substitutes free variable
$x$ in $Q$ with $k$ and replaces $X$ with the resulting process.
There is no effect on variable substitution for the inactive process.
In all the other cases process variable substitution is homorphic 
on the structure of the process.

\begin{figure}
\[
	\begin{array}{c}
		\begin{array}{c}
			\bout{s}{s'} P_1 \Par \binp{\dual{s}}{x} P_2 \red P_1 \Par P_2 \subst{s'}{x}\ \orule{ΝPass} \\
			\bout{s}{\abs{x}{P}} P_1 \Par \binp{\dual{s}}{\X} P_2 \red P_1 \Par P_2 \subst{\abs{x}{P}}{\X}\ \orule{APass}
			\\
			\tree{
				k \in I
			}{
				\bsel{s}{l_k} P \Par \bbra{\dual{s}}{l_i : P_i}_{i \in I} \red P \Par P_k 
			}\ \orule{Sel}
			\quad 
			\tree{
				P \red P'
			}{
				\news{s} P \red \news{s} P'
			}\ \orule{Sess}
			\\[4mm]
			\tree{
				P_1 \red P_1'
			}{
				P_1 \Par P_2 \red P_1' \Par P_2 
			}\ \orule{Par}
			\quad
			\tree{
				P \scong \red \scong P'
			}{
				P \red P'
			}\ \orule{Cong}
		\end{array}
		\\
		\begin{array}{c}
			\appl{X}{k} \subst{\abs{x}{Q}}{\X} = Q \subst{k}{x}
			\qquad
			\inact \subst{\abs{x}{Q}}{\X} = \inact
			\\
			\binp{s}{\Y} P \subst{\abs{x}{Q}}{\X} = \binp{s}{\Y} (P \subst{\abs{x}{Q}}{\X})
			\\
			(\bout{s}{\abs{y}{P_1}} P_2) \subst{\abs{x}{Q}}{\X} = \bout{s}{\abs{y}{P_1 \subst{\abs{x}{Q}}{\X}}} (P_2 \subst{\abs{x}{Q}}{\X})
			\\
			\bsel{s}{l} P \subst{\abs{x}{Q}}{\X} = \bsel{s}{l} (P \subst{\abs{x}{Q}}{\X})
			\\
			\bbra{s}{l_i : P_i}_{i \in I} \subst{\abs{x}{Q}}{\X} = \bbra{s}{l_i : P_i \subst{\abs{x}{Q}}{\X}}_{i \in I}
			\\
			(P_1 \Par P_2) \subst{\abs{x}{Q}}{\X} = P_1 \subst{\abs{x}{Q}}{\X} \Par P_2 \subst{\abs{x}{Q}}{\X}
			\qquad
			\news{s} P \subst{\abs{x}{Q}}{\X} = \news{s} (P \subst{\abs{x}{Q}}{\X})
		\end{array}
	\end{array}
\]
	\caption{The reduction semantics for $\HOp$ \label{fig:reduction}}
\end{figure}


There are three communication rules for the $\HOp$. The name passing
rule, $\orule{NPass}$ where a send prefix on channel $s$ sends
channel $s'$ to be received by parallel receive prefixed process
on the dual endpoint $\dual{s}$. After the reduction the continuation of the 
latter process substitutes $s'$ on the receive prefix object $x$.
The second communication axiom, $\orule{APass}$, describes
abstraction passing. An abstraction is sent on a send prefixed process
over channel $s$ and it is being received by a parallel $\dual{s}$-received
prefixed process. The latter process follows a process variable
substitution.
The third axiom is the standard select/branch action where
a process select label $l_k, k \in I$ on channel $s$ over a set of
labels $\set{l_i}_{i \in I}$ that are offered from a parallel process
on the dual session endpoint $\dual{s}$. The continuation of the latter
process is the corresponding process $P_k$.
The last three rules are congruence rules that preserve reduction
on the parallel, rule $\orule{Par}$ and name restriction, rule $\orule{Ses}$,
operators as well as reduction is closed under structural congruence, rule $\orule{Cong}$.



