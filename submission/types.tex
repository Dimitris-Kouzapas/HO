\section{Types for $\HOp$}

We define a session type system for $\HOp$, which is based on the type system developed by Mostrous
and Yoshida in~\cite{tlca07}.

\subsection{Session Types}
We consider a minimal type structure, a fragment of that defined in~\cite{tlca07}.
The only (but fundamental) differences are in the types for values: we focus on having 
$\shot{S}$ and $\lhot{S}$, whereas the structure in~\cite{tlca07} supports general functions $U \sharedop T$ and 
$U \lollipop T$.
\[
	\begin{array}{lcl}
		\text{\emph{Values}} & U ::= & S \bnfbar \lhot{S} \bnfbar \shot{S} \bnfbar \chtype{S} \qquad \quad \text{\emph{Terms}} \quad T ::= U  \bnfbar  \Proc\\
		\text{\emph{Sessions}} \ & S ::= &  \btout{U} S \bnfbar \btinp{U} S
		\bnfbar		\btsel{l_i:S_i}_{i \in I} \bnfbar \btbra{l_i:S_i}_{i \in I} \bnfbar \trec{t}{S} \bnfbar \vart{t}  \bnfbar \tinact 
	\end{array}
\]

There are four different value types $U$; session value $S$, linear higher order value $\lhot{S}$, 
shared higher order value $\shot{S}$; shared channel $\chtype{S}$. Terms can either have a
value type $U$ or a process type $\Proc$.

Session types follow the standard binary session types syntax \cite{}. Session send prefix $\btout{U} S$ 
denotes a session type that sends a value of type $U$ and continues as $S$. Dually receive prefix $\btinp{U} S$
denotes a session type that receives a value of type $U$ and continues as $S$. 
Set $\mathsf{ST}$ is the space of all session types.

\begin{definition}[Duality]
	Let function $F(R): \mathsf{ST} \longrightarrow \mathsf{ST}$ to be defined as:

	\begin{tabular}{rcl}
		$F(R)$ &$=$&		$\set{(\tinact, \tinact), (\vart{t}, \vart{t})}$\\
			&$\cup$&	$\set{(\btout{U};S_1, \btinp{U}; S_2), (\btinp{U};S_1, \btout{U}; S_2) \bnfbar S_1\ R\ S_2}$\\
			&$\cup$&	$\set{(\btsel{l_i: S_i}_{i \in I}, \btbra{l_j: S_j'}_{j \in J}) \bnfbar I \subseteq J, S_i\ R\ S_j'}$\\
			&$\cup$&	$\set{(\btbra{l_i: S_i}_{i \in I}, \btsel{l_j: S_j'}_{j \in J}) \bnfbar J \subseteq I, S_j\ R\ S_i'}$\\
			&$\cup$&	$\set{(\trec{t}{S_1}, \trec{t}{S_2}) \cup (S_1 \subst{\trec{t}{S}}{\vart{t}}, S_2), (S_1, S_2\subst{\trec{t}{S}}{\vart{t}}) \bnfbar S_1\ R\ S_2)}$
	\end{tabular}

	Standard arguments ensure that $F$ is monotone, thus the greatest fix point
	of $F$ exists and let duality defined as $\dualof = \nu X. F(X)$.
\end{definition}


Following our decision of focusing on functions $\shot{S}$ and $\lhot{S}$,
our environments are also simpler than those in~\cite{tlca07}:
\[
	\begin{array}{lcl}
		\text{Shared} & \Gamma \bnfis & \emptyset \bnfbar \Gamma \cat \varp{X}: \shot{S} \bnfbar \Gamma \cat k: \chtype{S} \bnfbar \Gamma \cat \rvar{X}: \Sigma \\
		\text{Linear} & \Lambda \bnfis & \emptyset \bnfbar \Lambda \cat \varp{X}: \lhot{S}  \\
		\text{Session} \quad & \Sigma \bnfis & \emptyset \bnfbar \Sigma \cat k:S  
	\end{array}
\]


With these environments the shape of judgments is exactly the same as in Mostrous and Yoshida's system:
\[
	\begin{array}{c}
		\Gamma; \Lambda ; \Sigma \proves P \hastype T
%		\Gamma; \Lambda; \Sigma \proves V \hastype U
	\end{array}
\]
As expected, weakening, contraction, and exchange principles apply to $\Gamma$;
environments $\Lambda$ and $\Sigma$ behave linearly, and are only subject to exchange.
We require that the domains of $\Gamma, \Lambda$ and $\Sigma$ are pairwise distinct.
%We focus on \emph{well-formed} judgments, which do not share elements in their domains.
%\newcommand{\jrule}[3]{\displaystyle \frac{#1 }{#2} & \trule{#3}}
\newcommand{\jrule}[3]{\displaystyle \trule{#3}~~\frac{#1 }{#2}}
\newcommand{\addenv}{\circ}

\begin{figure}[!t]
\[
	\begin{array}{c}
%		\jrule{ }{\Gamma ; \emptyset; \emptyset \vdash \UnitV \hastype \Unit}{Unit} 
%		\qquad\quad  
		\trule{Session}~~\Gamma; \emptyset; \set{k:S} \proves k \hastype S 
		\\[2mm]
		\trule{Shared}~~\Gamma \cat a : \chtype{S}; \emptyset; \emptyset \proves a \hastype \chtype{S}
		\qquad
		\trule{LVar}~~\Gamma; \set{X: \lhot{S}}; \emptyset \proves X \hastype \lhot{S} 
		\\[2mm]
		\trule{Prom}~~\tree{
			\Gamma; \emptyset; \emptyset \proves V \hastype \lhot{S}
		}{
			\Gamma; \emptyset; \emptyset \proves V \hastype \shot{S}
		} 
		\qquad\quad  
		\trule{Derelic}~~\tree{
			\Gamma; \Lambda \cat X{:}\lhot{S}; \Sigma \proves P \hastype \Proc
		}{
			\Gamma \cat X:\shot{S}; \Lambda; \Sigma \proves P \hastype \Proc
		} 
		\\[4mm]
%		\trule{Subt}~~\tree{
%			\Gamma; \Lambda; \Sigma \proves P \hastype T \quad \Sigma \subt \Sigma' \quad T \subt T'
%		}{
%			\Gamma ; \Lambda; \Sigma' \vdash P \hastype T'
%		} 
%		\qquad\quad

		\trule{Abs}~~\tree{
			\Gamma; \Lambda; \Sigma \cat x: S \proves P \hastype \Proc
		}{
			\Gamma; \Lambda; \Sigma \proves \abs{x}{P} \hastype \lhot{S}
		}
		\\[4mm]

		\trule{App}~~\tree{(U = \lhot{S}) \lor (U = \shot{S}) \quad \Gamma; \Lambda_1; \Sigma_1 \proves X \hastype U  \quad \Gamma; \Lambda_2; \Sigma_2 \proves k \hastype S
		}{
			\Gamma; \Lambda_1 \cup \Lambda_2; \Sigma_1 \cup \Sigma_2 \proves \appl{X}{k} \hastype \Proc
		} 
		\\[4mm]

		\trule{Send}~~\tree{
			\Gamma; \Lambda_1; \Sigma_1 \proves P \hastype \Proc  \quad \Gamma; \Lambda_2; \Sigma_2 \vdash V \hastype U  \quad (k:S \in \Sigma_1 \cup \Sigma_2)
		}{
			\Gamma; \Lambda_1 \cup \Lambda_2; (\Sigma_1 \cup \Sigma_2)\backslash\set{k:S} \cat k:\btout{U} S \proves \bout{k}{V} P \hastype \Proc
		}

		\\[4mm]
		\trule{Conn}~~\tree{
			\Gamma; \Lambda; \Sigma \cat x:S \proves P \hastype \Proc  \quad \Gamma; \emptyset; \emptyset \proves a \hastype \chtype{S}
		}{
			\Gamma; \Lambda; \Sigma \proves \binp{a}{x} P \hastype \Proc
		}
		\\[4mm]
%		\trule{ConnDual}~~\tree{
%			\Gamma; \Lambda; \Sigma \cat x: S_1 \proves P \hastype \Proc  \quad \Gamma; \emptyset; \emptyset \proves k \hastype \chtype{S_2} \quad S_1 \dualof S_2
%		}{
%			\Gamma; \Lambda; \Sigma \proves \bout{k}{x} P \hastype \Proc
%		}
%		\\[4mm]

		\trule{ConnDual}~~\tree{
			\Gamma; \Lambda; \Sigma \cat \dual{s}: S_1 \proves P \hastype \Proc  \quad \Gamma; \emptyset; \emptyset \proves a \hastype \chtype{S_2} \quad S_1 \dualof S_2
		}{
			\Gamma; \Lambda; \Sigma  \proves \bout{a}{\dual{s}} P \hastype \Proc
		}

		\\[4mm]

		\trule{NewSh}~~\tree{
			\Gamma\cat a:\chtype{S} ; \Lambda; \Sigma \proves P \hastype \Proc
		}{
			\Gamma; \Lambda; \Sigma \proves \news{a} P \hastype \Proc}
		\qquad\quad
		\trule{NewSes}~~\tree{
			\Gamma; \Lambda; \Sigma \cat s:S_1 \cat \dual{s}: S_2 \proves P \hastype \Proc \quad S_1 \dualof S_2
		}{
			\Gamma; \Lambda; \Sigma \proves \news{s} P \hastype \Proc
		}
		\\[4mm]

		\trule{RecvS}~~\tree{
			\Gamma; \Lambda; \Sigma \cat k: S_1 \cat x: S_2 \proves P \hastype \Proc
		}{
			\Gamma; \Lambda; \Sigma, k: \btinp{S_2} S_1  \vdash \binp{k}{x}P \hastype \Proc
		}
		\quad\quad 
		\trule{RecvL}~~\tree{
			\Gamma; \Lambda \cat X: \lhot{S}; \Sigma \cat k: S_1  \proves P \hastype \Proc
		}{
			\Gamma; \Lambda; \Sigma \cat k:\btinp{\lhot{S}}S_1  \proves \binp{k}{X}P \hastype \Proc
		}
		\\[4mm]

		\trule{RecvSh}~~\tree{
			\Gamma \cat X: \shot{S}; \Lambda; \Sigma \cat k: S_1  \proves P \hastype \Proc
		}{
			\Gamma; \Lambda; \Sigma \cat k:\btinp{\shot{S}}S_1  \proves \binp{k}{X}P \hastype \Proc
		}
		\quad ~~
		\trule{RecvShN}~~\tree{
			\Gamma \cat x: \chtype{S}; \Lambda; \Sigma \cat k: S_1  \proves P \hastype \Proc
		}{
			\Gamma; \Lambda; \Sigma \cat k:\btinp{\chtype{S}}S_1  \proves \binp{k}{x}P \hastype \Proc
		}
		\\[4mm]
		\trule{Par}~~\tree{
			\Gamma; \Lambda_{1}; \Sigma_{1} \proves P_{1} \hastype \Proc \quad \Gamma; \Lambda_{2}; \Sigma_{2} \proves P_{2} \hastype \Proc
		}{
			\Gamma; \Lambda_{1} \cup \Lambda_2; \Sigma_{1} \cup \Sigma_2 \proves P_1 \Par P_2 \hastype \Proc
		}
		\qquad\quad
		\trule{Close}~~\tree{
			\Gamma; \Lambda; \Sigma  \proves P \hastype T \quad k \not\in \dom{\Gamma, \Lambda,\Sigma}
		}{
			\Gamma; \Lambda; \Sigma \cat k: \tinact  \proves P \hastype \Proc
		}
		\\[4mm]
		\trule{Bra}~~\tree{
			 \forall i \in I \quad \Gamma; \Lambda; \Sigma \cat k:S_i \proves P_i \hastype \Proc
		}{
			\Gamma; \Lambda; \Sigma \cat k: \btbra{l_i:S_i}_{i \in I} \proves \bbra{k}{l_i:P_i}_{i \in I}\hastype \Proc
		}
		\qquad\quad 
	 	\trule{Sel}~~\tree{
			\Gamma; \Lambda; \Sigma \cat k: S_j  \proves P \hastype \Proc \quad j \in I
		}{
			\Gamma; \Lambda; \Sigma \cat k:\btsel{l_i:S_i}_{i \in I} \proves \bsel{s}{l_j} P \hastype \Proc
		}
		\\[4mm]

		\trule{Nil}~~\Gamma; \emptyset; \emptyset \proves \inact \hastype \Proc
\qquad \quad
		\trule{Var}~~\tree{
	
		}{
			\Gamma \cat \rvar{X}: \Sigma; \emptyset; \emptyset  \proves \rvar{X} \hastype \Proc
		}
		\qquad\quad 
%	 	\trule{Rec}~~\tree{
%			\Gamma \cat \rvar{X}: \Sigma; \emptyset; \emptyset  \proves P \hastype \Proc
%		}{
%			\Gamma ; \emptyset; \emptyset  \proves \recp{X}{P} \hastype \Proc
%		}
%		\\[4mm]

	 	\trule{Rec}~~\tree{
			\Gamma \cat \rvar{X}: \Sigma; \emptyset; \Sigma  \proves P \hastype \Proc
		}{
			\Gamma ; \emptyset; \Sigma  \proves \recp{X}{P} \hastype \Proc
		}


	\end{array}
\]
\caption{Typing Rules for $\HOp$\label{fig:typerulesmy}}
\end{figure}


The typing rules for our system are in Fig.~\ref{fig:typerulesmy}. 

\subsection{Type Soundness}
We state results for type safety:
we report instances of more general statements already proved by Mostrous and Yoshida in the asynchronous case.

\begin{lemma}[Substitution Lemma - Lemma C.10 in M\&Y]
	\begin{enumerate}[1.]
		\item	Suppose $\Gamma \cat X:\shot{S}; \Lambda; \Sigma  \proves P \hastype T$ and
			$\Gamma; \emptyset ; \emptyset  \proves V \hastype \shot{S}$.
			Then $\Gamma; \Lambda; \Sigma  \proves P\subst{V}{x} \hastype T$.
	 
		\item	Assume $\Gamma; \Lambda_1 \cat X:\lhot{S}; \Sigma_1  \proves P \hastype T$ 
			and $\Gamma; \Lambda_2; \Sigma_2  \proves V \hastype \lhot{S}$ with 
			$\Lambda_1, \Lambda_2$ and $\Sigma_1, \Sigma_2$ defined.  
			Then $\Gamma; \Lambda_1 \cup \Lambda_2; \Sigma_1 \cup \Sigma_2  \proves P\subst{V}{X} \hastype T$.

		\item	Suppose $\Gamma; \Lambda; \Sigma \cat x:S  \proves P \hastype T$ and
			$k \not\in \dom{\Gamma, \Lambda, \Sigma}$. 
			Then $\Gamma; \Lambda; \Sigma \cat k:S  \vdash P\subst{k}{x} \hastype T$.
			
		\item	Suppose $\Gamma \cat x:\chtype{S}; \Lambda; \Sigma \proves P \hastype T$ and
			$k \not\in \dom{\Gamma, \Lambda, \Sigma}$. 
			Then $\Gamma \cat k:\chtype{S}; \Lambda; \Sigma   \vdash P\subst{k}{x} \hastype T$.
	\end{enumerate}
\end{lemma}

\begin{proof}
In all four parts, we proceed by induction on the typing for $P$, with a case analysis on the last applied rule. Interesting parts are (1) and (2), where the definition of substitution in Fig.~\ref{fig:reduction} and strengthening are required. (TBC)
\qed
\end{proof}

\begin{definition}[Well-typed Session Environment]
	Let $\Sigma$ a session environment.
	We say that $\Sigma$ is {\em well-typed} if whenever
	$s: S_1, \dual{s}: S_2 \in \Sigma$ then $S_1 \dualof S_2$.
\end{definition}

\begin{definition}[Session Environment Reduction]
	We define relation $\red$ on session environments as:
	\begin{itemize}
		\item	$\Sigma \cat s: \btout{U} S_1 \cat \dual{s}: \btinp{U} S_2 \red \Sigma \cat s: S_1 \cat \dual{s}: S_2$
		\item	$\Sigma \cat s: \btsel{l_i: S_i}_{i \in I} \cat \dual{s}: \btbra{l_i: S_i'}_{i \in I} \red \Sigma \cat s: S_k \cat \dual{s}: S_k', \quad k \in I$.
	\end{itemize}
\end{definition}

We now state the instance of type soundness that we can derived from the Mostrous and Yoshida system.
It is worth noticing that M\&Y have a slightly richer definition of structural congruence.
Also, their statement for subject reduction relies on an ordering on typings associated to queues and other 
runtime elements (such extended typings are denoted $\Delta$ by M\&Y).
Since we are synchronous we can omit such an ordering.

\begin{theorem}[Type Soundness - Theorem 7.3 in M\&Y]
	\begin{enumerate}[1.]
		\item	(Subject Congruence) Suppose $\Gamma; \Lambda; \Sigma \proves P \hastype \Proc$.
			Then $P \scong P'$ implies $\Gamma; \Lambda; \Sigma \proves P' \hastype \Proc$.

		\item	(Subject Reduction) Suppose $\Gamma; \emptyset; \Sigma \proves P \hastype T$
			with
%			$\mathsf{balanced}(\Sigma)$. 
			well-typed $\Sigma$.
			Then $P \red P'$ implies $\Gamma; \emptyset; \Sigma_2  \proves P' \hastype T$
			and $\Sigma_1 \red \Sigma_2$ or $\Sigma_1 = \Sigma_2$.
	\end{enumerate}
\end{theorem}


