% !TEX root = main.tex
\noi We present two encodability results:
(1)~higher-order communication with both name-passing and recursion (\HOp) into 
higher-order communication without name-passing nor recursion (\HO) (\secref{subsec:HOpi_to_HO}); and 
(2)~\HOp into the first-order calculus with name-passing  
with recursion (\sessp) (\secref{subsec:HOp_to_sessp}).
We consider the following typed calculi (cf.~\defref{d:tcalculus}):
\begin{enumerate}[-]
	\item
	$\tyl{L}_{\HOp}=\calc{\HOp}{{\cal{T}}_1}{\hby{\tau}}{\hwb}{\proves}$,
	\item
	$\tyl{L}_{\HO}=\calc{\HO}{{\cal{T}}_2}{\hby{\tau}}{\hwb}{\proves}$,
	\item
	$\tyl{L}_{\sessp}=\calc{\sessp}{{\cal{T}}_3}{\hby{\tau}}{\fwb}{\proves}$ 
\end{enumerate}
%
where: 
${\cal{T}}_1$, ${\cal{T}}_2$, 
and ${\cal{T}}_3$
are sets of types of $\HOp$, $\HO$, and $\sessp$, respectively. 
The typing $\proves$ is defined in \figref{fig:typerulesmy}.
The LTSs follow the intuitions given in \secref{ss:equiv}.
%are as in \defref{def:rlts}, 
Moreover, 
$\hwb$ is as in \defref{d:hbw}, and 
$\fwb$ is as in \defref{d:fwb}.


\begin{comment}
\dk{
We further prove that the two encodings are precise.
An important issue to adress on the precissenes of
the encoding is that of type preservation. In the
context of types it is not enough to check that
the reduction semantics and the behavioural equivalences
correspond. Type preservation gives a corresponding result on the
type derivation of the encoded process, thus we can
see that the encoding is preserving a specific type
behaviour. For example consider the followin encoding
of first-order passing into higher-order passing:
\[
	\begin{array}{rcl}
		\map{\bout{n}{m} P} &=& \binp{n}{x} (\map{P} \Par \appl{x}{m})\\
		\map{\binp{n}{x} P} &=& \bout{n}{\abs{x} P} \inact
	\end{array}
\]
%
Using the above encoding we could simulate name passing, e.g:
\[
	\begin{array}{rcl}
		\bout{n}{m} P \Par \binp{\dual{n}}{x} Q &\red& P \Par Q \subst{m}{x}\\
		\map{\bout{n}{m} P \Par \binp{\dual{n}}{x} Q} &=&\\
		\binp{n}{x} (\map{P} \Par \appl{x}{m}) \Par \bout{\dual{n}}{\abs{x} \map{Q}} \inact &\red&
		\map{P} \Par \appl{\abs{x}{\map{Q}}}{m}\\
		&\red&
		\map{P} \Par \map{Q}\subst{m}{x}
	\end{array}
\]
The distinctive characteristic of this encoding is that
it encodes the output prefix into an input prefix, and the
input prefix into an output prefix which in turn has
an impact on the relation on the session type of the encoding.
The interaction structure of the session is not preserved by the
above encoding.
The mapping on types gives an insight on the behaviour of processes.
}
\end{comment}


\subsection{From \HOp to \HO}
\label{subsec:HOpi_to_HO}
\noi We show that $\HO$ is expressive enough to
represent the full
 \HOp-calculus.
The main challenge is to encode (1) name passing 
and (2) recursion, 
for which 
we only use  abstraction passing. For (1), we pass  
an % simple 
abstraction which enables to use the name upon application. 
For (2), we 
copy a process upon reception; the case of linear abstraction passing
%presents a limitation 
is \NY{delicate} 
because 
linear abstractions cannot be copied.
To handle linearity, we define 
%a preliminary tool which is a mapping from
a mapping from processes with free names to processes without free
names (but with free variables instead).
%from processes \jpc{with free names} to processes without free names (but with free variables) (\defref{d:auxmap}). 
To this end, we require two auxiliary definitions.

\smallskip 

\begin{definition}\rm 
\label{def:hop_to_ho}
	Let $\vmap{\cdot}: 2^{\mathcal{N}} \longrightarrow \mathcal{V}^\omega$
	be a map of sequences of lexicographically ordered names to sequences of variables, defined
	inductively as: 
	$\vmap{\epsilon} = \epsilon$ and $\vmap{n \cat \tilde{m}} = x_n \cat \vmap{\tilde{m}}$. 
\end{definition}

\smallskip 

%\noi The following auxiliary mapping transforms processes
%with free names into abstractions and it is
%used in \defref{d:enc:hopitoho}.
%
%\smallskip 

\begin{definition}[Auxiliary Mapping] \label{d:trabs}\label{d:auxmap}
	Let $\sigma$ be a set of session names.
	\figref{f:auxmap} defines an auxiliary mapping
	$\auxmapp{\cdot}{{}}{\sigma}: \HO \to \HO$.
\end{definition}

%
\begin{figure}[t]
\[%\small
	\begin{array}{c}
		\auxmapp{\inact}{{}}{\sigma}  \defeq  \inact
		\qquad \auxmapp{\bout{n}{\abs{x}{Q}} P}{{}}{\sigma} \defeq \bout{u}{\abs{x}{\auxmapp{Q}{{}}{\sigma}}} \auxmapp{P}{{}}{\sigma}
		\qquad
		\auxmapp{\news{n} P}{{}}{\sigma} \defeq \news{n} \auxmapp{P}{{}}{{\sigma \cat n}}
		\\[1mm]

		\auxmapp{P \Par Q}{{}}{\sigma} \defeq \auxmapp{P}{{}}{\sigma} \Par \auxmapp{Q}{{}}{\sigma} 
		\qquad
		\auxmapp{\appl{x}{n}}{{}}{\sigma} \defeq \appl{x}{u}
		\qquad
		\auxmapp{\appl{(\lambda x.Q)}{n}}{{}}{\sigma}  \defeq \appl{(\lambda x.\auxmapp{Q}{{}}{\sigma})}{u}
		\\[1mm]
		\auxmapp{\binp{n}{x} P}{{}}{\sigma} \defeq \binp{u}{x} \auxmapp{P}{{}}{\sigma} 
		\quad
		\auxmapp{\bsel{n}{l} P}{{}}{\sigma} \defeq \bsel{u}{l} \auxmapp{P}{{}}{\sigma} 
		\quad
		\auxmapp{\bbra{n}{l_i:P_i}_{i \in I}}{{}}{\sigma} \defeq \bbra{u}{l_i:\auxmapp{P_i}{{}}{\sigma}}_{i \in I}
	\end{array}
\]
\begin{center}
	{In all cases: $u = n$ if $n\in \sigma$; otherwise $u = x_n$.}
\end{center}
\vspace{-3mm}
\caption{\label{f:auxmap} Auxiliary mapping used to encode \HOp into \HO.}
\vspace{-1mm}
%\Hlinefig 
\end{figure}


\smallskip 

\noi This way, given an \HO
 process $P$ (with $\fn{P} = m_1, \cdots, m_k$),
 we are interested in its associated abstraction, 
defined as
$\abs{x_1\cdots x_n}{\auxmapp{P}{{}}{\emptyset} }$, where $\vmap{m_j} = x_j$, for all $j \in \{1, \ldots, k\}$. 
In the following we make this intuition precise.

%This transformation from processes into abstractions can be reverted by
%using abstraction and application with an appropriate sequence of session names:
%%
%\begin{proposition}\rm
%	Let $P$ be a \HOp process and 
%	suppose $\tilde{x} = \vmap{\tilde{n}}$ where 
%$\tilde{n} = \fn{P}$.
%	Then $P \scong \appl{(\abs{\tilde{x}}{\auxmapp{P}{{}}{\emptyset}})}{\tilde{n}}$.
%%	$\appl{X}{\smap{\fn{P}}} \subst{(\vmap{\fn{P}}) \map{P}^{\emptyset}}{X} \scong P$
%\end{proposition}

\begin{figure}[t!]
\noindent{\bf Types:}
\\
$
%\begin{array}{rclcrcl}
\begin{array}{c}
	\vtmap{{S}}{1} \defeq	\lhot{(\btinp{\lhot{\tmap{S}{1}}} \tinact)}
	\qquad
	\vtmap{\chtype{S}}{1} \defeq	\lhot{(\btinp{\shot{\chtype{\tmap{S}{1}}}} \tinact)}
	\\[1mm]

	\vtmap{\chtype{L}}{1} \defeq	\lhot{(\btinp{\shot{\chtype{\tmap{L}{1}}}} \tinact)}
	\qquad
	\vtmap{\lhot{C}}{1} \defeq \lhot{\tmap{C}{1}}
	\qquad
	\vtmap{\shot{C}}{1} \defeq \shot{\tmap{C}{1}}
	\\[1mm]

	\tmap{\chtype{S}}{1} \defeq	\chtype{\tmap{S}{1}} 
	\qquad
	\tmap{\chtype{L}}{1} \defeq	\chtype{\tmap{L}{1}}
	\\[1mm]

	\tmap{\btout{U} S}{1} \defeq \btout{{\vtmap{U}{1}}} \tmap{S}{1}
	\qquad
	\tmap{\btinp{U} S}{1} \defeq \btinp{{\vtmap{U}{1}}} \tmap{S}{1}
	\\[1mm]

	\tmap{\btsel{l_i: S_i}_{i \in I}}{1} \defeq \btsel{l_i: \tmap{S_i}{1}}_{i \in I}
	\qquad
	\tmap{\btbra{l_i: S_i}_{i \in I}}{1} \defeq \btbra{l_i: \tmap{S_i}{1}}_{i \in I}
	\\[1mm]

	\tmap{\vart{t}}{1} \defeq \vart{t} \quad \tmap{\trec{t}{S}}{1}  \defeq \trec{t}{\tmap{S}{1}}
	\qquad 
	\tmap{\tinact}{1}  \defeq  \tinact
	\\[1mm]
%	\hline
%	\noindent{\bf Labels:} \quad \quad
%		\mapa{(\nu \tilde{m})\bactout{n}{m}}^{1} &\!\!\!\!\defeq\!\!\!\!&   
%(\nu \tilde{m})\bactout{n}{\abs{z}{\,\binp{z}{x} (\appl{x}{m})} } \\
%		\mapa{\bactinp{n}{m}}^{1} &\!\!\!\!\defeq\!\!\!\!&   \bactinp{n}{\abs{z}{\,\binp{z}{x} (\appl{x}{m})} } \\
%	\mapa{(\nu \tilde{m})\bactout{n}{\abs{{x}}{P}}}^{1} &\!\!\!\!\defeq\!\!\!\!& 
%(\nu \tilde{m})\bactout{n}{\abs{{x}}{\pmapp{P}{1}{\es}}}\\
%			\mapa{\bactinp{n}{\abs{{x}}{P}}}^{1} &\!\!\!\!\defeq\!\!\!\!& \bactinp{n}{\abs{{x}}{\pmapp{P}{1}{\es}}}\\
%			\mapa{\bactsel{n}{l} }^{1} \!\!\defeq\!\! \bactsel{n}{l} 
%	\quad 
%			\mapa{\bactbra{n}{l} }^{1} &\!\!\!\!\defeq\!\!\!\!& \bactbra{n}{l} 
%	\quad \quad 
%			\mapa{\tau}^{1} \!\!\defeq\!\! \tau
%\\[1mm]
%\hline
\end{array}
$

\noi{\bf Terms} :
\\
$
\begin{array}{rclcrcl}
	\pmapp{\bout{u}{w} P}{1}{f}	&\defeq&	\bout{u}{ \abs{z}{\,\binp{z}{x} (\appl{x}{w})} } \pmapp{P}{1}{f}
	&&
	\pmapp{\binp{u}{\AT{x}{C}} Q}{1}{f}	&\defeq&	\binp{u}{y} \newsp{s}{\appl{y}{s} \Par \bout{\dual{s}}{\abs{x}{\pmapp{Q}{1}{f}}} \inact}
	\\[1mm]

	\pmapp{\bout{u}{\abs{{x}}{Q}} P}{1}{f}  &\defeq& \bout{u}{\abs{{x}}{\pmapp{Q}{1}{f}}} \pmapp{P}{1}{f}
	&&
	\pmapp{\binp{u}{\AT{x}{L}} P}{1}{f} &\defeq& \binp{u}{x} \pmapp{P}{1}{f}
	\\[1mm]

	\pmapp{\bsel{s}{l} P}{1}{f} &\defeq& \bsel{s}{l} \pmapp{P}{1}{f}
	&&
	\pmapp{\bbra{s}{l_i: P_i}_{i \in I}}{1}{f} &\defeq& \bbra{s}{l_i: \pmapp{P_i}{1}{f}}_{i \in I}
	\\[1mm]

	\pmapp{\inact}{1}{f} &\defeq& \inact
	&&
	\pmapp{\news{n} P}{1}{f} &\defeq& \news{n} \pmapp{P}{1}{f}
	\\[1mm]

	\pmapp{{x}\, {u}}{1}{f} &\defeq& \appl{x}{u}
	& &
	\pmapp{\appl{(\abs{x}{Q})}{u}}{1}{f} &\defeq& \appl{(\abs{x}{\pmapp{Q}{1}{f}})}{u}
	\\[1mm]

	\pmapp{P \Par Q}{1}{f} & \!\!\defeq\!\! &\pmapp{P}{1}{f} \Par \pmapp{Q}{1}{f}
	\\
	\pmapp{\recp{X}{P}}{1}{f} &\defeq& 
	\multicolumn{5}{l}{
		\newsp{s}{\bout{\dual{s}}{\abs{(\vmap{\tilde{n}}, y)} \,{\binp{y}{z_\X} \auxmapp{\pmapp{P}{1}{{f,\{\rvar{X}\to \tilde{n}\}}}}{{}}{\es}}} \inact \Par  \binp{s}{z_\X} \pmapp{P}{1}{{f,\{\rvar{X}\to \tilde{n}\}}}}
	\quad
	(\tilde{n} = \fn{P})
	}
	\\ 
	\pmapp{\rvar{X}}{1}{f} & \defeq &
	\multicolumn{5}{l}{
		\newsp{s}{\appl{z_X}{(\tilde{n}, s)} \Par \bbout{\dual{s}}{ \abs{(\vmap{\tilde{n}},y)}{\appl{z_X}{(\vmap{\tilde{n}}, y)}}} \inact}  \quad (\tilde{n} = f(\rvar{X}))
	}
\end{array}
$
\\[1mm]
%
Above $\fn{P}$ denotes a lexicographically ordered sequence  of free names in $P$.
The input bound variable $x$ is annotated by a type to distinguish first- and higher-order cases.
\vspace{-1mm}
\caption{\label{f:enc:hopi_to_ho}Encoding of \HOp into \HO (\defref{d:enc:hopitoho}).}
\vspace{-1mm}
%(cf.~\defref{d:enc:fotohorec}).
%Mappings 
%$\map{\cdot}^2$,
%$\mapt{\cdot}^2$, 
%and 
%$\mapa{\cdot}^2$
%are homomorphisms for the other processes/types/labels. 
%\Hlinefig
\end{figure}



\smallskip 

\begin{definition}[Typed Encoding of \HOp into \HO]
\label{d:enc:hopitoho}
Let $f$ be a function from process variables to sequences of name variables.
%
%Let $\tyl{L}_{\HOp}=\calc{\HOp}{{\cal{T}}_1}{\hby{\ell}}{\wb_H}{\proves}$
%and 
%$\tyl{L}_{\HO}=\calc{\HO}{{\cal{T}}_2}{\hby{\ell}}{\wb_H}{\proves}$. 
%where 
%${\cal{T}}_1$ and ${\cal{T}}_2$ are sets of types of $\HOp$ 
%and $\HO$, respectively, 
%the typing $\proves$ is defined in 
%\figref{fig:typerulesmy} 
%and $\hwb$ is defined in \defref{d:hbw}. 
We define the typed encoding 
$\enco{\map{\cdot}^{1}_f, \mapt{\cdot}^{1} %, \mapa{\cdot}^{1}
}: \tyl{L}_{\HOp} \to \tyl{L}_{\HO}$ in 
\figref{f:enc:hopi_to_ho}. 
We assume that the mapping $\mapt{\cdot}^{1}$ on types is extended to 
session environments $\Delta$
and
shared environments $\Gamma$ homomorphically with: 
\[
	\begin{array}{l}
%	    \mapt{\Delta \cat s: S}^{1} & =  & \mapt{\Delta}^{1} \cat s:\mapt{S}^{1} & \\
%		\mapt{\Gamma \cat u: \chtype{S}}^{1} & =  & \mapt{\Gamma}^{1} \cat u:\chtype{\mapt{S}^{1}} & \\
%		\mapt{\Gamma \cat u: \chtype{L}}^{1} & = &  \mapt{\Gamma}^{1} \cat u:\chtype{\mapt{L}^{1}} & \\
		\tmap{\Gamma \cat \varp{X}:\Delta}{1}  =  \tmap{\Gamma}{1} \cat z_X:\shot{(S_1,\ldots,S_m,S^*)} \ 
	\end{array}
\]
with
$S^* = \trec{t}{\btinp{\shot{(S_1,\ldots,S_m,\vart{t})}} \tinact}$
%and $\Delta = \{n_1:S_1, \ldots, n_m:S_m\}$. 
and $\Delta = \{n_i:S_i\}_{1\leq i\leq m}$. 
\end{definition}

\smallskip 
\noi Note that $\Delta$ in $\varp{X}:\Delta$ is mapped to a non-tail
recursive session type with variable $z_X$ (see recursion mappings in \figref{f:enc:hopi_to_ho}).
Non-tail
recursive session types \jpc{have been} studied in~\cite{DBLP:journals/corr/abs-1202-2086,TGC14};
\jpc{to our knowledge,}
this is the first application in the
context of higher-order session types.
%which carries type variable as the last argument.  
For simplicity of the presentation, %we use the polyadic name abstraction and passing.
we use name abstractions with polyadicity.
A precise encoding of polyadicity into \HO is given in~\secref{subsec:pho}.

We explain the mapping in \figref{f:enc:hopi_to_ho}, focusing 
on {\em name passing} ($\pmapp{\bout{u}{w} P}{1}{f}$, $\pmapp{\binp{u}{x} P}{1}{f}$),  and  
{\em recursion} ($\pmapp{\recp{X}{P}}{1}{f}$ and $\pmapp{\rvar{X}}{1}{f}$). 

\myparagraph{Name passing.}
A name $w$ is   passed as an input-guarded abstraction;
the abstraction receives a higher-order
value and continues with the application of $w$ over
the received higher-order value.
%A name $m$ is being passed as an input
%guarded abstraction. 
%The input prefix receives an abstraction and
%continues with the application of $n$ over the received abstraction.
On the receiver side ($\binp{u}{x} P$)
the encoding realises a mechanism that i) receives
the input guarded abstraction, then ii) applies it on a fresh  endpoint $s$, 
and iii)~uses
the dual endpoint $\dual{s}$ to send the continuation $P$ as the abstraction
$\abs{x}{P}$. 
\NY{Then} name substitution is achieved via name application.

\myparagraph{Recursion.}
The encoding of a recursive process $\recp{X}{P}$  is delicate, for it 
must preserve the linearity of session endpoints. To this end, we
\NY{first record a mapping from recursive variable $X$ to process variable $z_X$.
Then, we encode the recursion body $P$ as a name abstraction
in which free names of $P$ are converted into name variables, using \defref{d:auxmap}.
(Notice that $P$ is first encoded into \HO and then transformed using mapping
$\auxmapp{\cdot}{{}}{\sigma}$.)
Subsequently, this higher-order value is embedded in an input-guarded 
``duplicator'' process~\cite{ThomsenB:plachoasgcfhop}. Finally, we define the encoding of $X$ 
in such a way that it
simulates recursion unfolding by 
invoking the duplicator in a by-need fashion.
That is, upon reception, the \HO abstraction which encodes  the 
recursion body $P$
%containing $\auxmapp{P}{{}}{\sigma}$ 
is duplicated: 
one copy is used to reconstitute the original recursion body $P$ (through
the application of $\fn{P}$); another copy is used to re-invoke
the duplicator when needed. % to simulate recursion unfolding.
%An example of this typed encoding is detailed in~\cite{KouzapasPY15}.
We illustrate the encoding by means of an example.
}
%The idea follows 
%a classical recursion encoding \cite{ThomsenB:plachoasgcfhop}.  
%A mapping of process $P$ is parallel composed, 
%and also being passed as an input
%guarded abstraction, parameterised also by a sequence of trigger names $\tilde{n}$. 
%We record a mapping from $z_X$ (which is a fresh variable of $X$) 
%to $\tilde{n}$, so that 
%when the abstraction is substituted to $z_\rvar{X}$ 
%(which occurs in the mapping of $P\subst{z_X}{X}$), 
%the correct $\tilde{n}$ is applied. In this way, we can 
%send and receive an abstraction which holds $P$, repeatedly. 

 


% !TEX root = main.tex
\begin{example}[The Encoding 
$\pmapp{\cdot}{1}{f}$ At Work]
Let $P = \recp{X}{\bout{a}{m} \varp{X}}$ be an \HOp process.
Its associated encoding into \HO is as follows---we note that initially $f = \emptyset$.
\begin{eqnarray*}
	\pmapp{P}{1}{f} &=&
	\newsp{s_1}{ \binp{s_1}{x} \pmapp{\bout{a}{m} \varp{X}}{1}{{f'}} \Par \bout{\dual{s_1}}{ \abs{(x_a, x_m, z)} \binp{z}{x} \auxmapp{\pmapp{\bout{a}{m} \varp{X}}{1}{{f'}}}{{}}{\es} } \inact} \\
%	&&\bout{\dual{s_1}}{ \abs{(x_a, x_m, z)} \binp{z}{x} \auxmapp{\pmapp{\bout{a}{m} \varp{X}}{1}{{\varp{X} \rightarrow x_ax_m}}}{{}}{\es} } \inact}
%\end{eqnarray*}
%\begin{eqnarray*}	
\pmapp{\bout{a}{m} \varp{X}}{1}{{ f'}} &=&
	\bout{a}{\abs{z}{\binp{z}{x} (\appl{x}{m})}} \pmapp{\varp{X}}{1}{{f'}}
	\\
	&=& \bout{a}{\abs{z}{\binp{z}{x} (\appl{x}{m})}} \newsp{s_2}{\appl{x}{(a,m, s_2)}  \Par \bout{\dual{s_2}}{\abs{(x_a, x_m, z)}{\appl{x}{(x_a, x_m, z)}}} \inact} \\
	\auxmapp{\pmapp{\bout{a}{m} \varp{X}}{1}{{f'}}}{{}}{\es}
	  & = & \auxmapp{\bout{a}{\abs{z}{\binp{z}{x} (\appl{x}{m})}} \newsp{s_2}{\appl{x}{(a,m, s_2)}  \Par \bout{\dual{s_2}}{\abs{(x_a, x_m, z)}{\appl{x}{(x_a, x_m, z)}}} \inact}}{{}}{\es}
	\\
	 & = & \bout{x_a}{\abs{z}{\binp{z}{x} (\appl{x}{x_m})}} \auxmapp{\newsp{s_2}{\appl{x}{(a,m, s_2)}  \Par \bout{\dual{s_2}}{\abs{(x_a, x_m, z)}{\appl{x}{(x_a, x_m, z)}}} \inact}}{{}}{\es}
	\\
	& = & \bout{x_a}{\abs{z}{\binp{z}{x} (\appl{x}{x_m})}} \newsp{s_2}{\appl{x}{(x_a,x_m, s_2)}  \Par \bout{\dual{s_2}}{\abs{(x_a, x_m, z)}{\appl{x}{(x_a, x_m, z)}}} \inact}
\end{eqnarray*}
where $f' = \varp{X} \rightarrow x_ax_m$.
That is, by writing $V$ to denote the process
$$
\abs{(x_a, x_m, z)} \binp{z}{x} \bout{x_a}{\abs{z}{\binp{z}{x} (\appl{x}{x_m})}} \newsp{s_2}{\appl{x}{(x_a,x_m, s_2)}  \Par \bout{\dual{s_2}}{\abs{(x_a, x_m, z)}{\appl{x}{(x_a, x_m, z)}}} \inact}
$$
we would have that $P = \recp{X}{\bout{a}{m} \varp{X}}$ is mapped into the \HO process
$$
\newsp{s_1}{\binp{s_1}{x}  \bout{a}{\abs{z}{\binp{z}{x} (\appl{x}{m})}} \newsp{s_2}{\appl{x}{(a,m, s_2)}  \Par \bout{\dual{s_2}}{\abs{(x_a, x_m, z)}{\appl{x}{(x_a, x_m, z)}}} \inact}\Par \bout{\dual{s_1}}{V} \inact}
$$
We illustrate the behavior of the encoded process (below, we let $\lambda = \bactout{a}{\abs{z}{\binp{z}{x} (\appl{x}{m})}}$):
\begin{eqnarray*}
\pmapp{P}{1}{f} & \scong & \newsp{s_1}{\bout{\dual{s_1}}{V} \inact \Par \binp{s_1}{x} \bout{a}{\abs{z}{\binp{z}{x} (\appl{x}{m})}} \newsp{s_2}{\bout{\dual{s_2}}{\abs{(x_a, x_m, z)}{\appl{x}{(x_a, x_m, z)}}} \inact}  \\
& & \qquad \Par \appl{x}{(a,m, s_2)}} \\
& \by{\tau} & \bout{a}{\abs{z}{\binp{z}{x} (\appl{x}{m})}} \newsp{s_2}{\bout{\dual{s_2}}{V} \inact \Par \binp{s_2}{x} \bout{a}{\abs{z}{\binp{z}{x} (\appl{x}{m})}} \\
& & \qquad \qquad \quad \qquad \qquad \qquad \newsp{s_3}{\bout{\dual{s_3}}{\abs{(x_a, x_m, z)}{\appl{x}{(x_a, x_m, z)}}} \inact} \Par \appl{x}{(a,m, s_3)}} \\
& \scong_{\alpha} & \bout{a}{\abs{z}{\binp{z}{x} (\appl{x}{m})}} \newsp{s_1}{\bout{\dual{s_1}}{V} \inact \Par \binp{s_1}{x} \bout{a}{\abs{z}{\binp{z}{x} (\appl{x}{m})}} \\
& & \qquad \qquad \qquad \qquad \quad \qquad \newsp{s_2}{\bout{\dual{s_2}}{\abs{(x_a, x_m, z)}{\appl{x}{(x_a, x_m, z)}}} \inact} \Par \appl{x}{(a,m, s_2)}} \\
& \scong & 
		\bout{a}{\abs{z}{\binp{z}{x} (\appl{x}{m})}} \pmapp{\recp{X}{\bout{a}{m} \varp{X}}}{1}{f} \by{\lambda} 
		\pmapp{\recp{X}{\bout{a}{m} \varp{X}}}{1}{f}
%& \by{\lambda} & 
%		\pmapp{\recp{X}{\bout{a}{m} \varp{X}}}{1}{f}
\end{eqnarray*}
The encoding preserves also typing; associated derivations are given in \cite{KouzapasPY15}.
\qed
\end{example}



We now describe the properties of the encoding. 
First, we state operational correspondence with respect to reductions; 
the full statement (and proof) can be found in~\cite{KouzapasPY15}.

\begin{proposition}[\HOp into \HO: Operational Correspondence - Excerpt]%\myrm
	\label{prop:op_corr_HOp_to_HO}
	Let $P$ be an \HOp process such that $\Gamma; \emptyset; \Delta \proves P \hastype \Proc$.
	\begin{enumerate}[1.]
		\item Completeness: 
			Suppose $\horel{\Gamma}{\Delta}{P}{\hby{\tau}}{\Delta'}{P'}$. Then we have:
%
			\begin{enumerate}[a)]
				\item
					If  $P' \scong \newsp{\tilde{m}}{P_1 \Par P_2\subst{m}{x}}$
					then $\exists R$ s.t. \\
					$\horel{\tmap{\Gamma}{1}}{\tmap{\Delta}{1}}{\pmapp{P}{1}{f}}{\hby{\tau}}{\mapt{\Delta}^{1}}{\newsp{\tilde{m}}{\pmapp{P_1}{1}{f} \Par R}}$,
					and\\ 
					$\horel{\tmap{\Gamma}{1}}{\tmap{\Delta}{1}}{\newsp{\tilde{m}}{\pmapp{P_1}{1}{f} \Par R}}{\hby{\stau} \hby{\btau} \hby{\btau}}
					{\mapt{\Delta}^{1}}{\newsp{\tilde{m}}{\pmapp{P_1}{1}{f} \Par \pmapp{P_2}{1}{f}\subst{m}{x}}}$.
			
				\item
					If  $P' \scong \newsp{\tilde{m}}{P_1 \Par P_2 \subst{\abs{y}Q}{x}}$
					then \\
					$\horel{\tmap{\Gamma}{1}}{\tmap{\Delta}{1}}{\pmapp{P}{1}{f}}{\hby{\tau}}
					{\tmap{\Delta_1}{1}}{\newsp{\tilde{m}}{\pmapp{P_1}{1}{f}\Par \pmapp{P_2}{1}{f}\subst{\abs{y}\pmapp{Q}{1}{\emptyset}}{x}}}$.
			
				\item
					If   $P' \not\scong \newsp{\tilde{m}}{P_1 \Par P_2 \subst{m}{x}} \land P' \not\scong \newsp{\tilde{m}}{P_1 \Par P_2\subst{\abs{y}Q}{x}}$
					then \\
					$\horel{\tmap{\Gamma}{1}}{\tmap{\Delta}{1}}{\pmapp{P}{1}{f}}{\hby{\tau}}{\tmap{\Delta'_1}{1}}{ \pmapp{P'}{1}{f}}$.
			\end{enumerate}
			
		\item Soundness:	Suppose $\horel{\tmap{\Gamma}{1}}{\tmap{\Delta}{1}}{\pmapp{P}{1}{f}}{\hby{\tau}}{\tmap{\Delta'}{1}}{Q}$.
			Then $\Delta' = \Delta$ and 
					either
%
					\begin{enumerate}[a)]
						\item	$\exists P'$ s.t. 
							$\horel{\Gamma}{\Delta}{P}{\hby{\tau}}{\Delta}{P'}$,
							and $Q = \map{P'}^{1}_f$.	

						\item
							$\exists P_1, P_2, x, m, Q'$ s.t. 
							$\horel{\Gamma}{\Delta}{P}{\hby{\tau}}{\Delta}{\newsp{\tilde{m}}{P_1 \Par P_2\subst{m}{x}} }$, and\\
							$\horel{\tmap{\Gamma}{1}}{\tmap{\Delta}{1}}{Q}{\hby{\stau} \hby{\btau} \hby{\btau}}{\tmap{\Delta}{1}}{\pmapp{P_1}{1}{f} \Par \pmapp{P_2\subst{m}{x}}{1}{f}}$ 
%							$Q = \map{P_1}^{1}_f \Par Q'$, where $Q'  \Hby{} $.

%						\item $\exists P_1, P_2, x, R$ s.t. 
%						$\stytra{ \Gamma }{\tau}{ \Delta }{ P}{ \Delta}{ \news{\tilde{m}}(P_1 \Par P_2\subst{\abs{y}R}{x}) }$, and 
%						$Q = \map{\news{\tilde{m}}(P_1 \Par P_2\subst{\abs{y}R}{x})}^{1}_f$.
			\end{enumerate}
		    %\end{enumerate}
		    
%		\item   
%			If  $\wtytra{\mapt{\Gamma}^{1}}{\ell_2}{\mapt{\Delta}^{1}}{\pmapp{P}{1}{f}}{\mapt{\Delta'}^{1}}{Q}$
%			then $\exists \ell_1, P'$ s.t.  \\
%			(i)~$\stytra{\Gamma}{\ell_1}{\Delta}{P}{\Delta'}{P'}$,
%			(ii)~$\ell_2 = \mapa{\ell_1}^{1}$, 
%			(iii)~$\wbb{\mapt{\Gamma}^{1}}{\ell}{\mapt{\Delta'}^{1}}{\pmapp{P'}{1}{f}}{\mapt{\Delta'}^{1}}{Q}$.
	\end{enumerate}
\end{proposition}

\noi Observe how we can explicitly distinguish the role of finite, deterministic reductions 
(\defref{def:dettrans}) in both soundness and completeness statements.

Using operational correspondence, we can show \emph{full abstraction}:
\begin{proposition}[\HOp into \HO: Full Abstraction]%\myrm
	\label{prop:fulla_HOp_to_HO}
	Let $P_1, Q_1$ be \HOp processes. \\
	$\horel{\Gamma}{\Delta_1}{P_1}{\hwb}{\Delta_2}{Q_1}$
	if and only if
	$\horel{\tmap{\Gamma}{1}}{\tmap{\Delta_1}{1}}{\pmapp{P_1}{1}{f}}{\hwb}{\tmap{\Delta_2}{1}}{\pmapp{Q_1}{1}{f}}$.
\end{proposition}

%Based on these propositions, w
We may state the main result of this section. See~\cite{KouzapasPY15} for details. 

\begin{theorem}[Precise Encoding of \HOp into \HO]
\label{f:enc:hopitoho}
The encoding from $\tyl{L}_{\HOp}$ into $\tyl{L}_{\HO}$ (cf.~\defref{d:enc:hopitoho})
is precise. 
\end{theorem}


\subsection{From \HOp to \sessp}
\label{subsec:HOp_to_sessp}
\noi 
We now discuss the encodability of  $\HO$ into $\sessp$. % where
%:
We closely follow the encoding  by 
Sangiorgi~\cite{San92,SaWabook}, but casted in the setting of session-typed communications. 
Intuitively, such an encoding  represents the exchange of a process with the exchange of a freshly generated \emph{trigger name}. 
Trigger names may then be used to activate copies of the process, which becomes a persistent resource represented by an input-guarded replication.
%Consider the following (naive) adaptation of \cite{San92,SaWabook} 
%in which session names are used are triggers and 
%exchanged processes would be have to used exactly once:
%%
%\[
%\begin{array}{l}
%		\pmap{\bout{u}{\abs{x}{Q}} P}{n}  \defeq   \newsp{s}{\bout{u}{s} (\pmap{P}{n} \Par \binp{\dual{s}}{x} \pmap{Q}{n})} \\
%		\pmap{\binp{u}{x} P}{n}  \defeq \binp{u}{x} \pmap{P}{n}
%		\quad 
%		\pmap{\appl{x}{u}}{n}  \defeq  \bout{x}{u} \inact
%	\end{array}
%\]
%%
%with the remaining \HOp constructs being mapped homomorphically.
%Although $\pmap{\cdot}{n}$ captures the correct semantics when
%dealing with systems that allow only linear abstractions,
%it suffers from untypability in the presence
%of shared abstractions. For instance,
%mapping for $P = \bout{n}{\abs{x}{\bout{x}{m}\inact}} \inact \Par \binp{\dual{n}}{x} (\appl{x}{s_1} \Par \appl{x}{s_2})$
%would be:
%%
%\[
%	\pmap{P}{n} \defeq
%	\newsp{s}{\bout{n}{s} \binp{\dual{s}}{x} \bout{x}{m} \inact \Par \binp{\dual{n}}{x} (\bout{x}{s_1} \inact \Par \bout{x}{s_2} \inact)}
%\]
%%
%The above process is untypable since processes $(\bout{x}{s_1} \inact$ and $\bout{x}{s_2} \inact)$
%cannot be put in parallel because they do not have disjoint session environments.
On the other hand, session names are linear resources and cannot be replicated.
%The correct 
Our
approach %would be to 
then uses replicated names
as triggers for shared resources and non-replicated names
for linear resources.
%as triggers instead of session names, when dealing with shared abstractions. 

\smallskip 

\begin{definition}[Typed Encoding of \HOp into \sessp]
\label{d:enc:hopitopi}
%Let $\tyl{L}_{\sessp}=\calc{\sessp}{{\cal{T}}_3}{\hby{\ell}}{\fwb}{\proves}$ 
%where the typing is defined in 
%\figref{fig:typerulesmy} 
%and the equivalence $\fwb$ is defined in \defref{d:fwb}.
%${\cal{T}}_3$ is a set of types of $\sessp$.  
%%
We define the typed encoding 
$\enco{\map{\cdot}^{2}, \mapt{\cdot}^{2} %, \mapa{\cdot}^{2}
}: \tyl{L}_{\HOp} \to \tyl{L}_{\sessp}$  
%We define the mappings $\map{\cdot}^{2}$, $\mapt{\cdot}^{2}$, $\mapa{\cdot}^{2}$
in \figref{f:enc:ho_to_sessp}. 
\end{definition}

%\smallskip 
% !TEX root = ../journal16kpy.tex

\begin{figure}[t]
{\bf Terms:} 
\begin{align*}
	\pmap{\bout{u}{\abs{x}{Q}} P}{2} & \defeq  
	\begin{cases}
		\newsp{a}{\bout{u}{a} (\pmap{P}{2} \Par \repl{\binp{a}{y} \binp{y}{x} \pmap{Q}{2}})\,}\quad
		& \text{if $\fs{Q} = \emptyset$}
		\\
		\newsp{a}{\bout{u}{a} (\pmap{P}{2} \Par \binp{a}{y} \binp{y}{x} \pmap{Q}{2})\,}\quad
		& \text{otherwise} %\dk{Q \textrm{ linear}} \\
	\end{cases}
	\\
	\pmap{\binp{u}{x} P}{2} &\defeq  \binp{u}{x} \pmap{P}{2}
	%\quad \quad \pmap{\appl{(\abs{x}{P})}{u}}{2} \!\! \defeq \!\! \pmap{P\subst{u}{x}}{2}
	 \\
	\pmap{\appl{x}{u}}{2} &\defeq \newsp{s}{\bout{x}{s} \bout{\dual{s}}{u} \inact}
	\\
	\pmap{\appl{(\abs{x}{P})}{u}}{2} & \defeq  %\newsp{s}{\bout{a}{s} \bout{\dual{s}}{u} \inact} \\
	\newsp{s}{\binp{s}{x} \pmap{P}{2} \Par \bout{\dual{s}}{u} \inact}
\end{align*}
{\bf Types:}
\begin{align*}
		\tmap{\btout{\lhot{S}}S_1}{2} & \defeq \bbtout{\chtype{\btinp{\tmap{S}{2}}\tinact}}\tmap{S_1}{2}
		\\
		\tmap{\btinp{\lhot{S}}S_1}{2} & \defeq \bbtinp{\chtype{\btinp{\tmap{S}{2}}\tinact}}\tmap{S_1}{2}
%\noindent{\bf Labels:}\ 
%		\mapa{(\nu \tilde{m})\bactout{n}{\abs{ x}{P}} }^2  & \defeq & \news{m} \bactout{n}{m} \\
%		\mapa{\bactinp{n}{\abs{ x}{P}} }^2 &  \defeq & \bactinp{n}{m}
%\quad \quad m \text{ fresh}
%\\[1mm]
%\hline
\end{align*}
%\\[2mm]
%{Notice: $\repl{} P$ means $\recp{X}{(P \Par \rvar{X})}$. 
Elided mappings are homomorphic.
%for others.  %types, labels and processes    
%\vspace{-1mm}
\caption{Encoding of \HOp into \sessp (\defref{d:enc:hopitopi}). \label{f:enc:ho_to_sessp}}
%\vspace{-1mm}
%\Hlinefig
\end{figure}


\noi 
%Notice that $\mapa{\bactinp{n}{\abs{ x}{P}} }^2$ involves a fresh trigger name (linear or shared),  which denotes the location of $\pmap{P}{2}$. 
%(a $\sessp$ process).
Observe how $\pmap{\appl{(\abs{x}{P})}{u}}{2}$ naturally induces a name substitution.
We describe key properties of this encoding. First, operational correspondence:

\begin{proposition}[\HOp into \sessp: Operational Correspondence - Excerpt]%\myrm
	\label{prop:op_corr_HOp_to_p}
	Let $P$ be an  $\HOp$ process such that  $\Gamma; \emptyset; \Delta \proves P \hastype \Proc$.
	
\begin{enumerate}[1.]
\item Completeness: Suppose $\horel{\Gamma}{\Delta}{P}{\hby{\ell}}{\Delta'}{P'}$. Then either:
				\begin{enumerate}[a)]
				\item If $\ell = \tau$ then either
				\begin{enumerate}[-]
					\item	 %such that
						$
						\horel{\tmap{\Gamma}{2}}{\tmap{\Delta}{2}}{\pmap{P}{2}}
						{\hby{\tau}}
						{\tmap{\Delta'}{2}}{}{\newsp{\tilde{m}}{\pmap{P_1}{2} \Par \newsp{a}
						{\pmap{P_2}{2}\subst{a}{x} \Par \repl{} \binp{a}{y} \binp{y}{x} \pmap{Q}{2}}}}
						$, for some  $P_1, P_2, Q$;

					\item	%$\exists R$ such that
						$
						\horel{\tmap{\Gamma}{2}}{\tmap{\Delta}{2}}{\pmap{P}{2}}
						{\hby{\tau}}
						{\tmap{\Delta'}{2}}{}{\newsp{\tilde{m}}{\pmap{P_1}{2} \Par \newsp{s}
						{\pmap{P_2}{2}\subst{\dual{s}}{x} \Par \binp{s}{y} \binp{y}{x} \pmap{Q}{2}}}}
						$, for some  $P_1, P_2, Q$;

					\item	%$\ell_1 = \btau$ and
						$\horel{\tmap{\Gamma}{2}}{\tmap{\Delta}{2}}{\pmap{P}{2}}
						{\hby{\tau}}
						{\tmap{\Delta'}{2}}{}{{\pmap{P'}{2} }}
						$

				\end{enumerate}
				\item 	If $\ell = \btau$ then 
						$\horel{\tmap{\Gamma}{2}}{\tmap{\Delta}{2}}{\pmap{P}{2}}
						{\hby{\stau}}
						{\tmap{\Delta'}{2}}{}{{\pmap{P'}{2} }}
						$.
				\end{enumerate}
		
		%%%%%%% SOUNDNESSS
		\item Suppose 
		$\stytra{\mapt{\Gamma}^{2}}{\tau}{\mapt{\Delta}^{2}}{\map{P}^{2}}{\mapt{\Delta'}^{2}}{R}$.  \\
		Then $\exists P'$ such that
					$P \hby{\tau} P'$
					and $\horel{\mapt{\Gamma}^{2}}{\mapt{\Delta'}^{2}}{\map{P'}^{2}}{\hwb}{\mapt{\Delta'}^{2}}{R}$.
	\end{enumerate}
\end{proposition}

We can show full abstraction:

\begin{proposition}[\HOp to \sessp: Full Abstraction]%\myrm
	\label{prop:fulla_HOp_to_p}
	Let $P_1, Q_1$ be \HOp processes.
	$\horel{\Gamma}{\Delta_1}{P_1}{\hwb}{\Delta_2}{Q_1}$
	if and only if
	$\horel{\tmap{\Gamma}{2}}{\tmap{\Delta_1}{2}}{\pmap{P_1}{2}}{\fwb}{\tmap{\Delta_2}{2}}{\pmap{Q_1}{2}}$.
\end{proposition}

We can state:

\begin{theorem}[Precise Encoding of \HOp into \sessp]
\label{f:enc:hotopi}
The encoding from $\tyl{L}_{\HOp}$ into $\tyl{L}_{\sessp}$ (cf.~\defref{d:enc:hopitopi})
is precise. 
\end{theorem}

%\smallskip 
%
%\begin{remark}
%As stated in  \cite[Lem.\,5.2.2]{SangiorgiD:expmpa}, 
%due to the replicated trigger,  
%operational correspondence in \defref{def:ep} is refined to prove  
%full abstraction: 
%e.g., completeness of the case $\ell_1 \neq \tau$, is changed as follows.
%Suppose   
%$\stytraarg{\Gamma}{\ell_1}{\Delta}{P}{\Delta'}{P'}{}$:
%if $\ell_1 = (\nu \tilde{m})\bactout{n}{\abs{ x}{R}}$, 
%then %$\exists \ell_2, Q$ s.t. 
%$\stytraarg{\mapt{\Gamma}^2}{\ell_2}{\mapt{\Delta}^2}{\map{P}^2}{\mapt{\Delta'}^2}{Q}{}$,
%where 
%$\ell_2 = (\nu a)\bactout{n}{a}$ and
%$Q = \pmap{P' \Par  \repl{} \binp{a}{y} \binp{y}{x} R}{2}$.
%Similarly,
%if  
%%$\stytraarg{\Gamma}{\ell_1}{\Delta}{P}{\Delta'}{P'}{}$
%%with 
%$\ell_1 = \bactinp{n}{\abs{ x}{R}}$, 
%then %$\exists \ell_2, Q$ s.t. 
%$\stytraarg{\mapt{\Gamma}^2}{\ell_2}{\mapt{\Delta}^2}{\map{P}^2}{\mapt{\Delta'}^2}{Q}{}$,
%where 
%$\ell_2 = \bactout{n}{a}$ and
%$\pmap{P'}{2} \wb \news{a}(Q \Par  \repl{} \binp{a}{y} \binp{y}{x} \pmap{R}{2})$.
%Soundness is stated in a symmetric way; see \cite{KouzapasPY15}. 
%%Operational correspondence for the encoding in~\defref{d:enc:hopitopi}
%%is different from that in~\defref{def:ep}, due to triggers. 
%%In particular,  completeness differs when $\ell_1 \neq \tau$.
%%This way, e.g., if  
%%$\stytraarg{\Gamma}{\ell_1}{\Delta}{P}{\Delta'}{P'}{}$
%%with $\ell_1 = (\nu \tilde{m})\bactout{n}{\abs{ x}{R}}$, 
%%then %$\exists \ell_2, Q$ s.t. 
%%$\stytraarg{\mapt{\Gamma}^2}{\ell_2}{\mapt{\Delta}^2}{\map{P}^2}{\mapt{\Delta'}^2}{Q}{}$,
%%where 
%%$\ell_2 = (\nu a)\bactout{n}{a}$ and
%%$Q = \pmap{P' \Par  \repl{} \binp{a}{y} \binp{y}{x} R}{2}$.
%%This 
%%statement, essential in proofs of full abstraction,
%%is the same given by Sangiorgi~\cite{SangiorgiD:expmpa}.
%%Completeness is as in~\defref{def:ep} when  $\ell_1 = \tau$.
%%See~\cite{KouzapasPY15} for details.
%\end{remark}


