%%%%%%%%%%%%%%%%%%%%%%%%%%%%%%
%%%%% ESOP16
%%%%% CAMERA READY VERSION
%%%%%%%%%%%%%%%%%%%%%%%%%%%%%%

\documentclass[runningheads]{llncs} 
\usepackage[dvipsnames]{xcolor}
\usepackage{amsmath}
\usepackage{amssymb}
\usepackage{xspace}
\usepackage{graphicx}
\usepackage{latexsym}
\usepackage{listings}
\usepackage{multirow}
\usepackage{suffix}
\usepackage{url}
\usepackage{mathptmx}
\usepackage{mathrsfs}
\usepackage{comment}
\usepackage{enumerate}
\usepackage{txfonts}
\usepackage{hyperref}
\usepackage{fancybox}
%\usepackage{space}
\usepackage{color}      % use if color is used in text

\usepackage{mathpartir}

%\usepackage{tikz}	% for drawing figures

\usepackage{caption}	% for subfigures
\usepackage{subcaption}	% for subfigures

\begin{document}


\title{On the Relative Expressiveness of\\ 
Higher-Order Session Processes
}

\author{
	Dimitrios Kouzapas\inst{1}
	\and
	Jorge A. P\'{e}rez\inst{2}
	\and Nobuko Yoshida\inst{3}
}
\authorrunning{Dimitrios Kouzapas, Jorge A. P\'{e}rez, and Nobuko Yoshida}
\institute{University of Glasgow, UK
 \and University of Groningen, The Netherlands 
 \and Imperial College London, UK}
\maketitle


\input{macros}

%\pagestyle{plain}

%% !TEX root = main.tex
\begin{abstract}
This work proposes %efficient 
tractable
bisimulations 
for the higher-order $\pi$-calculus with session primitives~(\HOp).
We develop three typed bisimulations, which are shown to 
coincide with contextual equivalence.
These characterisations  
demonstrate that observing as inputs
only a specific finite set of higher-order values (which inhabit session types) suffices 
to reason about \HOp processes. 
\end{abstract}

\begin{abstract}
By integrating
constructs from the $\lambda$-calculus and 
the $\pi$-calculus,
in \emph{higher-order process calculi} exchanged values may contain processes.
This paper studies the relative expressiveness of \HOp, 
the higher-order $\pi$-calculus in
which communications are governed by \emph{session types}. 
Our main discovery is that \HO, a subcalculus of \HOp which lacks name-passing and recursion, 
can serve as a new core calculus for session-typed higher-order
concurrency. By exploring a new bisimulation for \HO, we show that
 \HO can encode \HOp fully abstractly 
(up to typed contextual \newj{equivalence})
more precisely and efficiently than the first-order 
session $\pi$-calculus (\sessp).
Overall, under session types, 
$\HOp$, $\HO$, and $\sessp$ 
are equally expressive; but 
$\HOp$ and $\HO$ are more tightly related than 
$\HOp$ and $\sessp$.
\end{abstract}

%%%%%%%%%%%%%%%%%%%%%%%%%%%%%%%%%%%%%%%%%%%%%%%%%%%%%%%%%%%%%%%%%%%%%%%%%%%%%%%%
%%%%%%%%%%%%%%%%%%%%%%%%%%%%%%%%%%%%%%%%%%%%%%%%%%%%%%%%%%%%%%%%%%%%%%%%%%%%%%%%

\section{Introduction}
\label{sec:intro}
%% !TEX root = main.tex

%This paper is about \emph{relative expressiveness} results for 
%\emph{higher-order process calculi}, core programming languages that 
%integrate name- and process-passing in communications.
%We focus on calculi coupled with \emph{session types} that denote interaction protocols. 
%Expressiveness results allows us to 
%identify
%a \emph{core %process language % concurrency %with session primitives, 
%calculus}
%that encompasses both first- and higher-order session communication.
%Establishing such results 
%in our typed setting 
%is challenging because 
% it entails defining 
% not only a translation 
%relating source and target languages (\emph{encoding}), but also a translation 
%relating their associated session types. 
%We aim at a very particular class of correct encodings: namely \emph{fully abstract} and \emph{type-preserving} encodings.
%Next, we elaborate on our aims,   approach, and contributions.

\emph{Type-preserving compilations} are important in the design of
functional and object-oriented languages: type information has been
used to, e.g., justify code optimizations and reason about programs
(see, e.g.,
\cite{DBLP:journals/toplas/MorrisettWCG99,DBLP:conf/pldi/ShaoA95,DBLP:journals/toplas/LeagueST02}).
A vast literature on 
{\em expressiveness} 
in concurrency theory  
(e.g.,~\cite{Palamidessi03,DBLP:journals/iandc/Gorla10,DBLP:journals/tcs/FuL10,DBLP:conf/icalp/LanesePSS10,DBLP:journals/corr/PetersG15})
also studies compilations (or \emph{encodings}):
they are used to transfer reasoning techniques 
from one calculus to another, and to identify 
constructs which may be implemented
using simpler ones. 
%To a large extent, however, this kind of \emph{expressiveness studies} concern only \emph{untyped process languages}.
In this work, we study 
{\em relative expressiveness} 
via \emph{type-preserving encodings} for \HOp, a \emph{higher-order} 
process language that integrates message-passing concurrency with functional features.
We consider source and target calculi coupled with \emph{session types} denoting interaction protocols. 
Building upon untyped frameworks for relative expressiveness
\cite{DBLP:journals/iandc/Gorla10}, 
we propose type preservation as a {new criteria} for \emph{precise encodings}.
We identify \HO, a new core calculus for higher-order session concurrency without
name passing. 
We show that \HO can encode \HOp precisely and efficiently. 
Requiring  
type preservation makes
this encoding far from trivial: our encoding crucially exploits advances on
session type duality~\cite{TGC14,DBLP:journals/corr/abs-1202-2086} and recent
characterisations of typed contextual equivalence \cite{characteristic_bis}.
We develop a full hierarchy of variants of \HOp based on 
precise encodings (see \figref{fig:express}):
our encodings are
type-preserving and fully abstract, up to typed
behavioural equalities. 

\begin{figure}[t]
\centering
\includegraphics[scale=1]{diag.pdf}

	\caption{Encodability in Higher-Order Sessions. 
	Precise encodings are defined in \defref{def:goodenc}.
	\label{fig:express}}
\vspace{-5mm}
\Hlinefig
\end{figure}

\jparagraph{Context}
In \emph{session-based concurrency}, interactions are organised into \emph{sessions}, basic communication units.
Interaction patterns can then be abstracted as expressive \emph{session types}~\cite{honda.vasconcelos.kubo:language-primitives}, against which  specifications may be checked. 
%These patterns are defined as %(possibly recursive) 
%sequences of communication actions: % (send/receive a value, offer/select a behavior).
%For instance, 
%session type $T_1 = \btinp{\mathsf{str}} \btout{\mathsf{int}}  \tinact$ may be intuitively read as: receive (?) a value of type $\mathsf{str}$,then output (!) a value of type $\mathsf{int}$, finally close the protocol.
Session type $\btinp{U} S$ (resp.  $\btout{U} S$)
describes a protocol that first receives (resp. sends) a value of type $U$ and then continues as protocol $S$.
Also, given an index set $I$, types $\btbra{l_i:S_i}_{i \in I}$ 
and $\btsel{l_i:S_i}_{i \in I}$ 
define %, respectively,
%a branching and selection constructs for  
 a labeled choice mechanism; types 
$\trec{t}{S}$ 
and 
$\tinact$ denote recursive and completed protocols, respectively.
%describes a protocol that offers
%(resp. ) 
%Type $\tinact$ denotes the completed protocol.
In the (first-order) $\pi$-calculus~\cite{MilnerR:calmp1}, 
session types describe the intended interactive behaviour of the names/channels in a process.
%names/channels are endowed with session types (such as $T_1$) representing their intended interactive behavior.

Session-based concurrency has also been casted in {higher-order} process
calculi which, by combining features from the $\lambda$-calculus and the $\pi$-calculus, 
enable the exchange of values 
that may contain processes~\cite{tlca07,DBLP:journals/jfp/GayV10}. 
%Higher-order calculi with sessions 
%naturally bridges concurrent and functional computation, 
%and enable the specification of protocols involving \emph{code mobility}, 
%commonplace in practice.
%The \HOp calculus enables 
%the specification of protocols involving \emph{code mobility}, 
%and includes
%Higher-order calculi with sessions 
The higher-order calculus with sessions studied here, denoted \HOp,
can specify protocols involving \emph{code mobility}: it includes
%equiped ping with 
constructs for 
synchronisation along shared names, 
session communication (value passing, labelled choice) along linear names,
recursion, 
 (first-order) abstractions 
 and applications.
 That is, 
 values in communications include names but also (first-order) abstractions---functions from name identifiers to processes. 
 %(In contrast, higher-order abstractions---functions from processes to processes---are disallowed.)
 (In contrast, we rule out higher-order abstractions---functions from processes to processes.)
Abstractions can be linear or shared; their types are  denoted $\lhot{C}$ and $\shot{C}$, respectively ($C$ 
%is a first-order type $C$ (say, a session name).
denotes a name). In \HOp we may have processes with a 
session type such as, e.g.,
%$T_2 = \btbra{upload:\btinp{\lhot{\mathsf{int}}}\tinact ~ , ~ sha:\btinp{\shot{\mathsf{int}}}\tinact}_{}$
$$S = \btbra{up:\btinp{\lhot{C}}\btout{\mathsf{ok}}\tinact ~ , ~ down:\btout{\shot{C}}\btout{\mathsf{ok}}\tinact ~ , ~quit:\btout{\mathsf{bye}}\tinact}_{}$$
that abstracts a server that offers different behaviours to clients: 
%  clients to select among distinct  behaviors: %namely, 
  to \emph{upload} a linear function, % (to be received by the server), 
  to \emph{download} a shared function, % (to be sent by the server),
   or to \emph{quit} the protocol. Subsequently, 
  the server sends a message ($\mathsf{ok}$ or $\mathsf{bye}$) before closing the session.


%\jparagraph{The Problem}
%%Roughly speaking, 
%  \HOp %, a higher-order process language that 
%extends Sangiorgi's higher-order $\pi$-calculus~\cite{SangiorgiD:expmpa} with session primitives.
%To be precise, %More precisely, 
%\HOp
%includes
%constructs for 
%%session establishment
%synchronisation along shared names, 
%session communication (value passing, labelled choice) along linear names,
%recursion, 
% (first-order) abstractions %(i.e., functions from name identifiers  to processes)
% and applications.
%% (denoted $\lambda x.P$ and $(\lambda x.P)a$, resp.).
%%While synchronization on shared names (useful to model session establishment) is 
%%non deterministic, session communication is deterministic and occurs on linear names.
%\HOp is therefore a rather rich language. This begs the question:
%%\begin{quote}
%is there a \emph{sub-calculus} of \HOp with equal expressivity? %hich is as expressive as the whole calculus? 
%%\end{quote}
%This question is of foundational interest, 
%for reasoning/validation techniques are more easily developed on small formalisms. 
%It also has practical ramifications, 
%as such a \emph{core calculus} could be taken as reference in 
%the design of %(functional) 
%programming languages with session types support.
%%implementations of languages with session primitives.
%Expressivity results may then help justifying useful connections 
%between foundational and practical advances on languages with concurrency and communication.
%
%%We have recently developed a behavioral theory  for \HOp~\cite{characteristic_bis}:
%%we introduced
%%\emph{characteristic bisimilarity}, a sound and complete 
%%characterization of contextual equivalence. % that enables tractable analyses.


\jparagraph{Expressiveness of \HOp}
%In this paper 
We study the type-preserving, 
relative expressivity of \HOp. % in relation. 
%to two 
%sub-calculi
%that distill first- and higher-order session-based concurrency. 
%\begin{enumerate}[-]
%\item 
As expected from 
known literature in the untyped setting \cite{SangiorgiD:expmpa}, 
the first-order session \sessp-calculus~\cite{honda.vasconcelos.kubo:language-primitives} {(here denoted~\sessp)} 
can encode  
\HOp preserving session types. 
%(\HOp without
%abstractions and applications) 
%\item 
In this paper, 
our \emph{main discovery} is 
that 
\HOp 
without
name-passing and recursion
can serve as a new core calculus    
for higher-order session concurrency.  
We call this core calculus \HO. 
We show that \HO can encode \HOp more efficiently 
than \sessp. In addition, in the higher-order session typed setting, 
\HO offers more tractable bisimulation techniques 
than \sessp (cf. \secref{ss:equiv})
%constitute 
%the main sources 
%of expressivity in \HOp. 
%: \emph{name passing} and constructs for \emph{infinite behavior} (i.e., recursion and replication). 
%On the one hand, t
%Indeed, the expressivity of name-passing calculi (untyped/typed) is well known; e.g., the $\pi$-calculus can express 
%the $\lambda$-calculus and 
%process-passing calculi~\cite{SangiorgiD:expmpa}. 
%In the $\pi$-calculus, recursion and replication can be expressed in terms of each other. 
%On the other hand, 
%Higher-order concurrency is quite expressive too: 
%calculi without name passing and recursion are Turing equivalent~\cite{DBLP:journals/iandc/LanesePSS11}.
%Also, 
%recursion/replication operators are redundant in higher-order calculi: they can be represented using process passing and duplication~\cite{ThomsenB:plachoasgcfhop}. 

%\figref{fig:express} summarises %our expressivity 
%our encodability results. 


%While encoding \HOp 
%into the $\pi$-calculus preserving session types 
%(extending  known  results for untyped processes~\cite{SangiorgiD:expmpa}) is 
%%\jpc{already}
%significant, 



\jparagraph{Challenges and Contributions}

We assess the expressivity  of \HOp, \HO, and \sessp as delineated by session types. 
We introduce \emph{type-preserving encodings}:
we use type information to define encodings
and to retain the semantics of session protocols. 
Indeed,  not only we require 
well-typed source processes are encoded into 
well-typed target processes: 
we demand that session type constructs (input, output, branching, select) used to type the source process
are preserved by the typing of the target process.
This criterion is included in 
our notion of \emph{precise encoding} (\defref{def:goodenc}), which 
extends encodability criteria for untyped processes with 
\emph{full abstraction}.
{Full abstraction results are stated
up to two
behavioural equalities that characterise barbed congruence:
\emph{characteristic bisimilarity} ($\fwb$, defined in~\cite{characteristic_bis})
and 
\emph{higher-order bisimilarity} ($\hwb$), introduced in this
work.
It turns out that $\hwb$ offers more direct  reasoning than $\fwb$. }
Using precise encodings we establish strong correspondences between 
\HOp and its variants---see \figref{fig:express}. 



Our main contribution is 
an encoding of \HOp into \HO (\secref{subsec:HOpi_to_HO}).  
Since \HO lacks 
both name-passing and recursion, this encoding involves two \emph{key challenges}:
\begin{enumerate}[a.]
\item In known (typed) 
encodings of name-passing into process-passing~\cite{SaWabook} %are limited: % in that 
%they come with restrictions on name usages;  
%they 
%work for %name-passing 
%calculi 
%with \emph{capability types} 
%in which 
only the output capability of names can be sent---a received name cannot be used in later inputs.
This is far too limiting in \HOp, where 
 session names %denoting arbitrary protocols 
 may be passed around (\emph{delegation})
and types describe interaction  \emph{structures}, rather than ``loose'' name capabilities. % at a given time.



\item %As mentioned above, recursion % and replication)
%can be encoded in untyped higher-order calculi using process duplication. Unfortunately, this kind of encodings 
Known encodings of recursion in untyped higher-order calculi
do not carry over to session typed calculi such as \HOp,
because linear abstractions cannot be copied/duplicated. Hence, the discipline of session types  limits 
the possibilities for representing infinite behaviours---even simple forms, such as input-guarded replication.
\end{enumerate}




%MOTIVATION FIRST ENCODING (). \emph{Still to highlight: recursive type required, no recursion, small example.

%--- 
\noi
%We illustrate our approach. % to these challenges.
Our encoding overcomes these two obstacles, as we discuss in the following section.

Additional technical contributions include: 
(i)~the encodability of \HO into \sessp (\secref{subsec:HOp_to_sessp}); 
(ii)~extensions of our encodability results to richer settings (\secref{sec:extension});
(iii)~a non encodability result showing that shared names strictly add expressive power to session calculi (\secref{ss:negative}).
In essence, (i) extends known  results for untyped processes~\cite{SangiorgiD:expmpa} to the session typed setting.
Concerning (ii), we develop extensions of our encodings to 
\begin{enumerate}[-]
\item The extension of \HOp with \emph{higher-order} abstractions (\HOpp); 
\item The extension of \HOp with polyadic name passing and abstraction (\PHOp); 
\item The super-calculus of \HOpp and \PHOp (\PHOpp), equivalent to the calculus in~\cite{tlca07}.
\end{enumerate}
%\figref{fig:express} summarises %our expressivity 
%our encodability results. 
%From a global standpoint, our 
These
encodability results connect \HOp with existing higher-order process calculi~\cite{tlca07}, and  
further highlight the status of \HO as the core calculus for session concurrency.
Finally, although (iii) may be somewhat expected, to our knowledge we are the first to prove this separation result, 
exploiting session determinacy and typed equivalences.




\jparagraph{Outline} 
%This paper  is structured as follows.
%\begin{enumerate}[$\bullet$]
\secref{sec:overview} overviews key ideas of the precise encoding of \HOp into \sessp.
%\item 
\secref{sec:calculus} presents \HOp and its 
subcalculi (\HO and \sessp); %, and extensions (\HOpp and \PHOp).  
\secref{sec:types} summarises their session type system.
\secref{sec:bt}~pres\-ents  behavioural equalities for \HOp:
we recall definitions of barbed congruence and characteristic bisimilarity~\cite{characteristic_bis}, 
and introduce higher-order bisimilarity.
We show that these three typed relations coincide (\thmref{t:coincide}).
%and states type soundness 
%for \HOp and its variants.
\secref{s:expr} defines \emph{precise %(typed) 
encodings} by extending encodability criteria  for untyped processes. %~(e.g.,~\cite{DBLP:journals/iandc/Gorla10}).
%\item 
\secref{sec:positive} %and \S\,\ref{sec:negative}
gives {precise encodings} of \HOp into \HO and of \HOp into~\sessp (Thms.~\ref{f:enc:hopitoho} and~\ref{f:enc:hotopi}).
Mutual encodings between \sessp and \HO are derivable; 
all these calculi are thus equally expressive.
By means of empirical and formal comparisons between these two precise encodings, in \secref{ss:compare} we establish that
\HOp and \HO are more tightly related than \HOp and \sessp (\thmref{t:tight}).
Moreover, we prove the impossibility of encoding communication along shared names
using linear names (\thmref{t:negative}).
%Exploiting determinacy and typed equivalences,
%\item
In \secref{sec:extension} %studies extensions of \HOp: 
we show that both \HOpp 
%(the extension with higher-order applications) 
and \PHOp 
%(the extension with polyadicity) 
are encodable in \HOp
(Thms.~\ref{f:enc:hopiptohopi} and \ref{f:enc:phopiptohopi}).
%This connects our work to the existing higher-order session calculus in~\cite{tlca07} (here denoted  $\PHOpp$).
%\item 
\secref{sec:relwork} collects concluding remarks and reviews related works.
%\secref{sec:concl} concludes.
The paper is self-contained. {\bf\em Omitted definitions and  proofs are in the Appendix and in~\cite{KouzapasPY15}.} 




\emph{Type-preserving compilations} are important in the design of
functional and object-oriented languages: type information has been
used to, e.g., justify code optimizations and reason about programs~\cite{DBLP:journals/toplas/MorrisettWCG99,DBLP:conf/pldi/ShaoA95,DBLP:journals/toplas/LeagueST02}.
A vast literature on 
{\em expressiveness} 
in concurrency theory
also studies compilations (or \emph{encodings})~\cite{Palamidessi03,DBLP:journals/iandc/Gorla10,DBLP:journals/tcs/FuL10,DBLP:conf/icalp/LanesePSS10,DBLP:journals/corr/PetersG15}:
they are used to transfer reasoning techniques 
%from one calculus to another, 
across calculi,
and to 
%identify constructs which may be implemented using simpler ones. 
implement process constructs using simpler ones.
%To a large extent, however, this kind of \emph{expressiveness studies} concern only \emph{untyped process languages}.
In this work, we study 
{\em relative expressiveness} 
via \emph{type-preserving encodings} for \HOp, a \emph{higher-order} 
process language that integrates message-passing concurrency with functional features.
We consider source and target calculi coupled with \emph{session types}~\cite{honda.vasconcelos.kubo:language-primitives} denoting interaction protocols. 
Building on untyped frameworks for relative expressiveness
\cite{DBLP:journals/iandc/Gorla10}, 
we propose type preservation as a {new criterion} for \emph{precise encodings}.
We identify \HO, a new core calculus for higher-order session concurrency without
name passing. 
We show that \HO can encode \HOp precisely and efficiently. 
Requiring  
type preservation makes
this encoding far from trivial: we crucially exploit advances on
session type duality~\cite{TGC14,DBLP:journals/corr/abs-1202-2086} and recent
characterisations of typed contextual equivalence \cite{characteristic_bis}.
We develop a full hierarchy of variants of \HOp based on 
precise encodings: % (see \figref{fig:express}):
our encodings are
type-preserving and fully abstract up to typed
behavioural equalities. 
\newj{\figref{fig:express} illustrates this hierarchy; the variants of \HOp are explained next.}

\begin{figure}[t]
\centering
\includegraphics[scale=1]{diag.pdf}

	\caption{Encodability in Higher-Order Sessions. 
	Precise encodings are defined in \defref{def:goodenc}.
	\label{fig:express}}
%\vspace{-5mm}
%\Hlinefig
\end{figure}

\paragraph{Context.}
In \emph{session-based concurrency}, interactions are organised into \emph{sessions}, basic communication units.
Interaction patterns can then be abstracted as \emph{session types}~\cite{honda.vasconcelos.kubo:language-primitives}, against which  specifications may be checked. 
%These patterns are defined as %(possibly recursive) 
%sequences of communication actions: % (send/receive a value, offer/select a behavior).
%For instance, 
%session type $T_1 = \btinp{\mathsf{str}} \btout{\mathsf{int}}  \tinact$ may be intuitively read as: receive (?) a value of type $\mathsf{str}$,then output (!) a value of type $\mathsf{int}$, finally close the protocol.
Session type $\btinp{U} S$ (resp.  $\btout{U} S$)
describes a protocol that first receives (resp. sends) a value of type $U$ and then continues as protocol $S$.
Also, given an index set $I$, types $\btbra{l_i:S_i}_{i \in I}$ 
and $\btsel{l_i:S_i}_{i \in I}$ 
define, %, respectively,
%a branching and selection constructs for  
\newj{respectively, external and internal choice constructs for}
 a labelled choice mechanism; types 
$\trec{t}{S}$ 
and 
$\tinact$ denote recursive and completed protocols, respectively.
%describes a protocol that offers
%(resp. ) 
%Type $\tinact$ denotes the completed protocol.
In the %(first-order) 
$\pi$-calculus, %~\cite{MilnerR:calmp1}, 
session types describe the intended interactive behaviour of the names %/channels 
in a process~\cite{honda.vasconcelos.kubo:language-primitives}.
%names/channels are endowed with session types (such as $T_1$) representing their intended interactive behavior.

Session-based concurrency has also been casted in {higher-order} process
calculi which, by combining features from the $\lambda$-calculus and the $\pi$-calculus, 
enable the exchange of values 
that may contain processes~\cite{tlca07,DBLP:journals/jfp/GayV10}. 
%Higher-order calculi with sessions 
%naturally bridges concurrent and functional computation, 
%and enable the specification of protocols involving \emph{code mobility}, 
%commonplace in practice.
%The \HOp calculus enables 
%the specification of protocols involving \emph{code mobility}, 
%and includes
%Higher-order calculi with sessions 
The higher-order calculus with sessions studied here, called \HOp,
can specify protocols involving \emph{code mobility}: it includes
%equiped ping with 
constructs for 
synchronisation along shared names, 
session communication (value passing, labelled choice) along linear names,
recursion, 
 (first-order) abstractions 
 and applications.
 That is, 
 values in communications include names but also (first-order) abstractions---functions from name identifiers to processes. 
 %(In contrast, higher-order abstractions---functions from processes to processes---are disallowed.)
 (In contrast, we rule out \emph{higher-order} abstractions---functions from processes to processes.)
Abstractions can be linear or shared; their types are  denoted $\lhot{C}$ and $\shot{C}$, respectively ($C$ 
%is a first-order type $C$ (say, a session name).
denotes a name). In \HOp we may have processes with a 
session type such as, e.g.,
%$T_2 = \btbra{upload:\btinp{\lhot{\mathsf{int}}}\tinact ~ , ~ sha:\btinp{\shot{\mathsf{int}}}\tinact}_{}$
$$S = \btbra{{up}:\btinp{\lhot{C}}\btout{\mathsf{ok}}\tinact ~ , ~ {down}:\btout{\shot{C}}\btout{\mathsf{ok}}\tinact ~ , ~{quit}:\btout{\mathsf{bye}}\tinact}_{}\,.$$
%that 
$S$ is the type of 
a server that offers ($\&$) \newj{three} different behaviours to a client: 
%  clients to select among distinct  behaviors: %namely, 
  to \emph{upload} a linear function, % (to be received by the server), 
  to \emph{download} a shared function, % (to be sent by the server),
   or to \emph{quit} the protocol. 
   %Subsequently, 
   \newj{Following a client's  selection ($\oplus$),}
  the server sends a message ($\mathsf{ok}$ or $\mathsf{bye}$) before closing the session.





\paragraph{Expressiveness of \HOp.}
%In this paper 
We study the type-preserving, 
relative expressivity of \HOp. % in relation. 
%to two 
%sub-calculi
%that distill first- and higher-order session-based concurrency. 
%\begin{enumerate}[-]
%\item 
As expected from 
known literature in the untyped setting \cite{SangiorgiD:expmpa}, 
the first-order session \sessp-calculus~\cite{honda.vasconcelos.kubo:language-primitives} {(here denoted~\sessp)} 
can encode  
\HOp preserving session types. 
%(\HOp without
%abstractions and applications) 
%\item 
In this paper, 
our \emph{main discovery} is 
that 
\HOp 
without
name-passing and recursion
%is a new 
can serve as a 
core calculus    
for higher-order session concurrency.  
We call this core calculus \HO. 
We show that \HO can encode \HOp more efficiently 
than \sessp. In addition, in the higher-order session typed setting, 
\HO offers more tractable bisimulation techniques 
than \sessp (cf. \secref{ss:equiv}).



\paragraph{Challenges and Contributions.}

We assess the expressivity  of \HOp, \HO, and \sessp as delineated by session types. 
We introduce \emph{type-preserving encodings}:
type information is used to define encodings
and to retain the semantics of session protocols. 
Indeed,  not only we require 
well-typed source processes are encoded into 
well-typed target processes: 
we demand that session type constructs (input, output, branching, select) used to type the source process
are preserved by the typing of the target process.
This criterion is included in 
our notion of \emph{precise encoding} (\defref{def:goodenc}), which 
extends encodability criteria for untyped processes with 
\emph{full abstraction}.
{Full abstraction results are stated
up to two
behavioural equalities that characterise barbed congruence:
\emph{characteristic bisimilarity} ($\fwb$, defined in~\cite{characteristic_bis})
and 
\emph{higher-order bisimilarity} ($\hwb$), introduced in this
work.
It turns out that $\hwb$ offers more direct  reasoning than $\fwb$. }
Using precise encodings we establish strong correspondences between 
\HOp and its variants---see 
%\figref{fig:express}. 
below.


One main contribution is 
an encoding of \HOp into \HO (\secref{subsec:HOpi_to_HO}).  
Since \HO lacks 
both name-passing and recursion, this encoding involves two \emph{key challenges}:
\begin{enumerate}[a.]
\item In known (typed) 
encodings of name-passing into process-passing~\cite{SaWabook} %are limited: % in that 
%they come with restrictions on name usages;  
%they 
%work for %name-passing 
%calculi 
%with \emph{capability types} 
%in which 
only the output capability of names can be sent---a received name cannot be used in later inputs.
This is far too limiting in \HOp, where 
 session names %denoting arbitrary protocols 
 may be passed around (\emph{delegation})
and types describe interaction  \emph{structures}, rather than ``loose'' name capabilities. % at a given time.



\item %As mentioned above, recursion % and replication)
%can be encoded in untyped higher-order calculi using process duplication. Unfortunately, this kind of encodings 
Known encodings of recursion in untyped higher-order calculi
do not carry over to session typed calculi such as \HOp,
because linear abstractions cannot be copied/duplicated. Hence, the discipline of session types  limits 
the possibilities for representing infinite behaviours---even simple forms, such as input-guarded replication.
\end{enumerate}




%MOTIVATION FIRST ENCODING (). \emph{Still to highlight: recursive type required, no recursion, small example.

%--- 
%\noi
%We illustrate our approach. % to these challenges.
Our encoding overcomes these two obstacles, as we discuss in \secref{sec:overview}.

Additional technical contributions include: 
(i)~the encodability of \HO into \sessp (\secref{subsec:HOp_to_sessp}); 
(ii)~extensions of our encodability results to richer settings (\secref{sec:extension});
(iii)~a non encodability result showing that shared names strictly add expressive power to session calculi (\secref{ss:negative}).
In essence, (i) extends known  results for untyped processes~\cite{SangiorgiD:expmpa} to the session typed setting.
Concerning (ii), we develop extensions of our encodings to 
\begin{enumerate}[-]
\item The extension of \HOp with \emph{higher-order} abstractions (\HOpp); 
\item The extension of \HOp with polyadic name passing and abstraction (\PHOp); 
\item The super-calculus of \HOpp and \PHOp (\PHOpp), equivalent to the calculus in~\cite{tlca07}.
\end{enumerate}
%\figref{fig:express} summarises %our expressivity 
%our encodability results. 
%From a global standpoint, our 

\newj{\figref{fig:express} summarises our encodability results: they}
%These encodability results 
connect \HOp with existing higher-order process calculi~\cite{tlca07}, and  
 highlight the status of \HO as the core calculus for session concurrency.
Finally, %although (iii) may be somewhat expected, 
to our knowledge we are the first to prove 
%this separation result, 
the non encodability result (iii),
exploiting session determinacy and typed equivalences.




\paragraph{Outline.} 
%This paper  is structured as follows.
%\begin{enumerate}[$\bullet$]
\secref{sec:overview} overviews key ideas of the precise encoding of \HOp into \sessp.
%\item 
\secref{sec:calculus} presents \HOp and its 
subcalculi (\HO and \sessp); %, and extensions (\HOpp and \PHOp).  
\secref{sec:types} summarises their session type system.
\secref{sec:bt}~pres\-ents  behavioural equalities for \HOp:
we recall definitions of barbed congruence and characteristic bisimilarity~\cite{characteristic_bis}, 
and introduce higher-order bisimilarity.
We show that these three typed relations coincide (\thmref{t:coincide}).
%and states type soundness 
%for \HOp and its variants.
\secref{s:expr} defines \emph{precise %(typed) 
encodings} by extending encodability criteria  for untyped processes. %~(e.g.,~\cite{DBLP:journals/iandc/Gorla10}).
%\item 
\secref{sec:positive} %and \S\,\ref{sec:negative}
gives {precise encodings} of \HOp into \HO and of \HOp into~\sessp (Thms.~\ref{f:enc:hopitoho} and~\ref{f:enc:hotopi}).
Mutual encodings between \sessp and \HO are derivable; 
all these calculi are thus equally expressive.
%By means of 
Via
empirical and formal comparisons between these two precise encodings, in \secref{ss:compare} we establish that
\HOp and \HO are more tightly related than \HOp and \sessp (\thmref{t:tight}).
Moreover, we prove the impossibility of encoding communication along shared names
using linear names (\thmref{t:negative}).
%Exploiting determinacy and typed equivalences,
%\item
In \secref{sec:extension} %studies extensions of \HOp: 
we show 
%that both \HOpp 
%(the extension with higher-order applications) 
%and \PHOp 
%(the extension with polyadicity) 
%are encodable 
encodings of \HOpp and \PHOp 
into \HOp
(Thms.~\ref{f:enc:hopiptohopi} and \ref{f:enc:phopiptohopi}).
%This connects our work to the existing higher-order session calculus in~\cite{tlca07} (here denoted  $\PHOpp$).
%\item 
\secref{sec:relwork} collects concluding remarks and reviews related works.
%\secref{sec:concl} concludes.
%The paper is self-contained. 
{Omitted definitions and  proofs are  %in the Appendix and 
in~\cite{KouzapasPY15}.} 

%%%%%%%%%%%%%%%%%%%%%%%%%%%%%%%%%%%%%%%%%%%%%%%%%%%%%%%%%%%%%%%%%%%%%%%%%%%%%%%%
%%%%%%%%%%%%%%%%%%%%%%%%%%%%%%%%%%%%%%%%%%%%%%%%%%%%%%%%%%%%%%%%%%%%%%%%%%%%%%%%

\section{Overview: Encoding Name Passing Into Process Passing}
\label{sec:overview}
%% !TEX root = main.tex

\jparagraph{A Precise Encoding of Name-Passing into Process-Passing}
As mentioned above, 
our encoding of \HOp into \HO (\secref{subsec:HOpi_to_HO}) should overcome two key challenges.
First, it should enable the communication of arbitrary names, as required to represent delegation.
Second, it should address the fact that linearity of session types limits the 
possibilities for representing infinite behavior. 
To encode name passing into \HO 
%to encode name output, 
we ``pack''
the name to be sent into a suitable abstraction; 
upon reception, the receiver must ``unpack'' this object following a precise protocol on a fresh  session:
%More precisely, our encoding \jpc{of name passing} in \HO is given as:
\begin{center}
\begin{tabular}{rcll}
  $\map{\bout{a}{b} P}$	&$=$&	$\bout{a}{ \abs{z}{\,\binp{z}{x} (\appl{x}{b})} } \map{P}$ \\
  $\map{\binp{a}{x} Q}$	&$=$&	$\binp{a}{y} \newsp{s}{\appl{y}{s} \Par \bout{\dual{s}}{\abs{x}{\map{Q}}} \inact}$
\end{tabular}
\end{center}
%and as a homomorphism for the other operators.
Above, 
%where
$a,b$ are names and $s$ and $\dual{s}$ are 
linear session names (\emph{endpoints}).
%$\lambda x.P$ is a name abstraction of $P$; $\appl{x}{a}$ is a name application; 
Processes $\bout{a}{V} P$ and 
$\binp{a}{x} P$ denote output and input at~$a$;   
abstractions and applications are denoted
$\lambda x.P$ and $(\lambda x.P)a$; %, respectively;
$\newsp{s}P$ and $\inact$ represent hiding and inaction. %, respectively.
%Intuitively, the output of a name $b$ along name $a$ is encoded by
%the output of an abstraction containing $b$; the input of a name is encoded 
%by the input of an abstraction
Thus, following a communication on $a$, %our encoding features 
a (deterministic) reduction between  
$s$ and $\dual{s}$ guarantees that name $b$ is properly unpacked by means of abstraction passing
and appropriate applications.
It is worth stressing how an output action in the source process is translated into an output action in the encoded process (and similarly for input).
This correspondence is key to ensure the preservation of session type operators mentioned above.

To preserve session linearity, we proceed as follows.
Given $\recp{X}{P}$, 
we encode the recursion body $P$ as an abstraction
in which free names of $P$ are converted into name variables.
%The encoding keeps track of these free names.
The resulting higher-order value is embedded in an input-guarded 
``duplicator'' process~\cite{ThomsenB:plachoasgcfhop}.
The recursion variable $X$ is then encoded 
in such a way that it
simulates recursion unfolding by 
invoking the duplicator in a by-need fashion.
That is, upon reception, the abstraction representing the 
recursion body $P$
is duplicated: 
one copy is used to reconstitute the original recursion body $P$ (through
the application of the free names of $P$); 
another copy is used to re-invoke the duplicator when needed. 
Interestingly, for this encoding to work 
we require non-tail recursive session types; to this end, 
we apply recent advances on the theory of duality for session types~\cite{TGC14,DBLP:journals/corr/abs-1202-2086}.

%To this end, we
%first record a mapping from recursive variable $X$ to process variable $z_X$.
%Then, we encode the recursion body $P$ as a name abstraction
%in which free names of $P$ are converted into name variables, using \defref{d:auxmap}.
%(Notice that $P$ is first encoded into \HO and then transformed using mapping
%$\auxmapp{\cdot}{{}}{\sigma}$.)
%Subsequently, this higher-order value is embedded in an input-guarded 
%``duplicator'' process~\cite{ThomsenB:plachoasgcfhop}. Finally, we define the encoding of $X$ 
%in such a way that it
%simulates recursion unfolding by 
%invoking the duplicator in a by-need fashion.
%That is, upon reception, the \HO abstraction which encodes  the 
%recursion body $P$
%%containing $\auxmapp{P}{{}}{\sigma}$ 
%is duplicated: 
%one copy is used to reconstitute the original recursion body $P$ (through
%the application of $\fn{P}$); another copy is used to re-invoke
%the duplicator when needed. 
%
%We encode recursion with non-tail recursive session types; for this 
%we apply recent advances on the theory of session duality~\cite{TGC14,DBLP:journals/corr/abs-1202-2086}.

\jparagraph{A Plausible Encoding That is Not Precise}
As motivated earlier,
we define a notion of \emph{precise encoding} (\secref{s:expr}) that
requires the translation of both process and types, and 
admits only process mappings that preserve session types
\emph{and} are fully abstract. Thus, our encodings 
not only exhibit   strong behavioral correspondences, but also 
 relate source and target processes with  
communication structures described by session types.
%Moreover, the notion of encoding includes full abstraction as encodability criteria.
These strict requirements make our developments more challenging.
In particular, requiring type preservation rules out other plausible encoding strategies.
To illustrate this point,
consider the  following encoding of %$\sessp$ 
name-passing 
into $\HO$:\footnote{This alternative  encoding was suggested by an anonymous reviewer of a previous version of this paper.} %defined as
\begin{center}
\begin{tabular}{rcll}
  $\umap{\bout{a}{b} P}$	&$=$&	$\binp{a}{x}( \appl{x}{b} \Par \umap{P})$ \\
  $\umap{\binp{a}{x} Q}$	&$=$&	$\bout{a}{\abs{x}{\umap{Q}}} \inact$
\end{tabular}
\end{center}
%and as a homomorphism for the other operators.
Intuitively, 
rather than sending a package with name $b$, 
this encoding sends the continuation of the input. Observe how this mapping entails  a 
``role inversion'': outputs are translated into inputs, and inputs are translated into outputs. 
Although perfectly reasonable, the encoding $\umap{\cdot}$  
%is far from desirable in a session typed setting: 
is \emph{not type preserving}. Consequently, it is also not \emph{precise}.
%Type preservation is intended to preserve the overall semantics of session types:
Since individual prefixes (input, output, branching, select) 
represent actions in a structured communication sequence (i.e., a protocol abstracted by a session type),
the encoding above would simply alter the meaning of the session protocol in the source language.





\paragraph{A Precise Encoding of Name-Passing into Process-Passing.}
As mentioned above, 
our encoding of \HOp into \HO (\secref{subsec:HOpi_to_HO}) should 
%overcome two key challenges. First, it should 
(a)~enable the communication of arbitrary names, as required to represent delegation,
and 
%Second, it should 
(b)~address the fact that linearity of session types limits the 
possibilities for representing infinite behaviour. 
To encode name passing into \HO 
%to encode name output, 
we ``pack''
the name to be sent into a suitable abstraction; 
upon reception, the receiver ``unpacks'' this object following a precise protocol on a fresh  session:
%More precisely, our encoding \jpc{of name passing} in \HO is given as:
\begin{center}
\begin{tabular}{rcll}
  $\map{\bout{a}{b} P}$	&$=$&	$\bout{a}{ \abs{z}{\,\binp{z}{x} (\appl{x}{b})} } \map{P}$ \\
  $\map{\binp{a}{x} Q}$	&$=$&	$\binp{a}{y} \newsp{s}{\appl{y}{s} \Par \bout{\dual{s}}{\abs{x}{\map{Q}}} \inact}$
\end{tabular}
\end{center}
%and as a homomorphism for the other operators.
Above, 
%where
$a,b$ are names and $s$ and $\dual{s}$ are 
linear session names (\emph{endpoints}).
%$\lambda x.P$ is a name abstraction of $P$; $\appl{x}{a}$ is a name application; 
Processes $\bout{a}{V} P$ and 
$\binp{a}{x} P$ denote output and input at~$a$;   
abstractions and applications are denoted
$\lambda x.P$ and $(\lambda x.P)a$. Processes %, respectively;
$\newsp{s}P$ and $\inact$ represent hiding and inaction. %, respectively.
%Intuitively, the output of a name $b$ along name $a$ is encoded by
%the output of an abstraction containing $b$; the input of a name is encoded 
%by the input of an abstraction
Thus, following a communication on $a$, %our encoding features 
a (deterministic) reduction between  
$s$ and $\dual{s}$ guarantees that $b$ is properly unpacked by means of abstraction passing
and appropriate applications.
Notice that 
\HO requires two extra reduction steps to mimic a name communication step in \HOp.
Also, an output action in the source process is translated into an output action in the encoded process (and similarly for input).
This is key to ensure the preservation of session type operators mentioned above (cf. \defref{def:tp}).

\newj{As hinted at above, 
a challenge in 
 encoding $\recp{X}{P}$ is how to 
preserve linearity  of session names.
Intuitively, we encode the recursion body $P$ as an abstraction 
$\abs{\tilde{x}}{\auxmapp{P}{{}}{\sigma}}$
in which each session name of $P$ is converted into a name variable in $\tilde{x}$, using mapping $\sigma$.
Since  
$\abs{\tilde{x}}{\auxmapp{P}{{}}{\sigma}}$
does not mention (linear) session names,
we may embed it into a 
``duplicator'' process
which implements recursion using higher-order communication~\cite{ThomsenB:plachoasgcfhop}. 
The encoding of the recursion variable $X$
invokes this duplicator in a by-need fashion:
it receives 
$\abs{\tilde{x}}{\auxmapp{P}{{}}{\sigma}}$ and uses two copies of it:
one copy allows us to obtain $P$
through the application of the session names of $P$; 
the other allows us
to invoke the duplicator when needed. 
Interestingly, for this encoding to work 
we require non-tail recursive session types; 
this exploits recent advances on the theory of duality for session types~\cite{TGC14,DBLP:journals/corr/abs-1202-2086}.}


%To preserve session linearity, we proceed as follows.
%Given $\recp{X}{P}$, 
%we encode the recursion body $P$ as an abstraction
%in which free names of $P$ are converted into name variables.
%The resulting higher-order value is embedded in an input-guarded 
%``duplicator'' process~\cite{ThomsenB:plachoasgcfhop}.
%The recursion variable $X$ is then encoded 
%in such a way that it
%simulates recursion unfolding by 
%invoking the duplicator in a by-need fashion.
%That is, upon reception, the abstraction representing the 
%recursion body $P$
%is duplicated: 
%one copy is used to reconstitute the original recursion body $P$ (through
%the application of the free names of $P$); 
%another copy is used to re-invoke the duplicator when needed. 
%Interestingly, for this encoding to work 
%we require non-tail recursive session types; to this end, 
%we apply recent advances on the theory of duality for session types~\cite{TGC14,DBLP:journals/corr/abs-1202-2086}.

%To this end, we
%first record a mapping from recursive variable $X$ to process variable $z_X$.
%Then, we encode the recursion body $P$ as a name abstraction
%in which free names of $P$ are converted into name variables, using \defref{d:auxmap}.
%(Notice that $P$ is first encoded into \HO and then transformed using mapping
%$\auxmapp{\cdot}{{}}{\sigma}$.)
%Subsequently, this higher-order value is embedded in an input-guarded 
%``duplicator'' process~\cite{ThomsenB:plachoasgcfhop}. Finally, we define the encoding of $X$ 
%in such a way that it
%simulates recursion unfolding by 
%invoking the duplicator in a by-need fashion.
%That is, upon reception, the \HO abstraction which encodes  the 
%recursion body $P$
%%containing $\auxmapp{P}{{}}{\sigma}$ 
%is duplicated: 
%one copy is used to reconstitute the original recursion body $P$ (through
%the application of $\fn{P}$); another copy is used to re-invoke
%the duplicator when needed. 
%
%We encode recursion with non-tail recursive session types; for this 
%we apply recent advances on the theory of session duality~\cite{TGC14,DBLP:journals/corr/abs-1202-2086}.

\paragraph{A Plausible Encoding That is Not Precise.}
Our notion of \emph{precise encoding} (\defref{def:goodenc}) 
requires the translation of both process and types; it  
admits only process mappings that preserve session types
\emph{and} are fully abstract. Thus, our encodings 
not only exhibit  strong behavioural correspondences, but also 
 relate source and target processes with  
communication structures described by session types.
%Moreover, the notion of encoding includes full abstraction as encodability criteria.
These requirements are demanding and make our developments far from trivial.
In particular, requiring type preservation may rule out other plausible encoding strategies.
To illustrate this point,
consider the  following encoding of %$\sessp$ 
name-passing 
into $\HO$:\footnote{This alternative  encoding was suggested by an anonymous reviewer of a previous version of this paper.} %defined as
\begin{center}
\begin{tabular}{rcll}
  $\umap{\binp{a}{x} Q}$	&$=$&	$\bout{a}{\abs{x}{\umap{Q}}} \inact$ \\
    $\umap{\bout{a}{b} P}$	&$=$&	$\binp{a}{x}( \appl{x}{b} \Par \umap{P})$ 
\end{tabular}
\end{center}
%and as a homomorphism for the other operators.
{Intuitively, 
the encoding of input takes the initiative by sending an abstraction containing the encoding of its continuation $Q$;
the encoding of output applies this received value to name $b$.}
%rather than sending a package with name $b$, this encoding sends the continuation of the input. 
Hence, this mapping entails  a 
``role inversion'': outputs are translated into inputs, and inputs are translated into outputs. 
Although fairly reasonable, we will see that the encoding $\umap{\cdot}$  
%is far from desirable in a session typed setting: 
is \emph{not type preserving}. Consequently, it is also not \emph{precise}.
%Type preservation is intended to preserve the overall semantics of session types:
Since individual prefixes (input, output, branching, select) 
represent actions in a structured communication sequence (i.e., a protocol abstracted by a session type),
the encoding above would simply alter the meaning of the session protocol in the source language.

%%%%%%%%%%%%%%%%%%%%%%%%%%%%%%%%%%%%%%%%%%%%%%%%%%%%%%%%%%%%%%%%%%%%%%%%%%%%%%%%
%%%%%%%%%%%%%%%%%%%%%%%%%%%%%%%%%%%%%%%%%%%%%%%%%%%%%%%%%%%%%%%%%%%%%%%%%%%%%%%%

%% !TEX root = main.tex
\section{Higher-Order Session $\pi$-Calculi}
\label{sec:calculus}

\noindent

We use the the 
\emph{Higher-Order Session $\pi$-Calculus} (\HOp)
which is a variant of the calculus originally introduced
in~\cite{characteristic_bis}.
\HOp includes both name- and abstraction-passing 
as well as recursion; it is a subcalculus of the language studied 
in~\cite{tlca07}.
The difference between \HOp and the calculus presented
in~\cite{characteristic_bis} is the fact that \HOp does
not allow higher-order value applications. We describe
the syntax in brief.
%Following the literature~\cite{JeffreyR05},
%for simplicity of the presentation
%we concentrate on the second-order call-by-value \HOp.  
(In \secref{sec:extension} we consider extensions of 
\HOp with higher-order abstractions 
and polyadicity in name-passing/abstractions.)
%We also introduce two subcalculi of \HOp. In particular, we define the 
%core higher-order session
%calculus (\HO), which 
%%. The \HO calculus is  minimal: it 
%includes constructs for shared name synchronisation and 
%%constructs for session establish\-ment/communication and 
%(monadic) name-abstraction, but lacks name-passing and recursion.

%Although minimal, in \secref{s:expr}
%the abstraction-passing capabilities of \HOp will prove 
%expressive enough to capture key features of session communication, 
%such as delegation and recursion.

\subsection{Syntax of \HOp}
\label{subsec:syntax}

\noindent
The syntax of \HOp is defined in \figref{fig:syntax}.

	\begin{figure}[t]
	\[ 
		\begin{array}{rcl}
			u,w  &\bnfis& n \bnfbar x,y,z
			\qquad \qquad
			n \bnfis a,b \bnfbar s, \dual{s} 
			\qquad \qquad
			V,W \bnfis \nonhosyntax{u} \bnfbar \abs{x}{P}
			\\[1mm]

			P,Q
			& \bnfis &
			\bout{u}{V}{P}  \bnfbar  \binp{u}{x}{P} \bnfbar
			\bsel{u}{l} P \bnfbar \bbra{u}{l_i:P_i}_{i \in I} \bnfbar \appl{V}{u}\bnfbar P\Par Q \bnfbar \news{n} P 
			\bnfbar \inact
			\\%[1mm]
			& \bnfbar &
			\nonhosyntax{\rvar{X} \bnfbar \recp{X}{P}}
		\end{array}
	\]
	\caption{Syntax of \HOp (\HO lacks the constructs in \nonhosyntax{\text{grey}}).}
	\label{fig:syntax}
%	\Hlinefig
\end{figure}



\myparagraph{Values}
Names $a,b,c, \dots$ (resp.~$s, \dual{s}, \dots$) 
range over shared (resp. session) names. 
Names $m, n, t, \dots$ are session or shared names.
Dual endpoints are $\dual{n}$ with
$\dual{\dual{s}} = s$ and $\dual{a} = a$.
%We define the dual operation over names $n$ as $\dual{n}$ with
%$\dual{\dual{s}} = s$ and $\dual{a} = a$.
%Intuitively, names $s$ and $\dual{s}$ are dual (two) \emph{endpoints} while 
%shared names represent shared (non-deterministic) points. 
Variables are denoted with $x, y, z, \dots$, 
and recursive variables are denoted with $\varp{X}, \varp{Y} \dots$.
An abstraction %(or higher-order value) 
$\abs{x}{P}$ is a process $P$ with name parameter $x$.
%Symbols $u, v, \dots$ range over identifiers; and  $V, W, \dots$ to denote values. 
Values $V,W$ include 
identifiers $u, v, \ldots$ %(first-order values) 
and 
abstractions $\abs{x}{P}$ (first- and higher-order values, resp.). 

\myparagraph{Terms} 
include the
$\pi$-calculus prefixes for sending and receiving values $V$.
%Process $\bout{u}{V} P$ denotes the output of value $V$
%over name $u$, with continuation $P$;
%process $\binp{u}{x} P$ denotes the input prefix on name $u$ of a value
%that 
%will substitute variable $x$ in continuation $P$. 
Recursion is expressed by $\recp{X}{P}$,
which binds the recursive variable $\varp{X}$ in process $P$.
Process 
%ny
%$\appl{x}{u}$ 
$\appl{V}{u}$ 
is the application
which substitutes name $u$ on the abstraction~$V$. 
Typing  ensures that $V$ is not a name.
Processes $\bsel{u}{l} P$ and $\bbra{u}{l_i: P_i}_{i \in I}$ are the
standard session processes for selecting and branching.
%Prefix $\bsel{u}{l} P$ selects label $l$ on name $u$ and then behaves as $P$.
%Given $i \in I$ 
%Process $\bbra{u}{l_i: P_i}_{i \in I}$ offers a choice on labels $l_i$ with
%continuation $P_i$, given that $i \in I$.
%Others are standard c
Constructs for 
inaction $\inact$,  parallel composition $P_1 \Par P_2$, and 
name restriction $\news{n} P$ are standard.
Session name restriction $\news{s} P$ simultaneously binds endpoints $s$ and $\dual{s}$ in $P$.
%A well-formed process relies on assumptions for
%guarded recursive processes.
Functions $\fv{P}$ and $\fn{P}$ denotet the sets of free 
%\jpc{recursion}
variables and names; 
and \dk{assume $V$ in $\bout{u}{V}{P}$ does not include free recursive 
variables $\rvar{X}$.}
If $\fv{P} = \emptyset$, we call $P$ {\em closed}.
%; and closed $P$ without 
%free session names a {\em program}. 

\subsection{Subcalculi of \HOp}
\label{subsec:subcalculi}
\noi
We define two subcalculi of \HOp. 
%We now define several sub-calculi of \HOp. 
\begin{enumerate}[$\bullet$]
	\item	The first subcalculus is the 
		{\em core higher-order session calculus} (denoted \HO),
		which lacks recursion and name passing; its 
		formal syntax is obtained from \figref{fig:syntax} by excluding 
		constructs in \nonhosyntax{\text{grey}}.

	\item	The second subcalculus is 
		the {\em session $\pi$-calculus} 
		(denoted $\sessp$), which 
		lacks  the
		higher-order constructs
		(i.e., abstraction passing and application), but includes recursion.

%	\item	The third sub-calculus, denoted \haskp, represents cloud Haskell:
%		\[
%			\begin{array}{rclllll}
%				V,W	& ::= &		u \bnfbar  \abs{x}{P}
%				\\
%				P,Q	& ::= &		\bout{u}{m}{P}  \bnfbar  \binp{u}{x}{P} \bnfbar
%							\bsel{u}{l} P \bnfbar \bbra{u}{l_i:P_i}_{i \in I}
%				\\[1mm]
%					& \bnfbar &	\appl{V}{V} \bnfbar P\Par Q \bnfbar \news{n} P \bnfbar \inact
%		\end{array}
%		\]
\end{enumerate}
%
Let $\CAL \in \{\HOp, \HO, \sessp\}$. We write 
$\CAL^{-\mathsf{sh}}$ for $\CAL$ without shared names
(we delete $a,b$ from $n$). 
We shall demonstrate in Section~\ref{sec:positive} that 
$\HOp$, $\HO$, and $\sessp$ have the same expressivity.


\subsection{Operational Semantics}
\label{subsec:semantics}

\begin{figure}[!t]
\[
	\begin{array}{rcllcrcll}
%	\begin{array}{c}
		\appl{(\abs{x}{P})}{u}  & \red & P \subst{u}{x} 
		& \orule{App}
%		\\[1mm]
		&&
		\bout{n}{V} P \Par \binp{\dual{n}}{x} Q & \red & P \Par Q \subst{V}{x} 
		& \orule{Pass}
		\\[1mm]

		\bsel{n}{l_j} Q \Par \bbra{\dual{n}}{l_i : P_i}_{i \in I} & \red & Q \Par P_j ~~(j \in I)~~ 
		& \orule{Sel}
%		\\[1mm]
		&&
		P \red P'\Rightarrow  \news{n} P   & \red  &  \news{n} P' 
		& \orule{Res}
		\\[1mm]

		P \red P' & \Rightarrow  &  P \Par Q  \red   P' \Par Q  
		& \orule{Par}
%		\\[1mm]
		&&
		P \scong Q \red Q' \scong P' & \Rightarrow & P  \red  P'
		& \orule{Cong}
	\end{array}
\]
{\small
\[
	\begin{array}{c}
		P \Par \inact \scong P
		\quad
		P_1 \Par P_2 \scong P_2 \Par P_1
		\quad
		P_1 \Par (P_2 \Par P_3) \scong (P_1 \Par P_2) \Par P_3
%		\\[1mm]
		\quad
		\news{n} \inact \scong \inact
%		\quad 
		\\[1mm]
		P \Par \news{n} Q \scong \news{n}(P \Par Q)
		\ (n \notin \fn{P})
		\quad 
		\recp{X}{P} \scong P\subst{\recp{X}{P}}{\rvar{X}}
%		\\[1mm]
		\quad
		P \scong Q \textrm{ if } P \scong_\alpha Q
%		\qquad
%		\dk{V \scong W \textrm{ if } V \scong_\alpha W
%\quad \abs{x}{P} \scong \abs{x}{Q} \textrm{ if } P \scong Q}
	\end{array}
\]
}
\caption{Operational Semantics of $\HOp$. 
\label{fig:reduction}}
%\Hlinefig
\end{figure}



\noindent \figref{fig:reduction} (top) defines the operational semantics 
of \HOp.
Rule $\orule{App}$ is the name application; 
Rule $\orule{Pass}$ defines a shared interaction at $n$ 
(with $\dual{n}=n$) or a session interaction;  
Rule $\orule{Sel}$ is the standard rule for labelled choice/selection;%:
%given an index set $I$, 
%a process selects label $l_j$ on name $n$ over a set of
%labels $\set{l_i}_{i \in I}$ offered by a branching 
%on the dual endpoint $\dual{n}$;
and other rules are standard $\pi$-calculus rules.
Rules for \emph{structural congruence} are defined in \figref{fig:reduction} (bottom). 
We assume the expected extension of $\scong$ to values $V$.
We write $\red^\ast$ for a multi-step reduction.


\section{Higher-Order Session $\pi$-Calculi}
\label{sec:calculus}

We introduce 
the \emph{higher-order session $\pi$-calculus} (\HOp).
We define 
syntax, operational semantics, and 
its sub-calculi (\sessp and \HO).
A type system and behavioural equivalences are introduced in 
\secref{sec:types} and \secref{sec:bt}. 
Extensions of \HOp %(\HOpp and \PHOp) 
are discussed in \secref{sec:extension}.


%We also introduce two subcalculi of \HOp. In particular, we define the 
%core higher-order session
%calculus (\HO), which 
%%. The \HO calculus is  minimal: it 
%includes constructs for shared name synchronisation and 
%%constructs for session establish\-ment/communication and 
%(monadic) name-abstraction, but lacks name-passing and recursion.

%Although minimal, in \secref{s:expr}
%the abstraction-passing capabilities of \HOp will prove 
%expressive enough to capture key features of session communication, 
%such as delegation and recursion.

\subsection{\HOp: Syntax, Operational Semantics, and Subcalculi}
\label{subsec:syntax}

\paragraph{Syntax.}
The syntax of \HOp is defined in \figref{fig:syntax}.
\HOp it is a subcalculus of the language studied 
in~\cite{tlca07}. It is also a variant of the language that we investigated in~\cite{characteristic_bis}, 
where higher-order value applications were considered. 


	\begin{figure}[t]
	\[ 
		\begin{array}{rcl}
			u,w  &\bnfis& n \bnfbar x,y,z
			\qquad \qquad
			n \bnfis a,b \bnfbar s, \dual{s} 
			\qquad \qquad
			V,W \bnfis \nonhosyntax{u} \bnfbar \abs{x}{P}
			\\[1mm]

			P,Q
			& \bnfis &
			\bout{u}{V}{P}  \bnfbar  \binp{u}{x}{P} \bnfbar
			\bsel{u}{l} P \bnfbar \bbra{u}{l_i:P_i}_{i \in I} \bnfbar \appl{V}{u}\bnfbar P\Par Q \bnfbar \news{n} P 
			\bnfbar \inact
			\\%[1mm]
			& \bnfbar &
			\nonhosyntax{\rvar{X} \bnfbar \recp{X}{P}}
		\end{array}
	\]
	\caption{Syntax of \HOp (\HO lacks the constructs in \nonhosyntax{\text{grey}}).}
	\label{fig:syntax}
%	\Hlinefig
\end{figure}



%\myparagraph{Values}
\emph{Names} $a,b,c, \dots$ (resp.~$s, \dual{s}, \dots$) 
range over shared (resp. session) names. 
Names $m, n, t, \dots$ are session or shared names.
Dual endpoints are $\dual{n}$ with
$\dual{\dual{s}} = s$ and $\dual{a} = a$.
%We define the dual operation over names $n$ as $\dual{n}$ with
%$\dual{\dual{s}} = s$ and $\dual{a} = a$.
%Intuitively, names $s$ and $\dual{s}$ are dual (two) \emph{endpoints} while 
%shared names represent shared (non-deterministic) points. 
Variables are denoted with $x, y, z, \dots$, 
and recursive variables are denoted with $\varp{X}, \varp{Y} \dots$.
An abstraction %(or higher-order value) 
$\abs{x}{P}$ is a process $P$ with name parameter $x$.
%Symbols $u, v, \dots$ range over identifiers; and  $V, W, \dots$ to denote values. 
\emph{Values} $V,W$ include 
identifiers $u, v, \ldots$ %(first-order values) 
and 
abstractions $\abs{x}{P}$ (first- and higher-order values, resp.). 

%\myparagraph{Terms} 

Terms
include $\pi$-calculus prefixes for sending and receiving values $V$.
%Process $\bout{u}{V} P$ denotes the output of value $V$
%over name $u$, with continuation $P$;
%process $\binp{u}{x} P$ denotes the input prefix on name $u$ of a value
%that 
%will substitute variable $x$ in continuation $P$. 
Recursion   $\recp{X}{P}$ binds the recursive variable $\varp{X}$ in process $P$.
Process 
%ny
%$\appl{x}{u}$ 
$\appl{V}{u}$ 
is the application
which substitutes name $u$ on the abstraction~$V$. 
Typing  ensures that $V$ is not a name.
Processes $\bsel{u}{l} P$ and $\bbra{u}{l_i: P_i}_{i \in I}$ are the
usual session processes for selecting and branching.
%Prefix $\bsel{u}{l} P$ selects label $l$ on name $u$ and then behaves as $P$.
%Given $i \in I$ 
%Process $\bbra{u}{l_i: P_i}_{i \in I}$ offers a choice on labels $l_i$ with
%continuation $P_i$, given that $i \in I$.
%Others are standard c
Constructs for 
inaction $\inact$,  parallel composition $P_1 \Par P_2$, and 
name restriction $\news{n} P$ are standard.
Session name restriction $\news{s} P$ simultaneously binds endpoints $s$ and $\dual{s}$ in $P$.
%A well-formed process relies on assumptions for
%guarded recursive processes.
Functions $\fv{P}$ and $\fn{P}$ denote the sets of free 
%\jpc{recursion}
variables and names.
We assume $V$ in $\bout{u}{V}{P}$ does not include free recursive 
variables $\rvar{X}$.
If $\fv{P} = \emptyset$, we call $P$ {\em closed}.
%; and closed $P$ without 
%free session names a {\em program}. 




%\subsection{Operational Semantics}
%\label{subsec:semantics}


\paragraph{Operational Semantics.}
The \emph{operational semantics} of \HOp is defined in terms of a reduction relation, 
denoted $\red$ and 
given in 
 \figref{fig:reduction} (top).
 We briefly explain the rules. 
Rule $\orule{App}$ defines  name application.
Rule $\orule{Pass}$ defines a shared interaction at $n$ 
(with $\dual{n}=n$) or a session interaction.
Rule $\orule{Sel}$ is the standard rule for labelled choice/selection.%:
%given an index set $I$, 
%a process selects label $l_j$ on name $n$ over a set of
%labels $\set{l_i}_{i \in I}$ offered by a branching 
%on the dual endpoint $\dual{n}$;
Other rules are standard $\pi$-calculus rules.
Reduction is closed under \emph{structural congruence} as defined in \figref{fig:reduction} (bottom). 
We assume the expected extension of $\scong$ to values $V$.
We write $\red^\ast$ for a multi-step reduction.

\begin{figure}[!t]
\[
	\begin{array}{rcllcrcll}
%	\begin{array}{c}
		\appl{(\abs{x}{P})}{u}  & \red & P \subst{u}{x} 
		& \orule{App}
%		\\[1mm]
		&&
		\bout{n}{V} P \Par \binp{\dual{n}}{x} Q & \red & P \Par Q \subst{V}{x} 
		& \orule{Pass}
		\\[1mm]

		\bsel{n}{l_j} Q \Par \bbra{\dual{n}}{l_i : P_i}_{i \in I} & \red & Q \Par P_j ~~(j \in I)~~ 
		& \orule{Sel}
%		\\[1mm]
		&&
		P \red P'\Rightarrow  \news{n} P   & \red  &  \news{n} P' 
		& \orule{Res}
		\\[1mm]

		P \red P' & \Rightarrow  &  P \Par Q  \red   P' \Par Q  
		& \orule{Par}
%		\\[1mm]
		&&
		P \scong Q \red Q' \scong P' & \Rightarrow & P  \red  P'
		& \orule{Cong}
	\end{array}
\]
{\small
\[
	\begin{array}{c}
		P \Par \inact \scong P
		\quad
		P_1 \Par P_2 \scong P_2 \Par P_1
		\quad
		P_1 \Par (P_2 \Par P_3) \scong (P_1 \Par P_2) \Par P_3
%		\\[1mm]
		\quad
		\news{n} \inact \scong \inact
%		\quad 
		\\[1mm]
		P \Par \news{n} Q \scong \news{n}(P \Par Q)
		\ (n \notin \fn{P})
		\quad 
		\recp{X}{P} \scong P\subst{\recp{X}{P}}{\rvar{X}}
%		\\[1mm]
		\quad
		P \scong Q \textrm{ if } P \scong_\alpha Q
%		\qquad
%		\dk{V \scong W \textrm{ if } V \scong_\alpha W
%\quad \abs{x}{P} \scong \abs{x}{Q} \textrm{ if } P \scong Q}
	\end{array}
\]
}
\caption{Operational Semantics of $\HOp$. 
\label{fig:reduction}}
%\Hlinefig
\end{figure}



\paragraph{Subcalculi.}
%\label{subsec:subcalculi}
%\noi 
As motivated in the introduction, 
we define two \emph{subcalculi} of \HOp: 
%We now define several sub-calculi of \HOp. 
\begin{enumerate}[$\bullet$]
	\item	The  
		{\em core higher-order session calculus}, denoted \HO,
		 lacks recursion and name passing; its 
		formal syntax is obtained from \figref{fig:syntax} by excluding 
		constructs in \nonhosyntax{\text{grey}}.

	\item	The   
		 {\em session $\pi$-calculus}, 
		denoted $\sessp$, 
		lacks  
		higher-order constructs
		(i.e., abstraction passing and application), but includes recursion.

%	\item	The third sub-calculus, denoted \haskp, represents cloud Haskell:
%		\[
%			\begin{array}{rclllll}
%				V,W	& ::= &		u \bnfbar  \abs{x}{P}
%				\\
%				P,Q	& ::= &		\bout{u}{m}{P}  \bnfbar  \binp{u}{x}{P} \bnfbar
%							\bsel{u}{l} P \bnfbar \bbra{u}{l_i:P_i}_{i \in I}
%				\\[1mm]
%					& \bnfbar &	\appl{V}{V} \bnfbar P\Par Q \bnfbar \news{n} P \bnfbar \inact
%		\end{array}
%		\]
\end{enumerate}
%
Let $\CAL \in \{\HOp, \HO, \sessp\}$. We write 
$\CAL^{-\mathsf{sh}}$ to denote the calculus $\CAL$ without shared names:
we delete $a,b$ from $n$. 
In \secref{sec:positive}
we shall demonstrate that 
$\HOp$, $\HO$, and $\sessp$ have the same expressivity,
and that $\CAL$ is strictly more expressive than $\CAL^{-\mathsf{sh}}$.


%%%%%%%%%%%%%%%%%%%%%%%%%%%%%%%%%%%%%%%%%%%%%%%%%%%%%%%%%%%%%%%%%%%%%%%%%%%%%%%%
%%%%%%%%%%%%%%%%%%%%%%%%%%%%%%%%%%%%%%%%%%%%%%%%%%%%%%%%%%%%%%%%%%%%%%%%%%%%%%%%

\section{Session Types for \HOp}
\label{sec:types}

We define a session type system for \HOp and state
\emph{type soundness} (\thmref{t:sr}), 
its main property.
Our system distills the key features of~\cite{tlca07,MostrousY15} and so it is simpler.


%The system almost identical with the system developed in~\cite{characteristic_bis}
%and we describe it in brief.
%Our system is simpler than that in~\cite{tlca07,MostrousY15}, thus distilling the key
%features of higher-order sessions. %communications. %in a session-typed setting.

%\smallskip 

%\subsection{Types}
%\label{subsec:types}
%\paragraph{Types.}
The syntax of types of \HOp follows. We write $\Proc$ to denote the process type.
\[
	\begin{array}{rcl}
%		\text{(value)} &
		U & \bnfis &	\nonhosyntax{C} \bnfbar L
%		\\[1mm]  % \bnfbar \Proc
		\qquad \qquad
%		\text{(name)} 
		C  \bnfis		S \bnfbar \chtype{S} \bnfbar \chtype{L}
%		\\[1mm]
		\qquad \qquad
%		\text{(abstr)}
		L \bnfis		\shot{C} \bnfbar \lhot{C}
		\\[1mm]

%		\text{(session)} 
		S & \bnfis &	\btout{U} S \bnfbar \btinp{U} S \bnfbar \btsel{l_i:S_i}_{i \in I}
%		\\ 
%						& \bnfbar & 
						\bnfbar \btbra{l_i:S_i}_{i \in I} \bnfbar  \trec{t}{S} \bnfbar \vart{t}  \bnfbar \tinact
	\end{array}
\]
Value type $U$ includes
  first-order types $C$ and  higher-order
types $L$.
%Note that we dissallow type $\chtype{U}$, thus
%in the type discipline shared names cannot carry shared names.
%In name types, $\chtype{U}$ is shared name types 
%which are sent via shared names. 
Types $\shot{C}$ and $\lhot{C}$ denote
{\em shared} and {\em linear} higher-order 
%\jpc{functional}
types, respectively.
Session types, denoted by $S$, follow the standard binary session type syntax~\cite{honda.vasconcelos.kubo:language-primitives}, with
the extension that carried types $U$ may be higher-order.
Shared channel types are denoted $\chtype{S}$ and $\chtype{L}$.
%,
%used to type abstraction values.
%$\lhot{C}$ \cite{tlca07,mostrous_phd} ensures values which contain free 
%session names used once. 
 %We write $S$ to denote %binary 
%session types.  {\em Output type}
%$\btout{U} S$ %is assigned to a name that 
%first sends a value of
%type $U$ and then follows the type described by $S$.  Dually,
%$\btinp{U} S$ denotes an {\em input type}. The {\em branching type}
%$\btbra{l_i:S_i}_{i \in I}$ and the {\em selection type}
%$\btsel{l_i:S_i}_{i \in I}$ define the labelled choice. 
%We assume the {\em recursive type} $\trec{t}{S}$ is guarded,
%i.e.,  $\trec{t}{\vart{t}}$ is not allowed. 
%%We stress that carried type $U$ in $\btout{U} S$ and
%%$\btinp{U} S$ can contain free type variables, which is crucial
%%to encode $\HOp$ into $\HO$.
%Type $\tinact$ is the termination type. 
Types of \HO exclude $\nonhosyntax{C}$ from 
value types of \HOp; the types of \sessp exclude $L$. 
From each $\CAL \in \{\HOp, \HO, \pi \}$, $\CAL^{-\mathsf{sh}}$ 
excludes shared name types ($\chtype{S}$ and $\chtype{L}$), 
from name type $C$.

\newj{We write $S \dualof S'$ if $S$ is the \emph{dual} of $S'$.   
Intuitively, session type duality is  obtained by 
dualising $!$ by $?$, $?$ by $!$, $\oplus$ by $\&$, and $\&$ by $\oplus$,  
including the fixed point construction.
We use the \emph{co-inductive} definition of duality given in \cite{TGC14}.}
%(see \defref{def:dual} in the Appendix). 

%\smallskip 

%\subsection{Typing System of \HOp}
%\label{subsec:typing}
%\paragraph{Typing Environments / Judgements}
We consider \emph{environments} denoted $\Gamma$, $\Lambda$, and $\Delta$:
\[
	\begin{array}{l}
		\Gamma  \bnfis  \emptyset \bnfbar \Gamma \cat \varp{x}: \shot{C} \bnfbar \Gamma \cat u: \chtype{S} \bnfbar \Gamma \cat u: \chtype{L} 
		\bnfbar \Gamma \cat \rvar{X}: \Delta
\\
		\Lambda \bnfis  \emptyset \bnfbar \Lambda \cat \AT{x}{\lhot{C}}
		 \\
		\Delta   \bnfis   \emptyset \bnfbar \Delta \cat \AT{u}{S}
	\end{array}
\]
%Environment 
$\Gamma$ maps variables and shared names to value types, and recursive 
variables to session environments (see below);  
it admits weakening, contraction, and exchange principles.
$\Lambda$ maps variables to 
%the
linear
%functional 
higher-order
types;   $\Delta$  maps   
session names to session types. 
Both $\Lambda$ and $\Delta$
%behave linearly: they 
are
only subject to exchange.  
We require that the domains of $\Gamma,
\Lambda$ and $\Delta$ are pairwise distinct. 
$\Delta_1\cdot \Delta_2$ denotes the disjoint union of $\Delta_1$ and $\Delta_2$.  
We focus on \emph{balanced} session environments: 
%that contain dual endpoints typed with dual types.
%The following definition ensures two session endpoints 
%are dual each other. 

%\smallskip

\begin{definition}[Balanced]\label{d:wtenv}%\rm
	%Let $\Delta$ be a session environment.
	We say that a session environment $\Delta$ is {\em balanced} if whenever
	$s: S_1, \dual{s}: S_2 \in \Delta$ then $S_1 \dualof S_2$.
\end{definition}

Given the above intuitions for environments, 
the typing judgements for values $V$ and processes $P$ are self-explanatory.
They are denoted 
$\Gamma; \Lambda; \Delta \proves V \hastype U$ and $\Gamma; \Lambda; \Delta \proves P \hastype \Proc$.
%
%\begin{center}
%	\begin{tabular}{c}
%		$\Gamma; \Lambda; \Delta \proves V \hastype U \qquad \qquad \qquad \qquad \Gamma; \Lambda; \Delta \proves P \hastype \Proc$
%	\end{tabular}
%\end{center}
%%
%\noi The first judgement states that under environments $\Gamma; \Lambda; \Delta$ value $V$
%has type $U$, whereas the second judgement states that under
%environments $\Gamma; \Lambda; \Delta$ process $P$ has the process type~$\Proc$. %
 
%\smallskip

% !TEX root = ../journal16kpy.tex
\begin{figure}[t]
\[
	\begin{array}{c}
		\inferrule[(Prom)]{
			\Gamma; \emptyset; \emptyset \proves V \hastype 
                         \lhot{C}
		}{
			\Gamma; \emptyset; \emptyset \proves V \hastype 
                         \shot{C}
		} 
		\quad
		\inferrule[(EProm)]{
		\Gamma; \Lambda \cat x : \lhot{C}; \Delta \proves P \hastype \Proc
		}{
			\Gamma \cat x:\shot{C}; \Lambda; \Delta \proves P \hastype \Proc
		}
		\quad
		\inferrule[(Abs)]{
			\Gamma; \Lambda; \Delta_1 \proves P \hastype \Proc
			\quad
			\Gamma; \es; \Delta_2 \proves x \hastype C
		}{
			\Gamma\backslash x; \Lambda; \Delta_1 \backslash \Delta_2 \proves \abs{{x}}{P} \hastype \lhot{{C}}
		}
		\\[1mm] \\
		\inferrule[(App)]{
			\begin{array}{c}
				U = \lhot{C} \lor \shot{C}
				\\
				\Gamma; \Lambda; \Delta_1 \proves V \hastype U \quad
				\Gamma; \es; \Delta_2 \proves u \hastype C
			\end{array}
		}{
			\Gamma; \Lambda; \Delta_1 \cat \Delta_2 \proves \appl{V}{u} \hastype \Proc
		} 
		~~
		\inferrule[(Send)]{
					\begin{array}{c}
					u:S \in \Delta_1 \cat \Delta_2 \\
			\Gamma; \Lambda_1; \Delta_1 \proves P \hastype \Proc
			\quad
			\Gamma; \Lambda_2; \Delta_2 \proves V \hastype U
			\end{array}
		}{
			\Gamma; \Lambda_1 \cat \Lambda_2; ((\Delta_1 \cat \Delta_2) \setminus u:S) \cat u:\btout{U} S \proves \bout{u}{V} P \hastype \Proc
		}
		\\[2mm] \\
		\inferrule*[left=(Rcv)]{
		\begin{array}{c}
			\Gamma; \Lambda_1; \Delta_1 \cat u: S \proves P \hastype \Proc
			\quad
			\Gamma; \Lambda_2; \Delta_2 \proves {x} \hastype {U}
			\end{array}
		}{
			\Gamma \backslash x; \Lambda_1\cat \Lambda_2; \Delta_1\backslash \Delta_2 \cat u: \btinp{U} S \vdash \binp{u}{{x}} P \hastype \Proc
		}
		\\[2mm] \\
		\inferrule[(Req)]{
			\begin{array}{c}
				\Gamma; \es; \es \proves u \hastype U_1
				\quad
				\Gamma; \Lambda; \Delta_1 \proves P \hastype \Proc
				\\
				\Gamma; \es; \Delta_2 \proves V \hastype U_2
				\\
				(U_1 = \chtype{S} 
                                \land %\Leftrightarrow 
                                U_2 = S)
				\lor
				 (U_1 = \chtype{L} 
                                \land %\Leftrightarrow 
                                %\Leftrightarrow 
                                 U_2 = L)
			\end{array}
		}{
			\Gamma; \Lambda; \Delta_1 \cat \Delta_2 \proves \bout{u}{V} P \hastype \Proc
		}
		~~
		\inferrule[(Acc)]{
			\begin{array}{c}
				\Gamma; \emptyset; \emptyset \proves u \hastype U_1 
				\quad
				\Gamma; \Lambda_1; \Delta_1 \proves P \hastype \Proc
				\\
				\Gamma; \Lambda_2; \Delta_2 \proves x \hastype U_2\\
				(U_1 = \chtype{S} 
                                \land %\Leftrightarrow 
                                U_2 = S)
				\lor
				 (U_1 = \chtype{L} 
                                \land %\Leftrightarrow 
                                %\Leftrightarrow 
                                 U_2 = L)
	               \end{array}
		}{
			\Gamma\backslash x; \Lambda_1 \backslash \Lambda_2; \Delta_1 \backslash \Delta_2 \proves \binp{u}{x} P \hastype \Proc
		}	
		\end{array}
\]
%\vspace{-3mm}
\caption{Selected Typing Rules for $\HOp$.
See \appref{app:types} for a full account.
\label{fig:typerulesmys}}
%\Hline
%\vspace{-1mm}
\end{figure}
%\myparagraph{Typing System of \HOp}




%\paragraph{Typing Rules} 

Selected typing rules are given in \figref{fig:typerulesmys}; 
%see \appref{app:typrules} 
see \cite{KouzapasPY15} for a full account.
%Types for session names/variables $u$ and
%directly derived from the linear part of the typing
%environment, i.e.~type maps $\Delta$ and $\Lambda$.
%Rules $\trule{Sess, Sh, LVar}$ are name and variable introduction rules. 
The shared type $\shot{C}$ %for shared higher order values $V$
is derived using rule \textsc{(Prom)} only  
if the value has a linear type with an empty linear
environment.
Rule~\textsc{(EProm)} allows us to freely use a \newj{shared
type variable as linear}.
%
Abstraction values are typed with rule~\textsc{(Abs)}.
%The key type for an abstraction is the type for
%the bound variables of the abstraction, i.e.~for
%bound variable type $C$ the abstraction
%has type $\lhot{C}$.
Application typing
is governed by rule~\textsc{(App)}: we expect
the type $C$ of an application name $u$ 
to match the type $\lhot{C}$ or $\shot{C}$
of the application variable $x$.
%
%A process prefixed with a session send operator $\bout{k}{V} P$
%is typed using rule $\trule{Send}$.
In rule~\textsc{(Send)}, 
the type $U$ of a send value $V$ should appear as a prefix
on the session type $\btout{U} S$ of $u$.
Rule~\textsc{(Rcv)} is its dual.  
%defined the typing for the 
%reception of values $\binp{u}{V} P$.
%the type $U$ of a receive value should 
%appear as a prefix on the session type $\btinp{U} S$ of $u$.
We use a similar approach with session prefixes
to type interaction between shared names as defined 
in rules~\textsc{(Req)} and~\textsc{(Acc)},
where the type of the sent/received object 
($S$ and $L$, resp.) should
match the type of the sent/received subject
($\chtype{S}$ and $\chtype{L}$, resp.).


\begin{definition}%[Reduction of Session Environments]%\rm
	\label{def:ses_red}
	We define the relation $\red$ on session environments as:
	\begin{eqnarray*}
			\Delta \cat s: \btout{U} S_1 \cat \dual{s}: \btinp{U} S_2 & \red &
			\Delta \cat s: S_1 \cat \dual{s}: S_2\\%[1mm]
			\Delta \cat s: \btsel{l_i: S_i}_{i \in I} \cat \dual{s}: \btbra{l_i: S_i'}_{i \in I} &\red& 
			 \Delta \cat s: S_k \cat \dual{s}: S_k' \ (k \in I)
		\end{eqnarray*}
	%\end{center}
%\begin{tabular}{rcl}
%	\setlength{\tabcolsep}{0pt}
%	$\Delta \cat s: \btout{U} S_1 \cat \dual{s}: \btinp{U} S_2$ & $\red$ & 
%	$\Delta \cat s: S_1 \cat \dual{s}: S_2$\\[1mm]
%	$\Delta \cat s: \btsel{l_i: S_i}_{i \in I} \cat \dual{s}: \btbra{l_i: S_i'}_{i \in I}$ & $\red$ & $\Delta \cat s: S_k \cat \dual{s}: S_k' \ (k \in I)$
%\end{tabular}
%\[
%\begin{array}{rcl}
%\Delta \cat s: \btout{U} S_1 \cat \dual{s}: \btinp{U} S_2 & \red & 
%\Delta \cat s: S_1 \cat \dual{s}: S_2\\[1mm]
%\Delta \cat s: \btsel{l_i: S_i}_{i \in I} \cat \dual{s}: \btbra{l_i: S_i'}_{i \in I} & \red & \Delta \cat s: S_k \cat \dual{s}: S_k' \ (k \in I)
%\end{array}
%\]
\end{definition}

%\smallskip

%The following result %Theorem 7.3 in M\&Y
%\noi 
We state  type soundness for \HOp; it implies 
type soundness for \HO, \sessp, and $\CAL^{-\mathsf{sh}}$. 

%\smallskip

\begin{theorem}[Type Soundness]\label{t:sr}\rm
%	\begin{enumerate}[1.]
%		\item	(Subject Congruence) Suppose $\Gamma; \es; \Delta \proves P \hastype \Proc$.
%			Then $P \scong P'$ implies $\Gamma; \es; \Delta \proves P' \hastype \Proc$.
%
%		\item
%			(Subject Reduction)
			Suppose $\Gamma; \es; \Delta \proves P \hastype \Proc$
			with
			$\Delta$ balanced. 
			Then $P \red P'$ implies $\Gamma; \es; \Delta'  \proves P' \hastype \Proc$
			and $\Delta = \Delta'$ or $\Delta \red \Delta'$
			with $\Delta'$ balanced. 
%	\end{enumerate}
\end{theorem}


%%%%%%%%%%%%%%%%%%%%%%%%%%%%%%%%%%%%%%%%%%%%%%%%%%%%%%%%%%%%%%%%%%%%%%%%%%%%%%%%
%%%%%%%%%%%%%%%%%%%%%%%%%%%%%%%%%%%%%%%%%%%%%%%%%%%%%%%%%%%%%%%%%%%%%%%%%%%%%%%%

\section{Behavioural Theory for \HOp}\label{sec:bt}
%% !TEX root = main.tex
\noi In this section we define reduction-closed, barbed congruence ($\cong$) as the
reference equivalence relation for \HOp processes.
Later on we will define two characterizations of $\cong$:
\emph{higher-order} and  
\emph{characteristic bisimilarities} (denoted $\hwb$ and $\fwb$, respectively). 
Here we focus on collecting intuitions; omitted details are in the Appendix and in~\cite{KouzapasPY15}.

\subsection{Reduction-Closed, Barbed Congruence}
\label{subsec:rc}
We first define \emph{confluence} over session environments $\Delta$:

\begin{definition}[Session Environment Confluence]
Let $\red^\ast$ denote multi-step reduction as in \defref{def:ses_red}.
	We denote $\Delta_1 \bistyp \Delta_2$ if there exists $\Delta$ such that
	$\Delta_1 \red^\ast \Delta$ and $\Delta_2 \red^\ast \Delta$.
\end{definition}

%\smallskip 
\noi Reduction-closed, barbed congruence is defined over typed
processes:

\begin{definition}[Typed Relation]
	We say that
	$\Gamma; \emptyset; \Delta_1 \proves P_1 \hastype \Proc\ \Re \ \Gamma; \emptyset; \Delta_2 \proves P_2 \hastype \Proc$
	is a {\em typed relation} whenever
	$P_1$ and $P_2$ are closed;
	$\Delta_1$ and $\Delta_2$ are balanced; and 
	$\Delta_1 \bistyp \Delta_2$.
	We write $\horel{\Gamma}{\Delta_1}{P_1}{\ \Re \ }{\Delta_2}{P_2}$
	for the typed relation $\Gamma; \emptyset; \Delta_1 \proves P_1 \hastype \Proc\ \Re \ \Gamma; \emptyset; \Delta_2 \proves P_2 \hastype \Proc$.
\end{definition}

%\smallskip 

\noi Observe that typed relations relate only closed terms whose
session environments %and the two session environments
are balanced  and confluent.
Next we define  {\em barbs}~\cite{MiSa92}
with respect to types. 

%\smallskip 

\begin{definition}[Barbs]\rm
	Let $P$ be a closed process. We define:
	\begin{enumerate}
		\item	$P \barb{n}$ if $P \scong \newsp{\tilde{m}}{\bout{n}{V} P_2 \Par P_3}, n \notin \tilde{m}$. %; $P \Barb{n}$ if $P \red^* \barb{n}$.

		\item	$\Gamma; \Delta \proves P \barb{n}$ if
			$\Gamma; \emptyset; \Delta \proves P \hastype \Proc$ with $P \barb{n}$ and $\dual{n} \notin \dom{\Delta}$.

			$\Gamma; \Delta \proves P \Barb{n}$ if $P \red^* P'$ and
			$\Gamma; \Delta' \proves P' \barb{n}$.			
	\end{enumerate}
\end{definition}

%\smallskip 

\noi A barb $\barb{n}$ is an observable on an output prefix with subject $n$.
Similarly a weak barb $\Barb{n}$ is a barb after a number of reduction steps.
Typed barbs $\barb{n}$ (resp.\ $\Barb{n}$)
occur on typed processes $\Gamma; \emptyset; \Delta \proves P \hastype \Proc$.
When $n$ is a session name we require that its dual endpoint $\dual{n}$ is not present
in the session environment $\Delta$.

To define a congruence relation, we introduce the family $\C$ of contexts:

\begin{definition}[Context]
	A context $\C$ is defined as:

	\begin{tabular}{rcl}
		$\C$ & $\bnfis$ & $\hole \bnfbar \bout{u}{V} \C \bnfbar \binp{u}{x} \C \bnfbar \bout{u}{\lambda x.\C} P \bnfbar \news{n} \C
		\bnfbar (\lambda x.\C)u \bnfbar \recp{X}{\C}$ 
		\\
		&$\bnfbar$& $\C \Par P \bnfbar P \Par \C
		\bnfbar \bsel{u}{l} \C \bnfbar \bbra{u}{l_1:P_1,\cdots,l_i:\C,\cdots,l_n:P_n}$
	\end{tabular}
%\smallskip 

\noi 
Notation $\context{\C}{P}$ replaces 
the hole $\hole$ in $\C$ with $P$.
\end{definition}

%\smallskip 

\noi We define reduction-closed, barbed congruence \cite{HondaKYoshida95}. 

%\smallskip 

\begin{definition}[Reduction-Closed, Barbed Congruence]
\label{def:rc}
	Typed relation
	$\horel{\Gamma}{\Delta_1}{P_1}{\ \Re\ }{\Delta_2}{P_2}$
	is a {\em reduction-closed, barbed congruence} whenever:
	\begin{enumerate}[1)]
		\item	If $P_1 \red P_1'$ then there exist $P_2', \Delta_2'$ such that $P_2 \red^* P_2'$ and
			$\horel{\Gamma}{\Delta_1'}{P_1'}{\ \Re\ }{\Delta_2'}{P_2'}$; 
			%and its symmetric case;
%		\item	If $P_2 \red P_2'$ then $\exists P_1', P_1 \red^* P_1'$ and
%		$\horel{\Gamma}{\Delta_1'}{P_1'}{\ \Re\ }{\Delta_2'}{P_2'}$
%		\end{itemize}

%		\item
%		\begin{itemize}
			\item	If $\Gamma;\Delta_1 \proves P_1 \barb{n}$ then $\Gamma;\Delta_2 \proves P_2 \Barb{n}$; %and its symmetric case; 

%			\item	If $\Gamma;\emptyset;\Delta \proves P_2 \barb{s}$ then $\Gamma;\emptyset;\Delta \proves P_1 \Barb{s}$.
%		\end{itemize}

		\item	For all $\C$, there exist $\Delta_1'',\Delta_2''$: $\horel{\Gamma}{\Delta_1''}{\context{\C}{P_1}}{\ \Re\ }{\Delta_2''}{\context{\C}{P_2}}$;
		                      \item	The symmetric cases of 1 and 2.                
	\end{enumerate}
	The largest such relation is denoted with $\cong$.
\end{definition}

{
\subsection{Two Equivalence Relations: $\fwb$ and $\hwb$}\label{ss:equiv}

\jparagraph{A Typed Labelled Transition System}
In~\cite{characteristic_bis} we have characterised
reduction-closed, barbed congruence for \HOp
via a typed relation called
{\em characteristic bisimilarity}.
%The definition of characteristic bisimilarity 
Its definition 
uses a \emph{typed}
labelled transition system (LTS) informed by session
types. 
Transitions in this LTS are denoted 
$\Gamma; \es; \Delta \proves P \hby{\ell} \Delta' \proves P' \hastype \Delta'$.
The main intuition %for its definition  
is that the transitions 
of a typed process should be enabled by its associated typing environment:
%
\[
	\tree {
		P \hby{\ell} P' \qquad (\Gamma, \Delta) \hby{\ell} (\Gamma, \Delta')
	}{
		\Gamma; \es; \Delta \proves P \hby{\ell} \Delta' \proves P' \hastype \Delta'
	}
\]
%
\noi As an example of how types enable transitions, consider the rule for input:
%
\[
	\tree{
		\dual{s} \notin \dom{\Delta} \qquad \Gamma; \Lambda'; \Delta' \proves V \hastype U
		\qquad
		V = m \vee  V \scong \omapchar{U} \vee V \scong \abs{{x}}{\binp{t}{y} (\appl{y}{{x}})}
					\textrm{ with } t \textrm{ fresh} 
	}{
		(\Gamma; \Lambda; \Delta \cat s: \btinp{U} S) \hby{\bactinp{s}{V}} (\Gamma; \Lambda\cat\Lambda'; \Delta\cat\Delta' \cat s: S)
	}
\]
\noi
This rule states that an session channel environment can input a value
if the channel is typed with an input prefix and the input value is either
a name $m$, a \emph{characteristic value} $\omapchar{U}$,  or a \emph{trigger value} (the abstraction
$\abs{{x}}{\binp{t}{y} (\appl{y}{{x}})}$). 
A characteristic value
is the {simplest} process that  inhabits a type (in this case, the
type $U$ carried by the session input prefix). The above rule is used to limit
the input actions that can be observed from a session input prefix.
For full details of the labelled transition system and the characteristic process definition
see \appref{app:behavioural} and~\cite{characteristic_bis}.
Moreover, we define two \emph{trigger processes}:
%
\begin{eqnarray*}
	\htrigger{t}{V}  & \defeq &  \hotrigger{t}{V} \label{eqb:0} \\
	\ftrigger{t}{V}{U} & \defeq &  \fotrigger{t}{x}{s}{\btinp{U} \tinact}{V} 	%\label{eqb:4}
\end{eqnarray*}
%
\noi
Trigger processes are defined as a process input prefixed on
a fresh name $t$. The   trigger process $\htrigger{t}{V}$ above applies a value $V$
on the receiving (characteristic) process.
The trigger process $\ftrigger{t}{V}{U}$ applies a value
on the (hard coded) \emph{characteristic process} $\map{\btinp{U} \tinact}^{s}$ (see~\cite{characteristic_bis} for details). 

\jparagraph{Characterisations of Barbed Congruence}
We now define \emph{characteristic} and \emph{higher-order} bisimilarity.
Observe that higher-order bisimilarity is a new typed equality, 
while 
characteristic bisimilarity was introduced in~\cite{characteristic_bis} (Def.~14).

\begin{definition}[Characteristic Bisimilarity]%\rm
	\label{d:fwb}
	A typed relation $\Re$ is a {\em  characteristic bisimulation} if 
	for all $\horel{\Gamma}{\Delta_1}{P_1}{\ \Re \ }{\Delta_2}{Q_1}$ 
	\begin{enumerate}[1)]
		\item 
				Whenever 
				$\horel{\Gamma}{\Delta_1}{P_1}{\hby{\news{\tilde{m_1}} \bactout{n}{V_1: U}}}{\Delta_1'}{P_2}$,
				there exist 
				$Q_2$, $V_2$, $\Delta'_2$
				such that \\
				$\horel{\Gamma}{\Delta_2}{Q_1}{\Hby{\news{\tilde{m_2}} \bactout{n}{V_2: U}}}{\Delta_2'}{Q_2}$ and, for fresh $t$, 
%
				\[
					\begin{array}{lrlll}
						\Gamma; \Delta''_1  \proves  {\newsp{\tilde{m_1}}{P_2 \Par \ftrigger{t}{V_1}{U_1}}}
						\ \Re \
						\Delta''_2 \proves {\newsp{\tilde{m_2}}{Q_2 \Par \ftrigger{t}{V_2}{U_2}}}
					\end{array}
				\]

		\item	
				For all $\horel{\Gamma}{\Delta_1}{P_1}{\hby{\ell}}{\Delta_1'}{P_2}$ such that 
				$\ell$ is not an output, 
				there exist $Q_2$, $\Delta'_2$ such that 
				$\horel{\Gamma}{\Delta_2}{Q_1}{\Hby{\hat{\ell}}}{\Delta_2'}{Q_2}$
				and
				$\horel{\Gamma}{\Delta_1'}{P_2}{\ \Re \ }{\Delta_2'}{Q_2}$; and 

		\item	The symmetric cases of 1 and 2.                
	\end{enumerate}
	The largest such bisimulation
	is called \emph{characteristic bisimilarity} \jpc{and} denoted by $\fwb$.
\end{definition}
 

%\noi 
%We define characteristic bisimilarity
%(cf.~Def.~14~in~\cite{characteristic_bis}):
% is given using characteristic trigger processes. 

\begin{definition}[Higher-Order Bisimilarity]%\rm
	\label{d:hbw}
	Higher-order bisimilarity, denoted by $\hwb$, is defined \jpc{by} replacing 
	Clause 1) in \defref{d:fwb} with the following clause:\\[1mm]
	Whenever 
	$\horel{\Gamma}{\Delta_1}{P_1}{\hby{\news{\tilde{m_1}} \bactout{n}{V_1}}}{\Delta_1'}{P_2}$ %with $\Gamma; \es; \Delta \proves V_1 \hastype U$,  
	then there exist 
	$Q_2$, $V_2$, $\Delta'_2$
	such that \\
	$\horel{\Gamma}{\Delta_2}{Q_1}{\Hby{\news{\tilde{m_2}}\bactout{n}{V_2}}}{\Delta_2'}{Q_2}$ %with $\Gamma; \es; \Delta' \proves V_2 \hastype U$,  
	and, for fresh $t$, \\[1mm]
	$
	\begin{array}{lrlll}
		\!\!\Gamma; \Delta''_1  \proves  {\newsp{\tilde{m_1}}{P_2 \Par 
		\htrigger{t}{V_1}}}
		\ \!\!\Re\!\!\ \Delta''_2
		\proves {\newsp{\tilde{m_2}}{Q_2 \Par \htrigger{t}{V_2}}}
	\end{array}
	$
\end{definition}

\begin{theorem}[\cite{KouzapasPY15}]
	Typed relations $\cong$, $\hwb$, and $\fwb$ coincide for \HOp processes.
\end{theorem}


\begin{remark}[Comparison between $\hwb$ and $\fwb$]
	The difference between higher order bisimilarity
	lies in the definition of the process trigger.
	\begin{enumerate}
		\item
				Higher-order bisimilarity trigger:
				\[
					\htrigger{t}{V}  \defeq  \hotrigger{t}{V}
				\]
				allows for the application of bound variable $x$,
				making $\hwb$ not defined in the \sessp sub-calculus
				of \HOp, but defined inside the \HO sub-calculus. 

		\item	Characteristic bisimilarity trigger:
				\[
					\ftrigger{t}{V}{U} \defeq  \fotrigger{t}{x}{s}{\btinp{U} \tinact}{V}
				\]
				solves the problem in (1) by not using $x$ in the continuation
				of fresh name input. This allows to input characteristic, first- or
				higher-order value on bound variable $x$.


		\item	Higher-order trigger lifts the need of observing
				types on the lts and including them in bisimulation
				closures. Higher-order bisimilarity adds to
				the simplicity and elegance of the \HO subcalculus
				over the \sessp subcalculus of \HOp.
	\end{enumerate}
\end{remark}

%\begin{definition}[Characteristic Bisimilarity]\rm
%	\label{d:fwb}
%
%	A typed relation $\Re$ is a {\em characteristic bisimulation} if 
%	for all $\horel{\Gamma}{\Delta_1}{P_1}{\ \Re \ }{\Delta_2}{Q_1}$ 
%	if whenever:
%	\begin{enumerate}[1)]
%		\item	$\horel{\Gamma}{\Delta_1}{P_1}{\hby{\news{\tilde{m_1}} \bactout{n}{V_1: U}}}{\Delta_1'}{P_2}$
%				then there exist  $Q_2$, $V_2$, $\Delta'_2$ such that 
%				$\horel{\Gamma}{\Delta_2}{Q_1}{\Hby{\news{\tilde{m_2}}\bactout{n}{V_2: U}}}{\Delta_2'}{Q_2}$
%				and, for fresh $t$,
%				$
%				\begin{array}{lrlll}
%					\Gamma; \Delta''_1 \proves {\newsp{\tilde{m_1}}{P_2 \Par  \ftrigger{t}{V_1}{U_1}}}
%					\ \Re\ \Delta''_2 \proves {\newsp{\tilde{m_2}}{Q_2 \Par \ftrigger{t}{V_2}{U_2}}}
%				\end{array}
%				$
%
%			\item	For all $\horel{\Gamma}{\Delta_1}{P_1}{\hby{\ell}}{\Delta_1'}{P_2}$ such that 
%					$\ell$ is not an output, there exist $Q_2$, $\Delta'_2$ such that 
%					$\horel{\Gamma}{\Delta_2}{Q_1}{\Hby{\hat{\ell}}}{\Delta_2'}{Q_2}$
%					and $\horel{\Gamma}{\Delta_1'}{P_2}{\ \Re \ }{\Delta_2'}{Q_2}$; and 
%
%			\item	The symmetric cases of 1 and 2.
%	\end{enumerate}
%	The largest such bisimulation is called \emph{characteristic bisimilarity} and denoted by $\fwb$.
%\end{definition}


\jparagraph{Up-to techniques}
We close this section by stating  a determinacy result for typed processes, useful in proving our expressiveness results (\secref{sec:positive}).
In our setting, processes that do not use shared names are inherently deterministic. 
In the sequel, we write 
 $\horel{\Gamma}{\Delta}{P}{\hby{\tau}}{\Delta'}{P'}$ to denote an internal (typed) transition.
The auxiliary definition below allows us to distinguish two kinds of  internal transitions:
\emph{session transitions} and \emph{$\beta$-transitions} (denoted 
$\horel{\Gamma}{\Delta}{P}{\hby{\stau}}{\Delta'}{P'}$
and $\horel{\Gamma}{\Delta}{P}{\hby{\btau}}{\Delta'}{P'}$, respectively).

\begin{definition}[Deterministic Transition]
\label{def:dettrans}
	Let  $\Gamma; \es; \Delta \proves P \hastype \Proc$ be a balanced \HOp process. 
	Transition $\horel{\Gamma}{\Delta}{P}{\hby{\tau}}{\Delta'}{P'}$ is called:
	\begin{enumerate}[$-$]
		\item	{\em Session transition}
				whenever the untyped transition $P \by{\tau} P'$ 
				is derived using  rule~$\ltsrule{Tau}$ 
				(where $\subj{\ell_1}$ and $\subj{\ell_2}$ in the premise are dual endpoints), 
				possibly followed by uses of
				$\ltsrule{Alpha}$, $\ltsrule{Res}$, $\ltsrule{Rec}$, or $\ltsrule{Par${}_L$}/
				\ltsrule{Par${}_R$}$.
		
		\item	{\em $\beta$-transition}
				whenever the untyped transition $P \by{\tau} P'$
				is derived using rule $\ltsrule{App}$,
				possibly followed by uses of  $\ltsrule{Alpha}$, $\ltsrule{Res}$, $\ltsrule{Rec}$, or $\ltsrule{Par${}_L$}/
				\ltsrule{Par${}_R$}$.
	\end{enumerate}
%
	We write
	$\horel{\Gamma}{\Delta}{P}{\hby{\stau}}{\Delta'}{P'}$
	and 
	$\horel{\Gamma}{\Delta}{P}{\hby{\btau}}{\Delta'}{P'}$
	to denote session and $\beta$-transitions, resp. Also, 
	 $\horel{\Gamma}{\Delta}{P}{\hby{\dtau}}{\Delta'}{P'}$ denotes
	either a session transition or a $\beta$-transition.
\end{definition}
%
%A transition $\horel{\Gamma}{\Delta}{P}{\hby{\tau}}{\Delta'}{P'}$ is said
%{\em deterministic} if it is derived using~$\ltsrule{App}$ or~$\ltsrule{Tau}$ 
%(where $\subj{\ell_1}$ and $\subj{\ell_2}$ in the premise  are dual endpoints), 
%possibly followed by uses of  $\ltsrule{Alpha}$, $\ltsrule{Res}$, $\ltsrule{Rec}$, or $\ltsrule{Par${}_L$}/\ltsrule{Par${}_R$}$.

\begin{lemma}[$\tau$-Inertness]\rm
	\label{lem:tau_inert}
	\begin{enumerate}[1)]
		\item
				Let $\horel{\Gamma}{\Delta}{P}{\hby{\dtau}}{\Delta'}{P'}$ be a deterministic transition,
				with balanced $\Delta$. \\ Then 
				$\Gamma; \Delta \proves P \cong \Delta'\proves P'$ 
				with $\Delta \red^\ast \Delta'$ balanced.

		\item 
				Let $P$ be an $\HOp^{-\mathsf{sh}}$ process. 
				Assume $\Gamma; \emptyset; \Delta \proves P \hastype \Proc$. \\ Then 
				$P \red^\ast P'$ implies $\Gamma; \Delta \proves 
				P \cong \Delta'\proves P'$ with $\Delta \red^\ast \Delta'$. 
	\end{enumerate}
\end{lemma}


\begin{proof}
	The proof uses the fact that processes of the
	form $\Gamma; \es; \Delta \proves_s \bout{s}{V} P_1 \Par \binp{k}{x} P_2$
	cannot have any typed transition observables and the fact
	that bisimulation is a congruence.
	See  \appref{app:sub_tau_inert} for details.
	The proof for Part 2 follows from Part 1.
	\qed
\end{proof}




\begin{lemma}[Up-to Deterministic Transition]%\myrm
	\label{lem:up_to_deterministic_transition}
	Let $\horel{\Gamma}{\Delta_1}{P_1}{\ \Re\ }{\Delta_2}{Q_1}$ such
	that if whenever:
%
	\begin{enumerate}
		\item	$\forall \news{\tilde{m_1}} \bactout{n}{V_1}$ such that
			$
				\horel{\Gamma}{\Delta_1}{P_1}{\hby{\news{\tilde{m_1}} \bactout{n}{V_1}}}{\Delta_3}{P_3}
			$
			implies that $\exists Q_2, V_2$ such that
			$
				\horel{\Gamma}{\Delta_2}{Q_1}{\Hby{\news{\tilde{m_2}} \bactout{n}{V_2}}}{\Delta_2'}{Q_2}
			$
			and
			$
				\horel{\Gamma}{\Delta_3}{P_3}{\Hby{\dtau}}{\Delta_1'}{P_2}
			$
			and for fresh $t$:
			$
				\horel{\Gamma}{\Delta_1''}{\newsp{\tilde{m_1}}{P_2 \Par \htrigger{t}{V_1}}}
				{\ \Re\ }
				{\Delta_2''}{}{\newsp{\tilde{m_2}}{Q_2 \Par \htrigger{t}{V_2}}}
%				\mhorel{\Gamma}{\Delta_1''}{\newsp{\tilde{m_1}}{P_2 \Par \hotrigger{t}{x}{s}{V_1}}}
%				{\ \Re\ }
%				{\Delta_2''}{}{\newsp{\tilde{m_2}}{Q_2 \Par \hotrigger{t}{x}{s}{V_2}}}
			$
%
		\item	$\forall \ell \not= \news{\tilde{m}} \bactout{n}{V}$ such that
			$
				\horel{\Gamma}{\Delta_1}{P_1}{\hby{\ell}}{\Delta_3}{P_3}
			$
			implies that $\exists Q_2$ such that 
			$
				\horel{\Gamma}{\Delta_1}{Q_1}{\hat{\Hby{\ell}}}{\Delta_2'}{Q_2}
			$
			and
			$
				\horel{\Gamma}{\Delta_3}{P_3}{\Hby{\dtau}}{\Delta_1'}{P_2}
			$
			and
			$\horel{\Gamma}{\Delta_1'}{P_2}{\ \Re\ }{\Delta_2'}{Q_2}$

		\item	The symmetric cases of 1 and 2.
	\end{enumerate}
	Then $\Re\ \subseteq\ \wb$.
\end{lemma}


\begin{proof}
	The proof is easy by considering the closure
	\[
		\Re^{\Hby{\dtau}} = \set{ \horel{\Gamma}{\Delta_1'}{P_2}{,}{\Delta_2'}{Q_1} \setbar \horel{\Gamma}{\Delta_1}{P_1}{\ \Re\ }{\Delta_2'}{Q_1},
		\horel{\Gamma}{\Delta_1}{P_1}{\Hby{\dtau}}{\Delta_1'}{P_2} }
	\]
	We verify that $\Re^{\Hby{\dtau}}$ is a bisimulation with
	the use of \propref{lem:tau_inert}.
	\qed
\end{proof}

\begin{example}[Up-to Deterministic Transition]
	Typed processes:
	\begin{eqnarray*}
		\Gamma; \es; \Delta, s': \tinact \proves P &=& \binp{n}{z_1} \newsp{s}{\binp{s}{x} \appl{(\abs{y}{\bout{n}{z_1} \inact})}{m} \Par \bout{\dual{s}}{s'} \inact} \hastype \Proc
		\\
		\Gamma; \es; \Delta \proves Q &=& \binp{n}{z_1} \binp{n}{z_2} \inact \hastype \Proc
	\end{eqnarray*}
	are bisimilar up-to deterministic transition because
	we can observe:
	\begin{eqnarray*}
		\Gamma; \Delta, s': \tinact \proves P &\hby{\bactinp{n}{m_1}}& \Delta', s: \tinact \proves \newsp{s}{\binp{s}{x} \appl{(\abs{y}{\bout{n}{z_2} \inact})}{m} \Par \bout{\dual{s}}{s'} \inact} \Hby{\dtau} \Delta' \proves \binp{n}{z_2} \inact
		\\
		\text{and}
		\\
		\Gamma; \es; \Delta \proves Q &\hby{\bactinp{n}{m_1}}& \Delta' \proves \binp{n}{z_2} \inact
	\end{eqnarray*}


%	Relation 
%	\[
%		\Re = \set{(\Gamma; \Delta, s': \tinact \proves P , \Delta \proves Q), (\Gamma; \Delta' \proves \binp{n}{z_2}, \Delta' \proves  \binp{n}{z_2})}
%	\]
%	is bisimulation up-to deterministic transition because
%	\begin{eqnarray*}
%		\Gamma; \Delta, s': \tinact \proves P &\hby{\bactinp{n}{s_1}}& \Delta', s: \tinact \proves \newsp{s}{\binp{s}{x} \appl{(\abs{y}{\bout{n}{y} \inact})}{s_1} \Par \bout{\dual{s}}{s'} \inact}
%		\\
%		\text{implies}&
%		\\
%		\Gamma; \es; \Delta \proves Q &\hby{\bactinp{n}{x}}& \Delta' \proves \binp{n}{z_2} \inact
%		\\
%		\text{and}&
%		\\
%		\Delta', s: \tinact \proves \newsp{s}{\binp{s}{x} \appl{(\abs{y}{\bout{n}{y} \inact})}{s_1} \Par \bout{\dual{s}}{s'} \inact}  \in \Re
%	\end{eqnarray*}
\end{example}

%\noi Precise encodings offer more detailed criteria and used for positive 
%encodability results (\secref{sec:positive}).
%In contrast, minimal encodings contains only 
%some of the criteria of precise encodings:    
%this reduced notion will be used 
%for the negative result in \secref{sec:negative}.




We first define reduction-closed, barbed congruence ($\cong$, \defref{def:rc}) as the
reference equivalence relation for \HOp processes.
We then define two characterizations of $\cong$:
\emph{characteristic} and
\emph{higher-order bisimilarities}   
 (denoted $\fwb$ and $\hwb$, cf. \defsref{d:fwb} and \ref{d:hbw}). 
%Here we focus on collecting intuitions; omitted details are in the Appendix and in~\cite{KouzapasPY15}.

\subsection{Reduction-Closed, Barbed Congruence ($\cong$)}
\label{subsec:rc}
%We first define \emph{confluence} over session environments $\Delta$:

We consider \emph{typed relations} $\Re$ that relate  closed terms whose
session environments %and the two session environments
are balanced  and confluent:

\begin{definition}[Session Environment Confluence]
Let $\red^\ast$ denote multi-step reduction as in \defref{def:ses_red}.
	We denote $\Delta_1 \bistyp \Delta_2$ if there exists $\Delta$ such that
	$\Delta_1 \red^\ast \Delta$ and $\Delta_2 \red^\ast \Delta$.
\end{definition}

%\smallskip 
%\noi Reduction-closed, barbed congruence is defined over typed
%processes:


\begin{definition}[Typed Relation]
	We say that
	$\Gamma; \emptyset; \Delta_1 \proves P_1 \hastype \Proc\ \Re \ \Gamma; \emptyset; \Delta_2 \proves P_2 \hastype \Proc$
	is a {\em typed relation} whenever
	$P_1$ and $P_2$ are closed;
	$\Delta_1$ and $\Delta_2$ are balanced; and 
	$\Delta_1 \bistyp \Delta_2$.
	We write $\horel{\Gamma}{\Delta_1}{P_1}{\ \Re \ }{\Delta_2}{P_2}$
	for the typed relation $\Gamma; \emptyset; \Delta_1 \proves P_1 \hastype \Proc\ \Re \ \Gamma; \emptyset; \Delta_2 \proves P_2 \hastype \Proc$.
\end{definition}

%\smallskip 

%\noi Next we define  {\em barbs}with respect to types. 
%\noi 
As usual, a \emph{barb} $\barb{n}$ is an observable on an output prefix with subject $n$~\cite{MiSa92}.
A \emph{weak barb} $\Barb{n}$ is a barb after zero or more reduction steps.
Typed barbs $\barb{n}$ (resp.\ $\Barb{n}$)
occur on typed processes $\Gamma; \emptyset; \Delta \proves P \hastype \Proc$.
When $n$ is a session name we require that its dual endpoint $\dual{n}$ is not present
in the session environment $\Delta$:
%\smallskip 

\begin{definition}[Barbs]%\rm
	Let $P$ be a closed process. We define:
	\begin{enumerate}[1.]
		\item	
		$P \barb{n}$ if $P \scong \newsp{\tilde{m}}{\bout{n}{V} P_2 \Par P_3}, n \notin \tilde{m}$. %; $P \Barb{n}$ if $P \red^* \barb{n}$.

		\item	$\Gamma; \Delta \proves P \barb{n}$ if
			$\Gamma; \emptyset; \Delta \proves P \hastype \Proc$ with $P \barb{n}$ and $\dual{n} \notin \dom{\Delta}$.

			$\Gamma; \Delta \proves P \Barb{n}$ if $P \red^* P'$ and
			$\Gamma; \Delta' \proves P' \barb{n}$.			
	\end{enumerate}
\end{definition}

%\smallskip 

%\noi 

To define a congruence relation, we introduce the family $\C$ of contexts:

\begin{definition}[Context]
	A context $\C$ is defined as:

	\begin{tabular}{rcl}
		$\C$ & $\bnfis$ & $\hole \bnfbar \bout{u}{V} \C \bnfbar \binp{u}{x} \C \bnfbar \bout{u}{\lambda x.\C} P \bnfbar \news{n} \C
		\bnfbar (\lambda x.\C)u \bnfbar \recp{X}{\C}$ 
		\\
		&$\bnfbar$& $\C \Par P \bnfbar P \Par \C
		\bnfbar \bsel{u}{l} \C \bnfbar \bbra{u}{l_1:P_1,\cdots,l_i:\C,\cdots,l_n:P_n}$
	\end{tabular}
%\smallskip 

%\noi 
Notation $\context{\C}{P}$ replaces 
the hole $\hole$ in $\C$ with $P$.
\end{definition}

%\smallskip 

%\noi 
We define reduction-closed, barbed congruence \cite{HondaKYoshida95}. 

%\smallskip 

\begin{definition}[Barbed Congruence]
\label{def:rc}
	Typed relation
	$\horel{\Gamma}{\Delta_1}{P}{\ \Re\ }{\Delta_2}{Q}$
	is a {\em reduction-closed, barbed congruence} whenever:
	\begin{enumerate}[1.]
		\item	If $P \red P'$ then there exist $Q', \Delta_1'$,  $\Delta_2'$ such that $Q \red^* Q'$ and
			$\horel{\Gamma}{\Delta_1'}{P'}{\ \Re\ }{\Delta_2'}{Q'}$; 
			%and its symmetric case;
%		\item	If $Q \red P_2'$ then $\exists P_1', P_1 \red^* P_1'$ and
%		$\horel{\Gamma}{\Delta_1'}{P_1'}{\ \Re\ }{\Delta_2'}{P_2'}$
%		\end{itemize}

%		\item
%		\begin{itemize}
			\item	If $\Gamma;\Delta_1 \proves P \barb{n}$ then $\Gamma;\Delta_2 \proves Q \Barb{n}$; %and its symmetric case; 

%			\item	If $\Gamma;\emptyset;\Delta \proves P_2 \barb{s}$ then $\Gamma;\emptyset;\Delta \proves P_1 \Barb{s}$.
%		\end{itemize}

		\item	For all $\C$, there exist $\Delta_1'',\Delta_2''$ such that  $\horel{\Gamma}{\Delta_1''}{\context{\C}{P}}{\ \Re\ }{\Delta_2''}{\context{\C}{Q}}$;
		                      \item	The symmetric cases of 1 and 2.                
	\end{enumerate}
	The largest such relation is denoted with $\cong$.
\end{definition}

{
\subsection{Two Equivalence Relations: $\fwb$ and $\hwb$}\label{ss:equiv}

\paragraph{A Typed Labelled Transition System.}
In~\cite{characteristic_bis} we have characterised
reduction-closed, barbed congruence for \HOp
via a typed relation called
{\em characteristic bisimilarity}.
%The definition of characteristic bisimilarity 
Its definition 
uses a \emph{typed}
labelled transition system (LTS) informed by session
types. 
Transitions in this LTS are denoted 
$\Gamma; \es; \Delta \proves P \hby{\ell} \Delta' \proves P'$.
(Weak transitions, defined as expected, are denoted 
$\Gamma; \es; \Delta \proves P \Hby{\ell} \Delta' \proves P'$.)
The main intuition %for its definition  
is that the transitions 
of a typed process should be enabled by its associated typing environment:
%
%\[
%	\tree {
%		P \hby{\ell} P' \qquad (\Gamma, \Delta) \hby{\ell} (\Gamma, \Delta')
%	}{
%		\Gamma; \es; \Delta \proves P \hby{\ell} \Delta' \proves P' \hastype \Delta'
%	}
%\]
$$
\text{if } P \hby{\ell} P' \text{ and } (\Gamma, \Delta) \hby{\ell} (\Gamma, \Delta') \text{ then }
\Gamma; \es; \Delta \proves P \hby{\ell} \Delta' \proves P'.
$$
%
%\noi 
As an example of how types enable transitions, consider the rule for input:
%
\[
	\tree{
		\dual{s} \notin \dom{\Delta} 
		\quad~~ 
		\Gamma; \Lambda'; \Delta' \proves V \hastype U
		\quad~~
		V = m \vee  V \scong \omapchar{U} \vee V \scong \abs{{x}}{\binp{t}{y} (\appl{y}{{x}})}
					\textrm{ with } t \textrm{ fresh} 
	}{
		(\Gamma; \Lambda; \Delta \cat s: \btinp{U} S) \hby{\bactinp{s}{V}} (\Gamma; \Lambda\cat\Lambda'; \Delta\cat\Delta' \cat s: S)
	}
\]
%\noi
This rule states that a session channel environment can input a value
if the channel is typed with an input prefix and the input value is either
a name $m$, a \emph{characteristic value} $\omapchar{U}$,  or a \emph{trigger value} (the abstraction
$\abs{{x}}{\binp{t}{y} (\appl{y}{{x}})}$). 
A characteristic value
is the {simplest} process that  inhabits a type (here, the
type $U$ carried by the input prefix). The above rule is used to limit
the input actions that can be observed from a session input prefix.
For details of the labelled transition system and the characteristic process definition
see% \appref{app:behavioural} and
~\cite{characteristic_bis}.
Moreover, we define a \emph{(first-order) trigger process}:
%
\begin{eqnarray}
%	\htrigger{t}{V}  & \defeq &  \hotrigger{t}{V} \label{eqb:0} \\
	\ftrigger{t}{V}{U} & \defeq &  \fotrigger{t}{x}{s}{\btinp{U} \tinact}{V} 	\label{eq:fot}
\end{eqnarray}
%
%\noi
The trigger process $\ftrigger{t}{V}{U}$ is 
is defined as a process input prefixed on
a fresh name $t$: it
%The   trigger process $\htrigger{t}{V}$ above applies a value $V$
%on the receiving (characteristic) process.
applies a value
on the %(hard coded) 
\emph{characteristic process} $\map{\btinp{U} \tinact}^{s}$ (see~\cite{characteristic_bis} for details). 

\paragraph{Characterisations of $\cong$.}
We now define \emph{characteristic} and \emph{higher-order} bisimilarities.
Observe that higher-order bisimilarity is a new typed equality, 
while 
characteristic bisimilarity was introduced in~\cite{characteristic_bis} (Def.~14).

\begin{definition}[Characteristic Bisimilarity]%\rm
	\label{d:fwb}
	A typed relation $\Re$ is called a {\em  characteristic bisimulation} if 
	for all $\horel{\Gamma}{\Delta_1}{P_1}{\ \Re \ }{\Delta_2}{Q_1}$ 
	\begin{enumerate}[1.]
		\item 
				Whenever 
				$\horel{\Gamma}{\Delta_1}{P_1}{\hby{\news{\tilde{m_1}} \bactout{n}{V_1: U}}}{\Delta_1'}{P_2}$,
				there exist 
				$Q_2$, $V_2$, $\Delta'_2$
				such that \\
				$\horel{\Gamma}{\Delta_2}{Q_1}{\Hby{\news{\tilde{m_2}} \bactout{n}{V_2: U}}}{\Delta_2'}{Q_2}$ and, for fresh $t$, 
%
				\[
					\begin{array}{lrlll}
						\Gamma; \Delta''_1  \proves  {\newsp{\tilde{m_1}}{P_2 \Par \ftrigger{t}{V_1}{U_1}}}
						\ \Re \
						\Delta''_2 \proves {\newsp{\tilde{m_2}}{Q_2 \Par \ftrigger{t}{V_2}{U_2}}}
					\end{array}
				\]

		\item	
				For all $\horel{\Gamma}{\Delta_1}{P_1}{\hby{\ell}}{\Delta_1'}{P_2}$ such that 
				$\ell$ is not an output, 
				there exist $Q_2$, $\Delta'_2$ such that 
				$\horel{\Gamma}{\Delta_2}{Q_1}{\Hby{\hat{\ell}}}{\Delta_2'}{Q_2}$
				and
				$\horel{\Gamma}{\Delta_1'}{P_2}{\ \Re \ }{\Delta_2'}{Q_2}$; and 

		\item	The symmetric cases of 1 and 2.                
	\end{enumerate}
	The largest such bisimulation
	is called \emph{characteristic bisimilarity} {and} denoted by $\fwb$.
\end{definition}
 

%\noi 
%We define characteristic bisimilarity
%(cf.~Def.~14~in~\cite{characteristic_bis}):
% is given using characteristic trigger processes. 

\newj{Interestingly, for reasoning about \HOp processes %not in \sessp 
we can also exploit the simpler \emph{higher-order bisimilarity}.}
We replace triggers as in \eqref{eq:fot}
with \emph{higher-order triggers}:
{\begin{eqnarray}
	\htrigger{t}{V}  & \defeq &  \hotrigger{t}{V} \label{eq:hot} 
%	\ftrigger{t}{V}{U} & \defeq &  \fotrigger{t}{x}{s}{\btinp{U} \tinact}{V} 	\label{eq:fot}
\end{eqnarray}}
We may then define:
\begin{definition}[Higher-Order Bisimilarity]%\rm
	\label{d:hbw}
	Higher-order bisimilarity, denoted by $\hwb$, is defined {by} replacing 
	Clause (1) in \defref{d:fwb} with the following clause:\\[1mm]
	%\begin{enumerate}[1.]
	%	\item 
		Whenever 
	$\horel{\Gamma}{\Delta_1}{P_1}{\hby{\news{\tilde{m_1}} \bactout{n}{V_1}}}{\Delta_1'}{P_2}$ %with $\Gamma; \es; \Delta \proves V_1 \hastype U$,  
	then there exist 
	$Q_2$, $V_2$, $\Delta'_2$
	such that \\
	$\horel{\Gamma}{\Delta_2}{Q_1}{\Hby{\news{\tilde{m_2}}\bactout{n}{V_2}}}{\Delta_2'}{Q_2}$ %with $\Gamma; \es; \Delta' \proves V_2 \hastype U$,  
	and, for fresh $t$, \\[1mm]
	$
	\begin{array}{lrlll}
		\!\!\Gamma; \Delta''_1  \proves  {\newsp{\tilde{m_1}}{P_2 \Par 
		\htrigger{t}{V_1}}}
		\ \Re\ \Delta''_2
		\proves {\newsp{\tilde{m_2}}{Q_2 \Par \htrigger{t}{V_2}}}
	\end{array}
	$
	%\end{enumerate}
\end{definition}

We state the following important result, which attests the significance of $\hwb$:
\begin{theorem}\label{t:coincide}
	Typed relations $\cong$, $\hwb$, and $\fwb$ coincide for \HOp processes.
\end{theorem}
\begin{proof}
Coincidence of $\cong$ and $\fwb$ was established in~\cite{characteristic_bis}.
Coincidence of $\hwb$ with $\cong$ and $\fwb$ is a new result: see \cite{KouzapasPY15}
for details. \qed
\end{proof}

\begin{remark}[Comparison between $\hwb$ and $\fwb$]
The key difference between $\hwb$ and $\fwb$ is in the trigger process considered. 
Because of the application in \eqref{eq:hot}, $\hwb$ 
%is only defined for processes in \HO; it 
cannot be used to reason about processes in \sessp. In contrast, $\fwb$ is more general:
it can uniformly input characteristic, first- or higher-order values.
This convenience comes at a price: the definition of \eqref{eq:fot} requires information on the 
type of $V$; in contrast, the higher-order trigger \eqref{eq:hot} is more generic and simple,
as it works independently of the given type.
%
%	The difference between higher order bisimilarity
%	lies in the definition of the process trigger.
%	\begin{enumerate}[(a)]
%		\item
%				The higher-order bisimilarity \eqref{eq:hot}
%%				\[
%%					\htrigger{t}{V}  \defeq  \hotrigger{t}{V}
%%				\]
%				allows for the application of bound variable $x$,
%				making $\hwb$ not defined in the \sessp sub-calculus
%				of \HOp, but defined inside the \HO sub-calculus. 
%
%		\item	The Characteristic bisimilarity trigger \eqref{eq:fot} 
%%				\[
%%					\ftrigger{t}{V}{U} \defeq  \fotrigger{t}{x}{s}{\btinp{U} \tinact}{V}
%%				\]
%				solves the problem in (a) by not using $x$ in the continuation
%				of fresh name input. This allows to input characteristic, first- or
%				higher-order value on bound variable $x$.
%
%
%		\item	Higher-order trigger lifts the need of observing
%				types on the lts and including them in bisimulation
%				closures. Higher-order bisimilarity adds to
%				the simplicity and elegance of the \HO subcalculus
%				over the \sessp subcalculus of \HOp.
%	\end{enumerate}
\end{remark}

%\begin{definition}[Characteristic Bisimilarity]\rm
%	\label{d:fwb}
%
%	A typed relation $\Re$ is a {\em characteristic bisimulation} if 
%	for all $\horel{\Gamma}{\Delta_1}{P_1}{\ \Re \ }{\Delta_2}{Q_1}$ 
%	if whenever:
%	\begin{enumerate}[1)]
%		\item	$\horel{\Gamma}{\Delta_1}{P_1}{\hby{\news{\tilde{m_1}} \bactout{n}{V_1: U}}}{\Delta_1'}{P_2}$
%				then there exist  $Q_2$, $V_2$, $\Delta'_2$ such that 
%				$\horel{\Gamma}{\Delta_2}{Q_1}{\Hby{\news{\tilde{m_2}}\bactout{n}{V_2: U}}}{\Delta_2'}{Q_2}$
%				and, for fresh $t$,
%				$
%				\begin{array}{lrlll}
%					\Gamma; \Delta''_1 \proves {\newsp{\tilde{m_1}}{P_2 \Par  \ftrigger{t}{V_1}{U_1}}}
%					\ \Re\ \Delta''_2 \proves {\newsp{\tilde{m_2}}{Q_2 \Par \ftrigger{t}{V_2}{U_2}}}
%				\end{array}
%				$
%
%			\item	For all $\horel{\Gamma}{\Delta_1}{P_1}{\hby{\ell}}{\Delta_1'}{P_2}$ such that 
%					$\ell$ is not an output, there exist $Q_2$, $\Delta'_2$ such that 
%					$\horel{\Gamma}{\Delta_2}{Q_1}{\Hby{\hat{\ell}}}{\Delta_2'}{Q_2}$
%					and $\horel{\Gamma}{\Delta_1'}{P_2}{\ \Re \ }{\Delta_2'}{Q_2}$; and 
%
%			\item	The symmetric cases of 1 and 2.
%	\end{enumerate}
%	The largest such bisimulation is called \emph{characteristic bisimilarity} and denoted by $\fwb$.
%\end{definition}


\paragraph{An up-to technique.}
In our setting, processes that do not use shared names are deterministic. 
The following up-to technique, based on determinacy properties, will be useful in proofs (\secref{sec:positive}).
Recall that $\horel{\Gamma}{\Delta}{P}{\hby{\tau}}{\Delta'}{P'}$ denotes an internal (typed) transition.
 
 \begin{notation}[Deterministic Transitions]
 \label{not:dettrans}
We distinguish two kinds of  internal transitions:
\emph{session transitions}, denoted 
$\horel{\Gamma}{\Delta}{P}{\hby{\stau}}{\Delta'}{P'}$,
and 
\emph{$\beta$-transitions}, denoted $\horel{\Gamma}{\Delta}{P}{\hby{\btau}}{\Delta'}{P'}$.
Intuitively, $\hby{\stau}$  results from a session communication (i.e., synchronization between
two dual endpoints); 
  $\hby{\btau}$ results from an application. 
 We write  $\horel{\Gamma}{\Delta}{P}{\hby{\dtau}}{\Delta'}{P'}$ to denote
	either a session transition or a $\beta$-transition.
	Formal definitions for $\hby{\btau}$  and $\hby{\stau}$ rely on an LTS for \HOp; see~\cite{KouzapasPY15} for details.
 \end{notation}
 
%The auxiliary definition below allows us to distinguish two kinds of  internal transitions:
%\emph{session transitions} and \emph{$\beta$-transitions} (denoted 
%$\horel{\Gamma}{\Delta}{P}{\hby{\stau}}{\Delta'}{P'}$
%and $\horel{\Gamma}{\Delta}{P}{\hby{\btau}}{\Delta'}{P'}$, respectively).
%
%\begin{definition}[Deterministic Transition]
%\label{def:dettrans}
%	Let  $\Gamma; \es; \Delta \proves P \hastype \Proc$ be a balanced \HOp process. 
%	Transition $\horel{\Gamma}{\Delta}{P}{\hby{\tau}}{\Delta'}{P'}$ is called:
%	\begin{enumerate}[$-$]
%		\item	{\em Session transition}
%				whenever the untyped transition $P \by{\tau} P'$ 
%				is derived using  rule~$\ltsrule{Tau}$ 
%				(where $\subj{\ell_1}$ and $\subj{\ell_2}$ in the premise are dual endpoints), 
%				possibly followed by uses of
%				$\ltsrule{Alpha}$, $\ltsrule{Res}$, $\ltsrule{Rec}$, or $\ltsrule{Par${}_L$}/
%				\ltsrule{Par${}_R$}$.
%		
%		\item	{\em $\beta$-transition}
%				whenever the untyped transition $P \by{\tau} P'$
%				is derived using rule $\ltsrule{App}$,
%				possibly followed by uses of  $\ltsrule{Alpha}$, $\ltsrule{Res}$, $\ltsrule{Rec}$, or $\ltsrule{Par${}_L$}/
%				\ltsrule{Par${}_R$}$.
%	\end{enumerate}
%%
%	We write
%	$\horel{\Gamma}{\Delta}{P}{\hby{\stau}}{\Delta'}{P'}$
%	and 
%	$\horel{\Gamma}{\Delta}{P}{\hby{\btau}}{\Delta'}{P'}$
%	to denote session and $\beta$-transitions, resp. Also, 
%	 $\horel{\Gamma}{\Delta}{P}{\hby{\dtau}}{\Delta'}{P'}$ denotes
%	either a session transition or a $\beta$-transition.
%\end{definition}
%
%A transition $\horel{\Gamma}{\Delta}{P}{\hby{\tau}}{\Delta'}{P'}$ is said
%{\em deterministic} if it is derived using~$\ltsrule{App}$ or~$\ltsrule{Tau}$ 
%(where $\subj{\ell_1}$ and $\subj{\ell_2}$ in the premise  are dual endpoints), 
%possibly followed by uses of  $\ltsrule{Alpha}$, $\ltsrule{Res}$, $\ltsrule{Rec}$, or $\ltsrule{Par${}_L$}/\ltsrule{Par${}_R$}$.

We have the following determinacy property;
%see  \appref{app:sub_tau_inert} 
see~\cite{KouzapasPY15} for details. 


\begin{lemma}[$\tau$-Inertness]%\rm
	\label{lem:tau_inert}
	(1)
%	\begin{enumerate}[1)]
%		\item
				Let $\horel{\Gamma}{\Delta}{P}{\hby{\dtau}}{\Delta'}{P'}$ be a deterministic transition,
				with balanced $\Delta$. Then 
				$\Gamma; \Delta \proves P \cong \Delta'\proves P'$ 
				with $\Delta \red^\ast \Delta'$ balanced.
%		\item 
				(2) Let $P$ be an $\HOp^{-\mathsf{sh}}$ process. 
				Assume $\Gamma; \emptyset; \Delta \proves P \hastype \Proc$. Then 
				$P \red^\ast P'$ implies $\Gamma; \Delta \proves 
				P \cong \Delta'\proves P'$ with $\Delta \red^\ast \Delta'$. 
%	\end{enumerate}
\end{lemma}


%\begin{proof}
%	The proof uses the fact that processes of the
%	form $\Gamma; \es; \Delta \proves_s \bout{s}{V} P_1 \Par \binp{k}{x} P_2$
%	cannot have any typed transition observables and the fact
%	that bisimulation is a congruence.
%	See  \appref{app:sub_tau_inert} for details.
%	The proof for Part 2 follows from Part 1.
%	\qed
%\end{proof}

\newj{We use 
\lemref{lem:tau_inert}
to prove \thmref{t:negative}, the negative result stated in
\secref{ss:negative}.
This property also enables us to define the following up-to technique, which is used in full abstraction proofs.
We write $\Hby{\dtau}$ to denote a (possibly empty) sequence of deterministic steps 
$\hby{\dtau}$.}


\begin{lemma}[Up-to Deterministic Transition]%\myrm
	\label{lem:up_to_deterministic_transition}
	Let $\horel{\Gamma}{\Delta_1}{P_1}{\ \Re\ }{\Delta_2}{Q_1}$ such
	that if whenever:
%
	\begin{enumerate}[1.]
		\item	$\forall \news{\tilde{m_1}} \bactout{n}{V_1}$ such that
			$
				\horel{\Gamma}{\Delta_1}{P_1}{\hby{\news{\tilde{m_1}} \bactout{n}{V_1}}}{\Delta_3}{P_3}
			$
			implies that $\exists Q_2, V_2$ such that
			$
				\horel{\Gamma}{\Delta_2}{Q_1}{\Hby{\news{\tilde{m_2}} \bactout{n}{V_2}}}{\Delta_2'}{Q_2}
			$
			and
			$
				\horel{\Gamma}{\Delta_3}{P_3}{\Hby{\dtau}}{\Delta_1'}{P_2}
			$
			and for fresh $t$:\\
			$
				\horel{\Gamma}{\Delta_1''}{\newsp{\tilde{m_1}}{P_2 \Par \htrigger{t}{V_1}}}
				{\ \Re\ }
				{\Delta_2''}{}{\newsp{\tilde{m_2}}{Q_2 \Par \htrigger{t}{V_2}}}
%				\mhorel{\Gamma}{\Delta_1''}{\newsp{\tilde{m_1}}{P_2 \Par \hotrigger{t}{x}{s}{V_1}}}
%				{\ \Re\ }
%				{\Delta_2''}{}{\newsp{\tilde{m_2}}{Q_2 \Par \hotrigger{t}{x}{s}{V_2}}}
			$.
%
		\item	$\forall \ell \not= \news{\tilde{m}} \bactout{n}{V}$ such that
			$
				\horel{\Gamma}{\Delta_1}{P_1}{\hby{\ell}}{\Delta_3}{P_3}
			$
			implies that $\exists Q_2$  \\ such that 
			$
				\horel{\Gamma}{\Delta_1}{Q_1}{\hat{\Hby{\ell}}}{\Delta_2'}{Q_2}
			$
			and
			$
				\horel{\Gamma}{\Delta_3}{P_3}{\Hby{\dtau}}{\Delta_1'}{P_2}
			$
			and
			$\horel{\Gamma}{\Delta_1'}{P_2}{\ \Re\ }{\Delta_2'}{Q_2}$.

		\item	The symmetric cases of 1 and 2.
	\end{enumerate}
	Then $\Re\ \subseteq\ \hwb$.
\end{lemma}


%\begin{proof}
%	The proof is easy by considering the closure
%	\[
%		\Re^{\Hby{\dtau}} = \set{ \horel{\Gamma}{\Delta_1'}{P_2}{,}{\Delta_2'}{Q_1} \setbar \horel{\Gamma}{\Delta_1}{P_1}{\ \Re\ }{\Delta_2'}{Q_1},
%		\horel{\Gamma}{\Delta_1}{P_1}{\Hby{\dtau}}{\Delta_1'}{P_2} }
%	\]
%	We verify that $\Re^{\Hby{\dtau}}$ is a bisimulation with
%	the use of \propref{lem:tau_inert}.
%	\qed
%\end{proof}
%
%\begin{example}[Up-to Deterministic Transition]
%	Typed processes:
%	\begin{eqnarray*}
%		\Gamma; \es; \Delta, s': \tinact \proves P &=& \binp{n}{z_1} \newsp{s}{\binp{s}{x} \appl{(\abs{y}{\bout{n}{z_1} \inact})}{m} \Par \bout{\dual{s}}{s'} \inact} \hastype \Proc
%		\\
%		\Gamma; \es; \Delta \proves Q &=& \binp{n}{z_1} \binp{n}{z_2} \inact \hastype \Proc
%	\end{eqnarray*}
%	are bisimilar up-to deterministic transition because
%	we can observe:
%	\begin{eqnarray*}
%		\Gamma; \Delta, s': \tinact \proves P &\hby{\bactinp{n}{m_1}}& \Delta', s: \tinact \proves \newsp{s}{\binp{s}{x} \appl{(\abs{y}{\bout{n}{z_2} \inact})}{m} \Par \bout{\dual{s}}{s'} \inact} \Hby{\dtau} \Delta' \proves \binp{n}{z_2} \inact
%		\\
%		\text{and}
%		\\
%		\Gamma; \es; \Delta \proves Q &\hby{\bactinp{n}{m_1}}& \Delta' \proves \binp{n}{z_2} \inact
%	\end{eqnarray*}


%	Relation 
%	\[
%		\Re = \set{(\Gamma; \Delta, s': \tinact \proves P , \Delta \proves Q), (\Gamma; \Delta' \proves \binp{n}{z_2}, \Delta' \proves  \binp{n}{z_2})}
%	\]
%	is bisimulation up-to deterministic transition because
%	\begin{eqnarray*}
%		\Gamma; \Delta, s': \tinact \proves P &\hby{\bactinp{n}{s_1}}& \Delta', s: \tinact \proves \newsp{s}{\binp{s}{x} \appl{(\abs{y}{\bout{n}{y} \inact})}{s_1} \Par \bout{\dual{s}}{s'} \inact}
%		\\
%		\text{implies}&
%		\\
%		\Gamma; \es; \Delta \proves Q &\hby{\bactinp{n}{x}}& \Delta' \proves \binp{n}{z_2} \inact
%		\\
%		\text{and}&
%		\\
%		\Delta', s: \tinact \proves \newsp{s}{\binp{s}{x} \appl{(\abs{y}{\bout{n}{y} \inact})}{s_1} \Par \bout{\dual{s}}{s'} \inact}  \in \Re
%	\end{eqnarray*}
%\end{example}

%\noi Precise encodings offer more detailed criteria and used for positive 
%encodability results (\secref{sec:positive}).
%In contrast, minimal encodings contains only 
%some of the criteria of precise encodings:    
%this reduced notion will be used 
%for the negative result in \secref{sec:negative}.




%%%%%%%%%%%%%%%%%%%%%%%%%%%%%%%%%%%%%%%%%%%%%%%%%%%%%%%%%%%%%%%%%%%%%%%%%%%%%%%%
%%%%%%%%%%%%%%%%%%%%%%%%%%%%%%%%%%%%%%%%%%%%%%%%%%%%%%%%%%%%%%%%%%%%%%%%%%%%%%%%


\section{Criteria for Typed Encodings}
\label{s:expr}
%% !TEX root = main.tex

\newpage
\section{Typed Encodings}\label{s:expr}

In this section we present a study of the expressiveness 
of the sub-calculi of $\HOp$.

We first define the notion of calculus.
We extend notions proposed elsewhere (cf.~Gorla\cite{})
by explicitly considering a type structure and a type system.

\begin{definition}[Typed Calculus]\label{d:tcalculus}\rm
	A \emph{typed calculus} $\tyl{L}$ is defined as a tuple:
%
	\[
		\calc{L}{T}{\red}{\wb}{\proves}
	\]
%
	where $L$ and $T$ are sets of processes and types, respectively; %$T_1$ is the set of types;
	$\red$ and $\wb$ denote a reduction semantics 
	and a typed equivalence
	on processes, respectively. Finally, $\proves$ denotes a type system for processes in $L$.
\end{definition}

We notice that in this paper we shall always consider languages with the same type system.
In the following, when writing $\tyl{L}_i$ we tacitly assume the existence of appropriate 
$L_i$, $T_i$, $\red_i$, $\wb_i$, and $\proves_i$.
We first define the notion of encoding over typed calculi.

\begin{definition}[Typed Encoding]\rm
	Let  $\tyl{L}_1$ % = \calc{L_1}{T_1}{\red_1}{\wb_1}{\proves_1}$
	and $\tyl{L}_2$ % =  \calc{L_2}{T_2}{\red_2}{\wb_2}{\proves_2}$ 
	be typed calculi.% as in Definition~\ref{d:tcalculus}.
	Given mappings $\map{\cdot}: L_1 \to L_2$ and
	$\mapt{\cdot}: T_1 \to T_2$, 
	we write 
	%$\enc{\cdot}{\cdot}: \calc{L_1}{T_1}{\red_1}{\wb_1}{\proves_1} \longrightarrow \calc{L_2}{T_2}{\red_2}{\wb_2}{\proves_2}$
%	for the encoding from $\calc{L_1}{T_1}{\red_1}{\wb_1}{\proves_1}$ to $\calc{L_2}{T_2}{\red_2}{\wb_2}{\proves_2}$.
	\[
		\enc{\cdot}{\cdot} : \tyl{L}_1 \to \tyl{L}_2
	\]
	to denote the \emph{typed encoding} of $\tyl{L}_1$ into $\tyl{L}_2$.
\end{definition}

\subsection{Encoding Properties}

We require that a {\em good} encoding should 
preserve not only the syntax but
also the operational, typing and behavioural
semantics. 

% ----> DK: The next notation is already defined
%\begin{notation}[Typed Equivalence]\rm
%	Let $P$ and $Q$ be two well-typed processes, i.e., 
%	there exist $\Gamma, \Sigma_1, \Sigma_2$ such that 
%	$\Gamma; \emptyset; \Sigma_1 \proves P \hastype \Proc$ 
%	and
%	$\Gamma; \emptyset; \Sigma_2 \proves Q \hastype \Proc$.
%	Then, to denote the fact that 
%	$P$ and $Q$ are related by behavioral equivalence $\wb$, we shall write
%	%Then we adopt the following notational convention:
%	\[
%		\Gamma; \Sigma_1 \wb \Sigma_2 \proves P \wb Q.
%	\]
%\end{notation}

\begin{definition}[Semantic Preserving Encoding]\rm
	\label{def:ep}
	We say that $\enc{\cdot}{\cdot} : \tyl{L}_1 \to \tyl{L}_2$ is a \emph{semantic preserving encoding}
	if it satisfies the following properties:
	%Let $\Gamma; \emptyset; \Sigma \proves P \hastype \Proc$ 
	%a process from calculus $\calc{L_1}{T_1}{\red_1}{\wb_1}{\proves_1}$
	%and an encoding 
	%$\enc{\cdot}{\cdot}: \calc{L_1}{T_1}{\red_1}{\wb_1}{\proves_1} \longrightarrow \calc{L_2}{T_2}{\red_2}{\wb_2}{\proves_2}$.
	
	\begin{enumerate}[1.]
		\item \emph{Type preservation}:	%We say that $\enc{\cdot}{\cdot}$ is \emph{type preserving}
		if
			$\Gamma; \emptyset; \Sigma \proves_1 P \hastype \Proc$ then $\mapt{\Gamma}; \emptyset; \mapt{\Sigma} \proves_2 \map{P} \hastype \Proc$ for any   $P$ in $L_1$.

		\item \emph{Operational Correspondence}: If $\Gamma; \emptyset; \Sigma \proves_1 P \hastype \Proc$ then
		\begin{enumerate}[-]
			\item	Completeness: If $P \red_1 P'$ then $\exists \Sigma'$ s.t.
				$\map{P} \Red_2 \map{P'}$ and
				$\mapt{\Gamma}; \emptyset; \mapt{\Sigma'} \proves_2 \map{P'} \hastype \Proc$.
			\item Soundness : If $\map{P} \red_2 Q$ then
				$\exists P'$ s.t. $P \red_1 P'$ and \\
				$\mapt{\Gamma}; \mapt{\Sigma_1} \wb_2 \mapt{\Sigma_2} \proves_2 \map{P'} \wb_2 Q$.
		\end{enumerate}
		
		\item \emph{Full Abstraction:}
		$\Gamma; \Sigma_1 \wb_1 \Sigma_2 \proves_1 P \wb_1 Q $ if and only if $\mapt{\Gamma}; \mapt{\Sigma_1} \wb_2 \mapt{\Sigma_2} \proves_2 \map{P} \wb_2 \map{Q} $.
	\end{enumerate}
\end{definition}


We show that the composition of encodings is closed on the above properties.

\begin{proposition}[Composability of Semantic Preserving Encodings]
	Let $\encod{\cdot}{\cdot}{1}: \tyl{L}_1 \to \tyl{L}_2$ and $\encod{\cdot}{\cdot}{2}: \tyl{L}_2 \to \tyl{L}_3$
	be two semantic preserving encodings.
	Then their composition, denoted 
	$\encod{\cdot}{\cdot}{1} \cdot \encod{\cdot}{\cdot}{2}: \tyl{L}_1 \to \tyl{L}_3$
	is also a semantic preserving encoding.
\end{proposition}

\begin{proof}
	Straightforward application of the definition of each property.
\end{proof}

\section{Positive Expressiveness Results}

\subsection{Languages Under Consideration}
We consider the following variants of \HOp:
\begin{enumerate}[-]
	\item	\HO: the second and third lines of the syntax of processes in Fig.~\ref{fig:syntax} (pure higher-order, monadic communication).
	\item	\sesp: the first and third lines of the syntax of processes in Fig.~\ref{fig:syntax} (first-order, monadic communication).
	\item	\sespnr: the finite sub-calculus of \sesp, i.e., name passing without recursion.
	\item	$\HO^{+\mathsf{p}}$: The polyadic \HO, i.~e.\ without polyadicity (polyadic abstraction/application).
	\item	$\sesp^{+\mathsf{p}}$: The polyadic \sesp, i.~e.\ with polyadicity (name passing)
	\item	$\HOp^{-\mathsf{p}}$: The monadic \HOp.
%	\item \pHOpnr: the finite variant of \pHOp 
%	\item \psesp: the variant of \sesp with polyadic communication.
%	\item \psespnr: the finite variant of \psesp with polyadic communication.
\end{enumerate}
\noindent
In the following we write $\pmap{\cdot}{i}$
and $\tmap{\cdot}{i}$ 
for mappings of processes and types, respectively.
Since we always consider variants and fragments of \HOp, the 
reduction semantics $\red$, the typed behavioral equivalence $\wb$,
and the type system $\proves$ are the same for all languages.

\subsection{Encoding \sespnr  into \HO}

The semantics of the $\HO$ are powerful enough to
express the semantics of the standard $\sesp$ calculus.

The name passing semantics of $\sesp$ have a rather straightforward
encoding from to $\HO$.
On the other hand to achieve the encoding of the recursion semantic
of $\sesp$, we need to extend
to the polyadic version of $\sesp$ as an intermediate step in order
to give a sound encoding of the recursion semantics to $\HO$.

We first encode the name passing semantics.
%Below, we use $n$ to stand for either a linear channel $k'$ or a shared name $a$.

\begin{definition}[\sespnr  into \HO]\rm
	Define $\encod{\cdot}{\cdot}{1}: \sespnr \to \HO$  as follows:
	\[
	\begin{array}{rcl}
		\pmap{\bout{k}{k'} P}{1}		&\defeq&	\bbout{k}{ \abs{z}{\,\binp{z}{X} \appl{X}{k'}} } \pmap{P}{1} \\
		\pmap{\binp{k}{x} Q}{1}			&\defeq&	\binp{k}{X} \newsp{s}{\appl{X}{s} \Par \bbout{\dual{s}}{\abs{x} \pmap{Q}{1}} \inact} \\
		\tmap{\btout{S_1} {S} }{1}		&\defeq&	\bbtout{\lhot{\btinp{\lhot{\tmap{S_1}{1}}}\tinact}} \tmap{S}{1}  \\
		\tmap{\btinp{S_1} S }{1}		&\defeq&	\bbtinp{\lhot{\btinp{\lhot{\tmap{S_1}{1}}}\tinact}} \tmap{S}{1} \\
		\tmap{\bbtout{\chtype{S_1}}{S}}{1}	&\defeq&	\bbtout{\shot{\btinp{\shot{\chtype{\tmap{S_1}{1}}}}\tinact}} \tmap{S}{1}  \\
		\tmap{\bbtinp{\chtype{S_1}}{S}}{1}	&\defeq&	\bbtinp{\shot{\btinp{\shot{\chtype{\tmap{S_1}{1}}}}\tinact}} \tmap{S}{1} 
	\end{array}
	\]
	where $\pmap{\cdot}{1}$ (resp. $\tmap{\cdot}{1}$) is an 
	homomorphism for the other process (resp. type) constructs.
\end{definition}

In the higher-order setting, a name $k$ is being passed as an input
guarded abstraction. The input prefix receives an abstraction and
continues with the application of $k$ over the received abstraction.
On the reception side $\binp{s}{x} P$ 
the encoding develops a mechanism that will receive
the input guarded abstraction, apply it on a fresh endpoint $s$ and use
the dual endpoint $\dual{s}$ to send the continuation $P$ as the abstraction
$\abs{x}{P}$. Name substitution is then achieved as application.

\begin{proposition}\rm
	Encoding $\encod{\cdot}{\cdot}{1}: \sespnr \to \HO$  is type-preserving (cf. Def.~\ref{def:ep}\,(1)).\rm
\end{proposition}

\begin{proof}
	Proof in Appendix~\ref{app:enc_sesspnr_to_ho_typing}.
	\qed
\end{proof}

\begin{comment}
\begin{proof}
	By induction on the structure of \sesp process $P$.
%
	\begin{enumerate}[1.]
		%%%% Output of (linear) channel
		\item	Case $P = \bout{k}{n}P'$. There are two sub-cases.
			In the first sub-case $n = k'$ (output of a linear channel). Then  
			we have the following typing in the source language:
			{\small
			\[
				\tree{
					\Gamma; \emptyset; \Sigma \cat k:S  \proves  P' \hastype \Proc \quad \Gamma ; \emptyset ; \{k' : S_1\} \proves  k' \hastype S_1}{
					\Gamma; \emptyset; \Sigma \cat k':S_1 \cat k:\btout{S_1}S \proves  \bout{k}{k'} P' \hastype \Proc}
			\]
			}
			The corresponding typing in the target language is as follows --- we write $U_1$ to stand for $\lhot{\btinp{\lhot{\tmap{S_1}{1}}}\tinact}$:
			{\small
			\[
				\tree{
					\tree{}{\tmap{\Gamma}{1}; \emptyset ; \tmap{\Sigma}{1} \cat k:\tmap{S}{1} \proves \pmap{P'}{1} \hastype \Proc}
					~~
					\tree{
						\tree{
							\tree{
								\tree{
									\tree{}{\tmap{\Gamma}{1} ; \{X : \lhot{\tmap{S_1}{1}}\} ; \emptyset \proves \X  \hastype \lhot{\tmap{S_1}{1}}} 
									\quad 
									\tree{}{\tmap{\Gamma}{1} ; \emptyset ; \{k' : \tmap{S_1}{1}\} \proves  k' \hastype \tmap{S_1}{1}}}{\tmap{\Gamma}{1} ; \{X : \lhot{\tmap{S_1}{1}}\} ; k' : \tmap{S_1}{1} \proves \appl{\X}{k'} \hastype \Proc}}{\tmap{\Gamma}{1} ; \{X : \lhot{\tmap{S_1}{1}}\} ; k' : 	\tmap{S_1}{1} \cat z:\tinact \proves \appl{\X}{k'} \hastype \Proc}
						}{
							\tmap{\Gamma}{1} ; \emptyset; k' : \tmap{S_1}{1} \cat z:\btinp{\lhot{\tmap{S_1}{1}}}\tinact \proves \binp{z}{X} \appl{\X}{k'} \hastype \Proc
						}
					}{
						\tmap{\Gamma}{1} ; \emptyset; k' : \tmap{S_1}{1} \proves \abs{z}{\binp{z}{X} \appl{\X}{k'}} \hastype U_1
					}
				}{
				\tmap{\Gamma}{1}; \emptyset; \tmap{\Sigma}{1} \cat k':\tmap{S_1}{1} \cat k:\btout{U_1}\tmap{S}{1} \proves  \bbout{k}{\abs{z}{\binp{z}{X} \appl{\X}{k'}}} \pmap{P'}{1} \hastype \Proc
				}
			\]
			}
	
			In the second sub-case, we have $n = a$ (output of a shared name). Then  
			we have the following typing in the source language:
			{\small
			\[
				\tree{
					\Gamma \cat a:\chtype{S_1}; \emptyset; \Sigma \cat k:S  \proves  P' \hastype \Proc \quad \Gamma \cat a:\chtype{S_1} ; \emptyset ; \emptyset \proves  a \hastype S_1}{
					\Gamma \cat a:\chtype{S_1} ; \emptyset; \Sigma  \cat k:\bbtout{\chtype{S_1}}S \proves  \bout{k}{a} P' \hastype \Proc}
			\]
			}
			The typing in the target language is derived similarly as in the first sub-case. \\
	
		%%%% Input of (linear) channel 
		\item	Case $P = \binp{k}{x}Q$. We have two sub-cases, depending on the type of $x$.
			In the first case, $x$ stands for a linear channel.
			Then we have the following typing in the source language:
			{\small
			\[
			 \tree{
				 \Gamma; \emptyset; \Sigma  \cat k:S \cat x:S_1 \proves   Q \hastype \Proc
			 	}{
				\Gamma; \emptyset; \Sigma  \cat k:\btinp{S_1}S \proves  \binp{k}{x} Q \hastype \Proc}
			 \]
			 }
			 The corresponding typing in the target language is as follows --- we write $U_1$ to stand for $\lhot{\btinp{\lhot{\tmap{S_1}{1}}}\tinact}$:
			{\small  
			\[
			 \tree{
				 \tree{
				 	\tree{
					\tree{
					\begin{array}{c}
					\tmap{\Gamma}{1}; \{X: U_1\};   \emptyset \proves X \hastype U_1 \\
					\tmap{\Gamma}{1}; \emptyset;   \cat s: \btinp{\lhot{\tmap{S_1}{1}}}\tinact \ \proves s \, \hastype  \btinp{\lhot{\tmap{S_1}{1}}}
					\tinact 
					\end{array}
					}{
					\tmap{\Gamma}{1}; \{X: U_1\};   \cat s: \btinp{\lhot{\tmap{S_1}{1}}}\tinact \ \proves \appl{X}{s}  \hastype \Proc
					} \quad 
					\tree{
					\tree{
					\tmap{\Gamma}{1}; \emptyset;  \emptyset \proves   \inact  \hastype \Proc}{
					\tmap{\Gamma}{1}; \emptyset;  \dual{s}: \tinact\proves   \inact  \hastype \Proc
					}
					\quad 
					\tree{
					\tmap{\Gamma}{1}; \emptyset;  \tmap{\Sigma}{1} \cat k:\tmap{S}{1}  x:\tmap{S_1}{1} \proves \pmap{Q}{1}   \hastype \Proc	 }{
					\tmap{\Gamma}{1}; \emptyset;  \tmap{\Sigma}{1} \cat k:\tmap{S}{1}   \proves \abs{x} \pmap{Q}{1}   \hastype \lhot{\tmap{S_1}{1}}			}
					}{
					\tmap{\Gamma}{1}; \emptyset;  \tmap{\Sigma}{1} \cat k:\tmap{S}{1}  \cat \dual{s}: \btout{\lhot{\tmap{S_1}{1}}}\tinact\proves  \bbout{\dual{s}}{\abs{x} \pmap{Q}{1}} \inact  \hastype \Proc
					}
					}{
					\tmap{\Gamma}{1}; \{X: U_1\};  \tmap{\Sigma}{1} \cat k:\tmap{S}{1} \cat s: \btinp{\lhot{\tmap{S_1}{1}}}\tinact \cat \dual{s}: \btout{\lhot{\tmap{S_1}{1}}}\tinact\proves \appl{X}{s} \Par \bbout{\dual{s}}{\abs{x} \pmap{Q}{1}} \inact  \hastype \Proc
					}
					}{
				 \tmap{\Gamma}{1}; \{X: U_1\};  \tmap{\Sigma}{1} \cat k:\tmap{S}{1} \proves \newsp{s}{\appl{X}{s} \Par \bbout{\dual{s}}{\abs{x} \pmap{Q}{1}} \inact}  \hastype \Proc
				 }
				 }{
				\tmap{\Gamma}{1}; \emptyset; \tmap{\Sigma}{1}  \cat k:\btinp{U_1}\tmap{S}{1} \proves  \binp{k}{X} \newsp{s}{\appl{X}{s} \Par \bbout{\dual{s}}{\abs{x} \pmap{Q}{1}} \inact}  \hastype \Proc
				}
			 \]
			 }
			 
			 In the second sub-case, $x$ stands for a shared name. Then we have the following typing in the source language:
			{\small
			\[
			 \tree{
				 \Gamma \cat x:\chtype{S_1} ; \emptyset; \Sigma  \cat k:S \proves   Q \hastype \Proc
			 	}{
				\Gamma ; \emptyset; \Sigma  \cat k:\btinp{\chtype{S_1}}S \proves  \binp{k}{x} Q \hastype \Proc}
			 \]
			 }
			 The typing in the target language is derived similarly as in the first sub-case.	
\end{enumerate}
%
\qed
\end{proof}
\end{comment}

\begin{proposition}\rm
	Encoding $\encod{\cdot}{\cdot}{1}: \sespnr \to \HO$  enjoys operational correspondence (cf. Def.~\ref{def:ep}\,(2)).
\end{proposition}


\begin{proof}
	Proof in Appendix~\ref{app:enc_sesspnr_to_ho_oc}.
	\qed
\end{proof}

\begin{comment}
\begin{proof}[Sketch]
	We must show completeness and soundness properties. 
	For completeness, it suffices to consider source process
	$P_0 = \bout{k}{k'} P \Par \binp{k}{x} Q$. We have that
%
	\[
		P_0 \red P \Par Q\subst{k'}{x}.
	\]
%
	By the definition of encoding we have:
	\begin{eqnarray*}
		\pmap{P_0}{1} & = & \bbout{k}{ \abs{z}{\,\binp{z}{X} \appl{X}{k'}} } \pmap{P}{1} \Par \binp{k}{X} \newsp{s}{\appl{X}{s} \Par \bbout{\dual{s}}{\abs{x} \pmap{Q}{1}} \inact}  \\
		& \red & \pmap{P}{1} \Par \newsp{s}{\appl{X}{s} \subst{\abs{z}{\,\binp{z}{X} \appl{X}{k'}}}{X} \Par \bbout{\dual{s}}{\abs{x} \pmap{Q}{1}} \inact} \\
		& = & \pmap{P}{1} \Par \newsp{s}{\,\binp{s}{X} \appl{X}{k'} \Par \bbout{\dual{s}}{\abs{x} \pmap{Q}{1}} \inact} \\
		& \red & \pmap{P}{1} \Par \appl{X}{k'} \subst{\abs{x} \pmap{Q}{1}}{X} \Par \inact \\
		& \scong & \pmap{P}{1} \Par \pmap{Q}{1}\subst{k'}{x}  
	\end{eqnarray*}
	For soundness, it suffices to notice that the encoding does not add new visible actions:
	the additional synchronizations induced by the encoding always occur on private (fresh) names.
	We assume weak bisimilarities, which abstract from internal actions used by the encoding,
	and so  constructing a relation witnessing behavioral equivalence is easy.
	\qed
\end{proof}
\end{comment}

%\subsection{Polyadic Into Monadic}
%The encoding from $\psesp$ to $\sesp$ is easier than the
%encoding of polyadic $\pi$-calculus in the $\pi$-calculus because
%we have linear session endpoints.
%
%\begin{definition}[$\psesp$ to $\sesp$]
%	We write $\encod{\cdot}{\cdot}{2}:\psesp \to \sesp$ whenever
%
%	\begin{tabular}{c}
%			$\map{\bout{k}{k'_1, \cdots, k'_n} P}^{2} \defeq \bout{k}{k'_1} \cdots ;  \bout{k}{k'_n}
%			\pmap{P}{2}$\\
%			$\map{\binp{k}{x_1, \cdots, x_n} P}^{2} \defeq \binp{k}{x_1} \cdots ; \binp{k}{x_n}  \pmap{P}{2}$ \\
%			$\tmap{\btout{S_1, \cdots, S_n} S}{2} \defeq \bbtout{\tmap{S_1}{2}} \cdots; \bbtout{\tmap{S_n}{2}} \tmap{S}{2}$\\
%			$\tmap{\btinp{S_1, \cdots, S_n} S}{2} \defeq \bbtinp{\tmap{S_1}{2}} \cdots; \bbtinp{\tmap{S_n}{2}} \tmap{S}{2}$
%%		\end{tabular}
%%		& \quad &
%%		\begin{tabular}{l}
%%			$\tmap{\btout{S_1 \cat \tilde{S}} S}{2} \defeq \btout{S_1} \tmap{\btout{\tilde{S}} S}{2}$\\
%%			$\tmap{\btinp{S_1 \cat \tilde{S}} S}{2} \defeq \btinp{S_1} \tmap{\btinp{\tilde{S}} S}{2}$
%%		\end{tabular}
%	\end{tabular}
%\end{definition}
%
%Polyadic name sending (resp.\ receive) is encoded as sequence of
%send (resp.\ receive) operations. Linearity of session endpoints
%ensures no race conditions, thus the encoding is sound.
%
%The encoding of the polyadic $\sesp$ semantics is as simple as the
%composition of the two former encodings.
%
%\begin{definition}[Encoding from $\psespnr$ to $\HO$]
%	We define $\encod{\cdot}{\cdot}{3}: \psespnr \longrightarrow \HO$
%	as $\encod{\cdot}{\cdot}{3} = \encod{\cdot}{\cdot}{1} \cat \encod{\cdot}{\cdot}{2}$.	
%\end{definition}

%So far we have consider name abstractions and applications which are \emph{monadic}.
%We now consider the \emph{polyadic} extension of these constructs, %name abstractions and applications.
%written $\abs{x_1, \ldots, x_n} P$ and $\appl{X}{k_1, \ldots, k_n}$, respectively.
%Next we give the encoding from $\HOp$ with polyadic name abstraction to $\HOp^{p}$.
%
%\begin{definition}[Encoding from $\pHOpnr$ to $\pHOp$]
%
%	\begin{tabular}{lcl}
%		$\map{\bout{k}{\abs{\tilde{x}} P_1} P_2}^4$ &$\defeq$& $\bout{k}{\abs{z} \binp{z}{\tilde{x}} \map{P_1}^4} \map{P_2}^4$\\
%		$\map{\appl{X}{\tilde{k}}}$ &$\defeq$& $\newsp{s}{\appl{X}{s} \Par \bout{\dual{s}}{\tilde{k}} \inact}$
%	\end{tabular}
%\end{definition}

%We compose the latter encoding with the generalisation $\map{\cdot}^3 : \HOp^{p-\mu} \longrightarrow \HO$
%of the encoding $\map{\cdot}^3 : \sesp^{p-\mu} \longrightarrow \HO$ to get a translation
%of $\HOp^{pa-\mu}$ to $\HO$.
%
%\begin{definition}[Encoding from $\HOp^{pa-\mu}$ to $\HO$]
%	We define $\encod{\cdot}{\cdot}{5}: \HOp^{pa-\mu} \longrightarrow \HO$
%	as $\encod{\cdot}{\cdot}{5} = \encod{\cdot}{\cdot}{4} \cat \encod{\cdot}{\cdot}{3}$.	
%\end{definition}

\subsection{Encode Polyadic Semantics (\HOp) to Monadic Semantics ($\HOp^{-\mathsf{p}}$)}

%In the extension of \HOp with 
%polyadic communication, denoted \pHOp, 
%one may pass in each synchronization 
%a tuple of values of length $n$, rather than a just single value.
%Thus, e.g., for $n = 2$ one would have
%%
%\begin{eqnarray*}
%	\bout{n}{m_1, m_2} P \Par \binp{\dual{n}}{x_1,x_2} Q  & \red &  P \Par Q \subst{m_1, m_2}{x_1, x_2} \\
%	\bout{n}{\abs{x_1, x_2}{P_1}} P \Par \binp{\dual{s}}{\X} Q & \red & P \Par Q \subst{\abs{x_1,x_2}{P_1}}{\X}
%\end{eqnarray*}
%%
%with $\appl{X}{k_1,k_2} \subst{\abs{x_1,x_2}{Q}}{\X}  =  Q \subst{k_1,k_2}{x_1,x_2} $.
%Thus, 
%\pHOp features tuple passing in intra-session (linear) communication,
%but also in abstractions/applications. 
%The session type system for \pHOp is an orthogonal
%extension of that in \S\,\ref{s:types}.
%The type syntax for values is extended as follows,
%where $\tilde{S}$ stands for a sequence $S_1, \ldots, S_n$ of 
%session types:
%
%\begin{eqnarray*}
%	U \bnfis  & \tilde{S} \bnfbar \lhot{\tilde{S}} \bnfbar \shot{\tilde{S}} \bnfbar \chtype{S}
%\end{eqnarray*}
%
%The syntax of session types would be kept unchanged.
%Typing rules require straightforward extensions. 
%For instance, the following rules would type 
%abstraction and application in the biadic case ($n = 2$):
%\[
%\trule{Abs2}~~\tree{
%			\Gamma; \Lambda; \Sigma \cat x_1: S_1, x_2: S_2 \proves P \hastype \Proc
%		}{
%			\Gamma; \Lambda; \Sigma \proves \abs{x_1, x_2}{P} \hastype \lhot{(S_1, S_2)}
%		}
%		\quad
%		\trule{App2}~~\tree{
%		\begin{array}{c}
%		(U = \lhot{(S_1,S_2)}) \lor (U = \shot{(S_1,S_2)}) \\
%		\Gamma; \Lambda; \Sigma \proves X \hastype U  \\
%		\Gamma; \Lambda_1; \Sigma_1 \proves k_1 \hastype S_1 \quad 		
%		\Gamma; \Lambda_2; \Sigma_2 \proves k_2 \hastype S_2
%		\end{array}
%		}{
%			\Gamma; \Lambda \cup  \Lambda_1 \cup \Lambda_2; \Sigma \cup \Sigma_1 \cup \Sigma_2 \proves \appl{X}{k_1,k_2} \hastype \Proc
%		} 
%\]

In the untyped $\pi$-calculus, polyadic communication
can be encoded into monadic name passing 
simply by performing $n$ monadic synchronizations on a fresh channel. 
In session-typed $\pi$-calculi this encoding is even simpler, 
thanks to the linearity of session endpoints~\cite{VascoFun}.
%The extension of the (monadic) session type system given in \S\,\ref{s:types}
%to handle polyadic communication is straightforward and follow expected lines.
%For this reason, we do not present a typing system for \pHOp in full detail; rather, 
%we shall define a syntactic transformation of \pHOp into \HOp, rather than as a typed encoding.
%\footnote{The definition of a polyadic semantics would only add visual clutter to our presentation,
%as all results extend easily from monadic to polyadic communication.}
We give the definition of the encoding of polyadic semantics to monadic semantics.
Because of the polyadic to monadic encoding %, denoted  $\auxmap{\cdot}{\mathsf{p}}$,
we are able to focus on monadic session processes,
and rely on polyadic constructs simply as convenient syntactic sugar.
In fact, we shall rely on polyadicity to encode recursive behaviors.
%
\begin{definition}[Polyadic Into Monadic]\rm
	Define the process mapping $\encod{\cdot}{\cdot}{\mathsf{p}}: \HOp \to \HOp^{-\mathsf{p}}$:
	%$\auxmap{\cdot}{\mathsf{p}}:\pHOp \to \HOp$ as
\[
	\begin{array}{rcl}
		\map{\bout{k}{k_1, \cdots, k_n} P}{\mathsf{p}}
		&\defeq&
		\bout{k}{k_1} \cdots ;  \bout{k}{k_n} \map{P}{\mathsf{p}}
		\\

		\map{\binp{k}{x_1, \cdots, x_n} P}{\mathsf{p}}
		&\defeq&
		\binp{k}{x_1} \cdots ; \binp{k}{x_n}  \map{P}{\mathsf{p}}
		\\

		\map{\bbout{k}{\abs{x_1, \cdots, x_n} Q} P}{\mathsf{p}}
		&\defeq&
		\bbout{k}{\abs{z}\binp{z}{x_1} \cdots ; \binp{z}{x_n} \map{Q}{\mathsf{p}}} \map{P}^{\mathsf{p}}
		\\

		\map{\appl{X}{k_1, \cdots, k_n}}{\mathsf{p}}
		&\defeq&
		\newsp{s}{\appl{X}{s} \Par \bout{\dual{s}}{k_1} \cdots ; \bout{\dual{s}}{k_n} \inact} 
	\end{array}
	\]
	and as an homomorphism for the remaining constructs in \HOp. 
	Define the mapping on types $\tmap{\cdot}{\mathsf{p}}$ as follows:
\[
	\begin{array}{rcl}
		\tmap{\btout{S_1, \cdots, S_n}S}{\mathsf{p}}
		&\defeq&
		\btout{\tmap{S_1}{\mathsf{p}}} \cdots \btout{\tmap{S_n}{\mathsf{p}}}\tmap{S}{\mathsf{p}}
		\\
		\tmap{\btinp{S_1, \cdots, S_n}S}{\mathsf{p}}
		&\defeq&
		\btinp{\tmap{S_1}{\mathsf{p}}} \cdots \btinp{\tmap{S_n}{\mathsf{p}}}\tmap{S}{\mathsf{p}}
		\\
		\tmap{\lhot{(S_1, \cdots, S_n)}}{\mathsf{p}}
		&\defeq&
		\lhot{\big(\btinp{\tmap{S_1}{\mathsf{p}}} \cdots \btinp{\tmap{S_n}{\mathsf{p}}}\tinact\big)}
		\\
		\tmap{\shot{(S_1, \cdots, S_n)}}{\mathsf{p}}
		&\defeq&
		\shot{\big(\btinp{\tmap{S_1}{\mathsf{p}}} \cdots \btinp{\tmap{S_n}{\mathsf{p}}}\tinact\big)}
	\end{array}
\]
	and as an homomorphism for the remaining type constructs.
	%\jp{I prefer to be explicit in the encoding of polyadic abstraction/applications. Previous version is commented.}
\end{definition}
%
Passing a list of names over session channels is established
with a corresponding list of sequential send (resp. receive) prefixes.
When we are dealing with an abstraction over a list of bound variables,
then we create a new abstraction name and we use it to receive in a polyadic
way the list of names on the abstraction. Similarly application will instantiate
the abstraction subject with a new session name and will use it 
to send the list of names that are going to be applied on the abstraction.
Note that we do not allow polyadic mapping on shared names.
The polyadic mapping, as presented here, is sound only on session names.
%The semantics might break if we apply this mapping on shared names.

%\begin{proposition}
%	$\Gamma; \emptyset; \Sigma \proves \map{P}^{p} \hastype \Proc$
%\end{proposition}


\subsection{Encoding Recursion into Abstraction Passing}

Encoding the constructs for recursion present in $\sesp$ as process-passing communication requires to follow the fundamental
principle of copying the process that needs to exhibit recursive behaviour.
The primitive recursor operation creates copies of a process and uses them
as continuations, e.g:
\[
	\recp{X}{\bout{n}{m} \rvar{X}} \scong \bout{n}{m} \recp{X}{\bout{n}{m} \rvar{X}}
\]
In the above example the scope of name $n$ includes the entire process so
the type for $n$ should be recursive. An alternative representation
of the above process would be:
\[
	\bout{a}{n} \inact \Par \recp{X}{\binp{a}{x} \bout{x}{m} (\bout{a}{x} \inact \Par \rvar{X})} \red \bout{n}{m} (\bout{a}{n} \inact \Par \recp{X}{\binp{a}{x} \bout{x}{m} (\bout{a}{x} \inact \Par \rvar{X}))}
\]
Endpoint $n$ is being passed sequentially on copies of the 
same process to achieve the effect of infinite sending of value $m$.
If we apply the same principles on higher order semantics we get:
\[
	\begin{array}{l}
		\newsp{s_1}{\bbout{s_1}{(z) \bout{n}{m} \binp{z}{X} \newsp{s_2}{\appl{X}{s_2} \Par \bout{\dual{s_2}}{(z) \appl{X}{z} } \inact} } \inact \Par \binp{\dual{s_1}}{X} \newsp{s_3}{\appl{X}{s_3} \Par \bout{\dual{s_3}}{(z) \appl{X}{z}} \inact } }
		\\
		\red
		\\
		\newsp{s_3}{\bout{n}{m} \binp{s_3}{X} \newsp{s_2}{\appl{X}{s_2} \Par \bout{\dual{s_2}}{(z) \appl{X}{z}} \inact} \Par \bout{\dual{s_3}}{(z) \bout{n}{m} \binp{z}{X} \newsp{s_2}{\appl{X}{s_2} \Par \bout{\dual{s_2}}{(z) \appl{X}{z} } \inact}  } \inact }
	\end{array}
\]
In the above encoding the abstraction
\[
	(z) \bout{n}{m} \binp{z}{X} \newsp{s_2}{\appl{X}{s_2} \Par \bout{\dual{s_2}}{(z) \appl{X}{z} } \inact}
\]
has a linear type due to the free occurrence of the session channel $n$.
But when passed, the latter abstraction is applied in a shared manner, i.e.\ two
copies of the abstraction are instantiated, thus the whole
encoding is untypable. The untypability problem would not exist
provided that the abstraction being passed were not linear.


A typable encoding of the example would be:
\[
	\begin{array}{l}
		\newsp{s_1}{\bout{s_1}{(z,x) \bout{x}{m} \binp{z}{X} \newsp{s_2}{\appl{X}{s_2,x} \Par \bout{\dual{s_2}}{(z,x) \appl{X}{z,x} } \inact} } \inact \Par \binp{\dual{s_1}}{X} \newsp{s_3}{\appl{X}{s_3, n} \Par \bout{\dual{s_3}}{(z,x) \appl{X}{z,x}} \inact } }
		\\
		\red
		\\
		\newsp{s_3}{\bout{n}{m} \binp{s_3}{X} \newsp{s_2}{\appl{X}{s_2, n} \Par \bout{\dual{s_2}}{(z, x) \appl{X}{z, x}} \inact} \Par \bout{\dual{s_3}}{(z, x) \bout{x}{m} \binp{z}{X} \newsp{s_2}{\appl{X}{s_2, x} \Par \bout{\dual{s_2}}{(z, x) \appl{X}{z, x} } \inact}  } \inact }
	\end{array}
\]

The abstraction now has become:
\[
	(z,x) \bout{x}{m} \binp{z}{X} \newsp{s_2}{\appl{X}{s_2,x} \Par \bout{\dual{s_2}}{(z,x) \appl{X}{z,x} } \inact}
\]
by replacing the free ocurrence of channel $n$ with variable $x$ and
bind $x$ as an abstraction variable. We can then instantiate
the above abstraction by passing session $n$ around following the same
principle as the name passing discipline.

A preliminary tool to encode the $\sesp$ recursion primitives would be to
provide a mapping from processes to processes with no free names.
We require some auxiliary definitions.
%
\begin{definition}\rm 
	Let $\vmap{\cdot}: 2^{\mathcal{N}} \longrightarrow \mathcal{V}^\omega$
	be a map of sequences of names to sequences of variables, defined
	inductively as follows:
%
\[
	\vmap{n} = x_n \qquad \qquad \qquad \vmap{n \cat \tilde{m}} = x_n \cat \vmap{\tilde{m}}
\]
\end{definition}

Given a process $P$, we write $\ofn{P}$ to denote the
\emph{sequence} of free names of $P$, lexicographically ordered.
Intuitively, the following mapping transforms processes
with free session names into abstractions:
%
\begin{definition}\label{d:trabs}\rm
	Let $\sigma$ be a set of session names.
	Define $\auxmapp{\cdot}{\mathsf{v}}{\sigma}: \HOp \to \HOp$  as follows
%
\[
	\begin{array}{rcl}
		\auxmapp{\news{n} P}{\sigma}{\mathsf{v}} &\bnfis& \news{n} \auxmapp{P}{\mathsf{v}}{{\sigma \cat n}}\\
		\auxmapp{\bout{n}{\abs{x} Q} P}{\mathsf{v}}{\sigma} &\bnfis&
		\left\{
		\begin{array}{rl}
			\bbout{x_n}{\abs{x,\vmap{ \ofn{P}}} \auxmapp{Q}{\mathsf{v}}{\sigma}} \auxmapp{P}{\mathsf{v}}{\sigma} & n \notin \sigma\\
			\bbout{n}{\abs{x,\vmap{\ofn{P}}} \auxmapp{Q}{\mathsf{v}}{\sigma}} \auxmapp{P}{\mathsf{v}}{\sigma} & n \in \sigma
		\end{array}
		\right.
		\\
		\auxmapp{\binp{n}{X} P}{\mathsf{v}}{\sigma} &\bnfis&
		\left\{
		\begin{array}{rl}
			\binp{x_n}{X} \auxmapp{P}{\mathsf{v}}{\sigma} & n \notin \sigma\\
			\binp{n}{X} \auxmapp{P}{\mathsf{v}}{\sigma} & n \in \sigma
		\end{array}
		\right.
		\\
		\auxmapp{\bsel{n}{l} P}{\mathsf{v}}{\sigma} &\bnfis&
		\left\{
		\begin{array}{rl}
			\bsel{x_n}{l} \auxmapp{P}{\mathsf{v}}{\sigma} & n \notin \sigma\\
			\bsel{n}{l} \auxmapp{P}{\mathsf{v}}{\sigma} & n \in \sigma
		\end{array}
		\right.
		\\
		\auxmapp{\bsel{n}{l} P}{\mathsf{v}}{\sigma} &\bnfis&
		\left\{
		\begin{array}{rl}
			\bsel{x_n}{l} \auxmapp{P}{\mathsf{v}}{\sigma} & n \notin \sigma\\
			\bsel{n}{l} \auxmapp{P}{\mathsf{v}}{\sigma} & n \in \sigma
		\end{array}
		\right.
		\\
		\auxmapp{\bout{n}{m}P}{\mathsf{v}}{\sigma} &\bnfis&
		\left\{
		\begin{array}{rl}
		    \bout{n}{m}\auxmapp{P}{\mathsf{v}}{\sigma} & n, m \in \sigma \\
		    \bout{x_n}{m}\auxmapp{P}{\mathsf{v}}{\sigma} & n \not\in \sigma, m \in \sigma \\
		    \bout{n}{x_m}\auxmapp{P}{\mathsf{v}}{\sigma} & n \in \sigma, m \not\in \sigma \\
		    \bout{x_n}{x_m}\auxmapp{P}{\mathsf{v}}{\sigma} & n, m \not\in \sigma 
		\end{array}
		\right.
		\\
		\auxmapp{\binp{n}{x}P}{\mathsf{v}}{\sigma} &\bnfis&
		\left\{
		\begin{array}{rl}
		    \binp{n}{x}\auxmapp{P}{\mathsf{v}}{\sigma} & n \in \sigma \\
		    \binp{x_n}{x}\auxmapp{P}{\mathsf{v}}{\sigma} & n \not\in \sigma 
		\end{array}
		\right.
		\\
		\auxmapp{\appl{\X}{n}}{\mathsf{v}}{\sigma} &\bnfis&
		\left\{
		\begin{array}{rl}
			\appl{\X}{x_n} & n \notin \sigma\\
			\appl{\X}{n} & n \in \sigma\\
		\end{array}
		\right. 
%		\auxmapp{\inact}{\mathsf{v}}{\sigma} &\bnfis& \inact\\
%		\auxmapp{P \Par Q}{\mathsf{v}}{\sigma} &\bnfis& \auxmapp{P}{\mathsf{v}}{\sigma} \Par \auxmapp{Q}{\mathsf{v}}{\sigma} 
	\end{array}
\]
and homomorphically for inaction and parallel composition.
\end{definition}

Given a process $P$ with $\ofn{P} = m_1, \cdots, m_n$, we are interested in its associated (polyadic) abstraction, which is defined as
$\abs{x_1, \cdots, x_n}{\auxmapp{P}{\mathsf{v}}{\es} }$, where $\vmap{m_j} = x_j$, for all $j \in \{1, \ldots, n\}$.
This transformation from processes into abstractions can be reverted by
using abstraction and application with an appropriate sequence of session names:
%
\begin{proposition}\rm
	Let $P$ be a \HOp process with $\tilde{n} = \ofn{P}$.
	Also, suppose $\tilde{x} = \vmap{\tilde{n}}$.
%	Also, let $A_P$ be the polyadic abstraction $\abs{\tilde{x}}\auxmapp{P}{\mathsf{v}}{\emptyset}$ (cf. Def.~\ref{d:trabs}).
	Then we have: $P \scong \appl{X}{\tilde{n}}\subst{\abs{\tilde{x}}\auxmapp{P}{\mathsf{v}}{\emptyset}}{X}$.
%	$\appl{X}{\smap{\fn{P}}} \subst{(\vmap{\fn{P}}) \map{P}^{\emptyset}}{X} \scong P$
\end{proposition}

\begin{proof}
	$\appl{X}{\smap{\fn{P}}} \subst{(\vmap{\fn{P}}) \map{P}^{\emptyset}}{X} =
	\map{P}^{\sigma} \subst{\smap{\fn{P}}}{\vmap{\fn{P}}}$ 
	\dk{TODO}
	\qed
\end{proof}

We are now ready to define the encoding of $\sesp$
(including constructs for recursion) into strict process-passing.
We stress that we use polyadicity in abstraction and application only
as syntactic sugar, to simplify presentation.
For the sake of completeness, we give again the encodings for 
finite processes and types, as
formalized by $\encod{\cdot}{\cdot}{1}: \sespnr \to \HO$.

\begin{definition}[From $\sesp$ to $\HO$]\rm
	Let $f$ be a function from recursion variables to sequences of name variables.
	Define $\fencod{\cdot}{\cdot}{2}{f}: \sesp \to \HO$ as
%
\[
	\begin{array}{rcll}
%	\map{\rec{X}{P}}^{2} &=& \newsp{s}{\binp{s}{\X} \map{P}^{2} \Par \bout{\dual{s}}{\abs{z \cat \vmap{\fn{P}}}{\binp{z}{\X} \map{P}^{\es}}} \inact}\\
%	\map{r}^{2} &=& \newsp{s}{\appl{\X}{s \cat \smap{\fn{P}}} \Par \bout{\dual{s}}{ \abs{z \cat \vmap{\fn{P}}}{\appl{X}{z \cat \vmap{\fn{P}}}}} \inact} \\
		\pmapp{\recp{X}{P}}{2}{f} &\defeq& \newsp{s}{\binp{s}{\X} \pmapp{P}{2}{{f,\{\rvar{X}\to \tilde{n}\}}} \Par \bbout{\dual{s}}{\abs{\vmap{\tilde{n}}, z } \,{\binp{z}{\X} \auxmapp{\pmapp{P}{2}{{f,\{\rvar{X}\to \tilde{n}\}}}}{\mathsf{v}}{\es}}} \inact} & \quad \tilde{n} = \ofn{P} \\ 
		\pmapp{\rvar{X}}{2}{f} &\defeq& \newsp{s}{\appl{\X}{\tilde{n}, s} \Par \bbout{\dual{s}}{ \abs{\vmap{\tilde{n}},z}\,\,{\appl{X}{ \vmap{\tilde{n}}, z}}} \inact} & \quad \tilde{n} = f(\rvar{X}) \\
		\pmapp{\bout{k}{n} P}{2}{f}	&\defeq&	\bbout{k}{ \abs{z}{\,\binp{z}{X} \appl{X}{n}} } \pmapp{P}{2}{f} \\
		\pmapp{\binp{k}{x} Q}{2}{f}	&\defeq&	\binp{k}{X} \newsp{s}{\appl{X}{s} \Par \bbout{\dual{s}}{\abs{x} \pmapp{Q}{2}{f}} \inact} \\
		\tmap{\btout{S_1} {S} }{2}	&\defeq&	\bbtout{\lhot{\btinp{\lhot{\tmap{S_1}{2}}}\tinact}} \tmap{S}{2}  \\
		\tmap{\btinp{S_1} S }{2}	&\defeq&	\bbtinp{\lhot{\btinp{\lhot{\tmap{S_1}{2}}}\tinact}} \tmap{S}{2} \\
		\tmap{\bbtout{\chtype{S_1}} {S} }{2}	&\defeq&	\bbtout{\shot{\btinp{\shot{\chtype{\tmap{S_1}{2}}}}\tinact}} \tmap{S}{2}  \\
		\tmap{\bbtinp{\chtype{S_1}} {S} }{2}	&\defeq&	\bbtinp{\shot{\btinp{\shot{\chtype{\tmap{S_1}{2}}}}\tinact}} \tmap{S}{2}
	\end{array}
\]
%
and as a homomorphism for the other process constructs. 
\end{definition}

\begin{remark}\rm
	Furthermore we define a mapping for environments $\Gamma$, as follows:
	\[
		\tmap{\Gamma \cat \rvar{X}:\Sigma}{2} = \tmap{\Gamma}{2} \cat X:\shot{(\tilde{S}_{\Sigma}, S^*)}
		%X:\trec{t}{\big(\shot{(\tilde{S}_{\Sigma}, \btinp{\vart{t}}\tinact)}\big)}
	\]
	where
	$S^* = \trec{t}{\big((\tilde{S}_{\Sigma}, \btinp{\vart{t}}\tinact)\big)}$
	and
	$\tilde{S}_{\Sigma} = S_1, \ldots, S_m$ for any $\Sigma = \{n_1:S_1, \ldots, n_m:S_m\}$.
\end{remark}

\begin{proposition}\rm
	Encoding $\fencod{\cdot}{\cdot}{2}{f}: \sesp \to \HO$  
	is type-preserving (cf. Def.~\ref{def:ep}\,(1)).
\end{proposition}

\begin{proof}
	Proof in Appendix~\ref{app:enc_sesp_to_HO_t}.
	\qed
\end{proof}

\begin{comment}
\begin{proof}
	By induction on the structure of \sesp process $P_0$. 
	\begin{enumerate}[1.]
		\item Case $P_0 = \rvar{X}$. Then we have the following typing in the source language:
	
		{\small
		\[
		\tree{
		}{
		\Gamma \cat \rvar{X}: \Sigma ;\, \es ;\, \es \proves \rvar{X} \hastype \Proc
		}
		\]
		}
	
		Then the typing of $\pmapp{\rvar{X}}{2}{f}$ is as follows, assuming $f(\rvar{X}) = \tilde{n}$ and $\tilde{x} = \vmap{\tilde{n}}$.
		Also, we write $\Sigma_{\tilde{n}}$ 
		and $\Sigma_{\tilde{x}}$ 
		to stand for 
		$n_1: S_1, \ldots, n_m: S_m$ and
			$x_1: S_1, \ldots, x_m: S_m$, respectively. 
		Below, we assume that $\Gamma = \Gamma' \cat X:\shot{\tilde{T}}$, 
		where  
		%$$\tilde{T} =  \trec{t}{\big(\tilde{S}, \btinp{\vart{t}}\tinact\big)}$$.
		\begin{eqnarray*}
		\tilde{T} & = & \big(\tilde{S}, S^*\big) \\
		S^* & = & \bbtinp{A}\tinact \\
		A & = & \trec{t}{(\tilde{S}, \btinp{\vart{t}}\tinact)}
		\end{eqnarray*}
				{\small
		\[
		\tree{
		\tree{
		\tree{
		\tree{
		}{
		\Gamma ;\, \es ;\, \es \proves X \hastype \shot{\tilde{T}}
		}
		\quad 
		\begin{array}{c}
		\Gamma ;\, \es ;\, \{n_i: S_i \} \proves n_i \hastype S_i \\
		\Gamma ;\, \es ;\, \{s: S^* \} \proves s\hastype S^*  \\
		\end{array}
		}{
		\Gamma  ;\, \es ;\, \Sigma_{\tilde{n}}, s:\btinp{\shot{\tilde{T}}}\tinact
		\proves  
		\appl{\X}{\tilde{n}, s} \hastype \Proc
		} 
		\quad 
		\tree{
		\tree{
		\Gamma  ;\, \es ;\,   \es \proves \inact \hastype \Proc
		}{
		\Gamma  ;\, \es ;\,   \dual{s}: \tinact \proves \inact \hastype \Proc
		} 
		\quad
		\tree{
		\tree{
		\begin{array}{c}
		\Gamma ;\, \es ;\, \{x_i: S_i \} \proves x_i \hastype S_i \\
		\Gamma ;\, \es ;\, \{z: S^*  \} \proves z\hastype S^*  \\
		\Gamma ;\, \es ;\, \es \proves X \hastype \shot{\tilde{T}}  \\
		\end{array}	}{
			\Gamma  ;\, \es ;\,   \Sigma_{\tilde{x}}, \, z:S^*
		\proves 
		 {\appl{X}{ \tilde{x}, z}} \hastype \Proc
		}
		}{
			\Gamma  ;\, \es ;\,   \es
		\proves 
		 \abs{\tilde{x},z}\,\,{\appl{X}{ \tilde{x}, z}} \hastype \shot{\tilde{T}}
		} 	
		}{
		\Gamma  ;\, \es ;\,   \dual{s}: \btout{\shot{\tilde{T}}}\tinact
		\proves 
		\bbout{\dual{s}}{ \abs{\tilde{x},z}\,\,{\appl{X}{ \tilde{x}, z}}} \inact \hastype \Proc
		}
		}{
		\Gamma  ;\, \es ;\, \Sigma_{\tilde{n}}, s:\btinp{\shot{\tilde{T}}}\tinact, \, \dual{s}: \btout{\shot{\tilde{T}}}\tinact
		\proves 
		\appl{\X}{\tilde{n}, s} \Par \bbout{\dual{s}}{ \abs{\tilde{x},z}\,\,{\appl{X}{ \tilde{x}, z}}} \inact \hastype \Proc
		}
		}{
		\Gamma  ;\, \es ;\, \Sigma_{\tilde{n}}
		\proves 
		\newsp{s}{\appl{\X}{\tilde{n}, s} \Par \bbout{\dual{s}}{ \abs{\tilde{x},z}\,\,{\appl{X}{ \tilde{x}, z}}} \inact} \hastype \Proc
		}
		\]
		}
	
		\item Case $P_0 = \recp{X}{P}$. Then we have the following typing in the source language:
	
		{\small
		\[
		\tree{
		\Gamma \cat \rvar{X}:\Sigma ;\, \es ;\,  \Sigma \proves P \hastype \Proc
		}{
		\Gamma  ;\, \es ;\,  \Sigma \proves \recp{X}{P} \hastype \Proc
		}
		\]
		}
	
		Then we have the following typing in the target language ---we write $R$ to stand for $\pmapp{P}{2}{{f,\{\rvar{X}\to \tilde{n}\}} }$
		and $\tilde{x}$ to stand for $\vmap{\ofn{P}}$.
		{\small 
		\[
		\tree{
		\tree{
		\tree{
		\tree{
		\tmap{\Gamma}{2}\cat X:\shot{\tilde{T}};\, \es;\, \tmap{\Sigma_{\tilde{n}}}{2}
		\proves
		 R  \hastype \Proc
		}{
		\tmap{\Gamma}{2}\cat X:\shot{\tilde{T}};\, \es;\, \tmap{\Sigma_{\tilde{n}}}{2}, s:\tinact 
		\proves
		 R  \hastype \Proc
		}
		}{
		\tmap{\Gamma}{2};\, \es;\, \tmap{\Sigma_{\tilde{n}}}{2}, s:\btinp{\shot{\tilde{T}}}\tinact 
		\proves
		\binp{s}{\X} R  \hastype \Proc
			} \quad 
		\tree{
		\tree{
		\tmap{\Gamma}{2};\, \es;\, \es
		\proves
		\inact \hastype \Proc
		}{
		\tmap{\Gamma}{2};\, \es;\, \dual{s}:\tinact
		\proves
		\inact \hastype \Proc
		} 
		\quad 
		\tree{
		\tree{
		\tree{
		\tmap{\Gamma}{2} \cat X: \shot{\tilde{T}};\, \es;\, \tmap{\Sigma_{\tilde{x}}}{2}
		\proves
		{{\auxmapp{R}{\mathsf{v}}{\es}}}  \hastype \Proc
		}{
		\tmap{\Gamma}{2} \cat X: \shot{\tilde{T}};\, \es;\, \tmap{\Sigma_{\tilde{x}}}{2},z: \tinact
		\proves
		{{\auxmapp{R}{\mathsf{v}}{\es}}}  \hastype \Proc
		}
		}{
		\tmap{\Gamma}{2};\, \es;\, \tmap{\Sigma_{\tilde{x}}}{2}, \, z: \btinp{A}\tinact
		\proves
		{{\binp{z}{\X} \auxmapp{R}{\mathsf{v}}{\es}}}  \hastype \Proc
		}
		}{
		\tmap{\Gamma}{2};\, \es;\, \es
		\proves
		{\abs{\tilde{x}, z } \,{\binp{z}{\X} \auxmapp{R}{\mathsf{v}}{\es}}}  \hastype \shot{\tilde{T}}
		}	
		}{
			\tmap{\Gamma}{2};\, \es;\, \dual{s}:\btout{\shot{\tilde{T}}}\tinact
		\proves
		\bbout{\dual{s}}{\abs{\tilde{x}, z } \,{\binp{z}{\X} \auxmapp{R}{\mathsf{v}}{\es}}} \inact \hastype \Proc
		}
		}{
		\tmap{\Gamma}{2};\, \es;\, \tmap{\Sigma_{\tilde{n}}}{2}, s:\btinp{\shot{\tilde{T}}}\tinact , \dual{s}:\btout{\shot{\tilde{T}}}\tinact
		\proves
		\binp{s}{\X} R \Par \bbout{\dual{s}}{\abs{\tilde{x}, z } \,{\binp{z}{\X} \auxmapp{R}{\mathsf{v}}{\es}}} \inact \hastype \Proc
		}
		}{
		\tmap{\Gamma}{2};\, \es;\, \tmap{\Sigma_{\tilde{n}}}{2} 
		\proves
		\newsp{s}{\binp{s}{\X} R \Par \bbout{\dual{s}}{\abs{\tilde{x}, z } \,{\binp{z}{\X} \auxmapp{R}{\mathsf{v}}{\es}}} \inact} \hastype \Proc
		}
		\]
		}
	\end{enumerate}
\qed
\end{proof}
\end{comment}

\begin{proposition}\rm
	Encoding $\fencod{\cdot}{\cdot}{2}{f}: \sesp \to \HO$ 
	enjoys operational correspondence (cf. Def.~\ref{def:ep}\,(2)).
\end{proposition}

\begin{proof}[Sketch]
	Proof in Appendix~\ref{app:enc_sesp_to_HO_oc}.
	\dk{TBD.}
	\qed
\end{proof}

\subsection{From $\HO$ to $\sesp$}

We now discuss the encodability of  $\HO$ into $\sesp$, 
i.e., how to encode a higher-order calculus with abstraction passing only
into a calculus with name passing only. 
We essentially follow the representability result put forward by 
Sangiorgi~\cite{San92,SaWabook}, but casted in the setting of session-typed communications. 
As we shall see, linearity of session endpoints will play a role in adaptating Sangiorgi's 
encodability strategy into a typed setting. 
Intuitively, such a strategy represents the exchange of a process with the exchange of 
a \emph{trigger}---a freshly generated names. 
Triggers may then be used to activate copies of the process, which now becomes a persistent 
resource represented by an input-guarded replication. In session-based communication, a session name 
is a linear resource and cannot be replicated. Consider the following (naive) adaptation of 
Sangiorgi's strategy in which session names are used are triggers and exchanged processes would be have to used exactly once:
%\begin{definition}[From $\HO$ to $\sesp$. Naive approach]
\[
	\begin{array}{lcl}
		\pmap{\bout{k}{\abs{x} P_1} Q}{n} & \defeq &  \newsp{s}{\bout{k}{s} (\pmap{Q}{n} \Par \binp{\dual{s}}{x} \pmap{P_1}{n})} \\
		\pmap{\binp{k}{X} P}{n} & \defeq& \binp{k}{x} \pmap{P}{n}\\
		\pmap{\appl{X}{k}}{n} & \defeq & \bout{x}{k} \inact
	\end{array}
	\]
%\end{definition}
%
%\begin{proposition}
%	Let $\Gamma;\emptyset;\Sigma \proves P \hastype \Proc$ with
%	the typing derivation to use only linear session types. Then
%	$\map{P}^8$ respects the properties of definition~\ref{def:ep}.
%\end{proposition}
%
%\begin{proof}
%	\dk{TODO}
%\end{proof}
(The mapping $\pmap{\cdot}{n}$ would be defined homomorphically for the remaining $\HO$ constructs.)
Although $\pmap{\cdot}{n}$ captures the correct semantics when
dealing with systems that allow only linear process variables,
it suffers from non-typability in the presence
of shared process variables. For instance,
let $P = \bbout{n}{\abs{x}{\bout{x}{m}\inact}} \inact \Par \binp{\dual{n}}{X} (\appl{X}{s_1} \Par \appl{X}{s_2})$.
We would have
\[
	\pmap{P}{n} \defeq
	\newsp{s}{\bout{n}{s} \binp{\dual{s}}{x} \bout{x}{m} \inact \Par \binp{\dual{n}}{x} (\bout{x}{s_1} \inact \Par \bout{x}{s_2} \inact)}
\]
The above process is non typable since processes $(\bout{x}{s_1} \inact$ and $\bout{x}{s_2} \inact)$
cannot be put in parallel because they do not have disjoint session environments.

The correct approach would be to use replicated shared names
as triggers instead of session names. 
Below we write $\repl{} P$ as a shorthand notation for $\recp{X}{(P \Par \rvar{X})}$.

\begin{definition}[From $\HO$ to $\sesp$]\rm
	Define $\encod{\cdot}{\cdot}{3}: \HO \to \sesp$ as follows
	\[
	\begin{array}{rcl}
		\pmap{\bbout{k}{\abs{x} Q} P}{3} & \defeq &  \newsp{a}{\bout{k}{a} (\pmap{P}{3} \Par \repl{} \binp{a}{y} \binp{y}{x} \pmap{Q}{3})\,} \\
		\pmap{\binp{k}{X} P}{3} &\defeq&  \binp{k}{x} \pmap{P}{3}\\
		\pmap{\appl{X}{k}}{3} & \defeq & \newsp{s}{\bout{x}{s} \bout{\dual{s}}{k} \inact}\\
		\tmap{\btout{\lhot{S}}S_1}{3} & \defeq & \bbtout{\chtype{\btinp{\tmap{S}{3}}\tinact}}\tmap{S_1}{3} \\
		\tmap{\btinp{\lhot{S}}S_1}{3} & \defeq & \bbtinp{\chtype{\btinp{\tmap{S}{3}}\tinact}}\tmap{S_1}{3}
	\end{array}
	\]
\end{definition}

\begin{proposition}\rm
	Encoding $\encod{\cdot}{\cdot}{3}: \HO \to \sesp$  is type-preserving (cf. Def.~\ref{def:ep}\,(1)).
\end{proposition}

\begin{proof}
	Proof in Appendix~\ref{app:enc_HO_to_sessp_t}.
	\qed
\end{proof}

\begin{comment}
\begin{proof}
By induction on the structure of \HO process $P$. 
\begin{enumerate}[1.]

%%%% Output of (linear) channel
	\item Case $P = \bbout{k}{\abs{x}Q}P$. Then we have two possibilities, depending on the typing for $\abs{x}Q$.
	The first case concerns a linear typing, and  
	we have the following typing in the source language:
	{\small
	\[
		\tree{
			\tree{}{\Gamma; \emptyset; \Sigma_1 \cat k:S  \proves  P \hastype \Proc} \quad \tree{\Gamma ; \emptyset ; \Sigma_2\cat x:S_1 \proves  Q \hastype \Proc}{\Gamma ; \emptyset ; \Sigma_2 \proves  \abs{x}Q \hastype \lhot{S_1}} }{
			\Gamma; \emptyset; \Sigma_1 \cat \Sigma_2 \cat k:\btout{\lhot{S_1}}S \proves  \bbout{k}{\abs{x}Q} P \hastype \Proc}
	\]
	}
	The corresponding typing in the target language is as follows --- we write $U_1$ to stand for 
	$\chtype{\btinp{\tmap{S_1}{3}}\tinact}$.
	We also write 
	\begin{eqnarray*}
	\tmap{\Gamma_1}{3} & = & \tmap{\Gamma}{3} \cup a:\chtype{\btinp{\tmap{S_1}{3}}\tinact} \\
	\tmap{\Gamma_2}{3} & = & \tmap{\Gamma_1}{3} \cup \rvar{X}:\tmap{\Sigma_2}{3}
	\end{eqnarray*}
	Also $(*)$ stands for $\tmap{\Gamma_1}{3}; \es ; \es \proves a \hastype U_1$; 
	$(**)$ stands for $\tmap{\Gamma_2}{3}; \es ; \es \proves a \hastype U_1$; and
	$(***)$ stands for $\tmap{\Gamma_2}{3}; \es ; \es \proves \rvar{X} \hastype \Proc$.
	
		{\small
	\[
		\tree{
		\tree{
		\tree{}{(*)}  \quad 
			\tree{
			\tree{}{
			\tmap{\Gamma_1}{3}; \es ; \tmap{\Sigma_1}{3}, k:\tmap{S}{3} 
		\proves 
		\pmap{P}{3}  \hastype \Proc
			}
			\quad 
			\tree{
			\tree{
			\tree{}{
			(***)
			} 
			\quad 
			\tree{
			\tree{
			\tree{
			\tree{
			}{
			\tmap{\Gamma_2}{3}; \es ; \tmap{\Sigma_2}{3},  x:\tmap{S_1}{3}
			\proves 
			\pmap{Q}{3} \hastype \Proc
			}
			}{
			\tmap{\Gamma_2}{3}; \es ; \tmap{\Sigma_2}{3}, y:\tinact, x:\tmap{S_1}{3}
			\proves 
			\pmap{Q}{3} \hastype \Proc
			}
			}{
			\tmap{\Gamma_2}{3}; \es ; \tmap{\Sigma_2}{3}, y: \btinp{\tmap{S_1}{3}}\tinact
			\proves 
			\binp{y}{x}\pmap{Q}{3} \hastype \Proc
			} 
			\quad 
			\tree{
			}{
			(**)
			}
			}{
			\tmap{\Gamma_2}{3}; \es ; \tmap{\Sigma_2}{3} 
			\proves 
			\binp{a}{y}\binp{y}{x}\pmap{Q}{3} \hastype \Proc
			} 
			}{
			\tmap{\Gamma_2}{3}; \es ; \tmap{\Sigma_2}{3} 
		\proves 
		\binp{a}{y}\binp{y}{x}\pmap{Q}{3} \Par \rvar{X} \hastype \Proc
			}
			}{
			\tmap{\Gamma_1}{3}; \es ; \tmap{\Sigma_2}{3} 
		\proves 
		\recp{X}{(\binp{a}{y}\binp{y}{x}\pmap{Q}{3} \Par \rvar{X})} \hastype \Proc
			}
			}{
			\tmap{\Gamma_1}{3}; \es ; \tmap{\Sigma_1, \Sigma_2}{3}, k:\tmap{S}{3} 
		\proves 
		\pmap{P}{3} \Par 
		\recp{X}{(\binp{a}{y}\binp{y}{x}\pmap{Q}{3} \Par \rvar{X})} \hastype \Proc
			}
		}{
		\tmap{\Gamma_1}{3}; \es ; \tmap{\Sigma_1, \Sigma_2}{3}, k:\bbtout{U_1}\tmap{S}{3} 
		\proves 
		\bout{k}{a}(\pmap{P}{3} \Par 
		\recp{X}{(\binp{a}{y}\binp{y}{x}\pmap{Q}{3} \Par \rvar{X}))} \hastype \Proc
		}
		}{
		\tmap{\Gamma}{3}; \es ; \tmap{\Sigma_1, \Sigma_2}{3}, k:\bbtout{U_1}\tmap{S}{3} 
		\proves 
		\newsp{a}{\bout{k}{a}( 
		\pmap{P}{3} \Par 
		\recp{X}{(\binp{a}{y}\binp{y}{x}\pmap{Q}{3} \Par \rvar{X}))}
		} \hastype \Proc
		}
	\]
	}
	 In the second case, $\abs{x}Q$ has a shared type. We have the following typing in the source language:
	{\small
	\[
		\tree{
			\tree{}{\Gamma; \emptyset; \Sigma \cat k:S  \proves  P \hastype \Proc} \quad 
			\tree{
			\tree{\Gamma ; \emptyset ; \cat x:S_1 \proves  Q \hastype \Proc}{\Gamma ; \emptyset ; \es \proves  \abs{x}Q \hastype \lhot{S_1} }			
			}{\Gamma ; \emptyset ; \es \proves  \abs{x}Q \hastype \shot{S_1} } }{
			\Gamma; \emptyset; \Sigma  \cat k:\btout{\shot{S_1}}S \proves  \bbout{k}{\abs{x}Q} P \hastype \Proc}
	\]
	}
	The corresponding typing in the target language can be derived similarly as in the first case.
	
	\item Case $P = \binp{k}{X} P$. Then there are two cases, depending on the type of $X$. 
	In the first case,
	we have the following typing in the source language:
	{\small
	\[
			\tree{\Gamma \cat X : \shot{S_1};\, \emptyset ;\, \Sigma \cat k:S \proves  P \hastype \Proc
			}{
			\Gamma;\, \emptyset;\, \Sigma\cat k:\btinp{\shot{S_1}}S \proves  \binp{k}{X} P \hastype \Proc}
	\]
	}
	The corresponding typing in the target language is as follows:
	% --- we write $\Gamma_0$ to stand for $\Gamma \setminus \{X: \lhot{S_1}\}$.
		{\small
	\[
			\tree{\tmap{\Gamma}{3} \cat x : \chtype{\btinp{\tmap{S_1}{3}}\tinact};\, \emptyset ;\, \Sigma \cat k:\tmap{S}{3} \proves  \tmap{P}{3} \hastype \Proc
			}{
			\tmap{\Gamma}{3};\, \emptyset; \, \tmap{\Sigma}{3}\cat k:\bbtinp{\chtype{\btinp{\tmap{S_1}{3}}\tinact}}\tmap{S}{3} \proves  \binp{k}{x} \pmap{P}{3} \hastype \Proc}
	\]
	}

   In the second case,  
	we have the following typing in the source language:
	{\small
	\[
			\tree{\Gamma;\, \{X : \lhot{S_1}\};\, \emptyset ;\, \Sigma \cat k:S \proves  P \hastype \Proc
			}{
			\Gamma;\, \emptyset;\, \Sigma\cat k:\btinp{\lhot{S_1}}S \proves  \binp{k}{X} P \hastype \Proc}
	\]
	}
	The corresponding typing in the target language is as follows:
	% --- we write $\Gamma_0$ to stand for $\Gamma \setminus \{X: \lhot{S_1}\}$.
		{\small
	\[
			\tree{\tmap{\Gamma}{3} \cat x : \chtype{\btinp{\tmap{S_1}{3}}\tinact};\, \emptyset ;\, \Sigma \cat k:\tmap{S}{3} \proves  \tmap{P}{3} \hastype \Proc
			}{
			\tmap{\Gamma}{3};\, \emptyset;\, \tmap{\Sigma}{3}\cat k:\bbtinp{\chtype{\btinp{\tmap{S_1}{3}}\tinact}}\tmap{S}{3} \proves  \binp{k}{x} \pmap{P}{3} \hastype \Proc}
	\]
	}

	
	\item Case $P = \appl{X}{k}$. Also here we have two cases, depending on whether $X$ has linear or shared type.
	In the first case, $X$ is linear and
	we have the following typing in the source language:
	{\small
	\[
			\tree{\Gamma ;\, \{X : \lhot{S_1}\};\,  \es \proves  X \hastype \lhot{S_1} \quad \Gamma; \es ; \{k:S_1\} \proves k \hastype S_1
			}{
			\Gamma;\, \{X : \lhot{S_1}\};\, k:S_1 \proves  \appl{X}{k} \hastype \Proc}
	\]
	}
	The corresponding typing in the target language is as follows  --- below we write $\tmap{\Gamma_1}{3}$ to stand for $\tmap{\Gamma}{3} \cat x:\chtype{\btout{\tmap{S_1}{3}}\tinact}$:
		{\small
	\[
			\tree{
			\tree{
			\tree{
			\tree{
			\tmap{\Gamma_1}{3};\, \es;\,  \es \proves  \inact \hastype \Proc
			}{
			\tmap{\Gamma_1}{3};\, \es;\,  \dual{s}:\tinact \proves  \inact \hastype \Proc
			} \quad 
			\tree{
			}{
			\tmap{\Gamma_1}{3};\, \es;\, \{k:\tmap{S_1}{3}\} \proves  k \hastype \tmap{S_1}{3} 
			}
			}{
			\tmap{\Gamma_1}{3};\, \es;\,\, k:\tmap{S_1}{3},\,  \dual{s}:\btout{\tmap{S_1}{3}}\tinact \proves  \bout{\dual{s}}{k}\inact \hastype \Proc
			} \quad \tree{}{\tmap{\Gamma_1}{3} ;\, \es ;\, \es \proves x \hastype \chtype{\btout{\tmap{S_1}{3}}\tinact}}
			}{
			\tmap{\Gamma_1}{3};\, \es;\, k:\tmap{S_1}{3}, s:\btinp{\tmap{S_1}{3}}\tinact , \dual{s}:\btout{\tmap{S_1}{3}}\tinact \proves  \bout{x}{s}\bout{\dual{s}}{k}\inact \hastype \Proc
			}
			}{
			\tmap{\Gamma_1}{3};\, \es;\, k:\tmap{S_1}{3} \proves  \news{s}{(\bout{x}{s}\bout{\dual{s}}{k}\inact)} \hastype \Proc}
	\]
	}
	In the second case, $X$ is shared, and
	we have the following typing in the source language:
	{\small
	\[
			\tree{\Gamma \cat  X : \lhot{S_1} ;\,  \es ;\,  \es \proves  X \hastype \shot{S_1} \quad \Gamma; \es ; k:S_1 \proves k \hastype S_1
			}{
			\Gamma \cat X : \shot{S_1};\, \es ;\, k:S_1 \proves  \appl{X}{k} \hastype \Proc}
	\]
	}
	The associated typing in the target language is obtained similarly as in the first case. \qed
	\end{enumerate}
	\end{proof}
\end{comment}


\begin{proposition}\rm
	Encoding $\encod{\cdot}{\cdot}{3}: \HO \to \sesp$ 
	enjoys operational correspondence (cf. Def.~\ref{def:ep}\,(2)).
\end{proposition}

\begin{proof}
	Proof in Appendix~\ref{app:enc_HO_to_sessp_oc}.
	\qed
\end{proof}

\begin{comment}
\begin{proof}[Sketch]
For completeness, we 
consider the \HO process $P = {\bbout{k}{\abs{x} Q} P_1} \Par \binp{k}{X} P_2$. We have that
\[
P \red P_1 \Par P_2 \subst{\abs{x}Q}{X}
\]
In the target language, this reduction is mimicked as follows:
\begin{eqnarray*}
\pmap{P}{2} & = & \newsp{a}{\bout{k}{a} (\pmap{P_1}{3} \Par \repl{} \binp{a}{y} \binp{y}{x} \pmap{Q}{3})\,} 
                  \Par \binp{k}{x} \pmap{P_2}{3} \\
            & \red & \newsp{a}{\pmap{P_1}{3} \Par \repl{} \binp{a}{y} \binp{y}{x} \pmap{Q}{3} 
                  \Par  \pmap{P_2}{3}\subst{a}{x}}
\end{eqnarray*}
\qed
\end{proof}
\end{comment}


At this point an open  question would be if we could find an encoding that maps
session names to session names without the creation of shared names.

\dk{put intuition??}


We now define the formal notion of \emph{encoding} by 
extending to a typed setting existing criteria for untyped processes (as in, e.g.,
\cite{Nestmann00,Palamidessi03,DBLP:conf/lics/PalamidessiSVV06,DBLP:journals/iandc/Gorla10,DBLP:conf/icalp/LanesePSS10,DBLP:journals/tcs/FuL10,DBLP:journals/corr/abs-1208-2750,DBLP:conf/esop/PetersNG13}). 
We first define a typed calculus parameterised by a syntax, operational semantics, and typing.
Based on this definition, later on we define concrete instances of (higher-order) typed calculi.

%\smallskip 

\begin{definition}[Typed Calculus]\label{d:tcalculus}%\rm
	A \emph{typed calculus} $\tyl{L}$ is a tuple
	$\calc{\CAL}{\cal{T}}{\hby{\tau}}{\wb}{\proves}$
	where $\CAL$ and $\cal{T}$ are sets of processes and types, 
	respectively; also, $\hby{\tau}$, $\wb$, and $\proves$ 
	denote a transition system, a typed equivalence,
	and a typing system for $\CAL$, respectively. 
\end{definition}

%\smallskip 
%
%\begin{definition}[Typed Calculus]\label{d:tcalculus}\rm
%A \emph{typed calculus} $\tyl{L}$ is a tuple
%          $\calc{\CAL}{\cal{T}}{\cal{A}}{\wb}{\proves}$
%	where $\CAL$,  $\cal{T}$, $\cal{A}$ are sets of processes, types, and action labels (of an underlying transition system),
%respectively; and $\wb$ and $\proves$ 
%	denote %a transition system (with set of labels $\mathcal{A}$), 
%	a typed equivalence and type system for~$\CAL$. 
%\end{definition}
%
%
%\smallskip 
%
%\begin{definition}[Typed Calculus]\label{d:tcalculus}\rm
%A \emph{typed calculus} $\tyl{L}$ is a tuple
%          $\calc{\CAL}{\cal{T}}{\hby{\ell}}{\wb}{\proves}$
%	where $\CAL$ and $\cal{T}$ are sets of processes and types, 
%respectively; and $\hby{\mathcal{A}}$, $\wb$, and $\proves$ 
%	denote a transition system (with set of labels $\mathcal{A}$), 
%	a typed equivalence, and type system for $\CAL$, resp. 
%	We write $\mathcal{A}$ is t`he set of labels used in relation $\hby{\ell}$.
%\end{definition}

%\smallskip 

%\noi 
As we explain later, we write $\hby{\tau}$ to denote an operational semantics defined in terms of
$\tau$-transitions (to characterise reductions).
Our notion of encoding considers mappings on processes 
and types; these are denoted $\map{\cdot}$ and $\mapt{\cdot}$, respectively: %, and transition labels: 

\begin{definition}[Typed Encoding]%\rm
\label{def:tenc}
        Consider typed calculi
        $\tyl{L}_1=\!\calc{\CAL_1}{{\cal{T}}_1}{\hby{\tau}_1}{\wb_1}{\proves_1}$
        and
        $\tyl{L}_2=\calc{\CAL_2}{{\cal{T}}_2}{\hby{\tau}_2}{\wb_2}{\proves_2}$.
        %Let $\mathcal{A}_{i}$ be the set of labels in $\hby{\ell}_i$ ($i=1,2$).
	Given mappings $\map{\cdot}: \CAL_1 \to \CAL_2$ and
	$\mapt{\cdot}: {\cal{T}}_1 \to {\cal{T}}_2$, 
%	and $\mapa{\cdot}: \mathcal{A}_1 \to \mathcal{A}_2$, 
	we write 
%	$\enco{\map{\cdot}, \mapt{\cdot}, \mapa{\cdot}} : 
		$\enco{\map{\cdot}, \mapt{\cdot}} : 
	\tyl{L}_1 \to \tyl{L}_2$ to denote the \emph{typed encoding} of $\tyl{L}_1$ into $\tyl{L}_2$.
\end{definition}

%\smallskip 

%\noi 
Mapping $\mapt{\cdot}$ extends to typing
environments, e.g., $\mapt{\Delta \cat u:S} = \mapt{\Delta} \cat u:\mapt{S}$.
We introduce syntactic criteria for typed encodings.
Let $\sigma$ denote a substitution of names for names (a renaming, as usual). Given environments $\Delta$ and $\Gamma$,
we write $\sigma(\Delta)$ and $\sigma(\Gamma)$ to denote the effect of applying $\sigma$ on the 
domains of $\Delta$ and $\Gamma$
(clearly, $\sigma(\Gamma)$ concerns only shared names in $\Gamma$: process and recursive variables in $\Gamma$ are not affected by $\sigma$). 

%\smallskip 

\begin{definition}[Syntax Preservation]%\rm
	\label{def:sep}
	We say that 
	typed encoding 
	$\enco{\map{\cdot}, \mapt{\cdot}}: \tyl{L}_1 \to \tyl{L}_2$ is \emph{syntax preserving}
	if it is:
	
	\begin{enumerate}[1.]
		\item	\emph{Homomorphic wrt parallel},   if 
		$\mapt{\Gamma}; \emptyset; \mapt{\Delta_1 \cat \Delta_2} \proves_2 \map{P_1 \Par P_2} \hastype \Proc$ \\
		then 
		$\mapt{\Gamma}; \emptyset; \mapt{\Delta_1} \cat \mapt{\Delta_2} \proves_2 \map{P_1} \Par \map{P_2} \hastype \Proc$.

		\item	\emph{Compositional wrt restriction},  if 
		$\mapt{\Gamma}; \emptyset; \mapt{\Delta} \proves_2 \map{\news{n}P} \hastype \Proc$ \\
		then 
		$\mapt{\Gamma}; \emptyset; \mapt{\Delta} \proves_2 \news{n}\map{P} \hastype \Proc$.
		
		\item \emph{Name invariant},   if
		$\mapt{\sigma(\Gamma)}; \emptyset; \mapt{\sigma(\Delta)} \proves_2 \map{\sigma(P)} \hastype \Proc$
		then \\
		$\sigma(\mapt{\Gamma}); \emptyset; \sigma(\mapt{\Delta}) \proves_2 \sigma(\map{P}) \hastype \Proc$, 
		for any injective renaming  of names $\sigma$.
	\end{enumerate}
\end{definition}

%\smallskip 

%\noi 
Homomorphism wrt parallel (used in, e.g.,~\cite{Palamidessi03,DBLP:conf/lics/PalamidessiSVV06})
expresses that encodings should preserve the distributed topology of source processes. This criterion
 is appropriate for both encodability and non encodability results; in our setting, it is
%it admits an elegant formulation, also 
induced by rules for typed composition.
Compositionality wrt restriction 
is also supported by typing and is 
useful in our encodability results (\secref{sec:positive}).
The name invariance criterion follows \cite{DBLP:journals/iandc/Gorla10,DBLP:conf/icalp/LanesePSS10}. 

\newj{We now state \emph{type preservation}, a static criterion on the mapping 
$\mapt{\cdot}: {\cal{T}}_1 \to {\cal{T}}_2$: % of our typed encodings. 
it ensures that type operators are preserved.
The source and target languages that we consider here share
five (session) type operators: input, output, recursion (binary operators); selection and 
branching ($n$-ary operators). 
Type preservation enables us to focus on 
mappings $\mapt{\cdot}$ that always translate a type operator into itself. 
This is key to retain the meaning of structured protocols:
as session types abstract communication behaviour, type preserving encodings help us in maintaining those abstractions
across translations.}

\begin{definition}[Type Preservation]
	\label{def:tp}
	The typed encoding 
	$\enco{\map{\cdot}, \mapt{\cdot}}: \tyl{L}_1 \to \tyl{L}_2$ is \emph{type preserving}
	if for every $k$-ary type operator $\mathtt{op}$ in ${\cal{T}}_1$ it holds that 
	 $$\mapt{\mathtt{op}(T_1, \cdots, T_k)} = \mathtt{op}(\mapt{T_1}, \cdots, \mapt{T_k})$$
	\end{definition}


\begin{example}
\newj{
Following the discussion in \secref{sec:overview}, let 
$\mapt{\cdot}_u$ 
be a mapping on session types 
such that 
$\mapt{\btout{U} S}_u = \btinp{\mapt{U}_u} \mapt{S}_u$ 
and 
$\mapt{\btinp{U} S}_u = \btout{\mapt{U}_u} \mapt{S}_u$ (other type operators are translated homomorphically).
That is,  
$\mapt{\cdot}_u$  translates the output type operator into an input type operator (and viceversa). 
%exchanges (inverts) input and output session type operators. % in \defref{def:tp}.
Therefore, $\mapt{\cdot}_u$ does not satisfy  type preservation. 
}
\end{example}

Next we define semantic criteria for typed encodings:

%\smallskip 

\begin{definition}[Semantic Preservation]%\rm
\label{def:ep}
       Consider two typed calculi $\tyl{L}_1$ and  $\tyl{L}_2$, defined as 
        $\tyl{L}_1=\calc{\CAL_1}{{\cal{T}}_1}{\hby{\tau}_1}{\wb_1}{\proves_1}$
       and $\tyl{L}_2=\calc{\CAL_2}{{\cal{T}}_2}{\hby{\tau}_2}{\wb_2}{\proves_2}$.
%       ($i=1,2$) be typed calculi. 
We say that the encoding $\enco{\map{\cdot}, \mapt{\cdot}}: \tyl{L}_1 \to \tyl{L}_2$ is   \emph{semantic preserving}
if it satisfies the properties below.
%Given a label $\ell \neq \tau$, we write 
%$\mathsf{sub}(\ell)$
%to denote the \emph{subject} of the action.
	
	\begin{enumerate}[1.]
		\item \emph{Type Soundness}:
	if
	$\Gamma; \emptyset; \Delta \proves_1 P \hastype \Proc$ then 
	$\mapt{\Gamma}; \emptyset; \mapt{\Delta} \proves_2 \map{P} \hastype \Proc$,  
	for any   $P$ in $\CAL_1$.
			%\item \emph{Subject preserving}: if $\subj{\ell} = u$ then $\subj{\mapa{\ell}} =u$.

			\item \emph{Barb Preserving}: if $\Gamma; \Delta \proves_1 P \barb{n}$
		then $\mapt{\Gamma}; \mapt{\Delta} \proves_2 \map{P} \Barb{n}$.

	\item \emph{Operational Correspondence}: If $\Gamma; \emptyset; \Delta \proves_1 P \hastype \Proc$ then
		\begin{enumerate}
			\item	\NY{Completeness: 
			   If  
$\stytraargi{\Gamma}{\tau}{\Delta}{P}{\Delta'}{P'}{1}{1}$
			   then  $\exists Q, \Delta''$ s.t. \\
 (i)~$\wtytraargi{\mapt{\Gamma}}{}{\mapt{\Delta}}{\map{P}}{\mapt{\Delta''}}{Q}{2}{2}$
			    %(ii)~$\ell_2 = \mapa{\ell_1}$, 
			    and 
				(ii)~${\mapt{\Gamma}};{\mapt{\Delta''}}\proves_2 {Q}{\wb_2}
{\mapt{\Delta'}}\proves_2 {\map{P'}}$.}
				
			\item	Soundness:   
				If  $\wtytraargi{\mapt{\Gamma}}{}{\mapt{\Delta}}{\map{P}}{\mapt{\Delta'}}{Q}{2}{2}$
				then  $\exists P', \Delta''$ s.t.  \\
				(i)~$\stytraargi{\Gamma}{\tau}{\Delta}{P}{\Delta''}{P'}{1}{1}$
				%(ii)~$\ell_2 = \mapa{\ell_1}$, 
				and 
				(ii)~
${\mapt{\Gamma}};{\mapt{\Delta''}}\proves_2 {\map{P'}}{\wb_2}
{\mapt{\Delta'}}\proves_2 {Q}$.

		\end{enumerate}
		
		\item \emph{Full Abstraction:} 
		\wbbarg{\Gamma}{}{\Delta}{P}{\Delta'}{Q}{1}
		if and only if 
		\wbbarg{\mapt{\Gamma}}{}{\mapt{\Delta}}{\map{P}}{\mapt{\Delta'}}{\map{Q}}{2}.
		
	\end{enumerate}
\end{definition}

%\smallskip 

%\noi 
Together with type preservation (\defref{def:tp}), type soundness is a distinguishing criterion in our notion of encoding.
% it enables us to focus on encodings which retain the communication structures denoted by session types.
%The other semantic
%criteria build upon analogous definitions in the untyped setting, as we explain now. 
\newj{Barb preservation, related to success sensitiveness in~\cite{DBLP:journals/iandc/Gorla10}, is convenient in our developments as all considered calculi have the same notion of barb.}
Operational correspondence, standardly divided into completeness and soundness, is based
%in the formulation given i
on~\cite{DBLP:journals/iandc/Gorla10,DBLP:conf/icalp/LanesePSS10};
it relies on 
%the typed LTS of \defref{def:rlts}, 
%labelled transitions rather than on 
$\tau$-labeled transitions (reductions).
Completeness ensures that a step of the source process is mimicked
by a step of its associated encoding; soundness is its converse.
%Soundness ensures that the source process is mimicked 
%by its associated encoding; completeness is its converse.
%Completeness and soundness rely on 
%the typed LTS of \defref{def:rlts}, 
%rather than on reductions;
%Labels are considered up to  mapping $\mapa{\cdot}$, which offers flexibility when comparing different calculi. We require that $\mapa{\cdot}$ preserves  subjects, in accordance with the criteria in~\cite{DBLP:conf/icalp/LanesePSS10}.
{Above, operational correspondence is stated in generic terms.}
It is worth stressing that 
the operational correspondence statements 
%given in \secref{sec:positive} 
for our encodings 
 are tailored to the specifics of each encoding, and so they
 are actually stronger than the criteria given above
 {(see \propsref{prop:op_corr_HOp_to_HO}, \ref{prop:op_corr_HOp_to_p}, \ref{prop:op_corr_HOpp_to_HOp}, \ref{prop:op_corr_pHOp_to_HOp}
 and ~\cite{KouzapasPY15} for details).}
Finally, following~\cite{SangiorgiD:expmpa,DBLP:conf/lics/PalamidessiSVV06,Yoshida96},
we consider full abstraction as an encodability criterion: this leads to 
stronger encodability results. 
%The completeness direction of full abstraction is dropped when we prove the negative result. 
%From the criteria in \defref{def:sep} and~\ref{def:ep}
%we have the following derived criterion: 

%\begin{proposition}[Barb Preservation]
%\label{p:barbpres}
%Let
%	$\enco{\map{\cdot}, \mapt{\cdot}, \mapa{\cdot}}: \tyl{L}_1 \to \tyl{L}_2$
%	be a typed encoding.
%	Suppose the encoding is both
% operationally complete (cf.~\defref{def:ep}-3(a)) 
% and subject preserving (cf.~\defref{def:ep}-2).
% Then, it is also \emph{barb preserving}, i.e., 
%$\Gamma; \Delta \proves_1 P \barb{n}$
%implies
%$\mapt{\Gamma}; \mapt{\Delta} \proves_2 \map{P} \Barb{n}$.
%\end{proposition}
%
%%\smallskip 
%
%\begin{proof}
%The proof
%follows from the definition of barbs, operational completeness, and subject preservation.
%\qed
%\end{proof}

We introduce 
\emph{precise} and \emph{minimal}
 encodings.
While we state strong positive encodability results % in \secref{sec:positive}, 
in terms of {\em precise} encodings,
to prove the non-encodability result in \secref{ss:negative}, 
we appeal to the weaker {\em minimal} encodings.  

\begin{definition}[Typed Encodings: Precise and Minimal]%\rm
\label{def:goodenc}
We say that 
	the typed encoding 
	$\enco{\map{\cdot}, \mapt{\cdot} %, \mapa{\cdot}
	}: \tyl{L}_1 \to \tyl{L}_2$ is 
	%\begin{enumerate}[$\bullet$]
	%\item 
	\emph{precise}, if it is syntax, type, and semantic preserving (\defsref{def:sep}, \ref{def:tp}, \ref{def:ep}).
	%\item 
	We say that the encoding is
	\emph{minimal}, if it is syntax preserving 
	(\defref{def:sep}),
	barb preserving (\defref{def:ep}-2), 
	and operationally complete (\defref{def:ep}-3(a)).
%	\end{enumerate}
\end{definition}

%\smallskip 

%\noi %As explained earlier, o
%Our encodability results %presented next 
%rely on precise encodings; 
%our non encodability result %, presented in \secref{sec:negative}, 
%uses minimal encodings.
%Further we have:

%\smallskip 

The following property will come in handy in \secref{sec:extension}:

\begin{proposition}%[Composability of Precise Encodings]%\rm
	\label{pro:composition}
	Let %encodings 
	$\enco{\map{\cdot}^{1}, \mapt{\cdot}^{1}%, \mapa{\cdot}^{1}
	}: \tyl{L}_1 \to \tyl{L}_2$
	and 
	$\enco{\map{\cdot}^{2}, \mapt{\cdot}^{2}%, \mapa{\cdot}^{2}
	}: \tyl{L}_2 \to \tyl{L}_3$
	be two precise %typed 
	encodings.
	Then their composition, denoted 
	$\enco{\map{\cdot}^{2} \circ \map{\cdot}^{1}, \mapt{\cdot}^{2} \circ \mapt{\cdot}^{1} %, \mapa{\cdot}^{2}\circ \mapa{\cdot}^{1}
	}: \tyl{L}_1 \to \tyl{L}_3$,
	is precise. 
\end{proposition}

%%%%%%%%%%%%%%%%%%%%%%%%%%%%%%%%%%%%%%%%%%%%%%%%%%%%%%%%%%%%%%%%%%%%%%%%%%%%%%%%
%%%%%%%%%%%%%%%%%%%%%%%%%%%%%%%%%%%%%%%%%%%%%%%%%%%%%%%%%%%%%%%%%%%%%%%%%%%%%%%%

\section{Expressiveness Results}
\label{sec:positive}
%We present two encodings:
\begin{enumerate}
\item The higher-order name-passing communication (\HOp) into 
the higher-order communication without name-passing nor 
recursions (\HO) (\S\,\ref{subsec:HOpi_to_HO})
\item the higher-order communication without 
name-passing and recursions (\HO)
into the first-order name-passing communication
with recursions (\sessp) (\S\,\ref{subsec:HO_to_sesspi})
\end{enumerate}
By (1), we can encode \sessp into \HO and by (2), 
we can encode \HOp into \sessp.  
Note that it is obvious that \HOp can encode both 
$\HO$ and $\sessp$ by the identitiy mapping. 

\subsection{From \HOp to \HO}
\label{subsec:HOpi_to_HO}

\begin{definition}\rm 
	Let $\vmap{\cdot}: 2^{\mathcal{N}} \longrightarrow \mathcal{V}^\omega$
	be a map of sequences of 
lexicographically ordered names to sequences of variables, defined
	inductively as: 
	$\vmap{\epsilon} = \epsilon$ and $\vmap{n \cat \tilde{m}} = x_n \cat \vmap{\tilde{m}}$. 
\end{definition}

\begin{definition}[Trigger Mapping] \label{d:trabs}\label{d:auxmap}
	Let $\sigma$ be a set of session names.
	We define a trigger mapping,  
$\auxmapp{\cdot}{{}}{\sigma}: \HO \to \HO$, in Fig.~\ref{f:auxmap}.
\end{definition}

%
\begin{figure}[t]
\[
\small
\begin{array}{rl}
	\auxmapp{\bout{n}{\abs{x}{Q}} P}{{}}{\sigma} &\!\!\!\!\!\!\defeq
		\bout{u}{\abs{x}{\auxmapp{Q}{{}}{\sigma}}} \auxmapp{P}{{}}{\sigma}
\\[1mm]
%\auxmapp{\bout{n}{m} P}{{}}{\sigma} \defeq
%	    \bout{u}{v}\auxmapp{P}{{}}{\sigma} 
	\auxmapp{\appl{x}{n}}{{}}{\sigma}  \defeq
		\appl{x}{u} \quad 
	\auxmapp{\inact}{{}}{\sigma}  \defeq  \inact
 & 
			\auxmapp{\binp{n}{x} P}{{}}{\sigma}\defeq
		\binp{u}{x} \auxmapp{P}{{}}{\sigma} 
\\[1mm]
	\auxmapp{\bsel{n}{l} P}{{}}{\sigma} \defeq
		\bsel{u}{l} \auxmapp{P}{{}}{\sigma} 
 & 
	\auxmapp{\bbra{n}{l_i:P_i}_{i \in I}}{{}}{\sigma}  \defeq 
		\bbra{u}{l_i:\auxmapp{P_i}{{}}{\sigma}}_{i \in I}
	\vspace{1mm} \\
\auxmapp{\news{n} P}{{}}{\sigma}  \defeq  \news{n} \auxmapp{P}{{}}{{\sigma \cat n}}
 & 
	\auxmapp{P \Par Q}{{}}{\sigma}  \defeq  \auxmapp{P}{{}}{\sigma} \Par \auxmapp{Q}{{}}{\sigma} 
\end{array}
\]
%\[
%	\begin{array}{rcl}
%          \auxmapp{\news{n} P}{{}}{\sigma} &\bnfis& \news{n} \auxmapp{P}{{}}{{\sigma \cat n}}
%		\vspace{1mm} \\
%		\auxmapp{\bout{n}{\abs{x}{Q}} P}{{}}{\sigma} &\bnfis&
%		\left\{
%		\begin{array}{rl}
%			\bout{x_n}{\abs{(x,\vmap{\fn{P}})}{\auxmapp{Q}{{}}{\sigma}}} \auxmapp{P}{{}}{\sigma} & n \notin \sigma\\
%			\bout{n}{\abs{(x,\vmap{\fn{P}})}{\auxmapp{Q}{{}}{\sigma}}} \auxmapp{P}{{}}{\sigma} & n \in \sigma
%		\end{array}
%		\right.
%			\vspace{1mm}	\\ 
%		\auxmapp{\bout{n}{m} P}{{}}{\sigma} &\bnfis&
%		\left\{
%		\begin{array}{rl}
%		    \bout{n}{m}\auxmapp{P}{{}}{\sigma} & n, m \in \sigma \\
%		    \bout{x_n}{m}\auxmapp{P}{{}}{\sigma} & n \not\in \sigma, m \in \sigma \\
%		    \bout{n}{x_m}\auxmapp{P}{{}}{\sigma} & n \in \sigma, m \not\in \sigma \\
%		    \bout{x_n}{x_m}\auxmapp{P}{{}}{\sigma} & n, m \not\in \sigma 
%		\end{array}
%		\right.
%		\vspace{1mm} \\ 
%				\auxmapp{\binp{n}{X} P}{{}}{\sigma} &\bnfis&
%		\left\{
%		\begin{array}{rl}
%			\binp{x_n}{X} \auxmapp{P}{{}}{\sigma} & n \notin \sigma\\
%			\binp{n}{X} \auxmapp{P}{{}}{\sigma} & n \in \sigma
%		\end{array}
%		\right.
%			\vspace{1mm}	\\ 
%		\auxmapp{\binp{n}{x}P}{{}}{\sigma} &\bnfis&
%		\left\{
%		\begin{array}{rl}
%		    \binp{n}{x}\auxmapp{P}{{}}{\sigma} & n \in \sigma \\
%		    \binp{x_n}{x}\auxmapp{P}{{}}{\sigma} & n \not\in \sigma 
%		\end{array}
%		\right.
%		\vspace{1mm} \\ 
%		\auxmapp{\bsel{n}{l} P}{{}}{\sigma} &\bnfis&
%		\left\{
%		\begin{array}{rl}
%			\bsel{x_n}{l} \auxmapp{P}{{}}{\sigma} & n \notin \sigma\\
%			\bsel{n}{l} \auxmapp{P}{{}}{\sigma} & n \in \sigma
%		\end{array}
%		\right.
%		\vspace{1mm} \\
%		\auxmapp{\bbra{n}{l_i:P_i}_{i \in I}}{{}}{\sigma} &\bnfis&
%		%\auxmapp{\bsel{n}{l} P}{{}}{\sigma} &\bnfis&
%		\left\{
%		\begin{array}{rl}
%			\bbra{x_n}{l_i:\auxmapp{P_i}{{}}{\sigma}}_{i \in I}  & n \notin \sigma\\
%			\bbra{n}{l_i:\auxmapp{P_i}{{}}{\sigma}}_{i \in I}  & n \in \sigma
%		\end{array}
%		\right.
%		\vspace{1mm} \\
%		\auxmapp{\appl{\X}{n}}{{}}{\sigma} &\bnfis&
%		\left\{
%		\begin{array}{rl}
%			\appl{\X}{x_n} & n \notin \sigma\\
%			\appl{\X}{n} & n \in \sigma\\
%		\end{array}
%		\right. \\
%		\auxmapp{\inact}{{}}{\sigma} &\bnfis& \inact\\
%		\auxmapp{P \Par Q}{{}}{\sigma} &\bnfis& \auxmapp{P}{{}}{\sigma} \Par \auxmapp{Q}{{}}{\sigma} 
% \end{array}
%\]
%The auxiliary map (cf. Definition~\ref{d:auxmap}) 
%used in the encoding of the higher-order communication 
%with recursive definitions into higher-order communication 
%without recursive definitions and (Definition~\ref{d:enc:fotohorec}).
$u = n$ if $n\in \sigma$; otherwise $u = x_n$, and  
$\fn{P}$ denotes a sequence of lexicopraphically ordered 
free names in $P$. 
%The mapping is defined homomorphically for inaction and parallel composition.
\caption{\label{f:auxmap} A triger mapping}
\end{figure}

Given a process $P$ with $\fn{P} = m_1, \cdots, m_n$, we are interested in its associated abstraction, which is defined as
$\abs{x_1, \cdots, x_n}{\auxmapp{P}{{}}{\epsilon} }$, where $\vmap{m_j} = x_j$, for all $j \in \{1, \ldots, n\}$.
This transformation from processes into abstractions can be reverted by
using abstraction and application with an appropriate sequence of session names:
%
\begin{proposition}\rm
	Let $P$ be a \HOp process and 
	suppose $\tilde{x} = \vmap{\tilde{n}}$ where 
$\tilde{n} = \fn{P}$.
	Then $P \scong \appl{(\abs{\tilde{x}}{\auxmapp{P}{{}}{\emptyset}})}{\tilde{n}}$.
%	$\appl{X}{\smap{\fn{P}}} \subst{(\vmap{\fn{P}}) \map{P}^{\emptyset}}{X} \scong P$
\end{proposition}

\begin{figure}[t]
\[
\begin{array}{rcll}
	\noindent{\bf Types:} \quad 
	\vtmap{{S}}{1}	&\!\!\defeq\!\!&	\lhot{(\btinp{\lhot{\tmap{S}{1}}} \tinact)} \\
	\vtmap{\chtype{S}}{1}&\!\!\defeq\!\!&	\lhot{(\btinp{\shot{\chtype{\tmap{S}{1}}}} \tinact)}  \\
	\vtmap{\chtype{L}}{1}&\!\!\defeq\!\!&	\lhot{(\btinp{\shot{\chtype{\tmap{L}{1}}}} \tinact)} \\
	\vtmap{\lhot{C}}{1} &\!\!\defeq\!\!& \lhot{\tmap{C}{1}}\\
	\vtmap{\shot{C}}{1} &\!\!\defeq\!\!& \shot{\tmap{C}{1}}\\
	\tmap{\chtype{S}}{1}&\!\!\defeq\!\!&	\chtype{\tmap{S}{1}}  \\
	\tmap{\chtype{L}}{1}&\!\!\defeq\!\!&	\chtype{\tmap{L}{1}}  \\
	%		\tmap{\btout{S_1} {S} }{1}	&\!\!\defeq\!\!&	\bbtout{\lhot{\btinp{\lhot{\tmap{S_1}{1}}}\tinact}} \tmap{S}{1}  \\
	%		\tmap{\btinp{S_1} S }{1}	&\!\!\defeq\!\!&	\bbtinp{\lhot{\btinp{\lhot{\tmap{S_1}{1}}}\tinact}} \tmap{S}{1} \\
	%		\tmap{\bbtout{\chtype{U}} {S} }{1}	&\!\!\defeq\!\!&	\bbtout{\shot{\btinp{\shot{\chtype{\tmap{U}{1}}}}\tinact}} \tmap{S}{1}  \\
	%		\tmap{\bbtinp{\chtype{U}} {S} }{1}	&\!\!\defeq\!\!&	\bbtinp{\shot{\btinp{\shot{\chtype{\tmap{U}{1}}}}\tinact}} \tmap{S}{1} \\

	\tmap{\btout{U} S}{1} &\!\!\defeq\!\!& \btout{{\vtmap{U}{1}}} \tmap{S}{1}\\
	\tmap{\btinp{U} S}{1} &\!\!\defeq\!\!& \btinp{{\vtmap{U}{1}}} \tmap{S}{1}\\
	\tmap{\btsel{l_i: S_i}_{i \in I}}{1} &\!\!\defeq\!\!& \btsel{l_i: \tmap{S_i}{1}}_{i \in I}\\
			\tmap{\btbra{l_i: S_i}_{i \in I}}{1} &\!\!\defeq\!\!& \btbra{l_i: \tmap{S_i}{1}}_{i \in I}\\
	\tmap{\vart{t}}{1} \defeq \vart{t} \quad 
			\tmap{\trec{t}{S}}{1}  &\!\!\defeq\!\!&
	\trec{t}{\tmap{S}{1}}\quad 
	\tmap{\tinact}{1}  \defeq  \tinact\\[1mm]
	\hline
	%\end{array}
	%\]
	%\[
	%\begin{array}{rcll}
	\noindent{\bf Labels:} \quad \quad 
		\mapa{\bactout{n}{m}}^{1} &\!\!\defeq\!\!&   \bactout{n}{\abs{z}{\,\binp{z}{x} \appl{x}{m}} } \\
		\mapa{\bactinp{n}{m}}^{1} &\!\!\defeq\!\!&   \bactinp{n}{\abs{z}{\,\binp{z}{x} \appl{x}{m}} } \\
			\mapa{\bactout{n}{\abs{{x}}{P}}}^{1} &\!\!\defeq\!\!& \bactout{n}{\abs{{x}}{\pmapp{P}{1}{\es}}}\\
			\mapa{\bactinp{n}{\abs{{x}}{P}}}^{1} &\!\!\defeq\!\!& \bactinp{n}{\abs{{x}}{\pmapp{P}{1}{\es}}}\\
			\mapa{\bactsel{n}{l} }^{1} \!\defeq\! \bactsel{n}{l} 
	\quad 
			\mapa{\bactbra{n}{l} }^{1} &\!\!\defeq\!\!& \bactbra{n}{l} 
	\quad \quad 
			\mapa{\tau}^{1} \!\defeq\! \tau
\\[1mm]
\hline
\end{array}
\]
{\bf Terms} : 
\[
\begin{array}{rcll}
  \pmapp{\bout{u}{w} P}{1}{f}	&\!\!\defeq\!\!&	\bout{u}{ \abs{z}{\,\binp{z}{x} (\appl{x}{w})} } \pmapp{P}{1}{f} \\
  \pmapp{\binp{u}{\AT{x}{C}} Q}{1}{f}	&\!\!\defeq\!\!&	\binp{u}{y} \newsp{s}{\appl{y}{s} \Par \bout{\dual{s}}{\abs{x}{\pmapp{Q}{1}{f}}} \inact} \\
		\pmapp{\bout{u}{\abs{{x}}{Q}} P}{1}{f}  
&\!\!\defeq\!\!& \bout{u}{\abs{{x}}{\pmapp{Q}{1}{f}}} \pmapp{P}{1}{f} \\
		\pmapp{\binp{u}{\AT{x}{L}} P}{1}{f} &\!\!\defeq\!\!& \binp{u}{x} \pmapp{P}{1}{f}\\
		\pmapp{\bsel{s}{l} P}{1}{f} &\!\!\defeq\!\!& \bsel{s}{l} \pmapp{P}{1}{f}\\
		\pmapp{\bbra{s}{l_i: P_i}_{i \in I}}{1}{f} &\!\!\defeq\!\!& \bbra{s}{l_i: \pmapp{P_i}{1}{f}}_{i \in I}\\
		\pmapp{\inact}{1}{f} \!\!\defeq\!\!\inact
& & 
		\pmapp{\news{n} P}{1}{f} \!\!\defeq\!\! \news{n} \pmapp{P}{1}{f}\\
\pmapp{{x}\, {u}}{1}{f}
 \!\!\defeq\!\!
{x}\, {u}
& & 		\pmapp{P \Par Q}{1}{f} \!\!\defeq\!\! \pmapp{P}{1}{f} \Par \pmapp{Q}{1}{f} \\
		\pmapp{\recp{X}{P}}{1}{f} &\!\!\defeq\!\!&\!\!\!\!\!\!
	\newsp{s}{\\
& &\!\!\!\!\!\!\bout{\dual{s}}{\abs{(\vmap{\tilde{n}}, y)} 
\,{\binp{y}{z_\X} \auxmapp{\pmapp{P\subst{z_\X}{\X}}{1}{{f,\{z_\rvar{X}\to \tilde{n}\}}}}{{}}{\es}}} \inact
\\ 
& & \!\!\!\!\!\!
 \Par 
\binp{s}{z_\X} \pmapp{P\subst{z_X}{X}}{1}{{f,\{z_\rvar{X}\to \tilde{n}\}}}
} 
\quad (\tilde{n} = \fn{P}) \\ 
\pmapp{z_\rvar{X}}{1}{f} &\!\!\defeq\!\!& \newsp{s}{
\appl{z_X}{(\tilde{n}, s)}\\
& &  \Par \bbout{\dual{s}}{ \abs{(\vmap{\tilde{n}},y)}{\appl{z_X}{(\vmap{\tilde{n}}, y)}}} \inact}  \quad (\tilde{n} = f(z_\rvar{X})) \\
\end{array}
\]
The input bound variable $x$ is annotated by a type to distinct the first-order and higher-order cases.
\caption{\label{f:enc:hopi_to_ho}
Encoding of \HOp into \HO.
%(cf.~Defintion~\ref{d:enc:fotohorec}).
%Mappings 
%$\map{\cdot}^2$,
%$\mapt{\cdot}^2$, 
%and 
%$\mapa{\cdot}^2$
%are homomorphisms for the other processes/types/labels. 
}
\end{figure}

\begin{definition}[Full Higher-Order Pi into Higher-Order]
\label{d:enc:hopitoho}
Let $f$ be a function from variables to sequences of name variables.
%
Let $\tyl{L}_{\HOp}=\calc{\HOp}{{\cal{T}}_1}{\hby{\ell}}{\wb_H}{\proves}$
and 
$\tyl{L}_{\HO}=\calc{\HO}{{\cal{T}}_2}{\hby{\ell}}{\wb_H}{\proves}$. 
where 
${\cal{T}}_1$ and ${\cal{T}}_2$ are sets of types of $\HOp$ 
and $\HO$, respectively, 
the typing $\proves$ is defined in 
Fig.~\ref{fig:typerulesmy} 
and the equivalence $\hwb$ is defined in Definition~\ref{d:bisim}.
We define the typed encoding 
the typed encoding $\enco{\map{\cdot}^{1}, \mapt{\cdot}^{1}, \mapa{\cdot}^{1}}: \HOp \to \HO$ in 
in Fig.~\ref{f:enc:hopi_to_ho}. 
\end{definition}

\begin{theorem}[Encoding of Full Higher-Order Pi into Higher-Order]
\label{f:enc:hopitoho}
The encoding from $\tyl{L}_{\HOp}$ into $\tyl{L}_{\HO}$ 
defined in Definition~\ref{d:enc:hopitoho}
is semantic preserving. 
\end{theorem}

\subsection{From \HO to \sessp}
\label{subsec:HO_to_sessp}

\begin{definition}[Higher-Order into First-Order Pi]
\label{d:enc:hopitopi}
$\tyl{L}_{\sessp}=\calc{\sessp}{{\cal{T}}_3}{\hby{\ell}}{\fwb}{\proves}$. 
where the typing is defined in 
Fig.~\ref{fig:typerulesmy} 
and the equivalence $\fwb$ is defined in Definition~\ref{d:fwb}.
${\cal{T}}_3$ is a set of types of $\sessp$.  
%
We define the mappings $\map{\cdot}^{2}$, $\mapt{\cdot}^{2}$, $\mapa{\cdot}^{2}$
in Fig.~\ref{f:enc:ho_to_sessp}. 
\end{definition}

\begin{figure}[t]
\[
\begin{array}{l}
	\begin{array}{rcl}
\noindent{\bf Types:}\quad 
		\tmap{\btout{\lhot{S}}S_1}{2} & \defeq & \bbtout{\chtype{\btinp{\tmap{S}{2}}\tinact}}\tmap{S_1}{2} \\
		\tmap{\btinp{\lhot{S}}S_1}{2} & \defeq & \bbtinp{\chtype{\btinp{\tmap{S}{2}}\tinact}}\tmap{S_1}{2} 
\\[1mm]
\hline
%\end{array}
%\]
%\[
%\begin{array}{rcll}
\noindent{\bf Labels:}\quad \quad 
		\mapa{\bactout{u}{\abs{ x}{P}} }^2  & \defeq & \news{a} \bactout{u}{a} \\
		\mapa{\bactinp{u}{\abs{ x}{P}} }^2 &  \defeq & \bactinp{u}{a}
\\[1mm]
\hline
\end{array}
\end{array}
\]
\hspace{4mm}{\bf Terms} :
\[
\begin{array}{rcll}
		\pmap{\bout{u}{\abs{x}{Q}} P}{2} &\!\!\!\! \defeq \!\!\!\!&  \left\{
		\begin{array}{r}
			\newsp{a}{\bout{u}{a} (\pmap{P}{2} \Par \repl{} \binp{a}{y} \binp{y}{x} \pmap{Q}{2})\,}\\
                  (s \notin \fn{Q}) \\
			\newsp{a}{\bout{u}{a} (\pmap{P}{2} \Par \binp{a}{y} \binp{y}{x} \pmap{Q}{2})\,}\quad\\
            \textrm{(otherwise)} %\dk{Q \textrm{ linear}} \\
		\end{array}
		\right.
		\\
\pmap{\binp{u}{x} P}{2} &\!\!\!\! \defeq \!\!\!\! &  \binp{u}{x} \pmap{P}{2}\\
\pmap{\appl{x}{u}}{2} & \!\!\!\! \defeq \!\!\!\! & \newsp{s}{\bout{x}{s} \bout{\dual{s}}{u} \inact}\\

	\end{array}
	\]
$\map{\cdot}^2$,
$\mapt{\cdot}^2$, 
and 
$\mapa{\cdot}^2$
are homomorphisms for the other types, labels and processes.   
	\caption{
Encoding of \HO into \sessp.
\label{f:enc:ho_to_sessp}
}
\end{figure}

\begin{theorem}[Encoding of Higher-Order into First-Order Pi]
\label{f:enc:hotopi}
The encoding from $\tyl{L}_{\HO}$ into $\tyl{L}_{\sessp}$ defined in 
Definition~\ref{d:enc:hopitopi} 
is semantic preserving. 
\end{theorem}

\begin{corollary}[Encoding of Higher-Order Pi into First-Order Pi]
The encoding from $\tyl{L}_{\HOp}$ into $\tyl{L}_{\sessp}$ is semantic preserving. 
\end{corollary}

Note that an encoding from $\tyl{L}_{\HOp}$ into $\tyl{L}_{\sessp}$
can be defined without using the first encoding result.  




 We first present two encodability results:
(1)~higher-order communication with recursion and name-passing   (\HOp) into 
higher-order communication without name-passing nor recursion (\HO) (\secref{subsec:HOpi_to_HO}); and 
(2)~\HOp into the first-order calculus with name-passing  
with recursion (\sessp) (\secref{subsec:HOp_to_sessp}).
We then compare these  encodings (\secref{ss:compare}). 
Moreover, in \secref{ss:negative} we state our impossibility result for shared/linear names.
We consider the typed calculi (cf.~\defref{d:tcalculus}):
%\begin{enumerate}[-]
%	\item
%	$\tyl{L}_{\HOp}=\calc{\HOp}{{\cal{T}}_1}{\hby{\tau}}{\hwb}{\proves}$,
%	\item
%	$\tyl{L}_{\HO}=\calc{\HO}{{\cal{T}}_2}{\hby{\tau}}{\hwb}{\proves}$,
%	\item
%	$\tyl{L}_{\sessp}=\calc{\sessp}{{\cal{T}}_3}{\hby{\tau}}{\fwb}{\proves}$ 
%\end{enumerate}
	$$\tyl{L}_{\HOp}=\calc{\HOp}{{\cal{T}}_1}{\hby{\tau}}{\hwb}{\proves}
	\quad
	\tyl{L}_{\HO}=\calc{\HO}{{\cal{T}}_2}{\hby{\tau}}{\hwb}{\proves}
	\quad
	\tyl{L}_{\sessp}=\calc{\sessp}{{\cal{T}}_3}{\hby{\tau}}{\fwb}{\proves}$$
where: 
${\cal{T}}_1$, ${\cal{T}}_2$, 
and ${\cal{T}}_3$
are sets of types of $\HOp$, $\HO$, and $\sessp$, respectively. 
The typing $\proves$ is defined in 
%\figref{fig:typerulesmy}.
\secref{sec:types}.
The LTSs follow the intuitions given in \secref{ss:equiv}.
%are as in \defref{def:rlts}, 
Moreover, 
$\hwb$ is as in \defref{d:hbw}, and 
$\fwb$ is as in \defref{d:fwb}.


\begin{comment}
\dk{
We further prove that the two encodings are precise.
An important issue to adress on the precissenes of
the encoding is that of type preservation. In the
context of types it is not enough to check that
the reduction semantics and the behavioural equivalences
correspond. Type preservation gives a corresponding result on the
type derivation of the encoded process, thus we can
see that the encoding is preserving a specific type
behaviour. For example consider the followin encoding
of first-order passing into higher-order passing:
\[
	\begin{array}{rcl}
		\map{\bout{n}{m} P} &=& \binp{n}{x} (\map{P} \Par \appl{x}{m})\\
		\map{\binp{n}{x} P} &=& \bout{n}{\abs{x} P} \inact
	\end{array}
\]
%
Using the above encoding we could simulate name passing, e.g:
\[
	\begin{array}{rcl}
		\bout{n}{m} P \Par \binp{\dual{n}}{x} Q &\red& P \Par Q \subst{m}{x}\\
		\map{\bout{n}{m} P \Par \binp{\dual{n}}{x} Q} &=&\\
		\binp{n}{x} (\map{P} \Par \appl{x}{m}) \Par \bout{\dual{n}}{\abs{x} \map{Q}} \inact &\red&
		\map{P} \Par \appl{\abs{x}{\map{Q}}}{m}\\
		&\red&
		\map{P} \Par \map{Q}\subst{m}{x}
	\end{array}
\]
The distinctive characteristic of this encoding is that
it encodes the output prefix into an input prefix, and the
input prefix into an output prefix which in turn has
an impact on the relation on the session type of the encoding.
The interaction structure of the session is not preserved by the
above encoding.
The mapping on types gives an insight on the behaviour of processes.
}
\end{comment}


\subsection{From \HOp to \HO}
\label{subsec:HOpi_to_HO}
$\HO$ is expressive enough to
precisely encode \HOp.
%the full \HOp-calculus.
As discussed above, the main challenges are to encode (1) name passing 
and (2) recursion, 
for which 
we only use  abstraction passing. 
 As explained in \secref{sec:overview}, for (1), we pass  
an % simple 
abstraction which enables to use the name upon application. 
For~(2), we 
copy a process upon reception; passing around linear abstractions
%presents a limitation 
is \NY{delicate} 
because 
they cannot be copied.
To handle linearity, we define the following auxiliary 
%a preliminary tool which is a mapping from
 mapping 
$\auxmapp{\cdot}{{}}{\sigma}$
from processes with free names to processes without free
names (but with free variables instead):
%from processes \jpc{with free names} to processes without free names (but with free variables) (\defref{d:auxmap}). 
%We require two auxiliary definitions.

%\smallskip 

%\begin{definition}\rm 
%%\label{def:hop_to_ho}
%	Let $\vmap{\cdot}: 2^{\mathcal{N}} \longrightarrow \mathcal{V}^\omega$
%	be a map of sequences of lexicographically ordered names to sequences of variables, defined
%	inductively as: 
%	$\vmap{\epsilon} = \epsilon$ and $\vmap{n \cat \tilde{m}} = x_n \cat \vmap{\tilde{m}}$. 
%\end{definition}
%
%\smallskip 

%\noi The following auxiliary mapping transforms processes
%with free names into abstractions and it is
%used in \defref{d:enc:hopitoho}.
%
%\smallskip 




\begin{definition}[Auxiliary Mapping] \label{d:trabs}\label{d:auxmap}
	Let $\vmap{\cdot}: 2^{\mathcal{N}} \longrightarrow \mathcal{V}^\omega$
	denote a map of sequences of lexicographically ordered names to sequences of variables, defined
	inductively 
	as: 
	$\vmap{\epsilon} = \epsilon$ and $\vmap{n \cat \tilde{m}} = x_n \cat \vmap{\tilde{m}}$. 
	Also, let $\sigma$ be a set of session names.
	\figref{f:auxmap} defines an auxiliary mapping
	$\auxmapp{\cdot}{{}}{\sigma}: \HO \to \HO$.
\end{definition}
% !TEX root = ../journal16kpy.tex
%\begin{figure}[t!]
%	\begin{align*}
%		\auxmapp{\bout{n}{\abs{x}{Q}} P}{{}}{\sigma} & \defeq \bout{u}{\abs{x}{\auxmapp{Q}{{}}{\sigma}}} \auxmapp{P}{{}}{\sigma}
%		&
%		\auxmapp{\bbra{n}{l_i:P_i}_{i \in I}}{{}}{\sigma} & \defeq \bbra{u}{l_i:\auxmapp{P_i}{{}}{\sigma}}_{i \in I}		
%		\\
%		\auxmapp{\binp{n}{x} P}{{}}{\sigma} & \defeq \binp{u}{x} \auxmapp{P}{{}}{\sigma} 
%		&
%		\auxmapp{\bsel{n}{l} P}{{}}{\sigma} & \defeq \bsel{u}{l} \auxmapp{P}{{}}{\sigma} 
%		\\
%		\auxmapp{\news{n} P}{{}}{\sigma} & \defeq \news{n} \auxmapp{P}{{}}{{\sigma \cat n}}
%		&
%		\auxmapp{\appl{(\lambda x.Q)}{n}}{{}}{\sigma}  & \defeq \appl{(\lambda x.\auxmapp{Q}{{}}{\sigma})}{u}
%		\\
%		\auxmapp{P \Par Q}{{}}{\sigma} & \defeq \auxmapp{P}{{}}{\sigma} \Par \auxmapp{Q}{{}}{\sigma} 
%		&
%		\auxmapp{\appl{x}{n}}{{}}{\sigma} & \defeq \appl{x}{u}
%		\\
%		\auxmapp{\inact}{{}}{\sigma}  & \defeq  \inact	
%		& 
%\end{align*}
%\begin{center}
%	{In all cases: $u = \begin{cases} n & \text{if $n\in \sigma$} \\ x_n & \text{otherwise ($x$ fresh)} \end{cases}$}
%\end{center}
%%\vspace{-3mm}
%\caption{\label{f:auxmap} Auxiliary mapping used to encode \HOp into \HO (\defref{d:auxmap}).}
%%\vspace{-1mm}
%%\Hlinefig 
%\end{figure}


\begin{figure}[t!]
	\begin{align*}
		\auxmapp{\bout{w}{\abs{x}{Q}} P}{{}}{\sigma} & \defeq \bout{u}{\abs{x}{\auxmapp{Q}{{}}{{\sigma \cat x}}}} \auxmapp{P}{{}}{\sigma}
		&
		\auxmapp{\bbra{w}{l_i:P_i}_{i \in I}}{{}}{\sigma} & \defeq \bbra{u}{l_i:\auxmapp{P_i}{{}}{\sigma}}_{i \in I}		
		\\
		\auxmapp{\binp{w}{x} P}{{}}{\sigma} & \defeq \binp{u}{x} \auxmapp{P}{{}}{\sigma} 
		&
		\auxmapp{\bsel{w}{l} P}{{}}{\sigma} & \defeq \bsel{u}{l} \auxmapp{P}{{}}{\sigma} 
		\\
		\auxmapp{\news{n} P}{{}}{\sigma} & \defeq \news{n} \auxmapp{P}{{}}{{\sigma \cat n}}
		&
		\auxmapp{\appl{(\lambda x.Q)}{w}}{{}}{\sigma}  & \defeq \appl{(\lambda x.\auxmapp{Q}{{}}{{\sigma \cat x}})}{u}
		\\
		\auxmapp{P \Par Q}{{}}{\sigma} & \defeq \auxmapp{P}{{}}{\sigma} \Par \auxmapp{Q}{{}}{\sigma} 
		&
		\auxmapp{\appl{x}{w}}{{}}{\sigma} & \defeq \appl{x}{u}
		\\
		\auxmapp{\inact}{{}}{\sigma}  & \defeq  \inact	
		& 
\end{align*}
\begin{center}
	{In all cases: $u = 
	\begin{cases} 
	x_n & \text{if $w$ is a name $n  \not\in \sigma$ ($x$ fresh)}
	\\
	w & \text{otherwise: $w$ is a variable or a name $n \in \sigma$} 
	 	\end{cases}$}
\end{center}
%\vspace{-3mm}
\caption{\label{f:auxmap} Auxiliary mapping used to encode \HOp into \HO (\defref{d:auxmap}).}
%\vspace{-1mm}
%\Hlinefig 
\end{figure}

%

\newj{Let $P$ be an \HOp process with $\fn{P} = \{n_1, \cdots, n_k\}$.
Intuitively, 
our encoding 
$\pmapp{\cdot}{1}{f}$ %: \HOp \to \HO$
%of \HOp into \HO %, givenin \figref{f:enc:hopi_to_ho}, 
%exploits 
% we are interested in 
%$P$ into \HO we 
exploits  
  the %\HO 
 abstraction
%defined as
$\abs{x_1,\cdots, x_k}{\auxmapp{\pmapp{P}{1}{f}}{{}}{\emptyset} }$, where $\vmap{n_j} = x_j$, for all $j \in \{1, \ldots, k\}$: }
%In the following we make this intuition precise.

%This transformation from processes into abstractions can be reverted by
%using abstraction and application with an appropriate sequence of session names:
%%
%\begin{proposition}\rm
%	Let $P$ be a \HOp process and 
%	suppose $\tilde{x} = \vmap{\tilde{n}}$ where 
%$\tilde{n} = \fn{P}$.
%	Then $P \scong \appl{(\abs{\tilde{x}}{\auxmapp{P}{{}}{\emptyset}})}{\tilde{n}}$.
%%	$\appl{X}{\smap{\fn{P}}} \subst{(\vmap{\fn{P}}) \map{P}^{\emptyset}}{X} \scong P$
%\end{proposition}



%\smallskip 

\begin{definition}[Typed Encoding of \HOp into \HO]
\label{d:enc:hopitoho}
Let $f$ be a map from process variables to sequences of name variables.
%
%Let $\tyl{L}_{\HOp}=\calc{\HOp}{{\cal{T}}_1}{\hby{\ell}}{\wb_H}{\proves}$
%and 
%$\tyl{L}_{\HO}=\calc{\HO}{{\cal{T}}_2}{\hby{\ell}}{\wb_H}{\proves}$. 
%where 
%${\cal{T}}_1$ and ${\cal{T}}_2$ are sets of types of $\HOp$ 
%and $\HO$, respectively, 
%the typing $\proves$ is defined in 
%\figref{fig:typerulesmy} 
%and $\hwb$ is defined in \defref{d:hbw}. 
The typed encoding 
$\enco{\map{\cdot}^{1}_f, \mapt{\cdot}^{1} %, \mapa{\cdot}^{1}
}: \tyl{L}_{\HOp} \to \tyl{L}_{\HO}$ is given in 
\figref{f:enc:hopi_to_ho}. 
Mapping $\mapt{\cdot}^{1}$ on types homomorphically extends to 
environments $\Delta$
and
$\Gamma$, with
$
\tmap{\Gamma \cat \varp{X}:\Delta_1}{1}  =  \tmap{\Gamma}{1} \cat z_X:\shot{(S_1,\ldots,S_m,S^*)}
$
%\[
%	\begin{array}{l}
%%	    \mapt{\Delta \cat s: S}^{1} & =  & \mapt{\Delta}^{1} \cat s:\mapt{S}^{1} & \\
%%		\mapt{\Gamma \cat u: \chtype{S}}^{1} & =  & \mapt{\Gamma}^{1} \cat u:\chtype{\mapt{S}^{1}} & \\
%%		\mapt{\Gamma \cat u: \chtype{L}}^{1} & = &  \mapt{\Gamma}^{1} \cat u:\chtype{\mapt{L}^{1}} & \\
%		\tmap{\Gamma \cat \varp{X}:\Delta}{1}  =  \tmap{\Gamma}{1} \cat z_X:\shot{(S_1,\ldots,S_m,S^*)} \ 
%	\end{array}
%\]
where  
$S^*$ is defined as $\trec{t}{\btinp{\shot{(S_1,\ldots,S_m,\vart{t})}} \tinact}$
provided that $\Delta_1 = \{n_i:S_i\}_{1\leq i\leq m}$.
%and $\Delta = \{n_1:S_1, \ldots, n_m:S_m\}$. 
\end{definition}

\begin{figure}[t!]
\noindent{\bf Types:}
\\
$
%\begin{array}{rclcrcl}
\begin{array}{c}
	\vtmap{{S}}{1} \defeq	\lhot{(\btinp{\lhot{\tmap{S}{1}}} \tinact)}
	\qquad
	\vtmap{\chtype{S}}{1} \defeq	\lhot{(\btinp{\shot{\chtype{\tmap{S}{1}}}} \tinact)}
	\\[1mm]

	\vtmap{\chtype{L}}{1} \defeq	\lhot{(\btinp{\shot{\chtype{\tmap{L}{1}}}} \tinact)}
	\qquad
	\vtmap{\lhot{C}}{1} \defeq \lhot{\tmap{C}{1}}
	\qquad
	\vtmap{\shot{C}}{1} \defeq \shot{\tmap{C}{1}}
	\\[1mm]

	\tmap{\chtype{S}}{1} \defeq	\chtype{\tmap{S}{1}} 
	\qquad
	\tmap{\chtype{L}}{1} \defeq	\chtype{\tmap{L}{1}}
	\\[1mm]

	\tmap{\btout{U} S}{1} \defeq \btout{{\vtmap{U}{1}}} \tmap{S}{1}
	\qquad
	\tmap{\btinp{U} S}{1} \defeq \btinp{{\vtmap{U}{1}}} \tmap{S}{1}
	\\[1mm]

	\tmap{\btsel{l_i: S_i}_{i \in I}}{1} \defeq \btsel{l_i: \tmap{S_i}{1}}_{i \in I}
	\qquad
	\tmap{\btbra{l_i: S_i}_{i \in I}}{1} \defeq \btbra{l_i: \tmap{S_i}{1}}_{i \in I}
	\\[1mm]

	\tmap{\vart{t}}{1} \defeq \vart{t} \quad \tmap{\trec{t}{S}}{1}  \defeq \trec{t}{\tmap{S}{1}}
	\qquad 
	\tmap{\tinact}{1}  \defeq  \tinact
	\\[1mm]
%	\hline
%	\noindent{\bf Labels:} \quad \quad
%		\mapa{(\nu \tilde{m})\bactout{n}{m}}^{1} &\!\!\!\!\defeq\!\!\!\!&   
%(\nu \tilde{m})\bactout{n}{\abs{z}{\,\binp{z}{x} (\appl{x}{m})} } \\
%		\mapa{\bactinp{n}{m}}^{1} &\!\!\!\!\defeq\!\!\!\!&   \bactinp{n}{\abs{z}{\,\binp{z}{x} (\appl{x}{m})} } \\
%	\mapa{(\nu \tilde{m})\bactout{n}{\abs{{x}}{P}}}^{1} &\!\!\!\!\defeq\!\!\!\!& 
%(\nu \tilde{m})\bactout{n}{\abs{{x}}{\pmapp{P}{1}{\es}}}\\
%			\mapa{\bactinp{n}{\abs{{x}}{P}}}^{1} &\!\!\!\!\defeq\!\!\!\!& \bactinp{n}{\abs{{x}}{\pmapp{P}{1}{\es}}}\\
%			\mapa{\bactsel{n}{l} }^{1} \!\!\defeq\!\! \bactsel{n}{l} 
%	\quad 
%			\mapa{\bactbra{n}{l} }^{1} &\!\!\!\!\defeq\!\!\!\!& \bactbra{n}{l} 
%	\quad \quad 
%			\mapa{\tau}^{1} \!\!\defeq\!\! \tau
%\\[1mm]
%\hline
\end{array}
$

\noi{\bf Terms} :
\\
$
\begin{array}{rclcrcl}
	\pmapp{\bout{u}{w} P}{1}{f}	&\defeq&	\!\bout{u}{ \abs{z}{\,\binp{z}{x} (\appl{x}{w})} } \pmapp{P}{1}{f}
	&&
	\pmapp{\binp{u}{\AT{x}{C}} Q}{1}{f}	&\defeq&	\!\binp{u}{y} \newsp{s}{\appl{y}{s} \!\Par\! \bout{\dual{s}}{\abs{x}{\pmapp{Q}{1}{f}}} \inact}
	\\[1mm]

	\pmapp{\bout{u}{\abs{{x}}{Q}} P}{1}{f}  &\defeq& \bout{u}{\abs{{x}}{\pmapp{Q}{1}{f}}} \pmapp{P}{1}{f}
	&&
	\pmapp{\binp{u}{\AT{x}{L}} P}{1}{f} &\defeq& \!\binp{u}{x} \pmapp{P}{1}{f}
	\\[1mm]

	\pmapp{\bsel{s}{l} P}{1}{f} &\defeq& \bsel{s}{l} \pmapp{P}{1}{f}
	&&
	\pmapp{\bbra{s}{l_i{:}P_i}_{i \in I}}{1}{f} &\defeq& \!\bbra{s}{l_i: \pmapp{P_i}{1}{f}}_{i \in I}
	\\[1mm]

	\pmapp{\inact}{1}{f} &\defeq& \inact
	&&
	\pmapp{\news{n} P}{1}{f} &\defeq& \!\news{n} \pmapp{P}{1}{f}
	\\[1mm]

	\pmapp{{x}\, {u}}{1}{f} &\defeq& \appl{x}{u}
	& &
	\pmapp{\appl{(\abs{x}{Q})}{u}}{1}{f} &\defeq& \!\appl{(\abs{x}{\pmapp{Q}{1}{f}})}{u}
	\\[1mm]

	\pmapp{P \Par Q}{1}{f} & \!\!\defeq\!\! &\pmapp{P}{1}{f} \Par \pmapp{Q}{1}{f}
	\\
	\pmapp{\recp{X}{P}}{1}{f} &\defeq& 
	\multicolumn{5}{l}{
		\newsp{s}{\bout{\dual{s}}{\abs{(\vmap{\tilde{n}}, y)} \,{\binp{y}{z_\X} \auxmapp{\pmapp{P}{1}{{f,\{\rvar{X}\to \tilde{n}\}}}}{{}}{\es}}} \inact \Par  \binp{s}{z_\X} \pmapp{P}{1}{{f,\{\rvar{X}\to \tilde{n}\}}}}
	\quad
	(\tilde{n} = \fn{P})
	}
	\\ 
	\pmapp{\rvar{X}}{1}{f} & \defeq &
	\multicolumn{5}{l}{
		\newsp{s}{\appl{z_X}{(\tilde{n}, s)} \Par \bbout{\dual{s}}{ \abs{(\vmap{\tilde{n}},y)}{\appl{z_X}{(\vmap{\tilde{n}}, y)}}} \inact}  \quad (\tilde{n} = f(\rvar{X}))
	}
\end{array}
$
\\[1mm]
%
Above $\fn{P}$ denotes a lexicographically ordered sequence  of free names in $P$.
The input bound variable $x$ is annotated by a type to distinguish first- and higher-order cases.
%\vspace{-1mm}
\caption{\label{f:enc:hopi_to_ho}Encoding of \HOp into \HO (\defref{d:enc:hopitoho}).}
%\vspace{-1mm}
%(cf.~\defref{d:enc:fotohorec}).
%Mappings 
%$\map{\cdot}^2$,
%$\mapt{\cdot}^2$, 
%and 
%$\mapa{\cdot}^2$
%are homomorphisms for the other processes/types/labels. 
%\Hlinefig
\end{figure}



%\noi 
Note that $\Delta$ in $\varp{X}:\Delta$ is mapped to a non-tail
recursive session type with variable $z_X$. % (see \figref{f:enc:hopi_to_ho}).
Non-tail
recursive session types {were} studied in~\cite{DBLP:journals/corr/abs-1202-2086,TGC14};
{to our knowledge,}
this is the first application in the
context of higher-order session types.
%which carries type variable as the last argument.  
For simplicity,  % of the presentation, %we use the polyadic name abstraction and passing.
we use polyadic name abstractions.
A precise encoding of polyadicity into \HO is given in~\secref{sec:extension}.

{Key elements in 
\figref{f:enc:hopi_to_ho} are encodings of 
{\em name passing} ($\pmapp{\bout{u}{w} P}{1}{f}$ and $\pmapp{\binp{u}{x} P}{1}{f}$)  and  
{\em recursion} ($\pmapp{\recp{X}{P}}{1}{f}$ and $\pmapp{\rvar{X}}{1}{f}$).
As motivated in \secref{sec:overview}, % we encode passing of name $w$  
a name $w$ is passed as an input-guarded abstraction;
on the receiver side,
the encoding realises a mechanism that i) receives
the abstraction; ii) applies to it a fresh  endpoint $s$;
iii)~uses the dual endpoint $\dual{s}$ to send the continuation $P$ as an abstraction.
%$\abs{x}{P}$. 
Thus, name substitution is achieved via name application.
As for recursion, to encode $\recp{X}{P}$ we
first record a mapping from recursive variable $X$ to process variable $z_X$.
Then, using 
$\auxmapp{\cdot}{{}}{\sigma}$ in 
\defref{d:auxmap}, we encode the recursion body $P$ as a name abstraction
in which free names of $P$ are converted into name variables.
(Notice that $P$ is first encoded into \HO and then transformed using mapping
$\auxmapp{\cdot}{{}}{\sigma}$.)
Subsequently, this higher-order value is embedded in an input-guarded 
``duplicator'' process. We encode $X$ 
in such a way that it
simulates recursion unfolding by 
invoking the duplicator in a by-need fashion.
That is, upon reception, the \HO abstraction encoding  
%recursion body 
$P$
%containing $\auxmapp{P}{{}}{\sigma}$ 
is duplicated: 
one copy is used to reconstitute the original recursion body $P$ (through
the application of $\fn{P}$); another copy is used to re-invoke
the duplicator when needed. % to simulate recursion unfolding.
%An example of this typed encoding is detailed in~\cite{KouzapasPY15}.
We illustrate the encoding by means of an example.}
%\end{description}


 


%% !TEX root = main.tex
\begin{example}[The Encoding 
$\pmapp{\cdot}{1}{f}$ At Work]
Let $P = \recp{X}{\bout{a}{m} \varp{X}}$ be an \HOp process.
Its associated encoding into \HO is as follows---we note that initially $f = \emptyset$.
\begin{eqnarray*}
	\pmapp{P}{1}{f} &=&
	\newsp{s_1}{ \binp{s_1}{x} \pmapp{\bout{a}{m} \varp{X}}{1}{{f'}} \Par \bout{\dual{s_1}}{ \abs{(x_a, x_m, z)} \binp{z}{x} \auxmapp{\pmapp{\bout{a}{m} \varp{X}}{1}{{f'}}}{{}}{\es} } \inact} \\
%	&&\bout{\dual{s_1}}{ \abs{(x_a, x_m, z)} \binp{z}{x} \auxmapp{\pmapp{\bout{a}{m} \varp{X}}{1}{{\varp{X} \rightarrow x_ax_m}}}{{}}{\es} } \inact}
%\end{eqnarray*}
%\begin{eqnarray*}	
\pmapp{\bout{a}{m} \varp{X}}{1}{{ f'}} &=&
	\bout{a}{\abs{z}{\binp{z}{x} (\appl{x}{m})}} \pmapp{\varp{X}}{1}{{f'}}
	\\
	&=& \bout{a}{\abs{z}{\binp{z}{x} (\appl{x}{m})}} \newsp{s_2}{\appl{x}{(a,m, s_2)}  \Par \bout{\dual{s_2}}{\abs{(x_a, x_m, z)}{\appl{x}{(x_a, x_m, z)}}} \inact} \\
	\auxmapp{\pmapp{\bout{a}{m} \varp{X}}{1}{{f'}}}{{}}{\es}
	  & = & \auxmapp{\bout{a}{\abs{z}{\binp{z}{x} (\appl{x}{m})}} \newsp{s_2}{\appl{x}{(a,m, s_2)}  \Par \bout{\dual{s_2}}{\abs{(x_a, x_m, z)}{\appl{x}{(x_a, x_m, z)}}} \inact}}{{}}{\es}
	\\
	 & = & \bout{x_a}{\abs{z}{\binp{z}{x} (\appl{x}{x_m})}} \auxmapp{\newsp{s_2}{\appl{x}{(a,m, s_2)}  \Par \bout{\dual{s_2}}{\abs{(x_a, x_m, z)}{\appl{x}{(x_a, x_m, z)}}} \inact}}{{}}{\es}
	\\
	& = & \bout{x_a}{\abs{z}{\binp{z}{x} (\appl{x}{x_m})}} \newsp{s_2}{\appl{x}{(x_a,x_m, s_2)}  \Par \bout{\dual{s_2}}{\abs{(x_a, x_m, z)}{\appl{x}{(x_a, x_m, z)}}} \inact}
\end{eqnarray*}
where $f' = \varp{X} \rightarrow x_ax_m$.
That is, by writing $V$ to denote the process
$$
\abs{(x_a, x_m, z)} \binp{z}{x} \bout{x_a}{\abs{z}{\binp{z}{x} (\appl{x}{x_m})}} \newsp{s_2}{\appl{x}{(x_a,x_m, s_2)}  \Par \bout{\dual{s_2}}{\abs{(x_a, x_m, z)}{\appl{x}{(x_a, x_m, z)}}} \inact}
$$
we would have that $P = \recp{X}{\bout{a}{m} \varp{X}}$ is mapped into the \HO process
$$
\newsp{s_1}{\binp{s_1}{x}  \bout{a}{\abs{z}{\binp{z}{x} (\appl{x}{m})}} \newsp{s_2}{\appl{x}{(a,m, s_2)}  \Par \bout{\dual{s_2}}{\abs{(x_a, x_m, z)}{\appl{x}{(x_a, x_m, z)}}} \inact}\Par \bout{\dual{s_1}}{V} \inact}
$$
We illustrate the behaviour of the encoded process (below, we let $\lambda = \bactout{a}{\abs{z}{\binp{z}{x} (\appl{x}{m})}}$):
\begin{eqnarray*}
\pmapp{P}{1}{f} & \scong & \newsp{s_1}{\bout{\dual{s_1}}{V} \inact \Par \binp{s_1}{x} \bout{a}{\abs{z}{\binp{z}{x} (\appl{x}{m})}} \newsp{s_2}{\bout{\dual{s_2}}{\abs{(x_a, x_m, z)}{\appl{x}{(x_a, x_m, z)}}} \inact}  \\
& & \qquad \Par \appl{x}{(a,m, s_2)}} \\
& \by{\tau} & \bout{a}{\abs{z}{\binp{z}{x} (\appl{x}{m})}} \newsp{s_2}{\bout{\dual{s_2}}{V} \inact \Par \binp{s_2}{x} \bout{a}{\abs{z}{\binp{z}{x} (\appl{x}{m})}} \\
& & \qquad \qquad \quad \qquad \qquad \qquad \newsp{s_3}{\bout{\dual{s_3}}{\abs{(x_a, x_m, z)}{\appl{x}{(x_a, x_m, z)}}} \inact} \Par \appl{x}{(a,m, s_3)}} \\
& \scong_{\alpha} & \bout{a}{\abs{z}{\binp{z}{x} (\appl{x}{m})}} \newsp{s_1}{\bout{\dual{s_1}}{V} \inact \Par \binp{s_1}{x} \bout{a}{\abs{z}{\binp{z}{x} (\appl{x}{m})}} \\
& & \qquad \qquad \qquad \qquad \quad \qquad \newsp{s_2}{\bout{\dual{s_2}}{\abs{(x_a, x_m, z)}{\appl{x}{(x_a, x_m, z)}}} \inact} \Par \appl{x}{(a,m, s_2)}} \\
& \scong & 
		\bout{a}{\abs{z}{\binp{z}{x} (\appl{x}{m})}} \pmapp{\recp{X}{\bout{a}{m} \varp{X}}}{1}{f} \by{\ell} 
		\pmapp{\recp{X}{\bout{a}{m} \varp{X}}}{1}{f}
%& \by{\lambda} & 
%		\pmapp{\recp{X}{\bout{a}{m} \varp{X}}}{1}{f}
\end{eqnarray*}
where $\ell$ stands for an output action.
The encoding preserves also typing; associated derivations are given in \cite{KouzapasPY15}.
\qed
\end{example}


\begin{example}[The Encoding 
$\pmapp{\cdot}{1}{f}$ At Work]
Let $P = \recp{X}{\bout{a}{m} \varp{X}}$ be an \HOp process.
Its encoding into \HO is given next; notice that $f = \emptyset$ and $f' = \varp{X} \rightarrow x_ax_m$.
\begin{eqnarray*}
	\pmapp{P}{1}{f} &=&
	\newsp{s_1}{ \binp{s_1}{x} \pmapp{\bout{a}{m} \varp{X}}{1}{{f'}} \Par \bout{\dual{s_1}}{ \abs{(x_a, x_m, z)} \binp{z}{x} \auxmapp{\pmapp{\bout{a}{m} \varp{X}}{1}{{f'}}}{{}}{\es} } \inact} \\
%	&&\bout{\dual{s_1}}{ \abs{(x_a, x_m, z)} \binp{z}{x} \auxmapp{\pmapp{\bout{a}{m} \varp{X}}{1}{{\varp{X} \rightarrow x_ax_m}}}{{}}{\es} } \inact}
%\end{eqnarray*}
%\begin{eqnarray*}	
\pmapp{\bout{a}{m} \varp{X}}{1}{{ f'}} &=&
%	\bout{a}{\abs{z}{\binp{z}{x} (\appl{x}{m})}} \pmapp{\varp{X}}{1}{{f'}}
%	\\
%	&=& 
	\bout{a}{\abs{z}{\binp{z}{x} (\appl{x}{m})}} \newsp{s_2}{\appl{x}{(a,m, s_2)}  \Par \bout{\dual{s_2}}{\abs{(x_a, x_m, z)}{\appl{x}{(x_a, x_m, z)}}} \inact} \\
	\auxmapp{\pmapp{\bout{a}{m} \varp{X}}{1}{{f'}}}{{}}{\es}
	  & = & 
%	  \auxmapp{\bout{a}{\abs{z}{\binp{z}{x} (\appl{x}{m})}} \newsp{s_2}{\appl{x}{(a,m, s_2)}  \Par \bout{\dual{s_2}}{\abs{(x_a, x_m, z)}{\appl{x}{(x_a, x_m, z)}}} \inact}}{{}}{\es}
%	\\
%	 & = & 
%	 \bout{x_a}{\abs{z}{\binp{z}{x} (\appl{x}{x_m})}} \auxmapp{\newsp{s_2}{\appl{x}{(a,m, s_2)}  \Par \bout{\dual{s_2}}{\abs{(x_a, x_m, z)}{\appl{x}{(x_a, x_m, z)}}} \inact}}{{}}{\es}
%	\\
%	& = & 
	\bout{x_a}{\abs{z}{\binp{z}{x} (\appl{x}{x_m})}} \newsp{s_2}{\appl{x}{(x_a,x_m, s_2)}  \Par \\
	& & \qquad \qquad \qquad \qquad \qquad \qquad \qquad \bout{\dual{s_2}}{\abs{(x_a, x_m, z)}{\appl{x}{(x_a, x_m, z)}}} \inact}
\end{eqnarray*}
That is, by writing $V$ to denote the process
$$
\abs{(x_a, x_m, z)} \binp{z}{x} \bout{x_a}{\abs{z}{\binp{z}{x} (\appl{x}{x_m})}} \newsp{s_2}{\appl{x}{(x_a,x_m, s_2)}  \!\Par\! \bout{\dual{s_2}}{\abs{(x_a, x_m, z)}{\appl{x}{(x_a, x_m, z)}}} \inact}
$$
we would have %that $P = \recp{X}{\bout{a}{m} \varp{X}}$ is mapped into the \HO process
\begin{eqnarray*}
\pmapp{P}{1}{f} & = & \newsp{s_1}{\binp{s_1}{x}  \bout{a}{\abs{z}{\binp{z}{x} (\appl{x}{m})}} \newsp{s_2}{\appl{x}{(a,m, s_2)}  \Par \\
& & \qquad \qquad \qquad \bout{\dual{s_2}}{\abs{(x_a, x_m, z)}{\appl{x}{(x_a, x_m, z)}}} \inact}\Par \bout{\dual{s_1}}{V} \inact}
\end{eqnarray*}
Next we illustrate the behaviour of $\pmapp{P}{1}{f}$; below $\ell$ stands for $\bactout{a}{\abs{z}{\binp{z}{x} (\appl{x}{m})}}$.
\begin{eqnarray*}
\pmapp{P}{1}{f} & \scong & \newsp{s_1}{\bout{\dual{s_1}}{V} \inact \Par \binp{s_1}{x} \bout{a}{\abs{z}{\binp{z}{x} (\appl{x}{m})}} \newsp{s_2}{\bout{\dual{s_2}}{\abs{(x_a, x_m, z)}{\\
& & \qquad \qquad \qquad \qquad \quad \quad  \appl{x}{(x_a, x_m, z)}}} \inact} 
\Par \appl{x}{(a,m, s_2)}} \\
& \by{\tau} & \bout{a}{\abs{z}{\binp{z}{x} (\appl{x}{m})}} \newsp{s_2}{\bout{\dual{s_2}}{V} \inact \Par \binp{s_2}{x} \bout{a}{\abs{z}{\binp{z}{x} (\appl{x}{m})}} \\
& & \qquad \qquad \quad \qquad \qquad \quad \newsp{s_3}{\bout{\dual{s_3}}{\abs{(x_a, x_m, z)}{\appl{x}{(x_a, x_m, z)}}} \inact} \Par \appl{x}{(a,m, s_3)}} \\
& \scong_{\alpha} & \bout{a}{\abs{z}{\binp{z}{x} (\appl{x}{m})}} \newsp{s_1}{\bout{\dual{s_1}}{V} \inact \Par \binp{s_1}{x} \bout{a}{\abs{z}{\binp{z}{x} (\appl{x}{m})}} \\
& & \qquad \qquad \qquad \qquad \quad \quad \newsp{s_2}{\bout{\dual{s_2}}{\abs{(x_a, x_m, z)}{\appl{x}{(x_a, x_m, z)}}} \inact} \Par \appl{x}{(a,m, s_2)}} \\
& \scong & 
		\bout{a}{\abs{z}{\binp{z}{x} (\appl{x}{m})}} \pmapp{\recp{X}{\bout{a}{m} \varp{X}}}{1}{f} \by{\ell} 
		\pmapp{\recp{X}{\bout{a}{m} \varp{X}}}{1}{f}.
%& \by{\lambda} & 
%		\pmapp{\recp{X}{\bout{a}{m} \varp{X}}}{1}{f}
\end{eqnarray*}
%where $\ell = \bactout{a}{\abs{z}{\binp{z}{x} (\appl{x}{m})}}$ is an output action.
%For type preservation/soundness see~\cite{KouzapasPY15}.
%\qed
\end{example}


We now describe the properties of the encoding. 
{Directly from \figref{f:enc:hopi_to_ho} we may state:
\begin{proposition}[\HOp into \HO: Type Preservation]
The encoding from $\tyl{L}_{\HOp}$ into $\tyl{L}_{\HO}$ (cf.~\defref{d:enc:hopitoho})
is type preserving.
\end{proposition}}

Now, we state operational correspondence with respect to reductions; 
the full statement (and proof) can be found in~\cite{KouzapasPY15}.
%Recall that $\hby{\stau}$ and $\hby{\btau}$ were defined in \notref{not:dettrans}.

\begin{proposition}[\HOp into \HO: Operational Correspondence - Excerpt]%\myrm
	\label{prop:op_corr_HOp_to_HO}
	Let $P$ be an \HOp process such that $\Gamma; \emptyset; \Delta \proves P \hastype \Proc$.
	\begin{enumerate}[1.]
		\item Completeness: 
			Suppose $\horel{\Gamma}{\Delta}{P}{\hby{\tau}}{\Delta'}{P'}$. Then we have:
%
			\begin{enumerate}[a)]
				\item
					If  $P' \scong \newsp{\tilde{m}}{P_1 \Par P_2\subst{m}{x}}$
					then $\exists R$ s.t. \\
					$\horel{\tmap{\Gamma}{1}}{\tmap{\Delta}{1}}{\pmapp{P}{1}{f}}{\hby{\tau}}{\mapt{\Delta}^{1}}{\newsp{\tilde{m}}{\pmapp{P_1}{1}{f} \Par R}}$,
					and\\ 
					$\horel{\tmap{\Gamma}{1}}{\tmap{\Delta}{1}}{\newsp{\tilde{m}}{\pmapp{P_1}{1}{f} \Par R}}{\hby{\stau} \hby{\btau} \hby{\btau}}
					{\mapt{\Delta}^{1}}{\newsp{\tilde{m}}{\pmapp{P_1}{1}{f} \Par \pmapp{P_2}{1}{f}\subst{m}{x}}}$.
			
				\item
					If  $P' \scong \newsp{\tilde{m}}{P_1 \Par P_2 \subst{\abs{y}Q}{x}}$
					then \\
					$\horel{\tmap{\Gamma}{1}}{\tmap{\Delta}{1}}{\pmapp{P}{1}{f}}{\hby{\tau}}
					{\tmap{\Delta_1}{1}}{\newsp{\tilde{m}}{\pmapp{P_1}{1}{f}\Par \pmapp{P_2}{1}{f}\subst{\abs{y}\pmapp{Q}{1}{\emptyset}}{x}}}$.
			
				\item
					If   $P' \not\scong \newsp{\tilde{m}}{P_1 \Par P_2 \subst{m}{x}} \land P' \not\scong \newsp{\tilde{m}}{P_1 \Par P_2\subst{\abs{y}Q}{x}}$
					then \\
					$\horel{\tmap{\Gamma}{1}}{\tmap{\Delta}{1}}{\pmapp{P}{1}{f}}{\hby{\tau}}{\tmap{\Delta'_1}{1}}{ \pmapp{P'}{1}{f}}$.
			\end{enumerate}
			
		\item Soundness:	Suppose $\horel{\tmap{\Gamma}{1}}{\tmap{\Delta}{1}}{\pmapp{P}{1}{f}}{\hby{\tau}}{\tmap{\Delta'}{1}}{Q}$.
			Then $\Delta' = \Delta$ and 
					either
%
					\begin{enumerate}[a)]
						\item	$\exists P'$ s.t. 
							$\horel{\Gamma}{\Delta}{P}{\hby{\tau}}{\Delta}{P'}$,
							and $Q = \map{P'}^{1}_f$.	

						\item
							$\exists P_1, P_2, x, m, Q'$ s.t. 
							$\horel{\Gamma}{\Delta}{P}{\hby{\tau}}{\Delta}{\newsp{\tilde{m}}{P_1 \Par P_2\subst{m}{x}} }$, and\\
							$\horel{\tmap{\Gamma}{1}}{\tmap{\Delta}{1}}{Q}{\hby{\stau} \hby{\btau} \hby{\btau}}{\tmap{\Delta}{1}}{\pmapp{P_1}{1}{f} \Par \pmapp{P_2\subst{m}{x}}{1}{f}}$ 
%							$Q = \map{P_1}^{1}_f \Par Q'$, where $Q'  \Hby{} $.

%						\item $\exists P_1, P_2, x, R$ s.t. 
%						$\stytra{ \Gamma }{\tau}{ \Delta }{ P}{ \Delta}{ \news{\tilde{m}}(P_1 \Par P_2\subst{\abs{y}R}{x}) }$, and 
%						$Q = \map{\news{\tilde{m}}(P_1 \Par P_2\subst{\abs{y}R}{x})}^{1}_f$.
			\end{enumerate}
		    %\end{enumerate}
		    
%		\item   
%			If  $\wtytra{\mapt{\Gamma}^{1}}{\ell_2}{\mapt{\Delta}^{1}}{\pmapp{P}{1}{f}}{\mapt{\Delta'}^{1}}{Q}$
%			then $\exists \ell_1, P'$ s.t.  \\
%			(i)~$\stytra{\Gamma}{\ell_1}{\Delta}{P}{\Delta'}{P'}$,
%			(ii)~$\ell_2 = \mapa{\ell_1}^{1}$, 
%			(iii)~$\wbb{\mapt{\Gamma}^{1}}{\ell}{\mapt{\Delta'}^{1}}{\pmapp{P'}{1}{f}}{\mapt{\Delta'}^{1}}{Q}$.
	\end{enumerate}
\end{proposition}

%\noi 
Observe how we can explicitly distinguish the role of finite, deterministic reductions 
($\hby{\stau}$ and $\hby{\btau}$, defined in \notref{not:dettrans}) in both soundness and completeness statements.

%Using operational correspondence, we can show \emph{full abstraction}:
\newj{The typed operational correspondence given above is an important component in the 
proof of \emph{full abstraction}, which we state next.}
\begin{proposition}[\HOp into \HO: Full Abstraction]%\myrm
	\label{prop:fulla_HOp_to_HO}
	Let $P_1, Q_1$ be \HOp processes. \\
	$\horel{\Gamma}{\Delta_1}{P_1}{\hwb}{\Delta_2}{Q_1}$
	if and only if
	$\horel{\tmap{\Gamma}{1}}{\tmap{\Delta_1}{1}}{\pmapp{P_1}{1}{f}}{\hwb}{\tmap{\Delta_2}{1}}{\pmapp{Q_1}{1}{f}}$.
\end{proposition}




%Based on these propositions, w
We may state the main result of this section. See~\cite{KouzapasPY15} for details. 

\begin{theorem}[Precise Encoding of \HOp into \HO]
\label{f:enc:hopitoho}
The encoding from $\tyl{L}_{\HOp}$ into $\tyl{L}_{\HO}$ (cf.~\defref{d:enc:hopitoho})
is precise. 
\end{theorem}


\subsection{From \HOp to \sessp}
\label{subsec:HOp_to_sessp}
\newj{We now discuss the precise encodability of  $\HOp$ into $\sessp$;
the non trivial issue is encoding higher-order communication, which is present in $\HOp$ but not in 
$\sessp$.}
We closely follow Sangiorgi's encoding~\cite{San92,SaWabook}, which represents 
%Intuitively, such an encoding  represents 
the exchange of a process with the exchange of a fresh \emph{trigger name}. 
Trigger names may then be used to activate copies of the process, which becomes a persistent resource represented by an input-guarded replication.
%Consider the following (naive) adaptation of \cite{San92,SaWabook} 
%in which session names are used are triggers and 
%exchanged processes would be have to used exactly once:
%%
%\[
%\begin{array}{l}
%		\pmap{\bout{u}{\abs{x}{Q}} P}{n}  \defeq   \newsp{s}{\bout{u}{s} (\pmap{P}{n} \Par \binp{\dual{s}}{x} \pmap{Q}{n})} \\
%		\pmap{\binp{u}{x} P}{n}  \defeq \binp{u}{x} \pmap{P}{n}
%		\quad 
%		\pmap{\appl{x}{u}}{n}  \defeq  \bout{x}{u} \inact
%	\end{array}
%\]
%%
%with the remaining \HOp constructs being mapped homomorphically.
%Although $\pmap{\cdot}{n}$ captures the correct semantics when
%dealing with systems that allow only linear abstractions,
%it suffers from untypability in the presence
%of shared abstractions. For instance,
%mapping for $P = \bout{n}{\abs{x}{\bout{x}{m}\inact}} \inact \Par \binp{\dual{n}}{x} (\appl{x}{s_1} \Par \appl{x}{s_2})$
%would be:
%%
%\[
%	\pmap{P}{n} \defeq
%	\newsp{s}{\bout{n}{s} \binp{\dual{s}}{x} \bout{x}{m} \inact \Par \binp{\dual{n}}{x} (\bout{x}{s_1} \inact \Par \bout{x}{s_2} \inact)}
%\]
%%
%The above process is untypable since processes $(\bout{x}{s_1} \inact$ and $\bout{x}{s_2} \inact)$
%cannot be put in parallel because they do not have disjoint session environments.
We cast this strategy in the setting of session-typed communications. 
In the presence of session names (which are linear  and cannot be replicated),
%The correct 
our
approach %would be to 
 uses replicated names
as triggers for shared resources and non-replicated names
for linear resources (cf. $\pmap{\bout{u}{\abs{x}{Q}} P}{2}$).
%as triggers instead of session names, when dealing with shared abstractions. 

%\smallskip 

\begin{definition}[Typed Encoding of \HOp into \sessp]
\label{d:enc:hopitopi}
%Let $\tyl{L}_{\sessp}=\calc{\sessp}{{\cal{T}}_3}{\hby{\ell}}{\fwb}{\proves}$ 
%where the typing is defined in 
%\figref{fig:typerulesmy} 
%and the equivalence $\fwb$ is defined in \defref{d:fwb}.
%${\cal{T}}_3$ is a set of types of $\sessp$.  
%%
The typed encoding 
$\enco{\map{\cdot}^{2}, \mapt{\cdot}^{2} %, \mapa{\cdot}^{2}
}: \tyl{L}_{\HOp} \to \tyl{L}_{\sessp}$  
%We define the mappings $\map{\cdot}^{2}$, $\mapt{\cdot}^{2}$, $\mapa{\cdot}^{2}$
is defined
in \figref{f:enc:ho_to_sessp}. 
\end{definition}

%\smallskip 
\begin{figure}[t]
{\bf Types:}

$
	\begin{array}{c}
		\tmap{\btout{\lhot{S}}S_1}{2} \defeq \bbtout{\chtype{\btinp{\tmap{S}{2}}\tinact}}\tmap{S_1}{2}
		\qquad
		\quad
		\tmap{\btinp{\lhot{S}}S_1}{2} \defeq \bbtinp{\chtype{\btinp{\tmap{S}{2}}\tinact}}\tmap{S_1}{2}
%\noindent{\bf Labels:}\ 
%		\mapa{(\nu \tilde{m})\bactout{n}{\abs{ x}{P}} }^2  & \defeq & \news{m} \bactout{n}{m} \\
%		\mapa{\bactinp{n}{\abs{ x}{P}} }^2 &  \defeq & \bactinp{n}{m}
%\quad \quad m \text{ fresh}
%\\[1mm]
%\hline
	\end{array}
$
\\[2mm]
{\bf Terms} :
\\
$
\begin{array}{rcll}
	\pmap{\bout{u}{\abs{x}{Q}} P}{2} & \defeq  &
	\left\{
	\begin{array}{r}
		\newsp{a}{\bout{u}{a} (\pmap{P}{2} \Par \repl{} \binp{a}{y} \binp{y}{x} \pmap{Q}{2})\,}\quad
		(s \notin \fn{Q})
		\\
		\newsp{a}{\bout{u}{a} (\pmap{P}{2} \Par \binp{a}{y} \binp{y}{x} \pmap{Q}{2})\,}\quad
		\textrm{(otherwise)} %\dk{Q \textrm{ linear}} \\
	\end{array}
	\right.
	\\[1mm]

	\pmap{\binp{u}{x} P}{2} &\defeq&  \binp{u}{x} \pmap{P}{2}
	%\quad \quad \pmap{\appl{(\abs{x}{P})}{u}}{2} \!\! \defeq \!\! \pmap{P\subst{u}{x}}{2}
	\qquad
	\pmap{\appl{x}{u}}{2} \defeq \newsp{s}{\bout{x}{s} \bout{\dual{s}}{u} \inact}
	\\[1mm]

	\pmap{\appl{(\abs{x}{P})}{u}}{2} & \defeq & %\newsp{s}{\bout{a}{s} \bout{\dual{s}}{u} \inact} \\
	\newsp{s}{\binp{s}{x} \pmap{P}{2} \Par \bout{\dual{s}}{u} \inact}
\end{array}
$
\\[2mm]
{Notice: $\repl{} P$ means $\recp{X}{(P \Par \rvar{X})}$. Elided mappings are homomorphic.}
%for others.  %types, labels and processes    
%\vspace{-1mm}
\caption{Encoding of \HOp into \sessp (\defref{d:enc:hopitopi}). \label{f:enc:ho_to_sessp}}
%\vspace{-1mm}
%\Hlinefig
\end{figure}


%\noi 
%Notice that $\mapa{\bactinp{n}{\abs{ x}{P}} }^2$ involves a fresh trigger name (linear or shared),  which denotes the location of $\pmap{P}{2}$. 
%(a $\sessp$ process).
Observe how $\pmap{\appl{(\abs{x}{P})}{u}}{2}$ naturally induces a name substitution.
We describe key properties of this encoding. First, type preservation and operational correspondence:

\begin{proposition}[\HOp into \sessp: Type Preservation]
The encoding from $\tyl{L}_{\HOp}$ into $\tyl{L}_{\sessp}$ (cf.~\defref{d:enc:hopitopi})
is type preserving.
\end{proposition}

\begin{proposition}[\HOp into \sessp: Operational Correspondence - Excerpt]%\myrm
	\label{prop:op_corr_HOp_to_p}
	Let $P$ be an  $\HOp$ process such that  $\Gamma; \emptyset; \Delta \proves P \hastype \Proc$.
	
\begin{enumerate}[1.]
\item Completeness: Suppose $\horel{\Gamma}{\Delta}{P}{\hby{\ell}}{\Delta'}{P'}$. Then either:
				\begin{enumerate}[a)]
				\item If $\ell = \tau$ then one of the following holds:
				\begin{enumerate}[-]
					\item	 %such that
						$
						\horel{\tmap{\Gamma}{2}}{\tmap{\Delta}{2}}{\pmap{P}{2}}
						{\hby{\tau}} \\
						{\tmap{\Delta'}{2}}{}{\newsp{\tilde{m}}{\pmap{P_1}{2} \!\Par\! \newsp{a}
						{\pmap{P_2}{2}\subst{a}{x} \!\Par\!\! \repl{} \binp{a}{y} \binp{y}{x} \pmap{Q}{2}}}}
						$, for some  $P_1, P_2, Q$;

					\item	%$\exists R$ such that
						$
						\horel{\tmap{\Gamma}{2}}{\tmap{\Delta}{2}}{\pmap{P}{2}}
						{\hby{\tau}}
						{\tmap{\Delta'}{2}}{}{\newsp{\tilde{m}}{\pmap{P_1}{2} \Par \newsp{s}
						{\pmap{P_2}{2}\subst{\dual{s}}{x} \!\Par\! \binp{s}{y} \binp{y}{x} \pmap{Q}{2}}}}
						$, for some  $P_1, P_2, Q$;

					\item	%$\ell_1 = \btau$ and
						$\horel{\tmap{\Gamma}{2}}{\tmap{\Delta}{2}}{\pmap{P}{2}}
						{\hby{\tau}}
						{\tmap{\Delta'}{2}}{}{{\pmap{P'}{2} }}
						$

				\end{enumerate}
				\item 	If $\ell = \btau$ then 
						$\horel{\tmap{\Gamma}{2}}{\tmap{\Delta}{2}}{\pmap{P}{2}}
						{\hby{\stau}}
						{\tmap{\Delta'}{2}}{}{{\pmap{P'}{2} }}
						$.
				\end{enumerate}
		
		%%%%%%% SOUNDNESSS
		\item Suppose 
		$\stytra{\mapt{\Gamma}^{2}}{\tau}{\mapt{\Delta}^{2}}{\map{P}^{2}}{\mapt{\Delta'}^{2}}{R}$.  \\
		Then $\exists P'$ such that
					$P \hby{\tau} P'$
					and $\horel{\mapt{\Gamma}^{2}}{\mapt{\Delta'}^{2}}{\map{P'}^{2}}{\hwb}{\mapt{\Delta'}^{2}}{R}$.
	\end{enumerate}
\end{proposition}

\newj{Exploiting the above properties (type preservation, typed operational correspondence), 
we can show that our typed encoding is fully abstract  and precise. }

\begin{proposition}[\HOp to \sessp: Full Abstraction]%\myrm
	\label{prop:fulla_HOp_to_p}
	Let $P_1, Q_1$ be \HOp processes.
	$\horel{\Gamma}{\Delta_1}{P_1}{\hwb}{\Delta_2}{Q_1}$
	if and only if
	$\horel{\tmap{\Gamma}{2}}{\tmap{\Delta_1}{2}}{\pmap{P_1}{2}}{\fwb}{\tmap{\Delta_2}{2}}{\pmap{Q_1}{2}}$.
\end{proposition}




\begin{theorem}[Precise Encoding of \HOp into \sessp]
\label{f:enc:hotopi}
The encoding from $\tyl{L}_{\HOp}$ into $\tyl{L}_{\sessp}$ (cf.~\defref{d:enc:hopitopi})
is precise. 
\end{theorem}

%\smallskip 
%
%\begin{remark}
%As stated in  \cite[Lem.\,5.2.2]{SangiorgiD:expmpa}, 
%due to the replicated trigger,  
%operational correspondence in \defref{def:ep} is refined to prove  
%full abstraction: 
%e.g., completeness of the case $\ell_1 \neq \tau$, is changed as follows.
%Suppose   
%$\stytraarg{\Gamma}{\ell_1}{\Delta}{P}{\Delta'}{P'}{}$:
%if $\ell_1 = (\nu \tilde{m})\bactout{n}{\abs{ x}{R}}$, 
%then %$\exists \ell_2, Q$ s.t. 
%$\stytraarg{\mapt{\Gamma}^2}{\ell_2}{\mapt{\Delta}^2}{\map{P}^2}{\mapt{\Delta'}^2}{Q}{}$,
%where 
%$\ell_2 = (\nu a)\bactout{n}{a}$ and
%$Q = \pmap{P' \Par  \repl{} \binp{a}{y} \binp{y}{x} R}{2}$.
%Similarly,
%if  
%%$\stytraarg{\Gamma}{\ell_1}{\Delta}{P}{\Delta'}{P'}{}$
%%with 
%$\ell_1 = \bactinp{n}{\abs{ x}{R}}$, 
%then %$\exists \ell_2, Q$ s.t. 
%$\stytraarg{\mapt{\Gamma}^2}{\ell_2}{\mapt{\Delta}^2}{\map{P}^2}{\mapt{\Delta'}^2}{Q}{}$,
%where 
%$\ell_2 = \bactout{n}{a}$ and
%$\pmap{P'}{2} \wb \news{a}(Q \Par  \repl{} \binp{a}{y} \binp{y}{x} \pmap{R}{2})$.
%Soundness is stated in a symmetric way; see \cite{KouzapasPY15}. 
%%Operational correspondence for the encoding in~\defref{d:enc:hopitopi}
%%is different from that in~\defref{def:ep}, due to triggers. 
%%In particular,  completeness differs when $\ell_1 \neq \tau$.
%%This way, e.g., if  
%%$\stytraarg{\Gamma}{\ell_1}{\Delta}{P}{\Delta'}{P'}{}$
%%with $\ell_1 = (\nu \tilde{m})\bactout{n}{\abs{ x}{R}}$, 
%%then %$\exists \ell_2, Q$ s.t. 
%%$\stytraarg{\mapt{\Gamma}^2}{\ell_2}{\mapt{\Delta}^2}{\map{P}^2}{\mapt{\Delta'}^2}{Q}{}$,
%%where 
%%$\ell_2 = (\nu a)\bactout{n}{a}$ and
%%$Q = \pmap{P' \Par  \repl{} \binp{a}{y} \binp{y}{x} R}{2}$.
%%This 
%%statement, essential in proofs of full abstraction,
%%is the same given by Sangiorgi~\cite{SangiorgiD:expmpa}.
%%Completeness is as in~\defref{def:ep} when  $\ell_1 = \tau$.
%%See~\cite{KouzapasPY15} for details.
%\end{remark}


%% !TEX root = main.tex

\subsection{Discussion: Comparing Encodings}
The precise encodings reported in  \secref{subsec:HOpi_to_HO} and \secref{subsec:HOp_to_sessp}
confirm that \HO and \sessp constitute two important sources of expressiveness in \HOp.
This naturally begs the question: which of the two sub-calculi is more tightly related to \HOp?
In this section, we compare our two encodings by contrasting the way in which 
\begin{enumerate}[a)]
\item the encoding from \HOp to \HO (\secref{subsec:HOpi_to_HO}) translates processes with name passing;
\item the encoding from \HOp to \sessp (\secref{subsec:HOp_to_sessp} translates processes with abstraction passing.
\end{enumerate}
Consider the \HOp processes:
\begin{eqnarray*}
P_1 & = & \bout{s}{a} \inact \Par \binp{\dual{s}}{x} (\bout{x}{s_1} \inact \Par \dots \Par \bout{x}{s_n} \inact) \\
P_2 & = & \bout{s}{\abs{x}{P}} \inact \Par \binp{\dual{s}}{x} (\appl{x}{s_1} \Par \dots \Par \appl{x}{s_n})
\end{eqnarray*}
\noi Observe that $P_1$ features \emph{pure} name passing (no abstraction-passing), whereas 
$P_2$ involves \emph{pure} abstraction passing (no name passing). In both cases, 
the intended communication on $s$ leads to $n$ usages of the communication object (name $a$ in $P_1$ and abstraction $\abs{x}{P}$ in $P_2$).
Consider now the reduction steps from $P_1$ and $P_2$:
\begin{eqnarray*}
P_1 & \hby{\tau} & \bout{a}{s_1} \inact \Par \dots \Par \bout{a}{s_n} \inact \\
P_2 & \hby{\tau}& \appl{(\abs{x}{P})}{s_1} \Par \dots \Par \appl{(\abs{x}{P})}{s_n} \quad \underbrace{\hby{\btau} \cdots \hby{\btau}}_{n} \quad P \subst{s_1}{x} \Par \dots \Par P \subst{s_1}{x} 
\end{eqnarray*}

%Let reduction on \sessp process:
%\begin{eqnarray*}
%	\bout{s}{a} \inact \Par \binp{\dual{s}}{x} (\bout{x}{s_1} \inact \Par \dots \Par \bout{x}{s_n} \inact)
%	\hby{\tau}
%	\bout{a}{s_1} \inact \Par \dots \Par \bout{a}{s_n} \inact
%\end{eqnarray*}
%and \HO process
%\begin{eqnarray*}
%	\bout{s}{\abs{x}{P}} \inact \Par \binp{\dual{s}}{x} (\appl{x}{s_1} \Par \dots \Par \appl{x}{s_n})
%	&\hby{\tau}&
%	\appl{(\abs{x}{P})}{s_1} \Par \dots \Par \appl{(\abs{x}{P})}{s_n}\\
%	&\Hby{\tau}_{n}&
%	P \subst{s_1}{x} \Par \dots \Par P \subst{s_1}{x}
%\end{eqnarray*}
\noi 
$P_1$ and $P_2$ essentially follow the same communication pattern; they both
reduce on a message passing action, with the
message being substituted $n$ times on the receing side.
Both $P_1$ and $P_2$ are \HOp processes.
If we consider the encodings of $P_1$ into \HO and of $P_2$ into \sessp,  
we obtain:
\[
\begin{array}{l}
	\bout{s}{\abs{z}{\binp{z}{y} \appl{y}{a}}} \inact \Par \binp{\dual{s}}{x} \newsp{t}{\appl{x}{t} \Par \bout{\dual{t}}{\abs{x}{(\bout{x}{\abs{z}{\binp{z}{y} \appl{y}{s_1}}} \inact \Par \dots \Par \bout{x}{\abs{z}{\binp{z}{y} \appl{y}{s_n}}} \inact)}} \inact}\\
	\hby{\stau}\\
	\newsp{t}{\appl{(\abs{z}{\binp{z}{y} \appl{y}{a}})}{t} \Par \bout{\dual{t}}{\abs{x}{(\bout{x}{\abs{z}{\binp{z}{y} \appl{y}{s_1}}} \inact \Par \dots \Par \bout{x}{\abs{z}{\binp{z}{y} \appl{y}{s_n}}} \inact)}} \inact}\\
	\hby{\btau}\\
	\newsp{t}{\binp{t}{y} \appl{y}{a} \Par \bout{\dual{t}}{\abs{x}{(\bout{x}{\abs{z}{\binp{z}{y} \appl{y}{s_1}}} \inact \Par \dots \Par \bout{x}{\abs{z}{\binp{z}{y} \appl{y}{s_n}}} \inact)}} \inact}\\
	\hby{\stau}\\
	\appl{\abs{x}{(\bout{x}{\abs{z}{\binp{z}{y} \appl{y}{s_1}}} \inact \Par \dots \Par \bout{x}{\abs{z}{\binp{z}{y} \appl{y}{s_n}}} \inact)}}{a}
	\\
	\hby{\btau}
	\\
	\bout{a}{\abs{z}{\binp{z}{y} \appl{y}{s_1}}} \inact \Par \dots \Par \bout{a}{\abs{z}{\binp{z}{y} \appl{y}{s_n}}} \inact
\end{array}
\]
and 
\[
\begin{array}{l}
	\newsp{b}{\bout{s}{b} \inact \Par \repl \binp{b}{y} \binp{y}{x} P} \Par \binp{\dual{s}}{x} (\newsp{s}{\bout{x}{s} \bout{\dual{s}}{s_1} \inact} \Par \dots \Par \newsp{s}{\bout{x}{s} \bout{\dual{s}}{s_1}\inact})
	\\
	\hby{\stau}
	\\
	\newsp{b}{\repl \binp{b}{y} \binp{y}{x} P \Par \newsp{s}{\bout{b}{s} \bout{\dual{s}}{s_1} \inact} \Par \dots \Par \newsp{s}{\bout{b}{s} \bout{\dual{s}}{s_1} \inact}}
	\\
	\hby{\stau}
	\\
	\newsp{b}{\repl \binp{b}{y} \binp{y}{x} P \Par \newsp{s}{\binp{s}{x} P \Par \bout{\dual{s}}{s_1} \inact} \Par \dots \Par \newsp{s}{\bout{b}{s} \bout{\dual{s}}{s_1} \inact}}
	\\
	\hby{\stau}
	\\
	\newsp{b}{\repl \binp{b}{y} \binp{y}{x} P \Par P\subst{s_1}{x} \Par \dots \Par \newsp{s}{\bout{b}{s} \bout{\dual{s}}{s_1} \inact}}
	\\
	\Hby{}_{2*(n - 1)}
	\\
	\newsp{b}{\repl \binp{b}{y} \binp{y}{x} P \Par P\subst{s_1}{x} \Par \dots \Par P\subst{s_n}{x}}
	
%	\red
%	\appl{V}{s_1} \Par \dots \Par \appl{V}{s_n}
\end{array}
\]

It is clear that encoding $P_1$ into \HO is more economical than 
encoding $P_2$ into \sessp. Not only moving to a pure higher-order setting requires less reduction steps than in the first-order concurrency of \sessp; in the presence of shared names, moving to a first-order setting brings the need of setting up and handling replicated processes which will eventually lead to garbage expressions. In contrast, the mechanism present in \HO works efficiently regardless of the linear or shared properties of the name that is ``packed'' into the abstraction. 
The use of deterministic transitions guarantees local synchronizations, arguably less expensive than point-to-point synchronizations induced by session communications.

Still, it is instructive to move our comparison 
to a purely linear setting, and to see what occurs. 
In the case of linear values we have:

\begin{eqnarray*}
	\bout{s'}{s} \inact \Par \binp{\dual{s'}}{x} \bout{x}{a} \inact
	\hby{\tau}
	\bout{s}{a} \inact
\end{eqnarray*}
and \HO process
\begin{eqnarray*}
	\bout{s}{\abs{x}{P}} \inact \Par \binp{\dual{s}}{x} \appl{x}{a}
	&\hby{\tau}&
	\appl{(\abs{x}{P})}{a}\\
	&\hby{\tau}&
	P \subst{a}{x}
\end{eqnarray*}

If we consider the encodings \HO and \sessp, respectively
we get:
\[
\begin{array}{l}
	\bout{s'}{\abs{z}{\binp{z}{y} \appl{y}{s}}} \inact \Par \binp{\dual{s'}}{x} \newsp{t}{\appl{x}{t} \Par \bout{\dual{t}}{\abs{x}{\bout{x}{\abs{z}{\binp{z}{y} \appl{y}{a}}} \inact}} \inact}\\
	\hby{\tau}\\
	\newsp{t}{\appl{(\abs{z}{\binp{z}{y} \appl{y}{s}})}{t} \Par \bout{\dual{t}}{\abs{x}{\bout{x}{\abs{z}{\binp{z}{y} \appl{y}{a}}} \inact)}} \inact}\\
	\hby{\tau}\\
	\newsp{t}{\binp{t}{y} \appl{y}{s} \Par \bout{\dual{t}}{\abs{x}{\bout{x}{\abs{z}{\binp{z}{y} \appl{y}{a}}} \inact}} \inact}\\
	\hby{\tau}\\
	\appl{\abs{x}{\bout{x}{\abs{z}{\binp{z}{y} \appl{y}{a}}} \inact}}{s}
	\\
	\hby{\tau}
	\\
	\bout{s}{\abs{z}{\binp{z}{y} \appl{y}{a}}} \inact
\end{array}
\]
and 
\[
\begin{array}{l}
	\newsp{t}{\bout{s}{t} \inact \Par \binp{\dual{t}}{y} \binp{y}{x} P} \Par \binp{\dual{s}}{x} \newsp{s}{\bout{x}{s} \bout{\dual{s}}{a} \inact}
	\\
	\hby{\tau}
	\\
	\newsp{t}{\binp{\dual{t}}{y} \binp{y}{x} P \Par \newsp{s}{\bout{t}{s} \bout{\dual{s}}{a} \inact}}
	\\
	\hby{\tau}
	\\
	\newsp{s}{\binp{s}{x} P \Par \bout{\dual{s}}{a} \inact}
	\\
	\hby{\tau}
	\\
	P\subst{a}{x}
\end{array}
\]


\subsection{Comparing Precise Encodings}\label{ss:compare}
The precise encodings  in  \secref{subsec:HOpi_to_HO} and \secref{subsec:HOp_to_sessp}
confirm that \HO and \sessp constitute two important sources of expressiveness in \HOp.
This naturally begs the question: which of the two sub-calculi is more tightly related to \HOp?
We argue, both empirically and formally, that when compared to \sessp, \HO   is more economical and satisfies tighter correspondences.

\paragraph{Empirical Comparison: Reduction Steps.}
We first contrast the way in which 
%\begin{enumerate}[a)]
%\item 
(a)~the encoding from \HOp to \HO (\secref{subsec:HOpi_to_HO}) translates processes with name passing;
%\item 
(b)~the encoding from \HOp to \sessp (\secref{subsec:HOp_to_sessp}) translates processes with abstraction passing.
%\end{enumerate}
Consider the \HOp processes:
$$
P_1  =  \bout{s}{a} \inact \Par \binp{\dual{s}}{x} (\bout{x}{s_1} \inact \Par \dots \Par \bout{x}{s_n} \inact) \qquad
P_2  =  \bout{s}{\abs{x}{P}} \inact \Par \binp{\dual{s}}{x} (\appl{x}{s_1} \Par \dots \Par \appl{x}{s_n})
$$

%\begin{eqnarray*}
%P_1 & = & \bout{s}{a} \inact \Par \binp{\dual{s}}{x} (\bout{x}{s_1} \inact \Par \dots \Par \bout{x}{s_n} \inact) \\
%P_2 & = & \bout{s}{\abs{x}{P}} \inact \Par \binp{\dual{s}}{x} (\appl{x}{s_1} \Par \dots \Par \appl{x}{s_n})
%\end{eqnarray*}
%\noi 
Observe that $P_1$ features \emph{pure} name passing (no abstraction-passing), whereas 
$P_2$ involves \emph{pure} abstraction passing (no name passing). In both cases, 
the intended communication on $s$ leads to $n$ usages of the communication object (name $a$ in $P_1$, abstraction $\abs{x}{P}$ in $P_2$).
Consider now the reduction steps from $P_1$ and $P_2$:
\begin{eqnarray*}
P_1 & \hby{\tau} & \bout{a}{s_1} \inact \Par \dots \Par \bout{a}{s_n} \inact \\
P_2 & \hby{\tau}& \appl{(\abs{x}{P})}{s_1} \Par \dots \Par \appl{(\abs{x}{P})}{s_n} \quad 
\underbrace{\hby{\btau}\hby{\btau} \cdots \hby{\btau}}_{n} 
%\hby{}}^{n}
%\stackrel{\btau}{\longmapsto^n}
\quad P \subst{s_1}{x} \Par \dots \Par P \subst{s_1}{x} 
\end{eqnarray*}

%Let reduction on \sessp process:
%\begin{eqnarray*}
%	\bout{s}{a} \inact \Par \binp{\dual{s}}{x} (\bout{x}{s_1} \inact \Par \dots \Par \bout{x}{s_n} \inact)
%	\hby{\tau}
%	\bout{a}{s_1} \inact \Par \dots \Par \bout{a}{s_n} \inact
%\end{eqnarray*}
%and \HO process
%\begin{eqnarray*}
%	\bout{s}{\abs{x}{P}} \inact \Par \binp{\dual{s}}{x} (\appl{x}{s_1} \Par \dots \Par \appl{x}{s_n})
%	&\hby{\tau}&
%	\appl{(\abs{x}{P})}{s_1} \Par \dots \Par \appl{(\abs{x}{P})}{s_n}\\
%	&\Hby{\tau}_{n}&
%	P \subst{s_1}{x} \Par \dots \Par P \subst{s_1}{x}
%\end{eqnarray*}
%\noi 
%$P_1$ and $P_2$ follow the same communication pattern; they both
%reduce on a message passing action, with the
%message being substituted $n$ times on the receing side.
%Both $P_1$ and $P_2$ are \HOp processes.
By considering the encoding of $P_1$ into \HO   
we obtain:
\begin{eqnarray*}
\map{P_1}^{1}_f & = &  	\bout{s}{\abs{z}{\binp{z}{y} \appl{y}{a}}} \inact \Par \\
& & \quad  \binp{\dual{s}}{x} \newsp{t}{\appl{x}{t} \Par \bout{\dual{t}}{\abs{x}{(\bout{x}{\abs{z}{\binp{z}{y} \appl{y}{s_1}}} \inact \Par \dots \Par \bout{x}{\abs{z}{\binp{z}{y} \appl{y}{s_n}}} \inact)}} \inact}\\
	& \hby{\stau} \hby{\btau} & 
%	\newsp{t}{\appl{(\abs{z}{\binp{z}{y} \appl{y}{a}})}{t} \Par \bout{\dual{t}}{\abs{x}{(\bout{x}{\abs{z}{\binp{z}{y} \appl{y}{s_1}}} \inact \Par \dots \Par \bout{x}{\abs{z}{\binp{z}{y} \appl{y}{s_n}}} \inact)}} \inact}\\
%	& \hby{\btau} & 
	\newsp{t}{\binp{t}{y} \appl{y}{a} \Par \bout{\dual{t}}{\abs{x}{(\bout{x}{\abs{z}{\binp{z}{y} \appl{y}{s_1}}} \inact \Par \dots \Par \bout{x}{\abs{z}{\binp{z}{y} \appl{y}{s_n}}} \inact)}} \inact}\\
	& \hby{\stau}\hby{\btau}  & 
%	\appl{\abs{x}{(\bout{x}{\abs{z}{\binp{z}{y} \appl{y}{s_1}}} \inact \Par \dots \Par \bout{x}{\abs{z}{\binp{z}{y} \appl{y}{s_n}}} \inact)}}{a}
%	\\
%	& \hby{\btau} & 
	\bout{a}{\abs{z}{\binp{z}{y} \appl{y}{s_1}}} \inact \Par \dots \Par \bout{a}{\abs{z}{\binp{z}{y} \appl{y}{s_n}}} \inact
\end{eqnarray*}
Now, we encode $P_2$ into \sessp:
\begin{eqnarray*}
\pmap{P_2}{2} & = & 	\newsp{b}{\bout{s}{b} \inact \Par \repl \binp{b}{y} \binp{y}{x} P} \Par \\
& & \qquad \qquad \binp{\dual{s}}{x} (\newsp{s}{\bout{x}{s} \bout{\dual{s}}{s_1} \inact} \Par \dots \Par \newsp{s}{\bout{x}{s} \bout{\dual{s}}{s_n}\inact})
	\\
%	& \hby{\stau} & 
%	\newsp{b}{\repl \binp{b}{y} \binp{y}{x} P \Par \newsp{s}{\bout{b}{s} \bout{\dual{s}}{s_1} \inact} \Par \dots \Par \newsp{s}{\bout{b}{s} \bout{\dual{s}}{s_n} \inact}}
%	\\
	& \hby{\stau}  \hby{\stau} \hby{\stau} & 
%	\newsp{b}{\repl \binp{b}{y} \binp{y}{x} P \Par \newsp{s}{\binp{s}{x} P \Par \bout{\dual{s}}{s_1} \inact} \Par \dots \Par \newsp{s}{\bout{b}{s} \bout{\dual{s}}{s_n} \inact}}
%	\\
%	& \hby{\stau} & 
	\newsp{b}{\repl \binp{b}{y} \binp{y}{x} P \Par P\subst{s_1}{x} \Par \dots \Par \newsp{s}{\bout{b}{s} \bout{\dual{s}}{s_n} \inact}}
	\\
	& \Hby{}_{2*(n - 1)} & 
	\newsp{b}{\repl \binp{b}{y} \binp{y}{x} P \Par P\subst{s_1}{x} \Par \dots \Par P\subst{s_n}{x}}
%	\red
%	\appl{V}{s_1} \Par \dots \Par \appl{V}{s_n}
\end{eqnarray*}
%\noi 
Clearly, encoding $P_1$ into \HO is more economical than 
encoding $P_2$ into \sessp. Not only moving to a pure higher-order setting requires less reduction steps than in the first-order concurrency of \sessp; in the presence of shared names, moving to a first-order setting brings the need of setting up and handling replicated processes which will eventually lead to garbage (stuck) processes (cf. $\repl \binp{b}{y} \binp{y}{x} P$ above). In contrast, the mechanism present in \HO works efficiently regardless of the linear or shared properties of the name that is ``packed'' into the abstraction. 
The use of $\beta$-transitions guarantees local synchronizations, which are arguably more economical than point-to-point, session synchronizations.

It is useful to move our comparison 
to a purely linear setting. % and to see what occurs. 
Consider processes:
%In the case of linear values we have:
\begin{eqnarray*}
	Q_1  =  \bout{s'}{s} \inact \Par \binp{\dual{s'}}{x} \bout{x}{a} \inact
	\hby{\tau}
	\bout{s}{a} \inact \quad~~
	Q_2  =  \bout{s}{\abs{x}{P}} \inact \Par \binp{\dual{s}}{x} \appl{x}{a}
	\hby{\tau}
%	\appl{(\abs{x}{P})}{a}
	\hby{\tau}
	P \subst{a}{x}
\end{eqnarray*}
$Q_1$ is a \sessp process; $Q_2$ is an \HO processs.
If we consider their encodings into \HO and \sessp, respectively,
we obtain:
\begin{eqnarray*}
	\map{Q_1}_f^{1} & = & \bout{s'}{\abs{z}{\binp{z}{y} \appl{y}{s}}} \inact \Par \binp{\dual{s'}}{x} \newsp{t}{\appl{x}{t} \Par \bout{\dual{t}}{\abs{x}{\bout{x}{\abs{z}{\binp{z}{y} \appl{y}{a}}} \inact}} \inact}\\
	& \hby{\stau} \hby{\btau}& 
%	\newsp{t}{\appl{(\abs{z}{\binp{z}{y} \appl{y}{s}})}{t} \Par \bout{\dual{t}}{\abs{x}{\bout{x}{\abs{z}{\binp{z}{y} \appl{y}{a}}} \inact)}} \inact}\\
%	& \hby{\btau} & 
	\newsp{t}{\binp{t}{y} \appl{y}{s} \Par \bout{\dual{t}}{\abs{x}{\bout{x}{\abs{z}{\binp{z}{y} \appl{y}{a}}} \inact}} \inact}\\
	& \hby{\stau} & 
	\appl{\abs{x}{\bout{x}{\abs{z}{\binp{z}{y} \appl{y}{a}}} \inact}}{s}
	~~\hby{\btau}~~
	\bout{s}{\abs{z}{\binp{z}{y} \appl{y}{a}}} \inact \\
%\end{eqnarray*}
%\begin{eqnarray*}
	\pmap{Q_2}{2} & = & \newsp{t}{\bout{s}{t} \inact \Par \binp{\dual{t}}{y} \binp{y}{x} P} \Par \binp{\dual{s}}{x} \newsp{s}{\bout{x}{s} \bout{\dual{s}}{a} \inact}
	\\
	& \hby{\stau} \hby{\stau} & 
%	\newsp{t}{\binp{\dual{t}}{y} \binp{y}{x} P \Par \newsp{s}{\bout{t}{s} \bout{\dual{s}}{a} \inact}}
%	\\
%	& \hby{\stau} & 
	\newsp{s}{\binp{s}{x} P \Par \bout{\dual{s}}{a} \inact}
	~~
	\hby{\stau} ~~
	P\subst{a}{x}
\end{eqnarray*}
%\noi 
In this case, the encoding $\pmap{\cdot}{2}$ is more efficient, as it induces less reduction steps.
Therefore, considering a fragment of \HOp without shared communications (linearity only)
has consequences in terms of reduction steps. Notice that we prove that linear communications do 
not suffice to encode shared communications (\secref{ss:negative}).

\paragraph{Formal Comparison: Labelled Transition Correspondence.}
%In addition to preciseness we can develop one more encodability
%result for the translation of \HOp into \HO, which is
%the correspondence of the labelled transition reduction
%system. As we will show such a correspondence does not
%hold for the \HOp to \sessp translation.
We now formally establish differences between $\map{\cdot}_f^1$ and $\pmap{\cdot}{2}$.
To this end, 
we introduce an extra encodability criterion: a form of operational correspondence 
for \emph{visible actions}. 
We shall write $\ell_1, \ell_2, \ldots$ to denote  
actions different from $\tau$
and  $\hby{\ell}$ to denote an LTS.
%with both visible and observable actions, denoted. % and $\hby{}_2$.
As actions from different calculi may be different, we also consider a mapping 
on action labels, denoted $\mapa{\cdot}$: 

\begin{definition}[Labelled Correspondence / Tight Encodings]%\rm
\label{def:lopco}
       Consider typed calculi $\tyl{L}_1$ and  $\tyl{L}_2$, defined as 
        $\tyl{L}_1=\calc{\CAL_1}{{\cal{T}}_1}{\hby{{\ell_1}}_1}{\wb_1}{\proves_1}$
       and $\tyl{L}_2=\calc{\CAL_2}{{\cal{T}}_2}{\hby{{\ell_2}}_2}{\wb_2}{\proves_2}$.
%       ($i=1,2$) be typed calculi. 
The encoding $\enco{\map{\cdot}, \mapt{\cdot}}: \tyl{L}_1 \to \tyl{L}_2$ satisfies
\emph{labelled operational correspondence}
if it satisfies:
	\begin{enumerate}[1.]
			\item
					If		$\stytraargi{\Gamma}{\ell_1}{\Delta}{P}{\Delta'}{P'}{1}{1}$
					then	$\exists Q$, $\Delta''$, $\ell_2$ s.t. 
							(i)~$\wtytraargi{\mapt{\Gamma}}{\ell_2}{\mapt{\Delta}}{\map{P}}{\mapt{\Delta''}}{Q}{2}{2}$;  \\
							(ii)~$\ell_2 = \mapa{\ell_1}$; 
							(iii)~${\mapt{\Gamma}};{\mapt{\Delta''}}\proves_2 {Q}{\wb_2}{\mapt{\Delta'}}\proves_2 {\map{P'}}$.
				
			\item
					If		$\wtytraargi{\mapt{\Gamma}}{\ell_2}{\mapt{\Delta}}{\map{P}}{\mapt{\Delta'}}{Q}{2}{2}$
					then	$\exists P'$, $\Delta''$, $\ell_1$ s.t. 
							(i)~$\stytraargi{\Gamma}{\ell_1}{\Delta}{P}{\Delta''}{P'}{1}{1}$;
							(ii)~$\ell_2 = \mapa{\ell_1}$;
							(iii)~${\mapt{\Gamma}};{\mapt{\Delta''}}\proves_2 {\map{P'}}{\wb_2}{\mapt{\Delta'}}\proves_2 {Q}$.
	\end{enumerate}
A \emph{tight encoding} is a typed 
encoding 
which is precise (\defref{def:goodenc}) and that also satisfies 
labelled operational correspondence as above.
\end{definition}

The following result attests that 
\HOp and \HO are more tightly related than \HOp and~\sessp:
\begin{theorem}[\HO Tightly Encodes \HOp]\label{t:tight}
While the encoding of \HOp into \HO (\defref{d:enc:hopitoho}) is tight, the encoding of \HOp into \sessp (\defref{d:enc:hopitopi}) is not tight.
\end{theorem}

%\noi 
To substantiate the above claim, we show that the encoding $\map{\cdot}^1_f$ enjoys 
labelled operational correspondence, whereas $\pmap{\cdot}{2}$ does not. 
Consider the following mapping:
\[
	\begin{array}{rclcrcl}
		\mapa{\news{\tilde{m_1}}\bactout{n}{m}}^{1}	&\defeq&	\news{\tilde{m_1}}\bactout{n}{\abs{z}{\,\binp{z}{x} \appl{x}{m}} }
		& &
		\mapa{\bactinp{n}{m}}^{1}			&\defeq&	\bactinp{n}{\abs{z}{\,\binp{z}{x} \appl{x}{m}} }
		\\
		\mapa{\news{\tilde{m}}\bactout{n}{\abs{x}{P}}}^{1} &\defeq& \news{\tilde{m}}\bactout{n}{\abs{x}{\pmapp{P}{1}{\es}}}
		& &
		\mapa{\bactinp{n}{\abs{x}{P}}}^{1} &\defeq& \bactinp{n}{\abs{x}{\pmapp{P}{1}{\es}}}
		\\
		\mapa{\bactsel{n}{l} }^{1} &\defeq& \bactsel{n}{l} 
		& &
		\mapa{\bactbra{n}{l} }^{1} &\defeq& \bactbra{n}{l} 
%		\\
%		\mapa{\tau}^{1} &\defeq& \tau
	\end{array}
\]



Then the following result, a complement of \propref{prop:op_corr_HOp_to_HO}, holds:

\begin{proposition}[Labelled Transition Correspondence, \HOp into \HO]
	\label{prop:lts_corr_HOp_to_HO}
	Let $P$ be an \HOp process.
	If $\Gamma; \emptyset; \Delta \proves P \hastype \Proc$ then:
%
	\begin{enumerate}[1.]
		\item
			Suppose $\horel{\Gamma}{\Delta}{P}{\hby{\ell_1}}{\Delta'}{P'}$. Then we have:
%
			\begin{enumerate}[a)]
				\item
					If $\ell_1 \in \set{\news{\tilde{m}}\bactout{n}{m}, \,\news{\tilde{m}}\bactout{n}{\abs{x}Q}, \,\bactsel{s}{l}, \,\bactbra{s}{l}}$
					then $\exists \ell_2$ s.t. \\
					$\horel{\tmap{\Gamma}{1}}{\tmap{\Delta}{1}}{\pmapp{P}{1}{f}}{\hby{\ell_2}}{\tmap{\Delta'}{1}}{\pmapp{P'}{1}{f}}$
					and $\ell_2 = \mapa{\ell_1}^{1}$.
			
				\item
					If $\ell_1 = \bactinp{n}{\abs{y}Q}$ and
					$P' = P_0 \subst{\abs{y}Q}{x}$
					then $\exists \ell_2$ s.t. \\
					$\horel{\tmap{\Gamma}{1}}{\tmap{\Delta}{1}}{\pmapp{P}{1}{f}}{\hby{\ell_2}}{\tmap{\Delta'}{1}}{\pmapp{P_0}{1}{f}\subst{\abs{y}\pmapp{Q}{1}{\emptyset}}{x}}$
					and $\ell_2 = \mapa{\ell_1}^{1}$.
			
				\item
					If $\ell_1 = \bactinp{n}{m}$
					and 
					$P' = P_0 \subst{m}{x}$
					then $\exists \ell_2$, $R$ s.t. 
					$\horel{\tmap{\Gamma}{1}}{\tmap{\Delta}{1}}{\pmapp{P}{1}{f}}{\hby{\ell_2}}{\tmap{\Delta'}{1}}{R}$, \\
					with $\ell_2 = \mapa{\ell_1}^{1}$, 
					and
					$\horel{\tmap{\Gamma}{1}}{\tmap{\Delta'}{1}}{R}{\hby{\stau} \hby{\btau} \hby{\btau}}
					{\tmap{\Delta'}{1}}{\pmapp{P_0}{1}{f}\subst{m}{x}}$.
						
%				\item
%					If $\ell_1 = \tau$
%					and $P' \scong \newsp{\tilde{m}}{P_1 \Par P_2\subst{m}{x}}$
%					then $\exists R$ s.t. \\
%					$\horel{\tmap{\Gamma}{1}}{\tmap{\Delta}{1}}{\pmapp{P}{1}{f}}{\hby{\tau}}{\mapt{\Delta}^{1}}{\newsp{\tilde{m}}{\pmapp{P_1}{1}{f} \Par R}}$,
%					and\\ 
%					$\horel{\tmap{\Gamma}{1}}{\tmap{\Delta}{1}}{\newsp{\tilde{m}}{\pmapp{P_1}{1}{f} \Par R}}{\hby{\btau} \hby{\stau} \hby{\btau}}
%					{\mapt{\Delta}^{1}}{\newsp{\tilde{m}}{\pmapp{P_1}{1}{f} \Par \pmapp{P_2}{1}{f}\subst{m}{x}}}$.
%			
%				\item
%					If $\ell_1 = \tau$
%					and $P' \scong \newsp{\tilde{m}}{P_1 \Par P_2 \subst{\abs{y}Q}{x}}$
%					then \\
%					$\horel{\tmap{\Gamma}{1}}{\tmap{\Delta}{1}}{\pmapp{P}{1}{f}}{\hby{\tau}}
%					{\tmap{\Delta_1}{1}}{\newsp{\tilde{m}}{\pmapp{P_1}{1}{f}\Par \pmapp{P_2}{1}{f}\subst{\abs{y}\pmapp{Q}{1}{\emptyset}}{x}}}$.
%			
%				\item
%					If $\ell_1 = \tau$
%					and $P' \not\scong \newsp{\tilde{m}}{P_1 \Par P_2 \subst{m}{x}} \land P' \not\scong \newsp{\tilde{m}}{P_1 \Par P_2\subst{\abs{y}Q}{x}}$
%					then \\
%					$\horel{\tmap{\Gamma}{1}}{\tmap{\Delta}{1}}{\pmapp{P}{1}{f}}{\hby{\tau}}{\tmap{\Delta'_1}{1}}{ \pmapp{P'}{1}{f}}$.
			\end{enumerate}
			
		\item	Suppose $\horel{\tmap{\Gamma}{1}}{\tmap{\Delta}{1}}{\pmapp{P}{1}{f}}{\hby{\ell_2}}{\tmap{\Delta'}{1}}{Q}$.
			Then we have:
%
			\begin{enumerate}[a)]
				\item 
					If $\ell_2 \in
					\set{\news{\tilde{m}}\bactout{n}{\abs{z}{\,\binp{z}{x} (\appl{x}{m})}}, \,\news{\tilde{m}} \bactout{n}{\abs{x}{R}}, \,\bactsel{s}{l}, \,\bactbra{s}{l}}$
					then $\exists \ell_1, P'$ s.t. \\
					$\horel{\Gamma}{\Delta}{P}{\hby{\ell_1}}{\Delta'}{P'}$, 
					$\ell_1 = \mapa{\ell_2}^{1}$, 
					and
					$Q = \pmapp{P'}{1}{f}$.
			
				\item 
					If $\ell_2 = \bactinp{n}{\abs{y} R}$ %(with $R \neq \binp{y}{x} \appl{x}{m}$)
					then either:
%
					\begin{enumerate}[(i)]
						\item	$\exists \ell_1, x, P', P''$ s.t. \\
							$\horel{\Gamma}{\Delta}{P}{\hby{\ell_1}}{\Delta'}{P' \subst{\abs{y}P''}{x}}$, 
							$\ell_1 = \mapa{\ell_2}^{1}$, $\pmapp{P''}{1}{\es} = R$, and $Q = \pmapp{P'}{1}{f}$.

						\item	$R \scong \binp{y}{x} (\appl{x}{m})$ and 
							$\exists \ell_1, z, P'$ s.t. 
							$\horel{\Gamma}{\Delta}{P}{\hby{\ell_1}}{\Delta'}{P' \subst{m}{z}}$, \\
							$\ell_1 = \mapa{\ell_2}^{1}$,
							and 
							$\horel{\tmap{\Gamma}{1}}{\tmap{\Delta'}{1}}{Q}{\hby{\stau} \hby{\btau} \hby{\btau}}{\tmap{\Delta''}{1}}{\pmapp{P'\subst{m}{z}}{1}{f}}$
					\end{enumerate}
			
%				\item 
%					If $\ell_2 = \tau$ 
%					then $\Delta' = \Delta$ and 
%					either
%%
%					\begin{enumerate}[(i)]
%						\item	$\exists P'$ s.t. 
%							$\horel{\Gamma}{\Delta}{P}{\hby{\tau}}{\Delta}{P'}$,
%							and $Q = \map{P'}^{1}_f$.	
%
%						\item
%							$\exists P_1, P_2, x, m, Q'$ s.t. 
%							$\horel{\Gamma}{\Delta}{P}{\hby{\tau}}{\Delta}{\newsp{\tilde{m}}{P_1 \Par P_2\subst{m}{x}} }$, and\\
%							$\horel{\tmap{\Gamma}{1}}{\tmap{\Delta}{1}}{Q}{\hby{\btau} \hby{\stau} \hby{\btau}}{\tmap{\Delta}{1}}{\pmapp{P_1}{1}{f} \Par \pmapp{P_2\subst{m}{x}}{1}{f}}$ 
%%							$Q = \map{P_1}^{1}_f \Par Q'$, where $Q'  \Hby{} $.
%
%%						\item $\exists P_1, P_2, x, R$ s.t. 
%%						$\stytra{ \Gamma }{\tau}{ \Delta }{ P}{ \Delta}{ \news{\tilde{m}}(P_1 \Par P_2\subst{\abs{y}R}{x}) }$, and 
%%						$Q = \map{\news{\tilde{m}}(P_1 \Par P_2\subst{\abs{y}R}{x})}^{1}_f$.
%			\end{enumerate}
		    \end{enumerate}
		    
%		\item   
%			If  $\wtytra{\mapt{\Gamma}^{1}}{\ell_2}{\mapt{\Delta}^{1}}{\pmapp{P}{1}{f}}{\mapt{\Delta'}^{1}}{Q}$
%			then $\exists \ell_1, P'$ s.t.  \\
%			(i)~$\stytra{\Gamma}{\ell_1}{\Delta}{P}{\Delta'}{P'}$,
%			(ii)~$\ell_2 = \mapa{\ell_1}^{1}$, 
%			(iii)~$\wbb{\mapt{\Gamma}^{1}}{\ell}{\mapt{\Delta'}^{1}}{\pmapp{P'}{1}{f}}{\mapt{\Delta'}^{1}}{Q}$.
	\end{enumerate}
\end{proposition}

The analog of \propref{prop:lts_corr_HOp_to_HO} does not hold for the encoding of \HOp into \sessp.
Consider the \HOp process:
\[
	\Gamma; \es; \Delta \proves \bout{s}{\abs{x}{P}} \inact \hastype \Proc \hby{\bactout{s}{\abs{x} P}} \es \proves \inact \not \hby{}
\]
with $\abs{x}{P}$ being a linear value.
We translate it into a \sessp process:
\[\tmap{\Gamma}{2}; \es; \tmap{\Delta}{2} \proves \newsp{a}{\bout{s}{a} \inact \Par \binp{a}{y} \binp{y}{x} P} \hastype \Proc
	 \hby{\bactout{s}{a}} \Delta' \proves \binp{a}{y} \binp{y}{x} P \hastype \Proc
\hby{\bactinp{a}{V}} \dots
\]

%\begin{eqnarray*}
%	&&\tmap{\Gamma}{2}; \es; \tmap{\Delta}{2} \proves \newsp{a}{\bout{s}{a} \inact \Par \binp{a}{y} \binp{y}{x} P} \hastype \Proc\\
%	&&\hby{\bactout{s}{a}}\\
%	&&\Delta' \proves \binp{a}{y} \binp{y}{x} P \hastype \Proc\\
%	&&\hby{\bactinp{a}{V}} \dots
%\end{eqnarray*}

%\noi 
The resulting processes have a mismatch both in the typing
environment ($\Delta' \not= \tmap{\es}{2}$)
and in the actions that they can %the two processes can
subsequently observe: the first process
cannot perform any action, while the second process
can perform actions of the encoding of $\abs{x}{P}$.



\subsection{A Negative Result}
\label{ss:negative}
%\section{Negative Results}

In the encoding from $\HOp$ to $\sessp$ we showed that
an easy and straightforward encoding would be to create
a new shared name for every abstraction we want to pass
in order to use it as a trigger that activate copies of
the abstraction.

At this point a reasonable question could be whether we can
encode shared name behaviour to session name behaviour and at
the same time maintain the type, operational and behavioural semantics.
If such result holds then its impact would be much bigger than
the encoding from $\HOp$ to $\sessp$, since it would
allow us to have session type systems without shared names
and still have the modelling convenience of shared names.

In this section we prove the intution among researches 
that a semantic preserving encoding between a calculus
with shared names and a calculus with only session names
does not exist.

\begin{theorem}\rm
	There is no encoding $\enco{\map{\cdot}, \mapt{\cdot}, \mapa{\cdot}}: \HOp \longrightarrow \HOp^{\minussh}$
	that enjoys operational correspondence and full abstraction.
\end{theorem}

\begin{proof}
	The proof is based on the fact that
	transitions on session channels are
	$\tau$-inert in contrast with shared
	channels which do not enjoy
	$\tau$-inertness.

	Details of the proof in Appendix~\ref{app:neg}
	\qed
\end{proof}


As most session calculi, 
\HOp includes communication on both shared and linear names.
The former enables non determinism, unrestricted behaviour; the latter allows to represent
deterministic, linear communication structures.
The expressiveness of shared names is also illustrated by our 
encoding from \HOp into \sessp (\figref{f:enc:ho_to_sessp}).
%Shared and linear names are fundamentally different; still, to the best of our knowledge,
%the status of shared communication, in terms of expressiveness, has not been formalized for session calculi.
This result begs the question: 
%\dk{can we further encode the replicated triggers in the encoding from \HOp to \HO using only session names?}
can we represent shared name interaction using session name interaction?
%are shared names truly indispensable for communication, or could they
%be encoded using linear communication?
It turns out that shared names  add expressiveness to \HOp:
we prove
the non existence of a minimal encoding 
(cf.~\defref{def:goodenc})
of shared name 
interaction into linear interaction. % (see \appref{app:neg} for details of the proof).
%for their behavior cannot be represented using purely deterministic processes.
%To this end, we show the non existence of a minimal encoding 
%(cf.~\defref{def:goodenc}(ii))
%of shared name communication into linear 
%communication. 

%\smallskip 

\begin{theorem}%\rm
	\label{t:negative}
	There is no minimal encoding from
		$\sessp$ to $\HOp^{\minussh}$.
	Hence, for any
		$\CAL_1,\CAL_2\in \{ \HOp, \HO, \sessp\}$, 
	there is no minimal encoding from  
	$\tyl{L}_{\CAL_1}$ 
	into
		$\tyl{L}_{\CAL_2^{-\mathsf{sh}}}$.
\end{theorem}

%\begin{proof}
%	The proof of \thmref{t:negative} relies on
%	%As described next, 
%	$\tau$-inertness (\lemref{lem:tau_inert})
%	%is critical in the proof.
%	%Recall that minimal encodings preserve barbs 
%	and barb preservation~(\propref{p:barbpres}).
%	See details in \appref{app:neg}.
%\end{proof}

By 
	\defref{d:enc:hopitoho} and~\ref{d:enc:hopitopi} 
	and \propref{f:enc:hopitoho} and~\ref{f:enc:hotopi}, 
we have:
\smallskip 
\NY{
\begin{corollary}
Let $\CAL_1,\CAL_2\in \{ \HOp, \HO, \sessp\}$.  
There is a precise encoding 
of
$\tyl{L}_{\CAL_1^{-\mathsf{sh}}}$ 
in $\tyl{L}_{\CAL_2^{-\mathsf{sh}}}$.
\end{corollary}
}


\begin{comment}
\begin{IEEEproof}[Proof]
	Let $\horel{\Gamma_1}{\Delta_1}{P_1}{\not\wb}{\Delta_2}{P_2}$
	with $P = \breq{a}{s} \inact \Par \bacc{a}{x} P_1 \Par \bacc{a}{x} P_2$ and	let $\Gamma; \emptyset; \Delta \proves P \hastype \Proc$.
	Assume also a encoding
	$\enc{\cdot}{\cdot}: \sessp \longrightarrow \HOp^{\minussh}$
is minimum. 
	From operational correspondence we obtain:
\[
\begin{array}{rcl}
		P \red P_1 \Par \bacc{a}{x} P_2 &\textrm{implies}& \map{P} \red \map{P_1 \Par \bacc{a}{x} P_2}\\
		P \red P_2 \Par \bacc{a}{x} P_1 &\textrm{implies}& \map{P} \red \map{P_2 \Par \bacc{a}{x} P_1}
\end{array}
\]
	From the fact that
	$\horel{\Gamma_1}{\Delta_1}{P_1}{\not\wb}{\Delta_2}{P_2}$
	we can derive that
%
	\[
		\horel{\Gamma_1'}{\Delta_1'}{P_1 \Par \bacc{a}{x} P_2}{\not\wb}{\Delta_2'}{P_2 \Par \bacc{a}{x} P_1}
	\]
%
	From \lemref{lem:tau_inert}(2) we know that
%
\[
\begin{array}{rcl}
		\horel{\mapt{\Gamma}}{\mapt{\Delta}}{\map{P}}{\wb}{\mapt{\Delta_1'}}{\map{P_1 \Par \bacc{a}{x} P_2}}\wb 
{\mapt{\Delta_2'}}\proves {\map{P_2 \Par \bacc{a}{x} P_1}}
\end{array}
\]
%
	%\noi 
	thus
$\horel{\mapt{\Gamma}}{\mapt{\Delta_1'}}{\map{P_1 \Par \bacc{a}{x} P_2}}{\wb}{\mapt{\Delta_2'}}{\map{P_2 \Par \bacc{a}{x} P_1}}$, 
%
	which contradicts the assumption. 
%	so there is no mapping $\map{\cdot}: \pHO \longrightarrow \spi$ that enjoys
%	the operational correspondence and full abstraction properties.
\end{IEEEproof}
\end{comment}




%%%%%%%%%%%%%%%%%%%%%%%%%%%%%%%%%%%%%%%%%%%%%%%%%%%%%%%%%%%%%%%%%%%%%%%%%%%%%%%%
%%%%%%%%%%%%%%%%%%%%%%%%%%%%%%%%%%%%%%%%%%%%%%%%%%%%%%%%%%%%%%%%%%%%%%%%%%%%%%%%

\section{Extensions: Higher-Order Abstractions and Polyadicity}
\label{sec:extension}
%% !TEX root = main.tex




In this section we extend \HOp to define two more higher-order
process calculi:
(i)~\HOpp is the extension of \HOp with higher-order applications/abstractions
and 
(ii)~\PHOp is the extension of \HOp
with polyadicity.
In both cases, we detail the
required modifications in the syntax and types.
The two extensions allow us to assume the \PHOpp
which is the polyadic extension of \HOpp.


\subsubsection{\HOp with Higher-Order Abstractions: The  $\HOpp$}
%\label{subsec:hop}
\noi 
The calculus \HOpp 
extends \HOp with higher-order abstractions and applications.
\HOpp is the calculus defined in~\cite{characteristic_bis}.

\myparagraph{Syntax, Operational Semantics and Types.}
\noi First, the syntax of \figref{fig:syntax} extends 
$\appl{V}{u}$ to $\appl{V}{W}$, including higher-order value $W$. 
Rule 
\[
	\appl{(\abs{x}{P})}{V} \red P \subst{V}{x}
\]
replaces rule $\orule{App}$ in \figref{fig:reduction}.
The syntax of types is modified as follows: %changes to include: 
%
\begin{center}
	\begin{tabular}{c}
		$L \bnfis \shot{U} \bnfbar \lhot{U}$
	\end{tabular}
\end{center}
These types can be easily accommodated in the type system in \figref{fig:typerulesmy}, 
we replace $C$ by $U$ in \trule{Abs} and $C$ by $U'$ in \trule{App}. Subject
reduction~(\thmref{t:sr}) holds for \HOpp (cf.~\cite{characteristic_bis})

\subsubsection{\HOp with Polyadic Communication: \PHOp}

\noi Embeddings of polyadic name passing into monadic name passing are
well-studied. % in the literature. 
Using a linear typing, precise
encodings (including full abstraction) can be obtained~\cite{Yoshida96}.
The syntax of 
$\HOp$ is extended with
polyadic name passing $\tilde{n}$ and $\abs{\tilde{x}}{Q}$ in the syntax 
of value $V$. The type syntax is extended to: 
%
\begin{center}
	\begin{tabular}{c}
	$	L \bnfis \shot{\tilde{C}} \bnfbar \lhot{\tilde{C}}
		\quad\quad
		S \bnfis  \btout{\tilde{U}} S \bnfbar \btinp{\tilde{U}} S \bnfbar \cdots$
	\end{tabular}
\end{center}
%
The type system disallows a shared name that directly carries polyadic
shared names as in \cite{tlca07,MostrousY15}.

The combined syntax, semantics and type syntax for \HOpp and \PHOp
define the \PHOpp which is the calculus that allows higher-order
abstractions and aplications, and polyadicity.


Here we extend \HOp in two directions: %to define two more higher-order
%process calculi:
(i)~\HOpp  extends   \HOp with higher-order applications/abstractions;
(ii)~\PHOp   extends  \HOp
with polyadicity.
In both cases, we detail the
required modifications in syntax and types.
Using encoding composability (\propref{pro:composition}), 
the two extensions may be combined into \PHOpp: the polyadic extension of \HOpp.


\paragraph{\HOp with Higher-Order Abstractions ($\HOpp$) and 
with Polyadicity (\PHOp).
}
%\label{subsec:hop}
We first introduce \HOpp, the  extension of \HOp with higher-order abstractions and applications.
This is the calculus that we studied in~\cite{characteristic_bis}. The syntax of \HOpp is obtained 
from  \figref{fig:syntax} by extending
$\appl{V}{u}$ to $\appl{V}{W}$, where  $W$ is a higher-order value. 
As for the reduction semantics, we keep the rules in \figref{fig:reduction}, except for 
 $\orule{App}$ which is replaced by 
$$
	\appl{(\abs{x}{P})}{V} \red P \subst{V}{x}
$$
The syntax of types is modified as follows: %changes to include: 
$
		L \bnfis \shot{U} \bnfbar \lhot{U}.
$
These types can be easily accommodated in the type system 
%in \figref{fig:typerulesmy}: 
in \secref{sec:types}:
we replace $C$ by $U$ in \trule{Abs} and $C$ by $U'$ in \trule{App}. Subject
reduction~(\thmref{t:sr}) holds for \HOpp (cf.~\cite{characteristic_bis})

%\subsection{\HOp with Polyadic Communication: \PHOp}
%
%\noi Embeddings of polyadic name passing into monadic name passing are
%well-studied. % in the literature. 
%Using a linear typing, precise
%encodings (including full abstraction) can be obtained~\cite{Yoshida96}.
The calculus  
$\PHOp$ 
extends $\HOp$ 
with polyadic name passing $\tilde{n}$ and $\abs{\tilde{x}}{Q}$ in the syntax 
of values $V$. 
The operational semantics is kept unchanged, with the expected use of the simultaneous substitution $\subst{\tilde{V}}{\tilde{x}}$.
The type syntax is extended to: 
%
\begin{center}
	\begin{tabular}{c}
	$	L \bnfis \shot{\tilde{C}} \bnfbar \lhot{\tilde{C}}
		\quad\quad
		S \bnfis  \btout{\tilde{U}} S \bnfbar \btinp{\tilde{U}} S \bnfbar \cdots$
	\end{tabular}
\end{center}
%
As in \cite{tlca07,MostrousY15},
the type system for \PHOp 
disallows a shared name that directly carries polyadic
shared names.

By combining \HOpp and \PHOp into a single calculus we obtain \PHOpp:
the extension of \HOp allows \emph{both} higher-order
abstractions/aplications and polyadicity.


%This section studies two extensions of \HOp, 
the functional abstraction (\HOpp)
and polyadic (\PHOp) calculi.  
 
\subsection{Encoding from $\HOpp$ to $\HOp$}
\label{subsec:hop}
\noi The functional abstraction \HOp-calculus, denoted by \HOpp, 
extends \HOp to %the $n$-higher-order 
functional abstractions and applications.

\myparagraph{Syntax, Operational Semantics and Types}
\noi First, the syntax of \figref{fig:syntax} extends 
$\appl{V}{u}$ to 
 $\appl{V}{W}$, including higher-order value $W$. 
We then replace rule $\orule{App}$ in \figref{fig:reduction}
with rule $\appl{(\abs{x}{P})}{V} \red P \subst{V}{x}$.
The syntax of types changes to include: 
\[ L \bnfis \shot{U} \bnfbar \lhot{U}\]  
We apply the straightforward extension of the typing  
system to accomodate the extended type syntax 
(we replace $C$ by $U$ in \trule{Abs} and $C$ by $U'$ in \trule{App} in \figref{fig:typerulesmy}).
\smallskip 

\myparagraph{Behavioural Semantics}
Labels remain the same. Rule $\ltsrule{App}$ in the untyped LTS
(\figref{fig:untyped_LTS}) 
is replaced with rule $\appl{(\abs{x}{P})}{V} \by{\tau} P \subst{V}{x}$.
Characteristic processes (\defref{def:char}) are extended with  
${\mapchar{\shot{U}}{x}} \!\!\defeq\!\! \mapchar{\lhot{U}}{x} \!\!\defeq\!\! {\appl{x}{\omapchar{U}}}$ and ${\omapchar{\shot{U}}} \defeq {\omapchar{\lhot{U}}} \!\!\defeq\!\! \abs{x}{\mapchar{U}{x}}$. 
Then we can use the same definitions for $\cong$, $\wbc$, $\hwb$ and $\fwb$. 

\smallskip 

\myparagraph{Encoding from \HOpp to \HOp} 
Let $\tyl{L}_{\HOpp}=\calc{\HOpp}{{\cal{T}}_4}{\hby{\ell}}{\wb_H}{\proves}$
where 
${\cal{T}}_4$ is a set of types of $\HOpp$;  
the typing $\proves$ is defined in 
\figref{fig:typerulesmy} with extended \trule{Abs} and \trule{App}. 
We define the typed encoding 
the typed encoding $\enco{\map{\cdot}^{3}, \mapt{\cdot}^{3}, \mapa{\cdot}^{3}}: \HOpp \to \HOp$ in 
\figref{f:enc:hopip_to_hopi}.
By \propref{pro:composition}, 
we derive the following theorem. 

\smallskip 

\begin{theorem}[Encoding of Functional Abstraction Higher-Order Pi into Pi]
\label{f:enc:hopiptohopi}
The encoding from $\tyl{L}_{\HOpp}$ into $\tyl{L}_{\HOp}$ 
is precise. Hence the encodings 
from $\tyl{L}_{\HOpp}$ to $\tyl{L}_{\HO}$ 
and $\tyl{L}_{\sessp}$ 
are also precise. 
\end{theorem}

\begin{figure}[t]
\[
\begin{array}{lrcll}
\noindent{\bf Types:} & 
		\tmap{\shot{L}}{3} &\defeq& \shot{\btinp{\tmap{L}{3}} \tinact}
		\\
%&		\tmap{\lhot{L}}{3} &\defeq& \lhot{\btinp{\tmap{L}{3}} \tinact}
%		\\
&		\tmap{\btout{\shot{L}} S}{3} &\defeq& \btout{\tmap{\shot{L}}{3}} \tmap{S}{3}
		\\
%&		\tmap{\btout{\lhot{L}} S}{3} &\defeq& \btout{\tmap{\lhot{L}}{3}} \tmap{S}{3}
%		\\
&		\tmap{\btinp{\shot{L}} S}{3} &\defeq& \btinp{\tmap{\shot{L}}{3}} \tmap{S}{3}
%		\\
%&		\tmap{\btinp{\lhot{L}} S}{3} &\defeq& \btinp{\tmap{\lhot{L}}{3}} \tmap{S}{3}
\\[1mm]
\hline
\noindent{\bf Labels:} & 
%		\mapa{\bactout{n}{\abs{x:C}{P}}}^{3} &\defeq& \bactout{n}{\abs{x}{\pmap{P}{3}}}
%		\\
%		\mapa{\bactinp{n}{\abs{x:C}{P}}}^{3} &\defeq& \bactinp{n}{\abs{x}{\pmap{P}{3}}}
%		\\
		\mapa{\bactout{n}{\abs{\AT{x}{L}}{P}}}^{3} &\defeq& \bactout{n}{\abs{z}{\binp{z}{x} \pmap{P}{3}}}
		\\
&		\mapa{\bactinp{n}{\abs{\AT{x}{L}}{P}}}^{3} &\defeq& \bactinp{n}{\abs{z}{\binp{z}{x} \pmap{P}{3}}}
\\[1mm]
\hline
{\bf Terms}: & 
	\pmap{\appl{x}{(\abs{x} P)}}{3} &\defeq& \newsp{s}{\appl{x}{s} \Par \bout{\dual{s}}{\abs{x} \pmap{P}{3}} \inact}
		\\
&	\pmap{\bout{u}{\abs{\AT{x}{L}}{Q}} P}{3} &\defeq& \bout{u}{\abs{z}{\binp{z}{x} \pmap{Q}{3}}} \pmap{P}{3}
%		\pmap{\bout{u}{\abs{x: C}{Q}} P}{3} &\defeq& \bout{u}{\abs{x}{\pmap{Q}{3}}} \pmap{P}{3}
	\end{array}
	\]
The mapping of types for $\lhot{L}$ is defined by replacing 
$\shot{L}$ by $\lhot{L}$. 
The case of $\abs{x:C}{P}$ in the label and term mappings 
are %defined 
as in \figref{f:enc:hopi_to_ho}, replacing 
$\tmap{\cdot}{1}$,
$\mapa{\cdot}^{1}$, and 
$\pmap{\cdot}{1}_f$, by  
$\tmap{\cdot}{3}$,
$\mapa{\cdot}^{3}$, and 
$\pmap{\cdot}{3}$. 
The other processes, types and labels are  homomorphic. 
\caption{\label{f:enc:hopip_to_hopi} 
Encoding of \HOpp into \HOp.
}
\Hline
\end{figure} 

\subsection{Encoding from Polyadic $\HOp$ to $\HOp$}
\label{subsec:pho}
\noi Embedding the polyadic name passing 
into the monadic name passing is well-studied in the literature.    
Using the linear typing, 
the preciseness (full abstraction) can be obtained \cite{Yoshida96}.
Here we summarise $\PHOp$ can be encoded into $\HOp$. 
The syntax of 
$\HOp$ is extended from \HOp by including 
polyadic name passing $\tilde{n}$ and $\abs{\tilde{x}}{Q}$ in the syntax 
of value $V$. The type syntax is extended to: 
%
\[
L ::= \shot{\tilde{C}} \ | \ \lhot{\tilde{C}}
\quad\quad S \ ::= \  \btout{\tilde{U}} S \bnfbar \btinp{\tilde{U}} S \bnfbar \cdots 
\]
%
\dk{The type system dissallows polyadic shared names.}
Other definitions are straightforwardly extended. 
We extend the mapping for labels 
($\mapa{\cdot}: \ell \to \tilde{\ell}$ in  
\defref{def:tenc}) to capture 
a sequence of labels  and \defref{def:ep} stays as the same
assuming that if 
$P \hby{\ell} P'$ and $\mapa{\ell} = \{\ell_1, \ell_2,  \cdots, \ell_m\}$ then
$\map{P} \Hby{\mapa{\ell}} \map{P'}$
should be understood as
$\map{P} \Hby{\ell_1} P_1 \Hby{\ell_2} P_2 \cdots \Hby{\ell_m} P_m =  \map{P'}$,
for some
$P_1, P_2, \ldots, P_m$.

Let $\tyl{L}_{\PHOp}=\calc{\PHOp}{{\cal{T}}_5}{\hby{\ell}}{\wb_H}{\proves}$
where 
${\cal{T}}_5$ is a set of types of $\HOpp$;  
the typing $\proves$ is defined in 
\figref{fig:typerulesmy} with polyadic types. 
We define the typed encoding 
the typed encoding $\enco{\map{\cdot}^{4}, \mapt{\cdot}^{4}, \mapa{\cdot}^{4}}: \PHOp \to \HOp$ 
in \figref{f:enc:poltomon}. 
Then we have:

\smallskip 

\begin{theorem}[Encoding of Polyadic Higher-Order Pi into Higher-Order Pi]
\label{f:enc:hopiptohopi}
The encoding from $\tyl{L}_{\PHOp}$ into $\tyl{L}_{\HOp}$ 
is precise. Hence the encodings 
from $\tyl{L}_{\PHOp}$ to 
$\tyl{L}_{\HO}$ 
and $\tyl{L}_{\sessp}$ 
are also precise. 
\end{theorem}

\begin{figure}[t]
\small
\[
\begin{array}{rcl}
% typed mapping starts here
{\bf Types:}\hspace{2.5cm}\\
		\tmap{\btout{S_1, \cdots, S_m}S}{4}
		&\!\!\defeq\!\!&
		\btout{\tmap{S_1}{4}} \cdots ; \btout{\tmap{S_m}{4}}\tmap{S}{4}
%		\\
%		\tmap{\btinp{S_1, \cdots, S_m}S}{4}
%		&\defeq&
%		\btinp{\tmap{S_1}{4}} \cdots ; \btinp{\tmap{S_m}{4}}\tmap{S}{4}
		\\
		\tmap{\bbtout{L} S}{4}
		&\!\!\defeq\!\!&
		\bbtout{\mapt{L}^{4}}\mapt{S}^{4}
%		\\
%		\tmap{\bbtinp{L} S}{4}
%		&\defeq&
%		\bbtinp{\mapt{L}^{4}}\mapt{S}^{4}
		\\
%		\tmap{\bbtout{\shot{(C_1, \cdots, C_m)}} S}{4}
%		&\defeq&
%		\bbtout{
%		\shot{\big(\btinp{\tmap{C_1}{4}} \cdots; \btinp{\tmap{C_m}{4}}\tinact\big)}}\mapt{S}^{4}
%		\\
%		\tmap{\bbtinp{\shot{(C_1, \cdots, C_m)}} S}{4}
%		&\defeq&
%		\bbtinp{
%		\shot{\big(\btinp{\tmap{C_1}{4}} \cdots; \btinp{\tmap{C_m}{4}}\tinact\big)}}\mapt{S}^{4}
%		\\
		\tmap{\shot{(C_1, \cdots, C_m)}}{4}
		&\!\!\defeq\!\!&
		\shot{\big(\btinp{\tmap{C_1}{4}} \cdots; \btinp{\tmap{C_m}{4}}\tinact\big)}
		\\
		\tmap{\lhot{(C_1, \cdots, C_m)}}{4}
		&\!\!\defeq\!\!&
		\lhot{\big(\btinp{\tmap{C_1}{4}} \cdots; \btinp{\tmap{C_m}{4}}\tinact\big)}
		\\
%		\tmap{\lhot{(C_1, \cdots, C_m)}}{4}
%		&\defeq&
%		\lhot{\big(\btinp{\tmap{C_1}{\mathsf{p}}} \cdots \btinp{\tmap{C_m}{\mathsf{p}}}\tinact\big)}
%		\\
%		\tmap{\shot{(C_1, \cdots, C_m)}}{\mathsf{p}}
%		&\defeq&
%		\shot{\big(\btinp{\tmap{C_1}{\mathsf{p}}} \cdots \btinp{\tmap{C_m}{\mathsf{p}}}\tinact\big)}
\hline
{\bf Labels:}\hspace{2.5cm}\\
%		\\ % action mapping starts here
		\mapa{\bactout{u}{u_1, \ldots, u_m}}^4 
		&\!\!\!\!\defeq\!\!\!\!&
 \bactout{u}{u_1} \cdots \bactout{u}{u_m} \\
%		\mapa{\bactinp{k}{k_1, \ldots, k_m}}^4 &\defeq&   \big\{\bactinp{k}{k_1}, \cdots, \bactinp{k}{k_m} \big\}\\
		\mapa{\bactout{u}{\abs{x_1, \ldots, x_m}{P}} }^4 
		&\!\!\defeq\!\!&
\bactout{u}{\abs{z}\binp{z}{x_1} \cdots \binp{z}{x_m} \map{P}^{4}}\\
%		\mapa{\bactinp{k}{\abs{x_1, \ldots, x_m}{P}} }^4 &\defeq&  \bactinp{k}{\abs{z}\binp{z}{x_1} \cdots ; \binp{z}{x_m} \map{P}^{4}} 
\hline
{\bf Terms:}\hspace{2.5cm}\\
		\map{\bout{u}{u_1, \cdots, u_m} P}^{4}
		&\!\!\defeq\!\!&
		\bout{u}{u_1} \cdots ;  \bout{u}{u_m} \map{P}^{4}
		\\
%			\map{\binp{k}{x_1, \cdots, x_m} P}^{4}
%		&\defeq&
%		\binp{k}{x_1} \cdots ; \binp{k}{x_m}  \map{P}^{4}
%		\\
		\map{\bbout{u}{\abs{(x_1, \cdots, x_m)} Q} P}^{4}
		&\!\!\defeq\!\!&
		\bbout{u}{\abs{z}\binp{z}{x_1}\cdots ; \binp{z}{x_m} \map{Q}^{4}} \map{P}^{4}
		\\ 
		\map{\appl{x}{(u_1, \cdots, u_m)}}^{4}
		&\!\!\defeq\!\!&
		\newsp{s}{\appl{x}{s} \Par \bout{\dual{s}}{u_1} \cdots ; \bout{\dual{s}}{u_m} \inact} 
        \\ 
	\end{array}
\]
The input cases are defined as the outputs replacing $!$ by $?$. 
The mappings for the other processes/types/labels are 
homomorphic. 
\caption{\label{f:enc:poltomon}
Encoding of \PHOp to \HOp.
}
\Hline 
\end{figure}


\paragraph{Precise Encodings of $\HOpp$ and $\PHOp$ into $\HOp$.}
%\noi 
We give  %two extensions of \HOp: 
encodings of \HOpp into \HOp
and into \PHOp, and show that they are precise. 
We use encoding composition (\propref{pro:composition}) to encode
\PHOpp into \HO and \sessp.
We consider the following 
typed calculi (cf.~\defref{d:tcalculus}):
\begin{enumerate}[-]
\item $\tyl{L}_{\HOpp}=\calc{\HOpp}{{\cal{T}}_4}{\hby{\ell}}{\hwb}{\proves}$,
where 
	${\cal{T}}_4$ is a set of types of $\HOpp$;  
the typing $\proves$ is defined in 
%\figref{fig:typerulesmy} 
\secref{sec:types}
with extended rules \trule{Abs} and \trule{App}.

\item 
	$\tyl{L}_{\PHOp}=\calc{\PHOp}{{\cal{T}}_5}{\hby{\ell}}{\hwb}{\proves}$,
where 
	${\cal{T}}_5$ is the set of types of $\PHOp$;  
the typing $\proves$ is defined in
%\figref{fig:typerulesmy} 
in \secref{sec:types}
with polyadic types.
\end{enumerate}

%\myparagraph{Syntax, Operational Semantics and Types.}
%\noi First, the syntax of \figref{fig:syntax} extends 
%$\appl{V}{u}$ to 
% $\appl{V}{W}$, including higher-order value $W$. 
%The rule $\appl{(\abs{x}{P})}{V} \red P \subst{V}{x}$
%replaces
%rule $\orule{App}$ in \figref{fig:reduction}.
%The syntax of types is modified as follows: %changes to include: 
%\begin{center}
%\begin{tabular}{c}
%$L \bnfis \shot{U} \bnfbar \lhot{U}$
%\end{tabular}
%\end{center}
%These types can be easily accommodated in the type system:
% in \figref{fig:typerulesmy}, 
%we replace $C$ by $U$ in \trule{Abs} and $C$ by $U'$ in \trule{App}.
%\smallskip 
%
%\myparagraph{Behavioural Semantics.}
%Labels remain the same. Rule $\ltsrule{App}$ in the untyped LTS
%(\figref{fig:untyped_LTS}) 
%is replaced with rule $\appl{(\abs{x}{P})}{V} \by{\tau} P \subst{V}{x}$.
%\defref{def:char} (characteristic processes) is extended with  
%${\mapchar{\shot{U}}{x}} \defeq\! \mapchar{\lhot{U}}{x} \defeq\! {\appl{x}{\omapchar{U}}}$ and 
%${\omapchar{\shot{U}}} \defeq {\omapchar{\lhot{U}}} \!\!\defeq\!\! \abs{x}{\mapchar{U}{x}}$. 
%We can then use the same definitions for $\cong$, $\wbc$, $\hwb$ and $\fwb$. 
%\smallskip 

%\paragraph{Encoding \HOpp into \HOp} 
First, the typed encoding
$\enco{\map{\cdot}^{3}, \mapt{\cdot}^{3}}: \HOpp \to \HOp$ is defined
in \figref{f:enc:hopip_to_hopi}.
It satisfies the following properties:
{
\begin{proposition}[\HOpp into \HOp: Type Preservation]
The encoding from $\tyl{L}_{\HOpp}$ into $\tyl{L}_{\HOp}$ (cf. \figref{f:enc:hopip_to_hopi})
is type preserving.
\end{proposition}}

%HERE WE NEED TO ADD AN EXCERPT OF \propref{app:prop:op_corr_HOpp_to_HOp}, BUT WE NEED TO FIX MACROS FIRST.

\begin{proposition}[Operational Correspondence: From \HOpp to \HOp - Excerpt] %\myrm
	\label{prop:op_corr_HOpp_to_HOp}
	Let $P$ be an \HOpp process such that $\Gamma; \es; \Delta \proves P$.
	\begin{enumerate}[1.]
		\item	Completeness: 
			$\horel{\Gamma}{\Delta}{P}{\hby{\ell}}{\Delta'}{P'}$ implies
			\begin{enumerate}[a)]
%				\item	If $\ell \in \set{\news{\tilde{m}} \bactout{n}{\abs{x}{Q}}, \bactinp{n}{\abs{x}{Q}}}$ then
%%					$\exists l' $ such that
%					$\horel{\tmap{\Gamma}{3}}{\tmap{\Delta}{3}}{\pmap{P}{3}}{\hby{\ell'}}
%					{\tmap{\Delta'}{3}}{\pmap{P'}{3}}$ with $\mapa{\ell}^{3} = \ell'$.
%				\item	If $\ell \notin \set{\news{\tilde{m}} \bactout{n}{\abs{x}{Q}}, \bactinp{n}{\abs{x}{Q}}, \tau}$ then
%					$\horel{\tmap{\Gamma}{3}}{\tmap{\Delta}{3}}{\pmap{P}{3}}{\hby{\ell}}
%					{\tmap{\Delta'}{3}}{\pmap{P'}{3}}$.
%
				\item	If $\ell = \btau$ then
					$\horel{\tmap{\Gamma}{3}}{\tmap{\Delta}{3}}{\pmap{P}{3}}{\hby{\tau}}
					{\Delta''}{R}$ and
					$\horel{\tmap{\Gamma}{3}}{\tmap{\Delta'}{3}}{\pmap{P'}{3}}{\hwb}{\Delta''}{R}$, for some $R$;

				\item	If $\ell = \tau$ and $\ell \not= \btau$ then %and $\hby{\ell}$ is not a \betatran then
					$\horel{\tmap{\Gamma}{3}}{\tmap{\Delta}{3}}{\pmap{P}{3}}{\hby{\tau}}
					{\tmap{\Delta'}{3}}{\pmap{P'}{3}}$.
			\end{enumerate}

		\item Soundness: $\horel{\tmap{\Gamma}{3}}{\tmap{\Delta}{3}}{\pmap{P}{3}}{\hby{\tau}}
			{\tmap{\Delta''}{3}}{Q}$ implies either 
%
			\begin{enumerate}[a)]
%				\item	If $\ell \in \set{\news{\tilde{m}} \bactout{n}{\abs{x}{Q}}, \bactinp{n}{\abs{x}{Q}}, \tau}$
%					then
%					$\horel{\Gamma}{\Delta}{P}{\hby{\ell'}}{\Delta'}{P'}$
%%					and $\horel{\tmap{\Gamma}{3}}{\tmap{\Delta''}{3}}{Q}{\hby{\hat{\ell}}}{\tmap{\Delta'}{3}}{\pmap{P'}{3}}$
%					with $\mapa{\ell'}^{3} = \ell$ and $Q \scong \pmap{P'}{3}$.
%
%				\item	If $\ell \notin \set{\news{\tilde{m}} \bactout{n}{\abs{x}{R}}, \bactinp{n}{\abs{x}{R}}, \tau}$
%					then
%					$\horel{\Gamma}{\Delta}{P}{\hby{\ell}}{\Delta'}{P'}$ and $Q \scong \pmap{P'}{3}$.
%%					and $\horel{\tmap{\Gamma}{3}}{\tmap{\Delta''}{3}}{Q}{\hby{\hat{\ell}}}{\tmap{\Delta'}{3}}{\pmap{P'}{3}}$.
%
				\item	
					$\horel{\Gamma}{\Delta}{P}{\hby{\tau}}{\Delta'}{P'}$ with $Q \scong \pmap{P'}{3}$
				\item 
					$\horel{\Gamma}{\Delta}{P}{\hby{\btau}}{\Delta'}{P'}$ and
					$\horel{\tmap{\Gamma}{3}}{\tmap{\Delta''}{3}}{Q}{\hby{\btau}}
					{\tmap{\Delta''}{3}}{\pmap{P'}{3}}$.
			\end{enumerate}
	\end{enumerate}
\end{proposition}




\begin{proposition}[Full Abstraction. From \HOpp to \HOp]%\myrm
	%\label{app:prop:fulla_HOpp_to_HOp}
	Let $P, Q$ be \HOpp processes with $\Gamma; \es; \Delta_1 \proves P \hastype \Proc$ and 
	$\Gamma; \es; \Delta_2 \proves Q \hastype \Proc$. \\
	Then 
	$\horel{\Gamma}{\Delta_1}{P}{\hwb}{\Delta_2}{Q}$ if and only if $\horel{\tmap{\Gamma}{3}}{\tmap{\Delta_1}{3}}{\pmap{P}{3}}{\hwb}{\tmap{\Delta_2}{3}}{\pmap{Q}{3}}$
\end{proposition}

\begin{figure}[t]
$
{%\small
\begin{array}{c}
	\multicolumn{1}{l}{\noindent{\bf Types:}}
	\\
	\tmap{\shot{L}}{3} \defeq \shot{\btinp{\tmap{L}{3}} \tinact}
	\qquad
	\tmap{\btout{\shot{L}} S}{3} \defeq \btout{\tmap{\shot{L}}{3}} \tmap{S}{3}
	\\
	\tmap{\btinp{\shot{L}} S}{3} \defeq \btinp{\tmap{\shot{L}}{3}} \tmap{S}{3}
	\\
%\hline
%\noindent{\bf Labels:} \  
%		\mapa{\news{\tilde{m}} \bactout{n}{\abs{\AT{x}{L}}{P}}}^{3} &\!\!\!\!\defeq\!\!\!\!& \news{\tilde{m}} \bactout{n}{\abs{z}{\binp{z}{x} \pmap{P}{3}}}
%		\\
%		\mapa{\bactinp{n}{\abs{\AT{x}{L}}{P}}}^{3} &\!\!\!\!\defeq\!\!\!\!& \bactinp{n}{\abs{z}{\binp{z}{x} \pmap{P}{3}}}
%\\
%\hline
	\multicolumn{1}{l}{\noindent{\bf Terms:}}
	\\
	\pmap{\appl{x}{(\abs{y} P)}}{3} \defeq \newsp{s}{\appl{x}{s} \Par \bout{\dual{s}}{\abs{y} \pmap{P}{3}} \inact}
	\qquad
	\pmap{\bout{u}{\abs{\AT{x}{L}}{Q}} P}{3} \defeq \bout{u}{\abs{z}{\binp{z}{x} \pmap{Q}{3}}} \pmap{P}{3}
	\\
%	\pmap{\bout{u}{\abs{x: C}{Q}} P}{3} &\!\!\!\!\defeq\!\!\!\!& \bout{u}{\abs{x}{\pmap{Q}{3}}} \pmap{P}{3}
	\pmap{\appl{(\abs{x}{P})}{(\abs{x}Q})}{3} \defeq \newsp{s}{\binp{s}{x} \pmap{P}{3} \Par  \bout{\dual{s}}{\abs{x} \pmap{Q}{3}} \inact}
%	[[(?x P_1) (?x P_2)]] = (? s)((?x [[P]])s | \bout{s}{?x P_2} 0 )
\end{array}
}
$

$\tmap{\lhot{L}}{3}$ is defined as $\tmap{\shot{L}}{3}$
by replacing $\shot{L}$ with~$\lhot{L}$.
Label and term mappings for $\abs{x:C}{P}$ are
%defined 
as in \figref{f:enc:hopi_to_ho}, replacing 
%The mapping of types for $\lhot{L}$ is defined by replacing 
%$\shot{L}$ by $\lhot{L}$. 
%The case of $\abs{x:C}{P}$ in the label and term mappings 
%are %defined 
%as in \figref{f:enc:hopi_to_ho}, replacing 
$\tmap{\cdot}{1}$,
$\mapa{\cdot}^{1}$, and 
$\pmap{\cdot}{1}_f$, by  
$\tmap{\cdot}{3}$,
$\mapa{\cdot}^{3}$, and 
$\pmap{\cdot}{3}$.
The other mappings for processes, types and labels are  homomorphic.

\caption{\label{f:enc:hopip_to_hopi} Encoding of \HOpp into \HOp.}
%\Hlinefig
\end{figure} 


Using the above propositions,  
\thmsref{f:enc:hopitoho}
and 
\ref{f:enc:hotopi},
and \propref{pro:composition}, 
we derive the following: % theorem:

%\smallskip 

\begin{theorem}[Encoding \HOpp into~\HOp]
	\label{f:enc:hopiptohopi}
	The encoding from $\tyl{L}_{\HOpp}$ into $\tyl{L}_{\HOp}$ (cf. \figref{f:enc:hopip_to_hopi})
	is precise. Hence, the encodings 
	from $\tyl{L}_{\HOpp}$ to $\tyl{L}_{\HO}$ 
	and $\tyl{L}_{\sessp}$ 
	are also precise. 
\end{theorem}
%\smallskip 



%\noi Embeddings of polyadic name passing into monadic name passing are
%well-studied. % in the literature. 
%Using a linear typing, precise
%encodings (including full abstraction) can be obtained~\cite{Yoshida96}.
%Here we summarise how $\PHOp$ can be encoded into $\HOp$. 
%The syntax of 
%$\HOp$ is extended %from \HOp by including 
%with
%polyadic name passing $\tilde{n}$ and $\abs{\tilde{x}}{Q}$ in the syntax 
%of value $V$. The type syntax is extended to: 
%
%\begin{center}
%\begin{tabular}{c}
%$
%L ::= \shot{\tilde{C}} \ | \ \lhot{\tilde{C}}
%\quad\quad S \ ::= \  \btout{\tilde{U}} S \bnfbar \btinp{\tilde{U}} S \bnfbar \cdots 
%$
%\end{tabular}
%\end{center}
%%
%The type system disallows a shared name that directly carries polyadic
%shared names as in \cite{tlca07,MostrousY15}.
%i.e. $\chtype{\tilde{\chtype{S}}}$ 
%and $\chtype{\tilde{\chtype{L}}}$ 
%are disallowed.
%Other definitions are straightforwardly extended. 
%\jpc{We slightly modify \defref{def:tenc} to capture that a 
%label $\ell$ may be mapped into a sequence of labels~$\tilde{\ell}$.}
%We extend the mapping for labels 
%($\mapa{\cdot}: \ell \to \tilde{\ell}$ in  
%\defref{def:tenc}) to capture 
%a sequence of labels  and 
%Also, \defref{def:ep} is kept unchanged, 
%assuming that if 
%$P \hby{\ell} P'$ and $\mapa{\ell} = \ell_1, \ell_2,  \cdots, \ell_m$ then
%$\map{P} \Hby{\mapa{\ell}} \map{P'}$
%should be understood as
%$\map{P} \Hby{\ell_1} P_1 \Hby{\ell_2} P_2 \cdots \Hby{\ell_m} P_m =  \map{P'}$,
%for some
%$P_1, P_2, \ldots, P_m$.

%\paragraph{Encoding $\PHOp$ into $\HOp$}
%\label{subsec:pho}
Second, we define the typed encoding
	$\enco{\map{\cdot}^{4}, \mapt{\cdot}^{4}}: \PHOp \to \HOp$ 
in \figref{f:enc:poltomon}.
For simplicity, we give the dyadic case (tuples of length 2);
the general case is as expected.
The encoding of $\PHOp$  satisfies the following properties:

%HERE WE NEED TO ADD AN EXCERPT OF \propref{app:prop:op_corr_pHOp_to_HOp}, BUT WE NEED TO FIX MACROS FIRST.

{
\begin{proposition}[\PHOp into \HOp: Type Preservation]
The encoding from
		$\tyl{L}_{\PHOp}$ into $\tyl{L}_{\HOp}$ (cf. \figref{f:enc:poltomon})
is type preserving.
\end{proposition}}

\begin{proposition}[Operational Correspondence: From \PHOp to \HOp - Excerpt]\myrm
	\label{prop:op_corr_pHOp_to_HOp}
Let $\Gamma; \es; \Delta \proves P$.
	\begin{enumerate}[1.]
		\item	Completeness: 
			$\horel{\Gamma}{\Delta}{P}{\hby{\ell}}{\Delta'}{P'}$ implies
%
			\begin{enumerate}[a)]
%				\item	If $\ell = \news{\tilde{m}'} \bactout{n}{\tilde{m}}$ then
%					$\horel{\tmap{\Gamma}{4}}{\tmap{\Delta}{4}}{\pmap{P}{4}}{\hby{\ell_1} \dots \hby{\ell_n}}{\tmap{\Delta'}{4}}{\pmap{P}{4}}$
%					with $\mapa{\ell}^{4} = \ell_1 \dots \ell_n$.
%
%				\item	If $\ell = \bactinp{n}{\tilde{m}}$ then
%					$\horel{\tmap{\Gamma}{4}}{\tmap{\Delta}{4}}{\pmap{P}{4}}{\hby{\ell_1} \dots \hby{\ell_n}}{\tmap{\Delta'}{4}}{\pmap{P}{4}}$
%					with $\mapa{\ell}^{4} = \ell_1 \dots \ell_n$.
%
%				\item	If $\ell \in \set{\news{\tilde{m}} \bactout{n}{\abs{\tilde{x}}{R}}, \bactinp{n}{\abs{\tilde{x}}{R}}}$ then
%%					$\exists l' $ such that
%					$\horel{\tmap{\Gamma}{4}}{\tmap{\Delta}{4}}{\pmap{P}{4}}{\hby{\ell'}}
%					{\tmap{\Delta'}{4}}{\pmap{P'}{4}}$ with $\mapa{\ell}^{4} = \ell'$.
%
%				\item	If $\ell \in \set{\bactsel{n}{l}, \bactbra{n}{l}}$ then
%					$\horel{\tmap{\Gamma}{4}}{\tmap{\Delta}{4}}{\pmap{P}{4}}{\hby{\ell}}
%					{\tmap{\Delta'}{4}}{\pmap{P'}{4}}$.

				\item	If $\ell = \btau$ then %either
					$\horel{\tmap{\Gamma}{4}}{\tmap{\Delta}{4}}{\pmap{P}{4}}{\hby{\btau} \hby{\stau}  \hby{\stau}}
					{\tmap{\Delta'}{4}}{\pmap{P'}{4}}$ %with $\mapa{\ell} = \btau, \stau \dots \stau$.

				\item	If $\ell = \tau$ then %and $\hby{\ell}$ is not a \betatran then
					$\horel{\tmap{\Gamma}{4}}{\tmap{\Delta}{4}}{\pmap{P}{4}}{\hby{\tau}\hby{\tau} \hby{\tau}}
					{\tmap{\Delta'}{4}}{\pmap{P'}{4}}$ %with $\mapa{\ell}^{4} = \tau \dots \tau$.
			\end{enumerate}

		\item	%Let $\Gamma; \es; \Delta \proves P$.
		Soundness: 
			$\horel{\tmap{\Gamma}{4}}{\tmap{\Delta}{4}}{\pmap{P}{4}}{\hby{\ell}}
			{\tmap{\Delta_1}{4}}{P_1}$ implies
%
			\begin{enumerate}[a)]
%				\item	If $\ell \in \set{\bactinp{n}{m}, \bactout{n}{m}, \news{m} \bactout{n}{m}}$ then
%					$\horel{\Gamma}{\Delta}{P}{\hby{\ell}}{\Delta'}{P'}$ and\\
%					$\horel{\tmap{\Gamma}{4}}{\tmap{\Delta_1}{4}}{P_1}{\hby{\ell_2} \dots \hby{\ell_n}}
%					{\tmap{\Delta'}{4}}{\tmap{P'}{4}}$ with $\mapa{\ell}^{4} = \ell_1 \dots \ell_n$.
%
%				\item	If $\ell \in \set{\news{\tilde{m}} \bactout{n}{\abs{x}{R}}, \bactinp{n}{\abs{x}{R}}}$
%					then
%					$\horel{\Gamma}{\Delta}{P}{\hby{\ell'}}{\Delta'}{P'}$
%					with $\mapa{\ell'}^{4} = \ell$ and $P_1 \scong \pmap{P'}{4}$.
%
%				\item	If $\ell \in \set{\bactsel{n}{l}, \bactbra{n}{l}}$
%					then
%					$\horel{\Gamma}{\Delta}{P}{\hby{\ell}}{\Delta'}{P'}$ and $P_1 \scong \pmap{P'}{4}$.
%%					and $\horel{\tmap{\Gamma}{3}}{\tmap{\Delta''}{3}}{Q}{\hby{\hat{\ell}}}{\tmap{\Delta'}{3}}{\pmap{P'}{3}}$.
%
				\item	If $\ell = \btau$ then
					$\horel{\Gamma}{\Delta}{P}{\hby{\btau}}{\Delta'}{P'}$ and
					$\horel{\tmap{\Gamma}{4}}{\tmap{\Delta_1}{4}}{P_1}{\hby{\stau} \hby{\stau}}
					{\tmap{\Delta'}{4}}{\tmap{P'}{4}}$ %with $\mapa{\ell}^{4} = \btau, \stau \dots \stau$.

				\item	If $\ell = \tau$ then
					$\horel{\Gamma}{\Delta}{P}{\hby{\tau}}{\Delta'}{P'}$ and
					$\horel{\tmap{\Gamma}{4}}{\tmap{\Delta_1}{4}}{P_1}{\hby{\tau} \hby{\tau} \hby{\tau}}
					{\tmap{\Delta'}{4}}{\tmap{P'}{4}}$ %with $\mapa{\ell}^{4} = \tau \dots \tau$.
			\end{enumerate}
	\end{enumerate}
\end{proposition}




\begin{proposition}[Full Abstraction: From \PHOp to \HOp]%\myrm
	%\label{app:prop:fulla_pHOp_to_HOp}
	Let $P, Q$ be \PHOp processes with $\Gamma; \es; \Delta_1 \proves P \hastype \Proc$ and 
	$\Gamma; \es; \Delta_2 \proves Q \hastype \Proc$. Then we have: \\
	$\horel{\Gamma}{\Delta_1}{P}{\hwb}{\Delta_2}{Q}$ if and only if $\horel{\tmap{\Gamma}{4}}{\tmap{\Delta_1}{4}}{\pmap{P}{4}}{\hwb}{\tmap{\Delta_2}{4}}{\pmap{Q}{4}}$.
\end{proposition}

%Based on these propositions,we have:
Using the above propositions,  
\thmsref{f:enc:hopitoho}
and 
\ref{f:enc:hotopi},
and \propref{pro:composition}, 
we derive the following: 
\begin{theorem}[Encoding of \PHOp into \HOp]
	\label{f:enc:phopiptohopi}
	The encoding from
		$\tyl{L}_{\PHOp}$ into $\tyl{L}_{\HOp}$ (cf. \figref{f:enc:poltomon})
	is precise. 
	Hence, the encodings 
	from
	$\tyl{L}_{\PHOp}$ to $\tyl{L}_{\HO}$ 
	and $\tyl{L}_{\sessp}$ 
	are also precise. 
\end{theorem}
By combining Thms.~\ref{f:enc:hopiptohopi} and~\ref{f:enc:phopiptohopi},
we can extend preciseness to the super-calculus
$\PHOpp$.
% (denoted by   in Fig.~\ref{fig:express}) 


% !TEX root = ../journal16kpy.tex

\begin{figure}[t]
{\bf Terms:} 
\begin{align*}
	 \pmap{\bout{u}{u_1, u_2} P}{4} &\defeq \bout{u}{u_1} \bout{u}{u_2} \pmap{P}{4}
	\\
	 \pmap{\bbout{u}{\abs{x_1, x_2} Q} P}{4} &\defeq \bbout{u}{\abs{z}\binp{z}{x_1} \binp{z}{x_2} \pmap{Q}{4}} \pmap{P}{4} %\qquad \text{($z$ fresh in $Q$)}
	\\
 \pmap{\appl{x}{(u_1, u_2)}}{4} &\defeq \newsp{s}{\appl{x}{s} \Par \bout{\dual{s}}{u_1}   \bout{\dual{s}}{u_2} \inact}
	\\
	\pmap{\appl{(\abs{x_1,x_2}{P})}{(u_1, u_2)}}{4} &\defeq
	\newsp{s}{\binp{s}{x_1}  \binp{s}{x_2} \pmap{P}{4} \Par \bout{\dual{s}}{u_1}  \bout{\dual{s}}{u_2} \inact} 
\end{align*}
{\bf Types:}
\begin{align*}
		\tmap{\btout{S_1, S_2}S}{4} &\defeq \btout{\tmap{S_1}{4}}  \btout{\tmap{S_2}{4}}\tmap{S}{4}
	\\
	 \tmap{\bbtout{L} S}{4} & \defeq  \bbtout{\mapt{L}^{4}}\mapt{S}^{4}
	\\
	  \tmap{\shot{(C_2,  C_2)}}{4} &\defeq \shot{\big(\btinp{\tmap{C_1}{4}} \btinp{\tmap{C_2}{4}}\tinact\big)}
	\\
	  \tmap{\lhot{(C_1,  C_2)}}{4} &\defeq \lhot{\big(\btinp{\tmap{C_1}{4}}  \btinp{\tmap{C_2}{4}}\tinact\big)}
\end{align*}
%We give the dyadic case; the general polyadic case is as expected.
The input cases are defined as the output cases by replacing $!$ by $?$. 
%Also, $\mapa{\tau}^4 =\tau, \tau$.
Elided mappings for  processes and types are 
homomorphic.
%\vspace{-2mm}
\caption{\label{f:enc:poltomon}Encoding of \PHOp (dyadic case) into \HOp. }
%\Hlinefig
%\vspace{-1mm} 
\end{figure}




%%%%%%%%%%%%%%%%%%%%%%%%%%%%%%%%%%%%%%%%%%%%%%%%%%%%%%%%%%%%%%%%%%%%%%%%%%%%%%%%
%%%%%%%%%%%%%%%%%%%%%%%%%%%%%%%%%%%%%%%%%%%%%%%%%%%%%%%%%%%%%%%%%%%%%%%%%%%%%%%%


\section{Concluding Remarks and Related Work}
\label{sec:relwork}
%% !TEX root = main.tex
\myparagraph{Expressiveness of Process Calculi.}
There is a vast literature on expressiveness studies for process calculi. 
Here we concentrate on closely related work; 
see~\cite{KouzapasPY15} for more detailed related work. 
To substantiate claims related to (relative) expressive power,
early works appealed to different definitions of encoding \cite{Palamidessi03}.
Later on, 
proposals of abstract 
frameworks which 
state associated syntactic and semantic criteria 
were put forward; 
two proposals are~\cite{DBLP:journals/iandc/Gorla10,DBLP:journals/tcs/FuL10}. 
These frameworks are applicable to different calculi, and 
have shown useful to clarify known results and to derive new ones.
Our formulation of (precise) typed encoding (Def.~\ref{def:goodenc}) 
builds upon existing proposals (including~\cite{Palamidessi03,DBLP:journals/iandc/Gorla10,DBLP:conf/icalp/LanesePSS10})
in order to account for the session type systems
associated to \HOp and its variants/extensions.

\myparagraph{Expressiveness of Higher-Order Process Calculi.}
Due to the close relationship between
higher-order process calculi and functional calculi, works devoted to
encoding (variants of) the $\lambda$-calculus into (variants of) the
$\pi$-calculus~ (e.g.,~\cite{San92,DBLP:journals/tcs/Fu99}) are broadly related.
The encoding of process passing into name passing is well-known~\cite{SangiorgiD:expmpa};
the encoding of in the reverse direction 
is studied in~\cite{SaWabook} for an asynchronous, localised $\pi$-calculus
(only the output capability of names can be sent around).  The
work~\cite{San96int} studies hierarchies for calculi with
\emph{internal} first-order mobility and with higher-order mobility
without name-passing (similarly as \HO). The
hierarchies are based on expressivity: formally defined according to
the order of types needed in typing, they describe different ``degrees
of mobility''.  Via fully abstract encodings, it is shown that that
name- and process-passing calculi with equal order of types have the
same expressiveness.  With respect to these previous results, our
approach based on session types has several important consequences and
allows us to derive new results.  Our study reinforces the intuitive
view of ``encodings as protocols'', namely session protocols which
enforce precise linear and shared disciplines for names, a distinction
not investigated in~\cite{SangiorgiD:expmpa,DBLP:journals/tcs/Sangiorgi01}. In
turn, the linear/shared distinction is central in proper definitions
of trigger processes, and are essential to encodings
(Def.~~\ref{d:enc:hopitopi}) and behavioral equivalences
(Defs.~\ref{d:hbw} and~\ref{d:fwb}).  More interestingly, we showed that
$\HO$ suffices to encode  the session
calculus with name passing ($\sessp$) but also $\HOp$ and its extension with
higher-order applications ($\HOpp$). Thus, using session types
all these calculi are shown to be equally expressive with fully
abstract encodings.  To our knowledge, these are the first
expressivity results of this kind.

The work~\cite{XuActa2012} studies the encodability of the higher-order $\pi$-calculus (extended with a relabeling operator) into the first-order $\pi$-calculus; encodings in the reverse direction are also proposed, following \cite{Tho90}.
A minimal calculus of higher-order concurrency is studied in~\cite{DBLP:journals/iandc/LanesePSS11}: it lacks restriction,  name passing, output prefix (so  communication is asynchronous), and constructs for infinite behavior. 
This calculus (a sublanguage of \HO) has 
a simple notion of (strong) bisimilarity which coincides with barbed congruence.
%be Turing complete, while 
%have a decidable notion of (strong) bisimilarity that coincides with barbed congruence. 

Our work is closely related in spirit to the expressiveness studies in~\cite{DBLP:conf/icalp/LanesePSS10,DBLP:conf/wsfm/XuYL13}.
In~\cite{DBLP:conf/icalp/LanesePSS10}
the core calculus in~\cite{DBLP:journals/iandc/LanesePSS11} is extended with restriction, output prefix (thus enabling synchronous communication), 
and polyadic communication. It is shown that 
synchronous communication can encode asynchronous communication, % (as in the first-order setting),
and that process passing polyadicity induces a hierarchy in expressive power, % (unlike the first-order setting).
A further extension with process abstractions of order one
(functions from processes to processes)
 is shown to strictly add expressive power with respect to passing of processes only.
The paper~\cite{DBLP:conf/wsfm/XuYL13} complements the study in~\cite{DBLP:conf/icalp/LanesePSS10} by deepening on the expressive power of second-order abstractions (similar to \HO). 
In that setting, name and process abstractions are distinguished and contrasted, also considering polyadicity of abstraction parameters (the same kind of polyadicity present in \pHOp). It is shown that polyadicity of process abstraction induces an expressiveness hierarchy. Moreover, it is shown that name abstraction can encode process abstraction, and therefore it may be considered as a more basic mechanism. 
The works~\cite{DBLP:conf/icalp/LanesePSS10,DBLP:conf/wsfm/XuYL13} focus on untyped processes;
therefore, our work complements their previous results by clarifying the status of typeful, resource-aware structured communications in
trigger-based representations of process passing, both in encodings and  equivalences.

\myparagraph{Session Typed Processes.}
Two works~\cite{DemangeonH11,Dardha:2012:STR:2370776.2370794} 
study encodings of binary session calculi into 
a linearly typed $\pi$-calculus. 
\cite{DemangeonH11}~gives a precise encoding of \sessp into a linear calculus 
based on~\cite{BHY},  
while~\cite{Dardha:2012:STR:2370776.2370794} 
gives the operational correspondence 
(without a full abstraction result)
for the first- and higher-order 
$\pi$-calculus (\cite{tlca07}) into \cite{LinearPi}. 
They investigate an embeddability of two different typing systems;
by the result of \cite{DemangeonH11}, \HOpp is further encodable precisely 
into the linearly typed $\pi$-calculi.     

The discipline developed for the $\HOp$ is a subset
of that in~\cite{tlca07,MostrousY15}.
\cite{tlca07} develops a full higher-order session calculus
with process abstractions and applications, hence  
treats the type 
$U=U_1 \rightarrow U_2 \dots U_n \rightarrow \Proc$ and its linear type 
$U^1$
which corresponds $\shot{\tilde{U}}$ and $\lhot{\tilde{U}}$ in 
a super-calculus of $\HOpp$ and $\PHOp$. 
%in~\cite{MostrousY15} in the asynchronous setting.
%The session type
%system considered is influenced by the type systems for $\lambda$-calculi and
%uses type syntax of the form $U_1 \rightarrow U_2 \dots U_n \rightarrow \Proc$
%for shared values and $(U_1 \rightarrow U_2 \dots U_n \rightarrow \Proc)^{1}$
%for linear values.
%Such a type is expressed in $\HOpp$
%terms using the type $\shot{U}$ (respectively, $\lhot{U}$)
%with $U$ being a nested higher-order type; and 
%the $\HOp$ uses only types of the form
%$\shot{C}$ and $\lhot{C}$ with $C$ being a first-order channel type.
Our results show
the calculus in~\cite{tlca07} is not only expressed but 
also reasoned in 
$\HO$ (with limited form of arrow types, $\shot{C}$ and $\lhot{C}$) and 
$\sessp$, via precise encodings. 

\myparagraph{Typed Behavioural Equivalences.}
The current work follows the principles for
session type behavioural semantics that were laid
by the previous works of the
authors~\cite{dkphdthesis,KYHH2015,KY2015,DBLP:journals/iandc/PerezCPT14}.
A bisimulation relation is defined on a labelled
transition system that assumes a session typed
observer.
The bisimilarity is characterised by the corresponding
reduction-closed, barb-preserving congruence using a
proof technique that is derived from~\cite{Hennessy07}.
The theory for higher-order session types developed here
differentiates from 
the work in~\cite{dkphdthesis,KYHH2015,KY2015}, which 
considers the behavioural semantics for first-order
binary and multiparty session types.
Also the work \cite{DBLP:journals/iandc/PerezCPT14} studies typed behavioral equivalencies for binary session types.
The underlying process languages does not have shared names which, as we have shown, strictly add expressive power. 
Moreover, for this deterministic language, confluence and $\tau$-inertness properties are established.

%The theory for higher-order session type quivalences is more challenging than
%their corresponding first-order bisimulation theory.
To cope with the challenges presented by the higher-order
session theory, 
our approach continues the line of research 
originally drawn by Sangiorgi~\cite{San96H,SangiorgiD:expmpa}
and later improved by Jeffrey and Rathke~\cite{JeffreyR05}.
The works %Sangiorgi as part of his Ph.D.~research
\cite{San96H,SangiorgiD:expmpa}
introduced the first fully abstract encoding from the higher-order 
$\pi$-calculus to the $\pi$-calculus. 
The replicated triggered process 
is also used in this work for the encoding of \HOp into \sessp (Definition~\ref{d:enc:hopitopi}).
Sangiorgi's encoding is based on the idea of a replicated input guarded process 
(called a trigger process). Operational correspondence for
the triggered encoding is shown using the contextual bisimulation
with first-order labels.
Although contextual bisimilarity has a satisfactory discriminative power,
its use is hindered by the universal quantification on output clauses.
Sangiorgi then proposed \emph{normal bisimilarity}, a tractable  equivalence 
on processes without universal quantification. 
To show the coincidence between contextual and normal bisimilarities, 
the use of triggered processes and bisimilarity is developed in \cite{San96H}.
%The encoding also motivates the definition of a form of
Triggered bisimulation is also defined on first-order labels
where the contextual bisimulation is restricted to arbitrary
trigger substitution rather than arbitrary process substitutions.
The triggered bisimulation was further refined by Jeffrey and
Rathke, who study calculi with recursive types, not addressed in~\cite{San96H,SangiorgiD:expmpa} and
relevant in our work.
They introduce their own version of a
bisimulation~\cite{JeffreyR05}
based on a LTS which is extended with trigger meta-notation.
%for a full higher-order $\pi$-calculus that allows
%higher-order applications.
Like Sangiorgi's approach, the labelled transition semantics
in~\cite{JeffreyR05}
observes first-order triggered values instead of
higher-order values, offering a more direct characterization of contextual equivalence
and lifting the restriction to finite types.


There are similarities and differences between of the characteristic bisimulation
and the bisimulation as defined by Jeffrey and Rathke
(below we use the meta-notation adopted in~\cite{JeffreyR05}):
%
\begin{enumerate}[i)]
	\item	The output of a higher-order value $\abs{x}{Q}$ on name
		$n$ in Jeffrey\&Rathke approach requires the output of
		a fresh trigger name $t$ (notation $\tau_t$)
		on channel $n$ 
		and then the introduction of a replicated triggered process
		(notation $(t \Leftarrow (x) Q)$)
		in the context of the acting process:
		%
		\[
			P \by{\news{t} \bactout{n}{\tau_{t}}} P' \Par (t \Leftarrow (x) Q) \by{\bactinp{t}{v}} P' \Par \appl{(x) Q}{v} \Par (t \Leftarrow (x) Q) 
		\]
		%
		In the characteristic bisimulation approach we only observe
		an output of a value that can be either first- or higher-order:
		%
		\[
			P \hby{\bactout{n}{V}} P' 
		\]
		%
		with $V = \abs{x}{Q}$ or $V = m$.
		A non-replicated triggered process appears in
		the parallel context of the acting process when
		we compare two processes for behavioural equality
		(cf.~characteristic bisimulation Definition~\ref{d:fwb}),
		$P' \Par \htrigger{t}{\abs{x}{Q}}$.
		In fact using the LTS in
		Definition~\ref{d:tlts} we can get:
		%
		\begin{eqnarray*}
			P' \Par \htrigger{t}{\abs{x}{Q}}
			&\by{\abs{z}{\binp{z}{y} \repl{} \binp{t}{x} \appl{y}{x}}}&
			P' \Par \newsp{s}{\binp{s}{y} \repl{} \binp{t}{x} \appl{y}{x} \Par \bout{s}{\abs{x}{Q}} \inact}\\
			&\by{\tau}&
			P' \Par \repl{}\binp{t}{y} \appl{\abs{x}{Q}}{y}
		\end{eqnarray*}
		%
		that simulates the Jeffrey\&Rathke approach.

		The characteristic bisimulation differentiates from
		the Jeffrey\&Rathke approach:
		\begin{enumerate}[$\bullet$]
			\item	The typed LTS predicts the case of linear
				output values and will never allow replication
				of such a value;
				if $V$ is linear the input action would have no replication
				operator, as
				$\abs{z}{\binp{z}{y} \binp{t}{x} \appl{y}{x}}$.

			\item	The characteristic bisimulation introduces a uniform approach
				not only for
				higher-order values but for first-order values
				as well. A triggered process can accept any
				process that can substitute a first-order value as well.
				This is derived from the fact that the $\HOp$
				calculus makes no use of a matching operator, in contrast
				to the calculus defined in~\cite{JeffreyR05})
				where name matching is crucial to prove completness
				of the bisimilarity relation.
				We chose not to include the matching operator
				because of the requirement of a minimal calculus.
				In the lack of matching we use types to inhabit
				a value so we can observe its simplest interaction
				with the process environment.

			\item	The \HOp calculus requires only first-order
				applications. Higher-order applications,
				as in the Jeffrey\&Rathke work,
				are presented as an extension in the \HOpp
				calculus.

			\item	The trigger process is non-replicated. In fact
				the trigger process transforms guards the output
				value with a higher-order input prefix. The
				functionality of the input is used to
				simulate the contextual bisimilarity that subsumes
				the replicated trigger approach.
				The transformation of an output action as an input
				action allows for treating an output
				using the restricted LTS~\ref{def:rlts}:
				%
				\[
					P' \Par \htrigger{t}{\abs{x}{Q}} \hby{\bactinp{t}{\abs{x}{\mapchar{U}{x}}}}
					P' \Par \news{s}{ \appl{\mapchar{U}{x}}{s} \Par \bout{\dual{s}}{\abs{x}{Q}} \inact}
				\]
		\end{enumerate}
		%
		%In essence we are transforming a replicated trigger into a process
		%that is input-prefixed on a fresh name that receives a higher-order
		%value;

	\item	The input of a higher-order value in the Jeffrey\&Rathke approach requires 
		the input of a fresh trigger name, which is substituted on the application
		variable, thus having a meta-suntax for triggered application instead
		of higher-order applications:
		%
		\[
			\binp{n}{x} P \by{\bactinp{n}{\tau_k}} \appl{\abs{x}{P}}{\tau_k} \by{\tau} P \subst{x}{\tau_k} 
		\]
		%
		with every instance of process variable $x$ in $P$ being substituted
		with trigger value $\tau_k$ to give a process of the form $\appl{\tau_k}{x}$.
		The approach in the characteristic bisimulation observes the
		triggered value
		$\abs{z}\binp{t}{x} \appl{x}{z}$ as an input instead of the
		trigger name:
		%
		\[
			P \hby{\bactinp{n}{\abs{z}\binp{t}{x} \appl{x}{z}}} P \subst{\abs{z}\binp{t}{x} \appl{x}{z}}{x}
		\]
		%
		with applications being transformed to
		$\abs{z}{\binp{t}{x} \appl{x}{z}}{v}$
		Note that in the characteristic bisimulation semantics
		we can also observe a characteristic process as an input.
		
	\item 	Triggered application in the Jeffrey\&Rahtke
		are observe using an output
		lead into an output observation of the
		application value over
		the fresh trigger name.
		%
		\[
			\appl{\tau_k}{v} \by{\bactout{k}{v}} \inact
		\]
		%
		In the characteristic bisimulation instead of observing an 
		application and its value as an action we observe:
		i) the name of the trigger through the trigger value
		application; and ii) the application
		value by inhabiting it in the characteristic value
		and observing the interaction of the corresponding
		characteristic process with its environment.
		%
		\begin{eqnarray*}
			\appl{\abs{z}{\binp{t}{x} \appl{x}{z}}}{v} &\by{\tau}& \binp{t}{x} \appl{x}{v}
			\by{\bactinp{t}{\abs{x}{\mapchar{U}{x}}}}
			\appl{\abs{x}{\mapchar{U}{x}}}{v}
			\by{\tau} \mapchar{U}{x} \subst{n}{x}
		\end{eqnarray*}
		%
\end{enumerate}

%The main differences of the triggered
%bisimulation approach comparing to our approach are:
%i) We use observe higher-order values on the LTS in contrast to first-order 
%values in~\cite{DBLP:journals/lmcs/JeffreyR05}.
%ii) In our approach we avoid the replicated triggered process,
%by transforming the output process into a higher-order guarded input.
%iii) The triggered bisimulation gives semantics for higher-order application,
%whereas in our approach we give semantics for first-order applications
%and show that higher-order applications are fully encodable.

%Boreale and Sangiorgi, 
%Deng and Hennessy, 
%Jeffrey and Rathke, Hennessy and Koutavas, Schmitt and Lenglet, Pi\E9rard and Sumii.
%Perez et al (bisimilarities for binary sessions), Kouzapas and Yoshida (bisimilarities for binary and multiparty sessions).
%Bisimilarities for HO processes: \cite{Xu07}.

Sangiorgi et al.~\cite{DBLP:conf/lics/SangiorgiKS07}, use a higher-order LTS 
to define an arguably complex bisimulation relation that stores the knowledge known to
the observer, thus the observation actions are based on the observer's knowledge
at any given time. 
The environmental bisimulation approach is simplified by Koutavas and
Hennessy in~\cite{DBLP:journals/cl/KoutavasH12,DBLP:conf/esop/KoutavasH11}
with the introduction
of a mapping from constants to higher-order values. This
technique allows for the observation of first-order values instead
of higher-order values. It differs from the approaches
in~\cite{San96H,JeffreyR05} because the
mapping between higher and first order values is no longer implicit.







We have thoroughly studied the expressivity of the higher-order $\pi$-calculus with sessions,
here denoted $\HOp$.
Unlike most previous works, 
%on the expressivity of (higher-order) process calculi, 
we have carried out our study in the setting of \emph{session types}. % for structured communications.
Types not only delineate and enable encodings; they 
inform the techniques required to reason about such encodings.
Our results cover a wide spectrum of features intrinsic to higher-order concurrency:
pure process-passing (with first- and higher-order abstractions), name-passing, polyadicity, 
linear/shared communication (cf.~\figref{fig:express}). 
Remarkably, the discipline embodied by 
session types turns out to be fundamental to show that all these languages are equally expressive, up to 
 strong typed bisimilarities. Indeed, although our encodings may be used in an untyped setting,
session type information is critical to establish key properties for preciseness, in particular full abstraction.

\paragraph{Related Work.}
There is a vast literature on expressiveness for process calculi; we refer to~\cite{DBLP:journals/entcs/Parrow08} 
and \cite[\S\,2.3]{PerezPhD10} for surveys.
%, both first- and higher-order. 
%For space reasons here 
%Below we concentrate on closely related work.
%In the untyped setting, the relative expressiveness of name-passing calculi with respect to 
%higher-order languages %(process- and abstraction-passing) 
%is well-known. 
Our study casts known results~\cite{SangiorgiD:expmpa} into a session typed setting, and
offers new encodability results.
Our work stresses the 
view of ``encodings as protocols'', namely session protocols which
enforce linear and shared disciplines for names, a distinction
little explored in %~\cite{SangiorgiD:expmpa,DBLP:journals/tcs/Sangiorgi01}. 
previous works.
This distinction %is key, as  
%in proper definitions of trigger processes, which are key to encodings (\defref{d:enc:hopitopi}). % and behavioural equivalences (\defref{d:hbw} and~\ref{d:fwb}).  
%in our technical developments: % in particular, 
%it 
enables us to obtain 
 refined 
operational correspondence results (cf. \propsref{prop:op_corr_HOp_to_HO}, \ref{prop:op_corr_HOp_to_p}, \ref{prop:op_corr_HOpp_to_HOp}, \ref{prop:op_corr_pHOp_to_HOp}).
We showed that
$\HO$ suffices to encode   the first-order session
calculus~\cite{honda.vasconcelos.kubo:language-primitives}, here denoted~\sessp. % with name passing ($\sessp$).
To our knowledge, this is a new result; %for session typed calculi: 
its significance is stressed by the demanding encodability criteria  considered, in particular full abstraction up to typed bisimilarities
($\hwb$/$\fwb$, cf. \propsref{prop:fulla_HOp_to_HO} and~\ref{prop:fulla_HOp_to_p}).
This encoding is relevant in a broader setting, as known encodings 
of name-passing into higher-order calculi~\cite{SaWabook,BundgaardHG06,DBLP:journals/entcs/MeredithR05,XuActa2012,DBLP:journals/corr/XuYL15}  require limitations
in source/target languages,
do not consider types,
 and/or fail to satisfy strong encodability criteria (see below). 
We also showed that $\HO$ can encode $\HOp$ and its extension with
higher-order applications ($\HOpp$). 
Thus, %using session types
all these  calculi are equally expressive with fully
abstract encodings (up to $\hwb$/$\fwb$).  
These appear to be the first results of this kind.

Early works on (relative) expressiveness appealed to different notions of encoding.
Later on, 
proposals of abstract 
frameworks which formalise the notion of encoding 
and state associated syntactic/semantic criteria 
were put forward; 
recent proposals include~\cite{DBLP:journals/iandc/Gorla10,DBLP:journals/tcs/FuL10,DBLP:journals/corr/abs-1208-2750,DBLP:conf/esop/PetersNG13,DBLP:journals/corr/PetersG15}. 
%These frameworks %are applicable to different calculi, and 
%have been used to clarify known results and to derive new ones.
Our formulation of precise encoding (\defref{def:goodenc}) 
builds upon existing proposals (e.g.,~\cite{Palamidessi03,DBLP:journals/iandc/Gorla10,DBLP:conf/icalp/LanesePSS10})
 to account for the session types
associated to \HOp. % and its variants.


Early expressiveness studies for higher-order calculi are~\cite{Tho90,SangiorgiD:expmpa}; 
recent works include~\cite{BundgaardHG06,DBLP:conf/icalp/LanesePSS10,DBLP:journals/iandc/LanesePSS11,XuActa2012,DBLP:conf/wsfm/XuYL13}.
Due to the close relationship between higher-order process calculi and functional calculi, 
%works devoted to 
encodings of (variants of) the $\lambda$-calculus into the $\pi$-calculus (see, e.g.,~\cite{San92,DBLP:journals/tcs/Fu99,DBLP:journals/iandc/YoshidaBH04,BHY,DBLP:conf/concur/SangiorgiX14}) are also related.
Sangiorgi's encoding of the higher-order $\pi$-calculus
into the  $\pi$-calculus~\cite{SangiorgiD:expmpa} 
is fully abstract with respect to reduction-closed, barbed congruence. 
We have shown in \secref{subsec:HOp_to_sessp} that the analogue of Sangiorgi's encoding for the session typed setting also satisfies full abstraction (up to $\hwb$/$\fwb$, cf. \propref{prop:op_corr_HOp_to_p}).
A basic form of input/output types is used in~\cite{DBLP:journals/tcs/Sangiorgi01}, where the encoding in~\cite{SangiorgiD:expmpa} is casted in the asynchronous setting, with output and applications coalesced in a single construct. Building upon~\cite{DBLP:journals/tcs/Sangiorgi01}, 
a simply typed encoding for synchronous processes is given in~\cite{SaWabook}; the reverse encoding (i.e.,  first-order communication into higher-order processes) is also studied  for an asynchronous, localised $\pi$-calculus (only the output capability of names can be sent around).
The work~\cite{San96int} studies hierarchies for calculi with \emph{internal} first-order mobility and 
with higher-order mobility without name-passing (similarly as the subcalculus \HO). 
The hierarchies are
defined according to the order of types needed in typing.
Via fully abstract encodings, it is shown that that name- and process-passing calculi with equal order of types have the same expressiveness.

%With respect to these previous results, our approach based on session types 
%has several important consequences and allows us to derive new results.  Our study reinforces the intuitive view of ``encodings as protocols'', namely session protocols which enforce precise linear and shared disciplines for names, a distinction not investigated in~\cite{SangiorgiD:expmpa,DBLP:journals/tcs/Sangiorgi01}. 
%In turn, the linear/shared distinctionbs central in proper definitions of trigger processes, which are essential to encodings and behavioural equivalences.
%More interestingly, we showed that $\HO$, a  minimal higher-order session calculus (no name passing, only first-order application) suffices to encode $\sessp$ (the session calculus with name passing) but also 
%$\HOp$  and 
%its extension  with higher-order applications (denoted $\HOpp$). 
%Thus, using session types all these calculi are shown to be equally expressive with fully abstract encodings.
%To our knowledge, these are the first expressiveness results of this kind.

Other related works are~\cite{BundgaardHG06,DBLP:journals/entcs/MeredithR05,XuActa2012,DBLP:journals/iandc/LanesePSS11}.
The paper~\cite{BundgaardHG06} gives a fully abstract %, contin\-u\-a\-tion-passing style 
encoding of the 
$\pi$-calculus into Homer, a higher-order  calculus with explicit locations, local names, and nested locations.
The paper~\cite{DBLP:journals/entcs/MeredithR05}
presents a \emph{reflective}  calculus with a ``quoting'' operator: names are quoted processes and represent the code 
of a process; name-passing is then a way of passing the code of a process. This reflective calculus 
can encode both first- and higher-order $\pi$-calculus.
Building upon~\cite{ThomsenB:plachoasgcfhop},
the work~\cite{XuActa2012} studies 
the (non)en\-co\-da\-bi\-lity of the untyped $\pi$-calculus into 
a higher-order $\pi$-calculus with a powerful 
name relabelling operator, which is 
%shown to be 
essential to encode name-passing. 
The paper~\cite{DBLP:journals/corr/XuYL15} defines an encoding of the (untyped) $\pi$-calculus 
without relabeling. 
This encoding is quite different from the one in~\secref{subsec:HOpi_to_HO}:
in~\cite{DBLP:journals/corr/XuYL15} names are encoded using polyadic name abstractions (called \emph{pipes}); 
guarded replication %(rather than recursion) 
enables infinite behaviours.
While our encoding satisfies full abstraction, % (\propref{prop:fulla_HOp_to_HO}), 
the encoding in~\cite{DBLP:journals/corr/XuYL15} does not: only 
divergence-reflection
and 
operational correspondence (soundness and completeness) properties 
are established.
Soundness is stated up-to \emph{pipe-bisimilarity}, an equivalence tailored to the encoding strategy;
the authors of~\cite{DBLP:journals/corr/XuYL15} describe this result as ``weak''.
%this result is described by the authors of~\cite{DBLP:journals/corr/XuYL15} as ``weak''.


A core higher-order calculus is studied in~\cite{DBLP:journals/iandc/LanesePSS11}: it lacks restriction,  name passing, output prefix, % (asynchronous communication), 
and %constructs for infinite behaviour.
replication/recursion. 
Still, this  subcalculus of \HO is Turing equivalent.
The work~\cite{DBLP:conf/icalp/LanesePSS10}
extends this core calculus with restriction,
output prefix, and polyadicity; it shows that 
synchronous communication can encode asynchronous communication, % (as in the first-order setting),
and that process passing polyadicity induces an expressiveness  hierarchy. % (unlike the first-order setting).
%A further extension with process abstractions of order one
%(functions from processes to processes)
% is shown to strictly add expressive power with respect to passing of processes only.
The paper~\cite{DBLP:conf/wsfm/XuYL13} 
complements~\cite{DBLP:conf/icalp/LanesePSS10} 
by studying the expressivity %of second-order abstractions.
%with replication ($!P$).  
%The work \cite{DBLP:conf/wsfm/XuYL13} focuses  
%%name and process abstractions are distinguished and contrasted, also 
%on expressiveness of the hirarchy of polyadic abstraction parameters. 
%(the same kind of polyadicity present in \pHOp)
%By adapting the encodings in~\cite{DBLP:conf/icalp/LanesePSS10} 
%Polyadicity 
of 
second-order process abstractions.
Polyadicity is shown to induce an expressiveness hierarchy; 
also,
by adapting the encoding in~\cite{SangiorgiD:expmpa},
process abstractions are encoded into name abstractions.
In contrast, here we 
give a fully abstract encoding of
 \PHOpp into \HO that preserves session types; this improves~\cite{DBLP:conf/icalp/LanesePSS10,DBLP:conf/wsfm/XuYL13}   
by enforcing linearity disciplines on process behaviour.
The focus of~\cite{DBLP:conf/icalp/LanesePSS10,XuActa2012,DBLP:conf/wsfm/XuYL13,DBLP:journals/corr/XuYL15} is on 
untyped, higher-order processes; they
%Moreover,~\cite{DBLP:conf/icalp/LanesePSS10,DBLP:conf/wsfm/XuYL13}
do not address communication disciplined by 
(session) type systems.
%therefore, our work complements their  results. 
% by clarifying the status of typeful %, resource-aware 
%structured communications. % in trigger-based representations of process passing, both in encodings and  equivalences.



Within session types, the works~\cite{DemangeonH11,Dardha:2012:STR:2370776.2370794} 
encode binary sessions into a linearly typed $\pi$-calculus. 
While~\cite{DemangeonH11}~gives an encoding of \sessp into a linear calculus 
(an extension of \cite{BHY}),  
the work~\cite{Dardha:2012:STR:2370776.2370794} 
gives  operational correspondence (without full abstraction)
for the first- and higher-order 
$\pi$-calculi into \cite{LinearPi}. 
%They investigate an embeddability of two different typing systems;
By the result of \cite{DemangeonH11}, 
\HOpp is encodable  into the linearly typed $\pi$-calculi.     
The syntax of $\HOp$ is a subset of that in~\cite{tlca07,MostrousY15}.
The work~\cite{tlca07} develops a higher-order session calculus
with process abstractions and applications; it admits the type 
$U=U_1 \rightarrow U_2 \dots U_n \rightarrow \Proc$ and its linear type 
$U^1$
which corresponds to $\shot{\tilde{U}}$ and $\lhot{\tilde{U}}$ in 
a super-calculus of $\HOpp$ and $\PHOp$. 
%in~\cite{MostrousY15} in the asynchronous setting.
%The session type
%system considered is influenced by the type systems for $\lambda$-calculi and
%uses type syntax of the form $U_1 \rightarrow U_2 \dots U_n \rightarrow \Proc$
%for shared values and $(U_1 \rightarrow U_2 \dots U_n \rightarrow \Proc)^{1}$
%for linear values.
%Such a type is expressed in $\HOpp$
%terms using the type $\shot{U}$ (respectively, $\lhot{U}$)
%with $U$ being a nested higher-order type; and 
%the $\HOp$ uses only types of the form
%$\shot{C}$ and $\lhot{C}$ with $C$ being a first-order channel type.
Our results show that
the calculus in~\cite{tlca07} is not only expressed but 
also reasoned in 
$\HO$ via precise encodings (with a limited form of arrow types: $\shot{C}$ and $\lhot{C}$). 
\newj{The recent work \cite{OY2016} studies two encodings:
from PCF with an effect system into a session-typed $\pi$-calculus, 
and its reverse. The reverse encoding is used to implement session channel passing in Concurrent Haskell. 
In future work we plan to use the core calculi 
studied in this paper 
to implement higher-order communication efficiently into Concurrent Haskell without losing its expressiveness.
}

\paragraph{Acknowledgments.}
We have benefited from feedback from the users of the Moca mailing list, in particular Greg Meredith and Xu Xian.
We are grateful to the anonymous reviewers for their useful remarks and suggestions.
This work has been partially sponsored by the Doctoral Prize Fellowship, EPSRC EP/K011715/1, EPSRC EP/K034413/1, and EPSRC EP/L00058X/1, EU project FP7-612985 UpScale, and EU COST Action IC1201 BETTY. P\'{e}rez is also affiliated to the NOVA Laboratory for Computer Science and Informatics (NOVA LINCS), Universidade Nova de Lisboa, Portugal.
 
%\newpage


%%%%%%%%%%%%%%%%%%%%%%%%%%%%%%%%%%%%%%%%%%%%%%%%%%%%%%%%%%%%%%%%%%%%%%%%%%%%%
% Bibliography.
%%%%%%%%%%%%%%%%%%%%%%%%%%%%%%%%%%%%%%%%%%%%%%%%%%%%%%%%%%%%%%%%%%%%%%%%%%%%%

\bibliographystyle{abbrv}
{\bibliography{session}}

%\newpage
%\onecolumn
%\setcounter{tocdepth}{4}
%\tableofcontents
%
%\appendix 
%\section{Appendix: the Typing System of \HOp}
%\label{app:types}
%%\begin{definition}[Type Equivalence]
%\label{def:iso}
%Let $\mathsf{ST}$ a set of closed session types. 
%Two types $S$ and $S'$ are said to be {\em isomorphic} if a pair $(S,S')$ is 
%in the largest fixed point of the monotone function
%$F:\mathcal{P}(\mathsf{ST}\times \mathsf{ST}) \to 
%\mathcal{P}(\mathsf{ST}\times \mathsf{ST})$ defined by:

%\begin{tabular}{rcl}
%$F(\Re)$ &$\!\!=\!\!$&	$\set{(\tinact, \tinact)}$\\
%         &$\!\!\cup\!\!$&	$\set{(\btout{U_1} S_1, \btout{U_2} S_2)
%\bnfbar (S_1, S_2),(U_1, U_2)\in \Re}$\\ 
%       &$\!\!\cup\!\!$&	$\set{(\btinp{U} S_1, \btinp{U} S_2)
%\bnfbar(S_1, S_2),,(U_1, U_2)\in \Re}$\\ 
%	&$\!\!\cup\!\!$&	$\set{(\btbra{l_i: S_i}_{i \in I} \,,\, \btbra{l_i: S_i'}_{i \in I}) \bnfbar \forall i\in I. (S_i, S_i')\in \Re}$\\
%	&$\!\!\cup\!\!$&	$\set{(\btsel{l_i: S_i}_{i \in I}\,,\, \btsel{l_i: S_i'}_{i \in I}) \bnfbar \forall i\in I. (S_i, S_i')\in \Re}$\\
%	&$\!\!\cup\!\!$&	$\set{(\trec{t}{S}, S')
%\bnfbar (S\subst{\trec{t}{S}}{\vart{t}},S')\in \Re}$\\
%	&$\!\!\cup\!\!$&	$\set{(S,\trec{t}{S'})
%\bnfbar (S,S'\subst{\trec{t}{S'}}{\vart{t}})\in \Re}$
%\end{tabular}
	
%\noindent
%Standard arguments ensure that $F$ is monotone, thus the greatest fixed point
%of $F$ exists. We write $S_1 \wb S_2$ if  $(S_1,S_2)\in \Re$. 
%\end{definition}

\begin{definition}[Duality]
\label{def:dual}
Let $\mathsf{ST}$ a set of closed session types. 
Two types $S$ and $S'$ are said to be {\em dual} if a pair $(S,S')$ is 
in the largest fixed point of the monotone function
$F:\mathcal{P}(\mathsf{ST}\times \mathsf{ST}) \to 
\mathcal{P}(\mathsf{ST}\times \mathsf{ST})$ defined by:
\begin{tabular}{rcl}
$F(\Re)$ &$\!\!=\!\!$&	$\set{(\tinact, \tinact)}$\\
         &$\!\!\cup\!\!$&	$\set{(\btout{U_1} S_1, \btinp{U_2} S_2)
\bnfbar(S_1, S_2)\in \Re, \  U_1 \wb U_2 }$\\ 
       &$\!\!\cup\!\!$&	$\set{(\btinp{U_1} S_1, \btout{U_2} S_2)
\bnfbar(S_1, S_2)\in \Re, \ U_1 \wb U_2}$\\ 
	&$\!\!\cup\!\!$&	$\set{(\btsel{l_i: S_i}_{i \in I} \,,\, \btbra{l_i: S_i'}_{i \in I}) \bnfbar \forall i\in I. (S_i, S_i')\in \Re}$\\
	&$\!\!\cup\!\!$&	$\set{(\btbra{l_i: S_i}_{i \in I}\,,\, \btsel{l_i: S_i'}_{i \in I}) \bnfbar \forall i\in I. (S_i, S_i')\in \Re}$\\
	&$\!\!\cup\!\!$&	$\set{(\trec{t}{S}, S')
\bnfbar (S\subst{\trec{t}{S}}{\vart{t}},S')\in \Re}$\\
	&$\!\!\cup\!\!$&	$\set{(S,\trec{t}{S'})
\bnfbar (S,S'\subst{\trec{t}{S'}}{\vart{t}})\in \Re}$
\end{tabular}
\noindent
where $U_1 \wb U_2$ means $U_1$ is type equivalent to $U_2$ \cite{yoshida.vasconcelos:language-primitives}.
Standard arguments ensure that $F$ is monotone, thus the greatest fixed point
of $F$ exists. We write $S_1 \dualof S_2$ if  $(S_1,S_2)\in \Re$. 
\end{definition}


%\section{Behavioural Semantics}

We present the proofs for the theorems in
Section~\ref{sec:beh_sem}.

\subsection{Proof for Theorem~\ref{the:coincidence}}

We split Theorem~\ref{the:coincidence} into 
Lemmas which we prove independently.
The combination of the lemmas is the proof for the parts
of the theorem.

The proof for Part 1 for the theorem is based on the
stratified definition of the bisimulation relations
$\wbc$ and $\wb$. The Knaster-Tarski theorem ensures
that both definitions are equivalent.
We give the stratified  definitions:

\begin{definition}[Stratified Contextual Bisimulation]\rm
	We define a set of relations $\mathcal{R}^c_n$
	on the following conditions:
%
	\begin{itemize}
		\item	$\mathcal{R}^c_0 = \bigcup_{\forall R} R, R$ is a typed relation.
		\item	$\Gamma; \emptyset; \Sigma_1\ \mathcal{R}^c_n\ \Sigma_2 \proves P_1\ \mathcal{R}^c_n\ P_2$
			whenever
			\begin{enumerate}
				\item	$\forall \news{\tilde{s}} \bactout{s}{\abs{x} P}$ such that
					\[
						\Gamma; \emptyset; \Sigma_1 \by{\news{\tilde{s}} \bactout{s}{\abs{x} P}} \Sigma_1' \proves P_1 \by{\news{\tilde{s}} \bactout{s}{\abs{x} P}} P_2
					\]
					$\exists Q_2, \abs{x}{Q}$ such that
					\[
						\Gamma; \emptyset; \Sigma_2 \By{\news{\tilde{s'}} \bactout{s}{\abs{x} Q}} \Sigma_2' \proves Q_1 \By{\news{\tilde{s}} \bactout{s}{\abs{x} Q}} Q_2
					\]
					and $\forall C, s'$
%					such that
%					\begin{eqnarray*}
%						\Gamma; \emptyset; \Sigma_1'' \proves \newsp{\tilde{s}}{P_2 \Par P \subst{s'}{x}} \hastype \Proc \\
%						\Gamma; \emptyset; \Sigma_2'' \proves \newsp{\tilde{s}}{Q_2 \Par Q \subst{s'}{x}} \hastype \Proc
%					\end{eqnarray*}
%					then
					\[
						\Gamma; \emptyset; \Sigma_1''\ \mathcal{R}^c_{n-1}\ \Sigma_2'' \proves
						\newsp{\tilde{s}}{P_2 \Par \context{C}{P \subst{s'}{x}}}\ \mathcal{R}^c_{n-1}\  \newsp{\tilde{s}}{Q_2 \Par \context{C}{Q \subst{s'}{x}}}
					\]

				\item	$\forall \news{\tilde{s}} \bactout{s}{s_1}$ such that
					\[
						\Gamma; \emptyset; \Sigma_1 \by{\news{\tilde{s}} \bactout{s}{s_1}} \Sigma_1' \proves P_1 \by{\news{\tilde{s}} \bactout{s}{s_1}} P_2
					\]
					%with $s_1: S \in \Sigma_1 \vee (\dual{s_2}: S' \in \Sigma_2 \wedge S \dualof S')$
					then $\exists Q_2, s_2$ such that
					\[
						\Gamma; \emptyset; \Sigma_2 \By{\news{\tilde{s'}} \bactout{s}{s_2}} \Sigma_2' \proves Q_1 \By{\news{\tilde{s'}} \bactout{s}{s_2}} Q_2
					\]
					%such that
		%			\begin{eqnarray*}
		%				\Gamma; \emptyset; \Sigma_1'' \proves \newsp{\tilde{s}}{P_2 \Par \context{C}{P \subst{s'}{x}}} \hastype \Proc \\
		%				\Gamma; \emptyset; \Sigma_2'' \proves \newsp{\tilde{s}}{Q_2 \Par \context{C}{Q \subst{s'}{x}}} \hastype \Proc
		%			\end{eqnarray*}
					and $\forall R, \set{x} = \fn{R}$
					\[
						\Gamma; \emptyset; \Sigma_1''\ \mathcal{R}^c_{n-1}\ \Sigma_2'' \proves \newsp{\tilde{s}}{P_2 \Par R\subst{s_1}{x}}\ \mathcal{R}^c_{n-1}\ 
						\newsp{\tilde{s'}}{Q_2 \Par R\subst{s_2}{x}}
					\]


				\item	$\forall \lambda \not= \news{\tilde{s}} \bactout{s}{\abs{x} P}$ such that
					\[
						\Gamma; \emptyset; \Sigma_1 \by{\lambda} \Sigma_1' \proves P_1 \by{\lambda} P_2
					\]
					$\exists Q_2$ such that 
					\[
						\Gamma; \emptyset; \Sigma_2 \by{\hat{\lambda}} \Sigma_2' \proves Q_1 \By{\hat{\lambda}} Q_2
					\]
					and
					$\Gamma; \emptyset; \Sigma_1'\ \mathcal{R}^c_{n-1}\ \Sigma_2' \proves P_2\ \mathcal{R}^c_{n-1}\ Q_2$.

				\item	The symmetric cases of 1, 2, 3.
			\end{enumerate}
	\end{itemize}
	\noi The above function is monotone and the Knaster-Tarski theorem ensures that a lattice is define
	with the largest $\mathcal{R}^c_n$ to be denote as $\wbc_n$ and the largest fix-point is equal to the
	contextual bisimilarity relation $\wbc = \bigcap_{i \geq 0} \wbc_i$.
\end{definition}


\begin{definition}[Stratified Bisimulation]\rm
	We define a set of relations $\mathcal{R}_n$
	on the following conditions:
%
	\begin{itemize}
		\item	$\mathcal{R}_0 = \bigcup_{\forall R} R, R$ is a typed relation.
		\item	$\Gamma; \emptyset; \Sigma_1\ \mathcal{R}_n\ \Sigma_2 \proves P_1\ \mathcal{R}_n\ P_2$
			whenever
			\begin{enumerate}
				\item	$\forall \news{\tilde{s}} \bactout{s}{\abs{x} P}$ such that
					\[
						\Gamma; \emptyset; \Sigma_1 \by{\news{\tilde{s}} \bactout{s}{\abs{x} P}} \Sigma_1' \proves P_1 \by{\news{\tilde{s}} \bactout{s}{\abs{x} P}} P_2
					\]
					$\exists Q_2, \abs{x}{Q}$ such that
					\[
						\Gamma; \emptyset; \Sigma_2 \By{\news{\tilde{s'}} \bactout{s}{\abs{x} Q}} \Sigma_2' \proves Q_1 \By{\news{\tilde{s}} \bactout{s}{\abs{x} Q}} Q_2
					\]
					and $\forall s'$
%					such that
%					\begin{eqnarray*}
%						\Gamma; \emptyset; \Sigma_1'' \proves \newsp{\tilde{s}}{P_2 \Par P \subst{s'}{x}} \hastype \Proc \\
%						\Gamma; \emptyset; \Sigma_2'' \proves \newsp{\tilde{s}}{Q_2 \Par Q \subst{s'}{x}} \hastype \Proc
%					\end{eqnarray*}
%					then
					\[
						\Gamma; \emptyset; \Sigma_1''\ \mathcal{R}_{n-1}\ \Sigma_2'' \proves
						\newsp{\tilde{s}}{P_2 \Par P \subst{s'}{x}}\ \mathcal{R}_{n-1}\  \newsp{\tilde{s}}{Q_2 \Par Q \subst{s'}{x}}
					\]

				\item	$\forall \news{\tilde{s}} \bactout{s}{s_1}$ such that
					\[
						\Gamma; \emptyset; \Sigma_1 \by{\news{\tilde{s}} \bactout{s}{s_1}} \Sigma_1' \proves P_1 \by{\news{\tilde{s}} \bactout{s}{s_1}} P_2
					\]
					with $s_1: S \in \Sigma_1 \vee (\dual{s_2}: S' \in \Sigma_2 \wedge S \dualof S')$
					then $\exists Q_2, s_2$ such that
					\[
						\Gamma; \emptyset; \Sigma_2 \By{\news{\tilde{s'}} \bactout{s}{s_2}} \Sigma_2' \proves Q_1 \By{\news{\tilde{s'}} \bactout{s}{s_2}} Q_2
					\]
					%such that
		%			\begin{eqnarray*}
		%				\Gamma; \emptyset; \Sigma_1'' \proves \newsp{\tilde{s}}{P_2 \Par \context{C}{P \subst{s'}{x}}} \hastype \Proc \\
		%				\Gamma; \emptyset; \Sigma_2'' \proves \newsp{\tilde{s}}{Q_2 \Par \context{C}{Q \subst{s'}{x}}} \hastype \Proc
		%			\end{eqnarray*}
					and
					\[
						\Gamma; \emptyset; \Sigma_1''\ \mathcal{R}_{n-1}\ \Sigma_2'' \proves \newsp{\tilde{s}}{P_2 \Par \map{S}^{s_1}}\ \mathcal{R}_{n-1}\ 
						\newsp{\tilde{s'}}{Q_2 \Par \map{S}^{s_2}}
					\]


				\item	$\forall \lambda \not= \news{\tilde{s}} \bactout{s}{\abs{x} P}$ such that
					\[
						\Gamma; \emptyset; \Sigma_1 \by{\lambda} \Sigma_1' \proves P_1 \by{\lambda} P_2
					\]
					$\exists Q_2$ such that 
					\[
						\Gamma; \emptyset; \Sigma_2 \by{\hat{\lambda}} \Sigma_2' \proves Q_1 \By{\hat{\lambda}} Q_2
					\]
					and
					$\Gamma; \emptyset; \Sigma_1'\ \mathcal{R}_{n-1}\ \Sigma_2' \proves P_2\ \mathcal{R}_{n-1}\ Q_2$.

				\item	The symmetric cases of 1, 2, 3.
			\end{enumerate}
	\end{itemize}
	\noi The above function is monotone and the Knaster-Tarski theorem ensures that a lattice is define
	with the largest $\mathcal{R}_n$ to be denote as $\wb_n$ and the largest fix-point is equal to the
	bisimilarity relation $\wb = \bigcap_{i \geq 0} \wb_i$.
\end{definition}


\begin{lemma}\rm
	$\wbc\ \subseteq\ \wb$
\end{lemma}

\begin{proof}
	Statement $\wbc \subseteq \wb$
	is equivalent to the statement $\forall n, \wbc_n \subseteq \wb_n$.
	From here the proof is done using induction on the definitions of $\wbc$.

	\noi {\bf Basic step:}
	From the definitions of $\wbc_0$ and $\wb_0$ we get that $\wbc_0 = \wb_0$.

	\noi {\bf Induction hypothesis:}
	$\wbc_n\ \subseteq\ \wb_n$.

	\noi {\bf Inductive step:}
	Let
%
	\begin{eqnarray*}
		\Gamma; \emptyset; \Sigma_1\ \wbc_{n+1}\ \Sigma_2 \proves P_1\ \wbc_{n+1}\ Q_1
	\end{eqnarray*}
%
	\noi We perform a case analysis on transition $\by\lambda$.

	%%%%%%%%%%%%%%%%%%%%%%%%%%%%%%%%%%%%%%%%%%%%%%%

	\noi - Case: $\lambda \notin \set{\news{\tilde{s}} \bactout{s}{\abs{x} P}, \news{\tilde{s}} \bactout{s}{s_1}}$
%
	\begin{eqnarray}
		\Gamma; \emptyset; \Sigma_1 \by{\lambda} \Sigma_1' \proves P_1 \by{\lambda} P_2 \label{lem:wbc_is_wb1}
	\end{eqnarray}
%
	\noi implies that 
	$\exists Q_1$ such that
%
	\begin{eqnarray}
		\Gamma; \emptyset; \Sigma_2 \by{\lambda} \Sigma_2' &\proves& Q_1 \by{\lambda} Q_2 \label{lem:wbc_is_wb2}\\
		\Gamma; \emptyset; \Sigma_1'\ \wbc_n\ \Sigma_2' &\proves& P_1\ \wbc_n\ Q_2
	\end{eqnarray}
%
	We apply the induction hypothesis to the latter judgement:
%
	\begin{eqnarray}
		\Gamma; \emptyset; \Sigma_1' \proves P_2\ \wb_n\ \Gamma; \emptyset; \Sigma_2' \proves Q_2 \hastype \Proc  \label{lem:wbc_is_wb3}
	\end{eqnarray}
%
	Assume $\mathcal{R}_{n+1} = \set{\Gamma; \emptyset; \Sigma_1 \proves P_1 \hastype \Proc, \Gamma; \emptyset; \Sigma_2 \proves Q_1 \hastype \Proc}$.

	\noi $\mathcal{R}_{n+1}$ satisfies the condition for the stratified definition of bisimulation
	because statement~\ref{lem:wbc_is_wb1} implies $\exists Q'$ such that
	statements~\ref{lem:wbc_is_wb2} holds and furthermore statement~\ref{lem:wbc_is_wb3} holds
%	if $\Gamma; \emptyset; \Sigma_1 \proves P \by{\lambda} \Gamma; \emptyset; \Sigma_1' \proves P' \hastype \Proc$ then
%	$\exists Q'$ such that
%	$\Gamma; \emptyset; \Sigma_2 \proves Q \by{\lambda} \Gamma; \emptyset; \Sigma_2' \proves Q' \hastype \Proc$
%	and \ref{pr:biscong_is_bis2}.

	\noi Because $\wb_{n+1}$ is the largest relation we get that $\mathcal{R}_{n+1} \subseteq \wb_{n+1}$ as required.

	%%%%%%%%%%%%%%%%%%%%%%%%%%%%%%%%%%%%%%%%%%%%%%%

	\noi - Case: $\lambda = \news{\tilde{s}} \bactout{s}{\abs{x} P}$
%
	\begin{eqnarray}
		\Gamma; \emptyset; \Sigma_1 \by{\news{\tilde{s}} \bactout{s}{\abs{x} P}} \Sigma_1' \proves P_1 \by{\news{\tilde{s}} \bactout{s}{\abs{x} P}} P_2 \label{lem:wbc_is_wb4}
	\end{eqnarray}
%
	\noi implies that
	$\exists Q_2, \abs{x}{Q}$ such that
	\begin{eqnarray}
		\Gamma; \emptyset; \Sigma_2 \By{\news{\tilde{s'}} \bactout{s}{\abs{x} Q}} \Sigma_2' \proves Q_1 \By{\news{\tilde{s'}} \bactout{s}{\abs{x} Q}} Q_2  \label{lem:wbc_is_wb5}
	\end{eqnarray}
	and $\forall C, s'$
%	such that
%	\begin{eqnarray*}
%		\Gamma; \emptyset; \Sigma_1'' \proves \newsp{\tilde{s}}{\context{C}{P' \Par P \subst{s'}{x}}} \hastype \Proc \\
%		\Gamma; \emptyset; \Sigma_2'' \proves \newsp{\tilde{s}}{\context{C}{Q' \Par Q \subst{s'}{x}}} \hastype \Proc
%	\end{eqnarray*}
%	then
%
	\begin{eqnarray*}
		\Gamma; \emptyset; \Sigma_1''\ \wbc_{n}\ \Sigma_2'' \proves \newsp{\tilde{s}}{\context{C}{P_2 \Par P \subst{s'}{x}}}\ \wbc_{n}\ 
		\newsp{\tilde{s'}}{\context{C}{Q_2 \Par Q \subst{s'}{x}}}
	\end{eqnarray*}
%
	\noi For $C = \hole$ we have that 
%
	\begin{eqnarray*}
		\Gamma; \emptyset; \Sigma_1''\ \wbc_{n}\ \Sigma_2'' \proves \newsp{\tilde{s}}{P_2 \Par P \subst{s'}{x}}\ \wbc_{n}\ 
		\newsp{\tilde{s}}{Q' \Par Q_2 \subst{s'}{x}}
	\end{eqnarray*}
%
	\dk{(prove that it is typable)}

	\noi If we apply the induction hypothesis to the latter statement we get
%
	\begin{eqnarray}
		\Gamma; \emptyset; \Sigma_1''\ \wb_{n}\ \Sigma_2'' \proves \newsp{\tilde{s}}{P_2 \Par P \subst{s'}{x}}\ \wb_{n}\ 
		\newsp{\tilde{s}}{Q_2 \Par Q \subst{s'}{x}}
		\label{lem:wbc_is_wb6}
	\end{eqnarray}
%
	\noi Assume $\mathcal{R}_{n+1} = \set{\Gamma; \emptyset; \Sigma_1 \proves P_1 \hastype \Proc, \Gamma; \emptyset; \Sigma_2 \proves Q_2 \hastype \Proc}$.

	\noi $\mathcal{R}_{n+1}$ satisfies the condition for the stratified definition for bisimulation
	because statement~\ref{lem:wbc_is_wb4} implies that
	$\exists Q', \abs{x}{Q}$ such that
	statement~\ref{lem:wbc_is_wb5} holds and furthemore statement~\ref{lem:wbc_is_wb6} holds.

	\noi Because $\wb_{n+1}$ is the largest relation we get that $\mathcal{R}_{n+1} \subseteq \wb_{n+1}$ as required.

	%%%%%%%%%%%%%%%%%%%%%%%%%%%%%%%%%%%%%%%%%%%%%%%

	\noi - Case: $\lambda = \news{\tilde{s}} \bactout{s}{s_1}$
%
	\begin{eqnarray}
		\Gamma; \emptyset; \Sigma_1 \by{\news{\tilde{s}} \bactout{s}{s_1}} \Sigma_1' \proves P_1 \by{\news{\tilde{s}} \bactout{s}{s_1}} P_2 \label{lem:wbc_is_wb7}
	\end{eqnarray}
%
	\noi \dk{with $s_1: S \in \Sigma_1 \vee (\dual{s_1}: S' \in \Sigma_1' \wedge S \dualof S')$} implies that
	$\exists Q', s_2$ such that
	\begin{eqnarray}
		\Gamma; \emptyset; \Sigma_2 \By{\news{\tilde{s'}} \bactout{s}{s_2}} \Sigma_2' \proves Q_1 \By{\news{\tilde{s'}} \bactout{s}{s_2}} Q_2 \label{lem:wbc_is_wb8}
	\end{eqnarray}
	and $\forall R$ with $\set{x} = \fn{P}$
%	such that
%	\begin{eqnarray*}
%		\Gamma; \emptyset; \Sigma_1'' \proves \newsp{\tilde{s}}{\context{C}{P' \Par P \subst{s'}{x}}} \hastype \Proc \\
%		\Gamma; \emptyset; \Sigma_2'' \proves \newsp{\tilde{s}}{\context{C}{Q' \Par Q \subst{s'}{x}}} \hastype \Proc
%	\end{eqnarray*}
%	then
%
	\begin{eqnarray*}
		\Gamma; \emptyset; \Sigma_1''\ \wbc_{n}\ \Sigma_2'' \proves \newsp{\tilde{s}}{P_2 \Par R \subst{s_1}{x}}\ \wbc_{n}\ 
		\newsp{\tilde{s'}}{Q_2 \Par R \subst{s_2}{x}}
	\end{eqnarray*}
%
	\noi From the latter statement we get
	\begin{eqnarray*}
		\Gamma; \emptyset; \Sigma_1''\ \wbc_{n}\ \Sigma_2'' \proves \newsp{\tilde{s}}{P_2 \Par \map{S}^{x} \subst{s_1}{x}}\ \wbc_{n}\ 
		\newsp{\tilde{s'}}{Q_2 \Par \map{S}^{x} \subst{s_2}{x}} \label{lem:wbc_is_wb9}
	\end{eqnarray*}
%
	\noi Assume $\mathcal{R}_{n+1} = \set{\Gamma; \emptyset; \Sigma_1 \proves P_1 \hastype \Proc, \Gamma; \emptyset; \Sigma_2 \proves Q_1 \hastype \Proc}$.

	\noi $\mathcal{R}_{n+1}$ satisfies the condition for the stratified definition for bisimulation
	because statement~\ref{lem:wbc_is_wb7} implies that
	$\exists Q', s_2$ such that
	statement~\ref{lem:wbc_is_wb8} holds and furthemore statement~\ref{lem:wbc_is_wb9} holds.
\end{proof}

\begin{lemma}\rm
	$\wb\ \subseteq\ \wbc$
\end{lemma}

\begin{proof}
	The statement $\wb \subseteq \wbc$.
	is equivalent to the statement $\forall n, \wb_n \subseteq \wbc_n$.
	The proof is done using induction on the definition $\wb$.

	\noi {\bf Basic step:} From the definitions of $\wbc_0$ and $\wb_0$ we get that $\wbc_0 = \wb_0$.

	\noi {\bf Induction hypothesis:} $\wb_n \subseteq \wbc_n$.

	\noi {\bf Inductive step:}
	Let 
	\[
		\Gamma; \emptyset; \Sigma_1\ \wb_{n+1}\ \Sigma_2 \proves P_1\ \wb_{n+1}\ Q_1
	\]
	We perform a case analysis on transition $\by{\lambda}$.

	\noi - Case: $\lambda \notin \set{\news{\tilde{s}} \bactout{s}{\abs{x} P}, \news{\tilde{s}} \bactout{s}{s_1}}$

	\noi Same arguments with the same case of the direction $\wbc \subseteq \wb$

	\noi - Case: $\lambda = \news{\tilde{s}} \bactout{s}{\abs{x} P}$
%
	\begin{eqnarray*}
		\Gamma; \emptyset; \Sigma_1 \by{\news{\tilde{s}} \bactout{s}{\abs{x} P}} \Sigma_1' \proves P_1 \by{\news{\tilde{s}} \bactout{s}{\abs{x} P}} P_2
	\end{eqnarray*}
%
	implies that
	$\exists Q_2, \abs{x}{Q}$ such that
	$\Gamma; \emptyset; \Sigma_2 \proves Q \by{\news{\tilde{s}} \bactout{s}{\abs{x} Q}} \Gamma; \emptyset; \Sigma_2' \proves Q' \hastype \Proc$
	and $s'$
	such that
	\begin{eqnarray*}
		\Gamma; \emptyset; \Sigma_1'' \proves \newsp{\tilde{s}}{P' \Par P \subst{s'}{x}} \hastype \Proc \\
		\Gamma; \emptyset; \Sigma_2'' \proves \newsp{\tilde{s}}{Q' \Par Q \subst{s'}{x}} \hastype \Proc
	\end{eqnarray*}
	then
	\[
		\Gamma; \emptyset; \Sigma_1'' \proves \newsp{\tilde{s}}{P' \Par P \subst{s'}{x}}\ \wb_{n}\ 
		\Gamma; \emptyset; \Sigma_2'' \proves \newsp{\tilde{s}}{Q' \Par Q \subst{s'}{x}} \hastype \Proc
	\]

	Let
	\[
		\begin{array}{rcl}
			R_{n + 1} &=& \set{	(\Gamma; \emptyset; \Sigma_1 \proves \newsp{\tilde{s_1}}{\bout{s}{\abs{x}{\context{C}{P}}} P''} \hastype \Proc,
						\Gamma; \emptyset; \Sigma_2 \proves \newsp{\tilde{s_2}}{\bout{s}{\abs{x}{\context{C}{P}}} Q''} \hastype \Proc) \setbar\\
				& &		\forall C, P' \scong \news{\tilde{s_1}} P'', Q' \scong \news{\tilde{s_2}} Q''}
		\end{array}
	\]

	\dk{prove that $R_{n+1}$ is typable}

	$R_{n+1}$ satisfies the stratified definition for bisimulation and furthermore we can deduce that
	$\forall C, s'$
	such that
	\begin{eqnarray*}
		\Gamma; \emptyset; \Sigma_1'' \proves \newsp{\tilde{s}}{\context{C}{P' \Par P \subst{s'}{x}}} \hastype \Proc \\
		\Gamma; \emptyset; \Sigma_2'' \proves \newsp{\tilde{s}}{\context{C}{Q' \Par Q \subst{s'}{x}}} \hastype \Proc
	\end{eqnarray*}
	then
	\[
		\Gamma; \emptyset; \Sigma_1'' \proves \newsp{\tilde{s}}{\context{C}{P' \Par P \subst{s'}{x}}}\ \wb_{n}\ 
		\Gamma; \emptyset; \Sigma_2'' \proves \newsp{\tilde{s}}{\context{C}{Q' \Par Q \subst{s'}{x}}} \hastype \Proc
	\]

	From the induction hypothesis we get that
	$\forall C, s'$
	such that
	\begin{eqnarray*}
		\Gamma; \emptyset; \Sigma_1'' \proves \newsp{\tilde{s}}{\context{C}{P' \Par P \subst{s'}{x}}} \hastype \Proc \\
		\Gamma; \emptyset; \Sigma_2'' \proves \newsp{\tilde{s}}{\context{C}{Q' \Par Q \subst{s'}{x}}} \hastype \Proc
	\end{eqnarray*}
	then
	\begin{eqnarray}
		\Gamma; \emptyset; \Sigma_1'' \proves \newsp{\tilde{s}}{\context{C}{P' \Par P \subst{s'}{x}}}\ \wbc_{n}\ 
		\Gamma; \emptyset; \Sigma_2'' \proves \newsp{\tilde{s}}{\context{C}{Q' \Par Q \subst{s'}{x}}} \hastype \Proc
		\label{pr:bis_is_contextbis}
	\end{eqnarray}

%	For $C = \hole$ we have that 
%	\[
%		\Gamma; \emptyset; \Sigma_1'' \proves \newsp{\tilde{s}}{P_2 \Par P \subst{s'}{x}}\ \wbc_{n}\ 
%		\Gamma; \emptyset; \Sigma_2'' \proves \newsp{\tilde{s}}{Q_2 \Par Q \subst{s'}{x}} \hastype \Proc
%	\]

	Assume $R^c_{n+1} = \set{\Gamma; \emptyset; \Sigma_1 \proves P \hastype \Proc, \Gamma; \emptyset; \Sigma_2 \proves Q \hastype \Proc}$.

	$R^c_{n+1}$ satisfies the condition for the stratified definition of the contextual bisimulation because of the fact that
	if $\Gamma; \emptyset; \Sigma_1 \proves P \by{\news{\tilde{s}} \bactout{s}{\abs{x} P}} \Gamma; \emptyset; \Sigma_1' \proves P' \hastype \Proc$ then
	$\exists Q', \abs{x}{Q}$ such that
	$\Gamma; \emptyset; \Sigma_2 \proves Q \by{\news{\tilde{s}} \bactout{s}{\abs{x} Q}} \Gamma; \emptyset; \Sigma_2' \proves Q' \hastype \Proc$
	and \ref{pr:bis_is_contextbis}.

	Because $\wbc_{n+1}$ is the largest relation we get that $R^c_{n+1} \subseteq \wb_{n+1}$ as required.
\end{proof}


%% !TEX root = ../main.tex
\section{Encoding Semantics}



\begin{comment}
%%%%%%%%%%%%%%%%%%%%%%%%%%%%%%%%%%%%%%%%%%%%%%%%%
% POLYADIC TO MONADIC
%%%%%%%%%%%%%%%%%%%%%%%%%%%%%%%%%%%%%%%%%%%%%%%%%

\subsection{Properties for $\encod{\cdot}{\cdot}{\mathsf{p}}$}
\label{app:polmon}
We study the properties of the typed encoding in
Def.~\ref{d:enc:poltomon} (Page~\pageref{d:enc:poltomon}).

We repeat the statement of Prop.~\ref{prop:typepresp}, as in Page \pageref{prop:typepresp}:
\begin{proposition}[Type Preservation, Polyadic to Monadic]
Let $P$ be an  $\HOp$ process.
If			$\Gamma; \emptyset; \Delta \proves P \hastype \Proc$ then 
			$\mapt{\Gamma}^{\mathsf{p}}; \emptyset; \mapt{\Delta}^{\mathsf{p}} \proves \map{P}^{\mathsf{p}} \hastype \Proc$. 
\end{proposition}

\begin{proof}
By induction on the inference $\Gamma; \emptyset; \Delta \proves P \hastype \Proc$.
We examine two representative cases, using biadic communications.

\begin{enumerate}[1.]
\item Case 
$P = \bout{k}{V} P'$ and 
$\Gamma; \emptyset; \Delta_1 \cat \Delta_2 \cat k:\btout{\lhot{(C_1,C_2)}} S \proves \bout{k}{V} P' \hastype \Proc$. Then either $V = Y$ or $V = \abs{x_1,x_2}Q$, for some $Q$. The case $V = Y$ is immediate; we give details for the case $V = \abs{x_1,x_2}Q$, for which we have the following typing:
\[
\tree{
\tree{}{
\Gamma; \emptyset; \Delta_1 \cat k:S \proves P' \hastype \Proc}
\quad
\tree{
\Gamma; \emptyset; \Delta_2 \cat x_1: C_1 \cat x_2:C_2 \proves Q \hastype \Proc
}{
\Gamma; \emptyset; \Delta_2 \proves \abs{x_1,x_2}Q \hastype \lhot{(C_1,C_2)}
}
}{
\Gamma; \emptyset; \Delta_1 \cat \Delta_2 \cat k:\btout{\lhot{(C_1,C_2)}} S \proves \bout{k}{\abs{x_1,x_2}Q} P \hastype \Proc
}
\]
We now show the typing for $\map{P}^{\mathsf{p}}$. By IH we have both:
\[
\mapt{\Gamma}^{\mathsf{p}}; \emptyset; \mapt{\Delta_1}^{\mathsf{p}} \cat k:\mapt{S}^{\mathsf{p}} \proves \map{P'}^{\mathsf{p}} \hastype \Proc
\qquad
\mapt{\Gamma}^{\mathsf{p}}; \emptyset; \mapt{\Delta_2}^{\mathsf{p}} \cat x_1: \mapt{C_1}^{\mathsf{p}} \cat x_2:\mapt{C_2}^{\mathsf{p}} \proves \map{Q}^{\mathsf{p}} \hastype \Proc
\]
Let $L = \lhot{(C_1,C_2)}$. 
By Def.~\ref{d:enc:poltomon} 
we have  
$\mapt{L}^{\mathsf{p}} = \lhot{\big(\btinp{\tmap{C_1}{\mathsf{p}}} \btinp{\tmap{C_2}{\mathsf{p}}}\tinact\big)}$
and
$\map{P}^{\mathsf{p}} = \bbout{k}{\abs{z}\binp{z}{x_1}\binp{z}{x_2} \map{Q}^{\mathsf{p}}} \map{P'}^{\mathsf{p}}$.
We can now infer the following typing derivation:
\[
\tree{
\tree{}
{
\mapt{\Gamma}^{\mathsf{p}}; \emptyset; \mapt{\Delta_1}^{\mathsf{p}} \cat k:\mapt{S}^{\mathsf{p}} \proves \map{P'}^{\mathsf{p}} \hastype \Proc}
\quad
\tree{
\tree{
\tree{
\tree{
\tree{}{\mapt{\Gamma}^{\mathsf{p}}; \emptyset; \mapt{\Delta_2}^{\mathsf{p}} \cat x_1: \tmap{C_1}{\mathsf{p}} \cat x_2: \tmap{C_2}{\mathsf{p}} \proves 
 \map{Q}^{\mathsf{p}} \hastype \Proc}
}{
\mapt{\Gamma}^{\mathsf{p}}; \emptyset; \mapt{\Delta_2}^{\mathsf{p}} \cat x_1: \tmap{C_1}{\mathsf{p}} \cat x_2: \tmap{C_2}{\mathsf{p}}
\cat z:\tinact \proves 
 \map{Q}^{\mathsf{p}} \hastype \Proc
}
}{
\mapt{\Gamma}^{\mathsf{p}}; \emptyset; \mapt{\Delta_2}^{\mathsf{p}} \cat x_1: \tmap{C_1}{\mathsf{p}}\cat z:\btinp{\tmap{C_2}{\mathsf{p}}}\tinact \proves 
\binp{z}{x_2} \map{Q}^{\mathsf{p}} \hastype \Proc
}
}{
\mapt{\Gamma}^{\mathsf{p}}; \emptyset; \mapt{\Delta_2}^{\mathsf{p}} \cat z:\btinp{\tmap{C_1}{\mathsf{p}}}\btinp{\tmap{C_2}{\mathsf{p}}}\tinact \proves 
\binp{z}{x_1}\binp{z}{x_2} \map{Q}^{\mathsf{p}} \hastype \Proc
}
}{
\mapt{\Gamma}^{\mathsf{p}}; \emptyset; \mapt{\Delta_2}^{\mathsf{p}}  \proves 
\abs{z}\binp{z}{x_1}\binp{z}{x_2} \map{Q}^{\mathsf{p}} \hastype \lhot{(\tmap{C_1}{\mathsf{p}},\tmap{C_2}{\mathsf{p}})}
}
}{
\mapt{\Gamma}^{\mathsf{p}}; \emptyset; \mapt{\Delta_1}^{\mathsf{p}} \cat \mapt{\Delta_2}^{\mathsf{p}} \cat k:\btout{\mapt{L}^{\mathsf{p}}} \mapt{S}^{\mathsf{p}} \proves \map{P}^{\mathsf{p}} \hastype \Proc
}
\]

\item Case $P = \binp{k}{x_1,x_2} P'$ 
and
$\Gamma; \emptyset; \Delta_1 \cat k: \btinp{(C_1, C_2)} S \proves \binp{k}{x_1,x_2} P' \hastype \Proc$.
We have the following typing derivation:
\[
\tree{
\Gamma; \emptyset; \Delta_1 \cat k:S \cat x_1: C_1 \cat x_2: C_2 \proves  P' \hastype \Proc
\quad
\Gamma; \emptyset;  \proves x_1, x_2 \hastype C_1,C_2
}{
\Gamma; \emptyset; \Delta_1 \cat k: \btinp{(C_1, C_2)} S \proves \binp{k}{x_1,x_2} P' \hastype \Proc
}
\]
By Def.~\ref{d:enc:poltomon} we have 
$\map{P}^{\mathsf{p}} = \binp{k}{x_1}\binp{k}{x_2}\map{P'}^{\mathsf{p}}$.
By IH we have 
$$
\mapt{\Gamma}^{\mathsf{p}}; \emptyset; \mapt{\Delta_1}^{\mathsf{p}} \cat k:\mapt{S}^{\mathsf{p}} \cat x_1: \tmap{C_1}{\mathsf{p}} \cat x_2: \tmap{C_2}{\mathsf{p}} \proves  \map{P'}^{\mathsf{p}} \hastype \Proc
$$
and the following type derivation:
\[
\tree{
\tree{
\tree{
}{
\mapt{\Gamma}^{\mathsf{p}}; \emptyset; \mapt{\Delta_1}^{\mathsf{p}} \cat x_1:\tmap{C_1}{\mathsf{p}} \cat x_2:\tmap{C_2}{\mathsf{p}} \cat k:\mapt{S}^{\mathsf{p}}  \proves  \map{P'}^{\mathsf{p}} \hastype \Proc
}
%\quad
%\tree{}{
%\mapt{\Gamma}^{\mathsf{p}}; \emptyset; x_2:\tmap{C_2}{\mathsf{p}}  \proves  x_2 \hastype \tmap{C_2}{\mathsf{p}}}
}{
\mapt{\Gamma}^{\mathsf{p}}; \emptyset; \mapt{\Delta_1}^{\mathsf{p}} \cat x_1:\tmap{C_1}{\mathsf{p}} \cat k:\btinp{\tmap{C_2}{\mathsf{p}}}\mapt{S}^{\mathsf{p}}  \proves  \binp{k}{x_2}\map{P'}^{\mathsf{p}} \hastype \Proc
}
%\quad
%\tree{}{
%\mapt{\Gamma}^{\mathsf{p}}; \emptyset; x_1:\tmap{C_1}{\mathsf{p}}  \proves  x_1 \hastype \tmap{C_1}{\mathsf{p}}}
}{
\mapt{\Gamma}^{\mathsf{p}}; \emptyset; \mapt{\Delta_1}^{\mathsf{p}} \cat k:\btinp{\tmap{C_1}{\mathsf{p}}}\btinp{\tmap{C_2}{\mathsf{p}}}\mapt{S}^{\mathsf{p}}  \proves  \map{P}^{\mathsf{p}} \hastype \Proc
}
\]
\end{enumerate}
\qed
\end{proof}

\end{comment}

%\subsection{Properties for $\encod{\cdot}{\cdot}{1}: \sessp^{-\mu} \to \HO$}
%\label{app:enc_sesspnr_to_ho}

%\begin{proposition}\rm
%	\label{app:enc_sesspnr_to_ho_typing}
%	Encoding $\encod{\cdot}{\cdot}{1}: \sessp^{-\mu} \to \HO$  is type-preserving (cf. Def.~\ref{def:ep}\,(1)).\rm
%\end{proposition}

%We repeat the statement of Prop.~\ref{prop:typepres1}, as in Page \pageref{prop:typepres1}:

%\begin{proposition}[Type Preservation, First-Order into Higher-Order]
%Let $P$ be a  $\sessp^{-\mu}$ process.
%If			$\Gamma; \emptyset; \Delta \proves P \hastype \Proc$ then 
%			$\mapt{\Gamma}^{1}; \emptyset; \mapt{\Delta}^{1} \proves \map{P}^{1} \hastype \Proc$. 
%\end{proposition}

%%%%%%%%%%%%%%%%%%%%%%%%%%%%%%%%%%%%%%%%%%%%%%%%%
% HOp TO HO
%%%%%%%%%%%%%%%%%%%%%%%%%%%%%%%%%%%%%%%%%%%%%%%%%

\subsection{Properties for $\enco{\pmapp{\cdot}{1}{f}, \tmap{\cdot}{1}, \mapa{\cdot}^{1}}: \HOp \to \HO$}
\label{app:enc_HOp_to_HO}

We repeat the statement of Prop.~\ref{prop:typepres_HOp_to_HO}, 
as in Page \pageref{prop:typepres_HOp_to_HO}:

%\begin{proposition}[Type Preservation, Full First-Order into Higher-Order]
%	Let $P$ be a  $\sessp$ process.
%	If	$\Gamma; \emptyset; \Delta \proves P \hastype \Proc$ then 
%		$\mapt{\Gamma}^{2}; \emptyset; \mapt{\Delta}^{2} \proves \map{P}_f^{2} \hastype \Proc$. 
%\end{proposition}

\begin{proposition}[Type Preservation, \HOp into \HO]
	Let $P$ be a \HOp process.
	If $\Gamma; \emptyset; \Delta \proves P \hastype \Proc$ then 
	$\mapt{\Gamma}^{1}; \emptyset; \mapt{\Delta}^{1} \proves \pmapp{P}{1}{f} \hastype \Proc$. 
\end{proposition}

\begin{proof}
	By induction on the   inference of $\Gamma; \emptyset; \Delta \proves P \hastype \Proc$. %\jp{TO BE ADJUSTED!}
%
	\begin{enumerate}[1.]
		%%%% Output of (linear) channel
		\item	Case $P = \bout{k}{n}P'$. There are two sub-cases.
			In the first sub-case $n = k'$ (output of a linear channel). Then  
			we have the following typing in the source language:
			{
			\[
				\tree{
					\Gamma; \emptyset; \Delta \cat k:S  \proves  P' \hastype \Proc \quad \Gamma ; \emptyset ; \{k' : S_1\} \proves  k' \hastype S_1}{
					\Gamma; \emptyset; \Delta \cat k':S_1 \cat k:\btout{S_1}S \proves  \bout{k}{k'} P' \hastype \Proc}
			\]
			}
			Thus, by IH we have
			$$
			\tmap{\Gamma}{1}; \emptyset ; \tmap{\Delta}{1} \cat k:\tmap{S}{1} \proves \pmap{P'}{1} \hastype \Proc
			$$
			Let us write $U_1$
			to stand for $\lhot{\btinp{\lhot{\tmap{S_1}{1}}}\tinact}$.
			The corresponding typing in the target language is as follows:
			\begin{eqnarray}
				\label{prop:sesspnr_to_HO_t1}
				\tree{
					\tree{
						\tree{
							\tree{
								\tmap{\Gamma}{1} ; \set{X : \lhot{\tmap{S_1}{1}}} ; \emptyset \proves \X  \hastype \lhot{\tmap{S_1}{1}}
								\qquad 
								\tmap{\Gamma}{1} ; \emptyset ; \set{k' : \tmap{S_1}{1}} \proves  k' \hastype \tmap{S_1}{1}
							}{
								\tmap{\Gamma}{1} ; \set{X : \lhot{\tmap{S_1}{1}}} ; k' : \tmap{S_1}{1} \proves \appl{\X}{k'} \hastype \Proc
							}
						}{
							\tmap{\Gamma}{1} ; \{X : \lhot{\tmap{S_1}{1}}\} ; k' : \tmap{S_1}{1} \cat z:\tinact \proves \appl{\X}{k'} \hastype \Proc
						}
					}{
						\tmap{\Gamma}{1} ; \emptyset; k' : \tmap{S_1}{1} \cat z:\btinp{\lhot{\tmap{S_1}{1}}}\tinact \proves \binp{z}{X} \appl{\X}{k'} \hastype \Proc
					}
				}{
					\tmap{\Gamma}{1} ; \emptyset; k' : \tmap{S_1}{1} \proves \abs{z}{\binp{z}{X} \appl{\X}{k'}} \hastype U_1
				}
			\end{eqnarray}
			\begin{eqnarray*}
				\tree{
					\tmap{\Gamma}{1}; \emptyset ; \tmap{\Delta}{1} \cat k:\tmap{S}{1} \proves \pmap{P'}{1} \hastype \Proc
					\qquad
					\tmap{\Gamma}{1} ; \emptyset; k' : \tmap{S_1}{1} \proves \abs{z}{\binp{z}{X} \appl{\X}{k'}} \hastype U_1 \ \eqref{prop:sesspnr_to_HO_t1}
				}{
					\tmap{\Gamma}{1}; \emptyset; \tmap{\Delta}{1} \cat k':\tmap{S_1}{1} \cat k:\btout{U_1}\tmap{S}{1} \proves  \bbout{k}{\abs{z}{\binp{z}{X} \appl{\X}{k'}}} \pmap{P'}{1} \hastype \Proc
				}
			\end{eqnarray*}
%
	
		%%%% Output of (shared) channel
			In the second sub-case, we have $n = a$ (output of a shared name). Then  
			we have the following typing in the source language:
			{
			\[
				\tree{
					\Gamma \cat a:\chtype{S_1}; \emptyset; \Delta \cat k:S  \proves
					P' \hastype \Proc \quad \Gamma \cat a:\chtype{S_1} ; \emptyset ; \emptyset \proves  a \hastype S_1
				}{
					\Gamma \cat a:\chtype{S_1} ; \emptyset; \Delta  \cat k:\bbtout{\chtype{S_1}}S \proves  \bout{k}{a} P' \hastype \Proc
				}
			\]
			}
			The typing in the target language is derived similarly as in the first sub-case. \\
	
		%%%% Input of (linear) channel 
		\item	Case $P = \binp{k}{x}Q$. We have two sub-cases, depending on the type of $x$.
			In the first case, $x$ stands for a linear channel.
			Then we have the following typing in the source language:
			{
			\[
				\tree{
					\Gamma; \emptyset; \Delta  \cat k:S \cat x:S_1 \proves   Q \hastype \Proc
				}{
					\Gamma; \emptyset; \Delta  \cat k:\btinp{S_1}S \proves  \binp{k}{x} Q \hastype \Proc
				}
			\]
			 }
			 Thus, by IH we have
			 $$
			 \tmap{\Gamma}{1}; \emptyset;  \tmap{\Delta}{1} \cat k:\tmap{S}{1}  \cat x:\tmap{S_1}{1} \proves  \pmap{Q}{1}   \hastype \Proc
			 $$
			 Let us write $U_1$ to stand for $\lhot{\btinp{\lhot{\tmap{S_1}{1}}}\tinact}$.
			 The corresponding typing in the target language is as follows:
			{\small
%
			\begin{eqnarray}
				\label{prop:sesspnr_to_HO_t2}
				\tree{
					\tmap{\Gamma}{1}; \{X: U_1\};   \emptyset \proves X \hastype U_1
					\qquad
					\tmap{\Gamma}{1}; \emptyset;   \cat s: \btinp{\lhot{\tmap{S_1}{1}}}\tinact \ \proves s \, \hastype  \btinp{\lhot{\tmap{S_1}{1}}} \tinact 
				}{
					\tmap{\Gamma}{1}; \{X: U_1\};   \cat s: \btinp{\lhot{\tmap{S_1}{1}}}\tinact \ \proves \appl{X}{s}  \hastype \Proc
				}
			\end{eqnarray}
%
			\begin{eqnarray}
				\label{prop:sesspnr_to_HO_t3}
				\tree{
					\tree{
						\tmap{\Gamma}{1}; \emptyset;  \emptyset \proves   \inact  \hastype \Proc
					}{
						\tmap{\Gamma}{1}; \emptyset;  \dual{s}: \tinact\proves   \inact  \hastype \Proc
					}
					\quad 
					\tree{
						\tmap{\Gamma}{1}; \emptyset;  \tmap{\Delta}{1} \cat k:\tmap{S}{1}  x:\tmap{S_1}{1} \proves \pmap{Q}{1}   \hastype \Proc
					}{
						\tmap{\Gamma}{1}; \emptyset;  \tmap{\Delta}{1} \cat k:\tmap{S}{1}   \proves \abs{x} \pmap{Q}{1}   \hastype \lhot{\tmap{S_1}{1}}
					}
				}{
					\tmap{\Gamma}{1}; \emptyset;  \tmap{\Delta}{1} \cat k:\tmap{S}{1}  \cat \dual{s}: \btout{\lhot{\tmap{S_1}{1}}}\tinact\proves  \bbout{\dual{s}}{\abs{x}{\pmap{Q}{1}}} \inact  \hastype \Proc
				}
			\end{eqnarray}
%
			\begin{eqnarray}
				\label{prop:sesspnr_to_HO_t4}
		 		\tree{
					\begin{array}{cl}
						\tmap{\Gamma}{1}; \{X: U_1\}; \cat s: \btinp{\lhot{\tmap{S_1}{1}}}\tinact \ \proves \appl{X}{s}  \hastype \Proc
						& \eqref{prop:sesspnr_to_HO_t2}
						\\
						\tmap{\Gamma}{1}; \emptyset; \tmap{\Delta}{1} \cat k:\tmap{S}{1} \cat \dual{s}: \btout{\lhot{\tmap{S_1}{1}}}\tinact \proves
						\bbout{\dual{s}}{\abs{x}{\pmap{Q}{1}}} \inact  \hastype \Proc
						& \eqref{prop:sesspnr_to_HO_t3}
					\end{array}
				}{
					\tmap{\Gamma}{1}; \{X: U_1\};  \tmap{\Delta}{1} \cat k:\tmap{S}{1} \cat s: \btinp{\lhot{\tmap{S_1}{1}}}\tinact \cat \dual{s}: \btout{\lhot{\tmap{S_1}{1}}}\tinact\proves \appl{X}{s} \Par \bbout{\dual{s}}{\abs{x}{\pmap{Q}{1}}} \inact  \hastype \Proc
			}
			\end{eqnarray}
%
			\begin{eqnarray*}
			\\
			 \tree{
				 \tree{
					\tmap{\Gamma}{1}; \{X: U_1\};  \tmap{\Delta}{1} \cat k:\tmap{S}{1} \cat s: \btinp{\lhot{\tmap{S_1}{1}}}\tinact \cat \dual{s}: \btout{\lhot{\tmap{S_1}{1}}}\tinact\proves \appl{X}{s} \Par \bbout{\dual{s}}{\abs{x}{\pmap{Q}{1}}} \inact  \hastype \Proc \quad \eqref{prop:sesspnr_to_HO_t4}
				}{
					\tmap{\Gamma}{1}; \{X: U_1\};  \tmap{\Delta}{1} \cat k:\tmap{S}{1} \proves \newsp{s}{\appl{X}{s} \Par \bbout{\dual{s}}{\abs{x}{\pmap{Q}{1}}} \inact}  \hastype \Proc
				}
			}{
				\tmap{\Gamma}{1}; \emptyset; \tmap{\Delta}{1}  \cat k:\btinp{U_1}\tmap{S}{1} \proves  \binp{k}{X} \newsp{s}{\appl{X}{s} \Par \bbout{\dual{s}}{\abs{x}{\pmap{Q}{1}}} \inact}  \hastype \Proc
			}
			\end{eqnarray*}
			 }
			 
			 In the second sub-case, $x$ stands for a shared name. Then we have the following typing in the source language:
			\[
			 \tree{
				\Gamma \cat x:\chtype{S_1} ; \emptyset; \Delta  \cat k:S \proves   Q \hastype \Proc
			 }{
				\Gamma ; \emptyset; \Delta  \cat k:\btinp{\chtype{S_1}}S \proves  \binp{k}{x} Q \hastype \Proc}
			 \]
			 The typing in the target language is derived similarly as in the first sub-case.	
%	\end{enumerate}
	%
%	\qed
%\end{proof}


%\begin{proposition}\rm
%	\label{app:enc_sesspnr_to_ho_oc}
%	Encoding $\encod{\cdot}{\cdot}{1}: \sessp^{-\mu} \to \HO$  enjoys operational correspondence (cf. Def.~\ref{def:ep}\,(2)).
%\end{proposition}
%
%\begin{proof}[Sketch]
%	We must show completeness and soundness properties. 
%	For completeness, it suffices to consider source process
%	$P_0 = \bout{k}{k'} P \Par \binp{k}{x} Q$. We have that
%%
%	\[
%		P_0 \red P \Par Q\subst{k'}{x}.
%	\]
%%
%	By the definition of encoding we have:
%	\begin{eqnarray*}
%		\pmap{P_0}{1} & = & \bbout{k}{ \abs{z}{\,\binp{z}{X} \appl{X}{k'}} } \pmap{P}{1} \Par \binp{k}{X} \newsp{s}{\appl{X}{s} \Par \bbout{\dual{s}}{\abs{x} \pmap{Q}{1}} \inact}  \\
%		& \red & \pmap{P}{1} \Par \newsp{s}{\appl{X}{s} \subst{\abs{z}{\,\binp{z}{X} \appl{X}{k'}}}{X} \Par \bbout{\dual{s}}{\abs{x}{\pmap{Q}{1}}} \inact} \\
%		& = & \pmap{P}{1} \Par \newsp{s}{\,\binp{s}{X} \appl{X}{k'} \Par \bbout{\dual{s}}{\abs{x}{\pmap{Q}{1}}} \inact} \\
%		& \red & \pmap{P}{1} \Par \appl{X}{k'} \subst{\abs{x} \pmap{Q}{1}}{X} \Par \inact \\
%		& \scong & \pmap{P}{1} \Par \pmap{Q}{1}\subst{k'}{x}  
%	\end{eqnarray*}
%	For soundness, it suffices to notice that the encoding does not add new visible actions:
%	the additional synchronizations induced by the encoding always occur on private (fresh) names.
%	We assume weak bisimilarities, which abstract from internal actions used by the encoding,
%	and so  constructing a relation witnessing behavioral equivalence is easy.
%	\qed
%\end{proof}



%\begin{proof}
%	By induction on the inference $\Gamma; \emptyset; \Delta \proves P \hastype \Proc$.
%	\begin{enumerate}[1.]
		\item	Case $P_0 = \rvar{X}$.
			Then we have the following typing in the source language:
%
			\[
				\Gamma \cat \rvar{X}: \Delta ;\, \es ;\, \es \proves \rvar{X} \hastype \Proc
			\]
%
			Then the typing of $\pmapp{\rvar{X}}{1}{f}$ is as follows,
			assuming $f(\rvar{X}) = \tilde{n}$ and $\tilde{x} = \vmap{\tilde{n}}$.
			Also, we write $\Delta_{\tilde{n}}$ 
			and $\Delta_{\tilde{x}}$ 
			to stand for 
			$n_1: S_1, \ldots, n_m: S_m$ and
			$x_1: S_1, \ldots, x_m: S_m$, respectively. 
			Below, we assume that $\Gamma = \Gamma' \cat X:\shot{\tilde{T}}$, 
			where  
			%$$\tilde{T} =  \trec{t}{\big(\tilde{S}, \btinp{\vart{t}}\tinact\big)}$$.
			\[
				\tilde{T} = \big(\tilde{S}, S^*\big) \qquad \quad
				S^* = \bbtinp{A}\tinact \qquad \quad
				A = \trec{t}{(\tilde{S}, \btinp{\vart{t}}\tinact)}
			\]
%
			\begin{eqnarray}
				\label{prop:sessp_to_HO_t1}
				\tree{
					\tree{
					}{
						\Gamma ;\, \es ;\, \es \proves X \hastype \shot{\tilde{T}}
					}
					\quad 
					\begin{array}{c}
						\Gamma ;\, \es ;\, \{n_i: S_i \} \proves n_i \hastype S_i \\
						\Gamma ;\, \es ;\, \{s: S^* \} \proves s\hastype S^*  \\
					\end{array}
				}{
					\Gamma  ;\, \es ;\, \Delta_{\tilde{n}}, s:\btinp{\shot{\tilde{T}}}\tinact
					\proves  
					\appl{\X}{\tilde{n}, s} \hastype \Proc
				} 
			\end{eqnarray}
%
			\begin{eqnarray}
				\label{prop:sessp_to_HO_t2}
				\tree{
					\tree{
						\Gamma  ;\, \es ;\,   \es \proves \inact \hastype \Proc
					}{
						\Gamma  ;\, \es ;\,   \dual{s}: \tinact \proves \inact \hastype \Proc
					} 
					\quad
					\tree{
						\tree{
							\begin{array}{c}
								\Gamma ;\, \es ;\, \{x_i: S_i \} \proves x_i \hastype S_i \\
								\Gamma ;\, \es ;\, \{z: S^*  \} \proves z\hastype S^*  \\
								\Gamma ;\, \es ;\, \es \proves X \hastype \shot{\tilde{T}}  \\
							\end{array}
						}{
							\Gamma  ;\, \es ;\,   \Delta_{\tilde{x}}, \, z:S^*
							\proves 
							 {\appl{X}{ \tilde{x}, z}} \hastype \Proc
						}
					}{
						\Gamma  ;\, \es ;\,   \es
						\proves 
						 \abs{\tilde{x},z}\,\,{\appl{X}{ \tilde{x}, z}} \hastype \shot{\tilde{T}}
					} 	
				}{
					\Gamma  ;\, \es ;\,   \dual{s}: \btout{\shot{\tilde{T}}}\tinact
					\proves 
					\bbout{\dual{s}}{ \abs{\tilde{x},z}\,\,{\appl{X}{ \tilde{x}, z}}} \inact \hastype \Proc
				}
			\end{eqnarray}
%
			\[
			\tree{
				\tree{
					\begin{array}{cc}
						\Gamma  ;\, \es ;\, \Delta_{\tilde{n}}, s:\btinp{\shot{\tilde{T}}}\tinact
						\proves  
						\appl{\X}{\tilde{n}, s} \hastype \Proc
						& \eqref{prop:sessp_to_HO_t1}
						\\ 
						\Gamma  ;\, \es ;\,   \dual{s}: \btout{\shot{\tilde{T}}}\tinact
						\proves 
						\bbout{\dual{s}}{ \abs{\tilde{x},z}\,\,{\appl{X}{ \tilde{x}, z}}} \inact \hastype \Proc
						& \eqref{prop:sessp_to_HO_t2}
					\end{array}
				}{
					\Gamma  ;\, \es ;\, \Delta_{\tilde{n}}, s:\btinp{\shot{\tilde{T}}}\tinact, \, \dual{s}: \btout{\shot{\tilde{T}}}\tinact
					\proves 
					\appl{\X}{\tilde{n}, s} \Par \bbout{\dual{s}}{ \abs{\tilde{x},z}\,\,{\appl{X}{ \tilde{x}, z}}} \inact \hastype \Proc
				}
			}{
				\Gamma  ;\, \es ;\, \Delta_{\tilde{n}}
				\proves 
				\newsp{s}{\appl{\X}{\tilde{n}, s} \Par \bbout{\dual{s}}{ \abs{\tilde{x},z}\,\,{\appl{X}{ \tilde{x}, z}}} \inact} \hastype \Proc
			}
			\]
%	
		\item	Case $P_0 = \recp{X}{P}$. Then we have the following typing in the source language:
%
			\[
				\tree{
					\Gamma \cat \rvar{X}:\Delta ;\, \es ;\,  \Delta \proves P \hastype \Proc
				}{
					\Gamma  ;\, \es ;\,  \Delta \proves \recp{X}{P} \hastype \Proc
				}
			\]
%	
			Then we have the following typing in the target language ---we write $R$
			to stand for $\pmapp{P}{1}{{f,\{\rvar{X}\to \tilde{n}\}} }$
			and $\tilde{x}$ to stand for $\vmap{\ofn{P}}$.
%
			\begin{eqnarray}
				\label{prop:sessp_to_HO_t4}
				\tree{
					\tree{
						\tmap{\Gamma}{1}\cat X:\shot{\tilde{T}};\, \es;\, \tmap{\Delta_{\tilde{n}}}{1}
						\proves
						 R  \hastype \Proc
					}{
						\tmap{\Gamma}{1}\cat X:\shot{\tilde{T}};\, \es;\, \tmap{\Delta_{\tilde{n}}}{1}, s:\tinact 
						\proves
						 R  \hastype \Proc
					}
				}{
					\tmap{\Gamma}{1};\, \es;\, \tmap{\Delta_{\tilde{n}}}{1}, s:\btinp{\shot{\tilde{T}}}\tinact 
					\proves
					\binp{s}{\X} R  \hastype \Proc
				}
			\end{eqnarray}
%
			\begin{eqnarray}
				\label{prop:sessp_to_HO_t5}
				\tree{
					\tree{
						\tmap{\Gamma}{1};\, \es;\, \es
						\proves
						\inact \hastype \Proc
					}{
						\tmap{\Gamma}{1};\, \es;\, \dual{s}:\tinact
						\proves
						\inact \hastype \Proc
					} 
					\quad 
					\tree{
						\tree{
							\tree{
								\tmap{\Gamma}{1} \cat X: \shot{\tilde{T}};\, \es;\, \tmap{\Delta_{\tilde{x}}}{1}
								\proves
								{{\auxmap{R}{\es}}}  \hastype \Proc
							}{
								\tmap{\Gamma}{1} \cat X: \shot{\tilde{T}};\, \es;\, \tmap{\Delta_{\tilde{x}}}{1},z: \tinact
								\proves
								{{\auxmap{R}{\es}}}  \hastype \Proc
							}
						}{
							\tmap{\Gamma}{1};\, \es;\, \tmap{\Delta_{\tilde{x}}}{1}, \, z: \btinp{A}\tinact
							\proves
							{{\binp{z}{\X} \auxmap{R}{\es}}}  \hastype \Proc
						}
					}{
						\tmap{\Gamma}{1};\, \es;\, \es
						\proves
						{\abs{\tilde{x}, z } \,{\binp{z}{\X} \auxmap{R}{\es}}}  \hastype \shot{\tilde{T}}
					}
				}{
					\tmap{\Gamma}{1};\, \es;\, \dual{s}:\btout{\shot{\tilde{T}}}\tinact
					\proves
					\bbout{\dual{s}}{\abs{\tilde{x}, z } \,{\binp{z}{\X} \auxmap{R}{\es}}} \inact \hastype \Proc
				}
			\end{eqnarray}
%
			\[
			\tree{
				\tree{
					\begin{array}{cc}
						\tmap{\Gamma}{1};\, \es;\, \tmap{\Delta_{\tilde{n}}}{1}, s:\btinp{\shot{\tilde{T}}}\tinact 
						\proves
						\binp{s}{\X} R  \hastype \Proc
						& \eqref{prop:sessp_to_HO_t4}
						\\
						\tmap{\Gamma}{1};\, \es;\, \dual{s}:\btout{\shot{\tilde{T}}}\tinact
						\proves
						\bbout{\dual{s}}{\abs{\tilde{x}, z } \,{\binp{z}{\X} \auxmap{R}{\es}}} \inact \hastype \Proc
						& \eqref{prop:sessp_to_HO_t5}
					\end{array}
				}{
					\tmap{\Gamma}{1};\, \es;\, \tmap{\Delta_{\tilde{n}}}{1}, s:\btinp{\shot{\tilde{T}}}\tinact , \dual{s}:\btout{\shot{\tilde{T}}}\tinact
					\proves
					\binp{s}{\X} R \Par \bbout{\dual{s}}{\abs{\tilde{x}, z } \,{\binp{z}{\X} \auxmap{R}{\es}}} \inact \hastype \Proc
				}
			}{
				\tmap{\Gamma}{1};\, \es;\, \tmap{\Delta_{\tilde{n}}}{1} 
				\proves
				\newsp{s}{\binp{s}{\X} R \Par \bbout{\dual{s}}{\abs{\tilde{x}, z } \,{\binp{z}{\X} \auxmap{R}{\es}}} \inact} \hastype \Proc
			}
			\]
	\end{enumerate}
	\qed
\end{proof}

%\begin{proposition}\rm
%	\label{app:enc_sesp_to_HO_oc}
%	Encoding $\fencod{\cdot}{\cdot}{2}{f}: \sessp \to \HO$ 
%	enjoys operational correspondence (cf. Def.~\ref{def:ep}\,(2)).
%\end{proposition}
%
%\begin{proof}[Sketch]
%\dk{TBD.}
%\end{proof}

%%%%%%%%%%%%%%%%%%%%%%%%%%%%%%%%%%%%%%%%%%%%%%%%%
% HOp TO SESSP
%%%%%%%%%%%%%%%%%%%%%%%%%%%%%%%%%%%%%%%%%%%%%%%%%


\subsection{Properties for $\enco{\pmap{\cdot}{2}, \tmap{\cdot}{}, \mapa{\cdot}^{2}}: \HOp \to \sessp$}
\label{app:enc:HOp_to_sessp}

We repeat the statement of Prop.~\ref{prop:typepres_HOp_to_FO},
as in Page \pageref{prop:typepres_HOp_to_FO}:

\begin{proposition}[Type Preservation, \HOp into \sessp]\rm
	Let $P$ be a \HOp process. 
	If $\Gamma; \emptyset; \Delta \proves P \hastype \Proc$ then 
	$\mapt{\Gamma}^{2}; \emptyset; \mapt{\Delta}^{2} \proves \map{P}^{2} \hastype \Proc$.
\end{proposition}


%\begin{proposition}[Type Preservation, Higher-Order into First-Order]
%Let $P$ be an  $\HO$ process. 
%If			$\Gamma; \emptyset; \Delta \proves P \hastype \Proc$ then 
%			$\mapt{\Gamma}^{2}; \emptyset; \mapt{\Delta}^{2} \proves \map{P}^{2} \hastype \Proc$. 
%\end{proposition}

\begin{proof}
	By induction on the inference $\Gamma; \emptyset; \Delta \proves P \hastype \Proc$.
%	By induction on the structure of \HO process $P$.  \jp{TO BE ADJUSTED!}
	\begin{enumerate}[1.]

	%%%% Output of (linear) channel
		\item	Case $P = \bbout{k}{\abs{x}{Q}}P$. Then we have two possibilities, depending on the typing for $\abs{x}Q$.
			The first case concerns a linear typing, and  
			we have the following typing in the source language:
%
			\[
				\tree{
					\Gamma; \emptyset; \Delta_1 \cat k:S  \proves  P \hastype \Proc
					\quad
					\tree{
						\Gamma ; \emptyset ; \Delta_2\cat x:S_1 \proves  Q \hastype \Proc
					}{
						\Gamma ; \emptyset ; \Delta_2 \proves  \abs{x}Q \hastype \lhot{S_1}
					}
				}{
					\Gamma; \emptyset; \Delta_1 \cat \Delta_2 \cat k:\btout{\lhot{S_1}}S \proves  \bbout{k}{\abs{x}{Q}} P \hastype \Proc
				}
			\]
%			
			This way, by IH we have
			$$
			\tmap{\Gamma}{2}; \es ; \tmap{\Delta_2}{2}, x:\tmap{S_1}{2}
									\proves 
									\pmap{Q}{2} \hastype \Proc
			$$
			Let us write 
			 $U_1$ to stand for 
			$\chtype{\btinp{\tmap{S_1}{2}}\tinact}$.
			The corresponding typing in the target language is as follows: 
%
			\begin{eqnarray*}
				\tmap{\Gamma_1}{2} & = & \tmap{\Gamma}{2} \cup a:\chtype{\btinp{\tmap{S_1}{2}}\tinact} \\
				\tmap{\Gamma_2}{2} & = & \tmap{\Gamma_1}{2} \cup \rvar{X}:\tmap{\Delta_2}{2}
			\end{eqnarray*}
%
			Also $(*)$ stands for $\tmap{\Gamma_1}{2}; \es ; \es \proves a \hastype U_1$; 
			$(**)$ stands for $\tmap{\Gamma_2}{2}; \es ; \es \proves a \hastype U_1$; and
			$(***)$ stands for $\tmap{\Gamma_2}{2}; \es ; \es \proves \rvar{X} \hastype \Proc$.
			\begin{eqnarray}
				\label{prop:HO_to_sessp_t1}
				\tree{
					\tree{
						\tree{
						}{
							(***)
						} 
						\quad 
						\tree{
							\tree{
								\tree{
									\tree{
									}{
										\tmap{\Gamma_2}{2}; \es ; \tmap{\Delta_2}{2},  x:\tmap{S_1}{2}
										\proves 
										\pmap{Q}{2} \hastype \Proc
									}
								}{
									\tmap{\Gamma_2}{2}; \es ; \tmap{\Delta_2}{2}, y:\tinact, x:\tmap{S_1}{2}
									\proves 
									\pmap{Q}{2} \hastype \Proc
								}
							}{
								\tmap{\Gamma_2}{2}; \es ; \tmap{\Delta_2}{2}, y: \btinp{\tmap{S_1}{2}}\tinact
								\proves 
								\binp{y}{x}\pmap{Q}{2} \hastype \Proc
							} 
							\quad 
							\tree{
							}{
								(**)
							}
						}{
							\tmap{\Gamma_2}{2}; \es ; \tmap{\Delta_2}{2} 
							\proves 
							\binp{a}{y}\binp{y}{x}\pmap{Q}{2} \hastype \Proc
						} 
					}{
						\tmap{\Gamma_2}{2}; \es ; \tmap{\Delta_2}{2} 
						\proves 
						\binp{a}{y}\binp{y}{x}\pmap{Q}{2} \Par \rvar{X} \hastype \Proc
					}
				}{
					\tmap{\Gamma_1}{2}; \es ; \tmap{\Delta_2}{2} 
					\proves 
					\recp{X}{(\binp{a}{y}\binp{y}{x}\pmap{Q}{2} \Par \rvar{X})} \hastype \Proc
				}
			\end{eqnarray}
%
			\begin{eqnarray}
				\label{prop:HO_to_sessp_t2}
				\tree{
					\begin{array}{c}
						\tmap{\Gamma_1}{2}; \es ; \tmap{\Delta_1}{2}, k:\tmap{S}{2} 
						\proves 
						\pmap{P}{2}  \hastype \Proc
						\\
						\tmap{\Gamma_1}{2}; \es ; \tmap{\Delta_2}{2} 
						\proves 
						\recp{X}{(\binp{a}{y}\binp{y}{x}\pmap{Q}{2} \Par \rvar{X})} \hastype \Proc
						\quad \eqref{prop:HO_to_sessp_t1}
					\end{array}
				}{
					\tmap{\Gamma_1}{2}; \es ; \tmap{\Delta_1, \Delta_2}{2}, k:\tmap{S}{2} 
					\proves 
					\pmap{P}{2} \Par 
					\recp{X}{(\binp{a}{y}\binp{y}{x}\pmap{Q}{2} \Par \rvar{X})} \hastype \Proc
				}
			\end{eqnarray}
%
			\[
				\tree{
					\tree{
						\begin{array}{c}
							\tmap{\Gamma_1}{2}; \es ; \es \proves a \hastype U_1
							\\
							\tmap{\Gamma_1}{2}; \es ; \tmap{\Delta_1, \Delta_2}{2}, k:\tmap{S}{2} 
							\proves 
							\pmap{P}{2} \Par 
							\recp{X}{(\binp{a}{y}\binp{y}{x}\pmap{Q}{2} \Par \rvar{X})} \hastype \Proc
							\quad \eqref{prop:HO_to_sessp_t2}
						\end{array}
					}{
						\tmap{\Gamma_1}{2}; \es ; \tmap{\Delta_1, \Delta_2}{2}, k:\bbtout{U_1}\tmap{S}{2} 
						\proves 
						\bout{k}{a}(\pmap{P}{2} \Par 
						\recp{X}{(\binp{a}{y}\binp{y}{x}\pmap{Q}{2} \Par \rvar{X}))} \hastype \Proc
					}
				}{
					\tmap{\Gamma}{2}; \es ; \tmap{\Delta_1, \Delta_2}{2}, k:\bbtout{U_1}\tmap{S}{2} 
					\proves 
					\newsp{a}{\bout{k}{a}( 
					\pmap{P}{2} \Par 
					\recp{X}{(\binp{a}{y}\binp{y}{x}\pmap{Q}{2} \Par \rvar{X}))}} \hastype \Proc
				}
			\]
%
			In the second case, $\abs{x}Q$ has a shared type. We have the following typing in the source language:
%
			\[
				\tree{
					\Gamma; \emptyset; \Delta \cat k:S  \proves  P \hastype \Proc
					\quad 
					\tree{
						\tree{
							\Gamma ; \emptyset ; \cat x:S_1 \proves  Q \hastype \Proc
						}{
							\Gamma ; \emptyset ; \es \proves  \abs{x}Q \hastype \lhot{S_1}
						}
					}{
						\Gamma ; \emptyset ; \es \proves  \abs{x}Q \hastype \shot{S_1}
					}
				}{
					\Gamma; \emptyset; \Delta  \cat k:\btout{\shot{S_1}}S \proves  \bbout{k}{\abs{x}{Q}} P \hastype \Proc
				}
			\]
%
			The corresponding typing in the target language can be derived similarly as in the first case.
	
		\item	Case $P = \binp{k}{X} P$. Then there are two cases, depending on the type of $X$. 
			In the first case,
			we have the following typing in the source language:
%
			\[
				\tree{
					\Gamma \cat X : \shot{S_1};\, \emptyset ;\, \Delta \cat k:S \proves  P \hastype \Proc
				}{
					\Gamma;\, \emptyset;\, \Delta\cat k:\btinp{\shot{S_1}}S \proves  \binp{k}{X} P \hastype \Proc
				}
			\]
			The corresponding typing in the target language is as follows:
			% --- we write $\Gamma_0$ to stand for $\Gamma \setminus \{X: \lhot{S_1}\}$.
%
			\[
				\tree{
					\tree{}{\tmap{\Gamma}{2} \cat x : \chtype{\btinp{\tmap{S_1}{2}}\tinact};\, \emptyset ;\, \Delta \cat k:\tmap{S}{2} \proves  \tmap{P}{2} \hastype \Proc}
				}{
					\tmap{\Gamma}{2};\, \emptyset; \, \tmap{\Delta}{2}\cat k:\bbtinp{\chtype{\btinp{\tmap{S_1}{2}}\tinact}}\tmap{S}{2} \proves
					\binp{k}{x} \pmap{P}{2} \hastype \Proc
				}
			\]
%
			In the second case,  
			we have the following typing in the source language:
%
			\[
				\tree{
					\Gamma;\, \{X : \lhot{S_1}\};\, \emptyset ;\, \Delta \cat k:S \proves  P \hastype \Proc
				}{
					\Gamma;\, \emptyset;\, \Delta\cat k:\btinp{\lhot{S_1}}S \proves  \binp{k}{X} P \hastype \Proc
				}
			\]
%
			The corresponding typing in the target language is as follows:
			% --- we write $\Gamma_0$ to stand for $\Gamma \setminus \{X: \lhot{S_1}\}$.
%
			\[
				\tree{
					\tmap{\Gamma}{2} \cat x : \chtype{\btinp{\tmap{S_1}{2}}\tinact};\, \emptyset ;\, \Delta \cat k:\tmap{S}{2} \proves  \tmap{P}{2} \hastype \Proc
				}{
					\tmap{\Gamma}{2};\, \emptyset;\, \tmap{\Delta}{2}\cat k:\bbtinp{\chtype{\btinp{\tmap{S_1}{2}}\tinact}}\tmap{S}{2} \proves
					\binp{k}{x} \pmap{P}{2} \hastype \Proc
				}
			\]
%
		\item	Case $P = \appl{X}{k}$. Also here we have two cases, depending on whether $X$ has linear or shared type.
			In the first case, $X$ is linear and
			we have the following typing in the source language:
%
			\[
				\tree{
					\Gamma ;\, \{X : \lhot{S_1}\};\,  \es \proves  X \hastype \lhot{S_1} \quad \Gamma; \es ; \{k:S_1\} \proves k \hastype S_1
				}{
					\Gamma;\, \{X : \lhot{S_1}\};\, k:S_1 \proves  \appl{X}{k} \hastype \Proc}
			\]
			Let us write
			$\tmap{\Gamma_1}{2}$ to stand for $\tmap{\Gamma}{2} \cat x:\chtype{\btout{\tmap{S_1}{2}}\tinact}$.
			The corresponding typing in the target language is as follows:
%
			\begin{eqnarray}
				\label{prop:HO_to_sessp_t11}
				\tree{
					\tree{
						\tmap{\Gamma_1}{2};\, \es;\,  \es \proves  \inact \hastype \Proc
					}{
						\tmap{\Gamma_1}{2};\, \es;\,  \dual{s}:\tinact \proves  \inact \hastype \Proc
					}
					\quad 
						\tmap{\Gamma_1}{2};\, \es;\, \{k:\tmap{S_1}{2}\} \proves  k \hastype \tmap{S_1}{2} 
				}{
					\tmap{\Gamma_1}{2};\, \es;\,\, k:\tmap{S_1}{2},\,  \dual{s}:\btout{\tmap{S_1}{2}}\tinact \proves  \bout{\dual{s}}{k}\inact \hastype \Proc
				}
			\end{eqnarray}
%
			\[
				\tree{
					\tree{
						\begin{array}{c}
							\tmap{\Gamma_1}{2};\, \es;\,\, k:\tmap{S_1}{2},\,  \dual{s}:\btout{\tmap{S_1}{2}}\tinact \proves
							\bout{\dual{s}}{k}\inact \hastype \Proc
							\quad \eqref{prop:HO_to_sessp_t11}
							\\
							\tmap{\Gamma_1}{2} ;\, \es ;\, \es \proves x \hastype \chtype{\btout{\tmap{S_1}{2}}\tinact}
						\end{array}
					}{
						\tmap{\Gamma_1}{2};\, \es;\, k:\tmap{S_1}{2}, s:\btinp{\tmap{S_1}{2}}\tinact , \dual{s}:\btout{\tmap{S_1}{2}}\tinact
						\proves
						\bout{x}{s}\bout{\dual{s}}{k}\inact \hastype \Proc
					}
				}{
					\tmap{\Gamma_1}{2};\, \es;\, k:\tmap{S_1}{2} \proves  \news{s}{(\bout{x}{s}\bout{\dual{s}}{k}\inact)} \hastype \Proc
				}
	\]
%
			In the second case, $X$ is shared, and
			we have the following typing in the source language:
%
			\[
				\tree{
					\Gamma \cat  X : \lhot{S_1} ;\,  \es ;\,  \es \proves  X \hastype \shot{S_1} \quad \Gamma; \es ; k:S_1 \proves k \hastype S_1
				}{
					\Gamma \cat X : \shot{S_1};\, \es ;\, k:S_1 \proves  \appl{X}{k} \hastype \Proc
				}
			\]
%
			The associated typing in the target language is obtained similarly as in the first case. \qed
	\end{enumerate}
\end{proof}


%\begin{proposition}\rm
%	\label{app:enc_HO_to_sessp_oc}
%	Encoding $\encod{\cdot}{\cdot}{2}: \HO \to \sessp$ 
%	enjoys operational correspondence (cf. Def.~\ref{def:ep}\,(2)).
%\end{proposition}
%
%\begin{proof}[Sketch]
%For completeness, we 
%consider the \HO process $P = {\bbout{k}{\abs{x}{Q}} P_1} \Par \binp{k}{X} P_2$. We have that
%\[
%P \red P_1 \Par P_2 \subst{\abs{x}Q}{X}
%\]
%In the target language, this reduction is mimicked as follows:
%\begin{eqnarray*}
%\pmap{P}{2} & = & \newsp{a}{\bout{k}{a} (\pmap{P_1}{2} \Par \repl{} \binp{a}{y} \binp{y}{x} \pmap{Q}{2})\,} 
%                  \Par \binp{k}{x} \pmap{P_2}{2} \\
%            & \red & \newsp{a}{\pmap{P_1}{2} \Par \repl{} \binp{a}{y} \binp{y}{x} \pmap{Q}{2} 
%                  \Par  \pmap{P_2}{2}\subst{a}{x}}
%\end{eqnarray*}
%\qed
%\end{proof}


%% !TEX root = ../journal16kpy.tex

\section{Negative Result}
\label{app:neg}

\begin{theorem}%\myrm
%	\label{thm:negative}
	Let $\CAL_1, \CAL_2 \in \set{\HOp, \HO, \sessp}$.
	There is no typed, minimal encoding from $\tyl{L}_{\CAL_1}$ into $\tyl{L}_{\CAL_2^{\minussh}}$
%	$\enco{\map{\cdot}, \mapt{\cdot}, \mapa{\cdot}}: \sessp \longrightarrow \HOp^{\minussh}$.
%	that enjoys: (i) homomorphism wrt parallel; (ii) barb preservation; (iii) operational completeness.
\end{theorem}

\begin{proof}
	Assume, towards a contradiction, that such a typed encoding indeed exists. 
	Consider the $\sessp$ process
	%
	\[
		P = \breq{a}{s} \inact \Par \bacc{a}{x} \bsel{n}{l_1} \inact \Par \bacc{a}{x} \bsel{m}{l_2} \inact \qquad \text{(with $n \neq m$)}
	\]
	%
	\noi such that 
	$\Gamma; \es; \Delta \proves P \hastype \Proc$.
	From process $P$ we have: %We then have both
	%
	\begin{eqnarray}
		& & \horel{\Gamma}{\Delta}{P}{\hby{\tau}}{\Delta'}{\bsel{n}{l_1} \inact \Par \bacc{a}{x} \bsel{m}{l_2} \inact = P_1} \label{eq:nn3} \\
		& & \horel{\Gamma}{\Delta}{P}{\hby{\tau}}{\Delta'}{\bsel{m}{l_2} \inact \Par \bacc{a}{x} \bsel{n}{l_1} \inact = P_2} \label{eq:nn4}
	\end{eqnarray}
	%
	Thus, by definition of typed barb we  have:
	%
	\begin{eqnarray}
		\Gamma; \Delta' \proves P_1 \barb{n} & \land & 
		\Gamma; \Delta' \proves P_1 \nbarb{m} \label{eq:nn1} \\
		\Gamma; \Delta' \proves P_2 \barb{m} & \land & 
		\Gamma; \Delta' \proves P_2 \nbarb{n} \label{eq:nn2}
	\end{eqnarray}
	%
	Consider now the $\HOp^{\minussh}$ process $\map{P}$.
	% = 
	% \map{\breq{a}{s} \inact} \Par \map{\bacc{a}{x} \bsel{n}{l_1} \inact} \Par \map{\bacc{a}{x} \bsel{m}{l_2}}$.
	By our assumption of operational completeness 
	(\defref{def:ep}-2(a)), 
	from \eqref{eq:nn3} with \eqref{eq:nn4}
	we infer that
	there exist $\HOp^{\minussh}$ processes $S_1$ and $S_2$ such that:
	%we have both:
	\begin{eqnarray}
		& & \horel{\mapt{\Gamma}}{\mapt{\Delta}}{\map{P}}{\Hby{\stau}}{\mapt{\Delta'}}{S_1 \WB \map{P_1}} \label{eq:n1} \\
		& & \horel{\mapt{\Gamma}}{\mapt{\Delta}}{\map{P}}{\Hby{\stau}}{\mapt{\Delta'}}{S_2 \WB \map{P_2}} \label{eq:n2}
		%\map{P} & \Hby{} &  S_1 \WB \map{P_1} \\
		%s\map{P} & \Hby{} & S_2 \WB \map{P_2}
	\end{eqnarray}
	By our assumption of barb preservation, 
	from \eqref{eq:nn1} with \eqref{eq:nn2}
	we infer: 
	%
	\begin{eqnarray}
		\mapt{\Gamma}; \mapt{\Delta'} \proves \map{P_1} \Barb{n} & \land & 
		\mapt{\Gamma}; \mapt{\Delta'} \proves \map{P_1} \nBarb{m} \label{eq:n3} \\
		\mapt{\Gamma}; \mapt{\Delta'} \proves \map{P_2} \Barb{m} & \land & 
		\mapt{\Gamma}; \mapt{\Delta'} \proves \map{P_2} \nBarb{n} \label{eq:n4}
	\end{eqnarray}
	%
	By definition of $\WB$, 
	by combining~\eqref{eq:n1} with~\eqref{eq:n3}
	and~\eqref{eq:n2} with~\eqref{eq:n4}, we infer barbs for $S_1$ and $S_2$:
	\begin{eqnarray}
		\mapt{\Gamma}; \mapt{\Delta'} \proves S_1 \Barb{n} & \land & 
		\mapt{\Gamma}; \mapt{\Delta'} \proves S_1 \nBarb{m} \label{eq:n5} \\
		\mapt{\Gamma}; \mapt{\Delta'} \proves S_2 \Barb{m} & \land & 
		\mapt{\Gamma}; \mapt{\Delta'} \proves S_2 \nBarb{n} \label{eq:n6}
	\end{eqnarray}
	That is, $S_1$ and $\map{P_1}$ 
	(resp. $S_2$ and $\map{P_2}$)
	have the same barbs.
	Now, by $\tau$-inertness (\propref{lem:tau_inert}), we have both 
	\begin{eqnarray}
		& & \horel{\mapt{\Gamma}}{\mapt{\Delta}}{S_1}{\WB}{\mapt{\Delta'}}{\map{P}} \label{eq:n7} \\
		& & \horel{\mapt{\Gamma}}{\mapt{\Delta}}{S_2}{\WB}{\mapt{\Delta'}}{\map{P}} \label{eq:n8}
	\end{eqnarray}
	Combining~\eqref{eq:n7} with~\eqref{eq:n8}, by transitivity of $\WB$,
	we have 
	\begin{equation}
		\horel{\mapt{\Gamma}}{\mapt{\Delta'}}{S_1}{\WB}{\mapt{\Delta'}}{S_2} \label{eq:n9}
	\end{equation}
	In turn, from~\eqref{eq:n9}
	we infer that 
	it must be the case that:
	\begin{eqnarray*}
		\mapt{\Gamma}; \mapt{\Delta'} \proves \map{P_1} \Barb{n} & \land & 
		\mapt{\Gamma}; \mapt{\Delta'} \proves \map{P_1} \Barb{m} \label{eq:n10} \\
		\mapt{\Gamma}; \mapt{\Delta'} \proves \map{P_2} \Barb{m} & \land & 
		\mapt{\Gamma}; \mapt{\Delta'} \proves \map{P_2} \Barb{n} \label{eq:n11}
	\end{eqnarray*}
	which clearly contradict \eqref{eq:n3} and \eqref{eq:n4} above.
	\qed
\end{proof}


%\begin{theorem}\rm
%	There is no encoding $\enco{\map{\cdot}, \mapt{\cdot}, \mapa{\cdot}}: \HOp \longrightarrow \HOp^{\minussh}$
%	that enjoys operational correspondence and full abstraction.
%\end{theorem}

%\begin{proof}
%	Let $\horel{\Gamma_1}{\Delta_1}{P_1}{\not\wb}{\Delta_2}{P_2}$
%	with $P = \breq{a}{s} \inact \Par \bacc{a}{x} P_1 \Par \bacc{a}{x} P_2$ and
%	let $\Gamma; \emptyset; \Delta \proves P \hastype \Proc$.
%	Assume also a encoding
%	$\enco{\map{\cdot}, \mapt{\cdot}, \mapa{\cdot}}: \HOp \longrightarrow \HOp^{\minussh}$
%	that enjoys
%	operational correspondence and full abstraction.
%
%	From operational correspondence we get that:
%	\begin{eqnarray*}
%		P \red P_1 \Par \bacc{a}{x} P_2 &\textrm{implies}& \map{P} \red \map{P_1 \Par \bacc{a}{x} P_2}\\
%		P \red P_2 \Par \bacc{a}{x} P_1 &\textrm{implies}& \map{P} \red \map{P_2 \Par \bacc{a}{x} P_1}
%	\end{eqnarray*}
%
%	From the fact that
%	$\horel{\Gamma_1}{\Delta_1}{P_1}{\not\wb}{\Delta_2}{P_2}$
%	we can derive that
%%
%	\[
%		\horel{\Gamma_1'}{\Delta_1'}{P_1 \Par \bacc{a}{x} P_2}{\not\wb}{\Delta_2'}{P_2 \Par \bacc{a}{x} P_1}
%	\]
%%
%	From Corollary~\ref{cor:tau_inert} we know that
%%
%	\begin{eqnarray*}
%		\horel{\mapt{\Gamma}}{\mapt{\Delta}}{\map{P}}{\wb}{\mapt{\Delta_1'}}{\map{P_1 \Par \bacc{a}{x} P_2}}\\
%		\horel{\mapt{\Gamma}}{\mapt{\Delta}}{\map{P}}{\wb}{\mapt{\Delta_2'}}{\map{P_2 \Par \bacc{a}{x} P_1}}
%	\end{eqnarray*}
%%
%	\noi thus
%	\[
%		\horel{\mapt{\Gamma}}{\mapt{\Delta_1'}}{\map{P_1 \Par \bacc{a}{x} P_2}}{\wb}{\mapt{\Delta_2'}}{\map{P_2 \Par \bacc{a}{x} P_1}}
%	\]
%%
%	From here we conclude that the full abstraction property does not hold,
%	which is a contradiction.
%	\qed
%%	so there is no mapping $\map{\cdot}: \pHO \longrightarrow \spi$ that enjoys
%%	the operational correspondence and full abstraction properties.
%\end{proof}


\end{document}


