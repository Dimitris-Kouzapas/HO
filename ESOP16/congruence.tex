\subsection{Reduction-Closed, Barbed Congruence}
\label{subsec:rc}

\noi In this section we define reduction-closed, barbed congruence as the
reference equivalence relation. We first define \emph{confluence}
over session environments $\Delta$:

\begin{definition}[Session Environment Confluence]
	We denote $\Delta_1 \bistyp \Delta_2$ if there exists $\Delta$ such that
	$\Delta_1 \red^\ast \Delta$ and $\Delta_2 \red^\ast \Delta$
	\jpc{(here we write $\red^\ast$ for the multi-step reduction in \defref{def:ses_red})}.
\end{definition}

%\smallskip 
\noi Reduction-closed, barbed congrunce is defined over typed
processes:

\begin{definition}[Typed Relation]
	We say that
	$\Gamma; \emptyset; \Delta_1 \proves P_1 \hastype \Proc\ \Re \ \Gamma; \emptyset; \Delta_2 \proves P_2 \hastype \Proc$
	is a {\em typed relation} whenever
	$P_1$ and $P_2$ are closed;
	$\Delta_1$ and $\Delta_2$ are balanced; and 
	$\Delta_1 \bistyp \Delta_2$.
	We write $\horel{\Gamma}{\Delta_1}{P_1}{\ \Re \ }{\Delta_2}{P_2}$
	for the typed relation $\Gamma; \emptyset; \Delta_1 \proves P_1 \hastype \Proc\ \Re \ \Gamma; \emptyset; \Delta_2 \proves P_2 \hastype \Proc$.
\end{definition}

%\smallskip 

\noi Typed relations relate only closed terms whose
session environments %and the two session environments
are balanced  and confluent.
Next we define  {\em barbs}~\cite{MiSa92}
with respect to types. 

%\smallskip 

\begin{definition}[Barbs]\rm
	Let $P$ be a closed process. We define:
	\begin{enumerate}
		\item	$P \barb{n}$ if $P \scong \newsp{\tilde{m}}{\bout{n}{V} P_2 \Par P_3}, n \notin \tilde{m}$. %; $P \Barb{n}$ if $P \red^* \barb{n}$.

		\item	$\Gamma; \Delta \proves P \barb{n}$ if
			$\Gamma; \emptyset; \Delta \proves P \hastype \Proc$ with $P \barb{n}$ and $\dual{n} \notin \dom{\Delta}$.

			$\Gamma; \Delta \proves P \Barb{n}$ if $P \red^* P'$ and
			$\Gamma; \Delta' \proves P' \barb{n}$.			
	\end{enumerate}
\end{definition}

%\smallskip 

\noi A barb $\barb{n}$ is an observable on an output prefix with subject $n$.
Similarly a weak barb $\Barb{n}$ is a barb after a number of reduction steps.
Typed barbs $\barb{n}$ (resp.\ $\Barb{n}$)
occur on typed processes $\Gamma; \emptyset; \Delta \proves P \hastype \Proc$.
When $n$ is a session name we require that its dual endpoint $\dual{n}$ is not present
in the session environment $\Delta$.

To define a congruence relation, we introduce the family $\C$ of contexts:

\begin{definition}[Context]
	A context $\C$ is defined as:

	\begin{tabular}{rcl}
		$\C$ & $\bnfis$ & $\hole \bnfbar \bout{u}{V} \C \bnfbar \binp{u}{x} \C \bnfbar \bout{u}{\lambda x.\C} P \bnfbar \news{n} \C
		\bnfbar (\lambda x.\C)u \bnfbar \recp{X}{\C}$ 
		\\
		&$\bnfbar$& $\C \Par P \bnfbar P \Par \C
		\bnfbar \bsel{u}{l} \C \bnfbar \bbra{u}{l_1:P_1,\cdots,l_i:\C,\cdots,l_n:P_n}$
	\end{tabular}
%\smallskip 

\noi 
Notation $\context{\C}{P}$ replaces 
the hole $\hole$ in $\C$ with $P$.
\end{definition}

%\smallskip 

\noi We define reduction-closed, barbed congruence \cite{HondaKYoshida95}. 

%\smallskip 

\begin{definition}[Reduction-Closed, Barbed Congruence]
\label{def:rc}
	Typed relation
	$\horel{\Gamma}{\Delta_1}{P_1}{\ \Re\ }{\Delta_2}{P_2}$
	is a {\em reduction-closed, barbed congruence} whenever:
	\begin{enumerate}[1)]
		\item	If $P_1 \red P_1'$ then there exist $P_2', \Delta_2'$ such that $P_2 \red^* P_2'$ and
			$\horel{\Gamma}{\Delta_1'}{P_1'}{\ \Re\ }{\Delta_2'}{P_2'}$; 
			%and its symmetric case;
%		\item	If $P_2 \red P_2'$ then $\exists P_1', P_1 \red^* P_1'$ and
%		$\horel{\Gamma}{\Delta_1'}{P_1'}{\ \Re\ }{\Delta_2'}{P_2'}$
%		\end{itemize}

%		\item
%		\begin{itemize}
			\item	If $\Gamma;\Delta_1 \proves P_1 \barb{n}$ then $\Gamma;\Delta_2 \proves P_2 \Barb{n}$; %and its symmetric case; 

%			\item	If $\Gamma;\emptyset;\Delta \proves P_2 \barb{s}$ then $\Gamma;\emptyset;\Delta \proves P_1 \Barb{s}$.
%		\end{itemize}

		\item	For all $\C$, there exist $\Delta_1'',\Delta_2''$: $\horel{\Gamma}{\Delta_1''}{\context{\C}{P_1}}{\ \Re\ }{\Delta_2''}{\context{\C}{P_2}}$;
		                      \item	The symmetric cases of 1 and 2.                
	\end{enumerate}
	The largest such relation is denoted with $\cong$.
\end{definition}

