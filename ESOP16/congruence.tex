% !TEX root = main.tex
\noi In this section we define reduction-closed, barbed congruence ($\cong$) as the
reference equivalence relation for \HOp processes.
Later on we will define two characterizations of $\cong$:
\emph{higher-order} and  
\emph{characteristic bisimilarities} (denoted $\hwb$ and $\fwb$, respectively). 
Here we focus on collecting intuitions; omitted details are in the Appendix and in~\cite{KouzapasPY15}.

\subsection{Reduction-Closed, Barbed Congruence}
\label{subsec:rc}
We first define \emph{confluence} over session environments $\Delta$:

\begin{definition}[Session Environment Confluence]
Let $\red^\ast$ denote multi-step reduction as in \defref{def:ses_red}.
	We denote $\Delta_1 \bistyp \Delta_2$ if there exists $\Delta$ such that
	$\Delta_1 \red^\ast \Delta$ and $\Delta_2 \red^\ast \Delta$.
\end{definition}

%\smallskip 
\noi Reduction-closed, barbed congruence is defined over typed
processes:

\begin{definition}[Typed Relation]
	We say that
	$\Gamma; \emptyset; \Delta_1 \proves P_1 \hastype \Proc\ \Re \ \Gamma; \emptyset; \Delta_2 \proves P_2 \hastype \Proc$
	is a {\em typed relation} whenever
	$P_1$ and $P_2$ are closed;
	$\Delta_1$ and $\Delta_2$ are balanced; and 
	$\Delta_1 \bistyp \Delta_2$.
	We write $\horel{\Gamma}{\Delta_1}{P_1}{\ \Re \ }{\Delta_2}{P_2}$
	for the typed relation $\Gamma; \emptyset; \Delta_1 \proves P_1 \hastype \Proc\ \Re \ \Gamma; \emptyset; \Delta_2 \proves P_2 \hastype \Proc$.
\end{definition}

%\smallskip 

\noi Observe that typed relations relate only closed terms whose
session environments %and the two session environments
are balanced  and confluent.
Next we define  {\em barbs}~\cite{MiSa92}
with respect to types. 

%\smallskip 

\begin{definition}[Barbs]\rm
	Let $P$ be a closed process. We define:
	\begin{enumerate}
		\item	$P \barb{n}$ if $P \scong \newsp{\tilde{m}}{\bout{n}{V} P_2 \Par P_3}, n \notin \tilde{m}$. %; $P \Barb{n}$ if $P \red^* \barb{n}$.

		\item	$\Gamma; \Delta \proves P \barb{n}$ if
			$\Gamma; \emptyset; \Delta \proves P \hastype \Proc$ with $P \barb{n}$ and $\dual{n} \notin \dom{\Delta}$.

			$\Gamma; \Delta \proves P \Barb{n}$ if $P \red^* P'$ and
			$\Gamma; \Delta' \proves P' \barb{n}$.			
	\end{enumerate}
\end{definition}

%\smallskip 

\noi A barb $\barb{n}$ is an observable on an output prefix with subject $n$.
Similarly a weak barb $\Barb{n}$ is a barb after a number of reduction steps.
Typed barbs $\barb{n}$ (resp.\ $\Barb{n}$)
occur on typed processes $\Gamma; \emptyset; \Delta \proves P \hastype \Proc$.
When $n$ is a session name we require that its dual endpoint $\dual{n}$ is not present
in the session environment $\Delta$.

To define a congruence relation, we introduce the family $\C$ of contexts:

\begin{definition}[Context]
	A context $\C$ is defined as:

	\begin{tabular}{rcl}
		$\C$ & $\bnfis$ & $\hole \bnfbar \bout{u}{V} \C \bnfbar \binp{u}{x} \C \bnfbar \bout{u}{\lambda x.\C} P \bnfbar \news{n} \C
		\bnfbar (\lambda x.\C)u \bnfbar \recp{X}{\C}$ 
		\\
		&$\bnfbar$& $\C \Par P \bnfbar P \Par \C
		\bnfbar \bsel{u}{l} \C \bnfbar \bbra{u}{l_1:P_1,\cdots,l_i:\C,\cdots,l_n:P_n}$
	\end{tabular}
%\smallskip 

\noi 
Notation $\context{\C}{P}$ replaces 
the hole $\hole$ in $\C$ with $P$.
\end{definition}

%\smallskip 

\noi We define reduction-closed, barbed congruence \cite{HondaKYoshida95}. 

%\smallskip 

\begin{definition}[Reduction-Closed, Barbed Congruence]
\label{def:rc}
	Typed relation
	$\horel{\Gamma}{\Delta_1}{P_1}{\ \Re\ }{\Delta_2}{P_2}$
	is a {\em reduction-closed, barbed congruence} whenever:
	\begin{enumerate}[1)]
		\item	If $P_1 \red P_1'$ then there exist $P_2', \Delta_2'$ such that $P_2 \red^* P_2'$ and
			$\horel{\Gamma}{\Delta_1'}{P_1'}{\ \Re\ }{\Delta_2'}{P_2'}$; 
			%and its symmetric case;
%		\item	If $P_2 \red P_2'$ then $\exists P_1', P_1 \red^* P_1'$ and
%		$\horel{\Gamma}{\Delta_1'}{P_1'}{\ \Re\ }{\Delta_2'}{P_2'}$
%		\end{itemize}

%		\item
%		\begin{itemize}
			\item	If $\Gamma;\Delta_1 \proves P_1 \barb{n}$ then $\Gamma;\Delta_2 \proves P_2 \Barb{n}$; %and its symmetric case; 

%			\item	If $\Gamma;\emptyset;\Delta \proves P_2 \barb{s}$ then $\Gamma;\emptyset;\Delta \proves P_1 \Barb{s}$.
%		\end{itemize}

		\item	For all $\C$, there exist $\Delta_1'',\Delta_2''$: $\horel{\Gamma}{\Delta_1''}{\context{\C}{P_1}}{\ \Re\ }{\Delta_2''}{\context{\C}{P_2}}$;
		                      \item	The symmetric cases of 1 and 2.                
	\end{enumerate}
	The largest such relation is denoted with $\cong$.
\end{definition}

{
\subsection{Two Equivalence Relations: $\fwb$ and $\hwb$}\label{ss:equiv}

\jparagraph{A Typed Labelled Transition System}
In~\cite{characteristic_bis} we have characterised
reduction-closed, barbed congruence for \HOp
via a typed relation called
{\em characteristic bisimilarity}.
%The definition of characteristic bisimilarity 
Its definition 
uses a \emph{typed}
labelled transition system (LTS) informed by session
types. 
Transitions in this LTS are denoted 
$\Gamma; \es; \Delta \proves P \hby{\ell} \Delta' \proves P' \hastype \Delta'$.
The main intuition %for its definition  
is that the transitions 
of a typed process should be enabled by its associated typing environment:
%
\[
	\tree {
		P \hby{\ell} P' \qquad (\Gamma, \Delta) \hby{\ell} (\Gamma, \Delta')
	}{
		\Gamma; \es; \Delta \proves P \hby{\ell} \Delta' \proves P' \hastype \Delta'
	}
\]
%
\noi As an example of how types enable transitions, consider the rule for input:
%
\[
	\tree{
		\dual{s} \notin \dom{\Delta} \qquad \Gamma; \Lambda'; \Delta' \proves V \hastype U
		\qquad
		V = m \vee  V \scong \omapchar{U} \vee V \scong \abs{{x}}{\binp{t}{y} (\appl{y}{{x}})}
					\textrm{ with } t \textrm{ fresh} 
	}{
		(\Gamma; \Lambda; \Delta \cat s: \btinp{U} S) \hby{\bactinp{s}{V}} (\Gamma; \Lambda\cat\Lambda'; \Delta\cat\Delta' \cat s: S)
	}
\]
\noi
This rule states that a session channel environment can input a value
if the channel is typed with an input prefix and the input value is either
a name $m$, a \emph{characteristic value} $\omapchar{U}$,  or a \emph{trigger value} (the abstraction
$\abs{{x}}{\binp{t}{y} (\appl{y}{{x}})}$). 
A characteristic value
is the {simplest} process that  inhabits a type (here the
type $U$ carried by the session input prefix). The above rule is used to limit
the input actions that can be observed from a session input prefix.
For details of the labelled transition system and the characteristic process definition
see \appref{app:behavioural} and~\cite{characteristic_bis}.
Moreover, we define a \emph{(first-order) trigger process}:
%
\begin{eqnarray}
%	\htrigger{t}{V}  & \defeq &  \hotrigger{t}{V} \label{eqb:0} \\
	\ftrigger{t}{V}{U} & \defeq &  \fotrigger{t}{x}{s}{\btinp{U} \tinact}{V} 	\label{eq:fot}
\end{eqnarray}
%
\noi
Trigger processes are defined as a process input prefixed on
a fresh name $t$. 
%The   trigger process $\htrigger{t}{V}$ above applies a value $V$
%on the receiving (characteristic) process.
The trigger process $\ftrigger{t}{V}{U}$ applies a value
on the (hard coded) \emph{characteristic process} $\map{\btinp{U} \tinact}^{s}$ (see~\cite{characteristic_bis} for details). 

\jparagraph{Characterisations of Barbed Congruence}
We now define \emph{characteristic} and \emph{higher-order} bisimilarity.
Observe that higher-order bisimilarity is a new typed equality, 
while 
characteristic bisimilarity was introduced in~\cite{characteristic_bis} (Def.~14).

\begin{definition}[Characteristic Bisimilarity]%\rm
	\label{d:fwb}
	A typed relation $\Re$ is a {\em  characteristic bisimulation} if 
	for all $\horel{\Gamma}{\Delta_1}{P_1}{\ \Re \ }{\Delta_2}{Q_1}$ 
	\begin{enumerate}[1)]
		\item 
				Whenever 
				$\horel{\Gamma}{\Delta_1}{P_1}{\hby{\news{\tilde{m_1}} \bactout{n}{V_1: U}}}{\Delta_1'}{P_2}$,
				there exist 
				$Q_2$, $V_2$, $\Delta'_2$
				such that \\
				$\horel{\Gamma}{\Delta_2}{Q_1}{\Hby{\news{\tilde{m_2}} \bactout{n}{V_2: U}}}{\Delta_2'}{Q_2}$ and, for fresh $t$, 
%
				\[
					\begin{array}{lrlll}
						\Gamma; \Delta''_1  \proves  {\newsp{\tilde{m_1}}{P_2 \Par \ftrigger{t}{V_1}{U_1}}}
						\ \Re \
						\Delta''_2 \proves {\newsp{\tilde{m_2}}{Q_2 \Par \ftrigger{t}{V_2}{U_2}}}
					\end{array}
				\]

		\item	
				For all $\horel{\Gamma}{\Delta_1}{P_1}{\hby{\ell}}{\Delta_1'}{P_2}$ such that 
				$\ell$ is not an output, 
				there exist $Q_2$, $\Delta'_2$ such that 
				$\horel{\Gamma}{\Delta_2}{Q_1}{\Hby{\hat{\ell}}}{\Delta_2'}{Q_2}$
				and
				$\horel{\Gamma}{\Delta_1'}{P_2}{\ \Re \ }{\Delta_2'}{Q_2}$; and 

		\item	The symmetric cases of 1 and 2.                
	\end{enumerate}
	The largest such bisimulation
	is called \emph{characteristic bisimilarity} \jpc{and} denoted by $\fwb$.
\end{definition}
 

%\noi 
%We define characteristic bisimilarity
%(cf.~Def.~14~in~\cite{characteristic_bis}):
% is given using characteristic trigger processes. 

Interestingly, for reasoning about \HO processes we can exploit a simpler technique: \emph{higher-order bisimilarity}.
We replace triggers as in \eqref{eq:fot}
with \emph{higher-order triggers}:
\begin{eqnarray}
	\htrigger{t}{V}  & \defeq &  \hotrigger{t}{V} \label{eq:hot} 
%	\ftrigger{t}{V}{U} & \defeq &  \fotrigger{t}{x}{s}{\btinp{U} \tinact}{V} 	\label{eq:fot}
\end{eqnarray}
We may then define:
\begin{definition}[Higher-Order Bisimilarity]%\rm
	\label{d:hbw}
	Higher-order bisimilarity, denoted by $\hwb$, is defined \jpc{by} replacing 
	Clause 1) in \defref{d:fwb} with the following clause:\\[1mm]
	Whenever 
	$\horel{\Gamma}{\Delta_1}{P_1}{\hby{\news{\tilde{m_1}} \bactout{n}{V_1}}}{\Delta_1'}{P_2}$ %with $\Gamma; \es; \Delta \proves V_1 \hastype U$,  
	then there exist 
	$Q_2$, $V_2$, $\Delta'_2$
	such that \\
	$\horel{\Gamma}{\Delta_2}{Q_1}{\Hby{\news{\tilde{m_2}}\bactout{n}{V_2}}}{\Delta_2'}{Q_2}$ %with $\Gamma; \es; \Delta' \proves V_2 \hastype U$,  
	and, for fresh $t$, \\[1mm]
	$
	\begin{array}{lrlll}
		\!\!\Gamma; \Delta''_1  \proves  {\newsp{\tilde{m_1}}{P_2 \Par 
		\htrigger{t}{V_1}}}
		\ \!\!\Re\!\!\ \Delta''_2
		\proves {\newsp{\tilde{m_2}}{Q_2 \Par \htrigger{t}{V_2}}}
	\end{array}
	$
\end{definition}

We state the following important result, which attests the significance of $\hwb$:
\begin{theorem}
	Typed relations $\cong$, $\hwb$, and $\fwb$ coincide for \HOp processes.
\end{theorem}
\begin{proof}
Coincidence of $\cong$ and $\fwb$ was established in~\cite{characteristic_bis}.
Coincidence of $\hwb$ with $\cong$ and $\fwb$ is a new result: see \cite{KouzapasPY15}
for details. \qed
\end{proof}

\begin{remark}[Comparison between $\hwb$ and $\fwb$]
	The difference between higher order bisimilarity
	lies in the definition of the process trigger.
	\begin{enumerate}[(a)]
		\item
				The higher-order bisimilarity trigger \eqref{eq:hot}
%				\[
%					\htrigger{t}{V}  \defeq  \hotrigger{t}{V}
%				\]
				allows for the application of bound variable $x$,
				making $\hwb$ not defined in the \sessp sub-calculus
				of \HOp, but defined inside the \HO sub-calculus. 

		\item	The Characteristic bisimilarity trigger \eqref{eq:fot} 
%				\[
%					\ftrigger{t}{V}{U} \defeq  \fotrigger{t}{x}{s}{\btinp{U} \tinact}{V}
%				\]
				solves the problem in (a) by not using $x$ in the continuation
				of fresh name input. This allows to input characteristic, first- or
				higher-order value on bound variable $x$.


		\item	Higher-order trigger lifts the need of observing
				types on the lts and including them in bisimulation
				closures. Higher-order bisimilarity adds to
				the simplicity and elegance of the \HO subcalculus
				over the \sessp subcalculus of \HOp.
	\end{enumerate}
\end{remark}

%\begin{definition}[Characteristic Bisimilarity]\rm
%	\label{d:fwb}
%
%	A typed relation $\Re$ is a {\em characteristic bisimulation} if 
%	for all $\horel{\Gamma}{\Delta_1}{P_1}{\ \Re \ }{\Delta_2}{Q_1}$ 
%	if whenever:
%	\begin{enumerate}[1)]
%		\item	$\horel{\Gamma}{\Delta_1}{P_1}{\hby{\news{\tilde{m_1}} \bactout{n}{V_1: U}}}{\Delta_1'}{P_2}$
%				then there exist  $Q_2$, $V_2$, $\Delta'_2$ such that 
%				$\horel{\Gamma}{\Delta_2}{Q_1}{\Hby{\news{\tilde{m_2}}\bactout{n}{V_2: U}}}{\Delta_2'}{Q_2}$
%				and, for fresh $t$,
%				$
%				\begin{array}{lrlll}
%					\Gamma; \Delta''_1 \proves {\newsp{\tilde{m_1}}{P_2 \Par  \ftrigger{t}{V_1}{U_1}}}
%					\ \Re\ \Delta''_2 \proves {\newsp{\tilde{m_2}}{Q_2 \Par \ftrigger{t}{V_2}{U_2}}}
%				\end{array}
%				$
%
%			\item	For all $\horel{\Gamma}{\Delta_1}{P_1}{\hby{\ell}}{\Delta_1'}{P_2}$ such that 
%					$\ell$ is not an output, there exist $Q_2$, $\Delta'_2$ such that 
%					$\horel{\Gamma}{\Delta_2}{Q_1}{\Hby{\hat{\ell}}}{\Delta_2'}{Q_2}$
%					and $\horel{\Gamma}{\Delta_1'}{P_2}{\ \Re \ }{\Delta_2'}{Q_2}$; and 
%
%			\item	The symmetric cases of 1 and 2.
%	\end{enumerate}
%	The largest such bisimulation is called \emph{characteristic bisimilarity} and denoted by $\fwb$.
%\end{definition}


\jparagraph{An up-to technique}
We close this section by stating  a determinacy result for typed processes, useful in proving our expressiveness results (\secref{sec:positive}).
In our setting, processes that do not use shared names are inherently deterministic. 
In the sequel, we write 
 $\horel{\Gamma}{\Delta}{P}{\hby{\tau}}{\Delta'}{P'}$ to denote an internal (typed) transition.
The auxiliary definition below allows us to distinguish two kinds of  internal transitions:
\emph{session transitions} and \emph{$\beta$-transitions} (denoted 
$\horel{\Gamma}{\Delta}{P}{\hby{\stau}}{\Delta'}{P'}$
and $\horel{\Gamma}{\Delta}{P}{\hby{\btau}}{\Delta'}{P'}$, respectively).

\begin{definition}[Deterministic Transition]
\label{def:dettrans}
	Let  $\Gamma; \es; \Delta \proves P \hastype \Proc$ be a balanced \HOp process. 
	Transition $\horel{\Gamma}{\Delta}{P}{\hby{\tau}}{\Delta'}{P'}$ is called:
	\begin{enumerate}[$-$]
		\item	{\em Session transition}
				whenever the untyped transition $P \by{\tau} P'$ 
				is derived using  rule~$\ltsrule{Tau}$ 
				(where $\subj{\ell_1}$ and $\subj{\ell_2}$ in the premise are dual endpoints), 
				possibly followed by uses of
				$\ltsrule{Alpha}$, $\ltsrule{Res}$, $\ltsrule{Rec}$, or $\ltsrule{Par${}_L$}/
				\ltsrule{Par${}_R$}$.
		
		\item	{\em $\beta$-transition}
				whenever the untyped transition $P \by{\tau} P'$
				is derived using rule $\ltsrule{App}$,
				possibly followed by uses of  $\ltsrule{Alpha}$, $\ltsrule{Res}$, $\ltsrule{Rec}$, or $\ltsrule{Par${}_L$}/
				\ltsrule{Par${}_R$}$.
	\end{enumerate}
%
	We write
	$\horel{\Gamma}{\Delta}{P}{\hby{\stau}}{\Delta'}{P'}$
	and 
	$\horel{\Gamma}{\Delta}{P}{\hby{\btau}}{\Delta'}{P'}$
	to denote session and $\beta$-transitions, resp. Also, 
	 $\horel{\Gamma}{\Delta}{P}{\hby{\dtau}}{\Delta'}{P'}$ denotes
	either a session transition or a $\beta$-transition.
\end{definition}
%
%A transition $\horel{\Gamma}{\Delta}{P}{\hby{\tau}}{\Delta'}{P'}$ is said
%{\em deterministic} if it is derived using~$\ltsrule{App}$ or~$\ltsrule{Tau}$ 
%(where $\subj{\ell_1}$ and $\subj{\ell_2}$ in the premise  are dual endpoints), 
%possibly followed by uses of  $\ltsrule{Alpha}$, $\ltsrule{Res}$, $\ltsrule{Rec}$, or $\ltsrule{Par${}_L$}/\ltsrule{Par${}_R$}$.

\begin{lemma}[$\tau$-Inertness]\rm
	\label{lem:tau_inert}
	\begin{enumerate}[1)]
		\item
				Let $\horel{\Gamma}{\Delta}{P}{\hby{\dtau}}{\Delta'}{P'}$ be a deterministic transition,
				with balanced $\Delta$. \\ Then 
				$\Gamma; \Delta \proves P \cong \Delta'\proves P'$ 
				with $\Delta \red^\ast \Delta'$ balanced.

		\item 
				Let $P$ be an $\HOp^{-\mathsf{sh}}$ process. 
				Assume $\Gamma; \emptyset; \Delta \proves P \hastype \Proc$. \\ Then 
				$P \red^\ast P'$ implies $\Gamma; \Delta \proves 
				P \cong \Delta'\proves P'$ with $\Delta \red^\ast \Delta'$. 
	\end{enumerate}
\end{lemma}


\begin{proof}
	The proof uses the fact that processes of the
	form $\Gamma; \es; \Delta \proves_s \bout{s}{V} P_1 \Par \binp{k}{x} P_2$
	cannot have any typed transition observables and the fact
	that bisimulation is a congruence.
	See  \appref{app:sub_tau_inert} for details.
	The proof for Part 2 follows from Part 1.
	\qed
\end{proof}

Using the above determinacy properties, we can state the following up-to technique.
We write $\Hby{\dtau}$ to denote a (possibly empty) sequence of deterministic steps 
$\hby{\dtau}$.


\begin{lemma}[Up-to Deterministic Transition]%\myrm
	\label{lem:up_to_deterministic_transition}
	Let $\horel{\Gamma}{\Delta_1}{P_1}{\ \Re\ }{\Delta_2}{Q_1}$ such
	that if whenever:
%
	\begin{enumerate}
		\item	$\forall \news{\tilde{m_1}} \bactout{n}{V_1}$ such that
			$
				\horel{\Gamma}{\Delta_1}{P_1}{\hby{\news{\tilde{m_1}} \bactout{n}{V_1}}}{\Delta_3}{P_3}
			$
			implies that $\exists Q_2, V_2$ such that
			$
				\horel{\Gamma}{\Delta_2}{Q_1}{\Hby{\news{\tilde{m_2}} \bactout{n}{V_2}}}{\Delta_2'}{Q_2}
			$
			and
			$
				\horel{\Gamma}{\Delta_3}{P_3}{\Hby{\dtau}}{\Delta_1'}{P_2}
			$
			and for fresh $t$:
			$
				\horel{\Gamma}{\Delta_1''}{\newsp{\tilde{m_1}}{P_2 \Par \htrigger{t}{V_1}}}
				{\ \Re\ }
				{\Delta_2''}{}{\newsp{\tilde{m_2}}{Q_2 \Par \htrigger{t}{V_2}}}
%				\mhorel{\Gamma}{\Delta_1''}{\newsp{\tilde{m_1}}{P_2 \Par \hotrigger{t}{x}{s}{V_1}}}
%				{\ \Re\ }
%				{\Delta_2''}{}{\newsp{\tilde{m_2}}{Q_2 \Par \hotrigger{t}{x}{s}{V_2}}}
			$
%
		\item	$\forall \ell \not= \news{\tilde{m}} \bactout{n}{V}$ such that
			$
				\horel{\Gamma}{\Delta_1}{P_1}{\hby{\ell}}{\Delta_3}{P_3}
			$
			implies that $\exists Q_2$ such that 
			$
				\horel{\Gamma}{\Delta_1}{Q_1}{\hat{\Hby{\ell}}}{\Delta_2'}{Q_2}
			$
			and
			$
				\horel{\Gamma}{\Delta_3}{P_3}{\Hby{\dtau}}{\Delta_1'}{P_2}
			$
			and
			$\horel{\Gamma}{\Delta_1'}{P_2}{\ \Re\ }{\Delta_2'}{Q_2}$

		\item	The symmetric cases of 1 and 2.
	\end{enumerate}
	Then $\Re\ \subseteq\ \wb$.
\end{lemma}


%\begin{proof}
%	The proof is easy by considering the closure
%	\[
%		\Re^{\Hby{\dtau}} = \set{ \horel{\Gamma}{\Delta_1'}{P_2}{,}{\Delta_2'}{Q_1} \setbar \horel{\Gamma}{\Delta_1}{P_1}{\ \Re\ }{\Delta_2'}{Q_1},
%		\horel{\Gamma}{\Delta_1}{P_1}{\Hby{\dtau}}{\Delta_1'}{P_2} }
%	\]
%	We verify that $\Re^{\Hby{\dtau}}$ is a bisimulation with
%	the use of \propref{lem:tau_inert}.
%	\qed
%\end{proof}
%
%\begin{example}[Up-to Deterministic Transition]
%	Typed processes:
%	\begin{eqnarray*}
%		\Gamma; \es; \Delta, s': \tinact \proves P &=& \binp{n}{z_1} \newsp{s}{\binp{s}{x} \appl{(\abs{y}{\bout{n}{z_1} \inact})}{m} \Par \bout{\dual{s}}{s'} \inact} \hastype \Proc
%		\\
%		\Gamma; \es; \Delta \proves Q &=& \binp{n}{z_1} \binp{n}{z_2} \inact \hastype \Proc
%	\end{eqnarray*}
%	are bisimilar up-to deterministic transition because
%	we can observe:
%	\begin{eqnarray*}
%		\Gamma; \Delta, s': \tinact \proves P &\hby{\bactinp{n}{m_1}}& \Delta', s: \tinact \proves \newsp{s}{\binp{s}{x} \appl{(\abs{y}{\bout{n}{z_2} \inact})}{m} \Par \bout{\dual{s}}{s'} \inact} \Hby{\dtau} \Delta' \proves \binp{n}{z_2} \inact
%		\\
%		\text{and}
%		\\
%		\Gamma; \es; \Delta \proves Q &\hby{\bactinp{n}{m_1}}& \Delta' \proves \binp{n}{z_2} \inact
%	\end{eqnarray*}


%	Relation 
%	\[
%		\Re = \set{(\Gamma; \Delta, s': \tinact \proves P , \Delta \proves Q), (\Gamma; \Delta' \proves \binp{n}{z_2}, \Delta' \proves  \binp{n}{z_2})}
%	\]
%	is bisimulation up-to deterministic transition because
%	\begin{eqnarray*}
%		\Gamma; \Delta, s': \tinact \proves P &\hby{\bactinp{n}{s_1}}& \Delta', s: \tinact \proves \newsp{s}{\binp{s}{x} \appl{(\abs{y}{\bout{n}{y} \inact})}{s_1} \Par \bout{\dual{s}}{s'} \inact}
%		\\
%		\text{implies}&
%		\\
%		\Gamma; \es; \Delta \proves Q &\hby{\bactinp{n}{x}}& \Delta' \proves \binp{n}{z_2} \inact
%		\\
%		\text{and}&
%		\\
%		\Delta', s: \tinact \proves \newsp{s}{\binp{s}{x} \appl{(\abs{y}{\bout{n}{y} \inact})}{s_1} \Par \bout{\dual{s}}{s'} \inact}  \in \Re
%	\end{eqnarray*}
%\end{example}

%\noi Precise encodings offer more detailed criteria and used for positive 
%encodability results (\secref{sec:positive}).
%In contrast, minimal encodings contains only 
%some of the criteria of precise encodings:    
%this reduced notion will be used 
%for the negative result in \secref{sec:negative}.


