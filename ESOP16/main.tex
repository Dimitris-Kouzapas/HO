%\documentclass{sigplanconf}
 \documentclass[runningheads]{llncs} 

\usepackage[dvipsnames]{xcolor}
\usepackage{amsmath}
\usepackage{amssymb}
\usepackage{xspace}
\usepackage{graphicx}
\usepackage{latexsym}
\usepackage{listings}
\usepackage{multirow}
\usepackage{suffix}
\usepackage{url}
\usepackage{mathptmx}
\usepackage{mathrsfs}
\usepackage{comment}
\usepackage{enumerate}
\usepackage{txfonts}
\usepackage{hyperref}
\usepackage{fancybox}
%\usepackage{space}
\usepackage{color}      % use if color is used in text

\usepackage{tikz}	% for drawing figures

\usepackage{caption}	% for subfigures
\usepackage{subcaption}	% for subfigures


% correct bad hyphenation here
\hyphenation{op-tical net-works semi-conduc-tor}

\begin{document}

%\special{papersize=8.5in,11in}
%\setlength{\pdfpageheight}{\paperheight}
%\setlength{\pdfpagewidth}{\paperwidth}

%\conferenceinfo{CONF 'yy}{Month d--d, 20yy, City, ST, Country} 
%\copyrightyear{20yy} 
%\copyrightdata{978-1-nnnn-nnnn-n/yy/mm} 
%\doi{nnnnnnn.nnnnnnn}


\title{On the Relative Expressiveness of\\ 
%Type-preserving Encodability for \\
Higher-Order Session Processes
}

\author{
	Dimitrios Kouzapas%}{University of Glasgow}{dimitrios.kouzapas@gla.ac.uk}
	\and
	Jorge A. P\'{e}rez
	\and Nobuko Yoshida
}
%	{University of Groningen \and Imperial College London}
%	{}
\institute{University of Glasgow, Imperial College London, and  University of Groningen }
\maketitle


\input{macros}

%\pagestyle{plain}

% As a general rule, do not put math, special symbols or citations
% in the abstract
%\begin{abstract}
% !TEX root = main.tex
\begin{abstract}
This work proposes %efficient 
tractable
bisimulations 
for the higher-order $\pi$-calculus with session primitives~(\HOp).
We develop three typed bisimulations, which are shown to 
coincide with contextual equivalence.
These characterisations  
demonstrate that observing as inputs
only a specific finite set of higher-order values (which inhabit session types) suffices 
to reason about \HOp processes. 
\end{abstract}
%\end{abstract}

%\pagestyle{plain}

% no keywords

% For peer review papers, you can put extra information on the cover
% page as needed:
% \ifCLASSOPTIONpeerreview
% \begin{center} \bfseries EDICS Category: 3-BBND \end{center}
% \fi
%
% For peerreview papers, this IEEEtran command inserts a page break and
% creates the second title. It will be ignored for other modes.
%\IEEEpeerreviewmaketitle

%\input{proposale}

\section{Introduction}
\label{sec:intro}
%% !TEX root = main.tex
%\myparagraph{Key points}
%\begin{enumerate}[1.]
%%	\item	Session $\pi$ calculus with process passing. DONE
%%	\item	Identify session $\pi$ and process passing subcalculi and their polyadic variants. DONE
%%	\item	Bisimulation theory for higher-order session semantics. DONE
%%	\item	New triggered bisimulation, related to J\&R's. DONE
%%	\item   Elementary values key to characterizations of behavioural equivalence. DONE
%	\item	Types provide techniques to prove completeness without matching. \jp{TBD}
%	\item	We are interested in encodings with properties a la Gorla. 
%                We extended them to typed setting. \jp{TBD}
%%	\item	Encode name-passing to pure process abstraction calculus, with name abstractions. DONE
%%	\item	Type of the recursion encoding uses non tail recursive type $\trec{t}{\btinp{t} \tinact}$. DONE
%%	\item	Encode higher-order semantics to first order semantics. DONE
%%	\item	Negative result. Cannot encode shared names using only shared names.
%%	\item   Extensions with higher-order abstractions and polyadicity also explored. DONE
%\end{enumerate}

%\smallskip 
%
%\myparagraph{Important things to explain}
%Explain our \HO is very small without containg name passing 
%\[ 
%\abs{x}.P \quad \appl{x}{u}
%\]

%Explain we input only characteristic processes.  
%
%\[
%\lambda x.\mapchar{S}{x}
%\]

%\subsection{Higher-Order Session Calculi}
\noi 
This paper is about \emph{relative expressiveness} results for 
\emph{higher-order process calculi}, core programming languages that 
integrate name- and process-passing in communications.
We focus on higher-order calculi coupled with \emph{session types} that denote interaction protocols. 
%The expressivity relations that we aim to are essential for the transference of reasoning techniques, notably behavioral equivalences, across different typed languages. 
%Our motivation for studying expressivity relations is to establish a firm foundation for the transference of reasoning techniques, notably behavioral equivalences, across different session typed languages. 
%Our results %of relative expressiveness 
%enable us to identify 
%a \emph{core} calculus of higher-order concurrency with session primitives, equally expressive as several other
%calculi with sessions.
%\emph{essential} constructs, i.e., constructs those not expressible in terms of other constructs. 
%They are also central ingredients in the   \emph{transference of reasoning techniques}, notably behavioral theories, across different languages.
Establishing expressiveness results is challenging partly because 
%Since session types denote protocols, 
 it formally entails defining 
 not only a translation (\emph{encoding})
relating source and target languages, but also a translation 
relating their associated session type systems. 
We thus aim at a very particular class of correct encodings: namely \emph{fully abstract} and \emph{type-preserving} encodings.
%This is a challenging task, as we elaborate next.
In the following, we elaborate on our aims,   approach, and contributions.

\jparagraph{Context}
In \emph{session-based concurrency}, concurrent interactions are organized into \emph{sessions}, basic communication units.
Interaction patterns can then be abstracted as expressive \emph{session types}~\cite{honda.vasconcelos.kubo:language-primitives}, against which process specifications may be checked. 
These patterns are defined as %(possibly recursive) 
sequences of communication actions: % (send/receive a value, offer/select a behavior).
%For instance, 
%session type $T_1 = \btinp{\mathsf{str}} \btout{\mathsf{int}}  \tinact$ may be intuitively read as: receive (?) a value of type $\mathsf{str}$,then output (!) a value of type $\mathsf{int}$, finally close the protocol.
type $\btinp{U} S$ (resp.  $\btout{U} S$)
describes a protocol that first receives (resp. sends) a value of type $U$ and then continues as protocol $S$.
Also, given an index set $I$, types $\btbra{l_i:S_i}_{i \in I}$ 
and $\btsel{l_i:S_i}_{i \in I}$ 
define %, respectively,
%a branching and selection constructs for  
 a labeled choice mechanism; types 
$\trec{t}{S}$ 
and 
$\tinact$ denote recursive and completed protocols, respectively.
%describes a protocol that offers
%(resp. ) 
%Type $\tinact$ denotes the completed protocol.
In the (first-order) $\pi$-calculus~\cite{MilnerR:calmp1}, 
session types describe the intended interactive behavior of the names/channels in a process.
%names/channels are endowed with session types (such as $T_1$) representing their intended interactive behavior.
Session-based concurrency has also been casted in the setting of \emph{higher-order} process
calculi which, by combining features from the $\lambda$-calculus and the $\pi$-calculus, 
enable the exchange of values that may contain processes~\cite{tlca07,DBLP:journals/jfp/GayV10}. 
Besides offering a natural bridge between concurrent and functional computation, 
higher-order calculi with sessions enable the specification of structured protocols involving \emph{code mobility}, 
frequent in practice.
In the main higher-order session language that we study here (denoted \HOp),
 values in communications include names but also (first-order) abstractions---functions from name identifiers to processes. 
 %(In contrast, higher-order abstractions---functions from processes to processes---are disallowed.)
 (In contrast, functions from processes to processes are disallowed.)
Abstractions can be linear or shared; their value types are  denoted $\lhot{C}$ and $\shot{C}$, respectively ($C$ 
%is a first-order type $C$ (say, a session name).
denotes a name). We may then have 
session types such as
%$T_2 = \btbra{upload:\btinp{\lhot{\mathsf{int}}}\tinact ~ , ~ sha:\btinp{\shot{\mathsf{int}}}\tinact}_{}$
$$\btbra{up:\btinp{\lhot{C}}\btout{\mathsf{ok}}\tinact ~ , ~ down:\btout{\shot{C}}\btout{\mathsf{ok}}\tinact ~ , ~quit:\btout{\mathsf{bye}}\tinact}_{}$$
which abstracts the protocol of a server that allows 
  clients to select among distinct  behaviors: %namely, 
  to \emph{upload} a linear function (which the server is ready to receive), to \emph{download} a shared function (which the server is ready to send), or to \emph{quit} the protocol.
  In each case, the server sends a confirmation message ($\mathsf{ok}$ or $\mathsf{bye}$) before closing the session.




%The study of higher-order concurrency has received significant attention, 
%from untyped and typed perspectives (see, e.g.,~\cite{ThomsenB:plachoasgcfhop,SangiorgiD:expmpa,San96int,MostrousY15,DBLP:journals/iandc/LanesePSS11,DBLP:conf/icalp/LanesePSS10,DBLP:conf/esop/KoutavasH11,XuActa2012}),
%in particular via comparisons with the (first-order) $\pi$-calculus~\cite{MilnerR:calmp1}. 
%Higher-order calculi with {session primitives}, put forward in~\cite{tlca07,DBLP:journals/jfp/GayV10},
%allow for the specification of structured, reciprocal exchanges (protocols) 
%which can be verified via type-checking using {session types}~\cite{honda.vasconcelos.kubo:language-primitives}.
%Although models of session-typed 
%communication with features of higher-order concurrency exist
%their \emph{relative expressiveness}
%remains little understood. 

\jparagraph{The Problem}
Roughly speaking, 
  \HOp %, a higher-order process language that 
extends Sangiorgi's higher-order $\pi$-calculus~\cite{SangiorgiD:expmpa} with session primitives.
More precisely, 
\HOp
includes
constructs for 
%session establishment
synchronisation along shared names, 
session communication (value passing, labelled choice) along linear names,
recursion, 
 (first-order) abstractions %(i.e., functions from name identifiers  to processes)
 and applications.
% (denoted $\lambda x.P$ and $(\lambda x.P)a$, resp.).
%While synchronization on shared names (useful to model session establishment) is 
%non deterministic, session communication is deterministic and occurs on linear names.
\HOp is therefore a rather rich process language. This immediately begs the question:
is there a \emph{sub-calculus} of \HOp with equal expressivity? %hich is as expressive as the whole calculus? 
Besides its intrinsic foundational interest, this question also has practical ramifications, 
as such a \emph{core calculus} could be taken as reference in 
implementations of languages with session primitives.
In this way, expressiveness results may help in justifying connections 
between theoretical and more
practical advances in the design of (functional) programming languages with session types support.

%We have recently developed a behavioral theory  for \HOp~\cite{characteristic_bis}:
%we introduced
%\emph{characteristic bisimilarity}, a sound and complete 
%characterization of contextual equivalence. % that enables tractable analyses.


\jparagraph{Approach}
Previous studies already suggest two main sources 
of expressivity in \HOp: \emph{name passing} and constructs for \emph{infinite behavior} (i.e., recursion and replication). 
On the one hand, name-passing calculi (both untyped and typed) are able to express the $\lambda$-calculus and 
process-passing calculi~\cite{SangiorgiD:expmpa}. 
%In the $\pi$-calculus, recursion and replication can be expressed in terms of each other. 
On the other hand, in higher-order calculi,
recursion and replication operators are redundant: they can be represented exploiting process passing and duplication~\cite{ThomsenB:plachoasgcfhop}. 
Higher-order concurrency is itself quite expressive: 
higher-order calculi which lack name passing and recursion are known to be Turing equivalent~\cite{DBLP:journals/iandc/LanesePSS11}.
%For these reasons, 
%Consequently, 
Based on these observations, 
in this paper we study the expressivity of \HOp in relation to two  sub-calculi
that distill the essence of first- and higher-order session-based concurrency:
\begin{enumerate}[-]
\item the session \sessp-calculus \jpc{(denoted~\sessp)}: \HOp without abstractions and applications;
\item the \HO-calculus: \HOp without recursion and name passing.
\end{enumerate}
%Interestingly, %\HO and \sessp distil the essence of higher- and first-order session-based concurrency:
While \sessp is, 
in essence, the calculus in~\cite{honda.vasconcelos.kubo:language-primitives}, 
%this paper shows 
our main discovery is 
that \HO  is a new core calculus 
for higher-order session concurrency.
We assess the expressivity 
 of \HOp, \HO, and \sessp as delineated by session typing. 
We establish strong correspondences between 
these calculi  via type-preserving, fully abstract encodings up to 
behavioural equalities. 
%While encoding \HOp 
%into the $\pi$-calculus preserving session types 
%(extending  known  results for untyped processes~\cite{SangiorgiD:expmpa}) is 
%%\jpc{already}
%significant, 

\jparagraph{Contributions}
Our main contribution is 
an encoding of \HOp into \HO (\secref{subsec:HOpi_to_HO}).  
Since \HO lacks 
both name-passing and recursion, this encoding involves two key challenges:
\begin{enumerate}[a.]
\item Known (typed) encodings of name-passing into process-passing~\cite[Ch.13]{SaWabook} are limited: % in that 
%they come with restrictions on name usages;  
they work for %name-passing 
calculi 
%with \emph{capability types} 
in which only the output capability of names is sent---a received name cannot be used in subsequent inputs.
This is far too limiting in \HOp, where 
 names denoting arbitrary protocols may be passed around (\emph{delegation})
and types describe  communication  \emph{structures}, rather than individual name capabilities. % at a given time.

\item As mentioned above, 
recursion % and replication)
can be encoded in untyped higher-order calculi using process duplication. 
Unfortunately, this kind of encodings do not carry over to session typed calculi such as \HOp,
because linear abstractions cannot be copied. Hence, the discipline of session types severely limits 
the possibilities for representing infinite behaviors---even simple forms, such as input-guarded replication.
\end{enumerate}




%MOTIVATION FIRST ENCODING (). \emph{Still to highlight: recursive type required, no recursion, small example.

%--- 
\noi
%We illustrate our approach. % to these challenges.
Concerning (a), %we illustrate the essence of 
to encode name passing into \HO, 
%to encode name output, 
we ``pack''
the name to be passed around into a suitable abstraction; 
upon reception, the receiver must ``unpack'' this object following a precise protocol on a fresh  session:
%More precisely, our encoding \jpc{of name passing} in \HO is given as:
\begin{center}
\begin{tabular}{rcll}
  $\map{\bout{a}{b} P}$	&$=$&	$\bout{a}{ \abs{z}{\,\binp{z}{x} (\appl{x}{b})} } \map{P}$ \\
  $\map{\binp{a}{x} Q}$	&$=$&	$\binp{a}{y} \newsp{s}{\appl{y}{s} \Par \bout{\dual{s}}{\abs{x}{\map{Q}}} \inact}$
\end{tabular}
\end{center}
%and as a homomorphism for the other operators.
Above, 
%where
$a,b$ are names and $s$ and $\dual{s}$ are 
linear session names (\emph{endpoints}).
%$\lambda x.P$ is a name abstraction of $P$; $\appl{x}{a}$ is a name application; 
Processes $\bout{a}{V} P$ and 
$\binp{a}{x} P$ denote output and input at~$a$;   
abstractions and applications are denoted
$\lambda x.P$ and $(\lambda x.P)a$, respectively;
$\newsp{s}P$ and $\inact$ represent hiding and inaction, respectively.
%Intuitively, the output of a name $b$ along name $a$ is encoded by
%the output of an abstraction containing $b$; the input of a name is encoded 
%by the input of an abstraction
Thus, following a communication on $a$, %our encoding features 
a (deterministic) reduction between  
$s$ and $\dual{s}$ which guarantees that name $b$ is properly unpacked by means of abstraction passing
and appropriate applications.



Concerning (b),
the encoding a recursive process  is \NY{also} challenging for 
%the linearity of endpoints in $P$ must be preserved.
%The encoding of a recursive process $\recp{X}{P}$  is delicate, for it 
it must preserve the linearity of session endpoints. 
Given $\recp{X}{P}$, 
we encode the recursion body $P$ as a name abstraction
in which free names of $P$ are converted into name variables.
%The encoding keeps track of these free names.
The resulting higher-order value is embedded in an input-guarded 
``duplicator'' process~\cite{ThomsenB:plachoasgcfhop}.
Each occurrence of the recursion variable $X$ is encoded 
in such a way that it
simulates recursion unfolding by 
invoking the duplicator in a by-need fashion.
That is, upon reception, the abstraction representing the 
recursion body $P$
is duplicated: 
one copy is used to reconstitute the original recursion body $P$ (through
the application of the free names of $P$); 
another copy is used to re-invoke the duplicator when needed. 
Interestingly, for this encoding strategy to work 
we require non-tail recursive session types; to this end, 
we apply recent advances on the theory of session duality~\cite{TGC14,DBLP:journals/corr/abs-1202-2086}.

%To this end, we
%first record a mapping from recursive variable $X$ to process variable $z_X$.
%Then, we encode the recursion body $P$ as a name abstraction
%in which free names of $P$ are converted into name variables, using \defref{d:auxmap}.
%(Notice that $P$ is first encoded into \HO and then transformed using mapping
%$\auxmapp{\cdot}{{}}{\sigma}$.)
%Subsequently, this higher-order value is embedded in an input-guarded 
%``duplicator'' process~\cite{ThomsenB:plachoasgcfhop}. Finally, we define the encoding of $X$ 
%in such a way that it
%simulates recursion unfolding by 
%invoking the duplicator in a by-need fashion.
%That is, upon reception, the \HO abstraction which encodes  the 
%recursion body $P$
%%containing $\auxmapp{P}{{}}{\sigma}$ 
%is duplicated: 
%one copy is used to reconstitute the original recursion body $P$ (through
%the application of $\fn{P}$); another copy is used to re-invoke
%the duplicator when needed. 
%
%We encode recursion with non-tail recursive session types; for this 
%we apply recent advances on the theory of session duality~\cite{TGC14,DBLP:journals/corr/abs-1202-2086}.
%
%
%
%TO EXPAND.
%\dk{Encoding of a general recursive process $\recp{X}{P}$ is another challenge, since 
%it must preserve the linearity of session endpoints appeared in $P$.
%We encode it with non-tail recursive session types,  
%for which we apply a recent advance on a session duality theory 
%\cite{BernardiH14,TGC14,DBLP:journals/corr/abs-1202-2086}.}

Additional results concern: (i)~the encodability of \HO into \sessp (\secref{subsec:HOp_to_sessp}); (ii)~a non encodability result showing that shared names strictly add expressive power to session calculi (\secref{ss:negative});
(iii)~extensions of our encodability results (\secref{sec:extension}). 
In essence, (i) extends known  results for untyped processes~\cite{SangiorgiD:expmpa} to the session typed setting.
Although (ii) may be somewhat expected, to our knowledge we are the first to offer a formal proof of this separation result, 
which exploits session determinacy and typed equivalences.
Finally, concerning (iii), we develop extensions of our encodings to 
\begin{enumerate}[-]
\item The extension of \HOp with \emph{higher-order} abstractions (\HOpp); 
\item The extension of \HOp with polyadic name passing and abstraction (\pHOp); 
\item The super-calculus of \HOpp and \pHOp (\PHOpp), equivalent to the calculus in~\cite{tlca07}).
\end{enumerate}
\figref{fig:express} summarises %our expressivity 
these encodability results.


\jparagraph{Precise Encodings}
We close this section by discussing the notion of encoding that we use in all the 
encodability results described above.
Building upon established notions for (untyped) processes (e.g.,~\cite{DBLP:journals/iandc/Gorla10}), 
we define a notion of \emph{precise encoding} (\secref{s:expr}) that 
requires the translation of both process and types, and 
admits only process mappings that preserve session types
\emph{and} are fully abstract. Thus, our encodings 
exhibit rather strong behavioral correspondences, and 
only relate source and target processes with  
proper communication structures (as described by session types).
%Moreover, the notion of encoding includes full abstraction as encodability criteria.
We argue that these requirements make our endeavor more challenging.
In particular, requiring type preservation rules out encoding strategies only plausible for untyped processes.
To illustrate this point,
consider the  following encoding of %$\sessp$ 
name-passing 
into $\HO$:\footnote{This alternative  encoding was suggested by an anonymous reviewer of a previous version of this paper.} %defined as
\begin{center}
\begin{tabular}{rcll}
  $\umap{\bout{a}{b} P}$	&$=$&	$\binp{a}{x}( \appl{x}{b} \Par \umap{P})$ \\
  $\umap{\binp{a}{x} Q}$	&$=$&	$\bout{a}{\abs{x}{\umap{Q}}} \inact$
\end{tabular}
\end{center}
%and as a homomorphism for the other operators.
Intuitively, 
rather than sending a package with name $b$, 
this encoding sends the continuation of the input. This entails  a 
``role inversion'': outputs are translated into inputs, and inputs are translated into outputs. 
Although perfectly reasonable in an  {untyped setting}, the encoding $\umap{\cdot}$  
%is far from desirable in a session typed setting: 
is not type preserving: 
since 
individual  prefixes represent actions in a structured communication sequence (abstracted by session types)
the role inversion defined by $\umap{\cdot}$ would alter the meaning of session protocols in the source language.

\begin{figure}[t]
\centering
\includegraphics[scale=1]{diag.pdf}
\vspace{1mm}
%%ADD~FIGURE!
%	\centering
%	\begin{subfigure}[b]{0.45\linewidth}
%		\centering
%		\begin{tikzpicture}
%
%			\node	(PHOpp)	at	(0, 0.8)		{\PHOpp};
%			\node	(HOpp)	at	(-1.5, 0.4)	{\HOpp};
%			\node	(PHOp)	at	(1.5, 0.4)	{\PHOp};
%			\node	(HOp)	at	(0, 0)		{$\mathsf{HO}{\pi}^{~}$};
%			\node	(HO)	at	(-1.5, -0.4)	{$\mathsf{HO}{~}^{~}$};
%			\node	(sessp)	at	(1.5, -0.4)	{$~~\pi^{~}$};
%
%			\draw[<-]	(PHOpp) -- (HOpp);	%(0, 1.25) -- (-2, 1);
%			\draw[<-]	(PHOpp) -- (PHOp);	%(0, 1.25) -- (2, 1);
%
%			\draw[<-]	(HOpp) -- (HOp);	%(2, 0.5) -- (0, 0.25);
%			\draw[<-]	(PHOp) -- (HOp);	%(-2, 0.5) -- (0, 0.25);
%
%			\draw[<-]	(HOp) -- (HO);		%(0, -0.25) -- (-2, -0.5);
%			\draw[<-]	(HOp) -- (sessp);	%(0, -0.25) -- (2, -0.5);
%
%%				\node	(PHOpp)	at	(0, 0.8)		{\PHOpp};
%%				\node	(HOpp)	at	(-2, 0.4)	{\HOpp};
%%				\node	(PHOp)	at	(2, 0.4)	{\PHOp};
%%				\node	(HOp)	at	(0, 0)		{$\mathsf{HO}{\pi}^{~}$};
%%				\node	(HO)	at	(-2, -0.4)	{$\mathsf{HO}{~}^{~}$};
%%				\node	(sessp)	at	(2, -0.4)	{$~~\pi^{~}$};
%%
%%				\draw[->]	(PHOpp) -- (HOpp);	%(0, 1.25) -- (-2, 1);
%%				\draw[->]	(PHOpp) -- (PHOp);	%(0, 1.25) -- (2, 1);
%%
%%				\draw[->]	(HOpp) -- (HOp);	%(2, 0.5) -- (0, 0.25);
%%				\draw[->]	(PHOp) -- (HOp);	%(-2, 0.5) -- (0, 0.25);
%%
%%				\draw[->]	(HOp) -- (HO);		%(0, -0.25) -- (-2, -0.5);
%%				\draw[->]	(HOp) -- (sessp);	%(0, -0.25) -- (2, -0.5);
%		\end{tikzpicture}
%	\end{subfigure}
%%	\hfill
%	\begin{subfigure}[b]{0.5\linewidth}
%		\small
%		The arrow start calculus wrt arrow tip calculus:
%		\begin{itemize}
%			\item	is a sub-calculus.
%			\item	encodes.
%		\end{itemize}
%		Arrow properties are transitive.
%	\end{subfigure}
%\\
	\caption{Encodability in Higher-Order Sessions. 
	Precise encodings are defined in \defref{def:goodenc}.
	\label{fig:express}}
\Hlinefig
\end{figure}

\smallskip

\jparagraph{Outline} This paper 
is structured as follows:
\begin{enumerate}[$\bullet$]
%\item \secref{sec:overview} overviews key ideas of our tractable bisimulations.
\item \secref{sec:calculus} presents the higher-order session calculus \HOp and its 
subcalculi \HO and \sessp.  Then, \secref{sec:types} gives the session type system
and states type soundness for \HOp and its variants.
%\item \secref{sec:behavioural} 
%develops \emph{higher-order} and \emph{characteristic} bisimilarities, our two
%tractable characterisations of contextual equivalence which 
%alleviate the issues of context bisimilarity~\cite{San96H}. These 
%relations are shown to coincide in \HOp (\thmref{the:coincidence}).
\item \secref{s:expr} defines \emph{precise (typed) encodings} by extending encodability criteria 
studied for
untyped processes~(e.g.~\cite{DBLP:journals/iandc/Gorla10,DBLP:conf/icalp/LanesePSS10}).
\item \secref{sec:positive} %and \S\,\ref{sec:negative}
gives encodings of \HOp into \HO and of \HOp into~\sessp.
These encodings 
are shown to be \emph{precise} (Thms.~\ref{f:enc:hopitoho} and~\ref{f:enc:hotopi}).
Mutual encodings between \sessp and \HO are derivable; 
all these calculi are thus equally expressive.
%Exploiting determinacy and typed equivalences,
We also prove the non-encodability of shared names
into linear names (\thmref{t:negative}).

\item \secref{sec:extension} studies extensions of \HOp. We show that 
both \HOpp (the extension with higher-order applications) 
and \pHOp (the extension with polyadicity) are encodable in \HOp
(Thms.~\ref{f:enc:hopiptohopi} and \ref{f:enc:phopiptohopi}).
This connects our work 
to the existing
higher-order session calculus in~\cite{tlca07} (here denoted  $\PHOpp$).

\item \secref{sec:relwork} concludes with related works. The appendix summarises the typing system. 
\end{enumerate}
\noi
The paper is self-contained. 
{\bf\em Additional related work, more examples, omitted definitions, and  proofs 
%can be found
are 
in~\cite{KouzapasPY15}.} 


% !TEX root = main.tex

%This paper is about \emph{relative expressiveness} results for 
%\emph{higher-order process calculi}, core programming languages that 
%integrate name- and process-passing in communications.
%We focus on calculi coupled with \emph{session types} that denote interaction protocols. 
%Expressiveness results allows us to 
%identify
%a \emph{core %process language % concurrency %with session primitives, 
%calculus}
%that encompasses both first- and higher-order session communication.
%Establishing such results 
%in our typed setting 
%is challenging because 
% it entails defining 
% not only a translation 
%relating source and target languages (\emph{encoding}), but also a translation 
%relating their associated session types. 
%We aim at a very particular class of correct encodings: namely \emph{fully abstract} and \emph{type-preserving} encodings.
%Next, we elaborate on our aims,   approach, and contributions.

\emph{Type-preserving compilations} are important in the design of
functional and object-oriented languages: type information has been
used to, e.g., justify code optimizations and reason about programs
(see, e.g.,
\cite{DBLP:journals/toplas/MorrisettWCG99,DBLP:conf/pldi/ShaoA95,DBLP:journals/toplas/LeagueST02}).
A vast literature on 
{\em expressiveness} 
in concurrency theory  
(e.g.,~\cite{Palamidessi03,DBLP:journals/iandc/Gorla10,DBLP:journals/tcs/FuL10,DBLP:conf/icalp/LanesePSS10,DBLP:journals/corr/PetersG15})
also studies compilations (or \emph{encodings}):
they are used to transfer reasoning techniques 
from one calculus to another, and to identify 
constructs which may be implemented
using simpler ones. 
%To a large extent, however, this kind of \emph{expressiveness studies} concern only \emph{untyped process languages}.
In this work, we study 
{\em relative expressiveness} 
via \emph{type-preserving encodings} for \HOp, a \emph{higher-order} 
process language that integrates message-passing concurrency with functional features.
We consider source and target calculi coupled with \emph{session types} denoting interaction protocols. 
Building upon untyped frameworks for relative expressiveness
\cite{DBLP:journals/iandc/Gorla10}, 
we propose type preservation as a {new criteria} for \emph{precise encodings}.
We identify \HO, a new core calculus for higher-order session concurrency without
name passing. 
We show that \HO can encode \HOp precisely and efficiently. 
Requiring  
type preservation makes
this encoding far from trivial: our encoding crucially exploits advances on
session type duality~\cite{TGC14,DBLP:journals/corr/abs-1202-2086} and recent
characterisations of typed contextual equivalence \cite{characteristic_bis}.
We develop a full hierarchy of variants of \HOp based on 
precise encodings (see \figref{fig:express}):
our encodings are
type-preserving and fully abstract, up to typed
behavioural equalities. 

\begin{figure}[t]
\centering
\includegraphics[scale=1]{diag.pdf}

	\caption{Encodability in Higher-Order Sessions. 
	Precise encodings are defined in \defref{def:goodenc}.
	\label{fig:express}}
\vspace{-5mm}
\Hlinefig
\end{figure}

\jparagraph{Context}
In \emph{session-based concurrency}, interactions are organised into \emph{sessions}, basic communication units.
Interaction patterns can then be abstracted as expressive \emph{session types}~\cite{honda.vasconcelos.kubo:language-primitives}, against which  specifications may be checked. 
%These patterns are defined as %(possibly recursive) 
%sequences of communication actions: % (send/receive a value, offer/select a behavior).
%For instance, 
%session type $T_1 = \btinp{\mathsf{str}} \btout{\mathsf{int}}  \tinact$ may be intuitively read as: receive (?) a value of type $\mathsf{str}$,then output (!) a value of type $\mathsf{int}$, finally close the protocol.
Session type $\btinp{U} S$ (resp.  $\btout{U} S$)
describes a protocol that first receives (resp. sends) a value of type $U$ and then continues as protocol $S$.
Also, given an index set $I$, types $\btbra{l_i:S_i}_{i \in I}$ 
and $\btsel{l_i:S_i}_{i \in I}$ 
define %, respectively,
%a branching and selection constructs for  
 a labeled choice mechanism; types 
$\trec{t}{S}$ 
and 
$\tinact$ denote recursive and completed protocols, respectively.
%describes a protocol that offers
%(resp. ) 
%Type $\tinact$ denotes the completed protocol.
In the (first-order) $\pi$-calculus~\cite{MilnerR:calmp1}, 
session types describe the intended interactive behaviour of the names/channels in a process.
%names/channels are endowed with session types (such as $T_1$) representing their intended interactive behavior.

Session-based concurrency has also been casted in {higher-order} process
calculi which, by combining features from the $\lambda$-calculus and the $\pi$-calculus, 
enable the exchange of values 
that may contain processes~\cite{tlca07,DBLP:journals/jfp/GayV10}. 
%Higher-order calculi with sessions 
%naturally bridges concurrent and functional computation, 
%and enable the specification of protocols involving \emph{code mobility}, 
%commonplace in practice.
%The \HOp calculus enables 
%the specification of protocols involving \emph{code mobility}, 
%and includes
%Higher-order calculi with sessions 
The higher-order calculus with sessions studied here, denoted \HOp,
can specify protocols involving \emph{code mobility}: it includes
%equiped ping with 
constructs for 
synchronisation along shared names, 
session communication (value passing, labelled choice) along linear names,
recursion, 
 (first-order) abstractions 
 and applications.
 That is, 
 values in communications include names but also (first-order) abstractions---functions from name identifiers to processes. 
 %(In contrast, higher-order abstractions---functions from processes to processes---are disallowed.)
 (In contrast, we rule out higher-order abstractions---functions from processes to processes.)
Abstractions can be linear or shared; their types are  denoted $\lhot{C}$ and $\shot{C}$, respectively ($C$ 
%is a first-order type $C$ (say, a session name).
denotes a name). In \HOp we may have processes with a 
session type such as, e.g.,
%$T_2 = \btbra{upload:\btinp{\lhot{\mathsf{int}}}\tinact ~ , ~ sha:\btinp{\shot{\mathsf{int}}}\tinact}_{}$
$$S = \btbra{up:\btinp{\lhot{C}}\btout{\mathsf{ok}}\tinact ~ , ~ down:\btout{\shot{C}}\btout{\mathsf{ok}}\tinact ~ , ~quit:\btout{\mathsf{bye}}\tinact}_{}$$
that abstracts a server that offers different behaviours to clients: 
%  clients to select among distinct  behaviors: %namely, 
  to \emph{upload} a linear function, % (to be received by the server), 
  to \emph{download} a shared function, % (to be sent by the server),
   or to \emph{quit} the protocol. Subsequently, 
  the server sends a message ($\mathsf{ok}$ or $\mathsf{bye}$) before closing the session.


%\jparagraph{The Problem}
%%Roughly speaking, 
%  \HOp %, a higher-order process language that 
%extends Sangiorgi's higher-order $\pi$-calculus~\cite{SangiorgiD:expmpa} with session primitives.
%To be precise, %More precisely, 
%\HOp
%includes
%constructs for 
%%session establishment
%synchronisation along shared names, 
%session communication (value passing, labelled choice) along linear names,
%recursion, 
% (first-order) abstractions %(i.e., functions from name identifiers  to processes)
% and applications.
%% (denoted $\lambda x.P$ and $(\lambda x.P)a$, resp.).
%%While synchronization on shared names (useful to model session establishment) is 
%%non deterministic, session communication is deterministic and occurs on linear names.
%\HOp is therefore a rather rich language. This begs the question:
%%\begin{quote}
%is there a \emph{sub-calculus} of \HOp with equal expressivity? %hich is as expressive as the whole calculus? 
%%\end{quote}
%This question is of foundational interest, 
%for reasoning/validation techniques are more easily developed on small formalisms. 
%It also has practical ramifications, 
%as such a \emph{core calculus} could be taken as reference in 
%the design of %(functional) 
%programming languages with session types support.
%%implementations of languages with session primitives.
%Expressivity results may then help justifying useful connections 
%between foundational and practical advances on languages with concurrency and communication.
%
%%We have recently developed a behavioral theory  for \HOp~\cite{characteristic_bis}:
%%we introduced
%%\emph{characteristic bisimilarity}, a sound and complete 
%%characterization of contextual equivalence. % that enables tractable analyses.


\jparagraph{Expressiveness of \HOp}
%In this paper 
We study the type-preserving, 
relative expressivity of \HOp. % in relation. 
%to two 
%sub-calculi
%that distill first- and higher-order session-based concurrency. 
%\begin{enumerate}[-]
%\item 
As expected from 
known literature in the untyped setting \cite{SangiorgiD:expmpa}, 
the first-order session \sessp-calculus~\cite{honda.vasconcelos.kubo:language-primitives} {(here denoted~\sessp)} 
can encode  
\HOp preserving session types. 
%(\HOp without
%abstractions and applications) 
%\item 
In this paper, 
our \emph{main discovery} is 
that 
\HOp 
without
name-passing and recursion
can serve as a new core calculus    
for higher-order session concurrency.  
We call this core calculus \HO. 
We show that \HO can encode \HOp more efficiently 
than \sessp. In addition, in the higher-order session typed setting, 
\HO offers more tractable bisimulation techniques 
than \sessp (cf. \secref{ss:equiv})
%constitute 
%the main sources 
%of expressivity in \HOp. 
%: \emph{name passing} and constructs for \emph{infinite behavior} (i.e., recursion and replication). 
%On the one hand, t
%Indeed, the expressivity of name-passing calculi (untyped/typed) is well known; e.g., the $\pi$-calculus can express 
%the $\lambda$-calculus and 
%process-passing calculi~\cite{SangiorgiD:expmpa}. 
%In the $\pi$-calculus, recursion and replication can be expressed in terms of each other. 
%On the other hand, 
%Higher-order concurrency is quite expressive too: 
%calculi without name passing and recursion are Turing equivalent~\cite{DBLP:journals/iandc/LanesePSS11}.
%Also, 
%recursion/replication operators are redundant in higher-order calculi: they can be represented using process passing and duplication~\cite{ThomsenB:plachoasgcfhop}. 

%\figref{fig:express} summarises %our expressivity 
%our encodability results. 


%While encoding \HOp 
%into the $\pi$-calculus preserving session types 
%(extending  known  results for untyped processes~\cite{SangiorgiD:expmpa}) is 
%%\jpc{already}
%significant, 



\jparagraph{Challenges and Contributions}

We assess the expressivity  of \HOp, \HO, and \sessp as delineated by session types. 
We introduce \emph{type-preserving encodings}:
we use type information to define encodings
and to retain the semantics of session protocols. 
Indeed,  not only we require 
well-typed source processes are encoded into 
well-typed target processes: 
we demand that session type constructs (input, output, branching, select) used to type the source process
are preserved by the typing of the target process.
This criterion is included in 
our notion of \emph{precise encoding} (\defref{def:goodenc}), which 
extends encodability criteria for untyped processes with 
\emph{full abstraction}.
{Full abstraction results are stated
up to two
behavioural equalities that characterise barbed congruence:
\emph{characteristic bisimilarity} ($\fwb$, defined in~\cite{characteristic_bis})
and 
\emph{higher-order bisimilarity} ($\hwb$), introduced in this
work.
It turns out that $\hwb$ offers more direct  reasoning than $\fwb$. }
Using precise encodings we establish strong correspondences between 
\HOp and its variants---see \figref{fig:express}. 



Our main contribution is 
an encoding of \HOp into \HO (\secref{subsec:HOpi_to_HO}).  
Since \HO lacks 
both name-passing and recursion, this encoding involves two \emph{key challenges}:
\begin{enumerate}[a.]
\item In known (typed) 
encodings of name-passing into process-passing~\cite{SaWabook} %are limited: % in that 
%they come with restrictions on name usages;  
%they 
%work for %name-passing 
%calculi 
%with \emph{capability types} 
%in which 
only the output capability of names can be sent---a received name cannot be used in later inputs.
This is far too limiting in \HOp, where 
 session names %denoting arbitrary protocols 
 may be passed around (\emph{delegation})
and types describe interaction  \emph{structures}, rather than ``loose'' name capabilities. % at a given time.



\item %As mentioned above, recursion % and replication)
%can be encoded in untyped higher-order calculi using process duplication. Unfortunately, this kind of encodings 
Known encodings of recursion in untyped higher-order calculi
do not carry over to session typed calculi such as \HOp,
because linear abstractions cannot be copied/duplicated. Hence, the discipline of session types  limits 
the possibilities for representing infinite behaviours---even simple forms, such as input-guarded replication.
\end{enumerate}




%MOTIVATION FIRST ENCODING (). \emph{Still to highlight: recursive type required, no recursion, small example.

%--- 
\noi
%We illustrate our approach. % to these challenges.
Our encoding overcomes these two obstacles, as we discuss in the following section.

Additional technical contributions include: 
(i)~the encodability of \HO into \sessp (\secref{subsec:HOp_to_sessp}); 
(ii)~extensions of our encodability results to richer settings (\secref{sec:extension});
(iii)~a non encodability result showing that shared names strictly add expressive power to session calculi (\secref{ss:negative}).
In essence, (i) extends known  results for untyped processes~\cite{SangiorgiD:expmpa} to the session typed setting.
Concerning (ii), we develop extensions of our encodings to 
\begin{enumerate}[-]
\item The extension of \HOp with \emph{higher-order} abstractions (\HOpp); 
\item The extension of \HOp with polyadic name passing and abstraction (\PHOp); 
\item The super-calculus of \HOpp and \PHOp (\PHOpp), equivalent to the calculus in~\cite{tlca07}.
\end{enumerate}
%\figref{fig:express} summarises %our expressivity 
%our encodability results. 
%From a global standpoint, our 
These
encodability results connect \HOp with existing higher-order process calculi~\cite{tlca07}, and  
further highlight the status of \HO as the core calculus for session concurrency.
Finally, although (iii) may be somewhat expected, to our knowledge we are the first to prove this separation result, 
exploiting session determinacy and typed equivalences.




\jparagraph{Outline} 
%This paper  is structured as follows.
%\begin{enumerate}[$\bullet$]
\secref{sec:overview} overviews key ideas of the precise encoding of \HOp into \sessp.
%\item 
\secref{sec:calculus} presents \HOp and its 
subcalculi (\HO and \sessp); %, and extensions (\HOpp and \PHOp).  
\secref{sec:types} summarises their session type system.
\secref{sec:bt}~pres\-ents  behavioural equalities for \HOp:
we recall definitions of barbed congruence and characteristic bisimilarity~\cite{characteristic_bis}, 
and introduce higher-order bisimilarity.
We show that these three typed relations coincide (\thmref{t:coincide}).
%and states type soundness 
%for \HOp and its variants.
\secref{s:expr} defines \emph{precise %(typed) 
encodings} by extending encodability criteria  for untyped processes. %~(e.g.,~\cite{DBLP:journals/iandc/Gorla10}).
%\item 
\secref{sec:positive} %and \S\,\ref{sec:negative}
gives {precise encodings} of \HOp into \HO and of \HOp into~\sessp (Thms.~\ref{f:enc:hopitoho} and~\ref{f:enc:hotopi}).
Mutual encodings between \sessp and \HO are derivable; 
all these calculi are thus equally expressive.
By means of empirical and formal comparisons between these two precise encodings, in \secref{ss:compare} we establish that
\HOp and \HO are more tightly related than \HOp and \sessp (\thmref{t:tight}).
Moreover, we prove the impossibility of encoding communication along shared names
using linear names (\thmref{t:negative}).
%Exploiting determinacy and typed equivalences,
%\item
In \secref{sec:extension} %studies extensions of \HOp: 
we show that both \HOpp 
%(the extension with higher-order applications) 
and \PHOp 
%(the extension with polyadicity) 
are encodable in \HOp
(Thms.~\ref{f:enc:hopiptohopi} and \ref{f:enc:phopiptohopi}).
%This connects our work to the existing higher-order session calculus in~\cite{tlca07} (here denoted  $\PHOpp$).
%\item 
\secref{sec:relwork} collects concluding remarks and reviews related works.
%\secref{sec:concl} concludes.
The paper is self-contained. {\bf\em Omitted definitions and  proofs are in the Appendix and in~\cite{KouzapasPY15}.} 



\section{Overview: Encoding Name Passing Into Process Passing} %: Tractable Higher-Order Session Bisimulations}
\label{sec:overview}
% !TEX root = main.tex

\jparagraph{A Precise Encoding of Name-Passing into Process-Passing}
As mentioned above, 
our encoding of \HOp into \HO (\secref{subsec:HOpi_to_HO}) should overcome two key challenges.
First, it should enable the communication of arbitrary names, as required to represent delegation.
Second, it should address the fact that linearity of session types limits the 
possibilities for representing infinite behavior. 
To encode name passing into \HO 
%to encode name output, 
we ``pack''
the name to be sent into a suitable abstraction; 
upon reception, the receiver must ``unpack'' this object following a precise protocol on a fresh  session:
%More precisely, our encoding \jpc{of name passing} in \HO is given as:
\begin{center}
\begin{tabular}{rcll}
  $\map{\bout{a}{b} P}$	&$=$&	$\bout{a}{ \abs{z}{\,\binp{z}{x} (\appl{x}{b})} } \map{P}$ \\
  $\map{\binp{a}{x} Q}$	&$=$&	$\binp{a}{y} \newsp{s}{\appl{y}{s} \Par \bout{\dual{s}}{\abs{x}{\map{Q}}} \inact}$
\end{tabular}
\end{center}
%and as a homomorphism for the other operators.
Above, 
%where
$a,b$ are names and $s$ and $\dual{s}$ are 
linear session names (\emph{endpoints}).
%$\lambda x.P$ is a name abstraction of $P$; $\appl{x}{a}$ is a name application; 
Processes $\bout{a}{V} P$ and 
$\binp{a}{x} P$ denote output and input at~$a$;   
abstractions and applications are denoted
$\lambda x.P$ and $(\lambda x.P)a$; %, respectively;
$\newsp{s}P$ and $\inact$ represent hiding and inaction. %, respectively.
%Intuitively, the output of a name $b$ along name $a$ is encoded by
%the output of an abstraction containing $b$; the input of a name is encoded 
%by the input of an abstraction
Thus, following a communication on $a$, %our encoding features 
a (deterministic) reduction between  
$s$ and $\dual{s}$ guarantees that name $b$ is properly unpacked by means of abstraction passing
and appropriate applications.
It is worth stressing how an output action in the source process is translated into an output action in the encoded process (and similarly for input).
This correspondence is key to ensure the preservation of session type operators mentioned above.

To preserve session linearity, we proceed as follows.
Given $\recp{X}{P}$, 
we encode the recursion body $P$ as an abstraction
in which free names of $P$ are converted into name variables.
%The encoding keeps track of these free names.
The resulting higher-order value is embedded in an input-guarded 
``duplicator'' process~\cite{ThomsenB:plachoasgcfhop}.
The recursion variable $X$ is then encoded 
in such a way that it
simulates recursion unfolding by 
invoking the duplicator in a by-need fashion.
That is, upon reception, the abstraction representing the 
recursion body $P$
is duplicated: 
one copy is used to reconstitute the original recursion body $P$ (through
the application of the free names of $P$); 
another copy is used to re-invoke the duplicator when needed. 
Interestingly, for this encoding to work 
we require non-tail recursive session types; to this end, 
we apply recent advances on the theory of duality for session types~\cite{TGC14,DBLP:journals/corr/abs-1202-2086}.

%To this end, we
%first record a mapping from recursive variable $X$ to process variable $z_X$.
%Then, we encode the recursion body $P$ as a name abstraction
%in which free names of $P$ are converted into name variables, using \defref{d:auxmap}.
%(Notice that $P$ is first encoded into \HO and then transformed using mapping
%$\auxmapp{\cdot}{{}}{\sigma}$.)
%Subsequently, this higher-order value is embedded in an input-guarded 
%``duplicator'' process~\cite{ThomsenB:plachoasgcfhop}. Finally, we define the encoding of $X$ 
%in such a way that it
%simulates recursion unfolding by 
%invoking the duplicator in a by-need fashion.
%That is, upon reception, the \HO abstraction which encodes  the 
%recursion body $P$
%%containing $\auxmapp{P}{{}}{\sigma}$ 
%is duplicated: 
%one copy is used to reconstitute the original recursion body $P$ (through
%the application of $\fn{P}$); another copy is used to re-invoke
%the duplicator when needed. 
%
%We encode recursion with non-tail recursive session types; for this 
%we apply recent advances on the theory of session duality~\cite{TGC14,DBLP:journals/corr/abs-1202-2086}.

\jparagraph{A Plausible Encoding That is Not Precise}
As motivated earlier,
we define a notion of \emph{precise encoding} (\secref{s:expr}) that
requires the translation of both process and types, and 
admits only process mappings that preserve session types
\emph{and} are fully abstract. Thus, our encodings 
not only exhibit   strong behavioral correspondences, but also 
 relate source and target processes with  
communication structures described by session types.
%Moreover, the notion of encoding includes full abstraction as encodability criteria.
These strict requirements make our developments more challenging.
In particular, requiring type preservation rules out other plausible encoding strategies.
To illustrate this point,
consider the  following encoding of %$\sessp$ 
name-passing 
into $\HO$:\footnote{This alternative  encoding was suggested by an anonymous reviewer of a previous version of this paper.} %defined as
\begin{center}
\begin{tabular}{rcll}
  $\umap{\bout{a}{b} P}$	&$=$&	$\binp{a}{x}( \appl{x}{b} \Par \umap{P})$ \\
  $\umap{\binp{a}{x} Q}$	&$=$&	$\bout{a}{\abs{x}{\umap{Q}}} \inact$
\end{tabular}
\end{center}
%and as a homomorphism for the other operators.
Intuitively, 
rather than sending a package with name $b$, 
this encoding sends the continuation of the input. Observe how this mapping entails  a 
``role inversion'': outputs are translated into inputs, and inputs are translated into outputs. 
Although perfectly reasonable, the encoding $\umap{\cdot}$  
%is far from desirable in a session typed setting: 
is \emph{not type preserving}. Consequently, it is also not \emph{precise}.
%Type preservation is intended to preserve the overall semantics of session types:
Since individual prefixes (input, output, branching, select) 
represent actions in a structured communication sequence (i.e., a protocol abstracted by a session type),
the encoding above would simply alter the meaning of the session protocol in the source language.






% !TEX root = main.tex
\section{Higher-Order Session $\pi$-Calculi}
\label{sec:calculus}

\noindent

We use the the 
\emph{Higher-Order Session $\pi$-Calculus} (\HOp)
which is a variant of the calculus originally introduced
in~\cite{characteristic_bis}.
\HOp includes both name- and abstraction-passing 
as well as recursion; it is a subcalculus of the language studied 
in~\cite{tlca07}.
The difference between \HOp and the calculus presented
in~\cite{characteristic_bis} is the fact that \HOp does
not allow higher-order value applications. We describe
the syntax in brief.
%Following the literature~\cite{JeffreyR05},
%for simplicity of the presentation
%we concentrate on the second-order call-by-value \HOp.  
(In \secref{sec:extension} we consider extensions of 
\HOp with higher-order abstractions 
and polyadicity in name-passing/abstractions.)
%We also introduce two subcalculi of \HOp. In particular, we define the 
%core higher-order session
%calculus (\HO), which 
%%. The \HO calculus is  minimal: it 
%includes constructs for shared name synchronisation and 
%%constructs for session establish\-ment/communication and 
%(monadic) name-abstraction, but lacks name-passing and recursion.

%Although minimal, in \secref{s:expr}
%the abstraction-passing capabilities of \HOp will prove 
%expressive enough to capture key features of session communication, 
%such as delegation and recursion.

\subsection{Syntax of \HOp}
\label{subsec:syntax}

\noindent
The syntax of \HOp is defined in \figref{fig:syntax}.

	\begin{figure}[t]
	\[ 
		\begin{array}{rcl}
			u,w  &\bnfis& n \bnfbar x,y,z
			\qquad \qquad
			n \bnfis a,b \bnfbar s, \dual{s} 
			\qquad \qquad
			V,W \bnfis \nonhosyntax{u} \bnfbar \abs{x}{P}
			\\[1mm]

			P,Q
			& \bnfis &
			\bout{u}{V}{P}  \bnfbar  \binp{u}{x}{P} \bnfbar
			\bsel{u}{l} P \bnfbar \bbra{u}{l_i:P_i}_{i \in I} \bnfbar \appl{V}{u}\bnfbar P\Par Q \bnfbar \news{n} P 
			\bnfbar \inact
			\\%[1mm]
			& \bnfbar &
			\nonhosyntax{\rvar{X} \bnfbar \recp{X}{P}}
		\end{array}
	\]
	\caption{Syntax of \HOp (\HO lacks the constructs in \nonhosyntax{\text{grey}}).}
	\label{fig:syntax}
%	\Hlinefig
\end{figure}



\myparagraph{Values}
Names $a,b,c, \dots$ (resp.~$s, \dual{s}, \dots$) 
range over shared (resp. session) names. 
Names $m, n, t, \dots$ are session or shared names.
Dual endpoints are $\dual{n}$ with
$\dual{\dual{s}} = s$ and $\dual{a} = a$.
%We define the dual operation over names $n$ as $\dual{n}$ with
%$\dual{\dual{s}} = s$ and $\dual{a} = a$.
%Intuitively, names $s$ and $\dual{s}$ are dual (two) \emph{endpoints} while 
%shared names represent shared (non-deterministic) points. 
Variables are denoted with $x, y, z, \dots$, 
and recursive variables are denoted with $\varp{X}, \varp{Y} \dots$.
An abstraction %(or higher-order value) 
$\abs{x}{P}$ is a process $P$ with name parameter $x$.
%Symbols $u, v, \dots$ range over identifiers; and  $V, W, \dots$ to denote values. 
Values $V,W$ include 
identifiers $u, v, \ldots$ %(first-order values) 
and 
abstractions $\abs{x}{P}$ (first- and higher-order values, resp.). 

\myparagraph{Terms} 
include the
$\pi$-calculus prefixes for sending and receiving values $V$.
%Process $\bout{u}{V} P$ denotes the output of value $V$
%over name $u$, with continuation $P$;
%process $\binp{u}{x} P$ denotes the input prefix on name $u$ of a value
%that 
%will substitute variable $x$ in continuation $P$. 
Recursion is expressed by $\recp{X}{P}$,
which binds the recursive variable $\varp{X}$ in process $P$.
Process 
%ny
%$\appl{x}{u}$ 
$\appl{V}{u}$ 
is the application
which substitutes name $u$ on the abstraction~$V$. 
Typing  ensures that $V$ is not a name.
Processes $\bsel{u}{l} P$ and $\bbra{u}{l_i: P_i}_{i \in I}$ are the
standard session processes for selecting and branching.
%Prefix $\bsel{u}{l} P$ selects label $l$ on name $u$ and then behaves as $P$.
%Given $i \in I$ 
%Process $\bbra{u}{l_i: P_i}_{i \in I}$ offers a choice on labels $l_i$ with
%continuation $P_i$, given that $i \in I$.
%Others are standard c
Constructs for 
inaction $\inact$,  parallel composition $P_1 \Par P_2$, and 
name restriction $\news{n} P$ are standard.
Session name restriction $\news{s} P$ simultaneously binds endpoints $s$ and $\dual{s}$ in $P$.
%A well-formed process relies on assumptions for
%guarded recursive processes.
Functions $\fv{P}$ and $\fn{P}$ denotet the sets of free 
%\jpc{recursion}
variables and names; 
and \dk{assume $V$ in $\bout{u}{V}{P}$ does not include free recursive 
variables $\rvar{X}$.}
If $\fv{P} = \emptyset$, we call $P$ {\em closed}.
%; and closed $P$ without 
%free session names a {\em program}. 

\subsection{Subcalculi of \HOp}
\label{subsec:subcalculi}
\noi
We define two subcalculi of \HOp. 
%We now define several sub-calculi of \HOp. 
\begin{enumerate}[$\bullet$]
	\item	The first subcalculus is the 
		{\em core higher-order session calculus} (denoted \HO),
		which lacks recursion and name passing; its 
		formal syntax is obtained from \figref{fig:syntax} by excluding 
		constructs in \nonhosyntax{\text{grey}}.

	\item	The second subcalculus is 
		the {\em session $\pi$-calculus} 
		(denoted $\sessp$), which 
		lacks  the
		higher-order constructs
		(i.e., abstraction passing and application), but includes recursion.

%	\item	The third sub-calculus, denoted \haskp, represents cloud Haskell:
%		\[
%			\begin{array}{rclllll}
%				V,W	& ::= &		u \bnfbar  \abs{x}{P}
%				\\
%				P,Q	& ::= &		\bout{u}{m}{P}  \bnfbar  \binp{u}{x}{P} \bnfbar
%							\bsel{u}{l} P \bnfbar \bbra{u}{l_i:P_i}_{i \in I}
%				\\[1mm]
%					& \bnfbar &	\appl{V}{V} \bnfbar P\Par Q \bnfbar \news{n} P \bnfbar \inact
%		\end{array}
%		\]
\end{enumerate}
%
Let $\CAL \in \{\HOp, \HO, \sessp\}$. We write 
$\CAL^{-\mathsf{sh}}$ for $\CAL$ without shared names
(we delete $a,b$ from $n$). 
We shall demonstrate in Section~\ref{sec:positive} that 
$\HOp$, $\HO$, and $\sessp$ have the same expressivity.


\subsection{Operational Semantics}
\label{subsec:semantics}

\begin{figure}[!t]
\[
	\begin{array}{rcllcrcll}
%	\begin{array}{c}
		\appl{(\abs{x}{P})}{u}  & \red & P \subst{u}{x} 
		& \orule{App}
%		\\[1mm]
		&&
		\bout{n}{V} P \Par \binp{\dual{n}}{x} Q & \red & P \Par Q \subst{V}{x} 
		& \orule{Pass}
		\\[1mm]

		\bsel{n}{l_j} Q \Par \bbra{\dual{n}}{l_i : P_i}_{i \in I} & \red & Q \Par P_j ~~(j \in I)~~ 
		& \orule{Sel}
%		\\[1mm]
		&&
		P \red P'\Rightarrow  \news{n} P   & \red  &  \news{n} P' 
		& \orule{Res}
		\\[1mm]

		P \red P' & \Rightarrow  &  P \Par Q  \red   P' \Par Q  
		& \orule{Par}
%		\\[1mm]
		&&
		P \scong Q \red Q' \scong P' & \Rightarrow & P  \red  P'
		& \orule{Cong}
	\end{array}
\]
{\small
\[
	\begin{array}{c}
		P \Par \inact \scong P
		\quad
		P_1 \Par P_2 \scong P_2 \Par P_1
		\quad
		P_1 \Par (P_2 \Par P_3) \scong (P_1 \Par P_2) \Par P_3
%		\\[1mm]
		\quad
		\news{n} \inact \scong \inact
%		\quad 
		\\[1mm]
		P \Par \news{n} Q \scong \news{n}(P \Par Q)
		\ (n \notin \fn{P})
		\quad 
		\recp{X}{P} \scong P\subst{\recp{X}{P}}{\rvar{X}}
%		\\[1mm]
		\quad
		P \scong Q \textrm{ if } P \scong_\alpha Q
%		\qquad
%		\dk{V \scong W \textrm{ if } V \scong_\alpha W
%\quad \abs{x}{P} \scong \abs{x}{Q} \textrm{ if } P \scong Q}
	\end{array}
\]
}
\caption{Operational Semantics of $\HOp$. 
\label{fig:reduction}}
%\Hlinefig
\end{figure}



\noindent \figref{fig:reduction} (top) defines the operational semantics 
of \HOp.
Rule $\orule{App}$ is the name application; 
Rule $\orule{Pass}$ defines a shared interaction at $n$ 
(with $\dual{n}=n$) or a session interaction;  
Rule $\orule{Sel}$ is the standard rule for labelled choice/selection;%:
%given an index set $I$, 
%a process selects label $l_j$ on name $n$ over a set of
%labels $\set{l_i}_{i \in I}$ offered by a branching 
%on the dual endpoint $\dual{n}$;
and other rules are standard $\pi$-calculus rules.
Rules for \emph{structural congruence} are defined in \figref{fig:reduction} (bottom). 
We assume the expected extension of $\scong$ to values $V$.
We write $\red^\ast$ for a multi-step reduction.

% !TEX root = main.tex
%\newpage
\section{Session Types for $\HOp$}
\label{sec:types}

In this section we define a session typing system for
$\HOp$ and establish its main properties. We use as
a reference the type system for higher-order session processes 
developed by Mostrous and Yoshida~\cite{tlca07,mostrous09sessionbased}.
Our system is simpler than that in~\cite{tlca07}, in order to distil the key
features of higher-order communication in a session-typed setting.

%%%%%%%%%%%%%%%%%%%%%%%%%%%%%%%%%%%%%%%%%%%%%%%%%%
%  SYNTAX FOR TYPES
%%%%%%%%%%%%%%%%%%%%%%%%%%%%%%%%%%%%%%%%%%%%%%%%%%

\subsection{Syntax}

We define the syntax of session types for \HOp.

\begin{definition}[Syntax of Types]\rm
	\label{def:types}
	The syntax of types is defined on the types for sessions $S$,
	and the types for values $U$:
	%
	\[
	\begin{array}{lrcl}
		\textrm{(value)} & U & \bnfis &		C \bnfbar L 
		\\

		\textrm{(names)} & C & \bnfis &	S \bnfbar \chtype{S} \bnfbar \chtype{L}
		\\

		\textrm{(lambda)} & L & \bnfis &	\shot{C} \bnfbar \lhot{C}
		\\

		\textrm{(session)} & S,T & \bnfis & 	\btout{U} S \bnfbar \btinp{U} S
							\bnfbar \btsel{l_i:S_i}_{i \in I} \bnfbar \btbra{l_i:S_i}_{i \in I}\\
					&&\bnfbar&	\trec{t}{S} \bnfbar \vart{t}  \bnfbar \tinact
	\end{array}
	\]
\end{definition}
%
\noi \myparagraph{Types for Values}
Types for values range over symbol $U$ which includes
first-order types $C$ and higher-order types $L$.
First-order types $C$ are used to type names;
session types $S$ type session names and shared types
$\chtype{S}$ or $\chtype{L}$ type shared names that
carry session values and higher-order values, respectively.
Higher-order types $L$ are used to type abstraction values;
$\shot{C}$ and $\lhot{C}$ denote
shared and linear abstraction types, respectively.

\myparagraph{Session Types}
The syntax of session types $S$ follows the usual
(binary) session types with
recursion~\cite{honda.vasconcelos.kubo:language-primitives,GH05}.
{\em Output type} $\btout{U} S$ is assigned to a name that
first sends a value of type $U$ and then follows
the type described by $S$.
Dually, the {\em input type} $\btinp{U} S$ is assigned to a name
that first receives a value of type $U$ and then continues as $S$. 
Session types for labelled choice and selection, 
%are standard: they are 
written $\btbra{l_i:S_i}_{i \in I}$ and $\btsel{l_i:S_i}_{i \in I}$, respectively,
require a set of types $\set{S_i}_{i \in I}$ that correspond to a set of
labels $\set{i \in I}_{i \in I}$. 
{\em Recursive session types} are defined using the primitive recursor.
We that a {\em type variable} is guarded, i.e.~$\trec{t}{\vart{t}}$ is not allowed.
We let $\mathsf{T}$ to be the set of all well-formed types and
Type $\tinact$ is the termination type.
\ST to be the set of all well-formed session types.

\begin{remark}[Restriction on Types for Values]
	The syntax for value types is restricted
	to dissallow types of the form:
	\begin{enumerate}[$\bullet$]
		\item	$\chtype{\chtype{U}}$: shared names
			cannot carry shared names; and

		\item  $\shot{U}$: abstractions do not
			bind higher-order variables.
	\end{enumerate}
\end{remark}

The difference between the syntax of process
in \HOp with the syntax of processes in~\cite{tlca07}
is also reflected on the two corresponding type syntax;
the type structure  in~\cite{tlca07}, 
supports the arrow types of the form $U \sharedop T$ and 
$U \lollipop T$, where $T$ denotes an arbitrary type of a term 
(i.e.~a value or a process).

%%%%%%%%%%%%%%%%%%%%%%%%%%%%%%%%%%%%%%%%%%%%%%%%%%
%  DUALITY
%%%%%%%%%%%%%%%%%%%%%%%%%%%%%%%%%%%%%%%%%%%%%%%%%%

\subsection{Duality}

Duality is defined following the co-inductive
approach following~\cite{GH05,TGC14}.
We first require the notion of type equivalence.
%
\begin{definition}[Type Equivalence]\rm
\label{def:type_equiv}
	Define function $F(\Re): \mathsf{T} \longrightarrow \mathsf{T}$:
	%
	\[
		\begin{array}{rcl}
			F(\Re) 	&=&	\set{(\tinact, \tinact)} \\
				&\cup&	\set{(\chtype{S}, \chtype{T}) \bnfbar S\ \Re\ T} \cup \set{(\chtype{L_1}, \chtype{L_2}) \bnfbar L_1\ \Re\ L_2}\\
				&\cup&	\set{(\shot{C_1}, \shot{C_2}), (\lhot{C_1}, \lhot{C_2}) \bnfbar C_1\ \Re\ C_2}\\
				&\cup&	\set{(\btout{U_1} S, \btout{U_2} T)\,,\, (\btinp{U_1} S, \btinp{U_1} T) \bnfbar U_1\ \Re\ U_2, S\ \Re\ T}\\
				&\cup&	\set{(\btsel{l_i: S_i}_{i \in I} \,,\, \btsel{l_i: T_i}_{i \in I}) \bnfbar  S_i\ \Re\ T_i}\\
				&\cup&	\set{(\btbra{l_i: S_i}_{i \in I}\,,\, \btbra{l_i: T_i}_{i \in I}) \bnfbar S_i\ \Re\ T_i}\\
				&\cup&	\set{(S\,,\, T) \bnfbar S\subst{\trec{t}{S}}{\vart{t}}\ \Re\ T)}
				\cup	\set{(S\,,\, T) \bnfbar S\ \Re\ T\subst{\trec{t}{T}}{\vart{t}})}
		\end{array}
	\]	
	\noi Standard arguments ensure that $F$ is monotone, thus the greatest fixed point
	of $F$ exists. Let type equivalence be defined as $\iso = \nu X. F(X)$.
\end{definition}
%
\noi Type equivalence is in a essence a co-inductive definition that
equates types up-to recursive unfolding.

The duality relation is defined in terms of type equivalence.
%
\begin{definition}[Duality]\rm
	Define function $F(\Re): \mathsf{ST} \longrightarrow \mathsf{ST}$:
	%
	\[
		\begin{array}{rcl}
			F(\Re) 	&=&	\set{(\tinact, \tinact)}\\
				&\cup&	\set{(\btout{U_1} S, \btinp{U_2} T)\,,\, (\btinp{U} S, \btout{U} T) \bnfbar U_1 \iso U_2, S\ \Re\ T}\\
				&\cup&	\set{(\btsel{l_i: S_i}_{i \in I} \,,\, \btbra{l_i: T_i}_{i \in I}) \bnfbar  S_i\ \Re\ T_i}\\
				&\cup&	\set{(\btbra{l_i: S_i}_{i \in I}\,,\, \btsel{l_i: T_i}_{i \in I}) \bnfbar S_i\ \Re\ T_i}\\
				&\cup&	\set{(S\,,\, T) \bnfbar S\subst{\trec{t}{S}}{\vart{t}}\ \Re\ T)}\\
				&\cup&	\set{(S\,,\, T) \bnfbar S\ \Re\ T\subst{\trec{t}{T}}{\vart{t}})}
		\end{array}
	\]	
	\noi Standard arguments ensure that $F$ is monotone, thus the greatest fixed point
	of $F$ exists. Let duality be defined as $\dualof = \nu X. F(X)$.
\end{definition}
%
Duality is applied co-inductively to session types
up-to recursive unfolding.
Dual session types are prefixed
on dual session type constructors
($!$ is dual to $?$ and $\oplus$ is dual to $\&$)
that carry equivalent types.
%The co-inductive definition of duality relates
%session types up-to recursive unfolding.

%%%%%%%%%%%%%%%%%%%%%%%%%%%%%%%%%%%%%%%%%%%%%%%%%%
%  TYPING SYSTEM
%%%%%%%%%%%%%%%%%%%%%%%%%%%%%%%%%%%%%%%%%%%%%%%%%%

\subsection{Type Environments and Judgements}
We follow~\cite{tlca07},
to defined the typing environment.
%
\begin{definition}[Typing environment]\rm
	We define the {\em shared type environment} $\Gamma$,
	the {\em linear type environment} $\Lambda$, and
	the {\em session type environment} $\Delta$ as:
	\[
	\begin{array}{llcl}
		\text{(Shared)}		& \Gamma  & \bnfis &	\emptyset \bnfbar \Gamma \cat x: \shot{C} \bnfbar \Gamma \cat u: \chtype{S} \bnfbar
								\Gamma \cat u: \chtype{L} \bnfbar \Gamma \cat \varp{X}: \Delta
		\\
		\text{(Linear)}		& \Lambda & \bnfis &	\emptyset \bnfbar \Lambda \cat x: \lhot{C}
		\\
		\text{(Session)}	& \Delta  & \bnfis &	\emptyset \bnfbar \Delta \cat u:S
	\end{array}
	\]
	We imply
	\begin{enumerate}[i.]
		\item	Domains of $\Gamma, \Lambda, \Delta$ are pairwise distinct.
		\item	Weakening, contraction and exchange apply to shared environment $\Gamma$.
		\item	Exchange applies to linear environments $\Lambda$ and $\Delta$. 
	\end{enumerate}
\end{definition}
%
\noi We define typing judgements for values $V$
and processes $P$:
%
\[	\begin{array}{c}
		\Gamma; \Lambda; \Delta \proves V \hastype U \qquad \qquad \qquad \qquad \Gamma; \Lambda; \Delta \proves P \hastype \Proc
	\end{array}
\]
%
\noi The first judgement asserts that under environment $\Gamma; \Lambda; \Delta$
values $V$ have type $U$,
whereas the second judgement asserts that under environment $\Gamma; \Lambda; \Delta$
process $P$ has the typed process type $\Proc$.

%%%%%%%%%%%
%	Type system description
%%%%%%%%%%%

\subsection{Typing Rules}

\begin{figure}[!t]
\[
	\begin{array}{c}
%		\jrule{ }{\Gamma ; \emptyset; \emptyset \vdash \UnitV \hastype \Unit}{Unit} 
%		\qquad\quad  
		\trule{Session}~~\Gamma; \emptyset; \set{k:S} \proves k \hastype S 
		\\[2mm]
		\trule{Shared}~~\Gamma \cat a : \chtype{S}; \emptyset; \emptyset \proves a \hastype \chtype{S}
		\qquad
		\trule{LVar}~~\Gamma; \set{X: \lhot{S}}; \emptyset \proves X \hastype \lhot{S} 
		\\[2mm]
		\trule{Prom}~~\tree{
			\Gamma; \emptyset; \emptyset \proves V \hastype \lhot{S}
		}{
			\Gamma; \emptyset; \emptyset \proves V \hastype \shot{S}
		} 
		\qquad\quad  
		\trule{Derelic}~~\tree{
			\Gamma; \Lambda \cat X{:}\lhot{S}; \Sigma \proves P \hastype \Proc
		}{
			\Gamma \cat X:\shot{S}; \Lambda; \Sigma \proves P \hastype \Proc
		} 
		\\[4mm]
%		\trule{Subt}~~\tree{
%			\Gamma; \Lambda; \Sigma \proves P \hastype T \quad \Sigma \subt \Sigma' \quad T \subt T'
%		}{
%			\Gamma ; \Lambda; \Sigma' \vdash P \hastype T'
%		} 
%		\qquad\quad

		\trule{Abs}~~\tree{
			\Gamma; \Lambda; \Sigma \cat x: S \proves P \hastype \Proc
		}{
			\Gamma; \Lambda; \Sigma \proves \abs{x}{P} \hastype \lhot{S}
		}
		\\[4mm]

		\trule{App}~~\tree{(U = \lhot{S}) \lor (U = \shot{S}) \quad \Gamma; \Lambda_1; \Sigma_1 \proves X \hastype U  \quad \Gamma; \Lambda_2; \Sigma_2 \proves k \hastype S
		}{
			\Gamma; \Lambda_1 \cup \Lambda_2; \Sigma_1 \cup \Sigma_2 \proves \appl{X}{k} \hastype \Proc
		} 
		\\[4mm]

		\trule{Send}~~\tree{
			\Gamma; \Lambda_1; \Sigma_1 \proves P \hastype \Proc  \quad \Gamma; \Lambda_2; \Sigma_2 \vdash V \hastype U  \quad (k:S \in \Sigma_1 \cup \Sigma_2)
		}{
			\Gamma; \Lambda_1 \cup \Lambda_2; (\Sigma_1 \cup \Sigma_2)\backslash\set{k:S} \cat k:\btout{U} S \proves \bout{k}{V} P \hastype \Proc
		}

		\\[4mm]
		\trule{Conn}~~\tree{
			\Gamma; \Lambda; \Sigma \cat x:S \proves P \hastype \Proc  \quad \Gamma; \emptyset; \emptyset \proves a \hastype \chtype{S}
		}{
			\Gamma; \Lambda; \Sigma \proves \binp{a}{x} P \hastype \Proc
		}
		\\[4mm]
%		\trule{ConnDual}~~\tree{
%			\Gamma; \Lambda; \Sigma \cat x: S_1 \proves P \hastype \Proc  \quad \Gamma; \emptyset; \emptyset \proves k \hastype \chtype{S_2} \quad S_1 \dualof S_2
%		}{
%			\Gamma; \Lambda; \Sigma \proves \bout{k}{x} P \hastype \Proc
%		}
%		\\[4mm]

		\trule{ConnDual}~~\tree{
			\Gamma; \Lambda; \Sigma \proves P \hastype \Proc  \quad \Gamma; \emptyset; \emptyset \proves a \hastype \chtype{S_2} \quad S_1 \dualof S_2
		}{
			\Gamma; \Lambda; \Sigma \cat k: S_1  \proves \bout{a}{k} P \hastype \Proc
		}

		\\[4mm]

		\trule{NewSh}~~\tree{
			\Gamma\cat a:\chtype{S} ; \Lambda; \Sigma \proves P \hastype \Proc
		}{
			\Gamma; \Lambda; \Sigma \proves \news{a} P \hastype \Proc}
		\qquad\quad
		\trule{NewSes}~~\tree{
			\Gamma; \Lambda; \Sigma \cat s:S_1 \cat \dual{s}: S_2 \proves P \hastype \Proc \quad S_1 \dualof S_2
		}{
			\Gamma; \Lambda; \Sigma \proves \news{s} P \hastype \Proc
		}
		\\[4mm]

		\trule{RecvS}~~\tree{
			\Gamma; \Lambda; \Sigma \cat k: S_1 \cat x: S_2 \proves P \hastype \Proc
		}{
			\Gamma; \Lambda; \Sigma, k: \btinp{S_2} S_1  \vdash \binp{k}{x}P \hastype \Proc
		}
		\quad\quad 
		\trule{RecvL}~~\tree{
			\Gamma; \Lambda \cat X: \lhot{S}; \Sigma \cat k: S_1  \proves P \hastype \Proc
		}{
			\Gamma; \Lambda; \Sigma \cat k:\btinp{\lhot{S}}S_1  \proves \binp{k}{X}P \hastype \Proc
		}
		\\[4mm]
		\trule{RecvShN}~~\tree{
			\Gamma \cat x: \chtype{S}; \Lambda; \Sigma \cat k: S_1  \proves P \hastype \Proc
		}{
			\Gamma; \Lambda; \Sigma \cat k:\btinp{\chtype{S}}S_1  \proves \binp{k}{x}P \hastype \Proc
		}
		
		\quad ~~
		\trule{RecvSh}~~\tree{
			\Gamma \cat X: \shot{S}; \Lambda; \Sigma \cat k: S_1  \proves P \hastype \Proc
		}{
			\Gamma; \Lambda; \Sigma \cat k:\btinp{\shot{S}}S_1  \proves \binp{k}{X}P \hastype \Proc
		}
		\\[4mm]
		\trule{Par}~~\tree{
			\Gamma; \Lambda_{1}; \Sigma_{1} \proves P_{1} \hastype \Proc \quad \Gamma; \Lambda_{2}; \Sigma_{2} \proves P_{2} \hastype \Proc
		}{
			\Gamma; \Lambda_{1} \cup \Lambda_2; \Sigma_{1} \cup \Sigma_2 \proves P_1 \Par P_2 \hastype \Proc
		}
		\qquad\quad
		\trule{Close}~~\tree{
			\Gamma; \Lambda; \Sigma  \proves P \hastype T \quad k \not\in \dom{\Gamma, \Lambda,\Sigma}
		}{
			\Gamma; \Lambda; \Sigma \cat k: \tinact  \proves P \hastype \Proc
		}
		\\[4mm]
		\trule{Bra}~~\tree{
			 \forall i \in I \quad \Gamma; \Lambda; \Sigma \cat k:S_i \proves P_i \hastype \Proc
		}{
			\Gamma; \Lambda; \Sigma \cat k: \btbra{l_i:S_i}_{i \in I} \proves \bbra{k}{l_i:P_i}_{i \in I}\hastype \Proc
		}
		\qquad\quad 
	 	\trule{Sel}~~\tree{
			\Gamma; \Lambda; \Sigma \cat k: S_j  \proves P \hastype \Proc \quad j \in I
		}{
			\Gamma; \Lambda; \Sigma \cat k:\btsel{l_i:S_i}_{i \in I} \proves \bsel{s}{l_j} P \hastype \Proc
		}
		\\[4mm]

		\trule{Nil}~~\Gamma; \emptyset; \emptyset \proves \inact \hastype \Proc
\qquad \quad
		\trule{Var}~~\tree{
	
		}{
			\Gamma \cat \rvar{X}: \Sigma; \emptyset; \emptyset  \proves \rvar{X} \hastype \Proc
		}
		\qquad\quad 
%	 	\trule{Rec}~~\tree{
%			\Gamma \cat \rvar{X}: \Sigma; \emptyset; \emptyset  \proves P \hastype \Proc
%		}{
%			\Gamma ; \emptyset; \emptyset  \proves \recp{X}{P} \hastype \Proc
%		}
%		\\[4mm]

	 	\trule{Rec}~~\tree{
			\Gamma \cat \rvar{X}: \Sigma; \emptyset; \Sigma  \proves P \hastype \Proc
		}{
			\Gamma ; \emptyset; \Sigma  \proves \recp{X}{P} \hastype \Proc
		}


	\end{array}
\]
\caption{Typing Rules for $\HOp$\label{fig:typerulesmy}}
\end{figure}


The type relation is defined in \figref{fig:typerulesmy}.
%Types for session names/variables $u$ and
%directly derived from the linear part of the typing
%environment, i.e.~type maps $\Delta$ and $\Lambda$.
Rules $\trule{Session}$ and $\trule{LVar}$ require,
respectively, the minimal:
i) session environment $\Delta$ to type session 
$u$ with type $S$; and,
ii) linear environment $\Lambda$ to type 
higher-order variable $x$ with type $\shot{C}$.
Rule $\trule{Shared}$
assigns the value type $U$
to shared names or shared variables $u$ 
if the map $u:U$ exists in environment
$\Gamma$. Rule $\trule{Shared}$ also requires 
that the linear environment is
be minimal, i.e.~empty.
The type $\shot{C}$ for shared higher-order values $V$
is derived using rule $\trule{Prom}$, where we require
a value with linear type to be typed without a linear
environment present in order to be used as a shared type.
Rule $\trule{EProm}$ allows to freely use a linear
type variable as shared type variable. 
%A value consisting of a tuple of names/variables is typed using the $\trule{Pol}$ rule,
%which requires theto type and combine each value in the tuple.
Abstraction values are typed with rule $\trule{Abs}$.
The key type for an abstraction is the type for
the bound variables of the abstraction, i.e.~for
bound variable with type $C$ the abstraction
has type $\lhot{C}$.
The dual of abstraction typing is application typing
governed by rule $\trule{App}$, where we expect
the type $C$ of an application name $u$ 
to match the type $\lhot{C}$ or $\shot{C}$
of the application variable $x$.

A process prefixed with a session send operator $\bout{u}{V} P$
is typed using rule $\trule{Send}$.
The type $U$ of a send value $V$ should appear as a prefix
on the session type $\btout{U} S$ of $s$.
Rule $\trule{Rcv}$
defines the typing for the 
reception of values $\binp{u}{V} P$.
The type $U$ of a receive value should 
appear as a prefix on the session type $\btinp{U} S$ of $u$.
We use a similar approach with session prefixes
to type interaction between shared channels as defined 
in rules $\trule{Req}$ and $\trule{Acc}$,
where the type of the sent/received object 
($S$ and $L$, respectively) should
match the type of the sent/received subject
($\chtype{S}$ and $\chtype{L}$, respectively).
%In the case of rule $\trule{Req}$ we require
%a duality condition for the communication of session names.
Select and branch prefixes are typed using the rules
$\trule{Sel}$ and $\trule{Bra}$ respectively. Both
rules prefix the session type with the selection
type $\btsel{l_i: S_i}_{i \in I}$ and
$\btbra{l_i:S_i}_{i \in I}$.

The creation of a
shared name $a$ requires to add
its type in environment $\Gamma$ as defined in 
rule \trule{Res}. 
Creation of a session name $s$
creates two endpoints with dual types and adds them to
the session environment 
$\Delta$ as defined in rule \trule{ResS}. 
Rule \trule{Par} concatanates the linear environment of
the parallel components of a parallel operator
to create a type for the entire process.
The disjointness of environments $\Lambda$ and $\Delta$
is implied. Rule \trule{End} allows a form of weakening 
for the session environment $\Delta$, provided that
the name added in $\Delta$ has the inactive
type $\tinact$. The inactive process $\inact$ has no
linear environment. The recursive variable is typed
directly from the shared environment $\Gamma$ as
in rule \trule{RVar}.
The recursive operator requires that the body of
a recursive process matches the type of the recursive
variable as in rule \trule{Rec}.

\begin{comment}
\subsection{Order of Types}

In~\cite{tlca07} the type syntax for values includes the definition
$U_1 \sharedop U_2$ and $U_1 \lollipop U_2$, that
allows us to define types of arbitrary order $k$.
An abstraction of $k$-order types requires to extend the syntax
to include higher-order applications:
\[
	\abs{z}{\binp{z}{x} \appl{x}{\abs{y} Q}}
\]
with with the type of $\abs{y}{Q}$ being of order
$k-1$. The type of of such an abstraction in the current setting would
be $\shot{U}$ (or $\lhot{U}$) with the order of the type being defined
as the number of nested higher-order types~\cite{San96int}.

In the type system we develop for the \HOp we only have
types of the form $\shot{C}$.
If we maintain the definition of counting the order
of the type as the nesting of higher-order types we
can still express $k$-order types, e.g:
\[
	\shot{(\btinp{U} \tinact)}
\]
with $U$ being of order $k-1$.
An $k$-order abstraction in \HOp would be:
\[
	\abs{z}{\binp{z}{x} \binp{x}{y} \appl{y}{n}}
\]
with $y$ being of order $k-1$.

\begin{definition}[Order of Value Type]\rm
	\label{def:order_type}
	Let type $U$ and value $V$ such that $\Gamma; \Lambda; \Delta \proves V \hastype U$.
	The order of $U$ is the number of using rule $\trule{Abs}$
	in the typing derivation $\Gamma; \Lambda; \Delta \proves V \hastype U$.
\end{definition}
\end{comment}

\subsection{Type Soundness}

%We state results for type safety:
Type safety result are instances of more general
statements already proved by
Mostrous and Yoshida~\cite{tlca07} in the asynchronous case.
%
\begin{lemma}[Substitution Lemma - Lemma C.10 in M\&Y]\rm
	\label{lem:subst}
	\begin{enumerate}[1.]
		\item	$\Gamma; \Lambda; \Delta \cat x:S  \proves P \hastype \Proc$ and
			$u \not\in \dom{\Gamma, \Lambda, \Delta}$
			implies
			$\Gamma; \Lambda; \Delta \cat u:S  \proves P\subst{u}{x} \hastype \Proc$.

		\item	$\Gamma \cat x:\chtype{U}; \Lambda; \Delta \proves P \hastype \Proc$ and
			$a \notin \dom{\Gamma, \Lambda, \Delta}$
			implies
			$\Gamma \cat a:\chtype{U}; \Lambda; \Delta \proves P\subst{a}{x} \hastype \Proc$.

		\item	If $\Gamma; \Lambda_1 \cat x:\lhot{C}; \Delta_1  \proves P \hastype \Proc$ 
			and $\Gamma; \Lambda_2; \Delta_2  \proves V \hastype \lhot{C}$ with 
			$\Lambda_1 \cat \Lambda_2$ and $\Delta_1 \cat \Delta_2$ defined,
			then $\Gamma; \Lambda_1 \cat \Lambda_2; \Delta_1 \cat \Delta_2  \proves P\subst{V}{x} \hastype \Proc$.

		\item	$\Gamma \cat x:\shot{C}; \Lambda; \Delta  \proves P \hastype \Proc$ and
			$\Gamma; \emptyset ; \emptyset \proves V \hastype \shot{C}$
			implies
			$\Gamma; \Lambda; \Delta \proves P\subst{V}{x} \hastype \Proc$.
		\end{enumerate}
\end{lemma}
%
\begin{proof}
	By induction on the typing for $P$, with a case analysis on the last used rule. 
	\qed
\end{proof}

We are interested in session environments that whenever they contain dual endpoints
their types are dual:
%
\begin{definition}[Balanced Session Environment]\label{d:wtenv}\rm
	We say that session environment $\Delta$ is {\em balanced} if
	$s: S_1, \dual{s}: S_2 \in \Delta$ implies $S_1 \dualof S_2$.
\end{definition}
%
The type soundness relies on the following auxiliary definition:
%
\begin{definition}[Session Environment Reduction]\rm
	\label{def:ses_red}
	The reduction relation $\red$ on session environments is defined as:
%
\[
	\begin{array}{rcl}
		\Delta \cat s: \btout{U} S_1 \cat \dual{s}: \btinp{U} S_2 &\red& \Delta \cat s: S_1 \cat \dual{s}: S_2
		\\
		\Delta \cat s: \btsel{l_i: S_i}_{i \in I} \cat \dual{s}: \btbra{l_i: S_i'}_{i \in I} &\red& \Delta \cat s: S_k \cat \dual{s}: S_k', \quad k \in I
	\end{array}
\]
%
	We write $\red^\ast$ for the multistep environment reduction.
\end{definition}
%
We now state the main soundness result as an instance
of type soundness from the system in~\cite{tlca07}.
It is worth noticing that in~\cite{tlca07} has a slightly richer
definition of structural congruence.
Also, their statement for subject reduction relies on an
ordering on typing associated to queues and other 
runtime elements. %(such extended typing is denoted as $\Delta$ by M\&Y).
Since we are dealing with synchronous semantics we can omit such an ordering.
The type soundness result implies soundness for the sub-calculi
\HO, \sessp, and $\CAL^{\minussh}$

\begin{theorem}[Type Soundness - Theorem 7.3 in M\&Y]\rm
	\label{thm:sr}
%
	\begin{enumerate}[1.]
		\item	(Subject Congruence)
			$\Gamma; \es; \Delta \proves P \hastype \Proc$
			and
			$P \scong P'$
			implies
			$\Gamma; \es; \Delta \proves P' \hastype \Proc$.

		\item	(Subject Reduction)
			$\Gamma; \es; \Delta \proves P \hastype \Proc$
			with
			balanced $\Delta$
			and
			$P \red P'$
			implies $\Gamma; \es; \Delta'  \proves P' \hastype \Proc$
			and either (i)~$\Delta = \Delta'$ or (ii)~$\Delta \red \Delta'$
			with $\Delta'$ balanced.
	\end{enumerate}
\end{theorem}

\begin{proof}
	See \appref{app:ts}.
	\qed
\end{proof}


\section{Behavioral Theory}
% !TEX root = main.tex
\noi In this section we define reduction-closed, barbed congruence ($\cong$) as the
reference equivalence relation for \HOp processes.
Later on we will define two characterizations of $\cong$:
\emph{higher-order} and  
\emph{characteristic bisimilarities} (denoted $\hwb$ and $\fwb$, respectively). 
Here we focus on collecting intuitions; omitted details are in the Appendix and in~\cite{KouzapasPY15}.

\subsection{Reduction-Closed, Barbed Congruence}
\label{subsec:rc}
We first define \emph{confluence} over session environments $\Delta$:

\begin{definition}[Session Environment Confluence]
Let $\red^\ast$ denote multi-step reduction as in \defref{def:ses_red}.
	We denote $\Delta_1 \bistyp \Delta_2$ if there exists $\Delta$ such that
	$\Delta_1 \red^\ast \Delta$ and $\Delta_2 \red^\ast \Delta$.
\end{definition}

%\smallskip 
\noi Reduction-closed, barbed congruence is defined over typed
processes:

\begin{definition}[Typed Relation]
	We say that
	$\Gamma; \emptyset; \Delta_1 \proves P_1 \hastype \Proc\ \Re \ \Gamma; \emptyset; \Delta_2 \proves P_2 \hastype \Proc$
	is a {\em typed relation} whenever
	$P_1$ and $P_2$ are closed;
	$\Delta_1$ and $\Delta_2$ are balanced; and 
	$\Delta_1 \bistyp \Delta_2$.
	We write $\horel{\Gamma}{\Delta_1}{P_1}{\ \Re \ }{\Delta_2}{P_2}$
	for the typed relation $\Gamma; \emptyset; \Delta_1 \proves P_1 \hastype \Proc\ \Re \ \Gamma; \emptyset; \Delta_2 \proves P_2 \hastype \Proc$.
\end{definition}

%\smallskip 

\noi Observe that typed relations relate only closed terms whose
session environments %and the two session environments
are balanced  and confluent.
Next we define  {\em barbs}~\cite{MiSa92}
with respect to types. 

%\smallskip 

\begin{definition}[Barbs]\rm
	Let $P$ be a closed process. We define:
	\begin{enumerate}
		\item	$P \barb{n}$ if $P \scong \newsp{\tilde{m}}{\bout{n}{V} P_2 \Par P_3}, n \notin \tilde{m}$. %; $P \Barb{n}$ if $P \red^* \barb{n}$.

		\item	$\Gamma; \Delta \proves P \barb{n}$ if
			$\Gamma; \emptyset; \Delta \proves P \hastype \Proc$ with $P \barb{n}$ and $\dual{n} \notin \dom{\Delta}$.

			$\Gamma; \Delta \proves P \Barb{n}$ if $P \red^* P'$ and
			$\Gamma; \Delta' \proves P' \barb{n}$.			
	\end{enumerate}
\end{definition}

%\smallskip 

\noi A barb $\barb{n}$ is an observable on an output prefix with subject $n$.
Similarly a weak barb $\Barb{n}$ is a barb after a number of reduction steps.
Typed barbs $\barb{n}$ (resp.\ $\Barb{n}$)
occur on typed processes $\Gamma; \emptyset; \Delta \proves P \hastype \Proc$.
When $n$ is a session name we require that its dual endpoint $\dual{n}$ is not present
in the session environment $\Delta$.

To define a congruence relation, we introduce the family $\C$ of contexts:

\begin{definition}[Context]
	A context $\C$ is defined as:

	\begin{tabular}{rcl}
		$\C$ & $\bnfis$ & $\hole \bnfbar \bout{u}{V} \C \bnfbar \binp{u}{x} \C \bnfbar \bout{u}{\lambda x.\C} P \bnfbar \news{n} \C
		\bnfbar (\lambda x.\C)u \bnfbar \recp{X}{\C}$ 
		\\
		&$\bnfbar$& $\C \Par P \bnfbar P \Par \C
		\bnfbar \bsel{u}{l} \C \bnfbar \bbra{u}{l_1:P_1,\cdots,l_i:\C,\cdots,l_n:P_n}$
	\end{tabular}
%\smallskip 

\noi 
Notation $\context{\C}{P}$ replaces 
the hole $\hole$ in $\C$ with $P$.
\end{definition}

%\smallskip 

\noi We define reduction-closed, barbed congruence \cite{HondaKYoshida95}. 

%\smallskip 

\begin{definition}[Reduction-Closed, Barbed Congruence]
\label{def:rc}
	Typed relation
	$\horel{\Gamma}{\Delta_1}{P_1}{\ \Re\ }{\Delta_2}{P_2}$
	is a {\em reduction-closed, barbed congruence} whenever:
	\begin{enumerate}[1)]
		\item	If $P_1 \red P_1'$ then there exist $P_2', \Delta_2'$ such that $P_2 \red^* P_2'$ and
			$\horel{\Gamma}{\Delta_1'}{P_1'}{\ \Re\ }{\Delta_2'}{P_2'}$; 
			%and its symmetric case;
%		\item	If $P_2 \red P_2'$ then $\exists P_1', P_1 \red^* P_1'$ and
%		$\horel{\Gamma}{\Delta_1'}{P_1'}{\ \Re\ }{\Delta_2'}{P_2'}$
%		\end{itemize}

%		\item
%		\begin{itemize}
			\item	If $\Gamma;\Delta_1 \proves P_1 \barb{n}$ then $\Gamma;\Delta_2 \proves P_2 \Barb{n}$; %and its symmetric case; 

%			\item	If $\Gamma;\emptyset;\Delta \proves P_2 \barb{s}$ then $\Gamma;\emptyset;\Delta \proves P_1 \Barb{s}$.
%		\end{itemize}

		\item	For all $\C$, there exist $\Delta_1'',\Delta_2''$: $\horel{\Gamma}{\Delta_1''}{\context{\C}{P_1}}{\ \Re\ }{\Delta_2''}{\context{\C}{P_2}}$;
		                      \item	The symmetric cases of 1 and 2.                
	\end{enumerate}
	The largest such relation is denoted with $\cong$.
\end{definition}

{
\subsection{Two Equivalence Relations: $\fwb$ and $\hwb$}\label{ss:equiv}

\jparagraph{A Typed Labelled Transition System}
In~\cite{characteristic_bis} we have characterised
reduction-closed, barbed congruence for \HOp
via a typed relation called
{\em characteristic bisimilarity}.
%The definition of characteristic bisimilarity 
Its definition 
uses a \emph{typed}
labelled transition system (LTS) informed by session
types. 
Transitions in this LTS are denoted 
$\Gamma; \es; \Delta \proves P \hby{\ell} \Delta' \proves P' \hastype \Delta'$.
The main intuition %for its definition  
is that the transitions 
of a typed process should be enabled by its associated typing environment:
%
\[
	\tree {
		P \hby{\ell} P' \qquad (\Gamma, \Delta) \hby{\ell} (\Gamma, \Delta')
	}{
		\Gamma; \es; \Delta \proves P \hby{\ell} \Delta' \proves P' \hastype \Delta'
	}
\]
%
\noi As an example of how types enable transitions, consider the rule for input:
%
\[
	\tree{
		\dual{s} \notin \dom{\Delta} \qquad \Gamma; \Lambda'; \Delta' \proves V \hastype U
		\qquad
		V = m \vee  V \scong \omapchar{U} \vee V \scong \abs{{x}}{\binp{t}{y} (\appl{y}{{x}})}
					\textrm{ with } t \textrm{ fresh} 
	}{
		(\Gamma; \Lambda; \Delta \cat s: \btinp{U} S) \hby{\bactinp{s}{V}} (\Gamma; \Lambda\cat\Lambda'; \Delta\cat\Delta' \cat s: S)
	}
\]
\noi
This rule states that an session channel environment can input a value
if the channel is typed with an input prefix and the input value is either
a name $m$, a \emph{characteristic value} $\omapchar{U}$,  or a \emph{trigger value} (the abstraction
$\abs{{x}}{\binp{t}{y} (\appl{y}{{x}})}$). 
A characteristic value
is the {simplest} process that  inhabits a type (in this case, the
type $U$ carried by the session input prefix). The above rule is used to limit
the input actions that can be observed from a session input prefix.
For full details of the labelled transition system and the characteristic process definition
see \appref{app:behavioural} and~\cite{characteristic_bis}.
Moreover, we define two \emph{trigger processes}:
%
\begin{eqnarray*}
	\htrigger{t}{V}  & \defeq &  \hotrigger{t}{V} \label{eqb:0} \\
	\ftrigger{t}{V}{U} & \defeq &  \fotrigger{t}{x}{s}{\btinp{U} \tinact}{V} 	%\label{eqb:4}
\end{eqnarray*}
%
\noi
Trigger processes are defined as a process input prefixed on
a fresh name $t$. The   trigger process $\htrigger{t}{V}$ above applies a value $V$
on the receiving (characteristic) process.
The trigger process $\ftrigger{t}{V}{U}$ applies a value
on the (hard coded) \emph{characteristic process} $\map{\btinp{U} \tinact}^{s}$ (see~\cite{characteristic_bis} for details). 

\jparagraph{Characterisations of Barbed Congruence}
We now define \emph{characteristic} and \emph{higher-order} bisimilarity.
Observe that higher-order bisimilarity is a new typed equality, 
while 
characteristic bisimilarity was introduced in~\cite{characteristic_bis} (Def.~14).

\begin{definition}[Characteristic Bisimilarity]%\rm
	\label{d:fwb}
	A typed relation $\Re$ is a {\em  characteristic bisimulation} if 
	for all $\horel{\Gamma}{\Delta_1}{P_1}{\ \Re \ }{\Delta_2}{Q_1}$ 
	\begin{enumerate}[1)]
		\item 
				Whenever 
				$\horel{\Gamma}{\Delta_1}{P_1}{\hby{\news{\tilde{m_1}} \bactout{n}{V_1: U}}}{\Delta_1'}{P_2}$,
				there exist 
				$Q_2$, $V_2$, $\Delta'_2$
				such that \\
				$\horel{\Gamma}{\Delta_2}{Q_1}{\Hby{\news{\tilde{m_2}} \bactout{n}{V_2: U}}}{\Delta_2'}{Q_2}$ and, for fresh $t$, 
%
				\[
					\begin{array}{lrlll}
						\Gamma; \Delta''_1  \proves  {\newsp{\tilde{m_1}}{P_2 \Par \ftrigger{t}{V_1}{U_1}}}
						\ \Re \
						\Delta''_2 \proves {\newsp{\tilde{m_2}}{Q_2 \Par \ftrigger{t}{V_2}{U_2}}}
					\end{array}
				\]

		\item	
				For all $\horel{\Gamma}{\Delta_1}{P_1}{\hby{\ell}}{\Delta_1'}{P_2}$ such that 
				$\ell$ is not an output, 
				there exist $Q_2$, $\Delta'_2$ such that 
				$\horel{\Gamma}{\Delta_2}{Q_1}{\Hby{\hat{\ell}}}{\Delta_2'}{Q_2}$
				and
				$\horel{\Gamma}{\Delta_1'}{P_2}{\ \Re \ }{\Delta_2'}{Q_2}$; and 

		\item	The symmetric cases of 1 and 2.                
	\end{enumerate}
	The largest such bisimulation
	is called \emph{characteristic bisimilarity} \jpc{and} denoted by $\fwb$.
\end{definition}
 

%\noi 
%We define characteristic bisimilarity
%(cf.~Def.~14~in~\cite{characteristic_bis}):
% is given using characteristic trigger processes. 

\begin{definition}[Higher-Order Bisimilarity]%\rm
	\label{d:hbw}
	Higher-order bisimilarity, denoted by $\hwb$, is defined \jpc{by} replacing 
	Clause 1) in \defref{d:fwb} with the following clause:\\[1mm]
	Whenever 
	$\horel{\Gamma}{\Delta_1}{P_1}{\hby{\news{\tilde{m_1}} \bactout{n}{V_1}}}{\Delta_1'}{P_2}$ %with $\Gamma; \es; \Delta \proves V_1 \hastype U$,  
	then there exist 
	$Q_2$, $V_2$, $\Delta'_2$
	such that \\
	$\horel{\Gamma}{\Delta_2}{Q_1}{\Hby{\news{\tilde{m_2}}\bactout{n}{V_2}}}{\Delta_2'}{Q_2}$ %with $\Gamma; \es; \Delta' \proves V_2 \hastype U$,  
	and, for fresh $t$, \\[1mm]
	$
	\begin{array}{lrlll}
		\!\!\Gamma; \Delta''_1  \proves  {\newsp{\tilde{m_1}}{P_2 \Par 
		\htrigger{t}{V_1}}}
		\ \!\!\Re\!\!\ \Delta''_2
		\proves {\newsp{\tilde{m_2}}{Q_2 \Par \htrigger{t}{V_2}}}
	\end{array}
	$
\end{definition}

\begin{theorem}[\cite{KouzapasPY15}]
	Typed relations $\cong$, $\hwb$, and $\fwb$ coincide for \HOp processes.
\end{theorem}


\begin{remark}[Comparison between $\hwb$ and $\fwb$]
	The difference between higher order bisimilarity
	lies in the definition of the process trigger.
	\begin{enumerate}
		\item
				Higher-order bisimilarity trigger:
				\[
					\htrigger{t}{V}  \defeq  \hotrigger{t}{V}
				\]
				allows for the application of bound variable $x$,
				making $\hwb$ not defined in the \sessp sub-calculus
				of \HOp, but defined inside the \HO sub-calculus. 

		\item	Characteristic bisimilarity trigger:
				\[
					\ftrigger{t}{V}{U} \defeq  \fotrigger{t}{x}{s}{\btinp{U} \tinact}{V}
				\]
				solves the problem in (1) by not using $x$ in the continuation
				of fresh name input. This allows to input characteristic, first- or
				higher-order value on bound variable $x$.


		\item	Higher-order trigger lifts the need of observing
				types on the lts and including them in bisimulation
				closures. Higher-order bisimilarity adds to
				the simplicity and elegance of the \HO subcalculus
				over the \sessp subcalculus of \HOp.
	\end{enumerate}
\end{remark}

%\begin{definition}[Characteristic Bisimilarity]\rm
%	\label{d:fwb}
%
%	A typed relation $\Re$ is a {\em characteristic bisimulation} if 
%	for all $\horel{\Gamma}{\Delta_1}{P_1}{\ \Re \ }{\Delta_2}{Q_1}$ 
%	if whenever:
%	\begin{enumerate}[1)]
%		\item	$\horel{\Gamma}{\Delta_1}{P_1}{\hby{\news{\tilde{m_1}} \bactout{n}{V_1: U}}}{\Delta_1'}{P_2}$
%				then there exist  $Q_2$, $V_2$, $\Delta'_2$ such that 
%				$\horel{\Gamma}{\Delta_2}{Q_1}{\Hby{\news{\tilde{m_2}}\bactout{n}{V_2: U}}}{\Delta_2'}{Q_2}$
%				and, for fresh $t$,
%				$
%				\begin{array}{lrlll}
%					\Gamma; \Delta''_1 \proves {\newsp{\tilde{m_1}}{P_2 \Par  \ftrigger{t}{V_1}{U_1}}}
%					\ \Re\ \Delta''_2 \proves {\newsp{\tilde{m_2}}{Q_2 \Par \ftrigger{t}{V_2}{U_2}}}
%				\end{array}
%				$
%
%			\item	For all $\horel{\Gamma}{\Delta_1}{P_1}{\hby{\ell}}{\Delta_1'}{P_2}$ such that 
%					$\ell$ is not an output, there exist $Q_2$, $\Delta'_2$ such that 
%					$\horel{\Gamma}{\Delta_2}{Q_1}{\Hby{\hat{\ell}}}{\Delta_2'}{Q_2}$
%					and $\horel{\Gamma}{\Delta_1'}{P_2}{\ \Re \ }{\Delta_2'}{Q_2}$; and 
%
%			\item	The symmetric cases of 1 and 2.
%	\end{enumerate}
%	The largest such bisimulation is called \emph{characteristic bisimilarity} and denoted by $\fwb$.
%\end{definition}


\jparagraph{Up-to techniques}
We close this section by stating  a determinacy result for typed processes, useful in proving our expressiveness results (\secref{sec:positive}).
In our setting, processes that do not use shared names are inherently deterministic. 
In the sequel, we write 
 $\horel{\Gamma}{\Delta}{P}{\hby{\tau}}{\Delta'}{P'}$ to denote an internal (typed) transition.
The auxiliary definition below allows us to distinguish two kinds of  internal transitions:
\emph{session transitions} and \emph{$\beta$-transitions} (denoted 
$\horel{\Gamma}{\Delta}{P}{\hby{\stau}}{\Delta'}{P'}$
and $\horel{\Gamma}{\Delta}{P}{\hby{\btau}}{\Delta'}{P'}$, respectively).

\begin{definition}[Deterministic Transition]
\label{def:dettrans}
	Let  $\Gamma; \es; \Delta \proves P \hastype \Proc$ be a balanced \HOp process. 
	Transition $\horel{\Gamma}{\Delta}{P}{\hby{\tau}}{\Delta'}{P'}$ is called:
	\begin{enumerate}[$-$]
		\item	{\em Session transition}
				whenever the untyped transition $P \by{\tau} P'$ 
				is derived using  rule~$\ltsrule{Tau}$ 
				(where $\subj{\ell_1}$ and $\subj{\ell_2}$ in the premise are dual endpoints), 
				possibly followed by uses of
				$\ltsrule{Alpha}$, $\ltsrule{Res}$, $\ltsrule{Rec}$, or $\ltsrule{Par${}_L$}/
				\ltsrule{Par${}_R$}$.
		
		\item	{\em $\beta$-transition}
				whenever the untyped transition $P \by{\tau} P'$
				is derived using rule $\ltsrule{App}$,
				possibly followed by uses of  $\ltsrule{Alpha}$, $\ltsrule{Res}$, $\ltsrule{Rec}$, or $\ltsrule{Par${}_L$}/
				\ltsrule{Par${}_R$}$.
	\end{enumerate}
%
	We write
	$\horel{\Gamma}{\Delta}{P}{\hby{\stau}}{\Delta'}{P'}$
	and 
	$\horel{\Gamma}{\Delta}{P}{\hby{\btau}}{\Delta'}{P'}$
	to denote session and $\beta$-transitions, resp. Also, 
	 $\horel{\Gamma}{\Delta}{P}{\hby{\dtau}}{\Delta'}{P'}$ denotes
	either a session transition or a $\beta$-transition.
\end{definition}
%
%A transition $\horel{\Gamma}{\Delta}{P}{\hby{\tau}}{\Delta'}{P'}$ is said
%{\em deterministic} if it is derived using~$\ltsrule{App}$ or~$\ltsrule{Tau}$ 
%(where $\subj{\ell_1}$ and $\subj{\ell_2}$ in the premise  are dual endpoints), 
%possibly followed by uses of  $\ltsrule{Alpha}$, $\ltsrule{Res}$, $\ltsrule{Rec}$, or $\ltsrule{Par${}_L$}/\ltsrule{Par${}_R$}$.

\begin{lemma}[$\tau$-Inertness]\rm
	\label{lem:tau_inert}
	\begin{enumerate}[1)]
		\item
				Let $\horel{\Gamma}{\Delta}{P}{\hby{\dtau}}{\Delta'}{P'}$ be a deterministic transition,
				with balanced $\Delta$. \\ Then 
				$\Gamma; \Delta \proves P \cong \Delta'\proves P'$ 
				with $\Delta \red^\ast \Delta'$ balanced.

		\item 
				Let $P$ be an $\HOp^{-\mathsf{sh}}$ process. 
				Assume $\Gamma; \emptyset; \Delta \proves P \hastype \Proc$. \\ Then 
				$P \red^\ast P'$ implies $\Gamma; \Delta \proves 
				P \cong \Delta'\proves P'$ with $\Delta \red^\ast \Delta'$. 
	\end{enumerate}
\end{lemma}


\begin{proof}
	The proof uses the fact that processes of the
	form $\Gamma; \es; \Delta \proves_s \bout{s}{V} P_1 \Par \binp{k}{x} P_2$
	cannot have any typed transition observables and the fact
	that bisimulation is a congruence.
	See  \appref{app:sub_tau_inert} for details.
	The proof for Part 2 follows from Part 1.
	\qed
\end{proof}




\begin{lemma}[Up-to Deterministic Transition]%\myrm
	\label{lem:up_to_deterministic_transition}
	Let $\horel{\Gamma}{\Delta_1}{P_1}{\ \Re\ }{\Delta_2}{Q_1}$ such
	that if whenever:
%
	\begin{enumerate}
		\item	$\forall \news{\tilde{m_1}} \bactout{n}{V_1}$ such that
			$
				\horel{\Gamma}{\Delta_1}{P_1}{\hby{\news{\tilde{m_1}} \bactout{n}{V_1}}}{\Delta_3}{P_3}
			$
			implies that $\exists Q_2, V_2$ such that
			$
				\horel{\Gamma}{\Delta_2}{Q_1}{\Hby{\news{\tilde{m_2}} \bactout{n}{V_2}}}{\Delta_2'}{Q_2}
			$
			and
			$
				\horel{\Gamma}{\Delta_3}{P_3}{\Hby{\dtau}}{\Delta_1'}{P_2}
			$
			and for fresh $t$:
			$
				\horel{\Gamma}{\Delta_1''}{\newsp{\tilde{m_1}}{P_2 \Par \htrigger{t}{V_1}}}
				{\ \Re\ }
				{\Delta_2''}{}{\newsp{\tilde{m_2}}{Q_2 \Par \htrigger{t}{V_2}}}
%				\mhorel{\Gamma}{\Delta_1''}{\newsp{\tilde{m_1}}{P_2 \Par \hotrigger{t}{x}{s}{V_1}}}
%				{\ \Re\ }
%				{\Delta_2''}{}{\newsp{\tilde{m_2}}{Q_2 \Par \hotrigger{t}{x}{s}{V_2}}}
			$
%
		\item	$\forall \ell \not= \news{\tilde{m}} \bactout{n}{V}$ such that
			$
				\horel{\Gamma}{\Delta_1}{P_1}{\hby{\ell}}{\Delta_3}{P_3}
			$
			implies that $\exists Q_2$ such that 
			$
				\horel{\Gamma}{\Delta_1}{Q_1}{\hat{\Hby{\ell}}}{\Delta_2'}{Q_2}
			$
			and
			$
				\horel{\Gamma}{\Delta_3}{P_3}{\Hby{\dtau}}{\Delta_1'}{P_2}
			$
			and
			$\horel{\Gamma}{\Delta_1'}{P_2}{\ \Re\ }{\Delta_2'}{Q_2}$

		\item	The symmetric cases of 1 and 2.
	\end{enumerate}
	Then $\Re\ \subseteq\ \wb$.
\end{lemma}


\begin{proof}
	The proof is easy by considering the closure
	\[
		\Re^{\Hby{\dtau}} = \set{ \horel{\Gamma}{\Delta_1'}{P_2}{,}{\Delta_2'}{Q_1} \setbar \horel{\Gamma}{\Delta_1}{P_1}{\ \Re\ }{\Delta_2'}{Q_1},
		\horel{\Gamma}{\Delta_1}{P_1}{\Hby{\dtau}}{\Delta_1'}{P_2} }
	\]
	We verify that $\Re^{\Hby{\dtau}}$ is a bisimulation with
	the use of \propref{lem:tau_inert}.
	\qed
\end{proof}

\begin{example}[Up-to Deterministic Transition]
	Typed processes:
	\begin{eqnarray*}
		\Gamma; \es; \Delta, s': \tinact \proves P &=& \binp{n}{z_1} \newsp{s}{\binp{s}{x} \appl{(\abs{y}{\bout{n}{z_1} \inact})}{m} \Par \bout{\dual{s}}{s'} \inact} \hastype \Proc
		\\
		\Gamma; \es; \Delta \proves Q &=& \binp{n}{z_1} \binp{n}{z_2} \inact \hastype \Proc
	\end{eqnarray*}
	are bisimilar up-to deterministic transition because
	we can observe:
	\begin{eqnarray*}
		\Gamma; \Delta, s': \tinact \proves P &\hby{\bactinp{n}{m_1}}& \Delta', s: \tinact \proves \newsp{s}{\binp{s}{x} \appl{(\abs{y}{\bout{n}{z_2} \inact})}{m} \Par \bout{\dual{s}}{s'} \inact} \Hby{\dtau} \Delta' \proves \binp{n}{z_2} \inact
		\\
		\text{and}
		\\
		\Gamma; \es; \Delta \proves Q &\hby{\bactinp{n}{m_1}}& \Delta' \proves \binp{n}{z_2} \inact
	\end{eqnarray*}


%	Relation 
%	\[
%		\Re = \set{(\Gamma; \Delta, s': \tinact \proves P , \Delta \proves Q), (\Gamma; \Delta' \proves \binp{n}{z_2}, \Delta' \proves  \binp{n}{z_2})}
%	\]
%	is bisimulation up-to deterministic transition because
%	\begin{eqnarray*}
%		\Gamma; \Delta, s': \tinact \proves P &\hby{\bactinp{n}{s_1}}& \Delta', s: \tinact \proves \newsp{s}{\binp{s}{x} \appl{(\abs{y}{\bout{n}{y} \inact})}{s_1} \Par \bout{\dual{s}}{s'} \inact}
%		\\
%		\text{implies}&
%		\\
%		\Gamma; \es; \Delta \proves Q &\hby{\bactinp{n}{x}}& \Delta' \proves \binp{n}{z_2} \inact
%		\\
%		\text{and}&
%		\\
%		\Delta', s: \tinact \proves \newsp{s}{\binp{s}{x} \appl{(\abs{y}{\bout{n}{y} \inact})}{s_1} \Par \bout{\dual{s}}{s'} \inact}  \in \Re
%	\end{eqnarray*}
\end{example}

%\noi Precise encodings offer more detailed criteria and used for positive 
%encodability results (\secref{sec:positive}).
%In contrast, minimal encodings contains only 
%some of the criteria of precise encodings:    
%this reduced notion will be used 
%for the negative result in \secref{sec:negative}.




%\subsection{Higher-Order Session Bisimulation}
%\label{sec:behavioural}
%\section{Behavioral Semantics}

In this section we define a theory for observational equivalence over
session typed $\HOp$ processes. The theory follows the principles
laid by the previous work of the authors
\cite{DBLP:conf/forte/KouzapasYH11,KY13,dkphdthesis}.
We require a bisimulation relation over typed processes that
is also characterised by the corresponding typed, reduction-closed,
barbed congruence relation.

\dk{(Jorge, I think you have a paper we can cite over session typed bisimulations)}

\subsection{Labelled Transition Semantics}

We define a relation $(P_1, \lambda, P_2) \in R$ over
(untyped) processes, that allows us to follow how a process may
interact with a process in its enviroment. The interaction
is defined on action $\lambda$:

\begin{tabular}{rcl}
		$\lambda$ &$\bnfis$& $\tau \bnfbar \bactout{s}{s'} \bnfbar \bactout{s}{\abs{x} P} \bnfbar\bactinp{s}{s} \bnfbar \bactinp{s}{\abs{x} P}$ \\
		&	$\bnfbar$ & $\bactsel{s}{l} \bnfbar \bactbra{s}{l} \bnfbar \news{\tilde{s}} \bactout{s}{s'} \bnfbar \news{\tilde{s}} \bactout{s}{\abs{x} P}$\\
		&$\bnfbar$ &	\dk{$\bactout{s}{\tilde{s}} \bnfbar \bactout{s}{\abs{\tilde{x}} P} \bactinp{s}{\tilde{s'}} \bnfbar \bactinp{s}{\abs{\tilde{x}} P}$}
\end{tabular}

The internal action is defined on label $\tau$.
Action $\bactout{s}{s'}$ denotes the sending of name $s'$ over channel $s$.
Similarly action $\bactout{s}{\abs{x}{P}}$ is the sending of abstraction $\abs{x}{P}$
over channel $s$. Dually actions for the reception of names and abstractions are
$\bactinp{s}{s'}$ and $\bactinp{s}{\abs{x}{P}}$ respectively. We also defined
actions for selecting a label $l$, $\bactsel{s}{l}$ and branching on a label
$l$, $\bactbra{s}{l}$. When output actions carry name restrictions a scope
opening is implied.

We define the notion of dual actions as the symmetric relation $\asymp$, that satisfies the rules:
\[
	\bactsel{s}{l} \asymp \bactbra{\dual{s}}{l} \qquad \news{\tilde{s}} \bactout{s}{\abs{x} P} \asymp \bactinp{\dual{s}}{\abs{x} P} \qquad \bactout{s}{s'} \asymp \bactinp{\dual{s}}{s'}
\]

Dual actions happen on subjects that are dual between them; they carry the same
object; and furthermore output action is dual with input action and 
select action is dual with branch action.

\paragraph{Untyped Labelled Transition System}

\begin{figure}
	\[
	\begin{array}{c}
		\bout{n}{\tilde{m}} P \by{\bactout{n}{\tilde{m}}} P\ \ltsrule{OutN}
		\qquad
		\binp{n}{\tilde{x}} P \by{\bactinp{n}{\tilde{m}}} P\subst{\tilde{m}}{\tilde{x}}\ \ltsrule{InN}
		\qquad
		\bout{n}{\abs{x}{Q}} P \by{\bactout{n}{\abs{x}{Q}}} P\ \ltsrule{OutA}
		\\[4mm]

		\binp{n}{X} P \by{\bactinp{n}{\abs{x}{Q}}} P\subst{\abs{x}Q}{X}\ \ltsrule{InA}
		\qquad
		\bsel{n}{l}{P} \by{\bactsel{n}{l}} P \ltsrule{Sel}
		\qquad
		\tree{
			j \in I
		}
		{
			\bbra{n}{l_i:P_i}_{i \in I} \by{\bactbra{n}{l_j}} P_j
		}\ \ltsrule{Bra}
		\\[6mm]

		\tree{
			P \by{\ell} P' \quad n \notin \fn{\ell}
		}{
			\news{n} P \by{\ell} \news{n} P' 
		}\ \ltsrule{Res}
		\qquad
		\tree{
			P \scong_\alpha P'' \quad P'' \by{\ell} P'
		}{
			P \by{\ell} P'
		}\ \ltsrule{Alpha}
		\qquad
		\tree{
			P \by{\news{\tilde{s}} \bactout{n}{\tilde{m}}} P' \quad s \in \tilde{m}
		}{
			\news{s} P \by{\news{s\cat\tilde{s}} \bactout{n}{\tilde{m}}} P'
		}\ \ltsrule{ScopeN}
		\\[6mm]

		\tree{
			P \by{\news{\tilde{m}} \bactout{n}{\abs{x} Q}} P' \quad s \in \fn{\abs{x} Q}
		}{
			\news{s} P \by{\news{s\cat\tilde{m}} \bactout{n}{\abs{x} Q}} P'
		}\ \ltsrule{ScopeA}
		\qquad
		\tree{
			P \by{\ell_1} P' \qquad Q \by{\ell_2} Q' \qquad \ell_1 \asymp \ell_2
		}{
			P \Par Q \by{\tau} \newsp{\bn{\ell_1} \cup \bn{\ell_2}}{P' \Par Q'}
		}\ \ltsrule{Tau}
		\\[6mm]

		\tree{

			P \by{\ell} P' \quad \bn{\ell} \cap \fn{Q} = \es
		}{
			P \Par Q \by{\ell} P' \Par Q
		}\ \ltsrule{LPar}
		\qquad
		\tree{
			Q \by{\ell} Q' \quad \bn{\ell} \cap \fn{P} = \es
		}{
			P \Par Q \by{\ell} P \Par Q'
		}\ \ltsrule{RPar}
%		\\[6mm]
	\end{array}
	\]
	\caption{The untyped Labelled Transition System \label{fig:untyped_LTS}}
\end{figure}


The labelled transition system, LTS, is defined in Figure~\ref{fig:untyped_LTS}.
A process with a send prefix can interact with the environment with a send
action that carries a name $s'$ or an abstraction $\abs{x}{Q}$. Dually
on a received prefixed process we can observe a receive action of a name or
an abstraction. Select and branch prefixed processes can trigger select
and branch actions respectively. The LTS is closed under the name creation
operator provided that the restricted name does not occur free in the LTS action.
If the restricted name occurs free in the LTS action then we observe a bound name
action and the continuation process performs scope opening. Similarly the LTS 
is closed on the parallel operator provided that the LTS action does not shared
any bound names with parallel processes. If two parallale processes can perform
dual actions then the two actions can synchronise to observe an internal transition.
Finally the LTS is closed under structural congruence.


\paragraph{Labeled Transition System for Typed Environments}

We define a relation
$((\Gamma, \Lambda_1, \Sigma_1), \lambda, (\Gamma, \Lambda_2, \Sigma_2)) \in R$
over type tuples, that allows us to follow the progress of types over actions $\lambda$.

\begin{figure}
	\[
	\begin{array}{c}
		\tree{
			\dual{s} \notin \dom{\Sigma} \quad \Gamma; \Lambda_1; \Sigma_1 \proves V \hastype U \quad \Sigma_1 \subseteq \Sigma \quad \Lambda_1 \subseteq \Lambda
		}{
			(\Gamma; \Lambda; \Sigma \cat s: \btout{U} S) \by{\bactout{s}{V}} (\Gamma; \Lambda\backslash\Lambda_1; \Sigma\backslash\Sigma_1 \cat s: S)
		}
		\\[6mm]
		\tree{
			\dual{s} \notin \dom{\Sigma} \quad  \Gamma; \Lambda_1; \Sigma_1 \proves V \hastype U
		}{
			(\Gamma; \Lambda; \Sigma \cat s: \btinp{U} S) \by{\bactinp{s}{U}} (\Gamma; \Lambda \cup \Lambda_1; \Sigma \cup \Sigma_1 \cat s: S)
		}
		\\[6mm]
		\tree{
			\dual{s} \notin \dom{\Sigma} \quad k \in I
		}{
			(\Gamma; \Lambda; \Sigma \cat s: \btsel{l_i: S_i}_{i \in I}) \by{\bactsel{s}{l_k}} (\Gamma; \Lambda; \Sigma \cat s:S_k)
		}
		\quad
		\tree{
			\dual{s} \notin \dom{\Sigma} \quad k \in I
		}{
			(\Gamma; \Lambda; \Sigma \cat s: \btbra{l_i: T_i}_{i \in I}) \by{\bactbra{s}{l_k}} (\Gamma; \Lambda; \Sigma \cat s:S_k)
		}
		\\[6mm]

		\tree{
			(\Gamma; \Lambda_1; \Sigma_1) \by{\news{\tilde{s}} \bactout{s}{V}} (\Gamma; \Lambda_2; \Sigma_2)
	%		\quad S_1 \dualof S_2
		}{
			(\Gamma; \Lambda_1; \Sigma_1) \by{\news{s' \cat \tilde{s'}} \bactout{s}{V}} (\Gamma; \Lambda_2; \Sigma_2 \cat \dual{s'}: S)
		}
		\quad
		\tree{
			\Sigma_1 \red \Sigma_2
		}{
			(\Gamma; \Lambda; \Sigma_1) \by{\tau} (\Gamma; \Lambda; \Sigma_2)
		}
	\end{array}
	\]
	\caption{Labelled Transition Semantics for Typed Enviroments \label{fig:envLTS}}
\end{figure}


\dk{describe env LTS}

\paragraph{Typed Transition System}

The transition system over processes is defined as a combination
of the untyped LTS and the LTS for typed environments:

\begin{definition}[Typed Transition System]\rm
	We write $\Gamma; \emptyset; \Sigma_1 \proves P_1 \hastype \Proc \by{\lambda} \Gamma; \emptyset; \Sigma_1 \proves P_2 \hastype \Proc$
	whenever:
	\begin{itemize}
		\item	$P_1 \by{\lambda} P_2$.
		\item	$(\Gamma, \emptyset, \Sigma_1) \by{\lambda} (\Gamma, \emptyset, \Sigma_2)$.
	\end{itemize}
\end{definition}

For notational convenience we can write
$\Gamma; \emptyset; \Sigma_1 \by{\lambda} \Sigma_2 \proves P_1 \by{\lambda} P_2$,
instead of $\Gamma; \emptyset; \Sigma_1 \proves P_1 \hastype \Proc \by{\lambda} \Gamma; \emptyset; \Sigma_1 \proves P_2 \hastype \Proc$.
We extend to $\By{}$ and $\By{\hat{\lambda}}$ in the \dk{standard way}.

The next invariant clarifies the soundness of the
typed transition system.

\begin{lemma}[Invariant]
	\begin{itemize}
		\item	If $\Gamma; \emptyset; \Sigma_1 \proves P_1 \hastype \Proc$ and
			$P_1 \by{\lambda} P_2$ and $(\Gamma; \emptyset; \Sigma_1) \by{\lambda} (\Gamma; \emptyset; \Sigma_2)$
			then $\Gamma; \emptyset; \Sigma_2 \proves P_2 \hastype \Proc$.
	\end{itemize}
\end{lemma}

\begin{proof}
	\dk{TODO}
\end{proof}

\subsection{Behavioural Semantics}

We use the typed labelled transition semantics to define
a set of relations over typed processes that allow us to compare
typed processes over a notion of observational equivalence.


We begin with a definition of a notion of confluence
over session environments $\Sigma$:
\begin{definition}[Session Environment Confluence]\rm
	We denote $\Sigma_1 \bistyp \Sigma_2$ whenever $\exists \Sigma$ such that
	$\Sigma_1 \red^* \Sigma$ and $\Sigma_2 \red^* \Sigma$.
\end{definition}

%\jp{The following definition is a bit too "loose". Need to add conditions on $\Sigma_1,\Sigma_2$, and a better notation not involving the empty $\Lambda$.}

A typed relation is a relation over typed programs:

\begin{definition}[Typed Relation]\rm
	We say that
	$\Gamma; \emptyset; \Sigma_1 \proves P_1 \hastype \Proc\ R\ \Gamma; \emptyset; \Sigma_2 \proves P_2 \hastype \Proc$
	is a typed relation whenever
	\begin{itemize}
		\item	$P_1$ and $P_2$ are programs.
		\item	$\Sigma_1$ and $\Sigma_2$ are well typed.
		\item	$\Sigma_1 \bistyp \Sigma_2$.
	\end{itemize}
\end{definition}

We require that relate only programs (i.e.\ processes with no free variables) with
well typed session environments and furthermore we require that two related processes
have confluent session environments.

For notational convenience we write $\Gamma; \emptyset; \Sigma_1\ R\ \Sigma_2 \proves P_1\ R\ P_2$
for expressing the typed relation $\Gamma; \emptyset; \Sigma_1 \proves P_1 \hastype \Proc\ R\ \Gamma; \emptyset; \Sigma_2 \proves P_2 \hastype \Proc$.

We define the notions of barb and typed barb.

\begin{definition}[Barbs]\rm
	Let program $P$.
	\begin{enumerate}
%		\item	We write $P \barb{s}$ if $P \scong \newsp{\tilde{s}}{\bout{s}{\abs{x} P_1} P_2 \Par P_3}, s \notin \tilde{s}$.
%			We write $P \Barb{s}$ if $P \red^* \barb{s}$.

		\item	We write $P \barb{s}$ if $P \scong \newsp{\tilde{s}}{\bout{s}{V} P_2 \Par P_3}, s \notin \tilde{s}$.
			We write $P \Barb{s}$ if $P \red^* \barb{s}$.

		\item	We write $\Gamma; \emptyset; \Sigma \proves P \barb{s}$ if
			$\Gamma; \emptyset; \Sigma \proves P \hastype \Proc$ with $P \barb{s}$ and $\dual{s} \notin \Sigma$.
			We write $\Gamma; \emptyset; \Sigma \proves P \Barb{s}$ if $P \red^* P'$ and
			$\Gamma; \emptyset; \Sigma' \proves P' \barb{s}$.			
	\end{enumerate}
\end{definition}

A barb $\barb{s}$ is an observable on an output prefix with subject $s$.
Similarly a weak barb $\Barb{s}$ is a barb after a number of reduction steps.
Typed barbs $\barb{s}$ (resp.\ $\Barb{s}$)
happen on typed processes $\Gamma; \emptyset; \Sigma \proves P \hastype \Proc$
where we require that the corresponding dual endpoint $\dual{s}$ is not present
in the session type $\Sigma$.

We define the notion of the context:

\begin{definition}[Context]\rm
	$C$ is a context defined on the grammar:

	\begin{tabular}{rcl}
		$C$ &$=$& $\hole \bnfbar P \bnfbar \bout{k}{V} C \bnfbar \binp{k}{X} C \bnfbar \binp{k}{x} C \bnfbar \news{s} C \bnfbar C \Par C \bnfbar \bsel{k}{l} C \bnfbar \bbra{k}{l_i:C_i}_{i \in I}$
	\end{tabular}
	Notation $\context{C}{P}$ replaces every $\hole$ in $C$ with $P$.
\end{definition}

A context is a function that takes a process and returns a new process
according to the above syntax.

%We extend the notion of context to the notion of typed context:
%\begin{definition}[Typed Context]
%	Let program $\Gamma; \emptyset; \Sigma \proves P \hastype \Proc$ then	
%\end{definition}

The first equivalence relation we define is reduction-closed, barbed congruence:
\begin{definition}[Reduction-closed, Barbed Congruence]\rm
	Typed relation $\Gamma; \emptyset; \Sigma_1\ R\ \Sigma \proves P_1 \ R\ P_2$ is a barbed congruence
	whenever:
	\begin{enumerate}
		\item
		\begin{itemize}
			\item	If $P_1 \red P_1'$ then $\exists P_2', P_2 \red^* P_1'$ and $\Gamma; \emptyset; \Sigma_1' \proves P_1'\ R\ \Gamma; \emptyset; \Sigma_2' \proves P_2' \hastype \Proc$.
			\item	If $P_2 \red P_2'$ then $\exists P_1', P_1 \red^* P_1'$ and $\Gamma; \emptyset; \Sigma_1' \proves P_1'\ R\ \Gamma; \emptyset; \Sigma_2' \proves P_2' \hastype \Proc$.
		\end{itemize}
		\item
		\begin{itemize}
			\item	If $\Gamma;\emptyset;\Sigma \proves P_1 \barb{s}$ then $\Gamma;\emptyset;\Sigma \proves P_2 \Barb{s}$.
			\item	If $\Gamma;\emptyset;\Sigma \proves P_2 \barb{s}$ then $\Gamma;\emptyset;\Sigma \proves P_1 \Barb{s}$.
		\end{itemize}
		\item	$\forall C, \Gamma; \emptyset; \Sigma_1'\ R\ \Sigma_2' \proves \context{C}{P_1}\ R\ \context{C}{P_2}$.
	\end{enumerate}
	The largest such congruence is denoted with $\cong$.
\end{definition}

Reduction-closed, barbed congruence has closed reduction semantics and 
preserves barbs under any context. In a sense no barb observer can distinguish
between two related processes.

We can use a session type to define the simplest process that is typed
under the given session type.
\begin{definition}[Simple Process]
	Let session type $S$ then we define a process $\map{S}^{x}$:
	\[
	\begin{array}{l}
		\map{\tinact}^{x} = \inact \qquad \map{\btinp{S'} S}^{x} = \binp{x}{y} (\map{S}^{x} \Par \map{S'}^{y}) \qquad
		\map{\btout{U} S}^{x} = \binp{x}{\dk{\map{U}}} \map{S}^{x}\\
		\map{\btsel{l : S}}^{x} = \bsel{x}{l} \map{S}^{x} \qquad \map{\btbra{l_i: S_i}_{i \in I}}^{x} = \bbra{x}{l_i: \map{S_i}^{x}}_{i \in I}\\
		\map{\tvar{t}}^{x} = \rvar{T} \qquad \map{\trec{t}{S}}^{x} = \recp{T}{\map{S}^{x}}
	\end{array}
	\] 
\end{definition}

The second equivalent relation is a bisimulation relation called
contextual bisimulation:
\begin{definition}[Contextual Bisimulation]\rm
	Let typed relation $\mathcal{R}$ such that $\Gamma; \emptyset; \Sigma_1\ \mathcal{R}\ \Sigma_2 \proves P_1\ \mathcal{R}\ Q_1$.
	$\mathcal{R}$ is a {\em contextual bisimulation} whenever:
	\begin{enumerate}
		\item	$\forall \news{\tilde{s}} \bactout{s}{\abs{x} P}$ such that
			\[
				\Gamma; \emptyset; \Sigma_1 \by{\news{\tilde{s}} \bactout{s}{\abs{x} P}} \Sigma_1' \proves P_1 \by{\news{\tilde{s}} \bactout{s}{\abs{x} P}} P_2
			\]
			$\exists Q_2, \abs{x}{Q}$ such that
			\[
				\Gamma; \emptyset; \Sigma_2 \By{\news{\tilde{s'}} \bactout{s}{\abs{x} Q}} \Sigma_2' \proves Q_1 \By{\news{\tilde{s'}} \bactout{s}{\abs{x} Q}} Q_2
			\]
			and $\forall C, s'$, %such that
%			\begin{eqnarray*}
%				\Gamma; \emptyset; \Sigma_1'' \proves \newsp{\tilde{s}}{P_2 \Par \context{C}{P \subst{s'}{x}}} \hastype \Proc \\
%				\Gamma; \emptyset; \Sigma_2'' \proves \newsp{\tilde{s}}{Q_2 \Par \context{C}{Q \subst{s'}{x}}} \hastype \Proc
%			\end{eqnarray*}
			then
			\[
				\Gamma; \emptyset; \Sigma_1''\ \mathcal{R}\ \Sigma_2'' \proves \newsp{\tilde{s}}{P_2 \Par \context{C}{P \subst{s'}{x}}}\ \mathcal{R}\ 
				\newsp{\tilde{s'}}{Q_2 \Par \context{C}{Q \subst{s'}{x}}}
			\]
		\item	$\forall \news{\tilde{s}} \bactout{s}{s_1}$ such that
			\[
				\Gamma; \emptyset; \Sigma_1 \by{\news{\tilde{s}} \bactout{s}{s_1}} \Sigma_1' \proves P_1 \by{\news{\tilde{s}} \bactout{s}{s_1}} P_2
			\]
			$\exists Q_2, s_2$ such that
			\[
				\Gamma; \emptyset; \Sigma_2 \By{\news{\tilde{s'}} \bactout{s}{s_2}} \Sigma_2' \proves Q_1 \By{\news{\tilde{s'}} \bactout{s}{s_2}} Q_2
			\]
			and $\forall R$ with $\set{x} = \fn{R}$, %such that
%			\begin{eqnarray*}
%				\Gamma; \emptyset; \Sigma_1'' \proves \newsp{\tilde{s}}{P_2 \Par \context{C}{P \subst{s'}{x}}} \hastype \Proc \\
%				\Gamma; \emptyset; \Sigma_2'' \proves \newsp{\tilde{s}}{Q_2 \Par \context{C}{Q \subst{s'}{x}}} \hastype \Proc
%			\end{eqnarray*}
			then
			\[
				\Gamma; \emptyset; \Sigma_1''\ \mathcal{R}\ \Sigma_2'' \proves \newsp{\tilde{s}}{P_2 \Par R \subst{s_1}{x}}\ \mathcal{R}\ 
				\newsp{\tilde{s'}}{Q_2 \Par R \subst{s_2}{x}}
			\]

		\item	$\forall \lambda \notin \set{\news{\tilde{s}} \bactout{s}{s'}, \news{\tilde{s}} \bactout{s}{\abs{x} P}}s$ such that
			\[
				\Gamma; \emptyset; \Sigma_1 \by{\lambda} \Sigma_1' \proves P_1 \by{\lambda} P_2
			\]
			$\exists Q_2$ such that 
			\[
				\Gamma; \emptyset; \Sigma_2 \by{\lambda} \Sigma_2' \proves Q_1 \By{\hat{\lambda}} Q_2
			\]
			and
			$\Gamma; \emptyset; \Sigma_1\ \mathcal{R}\ \Sigma_2 \proves P_2\ \mathcal{R}\ Q_2$.

		\item	The symmetric cases of 1, 2 and 3.
	\end{enumerate}
	The Knaster Tarski theorem ensures that the largest contextual bisimulation exists and is denoted by $\wb^c$.
\end{definition}





\section{The Notion of Typed Encoding}
\label{s:expr}
% !TEX root = main.tex

\newpage
\section{Typed Encodings}\label{s:expr}

In this section we present a study of the expressiveness 
of the sub-calculi of $\HOp$.

We first define the notion of calculus.
We extend notions proposed elsewhere (cf.~Gorla\cite{})
by explicitly considering a type structure and a type system.

\begin{definition}[Typed Calculus]\label{d:tcalculus}\rm
	A \emph{typed calculus} $\tyl{L}$ is defined as a tuple:
%
	\[
		\calc{L}{T}{\red}{\wb}{\proves}
	\]
%
	where $L$ and $T$ are sets of processes and types, respectively; %$T_1$ is the set of types;
	$\red$ and $\wb$ denote a reduction semantics 
	and a typed equivalence
	on processes, respectively. Finally, $\proves$ denotes a type system for processes in $L$.
\end{definition}

We notice that in this paper we shall always consider languages with the same type system.
In the following, when writing $\tyl{L}_i$ we tacitly assume the existence of appropriate 
$L_i$, $T_i$, $\red_i$, $\wb_i$, and $\proves_i$.
We first define the notion of encoding over typed calculi.

\begin{definition}[Typed Encoding]\rm
	Let  $\tyl{L}_1$ % = \calc{L_1}{T_1}{\red_1}{\wb_1}{\proves_1}$
	and $\tyl{L}_2$ % =  \calc{L_2}{T_2}{\red_2}{\wb_2}{\proves_2}$ 
	be typed calculi.% as in Definition~\ref{d:tcalculus}.
	Given mappings $\map{\cdot}: L_1 \to L_2$ and
	$\mapt{\cdot}: T_1 \to T_2$, 
	we write 
	%$\enc{\cdot}{\cdot}: \calc{L_1}{T_1}{\red_1}{\wb_1}{\proves_1} \longrightarrow \calc{L_2}{T_2}{\red_2}{\wb_2}{\proves_2}$
%	for the encoding from $\calc{L_1}{T_1}{\red_1}{\wb_1}{\proves_1}$ to $\calc{L_2}{T_2}{\red_2}{\wb_2}{\proves_2}$.
	\[
		\enc{\cdot}{\cdot} : \tyl{L}_1 \to \tyl{L}_2
	\]
	to denote the \emph{typed encoding} of $\tyl{L}_1$ into $\tyl{L}_2$.
\end{definition}

\subsection{Encoding Properties}

We require that a {\em good} encoding should 
preserve not only the syntax but
also the operational, typing and behavioural
semantics. 

% ----> DK: The next notation is already defined
%\begin{notation}[Typed Equivalence]\rm
%	Let $P$ and $Q$ be two well-typed processes, i.e., 
%	there exist $\Gamma, \Sigma_1, \Sigma_2$ such that 
%	$\Gamma; \emptyset; \Sigma_1 \proves P \hastype \Proc$ 
%	and
%	$\Gamma; \emptyset; \Sigma_2 \proves Q \hastype \Proc$.
%	Then, to denote the fact that 
%	$P$ and $Q$ are related by behavioral equivalence $\wb$, we shall write
%	%Then we adopt the following notational convention:
%	\[
%		\Gamma; \Sigma_1 \wb \Sigma_2 \proves P \wb Q.
%	\]
%\end{notation}

\begin{definition}[Semantic Preserving Encoding]\rm
	\label{def:ep}
	We say that $\enc{\cdot}{\cdot} : \tyl{L}_1 \to \tyl{L}_2$ is a \emph{semantic preserving encoding}
	if it satisfies the following properties:
	%Let $\Gamma; \emptyset; \Sigma \proves P \hastype \Proc$ 
	%a process from calculus $\calc{L_1}{T_1}{\red_1}{\wb_1}{\proves_1}$
	%and an encoding 
	%$\enc{\cdot}{\cdot}: \calc{L_1}{T_1}{\red_1}{\wb_1}{\proves_1} \longrightarrow \calc{L_2}{T_2}{\red_2}{\wb_2}{\proves_2}$.
	
	\begin{enumerate}[1.]
		\item \emph{Type preservation}:	%We say that $\enc{\cdot}{\cdot}$ is \emph{type preserving}
		if
			$\Gamma; \emptyset; \Sigma \proves_1 P \hastype \Proc$ then $\mapt{\Gamma}; \emptyset; \mapt{\Sigma} \proves_2 \map{P} \hastype \Proc$ for any   $P$ in $L_1$.

		\item \emph{Operational Correspondence}: If $\Gamma; \emptyset; \Sigma \proves_1 P \hastype \Proc$ then
		\begin{enumerate}[-]
			\item	Completeness: If $P \red_1 P'$ then $\exists \Sigma'$ s.t.
				$\map{P} \Red_2 \map{P'}$ and
				$\mapt{\Gamma}; \emptyset; \mapt{\Sigma'} \proves_2 \map{P'} \hastype \Proc$.
			\item Soundness : If $\map{P} \red_2 Q$ then
				$\exists P'$ s.t. $P \red_1 P'$ and \\
				$\mapt{\Gamma}; \mapt{\Sigma_1} \wb_2 \mapt{\Sigma_2} \proves_2 \map{P'} \wb_2 Q$.
		\end{enumerate}
		
		\item \emph{Full Abstraction:}
		$\Gamma; \Sigma_1 \wb_1 \Sigma_2 \proves_1 P \wb_1 Q $ if and only if $\mapt{\Gamma}; \mapt{\Sigma_1} \wb_2 \mapt{\Sigma_2} \proves_2 \map{P} \wb_2 \map{Q} $.
	\end{enumerate}
\end{definition}


We show that the composition of encodings is closed on the above properties.

\begin{proposition}[Composability of Semantic Preserving Encodings]
	Let $\encod{\cdot}{\cdot}{1}: \tyl{L}_1 \to \tyl{L}_2$ and $\encod{\cdot}{\cdot}{2}: \tyl{L}_2 \to \tyl{L}_3$
	be two semantic preserving encodings.
	Then their composition, denoted 
	$\encod{\cdot}{\cdot}{1} \cdot \encod{\cdot}{\cdot}{2}: \tyl{L}_1 \to \tyl{L}_3$
	is also a semantic preserving encoding.
\end{proposition}

\begin{proof}
	Straightforward application of the definition of each property.
\end{proof}

\section{Positive Expressiveness Results}

\subsection{Languages Under Consideration}
We consider the following variants of \HOp:
\begin{enumerate}[-]
	\item	\HO: the second and third lines of the syntax of processes in Fig.~\ref{fig:syntax} (pure higher-order, monadic communication).
	\item	\sesp: the first and third lines of the syntax of processes in Fig.~\ref{fig:syntax} (first-order, monadic communication).
	\item	\sespnr: the finite sub-calculus of \sesp, i.e., name passing without recursion.
	\item	$\HO^{+\mathsf{p}}$: The polyadic \HO, i.~e.\ without polyadicity (polyadic abstraction/application).
	\item	$\sesp^{+\mathsf{p}}$: The polyadic \sesp, i.~e.\ with polyadicity (name passing)
	\item	$\HOp^{-\mathsf{p}}$: The monadic \HOp.
%	\item \pHOpnr: the finite variant of \pHOp 
%	\item \psesp: the variant of \sesp with polyadic communication.
%	\item \psespnr: the finite variant of \psesp with polyadic communication.
\end{enumerate}
\noindent
In the following we write $\pmap{\cdot}{i}$
and $\tmap{\cdot}{i}$ 
for mappings of processes and types, respectively.
Since we always consider variants and fragments of \HOp, the 
reduction semantics $\red$, the typed behavioral equivalence $\wb$,
and the type system $\proves$ are the same for all languages.

\subsection{Encoding \sespnr  into \HO}

The semantics of the $\HO$ are powerful enough to
express the semantics of the standard $\sesp$ calculus.

The name passing semantics of $\sesp$ have a rather straightforward
encoding from to $\HO$.
On the other hand to achieve the encoding of the recursion semantic
of $\sesp$, we need to extend
to the polyadic version of $\sesp$ as an intermediate step in order
to give a sound encoding of the recursion semantics to $\HO$.

We first encode the name passing semantics.
%Below, we use $n$ to stand for either a linear channel $k'$ or a shared name $a$.

\begin{definition}[\sespnr  into \HO]\rm
	Define $\encod{\cdot}{\cdot}{1}: \sespnr \to \HO$  as follows:
	\[
	\begin{array}{rcl}
		\pmap{\bout{k}{k'} P}{1}		&\defeq&	\bbout{k}{ \abs{z}{\,\binp{z}{X} \appl{X}{k'}} } \pmap{P}{1} \\
		\pmap{\binp{k}{x} Q}{1}			&\defeq&	\binp{k}{X} \newsp{s}{\appl{X}{s} \Par \bbout{\dual{s}}{\abs{x} \pmap{Q}{1}} \inact} \\
		\tmap{\btout{S_1} {S} }{1}		&\defeq&	\bbtout{\lhot{\btinp{\lhot{\tmap{S_1}{1}}}\tinact}} \tmap{S}{1}  \\
		\tmap{\btinp{S_1} S }{1}		&\defeq&	\bbtinp{\lhot{\btinp{\lhot{\tmap{S_1}{1}}}\tinact}} \tmap{S}{1} \\
		\tmap{\bbtout{\chtype{S_1}}{S}}{1}	&\defeq&	\bbtout{\shot{\btinp{\shot{\chtype{\tmap{S_1}{1}}}}\tinact}} \tmap{S}{1}  \\
		\tmap{\bbtinp{\chtype{S_1}}{S}}{1}	&\defeq&	\bbtinp{\shot{\btinp{\shot{\chtype{\tmap{S_1}{1}}}}\tinact}} \tmap{S}{1} 
	\end{array}
	\]
	where $\pmap{\cdot}{1}$ (resp. $\tmap{\cdot}{1}$) is an 
	homomorphism for the other process (resp. type) constructs.
\end{definition}

In the higher-order setting, a name $k$ is being passed as an input
guarded abstraction. The input prefix receives an abstraction and
continues with the application of $k$ over the received abstraction.
On the reception side $\binp{s}{x} P$ 
the encoding develops a mechanism that will receive
the input guarded abstraction, apply it on a fresh endpoint $s$ and use
the dual endpoint $\dual{s}$ to send the continuation $P$ as the abstraction
$\abs{x}{P}$. Name substitution is then achieved as application.

\begin{proposition}\rm
	Encoding $\encod{\cdot}{\cdot}{1}: \sespnr \to \HO$  is type-preserving (cf. Def.~\ref{def:ep}\,(1)).\rm
\end{proposition}

\begin{proof}
	Proof in Appendix~\ref{app:enc_sesspnr_to_ho_typing}.
	\qed
\end{proof}

\begin{comment}
\begin{proof}
	By induction on the structure of \sesp process $P$.
%
	\begin{enumerate}[1.]
		%%%% Output of (linear) channel
		\item	Case $P = \bout{k}{n}P'$. There are two sub-cases.
			In the first sub-case $n = k'$ (output of a linear channel). Then  
			we have the following typing in the source language:
			{\small
			\[
				\tree{
					\Gamma; \emptyset; \Sigma \cat k:S  \proves  P' \hastype \Proc \quad \Gamma ; \emptyset ; \{k' : S_1\} \proves  k' \hastype S_1}{
					\Gamma; \emptyset; \Sigma \cat k':S_1 \cat k:\btout{S_1}S \proves  \bout{k}{k'} P' \hastype \Proc}
			\]
			}
			The corresponding typing in the target language is as follows --- we write $U_1$ to stand for $\lhot{\btinp{\lhot{\tmap{S_1}{1}}}\tinact}$:
			{\small
			\[
				\tree{
					\tree{}{\tmap{\Gamma}{1}; \emptyset ; \tmap{\Sigma}{1} \cat k:\tmap{S}{1} \proves \pmap{P'}{1} \hastype \Proc}
					~~
					\tree{
						\tree{
							\tree{
								\tree{
									\tree{}{\tmap{\Gamma}{1} ; \{X : \lhot{\tmap{S_1}{1}}\} ; \emptyset \proves \X  \hastype \lhot{\tmap{S_1}{1}}} 
									\quad 
									\tree{}{\tmap{\Gamma}{1} ; \emptyset ; \{k' : \tmap{S_1}{1}\} \proves  k' \hastype \tmap{S_1}{1}}}{\tmap{\Gamma}{1} ; \{X : \lhot{\tmap{S_1}{1}}\} ; k' : \tmap{S_1}{1} \proves \appl{\X}{k'} \hastype \Proc}}{\tmap{\Gamma}{1} ; \{X : \lhot{\tmap{S_1}{1}}\} ; k' : 	\tmap{S_1}{1} \cat z:\tinact \proves \appl{\X}{k'} \hastype \Proc}
						}{
							\tmap{\Gamma}{1} ; \emptyset; k' : \tmap{S_1}{1} \cat z:\btinp{\lhot{\tmap{S_1}{1}}}\tinact \proves \binp{z}{X} \appl{\X}{k'} \hastype \Proc
						}
					}{
						\tmap{\Gamma}{1} ; \emptyset; k' : \tmap{S_1}{1} \proves \abs{z}{\binp{z}{X} \appl{\X}{k'}} \hastype U_1
					}
				}{
				\tmap{\Gamma}{1}; \emptyset; \tmap{\Sigma}{1} \cat k':\tmap{S_1}{1} \cat k:\btout{U_1}\tmap{S}{1} \proves  \bbout{k}{\abs{z}{\binp{z}{X} \appl{\X}{k'}}} \pmap{P'}{1} \hastype \Proc
				}
			\]
			}
	
			In the second sub-case, we have $n = a$ (output of a shared name). Then  
			we have the following typing in the source language:
			{\small
			\[
				\tree{
					\Gamma \cat a:\chtype{S_1}; \emptyset; \Sigma \cat k:S  \proves  P' \hastype \Proc \quad \Gamma \cat a:\chtype{S_1} ; \emptyset ; \emptyset \proves  a \hastype S_1}{
					\Gamma \cat a:\chtype{S_1} ; \emptyset; \Sigma  \cat k:\bbtout{\chtype{S_1}}S \proves  \bout{k}{a} P' \hastype \Proc}
			\]
			}
			The typing in the target language is derived similarly as in the first sub-case. \\
	
		%%%% Input of (linear) channel 
		\item	Case $P = \binp{k}{x}Q$. We have two sub-cases, depending on the type of $x$.
			In the first case, $x$ stands for a linear channel.
			Then we have the following typing in the source language:
			{\small
			\[
			 \tree{
				 \Gamma; \emptyset; \Sigma  \cat k:S \cat x:S_1 \proves   Q \hastype \Proc
			 	}{
				\Gamma; \emptyset; \Sigma  \cat k:\btinp{S_1}S \proves  \binp{k}{x} Q \hastype \Proc}
			 \]
			 }
			 The corresponding typing in the target language is as follows --- we write $U_1$ to stand for $\lhot{\btinp{\lhot{\tmap{S_1}{1}}}\tinact}$:
			{\small  
			\[
			 \tree{
				 \tree{
				 	\tree{
					\tree{
					\begin{array}{c}
					\tmap{\Gamma}{1}; \{X: U_1\};   \emptyset \proves X \hastype U_1 \\
					\tmap{\Gamma}{1}; \emptyset;   \cat s: \btinp{\lhot{\tmap{S_1}{1}}}\tinact \ \proves s \, \hastype  \btinp{\lhot{\tmap{S_1}{1}}}
					\tinact 
					\end{array}
					}{
					\tmap{\Gamma}{1}; \{X: U_1\};   \cat s: \btinp{\lhot{\tmap{S_1}{1}}}\tinact \ \proves \appl{X}{s}  \hastype \Proc
					} \quad 
					\tree{
					\tree{
					\tmap{\Gamma}{1}; \emptyset;  \emptyset \proves   \inact  \hastype \Proc}{
					\tmap{\Gamma}{1}; \emptyset;  \dual{s}: \tinact\proves   \inact  \hastype \Proc
					}
					\quad 
					\tree{
					\tmap{\Gamma}{1}; \emptyset;  \tmap{\Sigma}{1} \cat k:\tmap{S}{1}  x:\tmap{S_1}{1} \proves \pmap{Q}{1}   \hastype \Proc	 }{
					\tmap{\Gamma}{1}; \emptyset;  \tmap{\Sigma}{1} \cat k:\tmap{S}{1}   \proves \abs{x} \pmap{Q}{1}   \hastype \lhot{\tmap{S_1}{1}}			}
					}{
					\tmap{\Gamma}{1}; \emptyset;  \tmap{\Sigma}{1} \cat k:\tmap{S}{1}  \cat \dual{s}: \btout{\lhot{\tmap{S_1}{1}}}\tinact\proves  \bbout{\dual{s}}{\abs{x} \pmap{Q}{1}} \inact  \hastype \Proc
					}
					}{
					\tmap{\Gamma}{1}; \{X: U_1\};  \tmap{\Sigma}{1} \cat k:\tmap{S}{1} \cat s: \btinp{\lhot{\tmap{S_1}{1}}}\tinact \cat \dual{s}: \btout{\lhot{\tmap{S_1}{1}}}\tinact\proves \appl{X}{s} \Par \bbout{\dual{s}}{\abs{x} \pmap{Q}{1}} \inact  \hastype \Proc
					}
					}{
				 \tmap{\Gamma}{1}; \{X: U_1\};  \tmap{\Sigma}{1} \cat k:\tmap{S}{1} \proves \newsp{s}{\appl{X}{s} \Par \bbout{\dual{s}}{\abs{x} \pmap{Q}{1}} \inact}  \hastype \Proc
				 }
				 }{
				\tmap{\Gamma}{1}; \emptyset; \tmap{\Sigma}{1}  \cat k:\btinp{U_1}\tmap{S}{1} \proves  \binp{k}{X} \newsp{s}{\appl{X}{s} \Par \bbout{\dual{s}}{\abs{x} \pmap{Q}{1}} \inact}  \hastype \Proc
				}
			 \]
			 }
			 
			 In the second sub-case, $x$ stands for a shared name. Then we have the following typing in the source language:
			{\small
			\[
			 \tree{
				 \Gamma \cat x:\chtype{S_1} ; \emptyset; \Sigma  \cat k:S \proves   Q \hastype \Proc
			 	}{
				\Gamma ; \emptyset; \Sigma  \cat k:\btinp{\chtype{S_1}}S \proves  \binp{k}{x} Q \hastype \Proc}
			 \]
			 }
			 The typing in the target language is derived similarly as in the first sub-case.	
\end{enumerate}
%
\qed
\end{proof}
\end{comment}

\begin{proposition}\rm
	Encoding $\encod{\cdot}{\cdot}{1}: \sespnr \to \HO$  enjoys operational correspondence (cf. Def.~\ref{def:ep}\,(2)).
\end{proposition}


\begin{proof}
	Proof in Appendix~\ref{app:enc_sesspnr_to_ho_oc}.
	\qed
\end{proof}

\begin{comment}
\begin{proof}[Sketch]
	We must show completeness and soundness properties. 
	For completeness, it suffices to consider source process
	$P_0 = \bout{k}{k'} P \Par \binp{k}{x} Q$. We have that
%
	\[
		P_0 \red P \Par Q\subst{k'}{x}.
	\]
%
	By the definition of encoding we have:
	\begin{eqnarray*}
		\pmap{P_0}{1} & = & \bbout{k}{ \abs{z}{\,\binp{z}{X} \appl{X}{k'}} } \pmap{P}{1} \Par \binp{k}{X} \newsp{s}{\appl{X}{s} \Par \bbout{\dual{s}}{\abs{x} \pmap{Q}{1}} \inact}  \\
		& \red & \pmap{P}{1} \Par \newsp{s}{\appl{X}{s} \subst{\abs{z}{\,\binp{z}{X} \appl{X}{k'}}}{X} \Par \bbout{\dual{s}}{\abs{x} \pmap{Q}{1}} \inact} \\
		& = & \pmap{P}{1} \Par \newsp{s}{\,\binp{s}{X} \appl{X}{k'} \Par \bbout{\dual{s}}{\abs{x} \pmap{Q}{1}} \inact} \\
		& \red & \pmap{P}{1} \Par \appl{X}{k'} \subst{\abs{x} \pmap{Q}{1}}{X} \Par \inact \\
		& \scong & \pmap{P}{1} \Par \pmap{Q}{1}\subst{k'}{x}  
	\end{eqnarray*}
	For soundness, it suffices to notice that the encoding does not add new visible actions:
	the additional synchronizations induced by the encoding always occur on private (fresh) names.
	We assume weak bisimilarities, which abstract from internal actions used by the encoding,
	and so  constructing a relation witnessing behavioral equivalence is easy.
	\qed
\end{proof}
\end{comment}

%\subsection{Polyadic Into Monadic}
%The encoding from $\psesp$ to $\sesp$ is easier than the
%encoding of polyadic $\pi$-calculus in the $\pi$-calculus because
%we have linear session endpoints.
%
%\begin{definition}[$\psesp$ to $\sesp$]
%	We write $\encod{\cdot}{\cdot}{2}:\psesp \to \sesp$ whenever
%
%	\begin{tabular}{c}
%			$\map{\bout{k}{k'_1, \cdots, k'_n} P}^{2} \defeq \bout{k}{k'_1} \cdots ;  \bout{k}{k'_n}
%			\pmap{P}{2}$\\
%			$\map{\binp{k}{x_1, \cdots, x_n} P}^{2} \defeq \binp{k}{x_1} \cdots ; \binp{k}{x_n}  \pmap{P}{2}$ \\
%			$\tmap{\btout{S_1, \cdots, S_n} S}{2} \defeq \bbtout{\tmap{S_1}{2}} \cdots; \bbtout{\tmap{S_n}{2}} \tmap{S}{2}$\\
%			$\tmap{\btinp{S_1, \cdots, S_n} S}{2} \defeq \bbtinp{\tmap{S_1}{2}} \cdots; \bbtinp{\tmap{S_n}{2}} \tmap{S}{2}$
%%		\end{tabular}
%%		& \quad &
%%		\begin{tabular}{l}
%%			$\tmap{\btout{S_1 \cat \tilde{S}} S}{2} \defeq \btout{S_1} \tmap{\btout{\tilde{S}} S}{2}$\\
%%			$\tmap{\btinp{S_1 \cat \tilde{S}} S}{2} \defeq \btinp{S_1} \tmap{\btinp{\tilde{S}} S}{2}$
%%		\end{tabular}
%	\end{tabular}
%\end{definition}
%
%Polyadic name sending (resp.\ receive) is encoded as sequence of
%send (resp.\ receive) operations. Linearity of session endpoints
%ensures no race conditions, thus the encoding is sound.
%
%The encoding of the polyadic $\sesp$ semantics is as simple as the
%composition of the two former encodings.
%
%\begin{definition}[Encoding from $\psespnr$ to $\HO$]
%	We define $\encod{\cdot}{\cdot}{3}: \psespnr \longrightarrow \HO$
%	as $\encod{\cdot}{\cdot}{3} = \encod{\cdot}{\cdot}{1} \cat \encod{\cdot}{\cdot}{2}$.	
%\end{definition}

%So far we have consider name abstractions and applications which are \emph{monadic}.
%We now consider the \emph{polyadic} extension of these constructs, %name abstractions and applications.
%written $\abs{x_1, \ldots, x_n} P$ and $\appl{X}{k_1, \ldots, k_n}$, respectively.
%Next we give the encoding from $\HOp$ with polyadic name abstraction to $\HOp^{p}$.
%
%\begin{definition}[Encoding from $\pHOpnr$ to $\pHOp$]
%
%	\begin{tabular}{lcl}
%		$\map{\bout{k}{\abs{\tilde{x}} P_1} P_2}^4$ &$\defeq$& $\bout{k}{\abs{z} \binp{z}{\tilde{x}} \map{P_1}^4} \map{P_2}^4$\\
%		$\map{\appl{X}{\tilde{k}}}$ &$\defeq$& $\newsp{s}{\appl{X}{s} \Par \bout{\dual{s}}{\tilde{k}} \inact}$
%	\end{tabular}
%\end{definition}

%We compose the latter encoding with the generalisation $\map{\cdot}^3 : \HOp^{p-\mu} \longrightarrow \HO$
%of the encoding $\map{\cdot}^3 : \sesp^{p-\mu} \longrightarrow \HO$ to get a translation
%of $\HOp^{pa-\mu}$ to $\HO$.
%
%\begin{definition}[Encoding from $\HOp^{pa-\mu}$ to $\HO$]
%	We define $\encod{\cdot}{\cdot}{5}: \HOp^{pa-\mu} \longrightarrow \HO$
%	as $\encod{\cdot}{\cdot}{5} = \encod{\cdot}{\cdot}{4} \cat \encod{\cdot}{\cdot}{3}$.	
%\end{definition}

\subsection{Encode Polyadic Semantics (\HOp) to Monadic Semantics ($\HOp^{-\mathsf{p}}$)}

%In the extension of \HOp with 
%polyadic communication, denoted \pHOp, 
%one may pass in each synchronization 
%a tuple of values of length $n$, rather than a just single value.
%Thus, e.g., for $n = 2$ one would have
%%
%\begin{eqnarray*}
%	\bout{n}{m_1, m_2} P \Par \binp{\dual{n}}{x_1,x_2} Q  & \red &  P \Par Q \subst{m_1, m_2}{x_1, x_2} \\
%	\bout{n}{\abs{x_1, x_2}{P_1}} P \Par \binp{\dual{s}}{\X} Q & \red & P \Par Q \subst{\abs{x_1,x_2}{P_1}}{\X}
%\end{eqnarray*}
%%
%with $\appl{X}{k_1,k_2} \subst{\abs{x_1,x_2}{Q}}{\X}  =  Q \subst{k_1,k_2}{x_1,x_2} $.
%Thus, 
%\pHOp features tuple passing in intra-session (linear) communication,
%but also in abstractions/applications. 
%The session type system for \pHOp is an orthogonal
%extension of that in \S\,\ref{s:types}.
%The type syntax for values is extended as follows,
%where $\tilde{S}$ stands for a sequence $S_1, \ldots, S_n$ of 
%session types:
%
%\begin{eqnarray*}
%	U \bnfis  & \tilde{S} \bnfbar \lhot{\tilde{S}} \bnfbar \shot{\tilde{S}} \bnfbar \chtype{S}
%\end{eqnarray*}
%
%The syntax of session types would be kept unchanged.
%Typing rules require straightforward extensions. 
%For instance, the following rules would type 
%abstraction and application in the biadic case ($n = 2$):
%\[
%\trule{Abs2}~~\tree{
%			\Gamma; \Lambda; \Sigma \cat x_1: S_1, x_2: S_2 \proves P \hastype \Proc
%		}{
%			\Gamma; \Lambda; \Sigma \proves \abs{x_1, x_2}{P} \hastype \lhot{(S_1, S_2)}
%		}
%		\quad
%		\trule{App2}~~\tree{
%		\begin{array}{c}
%		(U = \lhot{(S_1,S_2)}) \lor (U = \shot{(S_1,S_2)}) \\
%		\Gamma; \Lambda; \Sigma \proves X \hastype U  \\
%		\Gamma; \Lambda_1; \Sigma_1 \proves k_1 \hastype S_1 \quad 		
%		\Gamma; \Lambda_2; \Sigma_2 \proves k_2 \hastype S_2
%		\end{array}
%		}{
%			\Gamma; \Lambda \cup  \Lambda_1 \cup \Lambda_2; \Sigma \cup \Sigma_1 \cup \Sigma_2 \proves \appl{X}{k_1,k_2} \hastype \Proc
%		} 
%\]

In the untyped $\pi$-calculus, polyadic communication
can be encoded into monadic name passing 
simply by performing $n$ monadic synchronizations on a fresh channel. 
In session-typed $\pi$-calculi this encoding is even simpler, 
thanks to the linearity of session endpoints~\cite{VascoFun}.
%The extension of the (monadic) session type system given in \S\,\ref{s:types}
%to handle polyadic communication is straightforward and follow expected lines.
%For this reason, we do not present a typing system for \pHOp in full detail; rather, 
%we shall define a syntactic transformation of \pHOp into \HOp, rather than as a typed encoding.
%\footnote{The definition of a polyadic semantics would only add visual clutter to our presentation,
%as all results extend easily from monadic to polyadic communication.}
We give the definition of the encoding of polyadic semantics to monadic semantics.
Because of the polyadic to monadic encoding %, denoted  $\auxmap{\cdot}{\mathsf{p}}$,
we are able to focus on monadic session processes,
and rely on polyadic constructs simply as convenient syntactic sugar.
In fact, we shall rely on polyadicity to encode recursive behaviors.
%
\begin{definition}[Polyadic Into Monadic]\rm
	Define the process mapping $\encod{\cdot}{\cdot}{\mathsf{p}}: \HOp \to \HOp^{-\mathsf{p}}$:
	%$\auxmap{\cdot}{\mathsf{p}}:\pHOp \to \HOp$ as
\[
	\begin{array}{rcl}
		\map{\bout{k}{k_1, \cdots, k_n} P}{\mathsf{p}}
		&\defeq&
		\bout{k}{k_1} \cdots ;  \bout{k}{k_n} \map{P}{\mathsf{p}}
		\\

		\map{\binp{k}{x_1, \cdots, x_n} P}{\mathsf{p}}
		&\defeq&
		\binp{k}{x_1} \cdots ; \binp{k}{x_n}  \map{P}{\mathsf{p}}
		\\

		\map{\bbout{k}{\abs{x_1, \cdots, x_n} Q} P}{\mathsf{p}}
		&\defeq&
		\bbout{k}{\abs{z}\binp{z}{x_1} \cdots ; \binp{z}{x_n} \map{Q}{\mathsf{p}}} \map{P}^{\mathsf{p}}
		\\

		\map{\appl{X}{k_1, \cdots, k_n}}{\mathsf{p}}
		&\defeq&
		\newsp{s}{\appl{X}{s} \Par \bout{\dual{s}}{k_1} \cdots ; \bout{\dual{s}}{k_n} \inact} 
	\end{array}
	\]
	and as an homomorphism for the remaining constructs in \HOp. 
	Define the mapping on types $\tmap{\cdot}{\mathsf{p}}$ as follows:
\[
	\begin{array}{rcl}
		\tmap{\btout{S_1, \cdots, S_n}S}{\mathsf{p}}
		&\defeq&
		\btout{\tmap{S_1}{\mathsf{p}}} \cdots \btout{\tmap{S_n}{\mathsf{p}}}\tmap{S}{\mathsf{p}}
		\\
		\tmap{\btinp{S_1, \cdots, S_n}S}{\mathsf{p}}
		&\defeq&
		\btinp{\tmap{S_1}{\mathsf{p}}} \cdots \btinp{\tmap{S_n}{\mathsf{p}}}\tmap{S}{\mathsf{p}}
		\\
		\tmap{\lhot{(S_1, \cdots, S_n)}}{\mathsf{p}}
		&\defeq&
		\lhot{\big(\btinp{\tmap{S_1}{\mathsf{p}}} \cdots \btinp{\tmap{S_n}{\mathsf{p}}}\tinact\big)}
		\\
		\tmap{\shot{(S_1, \cdots, S_n)}}{\mathsf{p}}
		&\defeq&
		\shot{\big(\btinp{\tmap{S_1}{\mathsf{p}}} \cdots \btinp{\tmap{S_n}{\mathsf{p}}}\tinact\big)}
	\end{array}
\]
	and as an homomorphism for the remaining type constructs.
	%\jp{I prefer to be explicit in the encoding of polyadic abstraction/applications. Previous version is commented.}
\end{definition}
%
Passing a list of names over session channels is established
with a corresponding list of sequential send (resp. receive) prefixes.
When we are dealing with an abstraction over a list of bound variables,
then we create a new abstraction name and we use it to receive in a polyadic
way the list of names on the abstraction. Similarly application will instantiate
the abstraction subject with a new session name and will use it 
to send the list of names that are going to be applied on the abstraction.
Note that we do not allow polyadic mapping on shared names.
The polyadic mapping, as presented here, is sound only on session names.
%The semantics might break if we apply this mapping on shared names.

%\begin{proposition}
%	$\Gamma; \emptyset; \Sigma \proves \map{P}^{p} \hastype \Proc$
%\end{proposition}


\subsection{Encoding Recursion into Abstraction Passing}

Encoding the constructs for recursion present in $\sesp$ as process-passing communication requires to follow the fundamental
principle of copying the process that needs to exhibit recursive behaviour.
The primitive recursor operation creates copies of a process and uses them
as continuations, e.g:
\[
	\recp{X}{\bout{n}{m} \rvar{X}} \scong \bout{n}{m} \recp{X}{\bout{n}{m} \rvar{X}}
\]
In the above example the scope of name $n$ includes the entire process so
the type for $n$ should be recursive. An alternative representation
of the above process would be:
\[
	\bout{a}{n} \inact \Par \recp{X}{\binp{a}{x} \bout{x}{m} (\bout{a}{x} \inact \Par \rvar{X})} \red \bout{n}{m} (\bout{a}{n} \inact \Par \recp{X}{\binp{a}{x} \bout{x}{m} (\bout{a}{x} \inact \Par \rvar{X}))}
\]
Endpoint $n$ is being passed sequentially on copies of the 
same process to achieve the effect of infinite sending of value $m$.
If we apply the same principles on higher order semantics we get:
\[
	\begin{array}{l}
		\newsp{s_1}{\bbout{s_1}{(z) \bout{n}{m} \binp{z}{X} \newsp{s_2}{\appl{X}{s_2} \Par \bout{\dual{s_2}}{(z) \appl{X}{z} } \inact} } \inact \Par \binp{\dual{s_1}}{X} \newsp{s_3}{\appl{X}{s_3} \Par \bout{\dual{s_3}}{(z) \appl{X}{z}} \inact } }
		\\
		\red
		\\
		\newsp{s_3}{\bout{n}{m} \binp{s_3}{X} \newsp{s_2}{\appl{X}{s_2} \Par \bout{\dual{s_2}}{(z) \appl{X}{z}} \inact} \Par \bout{\dual{s_3}}{(z) \bout{n}{m} \binp{z}{X} \newsp{s_2}{\appl{X}{s_2} \Par \bout{\dual{s_2}}{(z) \appl{X}{z} } \inact}  } \inact }
	\end{array}
\]
In the above encoding the abstraction
\[
	(z) \bout{n}{m} \binp{z}{X} \newsp{s_2}{\appl{X}{s_2} \Par \bout{\dual{s_2}}{(z) \appl{X}{z} } \inact}
\]
has a linear type due to the free occurrence of the session channel $n$.
But when passed, the latter abstraction is applied in a shared manner, i.e.\ two
copies of the abstraction are instantiated, thus the whole
encoding is untypable. The untypability problem would not exist
provided that the abstraction being passed were not linear.


A typable encoding of the example would be:
\[
	\begin{array}{l}
		\newsp{s_1}{\bout{s_1}{(z,x) \bout{x}{m} \binp{z}{X} \newsp{s_2}{\appl{X}{s_2,x} \Par \bout{\dual{s_2}}{(z,x) \appl{X}{z,x} } \inact} } \inact \Par \binp{\dual{s_1}}{X} \newsp{s_3}{\appl{X}{s_3, n} \Par \bout{\dual{s_3}}{(z,x) \appl{X}{z,x}} \inact } }
		\\
		\red
		\\
		\newsp{s_3}{\bout{n}{m} \binp{s_3}{X} \newsp{s_2}{\appl{X}{s_2, n} \Par \bout{\dual{s_2}}{(z, x) \appl{X}{z, x}} \inact} \Par \bout{\dual{s_3}}{(z, x) \bout{x}{m} \binp{z}{X} \newsp{s_2}{\appl{X}{s_2, x} \Par \bout{\dual{s_2}}{(z, x) \appl{X}{z, x} } \inact}  } \inact }
	\end{array}
\]

The abstraction now has become:
\[
	(z,x) \bout{x}{m} \binp{z}{X} \newsp{s_2}{\appl{X}{s_2,x} \Par \bout{\dual{s_2}}{(z,x) \appl{X}{z,x} } \inact}
\]
by replacing the free ocurrence of channel $n$ with variable $x$ and
bind $x$ as an abstraction variable. We can then instantiate
the above abstraction by passing session $n$ around following the same
principle as the name passing discipline.

A preliminary tool to encode the $\sesp$ recursion primitives would be to
provide a mapping from processes to processes with no free names.
We require some auxiliary definitions.
%
\begin{definition}\rm 
	Let $\vmap{\cdot}: 2^{\mathcal{N}} \longrightarrow \mathcal{V}^\omega$
	be a map of sequences of names to sequences of variables, defined
	inductively as follows:
%
\[
	\vmap{n} = x_n \qquad \qquad \qquad \vmap{n \cat \tilde{m}} = x_n \cat \vmap{\tilde{m}}
\]
\end{definition}

Given a process $P$, we write $\ofn{P}$ to denote the
\emph{sequence} of free names of $P$, lexicographically ordered.
Intuitively, the following mapping transforms processes
with free session names into abstractions:
%
\begin{definition}\label{d:trabs}\rm
	Let $\sigma$ be a set of session names.
	Define $\auxmapp{\cdot}{\mathsf{v}}{\sigma}: \HOp \to \HOp$  as follows
%
\[
	\begin{array}{rcl}
		\auxmapp{\news{n} P}{\sigma}{\mathsf{v}} &\bnfis& \news{n} \auxmapp{P}{\mathsf{v}}{{\sigma \cat n}}\\
		\auxmapp{\bout{n}{\abs{x} Q} P}{\mathsf{v}}{\sigma} &\bnfis&
		\left\{
		\begin{array}{rl}
			\bbout{x_n}{\abs{x,\vmap{ \ofn{P}}} \auxmapp{Q}{\mathsf{v}}{\sigma}} \auxmapp{P}{\mathsf{v}}{\sigma} & n \notin \sigma\\
			\bbout{n}{\abs{x,\vmap{\ofn{P}}} \auxmapp{Q}{\mathsf{v}}{\sigma}} \auxmapp{P}{\mathsf{v}}{\sigma} & n \in \sigma
		\end{array}
		\right.
		\\
		\auxmapp{\binp{n}{X} P}{\mathsf{v}}{\sigma} &\bnfis&
		\left\{
		\begin{array}{rl}
			\binp{x_n}{X} \auxmapp{P}{\mathsf{v}}{\sigma} & n \notin \sigma\\
			\binp{n}{X} \auxmapp{P}{\mathsf{v}}{\sigma} & n \in \sigma
		\end{array}
		\right.
		\\
		\auxmapp{\bsel{n}{l} P}{\mathsf{v}}{\sigma} &\bnfis&
		\left\{
		\begin{array}{rl}
			\bsel{x_n}{l} \auxmapp{P}{\mathsf{v}}{\sigma} & n \notin \sigma\\
			\bsel{n}{l} \auxmapp{P}{\mathsf{v}}{\sigma} & n \in \sigma
		\end{array}
		\right.
		\\
		\auxmapp{\bsel{n}{l} P}{\mathsf{v}}{\sigma} &\bnfis&
		\left\{
		\begin{array}{rl}
			\bsel{x_n}{l} \auxmapp{P}{\mathsf{v}}{\sigma} & n \notin \sigma\\
			\bsel{n}{l} \auxmapp{P}{\mathsf{v}}{\sigma} & n \in \sigma
		\end{array}
		\right.
		\\
		\auxmapp{\bout{n}{m}P}{\mathsf{v}}{\sigma} &\bnfis&
		\left\{
		\begin{array}{rl}
		    \bout{n}{m}\auxmapp{P}{\mathsf{v}}{\sigma} & n, m \in \sigma \\
		    \bout{x_n}{m}\auxmapp{P}{\mathsf{v}}{\sigma} & n \not\in \sigma, m \in \sigma \\
		    \bout{n}{x_m}\auxmapp{P}{\mathsf{v}}{\sigma} & n \in \sigma, m \not\in \sigma \\
		    \bout{x_n}{x_m}\auxmapp{P}{\mathsf{v}}{\sigma} & n, m \not\in \sigma 
		\end{array}
		\right.
		\\
		\auxmapp{\binp{n}{x}P}{\mathsf{v}}{\sigma} &\bnfis&
		\left\{
		\begin{array}{rl}
		    \binp{n}{x}\auxmapp{P}{\mathsf{v}}{\sigma} & n \in \sigma \\
		    \binp{x_n}{x}\auxmapp{P}{\mathsf{v}}{\sigma} & n \not\in \sigma 
		\end{array}
		\right.
		\\
		\auxmapp{\appl{\X}{n}}{\mathsf{v}}{\sigma} &\bnfis&
		\left\{
		\begin{array}{rl}
			\appl{\X}{x_n} & n \notin \sigma\\
			\appl{\X}{n} & n \in \sigma\\
		\end{array}
		\right. 
%		\auxmapp{\inact}{\mathsf{v}}{\sigma} &\bnfis& \inact\\
%		\auxmapp{P \Par Q}{\mathsf{v}}{\sigma} &\bnfis& \auxmapp{P}{\mathsf{v}}{\sigma} \Par \auxmapp{Q}{\mathsf{v}}{\sigma} 
	\end{array}
\]
and homomorphically for inaction and parallel composition.
\end{definition}

Given a process $P$ with $\ofn{P} = m_1, \cdots, m_n$, we are interested in its associated (polyadic) abstraction, which is defined as
$\abs{x_1, \cdots, x_n}{\auxmapp{P}{\mathsf{v}}{\es} }$, where $\vmap{m_j} = x_j$, for all $j \in \{1, \ldots, n\}$.
This transformation from processes into abstractions can be reverted by
using abstraction and application with an appropriate sequence of session names:
%
\begin{proposition}\rm
	Let $P$ be a \HOp process with $\tilde{n} = \ofn{P}$.
	Also, suppose $\tilde{x} = \vmap{\tilde{n}}$.
%	Also, let $A_P$ be the polyadic abstraction $\abs{\tilde{x}}\auxmapp{P}{\mathsf{v}}{\emptyset}$ (cf. Def.~\ref{d:trabs}).
	Then we have: $P \scong \appl{X}{\tilde{n}}\subst{\abs{\tilde{x}}\auxmapp{P}{\mathsf{v}}{\emptyset}}{X}$.
%	$\appl{X}{\smap{\fn{P}}} \subst{(\vmap{\fn{P}}) \map{P}^{\emptyset}}{X} \scong P$
\end{proposition}

\begin{proof}
	$\appl{X}{\smap{\fn{P}}} \subst{(\vmap{\fn{P}}) \map{P}^{\emptyset}}{X} =
	\map{P}^{\sigma} \subst{\smap{\fn{P}}}{\vmap{\fn{P}}}$ 
	\dk{TODO}
	\qed
\end{proof}

We are now ready to define the encoding of $\sesp$
(including constructs for recursion) into strict process-passing.
We stress that we use polyadicity in abstraction and application only
as syntactic sugar, to simplify presentation.
For the sake of completeness, we give again the encodings for 
finite processes and types, as
formalized by $\encod{\cdot}{\cdot}{1}: \sespnr \to \HO$.

\begin{definition}[From $\sesp$ to $\HO$]\rm
	Let $f$ be a function from recursion variables to sequences of name variables.
	Define $\fencod{\cdot}{\cdot}{2}{f}: \sesp \to \HO$ as
%
\[
	\begin{array}{rcll}
%	\map{\rec{X}{P}}^{2} &=& \newsp{s}{\binp{s}{\X} \map{P}^{2} \Par \bout{\dual{s}}{\abs{z \cat \vmap{\fn{P}}}{\binp{z}{\X} \map{P}^{\es}}} \inact}\\
%	\map{r}^{2} &=& \newsp{s}{\appl{\X}{s \cat \smap{\fn{P}}} \Par \bout{\dual{s}}{ \abs{z \cat \vmap{\fn{P}}}{\appl{X}{z \cat \vmap{\fn{P}}}}} \inact} \\
		\pmapp{\recp{X}{P}}{2}{f} &\defeq& \newsp{s}{\binp{s}{\X} \pmapp{P}{2}{{f,\{\rvar{X}\to \tilde{n}\}}} \Par \bbout{\dual{s}}{\abs{\vmap{\tilde{n}}, z } \,{\binp{z}{\X} \auxmapp{\pmapp{P}{2}{{f,\{\rvar{X}\to \tilde{n}\}}}}{\mathsf{v}}{\es}}} \inact} & \quad \tilde{n} = \ofn{P} \\ 
		\pmapp{\rvar{X}}{2}{f} &\defeq& \newsp{s}{\appl{\X}{\tilde{n}, s} \Par \bbout{\dual{s}}{ \abs{\vmap{\tilde{n}},z}\,\,{\appl{X}{ \vmap{\tilde{n}}, z}}} \inact} & \quad \tilde{n} = f(\rvar{X}) \\
		\pmapp{\bout{k}{n} P}{2}{f}	&\defeq&	\bbout{k}{ \abs{z}{\,\binp{z}{X} \appl{X}{n}} } \pmapp{P}{2}{f} \\
		\pmapp{\binp{k}{x} Q}{2}{f}	&\defeq&	\binp{k}{X} \newsp{s}{\appl{X}{s} \Par \bbout{\dual{s}}{\abs{x} \pmapp{Q}{2}{f}} \inact} \\
		\tmap{\btout{S_1} {S} }{2}	&\defeq&	\bbtout{\lhot{\btinp{\lhot{\tmap{S_1}{2}}}\tinact}} \tmap{S}{2}  \\
		\tmap{\btinp{S_1} S }{2}	&\defeq&	\bbtinp{\lhot{\btinp{\lhot{\tmap{S_1}{2}}}\tinact}} \tmap{S}{2} \\
		\tmap{\bbtout{\chtype{S_1}} {S} }{2}	&\defeq&	\bbtout{\shot{\btinp{\shot{\chtype{\tmap{S_1}{2}}}}\tinact}} \tmap{S}{2}  \\
		\tmap{\bbtinp{\chtype{S_1}} {S} }{2}	&\defeq&	\bbtinp{\shot{\btinp{\shot{\chtype{\tmap{S_1}{2}}}}\tinact}} \tmap{S}{2}
	\end{array}
\]
%
and as a homomorphism for the other process constructs. 
\end{definition}

\begin{remark}\rm
	Furthermore we define a mapping for environments $\Gamma$, as follows:
	\[
		\tmap{\Gamma \cat \rvar{X}:\Sigma}{2} = \tmap{\Gamma}{2} \cat X:\shot{(\tilde{S}_{\Sigma}, S^*)}
		%X:\trec{t}{\big(\shot{(\tilde{S}_{\Sigma}, \btinp{\vart{t}}\tinact)}\big)}
	\]
	where
	$S^* = \trec{t}{\big((\tilde{S}_{\Sigma}, \btinp{\vart{t}}\tinact)\big)}$
	and
	$\tilde{S}_{\Sigma} = S_1, \ldots, S_m$ for any $\Sigma = \{n_1:S_1, \ldots, n_m:S_m\}$.
\end{remark}

\begin{proposition}\rm
	Encoding $\fencod{\cdot}{\cdot}{2}{f}: \sesp \to \HO$  
	is type-preserving (cf. Def.~\ref{def:ep}\,(1)).
\end{proposition}

\begin{proof}
	Proof in Appendix~\ref{app:enc_sesp_to_HO_t}.
	\qed
\end{proof}

\begin{comment}
\begin{proof}
	By induction on the structure of \sesp process $P_0$. 
	\begin{enumerate}[1.]
		\item Case $P_0 = \rvar{X}$. Then we have the following typing in the source language:
	
		{\small
		\[
		\tree{
		}{
		\Gamma \cat \rvar{X}: \Sigma ;\, \es ;\, \es \proves \rvar{X} \hastype \Proc
		}
		\]
		}
	
		Then the typing of $\pmapp{\rvar{X}}{2}{f}$ is as follows, assuming $f(\rvar{X}) = \tilde{n}$ and $\tilde{x} = \vmap{\tilde{n}}$.
		Also, we write $\Sigma_{\tilde{n}}$ 
		and $\Sigma_{\tilde{x}}$ 
		to stand for 
		$n_1: S_1, \ldots, n_m: S_m$ and
			$x_1: S_1, \ldots, x_m: S_m$, respectively. 
		Below, we assume that $\Gamma = \Gamma' \cat X:\shot{\tilde{T}}$, 
		where  
		%$$\tilde{T} =  \trec{t}{\big(\tilde{S}, \btinp{\vart{t}}\tinact\big)}$$.
		\begin{eqnarray*}
		\tilde{T} & = & \big(\tilde{S}, S^*\big) \\
		S^* & = & \bbtinp{A}\tinact \\
		A & = & \trec{t}{(\tilde{S}, \btinp{\vart{t}}\tinact)}
		\end{eqnarray*}
				{\small
		\[
		\tree{
		\tree{
		\tree{
		\tree{
		}{
		\Gamma ;\, \es ;\, \es \proves X \hastype \shot{\tilde{T}}
		}
		\quad 
		\begin{array}{c}
		\Gamma ;\, \es ;\, \{n_i: S_i \} \proves n_i \hastype S_i \\
		\Gamma ;\, \es ;\, \{s: S^* \} \proves s\hastype S^*  \\
		\end{array}
		}{
		\Gamma  ;\, \es ;\, \Sigma_{\tilde{n}}, s:\btinp{\shot{\tilde{T}}}\tinact
		\proves  
		\appl{\X}{\tilde{n}, s} \hastype \Proc
		} 
		\quad 
		\tree{
		\tree{
		\Gamma  ;\, \es ;\,   \es \proves \inact \hastype \Proc
		}{
		\Gamma  ;\, \es ;\,   \dual{s}: \tinact \proves \inact \hastype \Proc
		} 
		\quad
		\tree{
		\tree{
		\begin{array}{c}
		\Gamma ;\, \es ;\, \{x_i: S_i \} \proves x_i \hastype S_i \\
		\Gamma ;\, \es ;\, \{z: S^*  \} \proves z\hastype S^*  \\
		\Gamma ;\, \es ;\, \es \proves X \hastype \shot{\tilde{T}}  \\
		\end{array}	}{
			\Gamma  ;\, \es ;\,   \Sigma_{\tilde{x}}, \, z:S^*
		\proves 
		 {\appl{X}{ \tilde{x}, z}} \hastype \Proc
		}
		}{
			\Gamma  ;\, \es ;\,   \es
		\proves 
		 \abs{\tilde{x},z}\,\,{\appl{X}{ \tilde{x}, z}} \hastype \shot{\tilde{T}}
		} 	
		}{
		\Gamma  ;\, \es ;\,   \dual{s}: \btout{\shot{\tilde{T}}}\tinact
		\proves 
		\bbout{\dual{s}}{ \abs{\tilde{x},z}\,\,{\appl{X}{ \tilde{x}, z}}} \inact \hastype \Proc
		}
		}{
		\Gamma  ;\, \es ;\, \Sigma_{\tilde{n}}, s:\btinp{\shot{\tilde{T}}}\tinact, \, \dual{s}: \btout{\shot{\tilde{T}}}\tinact
		\proves 
		\appl{\X}{\tilde{n}, s} \Par \bbout{\dual{s}}{ \abs{\tilde{x},z}\,\,{\appl{X}{ \tilde{x}, z}}} \inact \hastype \Proc
		}
		}{
		\Gamma  ;\, \es ;\, \Sigma_{\tilde{n}}
		\proves 
		\newsp{s}{\appl{\X}{\tilde{n}, s} \Par \bbout{\dual{s}}{ \abs{\tilde{x},z}\,\,{\appl{X}{ \tilde{x}, z}}} \inact} \hastype \Proc
		}
		\]
		}
	
		\item Case $P_0 = \recp{X}{P}$. Then we have the following typing in the source language:
	
		{\small
		\[
		\tree{
		\Gamma \cat \rvar{X}:\Sigma ;\, \es ;\,  \Sigma \proves P \hastype \Proc
		}{
		\Gamma  ;\, \es ;\,  \Sigma \proves \recp{X}{P} \hastype \Proc
		}
		\]
		}
	
		Then we have the following typing in the target language ---we write $R$ to stand for $\pmapp{P}{2}{{f,\{\rvar{X}\to \tilde{n}\}} }$
		and $\tilde{x}$ to stand for $\vmap{\ofn{P}}$.
		{\small 
		\[
		\tree{
		\tree{
		\tree{
		\tree{
		\tmap{\Gamma}{2}\cat X:\shot{\tilde{T}};\, \es;\, \tmap{\Sigma_{\tilde{n}}}{2}
		\proves
		 R  \hastype \Proc
		}{
		\tmap{\Gamma}{2}\cat X:\shot{\tilde{T}};\, \es;\, \tmap{\Sigma_{\tilde{n}}}{2}, s:\tinact 
		\proves
		 R  \hastype \Proc
		}
		}{
		\tmap{\Gamma}{2};\, \es;\, \tmap{\Sigma_{\tilde{n}}}{2}, s:\btinp{\shot{\tilde{T}}}\tinact 
		\proves
		\binp{s}{\X} R  \hastype \Proc
			} \quad 
		\tree{
		\tree{
		\tmap{\Gamma}{2};\, \es;\, \es
		\proves
		\inact \hastype \Proc
		}{
		\tmap{\Gamma}{2};\, \es;\, \dual{s}:\tinact
		\proves
		\inact \hastype \Proc
		} 
		\quad 
		\tree{
		\tree{
		\tree{
		\tmap{\Gamma}{2} \cat X: \shot{\tilde{T}};\, \es;\, \tmap{\Sigma_{\tilde{x}}}{2}
		\proves
		{{\auxmapp{R}{\mathsf{v}}{\es}}}  \hastype \Proc
		}{
		\tmap{\Gamma}{2} \cat X: \shot{\tilde{T}};\, \es;\, \tmap{\Sigma_{\tilde{x}}}{2},z: \tinact
		\proves
		{{\auxmapp{R}{\mathsf{v}}{\es}}}  \hastype \Proc
		}
		}{
		\tmap{\Gamma}{2};\, \es;\, \tmap{\Sigma_{\tilde{x}}}{2}, \, z: \btinp{A}\tinact
		\proves
		{{\binp{z}{\X} \auxmapp{R}{\mathsf{v}}{\es}}}  \hastype \Proc
		}
		}{
		\tmap{\Gamma}{2};\, \es;\, \es
		\proves
		{\abs{\tilde{x}, z } \,{\binp{z}{\X} \auxmapp{R}{\mathsf{v}}{\es}}}  \hastype \shot{\tilde{T}}
		}	
		}{
			\tmap{\Gamma}{2};\, \es;\, \dual{s}:\btout{\shot{\tilde{T}}}\tinact
		\proves
		\bbout{\dual{s}}{\abs{\tilde{x}, z } \,{\binp{z}{\X} \auxmapp{R}{\mathsf{v}}{\es}}} \inact \hastype \Proc
		}
		}{
		\tmap{\Gamma}{2};\, \es;\, \tmap{\Sigma_{\tilde{n}}}{2}, s:\btinp{\shot{\tilde{T}}}\tinact , \dual{s}:\btout{\shot{\tilde{T}}}\tinact
		\proves
		\binp{s}{\X} R \Par \bbout{\dual{s}}{\abs{\tilde{x}, z } \,{\binp{z}{\X} \auxmapp{R}{\mathsf{v}}{\es}}} \inact \hastype \Proc
		}
		}{
		\tmap{\Gamma}{2};\, \es;\, \tmap{\Sigma_{\tilde{n}}}{2} 
		\proves
		\newsp{s}{\binp{s}{\X} R \Par \bbout{\dual{s}}{\abs{\tilde{x}, z } \,{\binp{z}{\X} \auxmapp{R}{\mathsf{v}}{\es}}} \inact} \hastype \Proc
		}
		\]
		}
	\end{enumerate}
\qed
\end{proof}
\end{comment}

\begin{proposition}\rm
	Encoding $\fencod{\cdot}{\cdot}{2}{f}: \sesp \to \HO$ 
	enjoys operational correspondence (cf. Def.~\ref{def:ep}\,(2)).
\end{proposition}

\begin{proof}[Sketch]
	Proof in Appendix~\ref{app:enc_sesp_to_HO_oc}.
	\dk{TBD.}
	\qed
\end{proof}

\subsection{From $\HO$ to $\sesp$}

We now discuss the encodability of  $\HO$ into $\sesp$, 
i.e., how to encode a higher-order calculus with abstraction passing only
into a calculus with name passing only. 
We essentially follow the representability result put forward by 
Sangiorgi~\cite{San92,SaWabook}, but casted in the setting of session-typed communications. 
As we shall see, linearity of session endpoints will play a role in adaptating Sangiorgi's 
encodability strategy into a typed setting. 
Intuitively, such a strategy represents the exchange of a process with the exchange of 
a \emph{trigger}---a freshly generated names. 
Triggers may then be used to activate copies of the process, which now becomes a persistent 
resource represented by an input-guarded replication. In session-based communication, a session name 
is a linear resource and cannot be replicated. Consider the following (naive) adaptation of 
Sangiorgi's strategy in which session names are used are triggers and exchanged processes would be have to used exactly once:
%\begin{definition}[From $\HO$ to $\sesp$. Naive approach]
\[
	\begin{array}{lcl}
		\pmap{\bout{k}{\abs{x} P_1} Q}{n} & \defeq &  \newsp{s}{\bout{k}{s} (\pmap{Q}{n} \Par \binp{\dual{s}}{x} \pmap{P_1}{n})} \\
		\pmap{\binp{k}{X} P}{n} & \defeq& \binp{k}{x} \pmap{P}{n}\\
		\pmap{\appl{X}{k}}{n} & \defeq & \bout{x}{k} \inact
	\end{array}
	\]
%\end{definition}
%
%\begin{proposition}
%	Let $\Gamma;\emptyset;\Sigma \proves P \hastype \Proc$ with
%	the typing derivation to use only linear session types. Then
%	$\map{P}^8$ respects the properties of definition~\ref{def:ep}.
%\end{proposition}
%
%\begin{proof}
%	\dk{TODO}
%\end{proof}
(The mapping $\pmap{\cdot}{n}$ would be defined homomorphically for the remaining $\HO$ constructs.)
Although $\pmap{\cdot}{n}$ captures the correct semantics when
dealing with systems that allow only linear process variables,
it suffers from non-typability in the presence
of shared process variables. For instance,
let $P = \bbout{n}{\abs{x}{\bout{x}{m}\inact}} \inact \Par \binp{\dual{n}}{X} (\appl{X}{s_1} \Par \appl{X}{s_2})$.
We would have
\[
	\pmap{P}{n} \defeq
	\newsp{s}{\bout{n}{s} \binp{\dual{s}}{x} \bout{x}{m} \inact \Par \binp{\dual{n}}{x} (\bout{x}{s_1} \inact \Par \bout{x}{s_2} \inact)}
\]
The above process is non typable since processes $(\bout{x}{s_1} \inact$ and $\bout{x}{s_2} \inact)$
cannot be put in parallel because they do not have disjoint session environments.

The correct approach would be to use replicated shared names
as triggers instead of session names. 
Below we write $\repl{} P$ as a shorthand notation for $\recp{X}{(P \Par \rvar{X})}$.

\begin{definition}[From $\HO$ to $\sesp$]\rm
	Define $\encod{\cdot}{\cdot}{3}: \HO \to \sesp$ as follows
	\[
	\begin{array}{rcl}
		\pmap{\bbout{k}{\abs{x} Q} P}{3} & \defeq &  \newsp{a}{\bout{k}{a} (\pmap{P}{3} \Par \repl{} \binp{a}{y} \binp{y}{x} \pmap{Q}{3})\,} \\
		\pmap{\binp{k}{X} P}{3} &\defeq&  \binp{k}{x} \pmap{P}{3}\\
		\pmap{\appl{X}{k}}{3} & \defeq & \newsp{s}{\bout{x}{s} \bout{\dual{s}}{k} \inact}\\
		\tmap{\btout{\lhot{S}}S_1}{3} & \defeq & \bbtout{\chtype{\btinp{\tmap{S}{3}}\tinact}}\tmap{S_1}{3} \\
		\tmap{\btinp{\lhot{S}}S_1}{3} & \defeq & \bbtinp{\chtype{\btinp{\tmap{S}{3}}\tinact}}\tmap{S_1}{3}
	\end{array}
	\]
\end{definition}

\begin{proposition}\rm
	Encoding $\encod{\cdot}{\cdot}{3}: \HO \to \sesp$  is type-preserving (cf. Def.~\ref{def:ep}\,(1)).
\end{proposition}

\begin{proof}
	Proof in Appendix~\ref{app:enc_HO_to_sessp_t}.
	\qed
\end{proof}

\begin{comment}
\begin{proof}
By induction on the structure of \HO process $P$. 
\begin{enumerate}[1.]

%%%% Output of (linear) channel
	\item Case $P = \bbout{k}{\abs{x}Q}P$. Then we have two possibilities, depending on the typing for $\abs{x}Q$.
	The first case concerns a linear typing, and  
	we have the following typing in the source language:
	{\small
	\[
		\tree{
			\tree{}{\Gamma; \emptyset; \Sigma_1 \cat k:S  \proves  P \hastype \Proc} \quad \tree{\Gamma ; \emptyset ; \Sigma_2\cat x:S_1 \proves  Q \hastype \Proc}{\Gamma ; \emptyset ; \Sigma_2 \proves  \abs{x}Q \hastype \lhot{S_1}} }{
			\Gamma; \emptyset; \Sigma_1 \cat \Sigma_2 \cat k:\btout{\lhot{S_1}}S \proves  \bbout{k}{\abs{x}Q} P \hastype \Proc}
	\]
	}
	The corresponding typing in the target language is as follows --- we write $U_1$ to stand for 
	$\chtype{\btinp{\tmap{S_1}{3}}\tinact}$.
	We also write 
	\begin{eqnarray*}
	\tmap{\Gamma_1}{3} & = & \tmap{\Gamma}{3} \cup a:\chtype{\btinp{\tmap{S_1}{3}}\tinact} \\
	\tmap{\Gamma_2}{3} & = & \tmap{\Gamma_1}{3} \cup \rvar{X}:\tmap{\Sigma_2}{3}
	\end{eqnarray*}
	Also $(*)$ stands for $\tmap{\Gamma_1}{3}; \es ; \es \proves a \hastype U_1$; 
	$(**)$ stands for $\tmap{\Gamma_2}{3}; \es ; \es \proves a \hastype U_1$; and
	$(***)$ stands for $\tmap{\Gamma_2}{3}; \es ; \es \proves \rvar{X} \hastype \Proc$.
	
		{\small
	\[
		\tree{
		\tree{
		\tree{}{(*)}  \quad 
			\tree{
			\tree{}{
			\tmap{\Gamma_1}{3}; \es ; \tmap{\Sigma_1}{3}, k:\tmap{S}{3} 
		\proves 
		\pmap{P}{3}  \hastype \Proc
			}
			\quad 
			\tree{
			\tree{
			\tree{}{
			(***)
			} 
			\quad 
			\tree{
			\tree{
			\tree{
			\tree{
			}{
			\tmap{\Gamma_2}{3}; \es ; \tmap{\Sigma_2}{3},  x:\tmap{S_1}{3}
			\proves 
			\pmap{Q}{3} \hastype \Proc
			}
			}{
			\tmap{\Gamma_2}{3}; \es ; \tmap{\Sigma_2}{3}, y:\tinact, x:\tmap{S_1}{3}
			\proves 
			\pmap{Q}{3} \hastype \Proc
			}
			}{
			\tmap{\Gamma_2}{3}; \es ; \tmap{\Sigma_2}{3}, y: \btinp{\tmap{S_1}{3}}\tinact
			\proves 
			\binp{y}{x}\pmap{Q}{3} \hastype \Proc
			} 
			\quad 
			\tree{
			}{
			(**)
			}
			}{
			\tmap{\Gamma_2}{3}; \es ; \tmap{\Sigma_2}{3} 
			\proves 
			\binp{a}{y}\binp{y}{x}\pmap{Q}{3} \hastype \Proc
			} 
			}{
			\tmap{\Gamma_2}{3}; \es ; \tmap{\Sigma_2}{3} 
		\proves 
		\binp{a}{y}\binp{y}{x}\pmap{Q}{3} \Par \rvar{X} \hastype \Proc
			}
			}{
			\tmap{\Gamma_1}{3}; \es ; \tmap{\Sigma_2}{3} 
		\proves 
		\recp{X}{(\binp{a}{y}\binp{y}{x}\pmap{Q}{3} \Par \rvar{X})} \hastype \Proc
			}
			}{
			\tmap{\Gamma_1}{3}; \es ; \tmap{\Sigma_1, \Sigma_2}{3}, k:\tmap{S}{3} 
		\proves 
		\pmap{P}{3} \Par 
		\recp{X}{(\binp{a}{y}\binp{y}{x}\pmap{Q}{3} \Par \rvar{X})} \hastype \Proc
			}
		}{
		\tmap{\Gamma_1}{3}; \es ; \tmap{\Sigma_1, \Sigma_2}{3}, k:\bbtout{U_1}\tmap{S}{3} 
		\proves 
		\bout{k}{a}(\pmap{P}{3} \Par 
		\recp{X}{(\binp{a}{y}\binp{y}{x}\pmap{Q}{3} \Par \rvar{X}))} \hastype \Proc
		}
		}{
		\tmap{\Gamma}{3}; \es ; \tmap{\Sigma_1, \Sigma_2}{3}, k:\bbtout{U_1}\tmap{S}{3} 
		\proves 
		\newsp{a}{\bout{k}{a}( 
		\pmap{P}{3} \Par 
		\recp{X}{(\binp{a}{y}\binp{y}{x}\pmap{Q}{3} \Par \rvar{X}))}
		} \hastype \Proc
		}
	\]
	}
	 In the second case, $\abs{x}Q$ has a shared type. We have the following typing in the source language:
	{\small
	\[
		\tree{
			\tree{}{\Gamma; \emptyset; \Sigma \cat k:S  \proves  P \hastype \Proc} \quad 
			\tree{
			\tree{\Gamma ; \emptyset ; \cat x:S_1 \proves  Q \hastype \Proc}{\Gamma ; \emptyset ; \es \proves  \abs{x}Q \hastype \lhot{S_1} }			
			}{\Gamma ; \emptyset ; \es \proves  \abs{x}Q \hastype \shot{S_1} } }{
			\Gamma; \emptyset; \Sigma  \cat k:\btout{\shot{S_1}}S \proves  \bbout{k}{\abs{x}Q} P \hastype \Proc}
	\]
	}
	The corresponding typing in the target language can be derived similarly as in the first case.
	
	\item Case $P = \binp{k}{X} P$. Then there are two cases, depending on the type of $X$. 
	In the first case,
	we have the following typing in the source language:
	{\small
	\[
			\tree{\Gamma \cat X : \shot{S_1};\, \emptyset ;\, \Sigma \cat k:S \proves  P \hastype \Proc
			}{
			\Gamma;\, \emptyset;\, \Sigma\cat k:\btinp{\shot{S_1}}S \proves  \binp{k}{X} P \hastype \Proc}
	\]
	}
	The corresponding typing in the target language is as follows:
	% --- we write $\Gamma_0$ to stand for $\Gamma \setminus \{X: \lhot{S_1}\}$.
		{\small
	\[
			\tree{\tmap{\Gamma}{3} \cat x : \chtype{\btinp{\tmap{S_1}{3}}\tinact};\, \emptyset ;\, \Sigma \cat k:\tmap{S}{3} \proves  \tmap{P}{3} \hastype \Proc
			}{
			\tmap{\Gamma}{3};\, \emptyset; \, \tmap{\Sigma}{3}\cat k:\bbtinp{\chtype{\btinp{\tmap{S_1}{3}}\tinact}}\tmap{S}{3} \proves  \binp{k}{x} \pmap{P}{3} \hastype \Proc}
	\]
	}

   In the second case,  
	we have the following typing in the source language:
	{\small
	\[
			\tree{\Gamma;\, \{X : \lhot{S_1}\};\, \emptyset ;\, \Sigma \cat k:S \proves  P \hastype \Proc
			}{
			\Gamma;\, \emptyset;\, \Sigma\cat k:\btinp{\lhot{S_1}}S \proves  \binp{k}{X} P \hastype \Proc}
	\]
	}
	The corresponding typing in the target language is as follows:
	% --- we write $\Gamma_0$ to stand for $\Gamma \setminus \{X: \lhot{S_1}\}$.
		{\small
	\[
			\tree{\tmap{\Gamma}{3} \cat x : \chtype{\btinp{\tmap{S_1}{3}}\tinact};\, \emptyset ;\, \Sigma \cat k:\tmap{S}{3} \proves  \tmap{P}{3} \hastype \Proc
			}{
			\tmap{\Gamma}{3};\, \emptyset;\, \tmap{\Sigma}{3}\cat k:\bbtinp{\chtype{\btinp{\tmap{S_1}{3}}\tinact}}\tmap{S}{3} \proves  \binp{k}{x} \pmap{P}{3} \hastype \Proc}
	\]
	}

	
	\item Case $P = \appl{X}{k}$. Also here we have two cases, depending on whether $X$ has linear or shared type.
	In the first case, $X$ is linear and
	we have the following typing in the source language:
	{\small
	\[
			\tree{\Gamma ;\, \{X : \lhot{S_1}\};\,  \es \proves  X \hastype \lhot{S_1} \quad \Gamma; \es ; \{k:S_1\} \proves k \hastype S_1
			}{
			\Gamma;\, \{X : \lhot{S_1}\};\, k:S_1 \proves  \appl{X}{k} \hastype \Proc}
	\]
	}
	The corresponding typing in the target language is as follows  --- below we write $\tmap{\Gamma_1}{3}$ to stand for $\tmap{\Gamma}{3} \cat x:\chtype{\btout{\tmap{S_1}{3}}\tinact}$:
		{\small
	\[
			\tree{
			\tree{
			\tree{
			\tree{
			\tmap{\Gamma_1}{3};\, \es;\,  \es \proves  \inact \hastype \Proc
			}{
			\tmap{\Gamma_1}{3};\, \es;\,  \dual{s}:\tinact \proves  \inact \hastype \Proc
			} \quad 
			\tree{
			}{
			\tmap{\Gamma_1}{3};\, \es;\, \{k:\tmap{S_1}{3}\} \proves  k \hastype \tmap{S_1}{3} 
			}
			}{
			\tmap{\Gamma_1}{3};\, \es;\,\, k:\tmap{S_1}{3},\,  \dual{s}:\btout{\tmap{S_1}{3}}\tinact \proves  \bout{\dual{s}}{k}\inact \hastype \Proc
			} \quad \tree{}{\tmap{\Gamma_1}{3} ;\, \es ;\, \es \proves x \hastype \chtype{\btout{\tmap{S_1}{3}}\tinact}}
			}{
			\tmap{\Gamma_1}{3};\, \es;\, k:\tmap{S_1}{3}, s:\btinp{\tmap{S_1}{3}}\tinact , \dual{s}:\btout{\tmap{S_1}{3}}\tinact \proves  \bout{x}{s}\bout{\dual{s}}{k}\inact \hastype \Proc
			}
			}{
			\tmap{\Gamma_1}{3};\, \es;\, k:\tmap{S_1}{3} \proves  \news{s}{(\bout{x}{s}\bout{\dual{s}}{k}\inact)} \hastype \Proc}
	\]
	}
	In the second case, $X$ is shared, and
	we have the following typing in the source language:
	{\small
	\[
			\tree{\Gamma \cat  X : \lhot{S_1} ;\,  \es ;\,  \es \proves  X \hastype \shot{S_1} \quad \Gamma; \es ; k:S_1 \proves k \hastype S_1
			}{
			\Gamma \cat X : \shot{S_1};\, \es ;\, k:S_1 \proves  \appl{X}{k} \hastype \Proc}
	\]
	}
	The associated typing in the target language is obtained similarly as in the first case. \qed
	\end{enumerate}
	\end{proof}
\end{comment}


\begin{proposition}\rm
	Encoding $\encod{\cdot}{\cdot}{3}: \HO \to \sesp$ 
	enjoys operational correspondence (cf. Def.~\ref{def:ep}\,(2)).
\end{proposition}

\begin{proof}
	Proof in Appendix~\ref{app:enc_HO_to_sessp_oc}.
	\qed
\end{proof}

\begin{comment}
\begin{proof}[Sketch]
For completeness, we 
consider the \HO process $P = {\bbout{k}{\abs{x} Q} P_1} \Par \binp{k}{X} P_2$. We have that
\[
P \red P_1 \Par P_2 \subst{\abs{x}Q}{X}
\]
In the target language, this reduction is mimicked as follows:
\begin{eqnarray*}
\pmap{P}{2} & = & \newsp{a}{\bout{k}{a} (\pmap{P_1}{3} \Par \repl{} \binp{a}{y} \binp{y}{x} \pmap{Q}{3})\,} 
                  \Par \binp{k}{x} \pmap{P_2}{3} \\
            & \red & \newsp{a}{\pmap{P_1}{3} \Par \repl{} \binp{a}{y} \binp{y}{x} \pmap{Q}{3} 
                  \Par  \pmap{P_2}{3}\subst{a}{x}}
\end{eqnarray*}
\qed
\end{proof}
\end{comment}


At this point an open  question would be if we could find an encoding that maps
session names to session names without the creation of shared names.

\dk{put intuition??}



\section{Expressiveness Results}
\label{sec:positive}
We present two encodings:
\begin{enumerate}
\item The higher-order name-passing communication (\HOp) into 
the higher-order communication without name-passing nor 
recursions (\HO) (\S\,\ref{subsec:HOpi_to_HO})
\item the higher-order communication without 
name-passing and recursions (\HO)
into the first-order name-passing communication
with recursions (\sessp) (\S\,\ref{subsec:HO_to_sesspi})
\end{enumerate}
By (1), we can encode \sessp into \HO and by (2), 
we can encode \HOp into \sessp.  
Note that it is obvious that \HOp can encode both 
$\HO$ and $\sessp$ by the identitiy mapping. 

\subsection{From \HOp to \HO}
\label{subsec:HOpi_to_HO}

\begin{definition}\rm 
	Let $\vmap{\cdot}: 2^{\mathcal{N}} \longrightarrow \mathcal{V}^\omega$
	be a map of sequences of 
lexicographically ordered names to sequences of variables, defined
	inductively as: 
	$\vmap{\epsilon} = \epsilon$ and $\vmap{n \cat \tilde{m}} = x_n \cat \vmap{\tilde{m}}$. 
\end{definition}

\begin{definition}[Trigger Mapping] \label{d:trabs}\label{d:auxmap}
	Let $\sigma$ be a set of session names.
	We define a trigger mapping,  
$\auxmapp{\cdot}{{}}{\sigma}: \HO \to \HO$, in Fig.~\ref{f:auxmap}.
\end{definition}

%
\begin{figure}[t]
\[
\small
\begin{array}{rl}
	\auxmapp{\bout{n}{\abs{x}{Q}} P}{{}}{\sigma} &\!\!\!\!\!\!\defeq
		\bout{u}{\abs{x}{\auxmapp{Q}{{}}{\sigma}}} \auxmapp{P}{{}}{\sigma}
\\[1mm]
%\auxmapp{\bout{n}{m} P}{{}}{\sigma} \defeq
%	    \bout{u}{v}\auxmapp{P}{{}}{\sigma} 
	\auxmapp{\appl{x}{n}}{{}}{\sigma}  \defeq
		\appl{x}{u} \quad 
	\auxmapp{\inact}{{}}{\sigma}  \defeq  \inact
 & 
			\auxmapp{\binp{n}{x} P}{{}}{\sigma}\defeq
		\binp{u}{x} \auxmapp{P}{{}}{\sigma} 
\\[1mm]
	\auxmapp{\bsel{n}{l} P}{{}}{\sigma} \defeq
		\bsel{u}{l} \auxmapp{P}{{}}{\sigma} 
 & 
	\auxmapp{\bbra{n}{l_i:P_i}_{i \in I}}{{}}{\sigma}  \defeq 
		\bbra{u}{l_i:\auxmapp{P_i}{{}}{\sigma}}_{i \in I}
	\vspace{1mm} \\
\auxmapp{\news{n} P}{{}}{\sigma}  \defeq  \news{n} \auxmapp{P}{{}}{{\sigma \cat n}}
 & 
	\auxmapp{P \Par Q}{{}}{\sigma}  \defeq  \auxmapp{P}{{}}{\sigma} \Par \auxmapp{Q}{{}}{\sigma} 
\end{array}
\]
%\[
%	\begin{array}{rcl}
%          \auxmapp{\news{n} P}{{}}{\sigma} &\bnfis& \news{n} \auxmapp{P}{{}}{{\sigma \cat n}}
%		\vspace{1mm} \\
%		\auxmapp{\bout{n}{\abs{x}{Q}} P}{{}}{\sigma} &\bnfis&
%		\left\{
%		\begin{array}{rl}
%			\bout{x_n}{\abs{(x,\vmap{\fn{P}})}{\auxmapp{Q}{{}}{\sigma}}} \auxmapp{P}{{}}{\sigma} & n \notin \sigma\\
%			\bout{n}{\abs{(x,\vmap{\fn{P}})}{\auxmapp{Q}{{}}{\sigma}}} \auxmapp{P}{{}}{\sigma} & n \in \sigma
%		\end{array}
%		\right.
%			\vspace{1mm}	\\ 
%		\auxmapp{\bout{n}{m} P}{{}}{\sigma} &\bnfis&
%		\left\{
%		\begin{array}{rl}
%		    \bout{n}{m}\auxmapp{P}{{}}{\sigma} & n, m \in \sigma \\
%		    \bout{x_n}{m}\auxmapp{P}{{}}{\sigma} & n \not\in \sigma, m \in \sigma \\
%		    \bout{n}{x_m}\auxmapp{P}{{}}{\sigma} & n \in \sigma, m \not\in \sigma \\
%		    \bout{x_n}{x_m}\auxmapp{P}{{}}{\sigma} & n, m \not\in \sigma 
%		\end{array}
%		\right.
%		\vspace{1mm} \\ 
%				\auxmapp{\binp{n}{X} P}{{}}{\sigma} &\bnfis&
%		\left\{
%		\begin{array}{rl}
%			\binp{x_n}{X} \auxmapp{P}{{}}{\sigma} & n \notin \sigma\\
%			\binp{n}{X} \auxmapp{P}{{}}{\sigma} & n \in \sigma
%		\end{array}
%		\right.
%			\vspace{1mm}	\\ 
%		\auxmapp{\binp{n}{x}P}{{}}{\sigma} &\bnfis&
%		\left\{
%		\begin{array}{rl}
%		    \binp{n}{x}\auxmapp{P}{{}}{\sigma} & n \in \sigma \\
%		    \binp{x_n}{x}\auxmapp{P}{{}}{\sigma} & n \not\in \sigma 
%		\end{array}
%		\right.
%		\vspace{1mm} \\ 
%		\auxmapp{\bsel{n}{l} P}{{}}{\sigma} &\bnfis&
%		\left\{
%		\begin{array}{rl}
%			\bsel{x_n}{l} \auxmapp{P}{{}}{\sigma} & n \notin \sigma\\
%			\bsel{n}{l} \auxmapp{P}{{}}{\sigma} & n \in \sigma
%		\end{array}
%		\right.
%		\vspace{1mm} \\
%		\auxmapp{\bbra{n}{l_i:P_i}_{i \in I}}{{}}{\sigma} &\bnfis&
%		%\auxmapp{\bsel{n}{l} P}{{}}{\sigma} &\bnfis&
%		\left\{
%		\begin{array}{rl}
%			\bbra{x_n}{l_i:\auxmapp{P_i}{{}}{\sigma}}_{i \in I}  & n \notin \sigma\\
%			\bbra{n}{l_i:\auxmapp{P_i}{{}}{\sigma}}_{i \in I}  & n \in \sigma
%		\end{array}
%		\right.
%		\vspace{1mm} \\
%		\auxmapp{\appl{\X}{n}}{{}}{\sigma} &\bnfis&
%		\left\{
%		\begin{array}{rl}
%			\appl{\X}{x_n} & n \notin \sigma\\
%			\appl{\X}{n} & n \in \sigma\\
%		\end{array}
%		\right. \\
%		\auxmapp{\inact}{{}}{\sigma} &\bnfis& \inact\\
%		\auxmapp{P \Par Q}{{}}{\sigma} &\bnfis& \auxmapp{P}{{}}{\sigma} \Par \auxmapp{Q}{{}}{\sigma} 
% \end{array}
%\]
%The auxiliary map (cf. Definition~\ref{d:auxmap}) 
%used in the encoding of the higher-order communication 
%with recursive definitions into higher-order communication 
%without recursive definitions and (Definition~\ref{d:enc:fotohorec}).
$u = n$ if $n\in \sigma$; otherwise $u = x_n$, and  
$\fn{P}$ denotes a sequence of lexicopraphically ordered 
free names in $P$. 
%The mapping is defined homomorphically for inaction and parallel composition.
\caption{\label{f:auxmap} A triger mapping}
\end{figure}

Given a process $P$ with $\fn{P} = m_1, \cdots, m_n$, we are interested in its associated abstraction, which is defined as
$\abs{x_1, \cdots, x_n}{\auxmapp{P}{{}}{\epsilon} }$, where $\vmap{m_j} = x_j$, for all $j \in \{1, \ldots, n\}$.
This transformation from processes into abstractions can be reverted by
using abstraction and application with an appropriate sequence of session names:
%
\begin{proposition}\rm
	Let $P$ be a \HOp process and 
	suppose $\tilde{x} = \vmap{\tilde{n}}$ where 
$\tilde{n} = \fn{P}$.
	Then $P \scong \appl{(\abs{\tilde{x}}{\auxmapp{P}{{}}{\emptyset}})}{\tilde{n}}$.
%	$\appl{X}{\smap{\fn{P}}} \subst{(\vmap{\fn{P}}) \map{P}^{\emptyset}}{X} \scong P$
\end{proposition}

\begin{figure}[t]
\[
\begin{array}{rcll}
	\noindent{\bf Types:} \quad 
	\vtmap{{S}}{1}	&\!\!\defeq\!\!&	\lhot{(\btinp{\lhot{\tmap{S}{1}}} \tinact)} \\
	\vtmap{\chtype{S}}{1}&\!\!\defeq\!\!&	\lhot{(\btinp{\shot{\chtype{\tmap{S}{1}}}} \tinact)}  \\
	\vtmap{\chtype{L}}{1}&\!\!\defeq\!\!&	\lhot{(\btinp{\shot{\chtype{\tmap{L}{1}}}} \tinact)} \\
	\vtmap{\lhot{C}}{1} &\!\!\defeq\!\!& \lhot{\tmap{C}{1}}\\
	\vtmap{\shot{C}}{1} &\!\!\defeq\!\!& \shot{\tmap{C}{1}}\\
	\tmap{\chtype{S}}{1}&\!\!\defeq\!\!&	\chtype{\tmap{S}{1}}  \\
	\tmap{\chtype{L}}{1}&\!\!\defeq\!\!&	\chtype{\tmap{L}{1}}  \\
	%		\tmap{\btout{S_1} {S} }{1}	&\!\!\defeq\!\!&	\bbtout{\lhot{\btinp{\lhot{\tmap{S_1}{1}}}\tinact}} \tmap{S}{1}  \\
	%		\tmap{\btinp{S_1} S }{1}	&\!\!\defeq\!\!&	\bbtinp{\lhot{\btinp{\lhot{\tmap{S_1}{1}}}\tinact}} \tmap{S}{1} \\
	%		\tmap{\bbtout{\chtype{U}} {S} }{1}	&\!\!\defeq\!\!&	\bbtout{\shot{\btinp{\shot{\chtype{\tmap{U}{1}}}}\tinact}} \tmap{S}{1}  \\
	%		\tmap{\bbtinp{\chtype{U}} {S} }{1}	&\!\!\defeq\!\!&	\bbtinp{\shot{\btinp{\shot{\chtype{\tmap{U}{1}}}}\tinact}} \tmap{S}{1} \\

	\tmap{\btout{U} S}{1} &\!\!\defeq\!\!& \btout{{\vtmap{U}{1}}} \tmap{S}{1}\\
	\tmap{\btinp{U} S}{1} &\!\!\defeq\!\!& \btinp{{\vtmap{U}{1}}} \tmap{S}{1}\\
	\tmap{\btsel{l_i: S_i}_{i \in I}}{1} &\!\!\defeq\!\!& \btsel{l_i: \tmap{S_i}{1}}_{i \in I}\\
			\tmap{\btbra{l_i: S_i}_{i \in I}}{1} &\!\!\defeq\!\!& \btbra{l_i: \tmap{S_i}{1}}_{i \in I}\\
	\tmap{\vart{t}}{1} \defeq \vart{t} \quad 
			\tmap{\trec{t}{S}}{1}  &\!\!\defeq\!\!&
	\trec{t}{\tmap{S}{1}}\quad 
	\tmap{\tinact}{1}  \defeq  \tinact\\[1mm]
	\hline
	%\end{array}
	%\]
	%\[
	%\begin{array}{rcll}
	\noindent{\bf Labels:} \quad \quad 
		\mapa{\bactout{n}{m}}^{1} &\!\!\defeq\!\!&   \bactout{n}{\abs{z}{\,\binp{z}{x} \appl{x}{m}} } \\
		\mapa{\bactinp{n}{m}}^{1} &\!\!\defeq\!\!&   \bactinp{n}{\abs{z}{\,\binp{z}{x} \appl{x}{m}} } \\
			\mapa{\bactout{n}{\abs{{x}}{P}}}^{1} &\!\!\defeq\!\!& \bactout{n}{\abs{{x}}{\pmapp{P}{1}{\es}}}\\
			\mapa{\bactinp{n}{\abs{{x}}{P}}}^{1} &\!\!\defeq\!\!& \bactinp{n}{\abs{{x}}{\pmapp{P}{1}{\es}}}\\
			\mapa{\bactsel{n}{l} }^{1} \!\defeq\! \bactsel{n}{l} 
	\quad 
			\mapa{\bactbra{n}{l} }^{1} &\!\!\defeq\!\!& \bactbra{n}{l} 
	\quad \quad 
			\mapa{\tau}^{1} \!\defeq\! \tau
\\[1mm]
\hline
\end{array}
\]
{\bf Terms} : 
\[
\begin{array}{rcll}
  \pmapp{\bout{u}{w} P}{1}{f}	&\!\!\defeq\!\!&	\bout{u}{ \abs{z}{\,\binp{z}{x} (\appl{x}{w})} } \pmapp{P}{1}{f} \\
  \pmapp{\binp{u}{\AT{x}{C}} Q}{1}{f}	&\!\!\defeq\!\!&	\binp{u}{y} \newsp{s}{\appl{y}{s} \Par \bout{\dual{s}}{\abs{x}{\pmapp{Q}{1}{f}}} \inact} \\
		\pmapp{\bout{u}{\abs{{x}}{Q}} P}{1}{f}  
&\!\!\defeq\!\!& \bout{u}{\abs{{x}}{\pmapp{Q}{1}{f}}} \pmapp{P}{1}{f} \\
		\pmapp{\binp{u}{\AT{x}{L}} P}{1}{f} &\!\!\defeq\!\!& \binp{u}{x} \pmapp{P}{1}{f}\\
		\pmapp{\bsel{s}{l} P}{1}{f} &\!\!\defeq\!\!& \bsel{s}{l} \pmapp{P}{1}{f}\\
		\pmapp{\bbra{s}{l_i: P_i}_{i \in I}}{1}{f} &\!\!\defeq\!\!& \bbra{s}{l_i: \pmapp{P_i}{1}{f}}_{i \in I}\\
		\pmapp{\inact}{1}{f} \!\!\defeq\!\!\inact
& & 
		\pmapp{\news{n} P}{1}{f} \!\!\defeq\!\! \news{n} \pmapp{P}{1}{f}\\
\pmapp{{x}\, {u}}{1}{f}
 \!\!\defeq\!\!
{x}\, {u}
& & 		\pmapp{P \Par Q}{1}{f} \!\!\defeq\!\! \pmapp{P}{1}{f} \Par \pmapp{Q}{1}{f} \\
		\pmapp{\recp{X}{P}}{1}{f} &\!\!\defeq\!\!&\!\!\!\!\!\!
	\newsp{s}{\\
& &\!\!\!\!\!\!\bout{\dual{s}}{\abs{(\vmap{\tilde{n}}, y)} 
\,{\binp{y}{z_\X} \auxmapp{\pmapp{P\subst{z_\X}{\X}}{1}{{f,\{z_\rvar{X}\to \tilde{n}\}}}}{{}}{\es}}} \inact
\\ 
& & \!\!\!\!\!\!
 \Par 
\binp{s}{z_\X} \pmapp{P\subst{z_X}{X}}{1}{{f,\{z_\rvar{X}\to \tilde{n}\}}}
} 
\quad (\tilde{n} = \fn{P}) \\ 
\pmapp{z_\rvar{X}}{1}{f} &\!\!\defeq\!\!& \newsp{s}{
\appl{z_X}{(\tilde{n}, s)}\\
& &  \Par \bbout{\dual{s}}{ \abs{(\vmap{\tilde{n}},y)}{\appl{z_X}{(\vmap{\tilde{n}}, y)}}} \inact}  \quad (\tilde{n} = f(z_\rvar{X})) \\
\end{array}
\]
The input bound variable $x$ is annotated by a type to distinct the first-order and higher-order cases.
\caption{\label{f:enc:hopi_to_ho}
Encoding of \HOp into \HO.
%(cf.~Defintion~\ref{d:enc:fotohorec}).
%Mappings 
%$\map{\cdot}^2$,
%$\mapt{\cdot}^2$, 
%and 
%$\mapa{\cdot}^2$
%are homomorphisms for the other processes/types/labels. 
}
\end{figure}

\begin{definition}[Full Higher-Order Pi into Higher-Order]
\label{d:enc:hopitoho}
Let $f$ be a function from variables to sequences of name variables.
%
Let $\tyl{L}_{\HOp}=\calc{\HOp}{{\cal{T}}_1}{\hby{\ell}}{\wb_H}{\proves}$
and 
$\tyl{L}_{\HO}=\calc{\HO}{{\cal{T}}_2}{\hby{\ell}}{\wb_H}{\proves}$. 
where 
${\cal{T}}_1$ and ${\cal{T}}_2$ are sets of types of $\HOp$ 
and $\HO$, respectively, 
the typing $\proves$ is defined in 
Fig.~\ref{fig:typerulesmy} 
and the equivalence $\hwb$ is defined in Definition~\ref{d:bisim}.
We define the typed encoding 
the typed encoding $\enco{\map{\cdot}^{1}, \mapt{\cdot}^{1}, \mapa{\cdot}^{1}}: \HOp \to \HO$ in 
in Fig.~\ref{f:enc:hopi_to_ho}. 
\end{definition}

\begin{theorem}[Encoding of Full Higher-Order Pi into Higher-Order]
\label{f:enc:hopitoho}
The encoding from $\tyl{L}_{\HOp}$ into $\tyl{L}_{\HO}$ 
defined in Definition~\ref{d:enc:hopitoho}
is semantic preserving. 
\end{theorem}

\subsection{From \HO to \sessp}
\label{subsec:HO_to_sessp}

\begin{definition}[Higher-Order into First-Order Pi]
\label{d:enc:hopitopi}
$\tyl{L}_{\sessp}=\calc{\sessp}{{\cal{T}}_3}{\hby{\ell}}{\fwb}{\proves}$. 
where the typing is defined in 
Fig.~\ref{fig:typerulesmy} 
and the equivalence $\fwb$ is defined in Definition~\ref{d:fwb}.
${\cal{T}}_3$ is a set of types of $\sessp$.  
%
We define the mappings $\map{\cdot}^{2}$, $\mapt{\cdot}^{2}$, $\mapa{\cdot}^{2}$
in Fig.~\ref{f:enc:ho_to_sessp}. 
\end{definition}

\begin{figure}[t]
\[
\begin{array}{l}
	\begin{array}{rcl}
\noindent{\bf Types:}\quad 
		\tmap{\btout{\lhot{S}}S_1}{2} & \defeq & \bbtout{\chtype{\btinp{\tmap{S}{2}}\tinact}}\tmap{S_1}{2} \\
		\tmap{\btinp{\lhot{S}}S_1}{2} & \defeq & \bbtinp{\chtype{\btinp{\tmap{S}{2}}\tinact}}\tmap{S_1}{2} 
\\[1mm]
\hline
%\end{array}
%\]
%\[
%\begin{array}{rcll}
\noindent{\bf Labels:}\quad \quad 
		\mapa{\bactout{u}{\abs{ x}{P}} }^2  & \defeq & \news{a} \bactout{u}{a} \\
		\mapa{\bactinp{u}{\abs{ x}{P}} }^2 &  \defeq & \bactinp{u}{a}
\\[1mm]
\hline
\end{array}
\end{array}
\]
\hspace{4mm}{\bf Terms} :
\[
\begin{array}{rcll}
		\pmap{\bout{u}{\abs{x}{Q}} P}{2} &\!\!\!\! \defeq \!\!\!\!&  \left\{
		\begin{array}{r}
			\newsp{a}{\bout{u}{a} (\pmap{P}{2} \Par \repl{} \binp{a}{y} \binp{y}{x} \pmap{Q}{2})\,}\\
                  (s \notin \fn{Q}) \\
			\newsp{a}{\bout{u}{a} (\pmap{P}{2} \Par \binp{a}{y} \binp{y}{x} \pmap{Q}{2})\,}\quad\\
            \textrm{(otherwise)} %\dk{Q \textrm{ linear}} \\
		\end{array}
		\right.
		\\
\pmap{\binp{u}{x} P}{2} &\!\!\!\! \defeq \!\!\!\! &  \binp{u}{x} \pmap{P}{2}\\
\pmap{\appl{x}{u}}{2} & \!\!\!\! \defeq \!\!\!\! & \newsp{s}{\bout{x}{s} \bout{\dual{s}}{u} \inact}\\

	\end{array}
	\]
$\map{\cdot}^2$,
$\mapt{\cdot}^2$, 
and 
$\mapa{\cdot}^2$
are homomorphisms for the other types, labels and processes.   
	\caption{
Encoding of \HO into \sessp.
\label{f:enc:ho_to_sessp}
}
\end{figure}

\begin{theorem}[Encoding of Higher-Order into First-Order Pi]
\label{f:enc:hotopi}
The encoding from $\tyl{L}_{\HO}$ into $\tyl{L}_{\sessp}$ defined in 
Definition~\ref{d:enc:hopitopi} 
is semantic preserving. 
\end{theorem}

\begin{corollary}[Encoding of Higher-Order Pi into First-Order Pi]
The encoding from $\tyl{L}_{\HOp}$ into $\tyl{L}_{\sessp}$ is semantic preserving. 
\end{corollary}

Note that an encoding from $\tyl{L}_{\HOp}$ into $\tyl{L}_{\sessp}$
can be defined without using the first encoding result.  





\section{Extensions}
\label{sec:extension}
\subsection{Extensions of \HOp}
% !TEX root = main.tex




In this section we extend \HOp to define two more higher-order
process calculi:
(i)~\HOpp is the extension of \HOp with higher-order applications/abstractions
and 
(ii)~\PHOp is the extension of \HOp
with polyadicity.
In both cases, we detail the
required modifications in the syntax and types.
The two extensions allow us to assume the \PHOpp
which is the polyadic extension of \HOpp.


\subsubsection{\HOp with Higher-Order Abstractions: The  $\HOpp$}
%\label{subsec:hop}
\noi 
The calculus \HOpp 
extends \HOp with higher-order abstractions and applications.
\HOpp is the calculus defined in~\cite{characteristic_bis}.

\myparagraph{Syntax, Operational Semantics and Types.}
\noi First, the syntax of \figref{fig:syntax} extends 
$\appl{V}{u}$ to $\appl{V}{W}$, including higher-order value $W$. 
Rule 
\[
	\appl{(\abs{x}{P})}{V} \red P \subst{V}{x}
\]
replaces rule $\orule{App}$ in \figref{fig:reduction}.
The syntax of types is modified as follows: %changes to include: 
%
\begin{center}
	\begin{tabular}{c}
		$L \bnfis \shot{U} \bnfbar \lhot{U}$
	\end{tabular}
\end{center}
These types can be easily accommodated in the type system in \figref{fig:typerulesmy}, 
we replace $C$ by $U$ in \trule{Abs} and $C$ by $U'$ in \trule{App}. Subject
reduction~(\thmref{t:sr}) holds for \HOpp (cf.~\cite{characteristic_bis})

\subsubsection{\HOp with Polyadic Communication: \PHOp}

\noi Embeddings of polyadic name passing into monadic name passing are
well-studied. % in the literature. 
Using a linear typing, precise
encodings (including full abstraction) can be obtained~\cite{Yoshida96}.
The syntax of 
$\HOp$ is extended with
polyadic name passing $\tilde{n}$ and $\abs{\tilde{x}}{Q}$ in the syntax 
of value $V$. The type syntax is extended to: 
%
\begin{center}
	\begin{tabular}{c}
	$	L \bnfis \shot{\tilde{C}} \bnfbar \lhot{\tilde{C}}
		\quad\quad
		S \bnfis  \btout{\tilde{U}} S \bnfbar \btinp{\tilde{U}} S \bnfbar \cdots$
	\end{tabular}
\end{center}
%
The type system disallows a shared name that directly carries polyadic
shared names as in \cite{tlca07,MostrousY15}.

The combined syntax, semantics and type syntax for \HOpp and \PHOp
define the \PHOpp which is the calculus that allows higher-order
abstractions and aplications, and polyadicity.

This section studies two extensions of \HOp, 
the functional abstraction (\HOpp)
and polyadic (\PHOp) calculi.  
 
\subsection{Encoding from $\HOpp$ to $\HOp$}
\label{subsec:hop}
\noi The functional abstraction \HOp-calculus, denoted by \HOpp, 
extends \HOp to %the $n$-higher-order 
functional abstractions and applications.

\myparagraph{Syntax, Operational Semantics and Types}
\noi First, the syntax of \figref{fig:syntax} extends 
$\appl{V}{u}$ to 
 $\appl{V}{W}$, including higher-order value $W$. 
We then replace rule $\orule{App}$ in \figref{fig:reduction}
with rule $\appl{(\abs{x}{P})}{V} \red P \subst{V}{x}$.
The syntax of types changes to include: 
\[ L \bnfis \shot{U} \bnfbar \lhot{U}\]  
We apply the straightforward extension of the typing  
system to accomodate the extended type syntax 
(we replace $C$ by $U$ in \trule{Abs} and $C$ by $U'$ in \trule{App} in \figref{fig:typerulesmy}).
\smallskip 

\myparagraph{Behavioural Semantics}
Labels remain the same. Rule $\ltsrule{App}$ in the untyped LTS
(\figref{fig:untyped_LTS}) 
is replaced with rule $\appl{(\abs{x}{P})}{V} \by{\tau} P \subst{V}{x}$.
Characteristic processes (\defref{def:char}) are extended with  
${\mapchar{\shot{U}}{x}} \!\!\defeq\!\! \mapchar{\lhot{U}}{x} \!\!\defeq\!\! {\appl{x}{\omapchar{U}}}$ and ${\omapchar{\shot{U}}} \defeq {\omapchar{\lhot{U}}} \!\!\defeq\!\! \abs{x}{\mapchar{U}{x}}$. 
Then we can use the same definitions for $\cong$, $\wbc$, $\hwb$ and $\fwb$. 

\smallskip 

\myparagraph{Encoding from \HOpp to \HOp} 
Let $\tyl{L}_{\HOpp}=\calc{\HOpp}{{\cal{T}}_4}{\hby{\ell}}{\wb_H}{\proves}$
where 
${\cal{T}}_4$ is a set of types of $\HOpp$;  
the typing $\proves$ is defined in 
\figref{fig:typerulesmy} with extended \trule{Abs} and \trule{App}. 
We define the typed encoding 
the typed encoding $\enco{\map{\cdot}^{3}, \mapt{\cdot}^{3}, \mapa{\cdot}^{3}}: \HOpp \to \HOp$ in 
\figref{f:enc:hopip_to_hopi}.
By \propref{pro:composition}, 
we derive the following theorem. 

\smallskip 

\begin{theorem}[Encoding of Functional Abstraction Higher-Order Pi into Pi]
\label{f:enc:hopiptohopi}
The encoding from $\tyl{L}_{\HOpp}$ into $\tyl{L}_{\HOp}$ 
is precise. Hence the encodings 
from $\tyl{L}_{\HOpp}$ to $\tyl{L}_{\HO}$ 
and $\tyl{L}_{\sessp}$ 
are also precise. 
\end{theorem}

\begin{figure}[t]
\[
\begin{array}{lrcll}
\noindent{\bf Types:} & 
		\tmap{\shot{L}}{3} &\defeq& \shot{\btinp{\tmap{L}{3}} \tinact}
		\\
%&		\tmap{\lhot{L}}{3} &\defeq& \lhot{\btinp{\tmap{L}{3}} \tinact}
%		\\
&		\tmap{\btout{\shot{L}} S}{3} &\defeq& \btout{\tmap{\shot{L}}{3}} \tmap{S}{3}
		\\
%&		\tmap{\btout{\lhot{L}} S}{3} &\defeq& \btout{\tmap{\lhot{L}}{3}} \tmap{S}{3}
%		\\
&		\tmap{\btinp{\shot{L}} S}{3} &\defeq& \btinp{\tmap{\shot{L}}{3}} \tmap{S}{3}
%		\\
%&		\tmap{\btinp{\lhot{L}} S}{3} &\defeq& \btinp{\tmap{\lhot{L}}{3}} \tmap{S}{3}
\\[1mm]
\hline
\noindent{\bf Labels:} & 
%		\mapa{\bactout{n}{\abs{x:C}{P}}}^{3} &\defeq& \bactout{n}{\abs{x}{\pmap{P}{3}}}
%		\\
%		\mapa{\bactinp{n}{\abs{x:C}{P}}}^{3} &\defeq& \bactinp{n}{\abs{x}{\pmap{P}{3}}}
%		\\
		\mapa{\bactout{n}{\abs{\AT{x}{L}}{P}}}^{3} &\defeq& \bactout{n}{\abs{z}{\binp{z}{x} \pmap{P}{3}}}
		\\
&		\mapa{\bactinp{n}{\abs{\AT{x}{L}}{P}}}^{3} &\defeq& \bactinp{n}{\abs{z}{\binp{z}{x} \pmap{P}{3}}}
\\[1mm]
\hline
{\bf Terms}: & 
	\pmap{\appl{x}{(\abs{x} P)}}{3} &\defeq& \newsp{s}{\appl{x}{s} \Par \bout{\dual{s}}{\abs{x} \pmap{P}{3}} \inact}
		\\
&	\pmap{\bout{u}{\abs{\AT{x}{L}}{Q}} P}{3} &\defeq& \bout{u}{\abs{z}{\binp{z}{x} \pmap{Q}{3}}} \pmap{P}{3}
%		\pmap{\bout{u}{\abs{x: C}{Q}} P}{3} &\defeq& \bout{u}{\abs{x}{\pmap{Q}{3}}} \pmap{P}{3}
	\end{array}
	\]
The mapping of types for $\lhot{L}$ is defined by replacing 
$\shot{L}$ by $\lhot{L}$. 
The case of $\abs{x:C}{P}$ in the label and term mappings 
are %defined 
as in \figref{f:enc:hopi_to_ho}, replacing 
$\tmap{\cdot}{1}$,
$\mapa{\cdot}^{1}$, and 
$\pmap{\cdot}{1}_f$, by  
$\tmap{\cdot}{3}$,
$\mapa{\cdot}^{3}$, and 
$\pmap{\cdot}{3}$. 
The other processes, types and labels are  homomorphic. 
\caption{\label{f:enc:hopip_to_hopi} 
Encoding of \HOpp into \HOp.
}
\Hline
\end{figure} 

\subsection{Encoding from Polyadic $\HOp$ to $\HOp$}
\label{subsec:pho}
\noi Embedding the polyadic name passing 
into the monadic name passing is well-studied in the literature.    
Using the linear typing, 
the preciseness (full abstraction) can be obtained \cite{Yoshida96}.
Here we summarise $\PHOp$ can be encoded into $\HOp$. 
The syntax of 
$\HOp$ is extended from \HOp by including 
polyadic name passing $\tilde{n}$ and $\abs{\tilde{x}}{Q}$ in the syntax 
of value $V$. The type syntax is extended to: 
%
\[
L ::= \shot{\tilde{C}} \ | \ \lhot{\tilde{C}}
\quad\quad S \ ::= \  \btout{\tilde{U}} S \bnfbar \btinp{\tilde{U}} S \bnfbar \cdots 
\]
%
\dk{The type system dissallows polyadic shared names.}
Other definitions are straightforwardly extended. 
We extend the mapping for labels 
($\mapa{\cdot}: \ell \to \tilde{\ell}$ in  
\defref{def:tenc}) to capture 
a sequence of labels  and \defref{def:ep} stays as the same
assuming that if 
$P \hby{\ell} P'$ and $\mapa{\ell} = \{\ell_1, \ell_2,  \cdots, \ell_m\}$ then
$\map{P} \Hby{\mapa{\ell}} \map{P'}$
should be understood as
$\map{P} \Hby{\ell_1} P_1 \Hby{\ell_2} P_2 \cdots \Hby{\ell_m} P_m =  \map{P'}$,
for some
$P_1, P_2, \ldots, P_m$.

Let $\tyl{L}_{\PHOp}=\calc{\PHOp}{{\cal{T}}_5}{\hby{\ell}}{\wb_H}{\proves}$
where 
${\cal{T}}_5$ is a set of types of $\HOpp$;  
the typing $\proves$ is defined in 
\figref{fig:typerulesmy} with polyadic types. 
We define the typed encoding 
the typed encoding $\enco{\map{\cdot}^{4}, \mapt{\cdot}^{4}, \mapa{\cdot}^{4}}: \PHOp \to \HOp$ 
in \figref{f:enc:poltomon}. 
Then we have:

\smallskip 

\begin{theorem}[Encoding of Polyadic Higher-Order Pi into Higher-Order Pi]
\label{f:enc:hopiptohopi}
The encoding from $\tyl{L}_{\PHOp}$ into $\tyl{L}_{\HOp}$ 
is precise. Hence the encodings 
from $\tyl{L}_{\PHOp}$ to 
$\tyl{L}_{\HO}$ 
and $\tyl{L}_{\sessp}$ 
are also precise. 
\end{theorem}

\begin{figure}[t]
\small
\[
\begin{array}{rcl}
% typed mapping starts here
{\bf Types:}\hspace{2.5cm}\\
		\tmap{\btout{S_1, \cdots, S_m}S}{4}
		&\!\!\defeq\!\!&
		\btout{\tmap{S_1}{4}} \cdots ; \btout{\tmap{S_m}{4}}\tmap{S}{4}
%		\\
%		\tmap{\btinp{S_1, \cdots, S_m}S}{4}
%		&\defeq&
%		\btinp{\tmap{S_1}{4}} \cdots ; \btinp{\tmap{S_m}{4}}\tmap{S}{4}
		\\
		\tmap{\bbtout{L} S}{4}
		&\!\!\defeq\!\!&
		\bbtout{\mapt{L}^{4}}\mapt{S}^{4}
%		\\
%		\tmap{\bbtinp{L} S}{4}
%		&\defeq&
%		\bbtinp{\mapt{L}^{4}}\mapt{S}^{4}
		\\
%		\tmap{\bbtout{\shot{(C_1, \cdots, C_m)}} S}{4}
%		&\defeq&
%		\bbtout{
%		\shot{\big(\btinp{\tmap{C_1}{4}} \cdots; \btinp{\tmap{C_m}{4}}\tinact\big)}}\mapt{S}^{4}
%		\\
%		\tmap{\bbtinp{\shot{(C_1, \cdots, C_m)}} S}{4}
%		&\defeq&
%		\bbtinp{
%		\shot{\big(\btinp{\tmap{C_1}{4}} \cdots; \btinp{\tmap{C_m}{4}}\tinact\big)}}\mapt{S}^{4}
%		\\
		\tmap{\shot{(C_1, \cdots, C_m)}}{4}
		&\!\!\defeq\!\!&
		\shot{\big(\btinp{\tmap{C_1}{4}} \cdots; \btinp{\tmap{C_m}{4}}\tinact\big)}
		\\
		\tmap{\lhot{(C_1, \cdots, C_m)}}{4}
		&\!\!\defeq\!\!&
		\lhot{\big(\btinp{\tmap{C_1}{4}} \cdots; \btinp{\tmap{C_m}{4}}\tinact\big)}
		\\
%		\tmap{\lhot{(C_1, \cdots, C_m)}}{4}
%		&\defeq&
%		\lhot{\big(\btinp{\tmap{C_1}{\mathsf{p}}} \cdots \btinp{\tmap{C_m}{\mathsf{p}}}\tinact\big)}
%		\\
%		\tmap{\shot{(C_1, \cdots, C_m)}}{\mathsf{p}}
%		&\defeq&
%		\shot{\big(\btinp{\tmap{C_1}{\mathsf{p}}} \cdots \btinp{\tmap{C_m}{\mathsf{p}}}\tinact\big)}
\hline
{\bf Labels:}\hspace{2.5cm}\\
%		\\ % action mapping starts here
		\mapa{\bactout{u}{u_1, \ldots, u_m}}^4 
		&\!\!\!\!\defeq\!\!\!\!&
 \bactout{u}{u_1} \cdots \bactout{u}{u_m} \\
%		\mapa{\bactinp{k}{k_1, \ldots, k_m}}^4 &\defeq&   \big\{\bactinp{k}{k_1}, \cdots, \bactinp{k}{k_m} \big\}\\
		\mapa{\bactout{u}{\abs{x_1, \ldots, x_m}{P}} }^4 
		&\!\!\defeq\!\!&
\bactout{u}{\abs{z}\binp{z}{x_1} \cdots \binp{z}{x_m} \map{P}^{4}}\\
%		\mapa{\bactinp{k}{\abs{x_1, \ldots, x_m}{P}} }^4 &\defeq&  \bactinp{k}{\abs{z}\binp{z}{x_1} \cdots ; \binp{z}{x_m} \map{P}^{4}} 
\hline
{\bf Terms:}\hspace{2.5cm}\\
		\map{\bout{u}{u_1, \cdots, u_m} P}^{4}
		&\!\!\defeq\!\!&
		\bout{u}{u_1} \cdots ;  \bout{u}{u_m} \map{P}^{4}
		\\
%			\map{\binp{k}{x_1, \cdots, x_m} P}^{4}
%		&\defeq&
%		\binp{k}{x_1} \cdots ; \binp{k}{x_m}  \map{P}^{4}
%		\\
		\map{\bbout{u}{\abs{(x_1, \cdots, x_m)} Q} P}^{4}
		&\!\!\defeq\!\!&
		\bbout{u}{\abs{z}\binp{z}{x_1}\cdots ; \binp{z}{x_m} \map{Q}^{4}} \map{P}^{4}
		\\ 
		\map{\appl{x}{(u_1, \cdots, u_m)}}^{4}
		&\!\!\defeq\!\!&
		\newsp{s}{\appl{x}{s} \Par \bout{\dual{s}}{u_1} \cdots ; \bout{\dual{s}}{u_m} \inact} 
        \\ 
	\end{array}
\]
The input cases are defined as the outputs replacing $!$ by $?$. 
The mappings for the other processes/types/labels are 
homomorphic. 
\caption{\label{f:enc:poltomon}
Encoding of \PHOp to \HOp.
}
\Hline 
\end{figure}



\section{Negative Results}

In the encoding from $\HOp$ to $\sessp$ we showed that
an easy and straightforward encoding would be to create
a new shared name for every abstraction we want to pass
in order to use it as a trigger that activate copies of
the abstraction.

At this point a reasonable question could be whether we can
encode shared name behaviour to session name behaviour and at
the same time maintain the type, operational and behavioural semantics.
If such result holds then its impact would be much bigger than
the encoding from $\HOp$ to $\sessp$, since it would
allow us to have session type systems without shared names
and still have the modelling convenience of shared names.

In this section we prove the intution among researches 
that a semantic preserving encoding between a calculus
with shared names and a calculus with only session names
does not exist.

\begin{theorem}\rm
	There is no encoding $\enco{\map{\cdot}, \mapt{\cdot}, \mapa{\cdot}}: \HOp \longrightarrow \HOp^{\minussh}$
	that enjoys operational correspondence and full abstraction.
\end{theorem}

\begin{proof}
	The proof is based on the fact that
	transitions on session channels are
	$\tau$-inert in contrast with shared
	channels which do not enjoy
	$\tau$-inertness.

	Details of the proof in Appendix~\ref{app:neg}
	\qed
\end{proof}


%\section{Representing a Programming Language}
%\label{implementation}
%The \PHOpp calculus has all the ingredients for a straightforward
representation of $\lambda$-terms:
\begin{itemize}
	\item	first and higher order abstractions/applications.
	\item	curried abstractions are represented as polyadic abstractions.
\end{itemize}
The \PHOpp calculus has channels that allow first- and higher-order
passing, along with the standard parallel composition operator.
Furthermore, the calculus makes use of a session type system to
type processes.
In a sense the \PHOpp can be used to have a straightforward representation
of session typed parallel $\lambda$-terms that communicated on channels.

On the other hand the Concurrent Haskell programming language uses the
Haskell programming language, where all its operators are encodable in
$\lambda$-terms and typed in System $F_{\omega}$ and a system
library which is typed using the $IO$ monad 
and allows for parallel Haskell processes that communicate on channels.

We can identify the correponding terms of \PHOpp in Concurrent Haskell:
\begin{itemize}
	\item	$\map{x} = x$
	\item	$\map{\abs{\tilde{x}}{P}} = \backslash x_1 \rightarrow \dots \backslash x_n \rightarrow \map{P}$
	\item	$\map{\appl{V}{V}} = \appl{\map{V}}{\map{V}}$
	\item	$\map{\bout{n}{V} P} = \dots$
	\item	$\map{\binp{n}{x} P} = \dots$
	\item	$\map{P\Par Q} = \mathsf{fork} \dots$
	\item	$\map{\inact} = \epsilon$
	\item	$\map{\bsel{s}{P}} = \dots$
	\item	$\map{\bbra{s}{l_i: P}_{i \in I}} = \dots \mathsf{case}\ x\ \mathsf{of}\ \set{l_i}: \map{P_i}$
	\item	$\map{\recp{X}{P}} = \mathsf{fixpoint\ operator}$
\end{itemize}

In the above encoding the Concurrent Haskell operators enjoy the full power
of $\lambda$-terms with the basic operators of the Concurrent Haskell library.
Furthermore, the types on the left are typed on session types and the
types on the right are typed on System $F_\omega$.

By implementing an embedding of the $\PHOpp$ session type system into 
System $F_\omega$ we can have a straightforward embedding of $\PHOpp$
in Concurrent Haskell.


The main contribution of this section comes by:
\begin{itemize}
	\item	reversing the above mapping
	\item	relying on the fact that session types embed (i.e.~are isomorphic) with polymorphic logic, \cite{DBLP:conf/esop/CairesPPT13}
\end{itemize}

Now we can claim that $\PHOpp$ is the first complete
theoretical model that can encode the full capabilities of Concurrent
Haskell.

Going beyond the $\PHOpp$ calculus we have developed a series
of encodings that allow us to encode $\PHOpp$ in a number
of its sub-calculus. So, we can further identify fragments of
the above encodings to have a core represantation of Concurrent
Haskell into session typed higher order calculus.
Our results show that Concurrent Haskell is encodable
in the terms of Concurrent Haskell that represent
$\HOp, \HO$ and, $\sessp$ with the latter two being
core representations.

We argue that it is more convenient to use $\HO$ rather
than $\sessp$ as a core theoretical model for Concurrent Haskell,
so \dk{we have implemented an encoding of Concurent Haskell $\lambda$-terms
and communication primitives in the fragment of Concurrent Haskell that
represents \HO}.

\dk{At this point maybe a correspondence between the type derivation
of the session type system and a type derivation of Concurrent Haskell
session typed processes is in order.}


\section{Related Work}
\label{sec:relwork}
% !TEX root = main.tex
\myparagraph{Expressiveness of Process Calculi.}
There is a vast literature on expressiveness studies for process calculi. 
Here we concentrate on closely related work; 
see~\cite{KouzapasPY15} for more detailed related work. 
To substantiate claims related to (relative) expressive power,
early works appealed to different definitions of encoding \cite{Palamidessi03}.
Later on, 
proposals of abstract 
frameworks which 
state associated syntactic and semantic criteria 
were put forward; 
two proposals are~\cite{DBLP:journals/iandc/Gorla10,DBLP:journals/tcs/FuL10}. 
These frameworks are applicable to different calculi, and 
have shown useful to clarify known results and to derive new ones.
Our formulation of (precise) typed encoding (Def.~\ref{def:goodenc}) 
builds upon existing proposals (including~\cite{Palamidessi03,DBLP:journals/iandc/Gorla10,DBLP:conf/icalp/LanesePSS10})
in order to account for the session type systems
associated to \HOp and its variants/extensions.

\myparagraph{Expressiveness of Higher-Order Process Calculi.}
Due to the close relationship between
higher-order process calculi and functional calculi, works devoted to
encoding (variants of) the $\lambda$-calculus into (variants of) the
$\pi$-calculus~ (e.g.,~\cite{San92,DBLP:journals/tcs/Fu99}) are broadly related.
The encoding of process passing into name passing is well-known~\cite{SangiorgiD:expmpa};
the encoding of in the reverse direction 
is studied in~\cite{SaWabook} for an asynchronous, localised $\pi$-calculus
(only the output capability of names can be sent around).  The
work~\cite{San96int} studies hierarchies for calculi with
\emph{internal} first-order mobility and with higher-order mobility
without name-passing (similarly as \HO). The
hierarchies are based on expressivity: formally defined according to
the order of types needed in typing, they describe different ``degrees
of mobility''.  Via fully abstract encodings, it is shown that that
name- and process-passing calculi with equal order of types have the
same expressiveness.  With respect to these previous results, our
approach based on session types has several important consequences and
allows us to derive new results.  Our study reinforces the intuitive
view of ``encodings as protocols'', namely session protocols which
enforce precise linear and shared disciplines for names, a distinction
not investigated in~\cite{SangiorgiD:expmpa,DBLP:journals/tcs/Sangiorgi01}. In
turn, the linear/shared distinction is central in proper definitions
of trigger processes, and are essential to encodings
(Def.~~\ref{d:enc:hopitopi}) and behavioral equivalences
(Defs.~\ref{d:hbw} and~\ref{d:fwb}).  More interestingly, we showed that
$\HO$ suffices to encode  the session
calculus with name passing ($\sessp$) but also $\HOp$ and its extension with
higher-order applications ($\HOpp$). Thus, using session types
all these calculi are shown to be equally expressive with fully
abstract encodings.  To our knowledge, these are the first
expressivity results of this kind.

The work~\cite{XuActa2012} studies the encodability of the higher-order $\pi$-calculus (extended with a relabeling operator) into the first-order $\pi$-calculus; encodings in the reverse direction are also proposed, following \cite{Tho90}.
A minimal calculus of higher-order concurrency is studied in~\cite{DBLP:journals/iandc/LanesePSS11}: it lacks restriction,  name passing, output prefix (so  communication is asynchronous), and constructs for infinite behavior. 
This calculus (a sublanguage of \HO) has 
a simple notion of (strong) bisimilarity which coincides with barbed congruence.
%be Turing complete, while 
%have a decidable notion of (strong) bisimilarity that coincides with barbed congruence. 

Our work is closely related in spirit to the expressiveness studies in~\cite{DBLP:conf/icalp/LanesePSS10,DBLP:conf/wsfm/XuYL13}.
In~\cite{DBLP:conf/icalp/LanesePSS10}
the core calculus in~\cite{DBLP:journals/iandc/LanesePSS11} is extended with restriction, output prefix (thus enabling synchronous communication), 
and polyadic communication. It is shown that 
synchronous communication can encode asynchronous communication, % (as in the first-order setting),
and that process passing polyadicity induces a hierarchy in expressive power, % (unlike the first-order setting).
A further extension with process abstractions of order one
(functions from processes to processes)
 is shown to strictly add expressive power with respect to passing of processes only.
The paper~\cite{DBLP:conf/wsfm/XuYL13} complements the study in~\cite{DBLP:conf/icalp/LanesePSS10} by deepening on the expressive power of second-order abstractions (similar to \HO). 
In that setting, name and process abstractions are distinguished and contrasted, also considering polyadicity of abstraction parameters (the same kind of polyadicity present in \pHOp). It is shown that polyadicity of process abstraction induces an expressiveness hierarchy. Moreover, it is shown that name abstraction can encode process abstraction, and therefore it may be considered as a more basic mechanism. 
The works~\cite{DBLP:conf/icalp/LanesePSS10,DBLP:conf/wsfm/XuYL13} focus on untyped processes;
therefore, our work complements their previous results by clarifying the status of typeful, resource-aware structured communications in
trigger-based representations of process passing, both in encodings and  equivalences.

\myparagraph{Session Typed Processes.}
Two works~\cite{DemangeonH11,Dardha:2012:STR:2370776.2370794} 
study encodings of binary session calculi into 
a linearly typed $\pi$-calculus. 
\cite{DemangeonH11}~gives a precise encoding of \sessp into a linear calculus 
based on~\cite{BHY},  
while~\cite{Dardha:2012:STR:2370776.2370794} 
gives the operational correspondence 
(without a full abstraction result)
for the first- and higher-order 
$\pi$-calculus (\cite{tlca07}) into \cite{LinearPi}. 
They investigate an embeddability of two different typing systems;
by the result of \cite{DemangeonH11}, \HOpp is further encodable precisely 
into the linearly typed $\pi$-calculi.     

The discipline developed for the $\HOp$ is a subset
of that in~\cite{tlca07,MostrousY15}.
\cite{tlca07} develops a full higher-order session calculus
with process abstractions and applications, hence  
treats the type 
$U=U_1 \rightarrow U_2 \dots U_n \rightarrow \Proc$ and its linear type 
$U^1$
which corresponds $\shot{\tilde{U}}$ and $\lhot{\tilde{U}}$ in 
a super-calculus of $\HOpp$ and $\PHOp$. 
%in~\cite{MostrousY15} in the asynchronous setting.
%The session type
%system considered is influenced by the type systems for $\lambda$-calculi and
%uses type syntax of the form $U_1 \rightarrow U_2 \dots U_n \rightarrow \Proc$
%for shared values and $(U_1 \rightarrow U_2 \dots U_n \rightarrow \Proc)^{1}$
%for linear values.
%Such a type is expressed in $\HOpp$
%terms using the type $\shot{U}$ (respectively, $\lhot{U}$)
%with $U$ being a nested higher-order type; and 
%the $\HOp$ uses only types of the form
%$\shot{C}$ and $\lhot{C}$ with $C$ being a first-order channel type.
Our results show
the calculus in~\cite{tlca07} is not only expressed but 
also reasoned in 
$\HO$ (with limited form of arrow types, $\shot{C}$ and $\lhot{C}$) and 
$\sessp$, via precise encodings. 

\myparagraph{Typed Behavioural Equivalences.}
The current work follows the principles for
session type behavioural semantics that were laid
by the previous works of the
authors~\cite{dkphdthesis,KYHH2015,KY2015,DBLP:journals/iandc/PerezCPT14}.
A bisimulation relation is defined on a labelled
transition system that assumes a session typed
observer.
The bisimilarity is characterised by the corresponding
reduction-closed, barb-preserving congruence using a
proof technique that is derived from~\cite{Hennessy07}.
The theory for higher-order session types developed here
differentiates from 
the work in~\cite{dkphdthesis,KYHH2015,KY2015}, which 
considers the behavioural semantics for first-order
binary and multiparty session types.
Also the work \cite{DBLP:journals/iandc/PerezCPT14} studies typed behavioral equivalencies for binary session types.
The underlying process languages does not have shared names which, as we have shown, strictly add expressive power. 
Moreover, for this deterministic language, confluence and $\tau$-inertness properties are established.

%The theory for higher-order session type quivalences is more challenging than
%their corresponding first-order bisimulation theory.
To cope with the challenges presented by the higher-order
session theory, 
our approach continues the line of research 
originally drawn by Sangiorgi~\cite{San96H,SangiorgiD:expmpa}
and later improved by Jeffrey and Rathke~\cite{JeffreyR05}.
The works %Sangiorgi as part of his Ph.D.~research
\cite{San96H,SangiorgiD:expmpa}
introduced the first fully abstract encoding from the higher-order 
$\pi$-calculus to the $\pi$-calculus. 
The replicated triggered process 
is also used in this work for the encoding of \HOp into \sessp (Definition~\ref{d:enc:hopitopi}).
Sangiorgi's encoding is based on the idea of a replicated input guarded process 
(called a trigger process). Operational correspondence for
the triggered encoding is shown using the contextual bisimulation
with first-order labels.
Although contextual bisimilarity has a satisfactory discriminative power,
its use is hindered by the universal quantification on output clauses.
Sangiorgi then proposed \emph{normal bisimilarity}, a tractable  equivalence 
on processes without universal quantification. 
To show the coincidence between contextual and normal bisimilarities, 
the use of triggered processes and bisimilarity is developed in \cite{San96H}.
%The encoding also motivates the definition of a form of
Triggered bisimulation is also defined on first-order labels
where the contextual bisimulation is restricted to arbitrary
trigger substitution rather than arbitrary process substitutions.
The triggered bisimulation was further refined by Jeffrey and
Rathke, who study calculi with recursive types, not addressed in~\cite{San96H,SangiorgiD:expmpa} and
relevant in our work.
They introduce their own version of a
bisimulation~\cite{JeffreyR05}
based on a LTS which is extended with trigger meta-notation.
%for a full higher-order $\pi$-calculus that allows
%higher-order applications.
Like Sangiorgi's approach, the labelled transition semantics
in~\cite{JeffreyR05}
observes first-order triggered values instead of
higher-order values, offering a more direct characterization of contextual equivalence
and lifting the restriction to finite types.


There are similarities and differences between of the characteristic bisimulation
and the bisimulation as defined by Jeffrey and Rathke
(below we use the meta-notation adopted in~\cite{JeffreyR05}):
%
\begin{enumerate}[i)]
	\item	The output of a higher-order value $\abs{x}{Q}$ on name
		$n$ in Jeffrey\&Rathke approach requires the output of
		a fresh trigger name $t$ (notation $\tau_t$)
		on channel $n$ 
		and then the introduction of a replicated triggered process
		(notation $(t \Leftarrow (x) Q)$)
		in the context of the acting process:
		%
		\[
			P \by{\news{t} \bactout{n}{\tau_{t}}} P' \Par (t \Leftarrow (x) Q) \by{\bactinp{t}{v}} P' \Par \appl{(x) Q}{v} \Par (t \Leftarrow (x) Q) 
		\]
		%
		In the characteristic bisimulation approach we only observe
		an output of a value that can be either first- or higher-order:
		%
		\[
			P \hby{\bactout{n}{V}} P' 
		\]
		%
		with $V = \abs{x}{Q}$ or $V = m$.
		A non-replicated triggered process appears in
		the parallel context of the acting process when
		we compare two processes for behavioural equality
		(cf.~characteristic bisimulation Definition~\ref{d:fwb}),
		$P' \Par \htrigger{t}{\abs{x}{Q}}$.
		In fact using the LTS in
		Definition~\ref{d:tlts} we can get:
		%
		\begin{eqnarray*}
			P' \Par \htrigger{t}{\abs{x}{Q}}
			&\by{\abs{z}{\binp{z}{y} \repl{} \binp{t}{x} \appl{y}{x}}}&
			P' \Par \newsp{s}{\binp{s}{y} \repl{} \binp{t}{x} \appl{y}{x} \Par \bout{s}{\abs{x}{Q}} \inact}\\
			&\by{\tau}&
			P' \Par \repl{}\binp{t}{y} \appl{\abs{x}{Q}}{y}
		\end{eqnarray*}
		%
		that simulates the Jeffrey\&Rathke approach.

		The characteristic bisimulation differentiates from
		the Jeffrey\&Rathke approach:
		\begin{enumerate}[$\bullet$]
			\item	The typed LTS predicts the case of linear
				output values and will never allow replication
				of such a value;
				if $V$ is linear the input action would have no replication
				operator, as
				$\abs{z}{\binp{z}{y} \binp{t}{x} \appl{y}{x}}$.

			\item	The characteristic bisimulation introduces a uniform approach
				not only for
				higher-order values but for first-order values
				as well. A triggered process can accept any
				process that can substitute a first-order value as well.
				This is derived from the fact that the $\HOp$
				calculus makes no use of a matching operator, in contrast
				to the calculus defined in~\cite{JeffreyR05})
				where name matching is crucial to prove completness
				of the bisimilarity relation.
				We chose not to include the matching operator
				because of the requirement of a minimal calculus.
				In the lack of matching we use types to inhabit
				a value so we can observe its simplest interaction
				with the process environment.

			\item	The \HOp calculus requires only first-order
				applications. Higher-order applications,
				as in the Jeffrey\&Rathke work,
				are presented as an extension in the \HOpp
				calculus.

			\item	The trigger process is non-replicated. In fact
				the trigger process transforms guards the output
				value with a higher-order input prefix. The
				functionality of the input is used to
				simulate the contextual bisimilarity that subsumes
				the replicated trigger approach.
				The transformation of an output action as an input
				action allows for treating an output
				using the restricted LTS~\ref{def:rlts}:
				%
				\[
					P' \Par \htrigger{t}{\abs{x}{Q}} \hby{\bactinp{t}{\abs{x}{\mapchar{U}{x}}}}
					P' \Par \news{s}{ \appl{\mapchar{U}{x}}{s} \Par \bout{\dual{s}}{\abs{x}{Q}} \inact}
				\]
		\end{enumerate}
		%
		%In essence we are transforming a replicated trigger into a process
		%that is input-prefixed on a fresh name that receives a higher-order
		%value;

	\item	The input of a higher-order value in the Jeffrey\&Rathke approach requires 
		the input of a fresh trigger name, which is substituted on the application
		variable, thus having a meta-suntax for triggered application instead
		of higher-order applications:
		%
		\[
			\binp{n}{x} P \by{\bactinp{n}{\tau_k}} \appl{\abs{x}{P}}{\tau_k} \by{\tau} P \subst{x}{\tau_k} 
		\]
		%
		with every instance of process variable $x$ in $P$ being substituted
		with trigger value $\tau_k$ to give a process of the form $\appl{\tau_k}{x}$.
		The approach in the characteristic bisimulation observes the
		triggered value
		$\abs{z}\binp{t}{x} \appl{x}{z}$ as an input instead of the
		trigger name:
		%
		\[
			P \hby{\bactinp{n}{\abs{z}\binp{t}{x} \appl{x}{z}}} P \subst{\abs{z}\binp{t}{x} \appl{x}{z}}{x}
		\]
		%
		with applications being transformed to
		$\abs{z}{\binp{t}{x} \appl{x}{z}}{v}$
		Note that in the characteristic bisimulation semantics
		we can also observe a characteristic process as an input.
		
	\item 	Triggered application in the Jeffrey\&Rahtke
		are observe using an output
		lead into an output observation of the
		application value over
		the fresh trigger name.
		%
		\[
			\appl{\tau_k}{v} \by{\bactout{k}{v}} \inact
		\]
		%
		In the characteristic bisimulation instead of observing an 
		application and its value as an action we observe:
		i) the name of the trigger through the trigger value
		application; and ii) the application
		value by inhabiting it in the characteristic value
		and observing the interaction of the corresponding
		characteristic process with its environment.
		%
		\begin{eqnarray*}
			\appl{\abs{z}{\binp{t}{x} \appl{x}{z}}}{v} &\by{\tau}& \binp{t}{x} \appl{x}{v}
			\by{\bactinp{t}{\abs{x}{\mapchar{U}{x}}}}
			\appl{\abs{x}{\mapchar{U}{x}}}{v}
			\by{\tau} \mapchar{U}{x} \subst{n}{x}
		\end{eqnarray*}
		%
\end{enumerate}

%The main differences of the triggered
%bisimulation approach comparing to our approach are:
%i) We use observe higher-order values on the LTS in contrast to first-order 
%values in~\cite{DBLP:journals/lmcs/JeffreyR05}.
%ii) In our approach we avoid the replicated triggered process,
%by transforming the output process into a higher-order guarded input.
%iii) The triggered bisimulation gives semantics for higher-order application,
%whereas in our approach we give semantics for first-order applications
%and show that higher-order applications are fully encodable.

%Boreale and Sangiorgi, 
%Deng and Hennessy, 
%Jeffrey and Rathke, Hennessy and Koutavas, Schmitt and Lenglet, Pi\E9rard and Sumii.
%Perez et al (bisimilarities for binary sessions), Kouzapas and Yoshida (bisimilarities for binary and multiparty sessions).
%Bisimilarities for HO processes: \cite{Xu07}.

Sangiorgi et al.~\cite{DBLP:conf/lics/SangiorgiKS07}, use a higher-order LTS 
to define an arguably complex bisimulation relation that stores the knowledge known to
the observer, thus the observation actions are based on the observer's knowledge
at any given time. 
The environmental bisimulation approach is simplified by Koutavas and
Hennessy in~\cite{DBLP:journals/cl/KoutavasH12,DBLP:conf/esop/KoutavasH11}
with the introduction
of a mapping from constants to higher-order values. This
technique allows for the observation of first-order values instead
of higher-order values. It differs from the approaches
in~\cite{San96H,JeffreyR05} because the
mapping between higher and first order values is no longer implicit.







\section{Concluding Remarks}
\label{sec:concl}
% !TEX root = main.tex

To Do

%%%%%%%%%%%%%%%%%%%%%%%%%%%%%%%%%%%%%%%%%%%%%%%%%%%%%%%%%%%%%%%%%%%%%%%%%%%%%
% Bibliography.
%%%%%%%%%%%%%%%%%%%%%%%%%%%%%%%%%%%%%%%%%%%%%%%%%%%%%%%%%%%%%%%%%%%%%%%%%%%%%

%\bibliographystyle{IEEEtran}
\bibliographystyle{abbrv}
{\bibliography{session}}


\newpage
\onecolumn
\setcounter{tocdepth}{4}
\tableofcontents

\appendix 
\section{Appendix: the Typing System of \HOp}
\label{app:types}
%\begin{definition}[Type Equivalence]
%\label{def:iso}
%Let $\mathsf{ST}$ a set of closed session types. 
%Two types $S$ and $S'$ are said to be {\em isomorphic} if a pair $(S,S')$ is 
%in the largest fixed point of the monotone function
%$F:\mathcal{P}(\mathsf{ST}\times \mathsf{ST}) \to 
%\mathcal{P}(\mathsf{ST}\times \mathsf{ST})$ defined by:

%\begin{tabular}{rcl}
%$F(\Re)$ &$\!\!=\!\!$&	$\set{(\tinact, \tinact)}$\\
%         &$\!\!\cup\!\!$&	$\set{(\btout{U_1} S_1, \btout{U_2} S_2)
%\bnfbar (S_1, S_2),(U_1, U_2)\in \Re}$\\ 
%       &$\!\!\cup\!\!$&	$\set{(\btinp{U} S_1, \btinp{U} S_2)
%\bnfbar(S_1, S_2),,(U_1, U_2)\in \Re}$\\ 
%	&$\!\!\cup\!\!$&	$\set{(\btbra{l_i: S_i}_{i \in I} \,,\, \btbra{l_i: S_i'}_{i \in I}) \bnfbar \forall i\in I. (S_i, S_i')\in \Re}$\\
%	&$\!\!\cup\!\!$&	$\set{(\btsel{l_i: S_i}_{i \in I}\,,\, \btsel{l_i: S_i'}_{i \in I}) \bnfbar \forall i\in I. (S_i, S_i')\in \Re}$\\
%	&$\!\!\cup\!\!$&	$\set{(\trec{t}{S}, S')
%\bnfbar (S\subst{\trec{t}{S}}{\vart{t}},S')\in \Re}$\\
%	&$\!\!\cup\!\!$&	$\set{(S,\trec{t}{S'})
%\bnfbar (S,S'\subst{\trec{t}{S'}}{\vart{t}})\in \Re}$
%\end{tabular}
	
%\noindent
%Standard arguments ensure that $F$ is monotone, thus the greatest fixed point
%of $F$ exists. We write $S_1 \wb S_2$ if  $(S_1,S_2)\in \Re$. 
%\end{definition}

\begin{definition}[Duality]
\label{def:dual}
Let $\mathsf{ST}$ a set of closed session types. 
Two types $S$ and $S'$ are said to be {\em dual} if a pair $(S,S')$ is 
in the largest fixed point of the monotone function
$F:\mathcal{P}(\mathsf{ST}\times \mathsf{ST}) \to 
\mathcal{P}(\mathsf{ST}\times \mathsf{ST})$ defined by:
\begin{tabular}{rcl}
$F(\Re)$ &$\!\!=\!\!$&	$\set{(\tinact, \tinact)}$\\
         &$\!\!\cup\!\!$&	$\set{(\btout{U_1} S_1, \btinp{U_2} S_2)
\bnfbar(S_1, S_2)\in \Re, \  U_1 \wb U_2 }$\\ 
       &$\!\!\cup\!\!$&	$\set{(\btinp{U_1} S_1, \btout{U_2} S_2)
\bnfbar(S_1, S_2)\in \Re, \ U_1 \wb U_2}$\\ 
	&$\!\!\cup\!\!$&	$\set{(\btsel{l_i: S_i}_{i \in I} \,,\, \btbra{l_i: S_i'}_{i \in I}) \bnfbar \forall i\in I. (S_i, S_i')\in \Re}$\\
	&$\!\!\cup\!\!$&	$\set{(\btbra{l_i: S_i}_{i \in I}\,,\, \btsel{l_i: S_i'}_{i \in I}) \bnfbar \forall i\in I. (S_i, S_i')\in \Re}$\\
	&$\!\!\cup\!\!$&	$\set{(\trec{t}{S}, S')
\bnfbar (S\subst{\trec{t}{S}}{\vart{t}},S')\in \Re}$\\
	&$\!\!\cup\!\!$&	$\set{(S,\trec{t}{S'})
\bnfbar (S,S'\subst{\trec{t}{S'}}{\vart{t}})\in \Re}$
\end{tabular}
\noindent
where $U_1 \wb U_2$ means $U_1$ is type equivalent to $U_2$ \cite{yoshida.vasconcelos:language-primitives}.
Standard arguments ensure that $F$ is monotone, thus the greatest fixed point
of $F$ exists. We write $S_1 \dualof S_2$ if  $(S_1,S_2)\in \Re$. 
\end{definition}


%\section{Type Soundness}\label{app:ts}
We state type soundness of our system.
As our typed process framework is a sub-calculus of that considered by Mostrous and Yoshida, the proof of type soundness requires notions and properties which are specific instances of those already shown in~\cite{}.
We begin by stating weakening and strengthening lemmas, which have standard proofs.



\begin{lemma}[Weakening - Lemma C.2 in M\&Y]\label{l:weak}
\begin{enumerate}[$-$]
\item If $\Gamma; \Lambda; \Sigma   \proves P \hastype \Proc$ and $X \not\in \dom{\Gamma,\Lambda,\Sigma}$ then $\Gamma\cat X: \shot{S}; \Lambda; \Sigma   \proves P \hastype \Proc$ 
\end{enumerate}
\end{lemma}

\begin{lemma}[Strengthening - Lemma C.3 and C.4 in M\&Y]\label{l:stren}
\begin{enumerate}[$-$]
\item If $\Gamma \cat X: \shot{S}; \Lambda; \Sigma   \proves P \hastype \Proc$ and $X \not\in \fpv{P}$ then $\Gamma; \Lambda; \Sigma   \proves P \hastype \Proc$ 
\item If $\Gamma; \Lambda; \Sigma \cat k: \tinact  \proves P \hastype \Proc$ and $k \not\in \fn{P}$ then $\Gamma; \Lambda; \Sigma \proves P \hastype \Proc$.
\end{enumerate}
\end{lemma}

\begin{lemma}[Substitution Lemma - Lemma C.10 in M\&Y]\label{l:subst}
	\begin{enumerate}[1.]
		\item	Suppose $\Gamma; \Lambda; \Sigma \cat x:S  \proves P \hastype \Proc$ and
			$k \not\in \dom{\Gamma, \Lambda, \Sigma}$. 
			Then $\Gamma; \Lambda; \Sigma \cat k:S  \vdash P\subst{k}{x} \hastype \Proc$.

		\item	Suppose $\Gamma \cat x:\chtype{S}; \Lambda; \Sigma \proves P \hastype \Proc$ and
			$a \not\in \dom{\Gamma, \Lambda, \Sigma}$. 
			Then $\Gamma \cat a:\chtype{S}; \Lambda; \Sigma   \vdash P\subst{a}{x} \hastype \Proc$.

		\item	Suppose $\Gamma; \Lambda_1 \cat X:\lhot{S}; \Sigma_1  \proves P \hastype \Proc$ 
			and $\Gamma; \Lambda_2; \Sigma_2  \proves V \hastype \lhot{S}$ with 
			$\Lambda_1, \Lambda_2$ and $\Sigma_1, \Sigma_2$ defined.  
			Then $\Gamma; \Lambda_1 \cat \Lambda_2; \Sigma_1 \cat \Sigma_2  \proves P\subst{V}{X} \hastype \Proc$.

		\item	Suppose $\Gamma \cat X:\shot{S}; \Lambda; \Sigma  \proves P \hastype \Proc$ and
			$\Gamma; \emptyset ; \emptyset  \proves V \hastype \shot{S}$.
			Then $\Gamma; \Lambda; \Sigma  \proves P\subst{V}{X} \hastype \Proc$.
		\end{enumerate}
\end{lemma}

\begin{proof}
In all four parts, we proceed by induction on the typing for $P$, with a case analysis on the last applied rule. 
Parts (1) and (2) are standard and omitted. 

In Part (3), we content ourselves by detailing only the case in which the last applied rule is \trule{App}. 
Then we have $P = \appl{X}{k}$ and. By typing inversion on the first assumption we infer that $\Lambda_1 = \emptyset$, $\Sigma_1 = \{k : S\}$, and also 
\begin{eqnarray}
\Gamma; \{X : \lhot{S} \} ; \emptyset & \proves &  X \hastype \shot{S} \nonumber \\
\Gamma; \emptyset; \{k : S\} & \proves & k \hastype S  \label{eq:subseq1}
\end{eqnarray}
By inversion on the second assumption we infer that either (i)\,$V = Y$ (for some process variable $Y$) or (ii)\,$V = (z)Q$, for some $Q$ such that
\begin{equation}
\Gamma; \Lambda_2 ; \Sigma_2 \cat z:S  \proves   Q \hastype \Proc \label{eq:subseq2}\\
\end{equation}
In possibility\,(i), we have a simple substitution on process variables and the thesis follows easily. 
In possibility\,(ii), we observe that $P\subst{V}{X} = \appl{X}{k}\subst{(z)Q}{X} = Q\subst{k}{z}$.
The thesis then follows by using Lemma~\ref{l:subst}\,(1) with \eqref{eq:subseq1} and \eqref{eq:subseq2} above to infer 
\begin{equation*}
\Gamma; \Lambda_2 ; \Sigma_2 \cat k:S  \proves   Q\subst{k}{z} \hastype \Proc .
\end{equation*}
The proof of Part (4) follows similar lines as that of Part (3).
\qed
\end{proof}

\begin{definition}[Well-typed Session Environment]
	Let $\Sigma$ a session environment.
	We say that $\Sigma$ is {\em well-typed} if whenever
	$s: S_1, \dual{s}: S_2 \in \Sigma$ then $S_1 \dualof S_2$.
\end{definition}

\begin{definition}[Session Environment Reduction]
	We define relation $\red$ on session environments as:
	\begin{itemize}
		\item	$\Sigma \cat s: \btout{U} S_1 \cat \dual{s}: \btinp{U} S_2 \red \Sigma \cat s: S_1 \cat \dual{s}: S_2$
		\item	$\Sigma \cat s: \btsel{l_i: S_i}_{i \in I} \cat \dual{s}: \btbra{l_i: S_i'}_{i \in I} \red \Sigma \cat s: S_j \cat \dual{s}: S_j', \quad (j \in I)$.
	\end{itemize}
\end{definition}

We now state the instance of type soundness that we can derived from the Mostrous and Yoshida system.
It is worth noticing that M\&Y have a slightly richer definition of structural congruence.
Also, their statement for subject reduction relies on an ordering on typings associated to queues and other 
runtime elements (such extended typings are denoted $\Delta$ by M\&Y).
Since we are synchronous we can omit such an ordering.


We now repeat the statement of Theorem~\ref{t:sr} in Page~\pageref{t:sr}:

\begin{theorem}[Type Soundness - Theorem~\ref{t:sr}]
	\begin{enumerate}[1.]
		\item	(Subject Congruence) Suppose $\Gamma; \Lambda; \Sigma \proves P \hastype \Proc$.
			Then $P \scong P'$ implies $\Gamma; \Lambda; \Sigma \proves P' \hastype \Proc$.

		\item	(Subject Reduction) Suppose $\Gamma; \emptyset; \Sigma \proves P \hastype \Proc$
			with
%			$\mathsf{balanced}(\Sigma)$. 
			well-typed $\Sigma$.
			Then $P \red P'$ implies $\Gamma; \emptyset; \Sigma_2  \proves P' \hastype \Proc$
			and $\Sigma_1 \red \Sigma_2$ or $\Sigma_1 = \Sigma_2$.
	\end{enumerate}
\end{theorem}

\begin{proof}
Part (1) is standard, using weakening and strengthening lemmas. Part (2) proceeds by induction on the last reduction rule used:
\begin{enumerate}[$-$]
\item Case \orule{NPass}: Then there are three sub-cases, depending on whether the 
communication subject is a shared name or a session channel. In turn, the second case should consider two possibilities, namely whether the communicated name is a shared name or a channel, that is, i.e., $U = S$ or $U = \chtype{S}$. 

In all cases, the proof is standard, using appropriate rules.
In the first case, we analyze the interaction of rules~\trule{Conn} and~\trule{ConnDual}, 
using Lemma~\ref{l:subst}(1).
This case amounts to delegation on shared channel and so the session environment does not reduce.
In the second case, we consider rule~\trule{Send} with two different rules 
 for typing reception of a name: when $U = S$ we use typing 
rule~\trule{RecvS} and Lemma~\ref{l:subst}(1); when $U = \chtype{S}$ we use typing 
rule~\trule{RecvShN} and Lemma~\ref{l:subst}(2). In this case, the session environment does reduce.

\item Case \orule{APass}: Then there are two sub-cases, depending on whether the communicated process abstraction is a shared or linear, that is, i.e., $U = \shot{S}$ or $U = \lhot{S}$. The proof is standard, using appropriate rules for typing reception of a process abstraction: 
when $U = \lhot{S}$ we use typing rule~\trule{RecvL} and Lemma~\ref{l:subst}(3); 
when $U = \shot{S}$ we use typing rule~\trule{RecvSh} and Lemma~\ref{l:subst}(4). The session environment reduces.

\item Case \orule{Sel}: The proof is standard, the session environment reduces.

\item Case \orule{Sess}:  The proof is standard, exploiting induction hypothesis. The session environment may remain invariant (channel restriction)  or reduce (name restriction).

\item Case \orule{Par}:
The proof is standard, exploiting induction hypothesis. 

\item Case \orule{Cong}: follows from Theorem~\ref{t:sr}\,(1).



\end{enumerate}
\qed
\end{proof}

\section{Behavioural Semantics}

We present the proofs for the theorems in
Section~\ref{sec:beh_sem}.

\subsection{Proof for Theorem~\ref{the:coincidence}}

We split Theorem~\ref{the:coincidence} into 
Lemmas which we prove independently.
The combination of the lemmas is the proof for the parts
of the theorem.

The proof for Part 1 for the theorem is based on the
stratified definition of the bisimulation relations
$\wbc$ and $\wb$. The Knaster-Tarski theorem ensures
that both definitions are equivalent.
We give the stratified  definitions:

\begin{definition}[Stratified Contextual Bisimulation]\rm
	We define a set of relations $\mathcal{R}^c_n$
	on the following conditions:
%
	\begin{itemize}
		\item	$\mathcal{R}^c_0 = \bigcup_{\forall R} R, R$ is a typed relation.
		\item	$\Gamma; \emptyset; \Sigma_1\ \mathcal{R}^c_n\ \Sigma_2 \proves P_1\ \mathcal{R}^c_n\ P_2$
			whenever
			\begin{enumerate}
				\item	$\forall \news{\tilde{s}} \bactout{s}{\abs{x} P}$ such that
					\[
						\Gamma; \emptyset; \Sigma_1 \by{\news{\tilde{s}} \bactout{s}{\abs{x} P}} \Sigma_1' \proves P_1 \by{\news{\tilde{s}} \bactout{s}{\abs{x} P}} P_2
					\]
					$\exists Q_2, \abs{x}{Q}$ such that
					\[
						\Gamma; \emptyset; \Sigma_2 \By{\news{\tilde{s'}} \bactout{s}{\abs{x} Q}} \Sigma_2' \proves Q_1 \By{\news{\tilde{s}} \bactout{s}{\abs{x} Q}} Q_2
					\]
					and $\forall C, s'$
%					such that
%					\begin{eqnarray*}
%						\Gamma; \emptyset; \Sigma_1'' \proves \newsp{\tilde{s}}{P_2 \Par P \subst{s'}{x}} \hastype \Proc \\
%						\Gamma; \emptyset; \Sigma_2'' \proves \newsp{\tilde{s}}{Q_2 \Par Q \subst{s'}{x}} \hastype \Proc
%					\end{eqnarray*}
%					then
					\[
						\Gamma; \emptyset; \Sigma_1''\ \mathcal{R}^c_{n-1}\ \Sigma_2'' \proves
						\newsp{\tilde{s}}{P_2 \Par \context{C}{P \subst{s'}{x}}}\ \mathcal{R}^c_{n-1}\  \newsp{\tilde{s}}{Q_2 \Par \context{C}{Q \subst{s'}{x}}}
					\]

				\item	$\forall \news{\tilde{s}} \bactout{s}{s_1}$ such that
					\[
						\Gamma; \emptyset; \Sigma_1 \by{\news{\tilde{s}} \bactout{s}{s_1}} \Sigma_1' \proves P_1 \by{\news{\tilde{s}} \bactout{s}{s_1}} P_2
					\]
					%with $s_1: S \in \Sigma_1 \vee (\dual{s_2}: S' \in \Sigma_2 \wedge S \dualof S')$
					then $\exists Q_2, s_2$ such that
					\[
						\Gamma; \emptyset; \Sigma_2 \By{\news{\tilde{s'}} \bactout{s}{s_2}} \Sigma_2' \proves Q_1 \By{\news{\tilde{s'}} \bactout{s}{s_2}} Q_2
					\]
					%such that
		%			\begin{eqnarray*}
		%				\Gamma; \emptyset; \Sigma_1'' \proves \newsp{\tilde{s}}{P_2 \Par \context{C}{P \subst{s'}{x}}} \hastype \Proc \\
		%				\Gamma; \emptyset; \Sigma_2'' \proves \newsp{\tilde{s}}{Q_2 \Par \context{C}{Q \subst{s'}{x}}} \hastype \Proc
		%			\end{eqnarray*}
					and $\forall R, \set{x} = \fn{R}$
					\[
						\Gamma; \emptyset; \Sigma_1''\ \mathcal{R}^c_{n-1}\ \Sigma_2'' \proves \newsp{\tilde{s}}{P_2 \Par R\subst{s_1}{x}}\ \mathcal{R}^c_{n-1}\ 
						\newsp{\tilde{s'}}{Q_2 \Par R\subst{s_2}{x}}
					\]


				\item	$\forall \lambda \not= \news{\tilde{s}} \bactout{s}{\abs{x} P}$ such that
					\[
						\Gamma; \emptyset; \Sigma_1 \by{\lambda} \Sigma_1' \proves P_1 \by{\lambda} P_2
					\]
					$\exists Q_2$ such that 
					\[
						\Gamma; \emptyset; \Sigma_2 \by{\hat{\lambda}} \Sigma_2' \proves Q_1 \By{\hat{\lambda}} Q_2
					\]
					and
					$\Gamma; \emptyset; \Sigma_1'\ \mathcal{R}^c_{n-1}\ \Sigma_2' \proves P_2\ \mathcal{R}^c_{n-1}\ Q_2$.

				\item	The symmetric cases of 1, 2, 3.
			\end{enumerate}
	\end{itemize}
	\noi The above function is monotone and the Knaster-Tarski theorem ensures that a lattice is define
	with the largest $\mathcal{R}^c_n$ to be denote as $\wbc_n$ and the largest fix-point is equal to the
	contextual bisimilarity relation $\wbc = \bigcap_{i \geq 0} \wbc_i$.
\end{definition}


\begin{definition}[Stratified Bisimulation]\rm
	We define a set of relations $\mathcal{R}_n$
	on the following conditions:
%
	\begin{itemize}
		\item	$\mathcal{R}_0 = \bigcup_{\forall R} R, R$ is a typed relation.
		\item	$\Gamma; \emptyset; \Sigma_1\ \mathcal{R}_n\ \Sigma_2 \proves P_1\ \mathcal{R}_n\ P_2$
			whenever
			\begin{enumerate}
				\item	$\forall \news{\tilde{s}} \bactout{s}{\abs{x} P}$ such that
					\[
						\Gamma; \emptyset; \Sigma_1 \by{\news{\tilde{s}} \bactout{s}{\abs{x} P}} \Sigma_1' \proves P_1 \by{\news{\tilde{s}} \bactout{s}{\abs{x} P}} P_2
					\]
					$\exists Q_2, \abs{x}{Q}$ such that
					\[
						\Gamma; \emptyset; \Sigma_2 \By{\news{\tilde{s'}} \bactout{s}{\abs{x} Q}} \Sigma_2' \proves Q_1 \By{\news{\tilde{s}} \bactout{s}{\abs{x} Q}} Q_2
					\]
					and $\forall s'$
%					such that
%					\begin{eqnarray*}
%						\Gamma; \emptyset; \Sigma_1'' \proves \newsp{\tilde{s}}{P_2 \Par P \subst{s'}{x}} \hastype \Proc \\
%						\Gamma; \emptyset; \Sigma_2'' \proves \newsp{\tilde{s}}{Q_2 \Par Q \subst{s'}{x}} \hastype \Proc
%					\end{eqnarray*}
%					then
					\[
						\Gamma; \emptyset; \Sigma_1''\ \mathcal{R}_{n-1}\ \Sigma_2'' \proves
						\newsp{\tilde{s}}{P_2 \Par P \subst{s'}{x}}\ \mathcal{R}_{n-1}\  \newsp{\tilde{s}}{Q_2 \Par Q \subst{s'}{x}}
					\]

				\item	$\forall \news{\tilde{s}} \bactout{s}{s_1}$ such that
					\[
						\Gamma; \emptyset; \Sigma_1 \by{\news{\tilde{s}} \bactout{s}{s_1}} \Sigma_1' \proves P_1 \by{\news{\tilde{s}} \bactout{s}{s_1}} P_2
					\]
					with $s_1: S \in \Sigma_1 \vee (\dual{s_2}: S' \in \Sigma_2 \wedge S \dualof S')$
					then $\exists Q_2, s_2$ such that
					\[
						\Gamma; \emptyset; \Sigma_2 \By{\news{\tilde{s'}} \bactout{s}{s_2}} \Sigma_2' \proves Q_1 \By{\news{\tilde{s'}} \bactout{s}{s_2}} Q_2
					\]
					%such that
		%			\begin{eqnarray*}
		%				\Gamma; \emptyset; \Sigma_1'' \proves \newsp{\tilde{s}}{P_2 \Par \context{C}{P \subst{s'}{x}}} \hastype \Proc \\
		%				\Gamma; \emptyset; \Sigma_2'' \proves \newsp{\tilde{s}}{Q_2 \Par \context{C}{Q \subst{s'}{x}}} \hastype \Proc
		%			\end{eqnarray*}
					and
					\[
						\Gamma; \emptyset; \Sigma_1''\ \mathcal{R}_{n-1}\ \Sigma_2'' \proves \newsp{\tilde{s}}{P_2 \Par \map{S}^{s_1}}\ \mathcal{R}_{n-1}\ 
						\newsp{\tilde{s'}}{Q_2 \Par \map{S}^{s_2}}
					\]


				\item	$\forall \lambda \not= \news{\tilde{s}} \bactout{s}{\abs{x} P}$ such that
					\[
						\Gamma; \emptyset; \Sigma_1 \by{\lambda} \Sigma_1' \proves P_1 \by{\lambda} P_2
					\]
					$\exists Q_2$ such that 
					\[
						\Gamma; \emptyset; \Sigma_2 \by{\hat{\lambda}} \Sigma_2' \proves Q_1 \By{\hat{\lambda}} Q_2
					\]
					and
					$\Gamma; \emptyset; \Sigma_1'\ \mathcal{R}_{n-1}\ \Sigma_2' \proves P_2\ \mathcal{R}_{n-1}\ Q_2$.

				\item	The symmetric cases of 1, 2, 3.
			\end{enumerate}
	\end{itemize}
	\noi The above function is monotone and the Knaster-Tarski theorem ensures that a lattice is define
	with the largest $\mathcal{R}_n$ to be denote as $\wb_n$ and the largest fix-point is equal to the
	bisimilarity relation $\wb = \bigcap_{i \geq 0} \wb_i$.
\end{definition}


\begin{lemma}\rm
	$\wbc\ \subseteq\ \wb$
\end{lemma}

\begin{proof}
	Statement $\wbc \subseteq \wb$
	is equivalent to the statement $\forall n, \wbc_n \subseteq \wb_n$.
	From here the proof is done using induction on the definitions of $\wbc$.

	\noi {\bf Basic step:}
	From the definitions of $\wbc_0$ and $\wb_0$ we get that $\wbc_0 = \wb_0$.

	\noi {\bf Induction hypothesis:}
	$\wbc_n\ \subseteq\ \wb_n$.

	\noi {\bf Inductive step:}
	Let
%
	\begin{eqnarray*}
		\Gamma; \emptyset; \Sigma_1\ \wbc_{n+1}\ \Sigma_2 \proves P_1\ \wbc_{n+1}\ Q_1
	\end{eqnarray*}
%
	\noi We perform a case analysis on transition $\by\lambda$.

	%%%%%%%%%%%%%%%%%%%%%%%%%%%%%%%%%%%%%%%%%%%%%%%

	\noi - Case: $\lambda \notin \set{\news{\tilde{s}} \bactout{s}{\abs{x} P}, \news{\tilde{s}} \bactout{s}{s_1}}$
%
	\begin{eqnarray}
		\Gamma; \emptyset; \Sigma_1 \by{\lambda} \Sigma_1' \proves P_1 \by{\lambda} P_2 \label{lem:wbc_is_wb1}
	\end{eqnarray}
%
	\noi implies that 
	$\exists Q_1$ such that
%
	\begin{eqnarray}
		\Gamma; \emptyset; \Sigma_2 \by{\lambda} \Sigma_2' &\proves& Q_1 \by{\lambda} Q_2 \label{lem:wbc_is_wb2}\\
		\Gamma; \emptyset; \Sigma_1'\ \wbc_n\ \Sigma_2' &\proves& P_1\ \wbc_n\ Q_2
	\end{eqnarray}
%
	We apply the induction hypothesis to the latter judgement:
%
	\begin{eqnarray}
		\Gamma; \emptyset; \Sigma_1' \proves P_2\ \wb_n\ \Gamma; \emptyset; \Sigma_2' \proves Q_2 \hastype \Proc  \label{lem:wbc_is_wb3}
	\end{eqnarray}
%
	Assume $\mathcal{R}_{n+1} = \set{\Gamma; \emptyset; \Sigma_1 \proves P_1 \hastype \Proc, \Gamma; \emptyset; \Sigma_2 \proves Q_1 \hastype \Proc}$.

	\noi $\mathcal{R}_{n+1}$ satisfies the condition for the stratified definition of bisimulation
	because statement~\ref{lem:wbc_is_wb1} implies $\exists Q'$ such that
	statements~\ref{lem:wbc_is_wb2} holds and furthermore statement~\ref{lem:wbc_is_wb3} holds
%	if $\Gamma; \emptyset; \Sigma_1 \proves P \by{\lambda} \Gamma; \emptyset; \Sigma_1' \proves P' \hastype \Proc$ then
%	$\exists Q'$ such that
%	$\Gamma; \emptyset; \Sigma_2 \proves Q \by{\lambda} \Gamma; \emptyset; \Sigma_2' \proves Q' \hastype \Proc$
%	and \ref{pr:biscong_is_bis2}.

	\noi Because $\wb_{n+1}$ is the largest relation we get that $\mathcal{R}_{n+1} \subseteq \wb_{n+1}$ as required.

	%%%%%%%%%%%%%%%%%%%%%%%%%%%%%%%%%%%%%%%%%%%%%%%

	\noi - Case: $\lambda = \news{\tilde{s}} \bactout{s}{\abs{x} P}$
%
	\begin{eqnarray}
		\Gamma; \emptyset; \Sigma_1 \by{\news{\tilde{s}} \bactout{s}{\abs{x} P}} \Sigma_1' \proves P_1 \by{\news{\tilde{s}} \bactout{s}{\abs{x} P}} P_2 \label{lem:wbc_is_wb4}
	\end{eqnarray}
%
	\noi implies that
	$\exists Q_2, \abs{x}{Q}$ such that
	\begin{eqnarray}
		\Gamma; \emptyset; \Sigma_2 \By{\news{\tilde{s'}} \bactout{s}{\abs{x} Q}} \Sigma_2' \proves Q_1 \By{\news{\tilde{s'}} \bactout{s}{\abs{x} Q}} Q_2  \label{lem:wbc_is_wb5}
	\end{eqnarray}
	and $\forall C, s'$
%	such that
%	\begin{eqnarray*}
%		\Gamma; \emptyset; \Sigma_1'' \proves \newsp{\tilde{s}}{\context{C}{P' \Par P \subst{s'}{x}}} \hastype \Proc \\
%		\Gamma; \emptyset; \Sigma_2'' \proves \newsp{\tilde{s}}{\context{C}{Q' \Par Q \subst{s'}{x}}} \hastype \Proc
%	\end{eqnarray*}
%	then
%
	\begin{eqnarray*}
		\Gamma; \emptyset; \Sigma_1''\ \wbc_{n}\ \Sigma_2'' \proves \newsp{\tilde{s}}{\context{C}{P_2 \Par P \subst{s'}{x}}}\ \wbc_{n}\ 
		\newsp{\tilde{s'}}{\context{C}{Q_2 \Par Q \subst{s'}{x}}}
	\end{eqnarray*}
%
	\noi For $C = \hole$ we have that 
%
	\begin{eqnarray*}
		\Gamma; \emptyset; \Sigma_1''\ \wbc_{n}\ \Sigma_2'' \proves \newsp{\tilde{s}}{P_2 \Par P \subst{s'}{x}}\ \wbc_{n}\ 
		\newsp{\tilde{s}}{Q' \Par Q_2 \subst{s'}{x}}
	\end{eqnarray*}
%
	\dk{(prove that it is typable)}

	\noi If we apply the induction hypothesis to the latter statement we get
%
	\begin{eqnarray}
		\Gamma; \emptyset; \Sigma_1''\ \wb_{n}\ \Sigma_2'' \proves \newsp{\tilde{s}}{P_2 \Par P \subst{s'}{x}}\ \wb_{n}\ 
		\newsp{\tilde{s}}{Q_2 \Par Q \subst{s'}{x}}
		\label{lem:wbc_is_wb6}
	\end{eqnarray}
%
	\noi Assume $\mathcal{R}_{n+1} = \set{\Gamma; \emptyset; \Sigma_1 \proves P_1 \hastype \Proc, \Gamma; \emptyset; \Sigma_2 \proves Q_2 \hastype \Proc}$.

	\noi $\mathcal{R}_{n+1}$ satisfies the condition for the stratified definition for bisimulation
	because statement~\ref{lem:wbc_is_wb4} implies that
	$\exists Q', \abs{x}{Q}$ such that
	statement~\ref{lem:wbc_is_wb5} holds and furthemore statement~\ref{lem:wbc_is_wb6} holds.

	\noi Because $\wb_{n+1}$ is the largest relation we get that $\mathcal{R}_{n+1} \subseteq \wb_{n+1}$ as required.

	%%%%%%%%%%%%%%%%%%%%%%%%%%%%%%%%%%%%%%%%%%%%%%%

	\noi - Case: $\lambda = \news{\tilde{s}} \bactout{s}{s_1}$
%
	\begin{eqnarray}
		\Gamma; \emptyset; \Sigma_1 \by{\news{\tilde{s}} \bactout{s}{s_1}} \Sigma_1' \proves P_1 \by{\news{\tilde{s}} \bactout{s}{s_1}} P_2 \label{lem:wbc_is_wb7}
	\end{eqnarray}
%
	\noi \dk{with $s_1: S \in \Sigma_1 \vee (\dual{s_1}: S' \in \Sigma_1' \wedge S \dualof S')$} implies that
	$\exists Q', s_2$ such that
	\begin{eqnarray}
		\Gamma; \emptyset; \Sigma_2 \By{\news{\tilde{s'}} \bactout{s}{s_2}} \Sigma_2' \proves Q_1 \By{\news{\tilde{s'}} \bactout{s}{s_2}} Q_2 \label{lem:wbc_is_wb8}
	\end{eqnarray}
	and $\forall R$ with $\set{x} = \fn{P}$
%	such that
%	\begin{eqnarray*}
%		\Gamma; \emptyset; \Sigma_1'' \proves \newsp{\tilde{s}}{\context{C}{P' \Par P \subst{s'}{x}}} \hastype \Proc \\
%		\Gamma; \emptyset; \Sigma_2'' \proves \newsp{\tilde{s}}{\context{C}{Q' \Par Q \subst{s'}{x}}} \hastype \Proc
%	\end{eqnarray*}
%	then
%
	\begin{eqnarray*}
		\Gamma; \emptyset; \Sigma_1''\ \wbc_{n}\ \Sigma_2'' \proves \newsp{\tilde{s}}{P_2 \Par R \subst{s_1}{x}}\ \wbc_{n}\ 
		\newsp{\tilde{s'}}{Q_2 \Par R \subst{s_2}{x}}
	\end{eqnarray*}
%
	\noi From the latter statement we get
	\begin{eqnarray*}
		\Gamma; \emptyset; \Sigma_1''\ \wbc_{n}\ \Sigma_2'' \proves \newsp{\tilde{s}}{P_2 \Par \map{S}^{x} \subst{s_1}{x}}\ \wbc_{n}\ 
		\newsp{\tilde{s'}}{Q_2 \Par \map{S}^{x} \subst{s_2}{x}} \label{lem:wbc_is_wb9}
	\end{eqnarray*}
%
	\noi Assume $\mathcal{R}_{n+1} = \set{\Gamma; \emptyset; \Sigma_1 \proves P_1 \hastype \Proc, \Gamma; \emptyset; \Sigma_2 \proves Q_1 \hastype \Proc}$.

	\noi $\mathcal{R}_{n+1}$ satisfies the condition for the stratified definition for bisimulation
	because statement~\ref{lem:wbc_is_wb7} implies that
	$\exists Q', s_2$ such that
	statement~\ref{lem:wbc_is_wb8} holds and furthemore statement~\ref{lem:wbc_is_wb9} holds.
\end{proof}

\begin{lemma}\rm
	$\wb\ \subseteq\ \wbc$
\end{lemma}

\begin{proof}
	The statement $\wb \subseteq \wbc$.
	is equivalent to the statement $\forall n, \wb_n \subseteq \wbc_n$.
	The proof is done using induction on the definition $\wb$.

	\noi {\bf Basic step:} From the definitions of $\wbc_0$ and $\wb_0$ we get that $\wbc_0 = \wb_0$.

	\noi {\bf Induction hypothesis:} $\wb_n \subseteq \wbc_n$.

	\noi {\bf Inductive step:}
	Let 
	\[
		\Gamma; \emptyset; \Sigma_1\ \wb_{n+1}\ \Sigma_2 \proves P_1\ \wb_{n+1}\ Q_1
	\]
	We perform a case analysis on transition $\by{\lambda}$.

	\noi - Case: $\lambda \notin \set{\news{\tilde{s}} \bactout{s}{\abs{x} P}, \news{\tilde{s}} \bactout{s}{s_1}}$

	\noi Same arguments with the same case of the direction $\wbc \subseteq \wb$

	\noi - Case: $\lambda = \news{\tilde{s}} \bactout{s}{\abs{x} P}$
%
	\begin{eqnarray*}
		\Gamma; \emptyset; \Sigma_1 \by{\news{\tilde{s}} \bactout{s}{\abs{x} P}} \Sigma_1' \proves P_1 \by{\news{\tilde{s}} \bactout{s}{\abs{x} P}} P_2
	\end{eqnarray*}
%
	implies that
	$\exists Q_2, \abs{x}{Q}$ such that
	$\Gamma; \emptyset; \Sigma_2 \proves Q \by{\news{\tilde{s}} \bactout{s}{\abs{x} Q}} \Gamma; \emptyset; \Sigma_2' \proves Q' \hastype \Proc$
	and $s'$
	such that
	\begin{eqnarray*}
		\Gamma; \emptyset; \Sigma_1'' \proves \newsp{\tilde{s}}{P' \Par P \subst{s'}{x}} \hastype \Proc \\
		\Gamma; \emptyset; \Sigma_2'' \proves \newsp{\tilde{s}}{Q' \Par Q \subst{s'}{x}} \hastype \Proc
	\end{eqnarray*}
	then
	\[
		\Gamma; \emptyset; \Sigma_1'' \proves \newsp{\tilde{s}}{P' \Par P \subst{s'}{x}}\ \wb_{n}\ 
		\Gamma; \emptyset; \Sigma_2'' \proves \newsp{\tilde{s}}{Q' \Par Q \subst{s'}{x}} \hastype \Proc
	\]

	Let
	\[
		\begin{array}{rcl}
			R_{n + 1} &=& \set{	(\Gamma; \emptyset; \Sigma_1 \proves \newsp{\tilde{s_1}}{\bout{s}{\abs{x}{\context{C}{P}}} P''} \hastype \Proc,
						\Gamma; \emptyset; \Sigma_2 \proves \newsp{\tilde{s_2}}{\bout{s}{\abs{x}{\context{C}{P}}} Q''} \hastype \Proc) \setbar\\
				& &		\forall C, P' \scong \news{\tilde{s_1}} P'', Q' \scong \news{\tilde{s_2}} Q''}
		\end{array}
	\]

	\dk{prove that $R_{n+1}$ is typable}

	$R_{n+1}$ satisfies the stratified definition for bisimulation and furthermore we can deduce that
	$\forall C, s'$
	such that
	\begin{eqnarray*}
		\Gamma; \emptyset; \Sigma_1'' \proves \newsp{\tilde{s}}{\context{C}{P' \Par P \subst{s'}{x}}} \hastype \Proc \\
		\Gamma; \emptyset; \Sigma_2'' \proves \newsp{\tilde{s}}{\context{C}{Q' \Par Q \subst{s'}{x}}} \hastype \Proc
	\end{eqnarray*}
	then
	\[
		\Gamma; \emptyset; \Sigma_1'' \proves \newsp{\tilde{s}}{\context{C}{P' \Par P \subst{s'}{x}}}\ \wb_{n}\ 
		\Gamma; \emptyset; \Sigma_2'' \proves \newsp{\tilde{s}}{\context{C}{Q' \Par Q \subst{s'}{x}}} \hastype \Proc
	\]

	From the induction hypothesis we get that
	$\forall C, s'$
	such that
	\begin{eqnarray*}
		\Gamma; \emptyset; \Sigma_1'' \proves \newsp{\tilde{s}}{\context{C}{P' \Par P \subst{s'}{x}}} \hastype \Proc \\
		\Gamma; \emptyset; \Sigma_2'' \proves \newsp{\tilde{s}}{\context{C}{Q' \Par Q \subst{s'}{x}}} \hastype \Proc
	\end{eqnarray*}
	then
	\begin{eqnarray}
		\Gamma; \emptyset; \Sigma_1'' \proves \newsp{\tilde{s}}{\context{C}{P' \Par P \subst{s'}{x}}}\ \wbc_{n}\ 
		\Gamma; \emptyset; \Sigma_2'' \proves \newsp{\tilde{s}}{\context{C}{Q' \Par Q \subst{s'}{x}}} \hastype \Proc
		\label{pr:bis_is_contextbis}
	\end{eqnarray}

%	For $C = \hole$ we have that 
%	\[
%		\Gamma; \emptyset; \Sigma_1'' \proves \newsp{\tilde{s}}{P_2 \Par P \subst{s'}{x}}\ \wbc_{n}\ 
%		\Gamma; \emptyset; \Sigma_2'' \proves \newsp{\tilde{s}}{Q_2 \Par Q \subst{s'}{x}} \hastype \Proc
%	\]

	Assume $R^c_{n+1} = \set{\Gamma; \emptyset; \Sigma_1 \proves P \hastype \Proc, \Gamma; \emptyset; \Sigma_2 \proves Q \hastype \Proc}$.

	$R^c_{n+1}$ satisfies the condition for the stratified definition of the contextual bisimulation because of the fact that
	if $\Gamma; \emptyset; \Sigma_1 \proves P \by{\news{\tilde{s}} \bactout{s}{\abs{x} P}} \Gamma; \emptyset; \Sigma_1' \proves P' \hastype \Proc$ then
	$\exists Q', \abs{x}{Q}$ such that
	$\Gamma; \emptyset; \Sigma_2 \proves Q \by{\news{\tilde{s}} \bactout{s}{\abs{x} Q}} \Gamma; \emptyset; \Sigma_2' \proves Q' \hastype \Proc$
	and \ref{pr:bis_is_contextbis}.

	Because $\wbc_{n+1}$ is the largest relation we get that $R^c_{n+1} \subseteq \wb_{n+1}$ as required.
\end{proof}


% !TEX root = ../main.tex
\section{Encoding Semantics}



\begin{comment}
%%%%%%%%%%%%%%%%%%%%%%%%%%%%%%%%%%%%%%%%%%%%%%%%%
% POLYADIC TO MONADIC
%%%%%%%%%%%%%%%%%%%%%%%%%%%%%%%%%%%%%%%%%%%%%%%%%

\subsection{Properties for $\encod{\cdot}{\cdot}{\mathsf{p}}$}
\label{app:polmon}
We study the properties of the typed encoding in
Def.~\ref{d:enc:poltomon} (Page~\pageref{d:enc:poltomon}).

We repeat the statement of Prop.~\ref{prop:typepresp}, as in Page \pageref{prop:typepresp}:
\begin{proposition}[Type Preservation, Polyadic to Monadic]
Let $P$ be an  $\HOp$ process.
If			$\Gamma; \emptyset; \Delta \proves P \hastype \Proc$ then 
			$\mapt{\Gamma}^{\mathsf{p}}; \emptyset; \mapt{\Delta}^{\mathsf{p}} \proves \map{P}^{\mathsf{p}} \hastype \Proc$. 
\end{proposition}

\begin{proof}
By induction on the inference $\Gamma; \emptyset; \Delta \proves P \hastype \Proc$.
We examine two representative cases, using biadic communications.

\begin{enumerate}[1.]
\item Case 
$P = \bout{k}{V} P'$ and 
$\Gamma; \emptyset; \Delta_1 \cat \Delta_2 \cat k:\btout{\lhot{(C_1,C_2)}} S \proves \bout{k}{V} P' \hastype \Proc$. Then either $V = Y$ or $V = \abs{x_1,x_2}Q$, for some $Q$. The case $V = Y$ is immediate; we give details for the case $V = \abs{x_1,x_2}Q$, for which we have the following typing:
\[
\tree{
\tree{}{
\Gamma; \emptyset; \Delta_1 \cat k:S \proves P' \hastype \Proc}
\quad
\tree{
\Gamma; \emptyset; \Delta_2 \cat x_1: C_1 \cat x_2:C_2 \proves Q \hastype \Proc
}{
\Gamma; \emptyset; \Delta_2 \proves \abs{x_1,x_2}Q \hastype \lhot{(C_1,C_2)}
}
}{
\Gamma; \emptyset; \Delta_1 \cat \Delta_2 \cat k:\btout{\lhot{(C_1,C_2)}} S \proves \bout{k}{\abs{x_1,x_2}Q} P \hastype \Proc
}
\]
We now show the typing for $\map{P}^{\mathsf{p}}$. By IH we have both:
\[
\mapt{\Gamma}^{\mathsf{p}}; \emptyset; \mapt{\Delta_1}^{\mathsf{p}} \cat k:\mapt{S}^{\mathsf{p}} \proves \map{P'}^{\mathsf{p}} \hastype \Proc
\qquad
\mapt{\Gamma}^{\mathsf{p}}; \emptyset; \mapt{\Delta_2}^{\mathsf{p}} \cat x_1: \mapt{C_1}^{\mathsf{p}} \cat x_2:\mapt{C_2}^{\mathsf{p}} \proves \map{Q}^{\mathsf{p}} \hastype \Proc
\]
Let $L = \lhot{(C_1,C_2)}$. 
By Def.~\ref{d:enc:poltomon} 
we have  
$\mapt{L}^{\mathsf{p}} = \lhot{\big(\btinp{\tmap{C_1}{\mathsf{p}}} \btinp{\tmap{C_2}{\mathsf{p}}}\tinact\big)}$
and
$\map{P}^{\mathsf{p}} = \bbout{k}{\abs{z}\binp{z}{x_1}\binp{z}{x_2} \map{Q}^{\mathsf{p}}} \map{P'}^{\mathsf{p}}$.
We can now infer the following typing derivation:
\[
\tree{
\tree{}
{
\mapt{\Gamma}^{\mathsf{p}}; \emptyset; \mapt{\Delta_1}^{\mathsf{p}} \cat k:\mapt{S}^{\mathsf{p}} \proves \map{P'}^{\mathsf{p}} \hastype \Proc}
\quad
\tree{
\tree{
\tree{
\tree{
\tree{}{\mapt{\Gamma}^{\mathsf{p}}; \emptyset; \mapt{\Delta_2}^{\mathsf{p}} \cat x_1: \tmap{C_1}{\mathsf{p}} \cat x_2: \tmap{C_2}{\mathsf{p}} \proves 
 \map{Q}^{\mathsf{p}} \hastype \Proc}
}{
\mapt{\Gamma}^{\mathsf{p}}; \emptyset; \mapt{\Delta_2}^{\mathsf{p}} \cat x_1: \tmap{C_1}{\mathsf{p}} \cat x_2: \tmap{C_2}{\mathsf{p}}
\cat z:\tinact \proves 
 \map{Q}^{\mathsf{p}} \hastype \Proc
}
}{
\mapt{\Gamma}^{\mathsf{p}}; \emptyset; \mapt{\Delta_2}^{\mathsf{p}} \cat x_1: \tmap{C_1}{\mathsf{p}}\cat z:\btinp{\tmap{C_2}{\mathsf{p}}}\tinact \proves 
\binp{z}{x_2} \map{Q}^{\mathsf{p}} \hastype \Proc
}
}{
\mapt{\Gamma}^{\mathsf{p}}; \emptyset; \mapt{\Delta_2}^{\mathsf{p}} \cat z:\btinp{\tmap{C_1}{\mathsf{p}}}\btinp{\tmap{C_2}{\mathsf{p}}}\tinact \proves 
\binp{z}{x_1}\binp{z}{x_2} \map{Q}^{\mathsf{p}} \hastype \Proc
}
}{
\mapt{\Gamma}^{\mathsf{p}}; \emptyset; \mapt{\Delta_2}^{\mathsf{p}}  \proves 
\abs{z}\binp{z}{x_1}\binp{z}{x_2} \map{Q}^{\mathsf{p}} \hastype \lhot{(\tmap{C_1}{\mathsf{p}},\tmap{C_2}{\mathsf{p}})}
}
}{
\mapt{\Gamma}^{\mathsf{p}}; \emptyset; \mapt{\Delta_1}^{\mathsf{p}} \cat \mapt{\Delta_2}^{\mathsf{p}} \cat k:\btout{\mapt{L}^{\mathsf{p}}} \mapt{S}^{\mathsf{p}} \proves \map{P}^{\mathsf{p}} \hastype \Proc
}
\]

\item Case $P = \binp{k}{x_1,x_2} P'$ 
and
$\Gamma; \emptyset; \Delta_1 \cat k: \btinp{(C_1, C_2)} S \proves \binp{k}{x_1,x_2} P' \hastype \Proc$.
We have the following typing derivation:
\[
\tree{
\Gamma; \emptyset; \Delta_1 \cat k:S \cat x_1: C_1 \cat x_2: C_2 \proves  P' \hastype \Proc
\quad
\Gamma; \emptyset;  \proves x_1, x_2 \hastype C_1,C_2
}{
\Gamma; \emptyset; \Delta_1 \cat k: \btinp{(C_1, C_2)} S \proves \binp{k}{x_1,x_2} P' \hastype \Proc
}
\]
By Def.~\ref{d:enc:poltomon} we have 
$\map{P}^{\mathsf{p}} = \binp{k}{x_1}\binp{k}{x_2}\map{P'}^{\mathsf{p}}$.
By IH we have 
$$
\mapt{\Gamma}^{\mathsf{p}}; \emptyset; \mapt{\Delta_1}^{\mathsf{p}} \cat k:\mapt{S}^{\mathsf{p}} \cat x_1: \tmap{C_1}{\mathsf{p}} \cat x_2: \tmap{C_2}{\mathsf{p}} \proves  \map{P'}^{\mathsf{p}} \hastype \Proc
$$
and the following type derivation:
\[
\tree{
\tree{
\tree{
}{
\mapt{\Gamma}^{\mathsf{p}}; \emptyset; \mapt{\Delta_1}^{\mathsf{p}} \cat x_1:\tmap{C_1}{\mathsf{p}} \cat x_2:\tmap{C_2}{\mathsf{p}} \cat k:\mapt{S}^{\mathsf{p}}  \proves  \map{P'}^{\mathsf{p}} \hastype \Proc
}
%\quad
%\tree{}{
%\mapt{\Gamma}^{\mathsf{p}}; \emptyset; x_2:\tmap{C_2}{\mathsf{p}}  \proves  x_2 \hastype \tmap{C_2}{\mathsf{p}}}
}{
\mapt{\Gamma}^{\mathsf{p}}; \emptyset; \mapt{\Delta_1}^{\mathsf{p}} \cat x_1:\tmap{C_1}{\mathsf{p}} \cat k:\btinp{\tmap{C_2}{\mathsf{p}}}\mapt{S}^{\mathsf{p}}  \proves  \binp{k}{x_2}\map{P'}^{\mathsf{p}} \hastype \Proc
}
%\quad
%\tree{}{
%\mapt{\Gamma}^{\mathsf{p}}; \emptyset; x_1:\tmap{C_1}{\mathsf{p}}  \proves  x_1 \hastype \tmap{C_1}{\mathsf{p}}}
}{
\mapt{\Gamma}^{\mathsf{p}}; \emptyset; \mapt{\Delta_1}^{\mathsf{p}} \cat k:\btinp{\tmap{C_1}{\mathsf{p}}}\btinp{\tmap{C_2}{\mathsf{p}}}\mapt{S}^{\mathsf{p}}  \proves  \map{P}^{\mathsf{p}} \hastype \Proc
}
\]
\end{enumerate}
\qed
\end{proof}

\end{comment}

%\subsection{Properties for $\encod{\cdot}{\cdot}{1}: \sessp^{-\mu} \to \HO$}
%\label{app:enc_sesspnr_to_ho}

%\begin{proposition}\rm
%	\label{app:enc_sesspnr_to_ho_typing}
%	Encoding $\encod{\cdot}{\cdot}{1}: \sessp^{-\mu} \to \HO$  is type-preserving (cf. Def.~\ref{def:ep}\,(1)).\rm
%\end{proposition}

%We repeat the statement of Prop.~\ref{prop:typepres1}, as in Page \pageref{prop:typepres1}:

%\begin{proposition}[Type Preservation, First-Order into Higher-Order]
%Let $P$ be a  $\sessp^{-\mu}$ process.
%If			$\Gamma; \emptyset; \Delta \proves P \hastype \Proc$ then 
%			$\mapt{\Gamma}^{1}; \emptyset; \mapt{\Delta}^{1} \proves \map{P}^{1} \hastype \Proc$. 
%\end{proposition}

%%%%%%%%%%%%%%%%%%%%%%%%%%%%%%%%%%%%%%%%%%%%%%%%%
% HOp TO HO
%%%%%%%%%%%%%%%%%%%%%%%%%%%%%%%%%%%%%%%%%%%%%%%%%

\subsection{Properties for $\enco{\pmapp{\cdot}{1}{f}, \tmap{\cdot}{1}, \mapa{\cdot}^{1}}: \HOp \to \HO$}
\label{app:enc_HOp_to_HO}

We repeat the statement of Prop.~\ref{prop:typepres_HOp_to_HO}, 
as in Page \pageref{prop:typepres_HOp_to_HO}:

%\begin{proposition}[Type Preservation, Full First-Order into Higher-Order]
%	Let $P$ be a  $\sessp$ process.
%	If	$\Gamma; \emptyset; \Delta \proves P \hastype \Proc$ then 
%		$\mapt{\Gamma}^{2}; \emptyset; \mapt{\Delta}^{2} \proves \map{P}_f^{2} \hastype \Proc$. 
%\end{proposition}

\begin{proposition}[Type Preservation, \HOp into \HO]
	Let $P$ be a \HOp process.
	If $\Gamma; \emptyset; \Delta \proves P \hastype \Proc$ then 
	$\mapt{\Gamma}^{1}; \emptyset; \mapt{\Delta}^{1} \proves \pmapp{P}{1}{f} \hastype \Proc$. 
\end{proposition}

\begin{proof}
	By induction on the   inference of $\Gamma; \emptyset; \Delta \proves P \hastype \Proc$. %\jp{TO BE ADJUSTED!}
%
	\begin{enumerate}[1.]
		%%%% Output of (linear) channel
		\item	Case $P = \bout{k}{n}P'$. There are two sub-cases.
			In the first sub-case $n = k'$ (output of a linear channel). Then  
			we have the following typing in the source language:
			{
			\[
				\tree{
					\Gamma; \emptyset; \Delta \cat k:S  \proves  P' \hastype \Proc \quad \Gamma ; \emptyset ; \{k' : S_1\} \proves  k' \hastype S_1}{
					\Gamma; \emptyset; \Delta \cat k':S_1 \cat k:\btout{S_1}S \proves  \bout{k}{k'} P' \hastype \Proc}
			\]
			}
			Thus, by IH we have
			$$
			\tmap{\Gamma}{1}; \emptyset ; \tmap{\Delta}{1} \cat k:\tmap{S}{1} \proves \pmap{P'}{1} \hastype \Proc
			$$
			Let us write $U_1$
			to stand for $\lhot{\btinp{\lhot{\tmap{S_1}{1}}}\tinact}$.
			The corresponding typing in the target language is as follows:
			\begin{eqnarray}
				\label{prop:sesspnr_to_HO_t1}
				\tree{
					\tree{
						\tree{
							\tree{
								\tmap{\Gamma}{1} ; \set{X : \lhot{\tmap{S_1}{1}}} ; \emptyset \proves \X  \hastype \lhot{\tmap{S_1}{1}}
								\qquad 
								\tmap{\Gamma}{1} ; \emptyset ; \set{k' : \tmap{S_1}{1}} \proves  k' \hastype \tmap{S_1}{1}
							}{
								\tmap{\Gamma}{1} ; \set{X : \lhot{\tmap{S_1}{1}}} ; k' : \tmap{S_1}{1} \proves \appl{\X}{k'} \hastype \Proc
							}
						}{
							\tmap{\Gamma}{1} ; \{X : \lhot{\tmap{S_1}{1}}\} ; k' : \tmap{S_1}{1} \cat z:\tinact \proves \appl{\X}{k'} \hastype \Proc
						}
					}{
						\tmap{\Gamma}{1} ; \emptyset; k' : \tmap{S_1}{1} \cat z:\btinp{\lhot{\tmap{S_1}{1}}}\tinact \proves \binp{z}{X} \appl{\X}{k'} \hastype \Proc
					}
				}{
					\tmap{\Gamma}{1} ; \emptyset; k' : \tmap{S_1}{1} \proves \abs{z}{\binp{z}{X} \appl{\X}{k'}} \hastype U_1
				}
			\end{eqnarray}
			\begin{eqnarray*}
				\tree{
					\tmap{\Gamma}{1}; \emptyset ; \tmap{\Delta}{1} \cat k:\tmap{S}{1} \proves \pmap{P'}{1} \hastype \Proc
					\qquad
					\tmap{\Gamma}{1} ; \emptyset; k' : \tmap{S_1}{1} \proves \abs{z}{\binp{z}{X} \appl{\X}{k'}} \hastype U_1 \ \eqref{prop:sesspnr_to_HO_t1}
				}{
					\tmap{\Gamma}{1}; \emptyset; \tmap{\Delta}{1} \cat k':\tmap{S_1}{1} \cat k:\btout{U_1}\tmap{S}{1} \proves  \bbout{k}{\abs{z}{\binp{z}{X} \appl{\X}{k'}}} \pmap{P'}{1} \hastype \Proc
				}
			\end{eqnarray*}
%
	
		%%%% Output of (shared) channel
			In the second sub-case, we have $n = a$ (output of a shared name). Then  
			we have the following typing in the source language:
			{
			\[
				\tree{
					\Gamma \cat a:\chtype{S_1}; \emptyset; \Delta \cat k:S  \proves
					P' \hastype \Proc \quad \Gamma \cat a:\chtype{S_1} ; \emptyset ; \emptyset \proves  a \hastype S_1
				}{
					\Gamma \cat a:\chtype{S_1} ; \emptyset; \Delta  \cat k:\bbtout{\chtype{S_1}}S \proves  \bout{k}{a} P' \hastype \Proc
				}
			\]
			}
			The typing in the target language is derived similarly as in the first sub-case. \\
	
		%%%% Input of (linear) channel 
		\item	Case $P = \binp{k}{x}Q$. We have two sub-cases, depending on the type of $x$.
			In the first case, $x$ stands for a linear channel.
			Then we have the following typing in the source language:
			{
			\[
				\tree{
					\Gamma; \emptyset; \Delta  \cat k:S \cat x:S_1 \proves   Q \hastype \Proc
				}{
					\Gamma; \emptyset; \Delta  \cat k:\btinp{S_1}S \proves  \binp{k}{x} Q \hastype \Proc
				}
			\]
			 }
			 Thus, by IH we have
			 $$
			 \tmap{\Gamma}{1}; \emptyset;  \tmap{\Delta}{1} \cat k:\tmap{S}{1}  \cat x:\tmap{S_1}{1} \proves  \pmap{Q}{1}   \hastype \Proc
			 $$
			 Let us write $U_1$ to stand for $\lhot{\btinp{\lhot{\tmap{S_1}{1}}}\tinact}$.
			 The corresponding typing in the target language is as follows:
			{\small
%
			\begin{eqnarray}
				\label{prop:sesspnr_to_HO_t2}
				\tree{
					\tmap{\Gamma}{1}; \{X: U_1\};   \emptyset \proves X \hastype U_1
					\qquad
					\tmap{\Gamma}{1}; \emptyset;   \cat s: \btinp{\lhot{\tmap{S_1}{1}}}\tinact \ \proves s \, \hastype  \btinp{\lhot{\tmap{S_1}{1}}} \tinact 
				}{
					\tmap{\Gamma}{1}; \{X: U_1\};   \cat s: \btinp{\lhot{\tmap{S_1}{1}}}\tinact \ \proves \appl{X}{s}  \hastype \Proc
				}
			\end{eqnarray}
%
			\begin{eqnarray}
				\label{prop:sesspnr_to_HO_t3}
				\tree{
					\tree{
						\tmap{\Gamma}{1}; \emptyset;  \emptyset \proves   \inact  \hastype \Proc
					}{
						\tmap{\Gamma}{1}; \emptyset;  \dual{s}: \tinact\proves   \inact  \hastype \Proc
					}
					\quad 
					\tree{
						\tmap{\Gamma}{1}; \emptyset;  \tmap{\Delta}{1} \cat k:\tmap{S}{1}  x:\tmap{S_1}{1} \proves \pmap{Q}{1}   \hastype \Proc
					}{
						\tmap{\Gamma}{1}; \emptyset;  \tmap{\Delta}{1} \cat k:\tmap{S}{1}   \proves \abs{x} \pmap{Q}{1}   \hastype \lhot{\tmap{S_1}{1}}
					}
				}{
					\tmap{\Gamma}{1}; \emptyset;  \tmap{\Delta}{1} \cat k:\tmap{S}{1}  \cat \dual{s}: \btout{\lhot{\tmap{S_1}{1}}}\tinact\proves  \bbout{\dual{s}}{\abs{x}{\pmap{Q}{1}}} \inact  \hastype \Proc
				}
			\end{eqnarray}
%
			\begin{eqnarray}
				\label{prop:sesspnr_to_HO_t4}
		 		\tree{
					\begin{array}{cl}
						\tmap{\Gamma}{1}; \{X: U_1\}; \cat s: \btinp{\lhot{\tmap{S_1}{1}}}\tinact \ \proves \appl{X}{s}  \hastype \Proc
						& \eqref{prop:sesspnr_to_HO_t2}
						\\
						\tmap{\Gamma}{1}; \emptyset; \tmap{\Delta}{1} \cat k:\tmap{S}{1} \cat \dual{s}: \btout{\lhot{\tmap{S_1}{1}}}\tinact \proves
						\bbout{\dual{s}}{\abs{x}{\pmap{Q}{1}}} \inact  \hastype \Proc
						& \eqref{prop:sesspnr_to_HO_t3}
					\end{array}
				}{
					\tmap{\Gamma}{1}; \{X: U_1\};  \tmap{\Delta}{1} \cat k:\tmap{S}{1} \cat s: \btinp{\lhot{\tmap{S_1}{1}}}\tinact \cat \dual{s}: \btout{\lhot{\tmap{S_1}{1}}}\tinact\proves \appl{X}{s} \Par \bbout{\dual{s}}{\abs{x}{\pmap{Q}{1}}} \inact  \hastype \Proc
			}
			\end{eqnarray}
%
			\begin{eqnarray*}
			\\
			 \tree{
				 \tree{
					\tmap{\Gamma}{1}; \{X: U_1\};  \tmap{\Delta}{1} \cat k:\tmap{S}{1} \cat s: \btinp{\lhot{\tmap{S_1}{1}}}\tinact \cat \dual{s}: \btout{\lhot{\tmap{S_1}{1}}}\tinact\proves \appl{X}{s} \Par \bbout{\dual{s}}{\abs{x}{\pmap{Q}{1}}} \inact  \hastype \Proc \quad \eqref{prop:sesspnr_to_HO_t4}
				}{
					\tmap{\Gamma}{1}; \{X: U_1\};  \tmap{\Delta}{1} \cat k:\tmap{S}{1} \proves \newsp{s}{\appl{X}{s} \Par \bbout{\dual{s}}{\abs{x}{\pmap{Q}{1}}} \inact}  \hastype \Proc
				}
			}{
				\tmap{\Gamma}{1}; \emptyset; \tmap{\Delta}{1}  \cat k:\btinp{U_1}\tmap{S}{1} \proves  \binp{k}{X} \newsp{s}{\appl{X}{s} \Par \bbout{\dual{s}}{\abs{x}{\pmap{Q}{1}}} \inact}  \hastype \Proc
			}
			\end{eqnarray*}
			 }
			 
			 In the second sub-case, $x$ stands for a shared name. Then we have the following typing in the source language:
			\[
			 \tree{
				\Gamma \cat x:\chtype{S_1} ; \emptyset; \Delta  \cat k:S \proves   Q \hastype \Proc
			 }{
				\Gamma ; \emptyset; \Delta  \cat k:\btinp{\chtype{S_1}}S \proves  \binp{k}{x} Q \hastype \Proc}
			 \]
			 The typing in the target language is derived similarly as in the first sub-case.	
%	\end{enumerate}
	%
%	\qed
%\end{proof}


%\begin{proposition}\rm
%	\label{app:enc_sesspnr_to_ho_oc}
%	Encoding $\encod{\cdot}{\cdot}{1}: \sessp^{-\mu} \to \HO$  enjoys operational correspondence (cf. Def.~\ref{def:ep}\,(2)).
%\end{proposition}
%
%\begin{proof}[Sketch]
%	We must show completeness and soundness properties. 
%	For completeness, it suffices to consider source process
%	$P_0 = \bout{k}{k'} P \Par \binp{k}{x} Q$. We have that
%%
%	\[
%		P_0 \red P \Par Q\subst{k'}{x}.
%	\]
%%
%	By the definition of encoding we have:
%	\begin{eqnarray*}
%		\pmap{P_0}{1} & = & \bbout{k}{ \abs{z}{\,\binp{z}{X} \appl{X}{k'}} } \pmap{P}{1} \Par \binp{k}{X} \newsp{s}{\appl{X}{s} \Par \bbout{\dual{s}}{\abs{x} \pmap{Q}{1}} \inact}  \\
%		& \red & \pmap{P}{1} \Par \newsp{s}{\appl{X}{s} \subst{\abs{z}{\,\binp{z}{X} \appl{X}{k'}}}{X} \Par \bbout{\dual{s}}{\abs{x}{\pmap{Q}{1}}} \inact} \\
%		& = & \pmap{P}{1} \Par \newsp{s}{\,\binp{s}{X} \appl{X}{k'} \Par \bbout{\dual{s}}{\abs{x}{\pmap{Q}{1}}} \inact} \\
%		& \red & \pmap{P}{1} \Par \appl{X}{k'} \subst{\abs{x} \pmap{Q}{1}}{X} \Par \inact \\
%		& \scong & \pmap{P}{1} \Par \pmap{Q}{1}\subst{k'}{x}  
%	\end{eqnarray*}
%	For soundness, it suffices to notice that the encoding does not add new visible actions:
%	the additional synchronizations induced by the encoding always occur on private (fresh) names.
%	We assume weak bisimilarities, which abstract from internal actions used by the encoding,
%	and so  constructing a relation witnessing behavioral equivalence is easy.
%	\qed
%\end{proof}



%\begin{proof}
%	By induction on the inference $\Gamma; \emptyset; \Delta \proves P \hastype \Proc$.
%	\begin{enumerate}[1.]
		\item	Case $P_0 = \rvar{X}$.
			Then we have the following typing in the source language:
%
			\[
				\Gamma \cat \rvar{X}: \Delta ;\, \es ;\, \es \proves \rvar{X} \hastype \Proc
			\]
%
			Then the typing of $\pmapp{\rvar{X}}{1}{f}$ is as follows,
			assuming $f(\rvar{X}) = \tilde{n}$ and $\tilde{x} = \vmap{\tilde{n}}$.
			Also, we write $\Delta_{\tilde{n}}$ 
			and $\Delta_{\tilde{x}}$ 
			to stand for 
			$n_1: S_1, \ldots, n_m: S_m$ and
			$x_1: S_1, \ldots, x_m: S_m$, respectively. 
			Below, we assume that $\Gamma = \Gamma' \cat X:\shot{\tilde{T}}$, 
			where  
			%$$\tilde{T} =  \trec{t}{\big(\tilde{S}, \btinp{\vart{t}}\tinact\big)}$$.
			\[
				\tilde{T} = \big(\tilde{S}, S^*\big) \qquad \quad
				S^* = \bbtinp{A}\tinact \qquad \quad
				A = \trec{t}{(\tilde{S}, \btinp{\vart{t}}\tinact)}
			\]
%
			\begin{eqnarray}
				\label{prop:sessp_to_HO_t1}
				\tree{
					\tree{
					}{
						\Gamma ;\, \es ;\, \es \proves X \hastype \shot{\tilde{T}}
					}
					\quad 
					\begin{array}{c}
						\Gamma ;\, \es ;\, \{n_i: S_i \} \proves n_i \hastype S_i \\
						\Gamma ;\, \es ;\, \{s: S^* \} \proves s\hastype S^*  \\
					\end{array}
				}{
					\Gamma  ;\, \es ;\, \Delta_{\tilde{n}}, s:\btinp{\shot{\tilde{T}}}\tinact
					\proves  
					\appl{\X}{\tilde{n}, s} \hastype \Proc
				} 
			\end{eqnarray}
%
			\begin{eqnarray}
				\label{prop:sessp_to_HO_t2}
				\tree{
					\tree{
						\Gamma  ;\, \es ;\,   \es \proves \inact \hastype \Proc
					}{
						\Gamma  ;\, \es ;\,   \dual{s}: \tinact \proves \inact \hastype \Proc
					} 
					\quad
					\tree{
						\tree{
							\begin{array}{c}
								\Gamma ;\, \es ;\, \{x_i: S_i \} \proves x_i \hastype S_i \\
								\Gamma ;\, \es ;\, \{z: S^*  \} \proves z\hastype S^*  \\
								\Gamma ;\, \es ;\, \es \proves X \hastype \shot{\tilde{T}}  \\
							\end{array}
						}{
							\Gamma  ;\, \es ;\,   \Delta_{\tilde{x}}, \, z:S^*
							\proves 
							 {\appl{X}{ \tilde{x}, z}} \hastype \Proc
						}
					}{
						\Gamma  ;\, \es ;\,   \es
						\proves 
						 \abs{\tilde{x},z}\,\,{\appl{X}{ \tilde{x}, z}} \hastype \shot{\tilde{T}}
					} 	
				}{
					\Gamma  ;\, \es ;\,   \dual{s}: \btout{\shot{\tilde{T}}}\tinact
					\proves 
					\bbout{\dual{s}}{ \abs{\tilde{x},z}\,\,{\appl{X}{ \tilde{x}, z}}} \inact \hastype \Proc
				}
			\end{eqnarray}
%
			\[
			\tree{
				\tree{
					\begin{array}{cc}
						\Gamma  ;\, \es ;\, \Delta_{\tilde{n}}, s:\btinp{\shot{\tilde{T}}}\tinact
						\proves  
						\appl{\X}{\tilde{n}, s} \hastype \Proc
						& \eqref{prop:sessp_to_HO_t1}
						\\ 
						\Gamma  ;\, \es ;\,   \dual{s}: \btout{\shot{\tilde{T}}}\tinact
						\proves 
						\bbout{\dual{s}}{ \abs{\tilde{x},z}\,\,{\appl{X}{ \tilde{x}, z}}} \inact \hastype \Proc
						& \eqref{prop:sessp_to_HO_t2}
					\end{array}
				}{
					\Gamma  ;\, \es ;\, \Delta_{\tilde{n}}, s:\btinp{\shot{\tilde{T}}}\tinact, \, \dual{s}: \btout{\shot{\tilde{T}}}\tinact
					\proves 
					\appl{\X}{\tilde{n}, s} \Par \bbout{\dual{s}}{ \abs{\tilde{x},z}\,\,{\appl{X}{ \tilde{x}, z}}} \inact \hastype \Proc
				}
			}{
				\Gamma  ;\, \es ;\, \Delta_{\tilde{n}}
				\proves 
				\newsp{s}{\appl{\X}{\tilde{n}, s} \Par \bbout{\dual{s}}{ \abs{\tilde{x},z}\,\,{\appl{X}{ \tilde{x}, z}}} \inact} \hastype \Proc
			}
			\]
%	
		\item	Case $P_0 = \recp{X}{P}$. Then we have the following typing in the source language:
%
			\[
				\tree{
					\Gamma \cat \rvar{X}:\Delta ;\, \es ;\,  \Delta \proves P \hastype \Proc
				}{
					\Gamma  ;\, \es ;\,  \Delta \proves \recp{X}{P} \hastype \Proc
				}
			\]
%	
			Then we have the following typing in the target language ---we write $R$
			to stand for $\pmapp{P}{1}{{f,\{\rvar{X}\to \tilde{n}\}} }$
			and $\tilde{x}$ to stand for $\vmap{\ofn{P}}$.
%
			\begin{eqnarray}
				\label{prop:sessp_to_HO_t4}
				\tree{
					\tree{
						\tmap{\Gamma}{1}\cat X:\shot{\tilde{T}};\, \es;\, \tmap{\Delta_{\tilde{n}}}{1}
						\proves
						 R  \hastype \Proc
					}{
						\tmap{\Gamma}{1}\cat X:\shot{\tilde{T}};\, \es;\, \tmap{\Delta_{\tilde{n}}}{1}, s:\tinact 
						\proves
						 R  \hastype \Proc
					}
				}{
					\tmap{\Gamma}{1};\, \es;\, \tmap{\Delta_{\tilde{n}}}{1}, s:\btinp{\shot{\tilde{T}}}\tinact 
					\proves
					\binp{s}{\X} R  \hastype \Proc
				}
			\end{eqnarray}
%
			\begin{eqnarray}
				\label{prop:sessp_to_HO_t5}
				\tree{
					\tree{
						\tmap{\Gamma}{1};\, \es;\, \es
						\proves
						\inact \hastype \Proc
					}{
						\tmap{\Gamma}{1};\, \es;\, \dual{s}:\tinact
						\proves
						\inact \hastype \Proc
					} 
					\quad 
					\tree{
						\tree{
							\tree{
								\tmap{\Gamma}{1} \cat X: \shot{\tilde{T}};\, \es;\, \tmap{\Delta_{\tilde{x}}}{1}
								\proves
								{{\auxmap{R}{\es}}}  \hastype \Proc
							}{
								\tmap{\Gamma}{1} \cat X: \shot{\tilde{T}};\, \es;\, \tmap{\Delta_{\tilde{x}}}{1},z: \tinact
								\proves
								{{\auxmap{R}{\es}}}  \hastype \Proc
							}
						}{
							\tmap{\Gamma}{1};\, \es;\, \tmap{\Delta_{\tilde{x}}}{1}, \, z: \btinp{A}\tinact
							\proves
							{{\binp{z}{\X} \auxmap{R}{\es}}}  \hastype \Proc
						}
					}{
						\tmap{\Gamma}{1};\, \es;\, \es
						\proves
						{\abs{\tilde{x}, z } \,{\binp{z}{\X} \auxmap{R}{\es}}}  \hastype \shot{\tilde{T}}
					}
				}{
					\tmap{\Gamma}{1};\, \es;\, \dual{s}:\btout{\shot{\tilde{T}}}\tinact
					\proves
					\bbout{\dual{s}}{\abs{\tilde{x}, z } \,{\binp{z}{\X} \auxmap{R}{\es}}} \inact \hastype \Proc
				}
			\end{eqnarray}
%
			\[
			\tree{
				\tree{
					\begin{array}{cc}
						\tmap{\Gamma}{1};\, \es;\, \tmap{\Delta_{\tilde{n}}}{1}, s:\btinp{\shot{\tilde{T}}}\tinact 
						\proves
						\binp{s}{\X} R  \hastype \Proc
						& \eqref{prop:sessp_to_HO_t4}
						\\
						\tmap{\Gamma}{1};\, \es;\, \dual{s}:\btout{\shot{\tilde{T}}}\tinact
						\proves
						\bbout{\dual{s}}{\abs{\tilde{x}, z } \,{\binp{z}{\X} \auxmap{R}{\es}}} \inact \hastype \Proc
						& \eqref{prop:sessp_to_HO_t5}
					\end{array}
				}{
					\tmap{\Gamma}{1};\, \es;\, \tmap{\Delta_{\tilde{n}}}{1}, s:\btinp{\shot{\tilde{T}}}\tinact , \dual{s}:\btout{\shot{\tilde{T}}}\tinact
					\proves
					\binp{s}{\X} R \Par \bbout{\dual{s}}{\abs{\tilde{x}, z } \,{\binp{z}{\X} \auxmap{R}{\es}}} \inact \hastype \Proc
				}
			}{
				\tmap{\Gamma}{1};\, \es;\, \tmap{\Delta_{\tilde{n}}}{1} 
				\proves
				\newsp{s}{\binp{s}{\X} R \Par \bbout{\dual{s}}{\abs{\tilde{x}, z } \,{\binp{z}{\X} \auxmap{R}{\es}}} \inact} \hastype \Proc
			}
			\]
	\end{enumerate}
	\qed
\end{proof}

%\begin{proposition}\rm
%	\label{app:enc_sesp_to_HO_oc}
%	Encoding $\fencod{\cdot}{\cdot}{2}{f}: \sessp \to \HO$ 
%	enjoys operational correspondence (cf. Def.~\ref{def:ep}\,(2)).
%\end{proposition}
%
%\begin{proof}[Sketch]
%\dk{TBD.}
%\end{proof}

%%%%%%%%%%%%%%%%%%%%%%%%%%%%%%%%%%%%%%%%%%%%%%%%%
% HOp TO SESSP
%%%%%%%%%%%%%%%%%%%%%%%%%%%%%%%%%%%%%%%%%%%%%%%%%


\subsection{Properties for $\enco{\pmap{\cdot}{2}, \tmap{\cdot}{}, \mapa{\cdot}^{2}}: \HOp \to \sessp$}
\label{app:enc:HOp_to_sessp}

We repeat the statement of Prop.~\ref{prop:typepres_HOp_to_FO},
as in Page \pageref{prop:typepres_HOp_to_FO}:

\begin{proposition}[Type Preservation, \HOp into \sessp]\rm
	Let $P$ be a \HOp process. 
	If $\Gamma; \emptyset; \Delta \proves P \hastype \Proc$ then 
	$\mapt{\Gamma}^{2}; \emptyset; \mapt{\Delta}^{2} \proves \map{P}^{2} \hastype \Proc$.
\end{proposition}


%\begin{proposition}[Type Preservation, Higher-Order into First-Order]
%Let $P$ be an  $\HO$ process. 
%If			$\Gamma; \emptyset; \Delta \proves P \hastype \Proc$ then 
%			$\mapt{\Gamma}^{2}; \emptyset; \mapt{\Delta}^{2} \proves \map{P}^{2} \hastype \Proc$. 
%\end{proposition}

\begin{proof}
	By induction on the inference $\Gamma; \emptyset; \Delta \proves P \hastype \Proc$.
%	By induction on the structure of \HO process $P$.  \jp{TO BE ADJUSTED!}
	\begin{enumerate}[1.]

	%%%% Output of (linear) channel
		\item	Case $P = \bbout{k}{\abs{x}{Q}}P$. Then we have two possibilities, depending on the typing for $\abs{x}Q$.
			The first case concerns a linear typing, and  
			we have the following typing in the source language:
%
			\[
				\tree{
					\Gamma; \emptyset; \Delta_1 \cat k:S  \proves  P \hastype \Proc
					\quad
					\tree{
						\Gamma ; \emptyset ; \Delta_2\cat x:S_1 \proves  Q \hastype \Proc
					}{
						\Gamma ; \emptyset ; \Delta_2 \proves  \abs{x}Q \hastype \lhot{S_1}
					}
				}{
					\Gamma; \emptyset; \Delta_1 \cat \Delta_2 \cat k:\btout{\lhot{S_1}}S \proves  \bbout{k}{\abs{x}{Q}} P \hastype \Proc
				}
			\]
%			
			This way, by IH we have
			$$
			\tmap{\Gamma}{2}; \es ; \tmap{\Delta_2}{2}, x:\tmap{S_1}{2}
									\proves 
									\pmap{Q}{2} \hastype \Proc
			$$
			Let us write 
			 $U_1$ to stand for 
			$\chtype{\btinp{\tmap{S_1}{2}}\tinact}$.
			The corresponding typing in the target language is as follows: 
%
			\begin{eqnarray*}
				\tmap{\Gamma_1}{2} & = & \tmap{\Gamma}{2} \cup a:\chtype{\btinp{\tmap{S_1}{2}}\tinact} \\
				\tmap{\Gamma_2}{2} & = & \tmap{\Gamma_1}{2} \cup \rvar{X}:\tmap{\Delta_2}{2}
			\end{eqnarray*}
%
			Also $(*)$ stands for $\tmap{\Gamma_1}{2}; \es ; \es \proves a \hastype U_1$; 
			$(**)$ stands for $\tmap{\Gamma_2}{2}; \es ; \es \proves a \hastype U_1$; and
			$(***)$ stands for $\tmap{\Gamma_2}{2}; \es ; \es \proves \rvar{X} \hastype \Proc$.
			\begin{eqnarray}
				\label{prop:HO_to_sessp_t1}
				\tree{
					\tree{
						\tree{
						}{
							(***)
						} 
						\quad 
						\tree{
							\tree{
								\tree{
									\tree{
									}{
										\tmap{\Gamma_2}{2}; \es ; \tmap{\Delta_2}{2},  x:\tmap{S_1}{2}
										\proves 
										\pmap{Q}{2} \hastype \Proc
									}
								}{
									\tmap{\Gamma_2}{2}; \es ; \tmap{\Delta_2}{2}, y:\tinact, x:\tmap{S_1}{2}
									\proves 
									\pmap{Q}{2} \hastype \Proc
								}
							}{
								\tmap{\Gamma_2}{2}; \es ; \tmap{\Delta_2}{2}, y: \btinp{\tmap{S_1}{2}}\tinact
								\proves 
								\binp{y}{x}\pmap{Q}{2} \hastype \Proc
							} 
							\quad 
							\tree{
							}{
								(**)
							}
						}{
							\tmap{\Gamma_2}{2}; \es ; \tmap{\Delta_2}{2} 
							\proves 
							\binp{a}{y}\binp{y}{x}\pmap{Q}{2} \hastype \Proc
						} 
					}{
						\tmap{\Gamma_2}{2}; \es ; \tmap{\Delta_2}{2} 
						\proves 
						\binp{a}{y}\binp{y}{x}\pmap{Q}{2} \Par \rvar{X} \hastype \Proc
					}
				}{
					\tmap{\Gamma_1}{2}; \es ; \tmap{\Delta_2}{2} 
					\proves 
					\recp{X}{(\binp{a}{y}\binp{y}{x}\pmap{Q}{2} \Par \rvar{X})} \hastype \Proc
				}
			\end{eqnarray}
%
			\begin{eqnarray}
				\label{prop:HO_to_sessp_t2}
				\tree{
					\begin{array}{c}
						\tmap{\Gamma_1}{2}; \es ; \tmap{\Delta_1}{2}, k:\tmap{S}{2} 
						\proves 
						\pmap{P}{2}  \hastype \Proc
						\\
						\tmap{\Gamma_1}{2}; \es ; \tmap{\Delta_2}{2} 
						\proves 
						\recp{X}{(\binp{a}{y}\binp{y}{x}\pmap{Q}{2} \Par \rvar{X})} \hastype \Proc
						\quad \eqref{prop:HO_to_sessp_t1}
					\end{array}
				}{
					\tmap{\Gamma_1}{2}; \es ; \tmap{\Delta_1, \Delta_2}{2}, k:\tmap{S}{2} 
					\proves 
					\pmap{P}{2} \Par 
					\recp{X}{(\binp{a}{y}\binp{y}{x}\pmap{Q}{2} \Par \rvar{X})} \hastype \Proc
				}
			\end{eqnarray}
%
			\[
				\tree{
					\tree{
						\begin{array}{c}
							\tmap{\Gamma_1}{2}; \es ; \es \proves a \hastype U_1
							\\
							\tmap{\Gamma_1}{2}; \es ; \tmap{\Delta_1, \Delta_2}{2}, k:\tmap{S}{2} 
							\proves 
							\pmap{P}{2} \Par 
							\recp{X}{(\binp{a}{y}\binp{y}{x}\pmap{Q}{2} \Par \rvar{X})} \hastype \Proc
							\quad \eqref{prop:HO_to_sessp_t2}
						\end{array}
					}{
						\tmap{\Gamma_1}{2}; \es ; \tmap{\Delta_1, \Delta_2}{2}, k:\bbtout{U_1}\tmap{S}{2} 
						\proves 
						\bout{k}{a}(\pmap{P}{2} \Par 
						\recp{X}{(\binp{a}{y}\binp{y}{x}\pmap{Q}{2} \Par \rvar{X}))} \hastype \Proc
					}
				}{
					\tmap{\Gamma}{2}; \es ; \tmap{\Delta_1, \Delta_2}{2}, k:\bbtout{U_1}\tmap{S}{2} 
					\proves 
					\newsp{a}{\bout{k}{a}( 
					\pmap{P}{2} \Par 
					\recp{X}{(\binp{a}{y}\binp{y}{x}\pmap{Q}{2} \Par \rvar{X}))}} \hastype \Proc
				}
			\]
%
			In the second case, $\abs{x}Q$ has a shared type. We have the following typing in the source language:
%
			\[
				\tree{
					\Gamma; \emptyset; \Delta \cat k:S  \proves  P \hastype \Proc
					\quad 
					\tree{
						\tree{
							\Gamma ; \emptyset ; \cat x:S_1 \proves  Q \hastype \Proc
						}{
							\Gamma ; \emptyset ; \es \proves  \abs{x}Q \hastype \lhot{S_1}
						}
					}{
						\Gamma ; \emptyset ; \es \proves  \abs{x}Q \hastype \shot{S_1}
					}
				}{
					\Gamma; \emptyset; \Delta  \cat k:\btout{\shot{S_1}}S \proves  \bbout{k}{\abs{x}{Q}} P \hastype \Proc
				}
			\]
%
			The corresponding typing in the target language can be derived similarly as in the first case.
	
		\item	Case $P = \binp{k}{X} P$. Then there are two cases, depending on the type of $X$. 
			In the first case,
			we have the following typing in the source language:
%
			\[
				\tree{
					\Gamma \cat X : \shot{S_1};\, \emptyset ;\, \Delta \cat k:S \proves  P \hastype \Proc
				}{
					\Gamma;\, \emptyset;\, \Delta\cat k:\btinp{\shot{S_1}}S \proves  \binp{k}{X} P \hastype \Proc
				}
			\]
			The corresponding typing in the target language is as follows:
			% --- we write $\Gamma_0$ to stand for $\Gamma \setminus \{X: \lhot{S_1}\}$.
%
			\[
				\tree{
					\tree{}{\tmap{\Gamma}{2} \cat x : \chtype{\btinp{\tmap{S_1}{2}}\tinact};\, \emptyset ;\, \Delta \cat k:\tmap{S}{2} \proves  \tmap{P}{2} \hastype \Proc}
				}{
					\tmap{\Gamma}{2};\, \emptyset; \, \tmap{\Delta}{2}\cat k:\bbtinp{\chtype{\btinp{\tmap{S_1}{2}}\tinact}}\tmap{S}{2} \proves
					\binp{k}{x} \pmap{P}{2} \hastype \Proc
				}
			\]
%
			In the second case,  
			we have the following typing in the source language:
%
			\[
				\tree{
					\Gamma;\, \{X : \lhot{S_1}\};\, \emptyset ;\, \Delta \cat k:S \proves  P \hastype \Proc
				}{
					\Gamma;\, \emptyset;\, \Delta\cat k:\btinp{\lhot{S_1}}S \proves  \binp{k}{X} P \hastype \Proc
				}
			\]
%
			The corresponding typing in the target language is as follows:
			% --- we write $\Gamma_0$ to stand for $\Gamma \setminus \{X: \lhot{S_1}\}$.
%
			\[
				\tree{
					\tmap{\Gamma}{2} \cat x : \chtype{\btinp{\tmap{S_1}{2}}\tinact};\, \emptyset ;\, \Delta \cat k:\tmap{S}{2} \proves  \tmap{P}{2} \hastype \Proc
				}{
					\tmap{\Gamma}{2};\, \emptyset;\, \tmap{\Delta}{2}\cat k:\bbtinp{\chtype{\btinp{\tmap{S_1}{2}}\tinact}}\tmap{S}{2} \proves
					\binp{k}{x} \pmap{P}{2} \hastype \Proc
				}
			\]
%
		\item	Case $P = \appl{X}{k}$. Also here we have two cases, depending on whether $X$ has linear or shared type.
			In the first case, $X$ is linear and
			we have the following typing in the source language:
%
			\[
				\tree{
					\Gamma ;\, \{X : \lhot{S_1}\};\,  \es \proves  X \hastype \lhot{S_1} \quad \Gamma; \es ; \{k:S_1\} \proves k \hastype S_1
				}{
					\Gamma;\, \{X : \lhot{S_1}\};\, k:S_1 \proves  \appl{X}{k} \hastype \Proc}
			\]
			Let us write
			$\tmap{\Gamma_1}{2}$ to stand for $\tmap{\Gamma}{2} \cat x:\chtype{\btout{\tmap{S_1}{2}}\tinact}$.
			The corresponding typing in the target language is as follows:
%
			\begin{eqnarray}
				\label{prop:HO_to_sessp_t11}
				\tree{
					\tree{
						\tmap{\Gamma_1}{2};\, \es;\,  \es \proves  \inact \hastype \Proc
					}{
						\tmap{\Gamma_1}{2};\, \es;\,  \dual{s}:\tinact \proves  \inact \hastype \Proc
					}
					\quad 
						\tmap{\Gamma_1}{2};\, \es;\, \{k:\tmap{S_1}{2}\} \proves  k \hastype \tmap{S_1}{2} 
				}{
					\tmap{\Gamma_1}{2};\, \es;\,\, k:\tmap{S_1}{2},\,  \dual{s}:\btout{\tmap{S_1}{2}}\tinact \proves  \bout{\dual{s}}{k}\inact \hastype \Proc
				}
			\end{eqnarray}
%
			\[
				\tree{
					\tree{
						\begin{array}{c}
							\tmap{\Gamma_1}{2};\, \es;\,\, k:\tmap{S_1}{2},\,  \dual{s}:\btout{\tmap{S_1}{2}}\tinact \proves
							\bout{\dual{s}}{k}\inact \hastype \Proc
							\quad \eqref{prop:HO_to_sessp_t11}
							\\
							\tmap{\Gamma_1}{2} ;\, \es ;\, \es \proves x \hastype \chtype{\btout{\tmap{S_1}{2}}\tinact}
						\end{array}
					}{
						\tmap{\Gamma_1}{2};\, \es;\, k:\tmap{S_1}{2}, s:\btinp{\tmap{S_1}{2}}\tinact , \dual{s}:\btout{\tmap{S_1}{2}}\tinact
						\proves
						\bout{x}{s}\bout{\dual{s}}{k}\inact \hastype \Proc
					}
				}{
					\tmap{\Gamma_1}{2};\, \es;\, k:\tmap{S_1}{2} \proves  \news{s}{(\bout{x}{s}\bout{\dual{s}}{k}\inact)} \hastype \Proc
				}
	\]
%
			In the second case, $X$ is shared, and
			we have the following typing in the source language:
%
			\[
				\tree{
					\Gamma \cat  X : \lhot{S_1} ;\,  \es ;\,  \es \proves  X \hastype \shot{S_1} \quad \Gamma; \es ; k:S_1 \proves k \hastype S_1
				}{
					\Gamma \cat X : \shot{S_1};\, \es ;\, k:S_1 \proves  \appl{X}{k} \hastype \Proc
				}
			\]
%
			The associated typing in the target language is obtained similarly as in the first case. \qed
	\end{enumerate}
\end{proof}


%\begin{proposition}\rm
%	\label{app:enc_HO_to_sessp_oc}
%	Encoding $\encod{\cdot}{\cdot}{2}: \HO \to \sessp$ 
%	enjoys operational correspondence (cf. Def.~\ref{def:ep}\,(2)).
%\end{proposition}
%
%\begin{proof}[Sketch]
%For completeness, we 
%consider the \HO process $P = {\bbout{k}{\abs{x}{Q}} P_1} \Par \binp{k}{X} P_2$. We have that
%\[
%P \red P_1 \Par P_2 \subst{\abs{x}Q}{X}
%\]
%In the target language, this reduction is mimicked as follows:
%\begin{eqnarray*}
%\pmap{P}{2} & = & \newsp{a}{\bout{k}{a} (\pmap{P_1}{2} \Par \repl{} \binp{a}{y} \binp{y}{x} \pmap{Q}{2})\,} 
%                  \Par \binp{k}{x} \pmap{P_2}{2} \\
%            & \red & \newsp{a}{\pmap{P_1}{2} \Par \repl{} \binp{a}{y} \binp{y}{x} \pmap{Q}{2} 
%                  \Par  \pmap{P_2}{2}\subst{a}{x}}
%\end{eqnarray*}
%\qed
%\end{proof}


% !TEX root = ../journal16kpy.tex

\section{Negative Result}
\label{app:neg}

\begin{theorem}%\myrm
%	\label{thm:negative}
	Let $\CAL_1, \CAL_2 \in \set{\HOp, \HO, \sessp}$.
	There is no typed, minimal encoding from $\tyl{L}_{\CAL_1}$ into $\tyl{L}_{\CAL_2^{\minussh}}$
%	$\enco{\map{\cdot}, \mapt{\cdot}, \mapa{\cdot}}: \sessp \longrightarrow \HOp^{\minussh}$.
%	that enjoys: (i) homomorphism wrt parallel; (ii) barb preservation; (iii) operational completeness.
\end{theorem}

\begin{proof}
	Assume, towards a contradiction, that such a typed encoding indeed exists. 
	Consider the $\sessp$ process
	%
	\[
		P = \breq{a}{s} \inact \Par \bacc{a}{x} \bsel{n}{l_1} \inact \Par \bacc{a}{x} \bsel{m}{l_2} \inact \qquad \text{(with $n \neq m$)}
	\]
	%
	\noi such that 
	$\Gamma; \es; \Delta \proves P \hastype \Proc$.
	From process $P$ we have: %We then have both
	%
	\begin{eqnarray}
		& & \horel{\Gamma}{\Delta}{P}{\hby{\tau}}{\Delta'}{\bsel{n}{l_1} \inact \Par \bacc{a}{x} \bsel{m}{l_2} \inact = P_1} \label{eq:nn3} \\
		& & \horel{\Gamma}{\Delta}{P}{\hby{\tau}}{\Delta'}{\bsel{m}{l_2} \inact \Par \bacc{a}{x} \bsel{n}{l_1} \inact = P_2} \label{eq:nn4}
	\end{eqnarray}
	%
	Thus, by definition of typed barb we  have:
	%
	\begin{eqnarray}
		\Gamma; \Delta' \proves P_1 \barb{n} & \land & 
		\Gamma; \Delta' \proves P_1 \nbarb{m} \label{eq:nn1} \\
		\Gamma; \Delta' \proves P_2 \barb{m} & \land & 
		\Gamma; \Delta' \proves P_2 \nbarb{n} \label{eq:nn2}
	\end{eqnarray}
	%
	Consider now the $\HOp^{\minussh}$ process $\map{P}$.
	% = 
	% \map{\breq{a}{s} \inact} \Par \map{\bacc{a}{x} \bsel{n}{l_1} \inact} \Par \map{\bacc{a}{x} \bsel{m}{l_2}}$.
	By our assumption of operational completeness 
	(\defref{def:ep}-2(a)), 
	from \eqref{eq:nn3} with \eqref{eq:nn4}
	we infer that
	there exist $\HOp^{\minussh}$ processes $S_1$ and $S_2$ such that:
	%we have both:
	\begin{eqnarray}
		& & \horel{\mapt{\Gamma}}{\mapt{\Delta}}{\map{P}}{\Hby{\stau}}{\mapt{\Delta'}}{S_1 \WB \map{P_1}} \label{eq:n1} \\
		& & \horel{\mapt{\Gamma}}{\mapt{\Delta}}{\map{P}}{\Hby{\stau}}{\mapt{\Delta'}}{S_2 \WB \map{P_2}} \label{eq:n2}
		%\map{P} & \Hby{} &  S_1 \WB \map{P_1} \\
		%s\map{P} & \Hby{} & S_2 \WB \map{P_2}
	\end{eqnarray}
	By our assumption of barb preservation, 
	from \eqref{eq:nn1} with \eqref{eq:nn2}
	we infer: 
	%
	\begin{eqnarray}
		\mapt{\Gamma}; \mapt{\Delta'} \proves \map{P_1} \Barb{n} & \land & 
		\mapt{\Gamma}; \mapt{\Delta'} \proves \map{P_1} \nBarb{m} \label{eq:n3} \\
		\mapt{\Gamma}; \mapt{\Delta'} \proves \map{P_2} \Barb{m} & \land & 
		\mapt{\Gamma}; \mapt{\Delta'} \proves \map{P_2} \nBarb{n} \label{eq:n4}
	\end{eqnarray}
	%
	By definition of $\WB$, 
	by combining~\eqref{eq:n1} with~\eqref{eq:n3}
	and~\eqref{eq:n2} with~\eqref{eq:n4}, we infer barbs for $S_1$ and $S_2$:
	\begin{eqnarray}
		\mapt{\Gamma}; \mapt{\Delta'} \proves S_1 \Barb{n} & \land & 
		\mapt{\Gamma}; \mapt{\Delta'} \proves S_1 \nBarb{m} \label{eq:n5} \\
		\mapt{\Gamma}; \mapt{\Delta'} \proves S_2 \Barb{m} & \land & 
		\mapt{\Gamma}; \mapt{\Delta'} \proves S_2 \nBarb{n} \label{eq:n6}
	\end{eqnarray}
	That is, $S_1$ and $\map{P_1}$ 
	(resp. $S_2$ and $\map{P_2}$)
	have the same barbs.
	Now, by $\tau$-inertness (\propref{lem:tau_inert}), we have both 
	\begin{eqnarray}
		& & \horel{\mapt{\Gamma}}{\mapt{\Delta}}{S_1}{\WB}{\mapt{\Delta'}}{\map{P}} \label{eq:n7} \\
		& & \horel{\mapt{\Gamma}}{\mapt{\Delta}}{S_2}{\WB}{\mapt{\Delta'}}{\map{P}} \label{eq:n8}
	\end{eqnarray}
	Combining~\eqref{eq:n7} with~\eqref{eq:n8}, by transitivity of $\WB$,
	we have 
	\begin{equation}
		\horel{\mapt{\Gamma}}{\mapt{\Delta'}}{S_1}{\WB}{\mapt{\Delta'}}{S_2} \label{eq:n9}
	\end{equation}
	In turn, from~\eqref{eq:n9}
	we infer that 
	it must be the case that:
	\begin{eqnarray*}
		\mapt{\Gamma}; \mapt{\Delta'} \proves \map{P_1} \Barb{n} & \land & 
		\mapt{\Gamma}; \mapt{\Delta'} \proves \map{P_1} \Barb{m} \label{eq:n10} \\
		\mapt{\Gamma}; \mapt{\Delta'} \proves \map{P_2} \Barb{m} & \land & 
		\mapt{\Gamma}; \mapt{\Delta'} \proves \map{P_2} \Barb{n} \label{eq:n11}
	\end{eqnarray*}
	which clearly contradict \eqref{eq:n3} and \eqref{eq:n4} above.
	\qed
\end{proof}


%\begin{theorem}\rm
%	There is no encoding $\enco{\map{\cdot}, \mapt{\cdot}, \mapa{\cdot}}: \HOp \longrightarrow \HOp^{\minussh}$
%	that enjoys operational correspondence and full abstraction.
%\end{theorem}

%\begin{proof}
%	Let $\horel{\Gamma_1}{\Delta_1}{P_1}{\not\wb}{\Delta_2}{P_2}$
%	with $P = \breq{a}{s} \inact \Par \bacc{a}{x} P_1 \Par \bacc{a}{x} P_2$ and
%	let $\Gamma; \emptyset; \Delta \proves P \hastype \Proc$.
%	Assume also a encoding
%	$\enco{\map{\cdot}, \mapt{\cdot}, \mapa{\cdot}}: \HOp \longrightarrow \HOp^{\minussh}$
%	that enjoys
%	operational correspondence and full abstraction.
%
%	From operational correspondence we get that:
%	\begin{eqnarray*}
%		P \red P_1 \Par \bacc{a}{x} P_2 &\textrm{implies}& \map{P} \red \map{P_1 \Par \bacc{a}{x} P_2}\\
%		P \red P_2 \Par \bacc{a}{x} P_1 &\textrm{implies}& \map{P} \red \map{P_2 \Par \bacc{a}{x} P_1}
%	\end{eqnarray*}
%
%	From the fact that
%	$\horel{\Gamma_1}{\Delta_1}{P_1}{\not\wb}{\Delta_2}{P_2}$
%	we can derive that
%%
%	\[
%		\horel{\Gamma_1'}{\Delta_1'}{P_1 \Par \bacc{a}{x} P_2}{\not\wb}{\Delta_2'}{P_2 \Par \bacc{a}{x} P_1}
%	\]
%%
%	From Corollary~\ref{cor:tau_inert} we know that
%%
%	\begin{eqnarray*}
%		\horel{\mapt{\Gamma}}{\mapt{\Delta}}{\map{P}}{\wb}{\mapt{\Delta_1'}}{\map{P_1 \Par \bacc{a}{x} P_2}}\\
%		\horel{\mapt{\Gamma}}{\mapt{\Delta}}{\map{P}}{\wb}{\mapt{\Delta_2'}}{\map{P_2 \Par \bacc{a}{x} P_1}}
%	\end{eqnarray*}
%%
%	\noi thus
%	\[
%		\horel{\mapt{\Gamma}}{\mapt{\Delta_1'}}{\map{P_1 \Par \bacc{a}{x} P_2}}{\wb}{\mapt{\Delta_2'}}{\map{P_2 \Par \bacc{a}{x} P_1}}
%	\]
%%
%	From here we conclude that the full abstraction property does not hold,
%	which is a contradiction.
%	\qed
%%	so there is no mapping $\map{\cdot}: \pHO \longrightarrow \spi$ that enjoys
%%	the operational correspondence and full abstraction properties.
%\end{proof}


\end{document}


