% !TEX root = main.tex

\jparagraph{A Precise Encoding of Name-Passing into Process-Passing}
As mentioned above, 
our encoding of \HOp into \HO (\secref{subsec:HOpi_to_HO}) should 
%overcome two key challenges. First, it should 
(a)~enable the communication of arbitrary names, as required to represent delegation,
and 
%Second, it should 
(b)~address the fact that linearity of session types limits the 
possibilities for representing infinite behavior. 
To encode name passing into \HO 
%to encode name output, 
we ``pack''
the name to be sent into a suitable abstraction; 
upon reception, the receiver ``unpacks'' this object following a precise protocol on a fresh  session:
%More precisely, our encoding \jpc{of name passing} in \HO is given as:
\begin{center}
\begin{tabular}{rcll}
  $\map{\bout{a}{b} P}$	&$=$&	$\bout{a}{ \abs{z}{\,\binp{z}{x} (\appl{x}{b})} } \map{P}$ \\
  $\map{\binp{a}{x} Q}$	&$=$&	$\binp{a}{y} \newsp{s}{\appl{y}{s} \Par \bout{\dual{s}}{\abs{x}{\map{Q}}} \inact}$
\end{tabular}
\end{center}
%and as a homomorphism for the other operators.
Above, 
%where
$a,b$ are names and $s$ and $\dual{s}$ are 
linear session names (\emph{endpoints}).
%$\lambda x.P$ is a name abstraction of $P$; $\appl{x}{a}$ is a name application; 
Processes $\bout{a}{V} P$ and 
$\binp{a}{x} P$ denote output and input at~$a$;   
abstractions and applications are denoted
$\lambda x.P$ and $(\lambda x.P)a$; %, respectively;
$\newsp{s}P$ and $\inact$ represent hiding and inaction. %, respectively.
%Intuitively, the output of a name $b$ along name $a$ is encoded by
%the output of an abstraction containing $b$; the input of a name is encoded 
%by the input of an abstraction
Thus, following a communication on $a$, %our encoding features 
a (deterministic) reduction between  
$s$ and $\dual{s}$ guarantees that name $b$ is properly unpacked by means of abstraction passing
and appropriate applications.
Observe that 
\HO requires two extra reduction steps to mimic a name communication step in \HOp.
Also, it is worth stressing how an output action in the source process is translated into an output action in the encoded process (and similarly for input).
This is key to ensure the preservation of session type operators mentioned above (cf. \defref{def:tp}).

To preserve session linearity, we proceed as follows.
Given $\recp{X}{P}$, 
we encode the recursion body $P$ as an abstraction
in which free names of $P$ are converted into name variables.
%The encoding keeps track of these free names.
The resulting higher-order value is embedded in an input-guarded 
``duplicator'' process~\cite{ThomsenB:plachoasgcfhop}.
The recursion variable $X$ is then encoded 
in such a way that it
simulates recursion unfolding by 
invoking the duplicator in a by-need fashion.
That is, upon reception, the abstraction representing the 
recursion body $P$
is duplicated: 
one copy is used to reconstitute the original recursion body $P$ (through
the application of the free names of $P$); 
another copy is used to re-invoke the duplicator when needed. 
Interestingly, for this encoding to work 
we require non-tail recursive session types; to this end, 
we apply recent advances on the theory of duality for session types~\cite{TGC14,DBLP:journals/corr/abs-1202-2086}.

%To this end, we
%first record a mapping from recursive variable $X$ to process variable $z_X$.
%Then, we encode the recursion body $P$ as a name abstraction
%in which free names of $P$ are converted into name variables, using \defref{d:auxmap}.
%(Notice that $P$ is first encoded into \HO and then transformed using mapping
%$\auxmapp{\cdot}{{}}{\sigma}$.)
%Subsequently, this higher-order value is embedded in an input-guarded 
%``duplicator'' process~\cite{ThomsenB:plachoasgcfhop}. Finally, we define the encoding of $X$ 
%in such a way that it
%simulates recursion unfolding by 
%invoking the duplicator in a by-need fashion.
%That is, upon reception, the \HO abstraction which encodes  the 
%recursion body $P$
%%containing $\auxmapp{P}{{}}{\sigma}$ 
%is duplicated: 
%one copy is used to reconstitute the original recursion body $P$ (through
%the application of $\fn{P}$); another copy is used to re-invoke
%the duplicator when needed. 
%
%We encode recursion with non-tail recursive session types; for this 
%we apply recent advances on the theory of session duality~\cite{TGC14,DBLP:journals/corr/abs-1202-2086}.

\jparagraph{A Plausible Encoding That is Not Precise}
Our notion of \emph{precise encoding} (\secref{s:expr}) that
requires the translation of both process and types, and 
admits only process mappings that preserve session types
\emph{and} are fully abstract. Thus, our encodings 
not only exhibit  strong behavioral correspondences, but also 
 relate source and target processes with  
communication structures described by session types.
%Moreover, the notion of encoding includes full abstraction as encodability criteria.
These strict requirements make our developments far from trivial.
In particular, requiring type preservation rules out other plausible encoding strategies.
To illustrate this point,
consider the  following encoding of %$\sessp$ 
name-passing 
into $\HO$:\footnote{This alternative  encoding was suggested by an anonymous reviewer of a previous version of this paper.} %defined as
\begin{center}
\begin{tabular}{rcll}
  $\umap{\bout{a}{b} P}$	&$=$&	$\binp{a}{x}( \appl{x}{b} \Par \umap{P})$ \\
  $\umap{\binp{a}{x} Q}$	&$=$&	$\bout{a}{\abs{x}{\umap{Q}}} \inact$
\end{tabular}
\end{center}
%and as a homomorphism for the other operators.
Intuitively, 
rather than sending a package with name $b$, 
this encoding sends the continuation of the input. Observe how this mapping entails  a 
``role inversion'': outputs are translated into inputs, and inputs are translated into outputs. 
Although perfectly reasonable, the encoding $\umap{\cdot}$  
%is far from desirable in a session typed setting: 
is \emph{not type preserving}. Consequently, it is also not \emph{precise}.
%Type preservation is intended to preserve the overall semantics of session types:
Since individual prefixes (input, output, branching, select) 
represent actions in a structured communication sequence (i.e., a protocol abstracted by a session type),
the encoding above would simply alter the meaning of the session protocol in the source language.




