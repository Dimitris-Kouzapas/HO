% !TEX root = main.tex

\subsection{Comparing Precise Encodings}\label{ss:compare}
The precise encodings reported in  \secref{subsec:HOpi_to_HO} and \secref{subsec:HOp_to_sessp}
confirm that \HO and \sessp constitute two important sources of expressiveness in \HOp.
This naturally begs the question: which of the two sub-calculi is more tightly related to \HOp?
We empirically and formally argue that when compared to \sessp, \HO   is more economical and satisfies tighter correspondences.

\jparagraph{Empirical Comparison: Reduction Steps}
We first contrast the way in which 
%\begin{enumerate}[a)]
%\item 
(a)~the encoding from \HOp to \HO (\secref{subsec:HOpi_to_HO}) translates processes with name passing;
%\item 
(b)~the encoding from \HOp to \sessp (\secref{subsec:HOp_to_sessp}) translates processes with abstraction passing.
%\end{enumerate}
Consider the \HOp processes:
$$
P_1  =  \bout{s}{a} \inact \Par \binp{\dual{s}}{x} (\bout{x}{s_1} \inact \Par \dots \Par \bout{x}{s_n} \inact) \qquad
P_2  =  \bout{s}{\abs{x}{P}} \inact \Par \binp{\dual{s}}{x} (\appl{x}{s_1} \Par \dots \Par \appl{x}{s_n})
$$

%\begin{eqnarray*}
%P_1 & = & \bout{s}{a} \inact \Par \binp{\dual{s}}{x} (\bout{x}{s_1} \inact \Par \dots \Par \bout{x}{s_n} \inact) \\
%P_2 & = & \bout{s}{\abs{x}{P}} \inact \Par \binp{\dual{s}}{x} (\appl{x}{s_1} \Par \dots \Par \appl{x}{s_n})
%\end{eqnarray*}
\noi Observe that $P_1$ features \emph{pure} name passing (no abstraction-passing), whereas 
$P_2$ involves \emph{pure} abstraction passing (no name passing). In both cases, 
the intended communication on $s$ leads to $n$ usages of the communication object (name $a$ in $P_1$, abstraction $\abs{x}{P}$ in $P_2$).
Consider now the reduction steps from $P_1$ and $P_2$:
\begin{eqnarray*}
P_1 & \hby{\tau} & \bout{a}{s_1} \inact \Par \dots \Par \bout{a}{s_n} \inact \\
P_2 & \hby{\tau}& \appl{(\abs{x}{P})}{s_1} \Par \dots \Par \appl{(\abs{x}{P})}{s_n} \quad 
\underbrace{\hby{\btau}\hby{\btau} \cdots \hby{\btau}}_{n} 
%\hby{}}^{n}
%\stackrel{\btau}{\longmapsto^n}
\quad P \subst{s_1}{x} \Par \dots \Par P \subst{s_1}{x} 
\end{eqnarray*}

%Let reduction on \sessp process:
%\begin{eqnarray*}
%	\bout{s}{a} \inact \Par \binp{\dual{s}}{x} (\bout{x}{s_1} \inact \Par \dots \Par \bout{x}{s_n} \inact)
%	\hby{\tau}
%	\bout{a}{s_1} \inact \Par \dots \Par \bout{a}{s_n} \inact
%\end{eqnarray*}
%and \HO process
%\begin{eqnarray*}
%	\bout{s}{\abs{x}{P}} \inact \Par \binp{\dual{s}}{x} (\appl{x}{s_1} \Par \dots \Par \appl{x}{s_n})
%	&\hby{\tau}&
%	\appl{(\abs{x}{P})}{s_1} \Par \dots \Par \appl{(\abs{x}{P})}{s_n}\\
%	&\Hby{\tau}_{n}&
%	P \subst{s_1}{x} \Par \dots \Par P \subst{s_1}{x}
%\end{eqnarray*}
\noi 
%$P_1$ and $P_2$ follow the same communication pattern; they both
%reduce on a message passing action, with the
%message being substituted $n$ times on the receing side.
%Both $P_1$ and $P_2$ are \HOp processes.
By considering the encoding of $P_1$ into \HO   
we obtain:
\begin{eqnarray*}
\map{P_1}^{1}_f & = &  	\bout{s}{\abs{z}{\binp{z}{y} \appl{y}{a}}} \inact \Par \\
& & \quad \quad \binp{\dual{s}}{x} \newsp{t}{\appl{x}{t} \Par \bout{\dual{t}}{\abs{x}{(\bout{x}{\abs{z}{\binp{z}{y} \appl{y}{s_1}}} \inact \Par \dots \Par \bout{x}{\abs{z}{\binp{z}{y} \appl{y}{s_n}}} \inact)}} \inact}\\
	& \hby{\stau} \hby{\btau} & 
%	\newsp{t}{\appl{(\abs{z}{\binp{z}{y} \appl{y}{a}})}{t} \Par \bout{\dual{t}}{\abs{x}{(\bout{x}{\abs{z}{\binp{z}{y} \appl{y}{s_1}}} \inact \Par \dots \Par \bout{x}{\abs{z}{\binp{z}{y} \appl{y}{s_n}}} \inact)}} \inact}\\
%	& \hby{\btau} & 
	\newsp{t}{\binp{t}{y} \appl{y}{a} \Par \bout{\dual{t}}{\abs{x}{(\bout{x}{\abs{z}{\binp{z}{y} \appl{y}{s_1}}} \inact \Par \dots \Par \bout{x}{\abs{z}{\binp{z}{y} \appl{y}{s_n}}} \inact)}} \inact}\\
	& \hby{\stau}\hby{\btau}  & 
%	\appl{\abs{x}{(\bout{x}{\abs{z}{\binp{z}{y} \appl{y}{s_1}}} \inact \Par \dots \Par \bout{x}{\abs{z}{\binp{z}{y} \appl{y}{s_n}}} \inact)}}{a}
%	\\
%	& \hby{\btau} & 
	\bout{a}{\abs{z}{\binp{z}{y} \appl{y}{s_1}}} \inact \Par \dots \Par \bout{a}{\abs{z}{\binp{z}{y} \appl{y}{s_n}}} \inact
\end{eqnarray*}
Now, we encode $P_2$ into \sessp:
\begin{eqnarray*}
\pmap{P_2}{2} & = & 	\newsp{b}{\bout{s}{b} \inact \Par \repl \binp{b}{y} \binp{y}{x} P} \Par \\
& & \qquad \qquad \binp{\dual{s}}{x} (\newsp{s}{\bout{x}{s} \bout{\dual{s}}{s_1} \inact} \Par \dots \Par \newsp{s}{\bout{x}{s} \bout{\dual{s}}{s_n}\inact})
	\\
%	& \hby{\stau} & 
%	\newsp{b}{\repl \binp{b}{y} \binp{y}{x} P \Par \newsp{s}{\bout{b}{s} \bout{\dual{s}}{s_1} \inact} \Par \dots \Par \newsp{s}{\bout{b}{s} \bout{\dual{s}}{s_n} \inact}}
%	\\
	& \hby{\stau}  \hby{\stau} \hby{\stau} & 
%	\newsp{b}{\repl \binp{b}{y} \binp{y}{x} P \Par \newsp{s}{\binp{s}{x} P \Par \bout{\dual{s}}{s_1} \inact} \Par \dots \Par \newsp{s}{\bout{b}{s} \bout{\dual{s}}{s_n} \inact}}
%	\\
%	& \hby{\stau} & 
	\newsp{b}{\repl \binp{b}{y} \binp{y}{x} P \Par P\subst{s_1}{x} \Par \dots \Par \newsp{s}{\bout{b}{s} \bout{\dual{s}}{s_n} \inact}}
	\\
	& \Hby{}_{2*(n - 1)} & 
	\newsp{b}{\repl \binp{b}{y} \binp{y}{x} P \Par P\subst{s_1}{x} \Par \dots \Par P\subst{s_n}{x}}
%	\red
%	\appl{V}{s_1} \Par \dots \Par \appl{V}{s_n}
\end{eqnarray*}
\noi 
It is clear that encoding $P_1$ into \HO is more economical than 
encoding $P_2$ into \sessp. Not only moving to a pure higher-order setting requires less reduction steps than in the first-order concurrency of \sessp; in the presence of shared names, moving to a first-order setting brings the need of setting up and handling replicated processes which will eventually lead to garbage processes. In contrast, the mechanism present in \HO works efficiently regardless of the linear or shared properties of the name that is ``packed'' into the abstraction. 
The use of $\beta$-transitions guarantees local synchronizations, which are arguably more economical than point-to-point, session synchronizations.

It is useful to move our comparison 
to a purely linear setting. % and to see what occurs. 
Consider processes:
%In the case of linear values we have:
\begin{eqnarray*}
	Q_1  =  \bout{s'}{s} \inact \Par \binp{\dual{s'}}{x} \bout{x}{a} \inact
	\hby{\tau}
	\bout{s}{a} \inact \qquad
	Q_2  =  \bout{s}{\abs{x}{P}} \inact \Par \binp{\dual{s}}{x} \appl{x}{a}
	\hby{\tau}
%	\appl{(\abs{x}{P})}{a}
	\hby{\tau}
	P \subst{a}{x}
\end{eqnarray*}
$Q_1$ is a \sessp process; $Q_2$ is an \HO processs.
If we consider their encodings into \HO and \sessp, respectively,
we obtain:
\begin{eqnarray*}
	\map{Q_1}_f^{1} & = & \bout{s'}{\abs{z}{\binp{z}{y} \appl{y}{s}}} \inact \Par \binp{\dual{s'}}{x} \newsp{t}{\appl{x}{t} \Par \bout{\dual{t}}{\abs{x}{\bout{x}{\abs{z}{\binp{z}{y} \appl{y}{a}}} \inact}} \inact}\\
	& \hby{\stau} \hby{\btau}& 
%	\newsp{t}{\appl{(\abs{z}{\binp{z}{y} \appl{y}{s}})}{t} \Par \bout{\dual{t}}{\abs{x}{\bout{x}{\abs{z}{\binp{z}{y} \appl{y}{a}}} \inact)}} \inact}\\
%	& \hby{\btau} & 
	\newsp{t}{\binp{t}{y} \appl{y}{s} \Par \bout{\dual{t}}{\abs{x}{\bout{x}{\abs{z}{\binp{z}{y} \appl{y}{a}}} \inact}} \inact}\\
	& \hby{\stau} & 
	\appl{\abs{x}{\bout{x}{\abs{z}{\binp{z}{y} \appl{y}{a}}} \inact}}{s}
	~~\hby{\btau}~~
	\bout{s}{\abs{z}{\binp{z}{y} \appl{y}{a}}} \inact \\
%\end{eqnarray*}
%\begin{eqnarray*}
	\pmap{Q_2}{2} & = & \newsp{t}{\bout{s}{t} \inact \Par \binp{\dual{t}}{y} \binp{y}{x} P} \Par \binp{\dual{s}}{x} \newsp{s}{\bout{x}{s} \bout{\dual{s}}{a} \inact}
	\\
	& \hby{\stau} \hby{\stau} & 
%	\newsp{t}{\binp{\dual{t}}{y} \binp{y}{x} P \Par \newsp{s}{\bout{t}{s} \bout{\dual{s}}{a} \inact}}
%	\\
%	& \hby{\stau} & 
	\newsp{s}{\binp{s}{x} P \Par \bout{\dual{s}}{a} \inact}
	~~
	\hby{\stau} ~~
	P\subst{a}{x}
\end{eqnarray*}
\noi In this case, the encoding $\pmap{\cdot}{2}$ is more efficient: it induces less reduction steps.
Therefore considering a fragment of \HOp without shared communications (linearity only)
has consequences in terms of reduction steps. Notice that we prove that linear communications do 
not suffice to encode shared communications (\secref{ss:negative}).

\jparagraph{Formal Comparison: Labelled Transition Correspondence}
%In addition to preciseness we can develop one more encodability
%result for the translation of \HOp into \HO, which is
%the correspondence of the labelled transition reduction
%system. As we will show such a correspondence does not
%hold for the \HOp to \sessp translation.
We now formally establish differences between $\map{\cdot}_f^1$ and $\pmap{\cdot}{2}$.
To this end, 
we introduce an extra encodability criteria: a form of operational correspondence 
for \emph{visible actions}. 
We shall write $\ell_1, \ell_2, \ldots$ to denote  
actions different from $\tau$
and  $\hby{\ell}$ to denote a LTS.
%with both visible and observable actions, denoted. % and $\hby{}_2$.
As actions from different calculi may be different, we also consider a mapping $\mapa{\cdot}$
on action labels. 

\begin{definition}[Labelled Correspondence / Tight Encodings]%\rm
\label{def:lopco}
       Consider typed calculi $\tyl{L}_1$ and  $\tyl{L}_2$, defined as 
        $\tyl{L}_1=\calc{\CAL_1}{{\cal{T}}_1}{\hby{{\ell_1}}_1}{\wb_1}{\proves_1}$
       and $\tyl{L}_2=\calc{\CAL_2}{{\cal{T}}_2}{\hby{{\ell_2}}_2}{\wb_2}{\proves_2}$.
%       ($i=1,2$) be typed calculi. 
The encoding $\enco{\map{\cdot}, \mapt{\cdot}}: \tyl{L}_1 \to \tyl{L}_2$ satisfies
\emph{labelled operational correspondence}
if it satisfies:
	\begin{enumerate}[1.]
			\item
					If		$\stytraargi{\Gamma}{\ell_1}{\Delta}{P}{\Delta'}{P'}{1}{1}$
					then	$\exists Q$, $\Delta''$, $\ell_2$ s.t. 
							(i)~$\wtytraargi{\mapt{\Gamma}}{\ell_2}{\mapt{\Delta}}{\map{P}}{\mapt{\Delta''}}{Q}{2}{2}$;  \\
							(ii)~$\ell_2 = \mapa{\ell_1}$; 
							(iii)~${\mapt{\Gamma}};{\mapt{\Delta''}}\proves_2 {Q}{\wb_2}{\mapt{\Delta'}}\proves_2 {\map{P'}}$.
				
			\item
					If		$\wtytraargi{\mapt{\Gamma}}{\ell_2}{\mapt{\Delta}}{\map{P}}{\mapt{\Delta'}}{Q}{2}{2}$
					then	$\exists P'$, $\Delta''$, $\ell_1$ s.t. 
							(i)~$\stytraargi{\Gamma}{\ell_1}{\Delta}{P}{\Delta''}{P'}{1}{1}$;
							(ii)~$\ell_2 = \mapa{\ell_1}$;
							(iii)~${\mapt{\Gamma}};{\mapt{\Delta''}}\proves_2 {\map{P'}}{\wb_2}{\mapt{\Delta'}}\proves_2 {Q}$.
	\end{enumerate}
A \emph{tight encoding} is a typed 
encoding 
which is precise (\defref{def:goodenc}) and that also satisfies 
labelled operational correspondence as above.
\end{definition}

We have the following result, which attests that 
\HOp and \HO are more tightly related than \HOp and \sessp:
\begin{theorem}\label{t:tight}
While the encoding of \HOp into \HO (\defref{d:enc:hopitoho}) is tight, the encoding of \HOp into \sessp (\defref{d:enc:hopitopi}) is not tight.
\end{theorem}

\noi We substantiate the above claim by showing that the encoding $\map{\cdot}^1_f$ enjoys 
labelled operational correspondence, whereas $\pmap{\cdot}{2}$ does not. 
Consider the following mapping:
\[
	\begin{array}{rclcrcl}
		\mapa{\news{\tilde{m_1}}\bactout{n}{m}}^{1}	&\defeq&	\news{\tilde{m_1}}\bactout{n}{\abs{z}{\,\binp{z}{x} \appl{x}{m}} }
		& &
		\mapa{\bactinp{n}{m}}^{1}			&\defeq&	\bactinp{n}{\abs{z}{\,\binp{z}{x} \appl{x}{m}} }
		\\
		\mapa{\news{\tilde{m}}\bactout{n}{\abs{x}{P}}}^{1} &\defeq& \news{\tilde{m}}\bactout{n}{\abs{x}{\pmapp{P}{1}{\es}}}
		& &
		\mapa{\bactinp{n}{\abs{x}{P}}}^{1} &\defeq& \bactinp{n}{\abs{x}{\pmapp{P}{1}{\es}}}
		\\
		\mapa{\bactsel{n}{l} }^{1} &\defeq& \bactsel{n}{l} 
		& &
		\mapa{\bactbra{n}{l} }^{1} &\defeq& \bactbra{n}{l} 
%		\\
%		\mapa{\tau}^{1} &\defeq& \tau
	\end{array}
\]



Then the following result, a complement of \propref{prop:op_corr_HOp_to_HO}, holds:

\begin{proposition}[Labelled Transition Correspondence, \HOp into \HO]
	\label{prop:lts_corr_HOp_to_HO}
	Let $P$ be an \HOp process.
	If $\Gamma; \emptyset; \Delta \proves P \hastype \Proc$ then:
%
	\begin{enumerate}[1.]
		\item
			Suppose $\horel{\Gamma}{\Delta}{P}{\hby{\ell_1}}{\Delta'}{P'}$. Then we have:
%
			\begin{enumerate}[a)]
				\item
					If $\ell_1 \in \set{\news{\tilde{m}}\bactout{n}{m}, \,\news{\tilde{m}}\bactout{n}{\abs{x}Q}, \,\bactsel{s}{l}, \,\bactbra{s}{l}}$
					then $\exists \ell_2$ s.t. \\
					$\horel{\tmap{\Gamma}{1}}{\tmap{\Delta}{1}}{\pmapp{P}{1}{f}}{\hby{\ell_2}}{\tmap{\Delta'}{1}}{\pmapp{P'}{1}{f}}$
					and $\ell_2 = \mapa{\ell_1}^{1}$.
			
				\item
					If $\ell_1 = \bactinp{n}{\abs{y}Q}$ and
					$P' = P_0 \subst{\abs{y}Q}{x}$
					then $\exists \ell_2$ s.t. \\
					$\horel{\tmap{\Gamma}{1}}{\tmap{\Delta}{1}}{\pmapp{P}{1}{f}}{\hby{\ell_2}}{\tmap{\Delta'}{1}}{\pmapp{P_0}{1}{f}\subst{\abs{y}\pmapp{Q}{1}{\emptyset}}{x}}$
					and $\ell_2 = \mapa{\ell_1}^{1}$.
			
				\item
					If $\ell_1 = \bactinp{n}{m}$
					and 
					$P' = P_0 \subst{m}{x}$
					then $\exists \ell_2$, $R$ s.t. 
					$\horel{\tmap{\Gamma}{1}}{\tmap{\Delta}{1}}{\pmapp{P}{1}{f}}{\hby{\ell_2}}{\tmap{\Delta'}{1}}{R}$, \\
					with $\ell_2 = \mapa{\ell_1}^{1}$, 
					and
					$\horel{\tmap{\Gamma}{1}}{\tmap{\Delta'}{1}}{R}{\hby{\stau} \hby{\btau} \hby{\btau}}
					{\tmap{\Delta'}{1}}{\pmapp{P_0}{1}{f}\subst{m}{x}}$.
						
%				\item
%					If $\ell_1 = \tau$
%					and $P' \scong \newsp{\tilde{m}}{P_1 \Par P_2\subst{m}{x}}$
%					then $\exists R$ s.t. \\
%					$\horel{\tmap{\Gamma}{1}}{\tmap{\Delta}{1}}{\pmapp{P}{1}{f}}{\hby{\tau}}{\mapt{\Delta}^{1}}{\newsp{\tilde{m}}{\pmapp{P_1}{1}{f} \Par R}}$,
%					and\\ 
%					$\horel{\tmap{\Gamma}{1}}{\tmap{\Delta}{1}}{\newsp{\tilde{m}}{\pmapp{P_1}{1}{f} \Par R}}{\hby{\btau} \hby{\stau} \hby{\btau}}
%					{\mapt{\Delta}^{1}}{\newsp{\tilde{m}}{\pmapp{P_1}{1}{f} \Par \pmapp{P_2}{1}{f}\subst{m}{x}}}$.
%			
%				\item
%					If $\ell_1 = \tau$
%					and $P' \scong \newsp{\tilde{m}}{P_1 \Par P_2 \subst{\abs{y}Q}{x}}$
%					then \\
%					$\horel{\tmap{\Gamma}{1}}{\tmap{\Delta}{1}}{\pmapp{P}{1}{f}}{\hby{\tau}}
%					{\tmap{\Delta_1}{1}}{\newsp{\tilde{m}}{\pmapp{P_1}{1}{f}\Par \pmapp{P_2}{1}{f}\subst{\abs{y}\pmapp{Q}{1}{\emptyset}}{x}}}$.
%			
%				\item
%					If $\ell_1 = \tau$
%					and $P' \not\scong \newsp{\tilde{m}}{P_1 \Par P_2 \subst{m}{x}} \land P' \not\scong \newsp{\tilde{m}}{P_1 \Par P_2\subst{\abs{y}Q}{x}}$
%					then \\
%					$\horel{\tmap{\Gamma}{1}}{\tmap{\Delta}{1}}{\pmapp{P}{1}{f}}{\hby{\tau}}{\tmap{\Delta'_1}{1}}{ \pmapp{P'}{1}{f}}$.
			\end{enumerate}
			
		\item	Suppose $\horel{\tmap{\Gamma}{1}}{\tmap{\Delta}{1}}{\pmapp{P}{1}{f}}{\hby{\ell_2}}{\tmap{\Delta'}{1}}{Q}$.
			Then we have:
%
			\begin{enumerate}[a)]
				\item 
					If $\ell_2 \in
					\set{\news{\tilde{m}}\bactout{n}{\abs{z}{\,\binp{z}{x} (\appl{x}{m})}}, \,\news{\tilde{m}} \bactout{n}{\abs{x}{R}}, \,\bactsel{s}{l}, \,\bactbra{s}{l}}$
					then $\exists \ell_1, P'$ s.t. \\
					$\horel{\Gamma}{\Delta}{P}{\hby{\ell_1}}{\Delta'}{P'}$, 
					$\ell_1 = \mapa{\ell_2}^{1}$, 
					and
					$Q = \pmapp{P'}{1}{f}$.
			
				\item 
					If $\ell_2 = \bactinp{n}{\abs{y} R}$ %(with $R \neq \binp{y}{x} \appl{x}{m}$)
					then either:
%
					\begin{enumerate}[(i)]
						\item	$\exists \ell_1, x, P', P''$ s.t. \\
							$\horel{\Gamma}{\Delta}{P}{\hby{\ell_1}}{\Delta'}{P' \subst{\abs{y}P''}{x}}$, 
							$\ell_1 = \mapa{\ell_2}^{1}$, $\pmapp{P''}{1}{\es} = R$, and $Q = \pmapp{P'}{1}{f}$.

						\item	$R \scong \binp{y}{x} (\appl{x}{m})$ and 
							$\exists \ell_1, z, P'$ s.t. 
							$\horel{\Gamma}{\Delta}{P}{\hby{\ell_1}}{\Delta'}{P' \subst{m}{z}}$, \\
							$\ell_1 = \mapa{\ell_2}^{1}$,
							and 
							$\horel{\tmap{\Gamma}{1}}{\tmap{\Delta'}{1}}{Q}{\hby{\stau} \hby{\btau} \hby{\btau}}{\tmap{\Delta''}{1}}{\pmapp{P'\subst{m}{z}}{1}{f}}$
					\end{enumerate}
			
%				\item 
%					If $\ell_2 = \tau$ 
%					then $\Delta' = \Delta$ and 
%					either
%%
%					\begin{enumerate}[(i)]
%						\item	$\exists P'$ s.t. 
%							$\horel{\Gamma}{\Delta}{P}{\hby{\tau}}{\Delta}{P'}$,
%							and $Q = \map{P'}^{1}_f$.	
%
%						\item
%							$\exists P_1, P_2, x, m, Q'$ s.t. 
%							$\horel{\Gamma}{\Delta}{P}{\hby{\tau}}{\Delta}{\newsp{\tilde{m}}{P_1 \Par P_2\subst{m}{x}} }$, and\\
%							$\horel{\tmap{\Gamma}{1}}{\tmap{\Delta}{1}}{Q}{\hby{\btau} \hby{\stau} \hby{\btau}}{\tmap{\Delta}{1}}{\pmapp{P_1}{1}{f} \Par \pmapp{P_2\subst{m}{x}}{1}{f}}$ 
%%							$Q = \map{P_1}^{1}_f \Par Q'$, where $Q'  \Hby{} $.
%
%%						\item $\exists P_1, P_2, x, R$ s.t. 
%%						$\stytra{ \Gamma }{\tau}{ \Delta }{ P}{ \Delta}{ \news{\tilde{m}}(P_1 \Par P_2\subst{\abs{y}R}{x}) }$, and 
%%						$Q = \map{\news{\tilde{m}}(P_1 \Par P_2\subst{\abs{y}R}{x})}^{1}_f$.
%			\end{enumerate}
		    \end{enumerate}
		    
%		\item   
%			If  $\wtytra{\mapt{\Gamma}^{1}}{\ell_2}{\mapt{\Delta}^{1}}{\pmapp{P}{1}{f}}{\mapt{\Delta'}^{1}}{Q}$
%			then $\exists \ell_1, P'$ s.t.  \\
%			(i)~$\stytra{\Gamma}{\ell_1}{\Delta}{P}{\Delta'}{P'}$,
%			(ii)~$\ell_2 = \mapa{\ell_1}^{1}$, 
%			(iii)~$\wbb{\mapt{\Gamma}^{1}}{\ell}{\mapt{\Delta'}^{1}}{\pmapp{P'}{1}{f}}{\mapt{\Delta'}^{1}}{Q}$.
	\end{enumerate}
\end{proposition}

The analog of \propref{prop:lts_corr_HOp_to_HO} does not hold for the encoding of \HOp into \sessp.
Consider the \HOp process:
\[
	\Gamma; \es; \Delta \proves \bout{s}{\abs{x}{P}} \inact \hastype \Proc \hby{\bactout{s}{\abs{x} P}} \es \proves \inact \not \hby{}
\]
with $\abs{x}{P}$ being a linear value.
We translated into \sessp process:
\[\tmap{\Gamma}{2}; \es; \tmap{\Delta}{2} \proves \newsp{a}{\bout{s}{a} \inact \Par \binp{a}{y} \binp{y}{x} P} \hastype \Proc
	 \hby{\bactout{s}{a}} \Delta' \proves \binp{a}{y} \binp{y}{x} P \hastype \Proc
\hby{\bactinp{a}{V}} \dots
\]

%\begin{eqnarray*}
%	&&\tmap{\Gamma}{2}; \es; \tmap{\Delta}{2} \proves \newsp{a}{\bout{s}{a} \inact \Par \binp{a}{y} \binp{y}{x} P} \hastype \Proc\\
%	&&\hby{\bactout{s}{a}}\\
%	&&\Delta' \proves \binp{a}{y} \binp{y}{x} P \hastype \Proc\\
%	&&\hby{\bactinp{a}{V}} \dots
%\end{eqnarray*}

\noi The resulting processes have a mismatch both in the typing
environment ($\Delta' \not= \tmap{\es}{2}$)
and in the actions that they can %the two processes can
subsequently observe: the first process
cannot perform any action, while the second process
can performs actions of the encoding of $\abs{x}{P}$.

