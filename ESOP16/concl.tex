% !TEX root = main.tex

We have provided a thorough study of the expressivity of the higher-order $\pi$-calculus with sessions, here denoted $\HOp$.
Unlink most previous works on the expressivity of 
(higher-order) process calculi, we have carried out our study in the setting of \emph{session types} for structured communications.
Types not only delineate process translations; they 
inform the reasoning techniques required to reason about such translations.

Prerequisites for our developments are notions of typed behavioral equivalences and abstract notions of 
\emph{encodings} that extend encodability criteria with (session) types.
By incorporating session types, we defined the notion of \emph{precise  encodings}.
To the best of our knowledge, ours is the first definition of its kind, and builds upon well-established notions, for instance~\cite{DBLP:journals/iandc/Gorla10}. We explored precise encodability in the most representative subcalculi of \HOp, 
namely its first- and higher-order fragments, here denoted \sessp and \HO, respectively. These two sub-calculi are known to be fairly expressive in their own. A main result of precise encodability is our encoding of \HOp into \HO:
since \HO lacks name passing and recursion, this typed encoding is far from trivial, also because precise encodability imposes strict conditions, which rule out plausible encoding strategies that do not preserve (session) types. Our study also includes extensions of \HOp with higher-order abstractions (functions from processes to processes) and polyadicity, here denoted \HOpp and \PHOp, respectively. We extend these results to these super-calculi, therefore connecting with existing higher-order process calculi with sessions~\cite{tlca07}.

Overall, 
our results cover the whole spectrum of features intrinsic to higher-order session-based concurrency:
pure process-passing (with first- and higher-order abstractions), pure name-passing, polyadicity, 
linear/shared communication, and their combination (see \figref{fig:express}). Remarkably, 
the behavioral discipline embodied by 
session types turns out to be fundamental to show that all these languages are equally expressive, up to 
very strong behavioral correspondences. Indeed, although our encodings may be exploited in an untyped setting,
session type information is critical to prove key properties for preciseness, in particular full abstraction.
