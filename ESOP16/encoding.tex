% !TEX root = main.tex
\noi Here we define the formal notion of \emph{encoding} by 
extending to a typed setting existing criteria for untyped processes (as in, e.g.,
\cite{Nestmann00,Palamidessi03,DBLP:conf/lics/PalamidessiSVV06,DBLP:journals/iandc/Gorla10,DBLP:conf/icalp/LanesePSS10,DBLP:journals/tcs/FuL10,DBLP:journals/corr/abs-1208-2750}). 
We first define a typed calculus parameterised by a syntax, operational semantics, and typing.

\smallskip 

\begin{definition}[Typed Calculus]\label{d:tcalculus}\rm
	A \emph{typed calculus} $\tyl{L}$ is a tuple
	$\calc{\CAL}{\cal{T}}{\hby{\tau}}{\wb}{\proves}$
	where $\CAL$ and $\cal{T}$ are sets of processes and types, 
	respectively; and $\hby{\tau}$, $\wb$, and $\proves$ 
	denote a transition system, a typed equivalence,
	and a typing system for $\CAL$, respectively. 
\end{definition}

%\smallskip 
%
%\begin{definition}[Typed Calculus]\label{d:tcalculus}\rm
%A \emph{typed calculus} $\tyl{L}$ is a tuple
%          $\calc{\CAL}{\cal{T}}{\cal{A}}{\wb}{\proves}$
%	where $\CAL$,  $\cal{T}$, $\cal{A}$ are sets of processes, types, and action labels (of an underlying transition system),
%respectively; and $\wb$ and $\proves$ 
%	denote %a transition system (with set of labels $\mathcal{A}$), 
%	a typed equivalence and type system for~$\CAL$. 
%\end{definition}
%
%
%\smallskip 
%
%\begin{definition}[Typed Calculus]\label{d:tcalculus}\rm
%A \emph{typed calculus} $\tyl{L}$ is a tuple
%          $\calc{\CAL}{\cal{T}}{\hby{\ell}}{\wb}{\proves}$
%	where $\CAL$ and $\cal{T}$ are sets of processes and types, 
%respectively; and $\hby{\mathcal{A}}$, $\wb$, and $\proves$ 
%	denote a transition system (with set of labels $\mathcal{A}$), 
%	a typed equivalence, and type system for $\CAL$, resp. 
%	We write $\mathcal{A}$ is t`he set of labels used in relation $\hby{\ell}$.
%\end{definition}

\smallskip 

\noi 
As we explain later, we write $\hby{\tau}$ to denote an operational semantics defined in terms of
$\tau$-transitions (to characterise reductions).
Based on this definition, later on we define concrete instances of (higher-order) typed calculi.
Our notion of encoding considers mappings on processes, 
types, and transition labels: 

\begin{definition}[Typed Encoding]\rm
\label{def:tenc}
        Let  
        $\tyl{L}_1=\calc{\!\CAL_1}{{\cal{T}}_1}{\hby{\tau}_1}{\wb_1}{\proves_1}$
        and
        $\tyl{L}_2=\calc{\CAL_2}{{\cal{T}}_2}{\hby{\tau}_2}{\wb_2}{\proves_2}$
         be typed calculi. 
        %Let $\mathcal{A}_{i}$ be the set of labels in $\hby{\ell}_i$ ($i=1,2$).
	Given mappings $\map{\cdot}: \CAL_1 \to \CAL_2$ and
	$\mapt{\cdot}: {\cal{T}}_1 \to {\cal{T}}_2$, 
%	and $\mapa{\cdot}: \mathcal{A}_1 \to \mathcal{A}_2$, 
	we write 
%	$\enco{\map{\cdot}, \mapt{\cdot}, \mapa{\cdot}} : 
		$\enco{\map{\cdot}, \mapt{\cdot}} : 
	\tyl{L}_1 \to \tyl{L}_2$ to denote the \emph{typed encoding} of $\tyl{L}_1$ into $\tyl{L}_2$.
\end{definition}

\smallskip 

\noi We will often assume that  $\mapt{\cdot}$ extends to typing
environments as expected. This way, e.g., $\mapt{\Delta \cat u:S} = \mapt{\Delta} \cat u:\mapt{S}$.

We introduce two classes of typed encodings, which 
serve different purposes. 
%Both  consist of syntactic and semantic criteria 
%first proposed for untyped processes~\cite{Palamidessi03,DBLP:journals/iandc/Gorla10,DBLP:conf/icalp/LanesePSS10}, and here extended to account for (higher-order) session types.
First, for stating stronger positive encodability results, % in \secref{sec:positive}, 
we define the notion of {\em precise} encodings.
Then, 
for proving the strong non-encodability result, % in \secref{sec:negative}, 
precise encodings are relaxed into the weaker {\em minimal} encodings. 

We first state the 
syntactic criteria. 
Let $\sigma$ denote a substitution of names for names (a renaming, in the usual sense). Given environments $\Delta$ and $\Gamma$,
we write $\sigma(\Delta)$ and $\sigma(\Gamma)$ to denote the effect of applying $\sigma$ on the 
domains of $\Delta$ and $\Gamma$
(clearly, $\sigma(\Gamma)$ concerns only shared names in $\Gamma$: process and recursive variables in $\Gamma$ are not affected by $\sigma$). 

\smallskip 

\begin{definition}[Syntax Preserving Encoding]\rm
	\label{def:sep}
	%We say that 
	The typed encoding 
	$\enco{\map{\cdot}, \mapt{\cdot}}: \tyl{L}_1 \to \tyl{L}_2$ is \emph{syntax preserving}
	if it is:
	
	\begin{enumerate}[1.]
		\item	\emph{Homomorphic wrt parallel},   if 
		$\mapt{\Gamma}; \emptyset; \mapt{\Delta_1 \cat \Delta_2} \proves_2 \map{P_1 \Par P_2} \hastype \Proc$ \\
		then 
		$\mapt{\Gamma}; \emptyset; \mapt{\Delta_1} \cat \mapt{\Delta_2} \proves_2 \map{P_1} \Par \map{P_2} \hastype \Proc$.

		\item	\emph{Compositional wrt restriction},  if 
		$\mapt{\Gamma}; \emptyset; \mapt{\Delta} \proves_2 \map{\news{n}P} \hastype \Proc$ \\
		then 
		$\mapt{\Gamma}; \emptyset; \mapt{\Delta} \proves_2 \news{n}\map{P} \hastype \Proc$.
		
		\item \emph{Name invariant},   if
		$\mapt{\sigma(\Gamma)}; \emptyset; \mapt{\sigma(\Delta)} \proves_2 \map{\sigma(P)} \hastype \Proc$
		then \\
		$\sigma(\mapt{\Gamma}); \emptyset; \sigma(\mapt{\Delta}) \proves_2 \sigma(\map{P}) \hastype \Proc$, 
		for any injective renaming  of names $\sigma$.
	\end{enumerate}
\end{definition}

\smallskip 

\noi Homomorphism wrt parallel (used in, e.g.,~\cite{Palamidessi03,DBLP:conf/lics/PalamidessiSVV06})
expresses that encodings should preserve the distributed topology of source processes. This criteria 
 is appropriate for both encodability and non encodability results; in our setting, it is
%it admits an elegant formulation, also 
induced by rules for typed composition.
Compositionality wrt restriction 
is also supported by typing and turns out to be 
useful in our encodability results (\secref{sec:positive}).
The name invariance criteria follows \cite{DBLP:journals/iandc/Gorla10,DBLP:conf/icalp/LanesePSS10}. 

We now state a static criteria on the mapping 
$\mapt{\cdot}: {\cal{T}}_1 \to {\cal{T}}_2$. % of our typed encodings. 
The \emph{type preservation} criteria below ensures that type operators are preserved.
In our setting, 
we have five type operators: input, output, recursion (binary operators); selection and 
branching ($n$-ary operators). 
Observe that the source and target languages we shall consider share these (session) type operators.
Type preservation enables us to focus on 
mappings $\mapt{\cdot}$ in which a session type operator is always translated into itself. 
In turn, this is key to retain the meaning of structured protocols across typed encodings.

\begin{definition}[Type Preserving Encoding]
	\label{def:tp}
	The typed encoding 
	$\enco{\map{\cdot}, \mapt{\cdot}}: \tyl{L}_1 \to \tyl{L}_2$ is \emph{type preserving}
	if for every $k$-ary type operator $\mathtt{op}$ in ${\cal{T}}_1$ it holds that 
	 $$\mapt{\mathtt{op}(T_1, \cdots, T_k)} = \mathtt{op}(\mapt{T_1}, \cdots, \mapt{T_k})$$
	\end{definition}
This way, e.g., 
consider a mapping of types $\mapt{\cdot}_1$ such that 
$\mapt{\btout{U} S}_1 = \btinp{\mapt{U}_1} \mapt{S}_1$ 
and 
$\mapt{\btinp{U} S}_1 = \btout{\mapt{U}_1} \mapt{S}_1$. This a mapping which exchanges
input and output session type operators and therefore does not
satisfy the type preservation criteria above. % in \defref{def:tp}.

Next we define semantic criteria: %for typed encodings.

%\smallskip 

\begin{definition}[Semantic Preserving Encoding]\rm
\label{def:ep}
       Consider typed calculi
        $\tyl{L}_1=\calc{\CAL_1}{{\cal{T}}_1}{\hby{\tau}_1}{\wb_1}{\proves_1}$
       and $\tyl{L}_2=\calc{\CAL_2}{{\cal{T}}_2}{\hby{\tau}_2}{\wb_2}{\proves_2}$.
%       ($i=1,2$) be typed calculi. 
We say that $\enco{\map{\cdot}, \mapt{\cdot}}: \tyl{L}_1 \to \tyl{L}_2$ is a \emph{semantic preserving encoding}
if it satisfies the properties below.
%Given a label $\ell \neq \tau$, we write 
%$\mathsf{sub}(\ell)$
%to denote the \emph{subject} of the action.
	
	\begin{enumerate}[1.]
		\item \emph{Type Soundness}:
	if
	$\Gamma; \emptyset; \Delta \proves_1 P \hastype \Proc$ then 
	$\mapt{\Gamma}; \emptyset; \mapt{\Delta} \proves_2 \map{P} \hastype \Proc$,  
	for any   $P$ in $\CAL_1$.
			%\item \emph{Subject preserving}: if $\subj{\ell} = u$ then $\subj{\mapa{\ell}} =u$.

			\item \emph{Barb preserving}: if $\Gamma; \Delta \proves_1 P \barb{n}$
		then $\mapt{\Gamma}; \mapt{\Delta} \proves_2 \map{P} \Barb{n}$

	\item \emph{Operational Correspondence}: If $\Gamma; \emptyset; \Delta \proves_1 P \hastype \Proc$ then
		\begin{enumerate}
			\item	\NY{Completeness: 
			   If  
$\stytraargi{\Gamma}{\tau}{\Delta}{P}{\Delta'}{P'}{1}{1}$
			   then  $\exists Q, \Delta''$ s.t. \\
 (i)~$\wtytraargi{\mapt{\Gamma}}{}{\mapt{\Delta}}{\map{P}}{\mapt{\Delta''}}{Q}{2}{2}$
			    %(ii)~$\ell_2 = \mapa{\ell_1}$, 
			    and 
				(ii)~${\mapt{\Gamma}};{\mapt{\Delta''}}\proves_2 {Q}{\wb_2}
{\mapt{\Delta'}}\proves_2 {\map{P'}}$.}
				
			\item	Soundness:   
				If  $\wtytraargi{\mapt{\Gamma}}{}{\mapt{\Delta}}{\map{P}}{\mapt{\Delta'}}{Q}{2}{2}$
				then  $\exists P', \Delta''$ s.t.  \\
				(i)~$\stytraargi{\Gamma}{\tau}{\Delta}{P}{\Delta''}{P'}{1}{1}$
				%(ii)~$\ell_2 = \mapa{\ell_1}$, 
				and 
				(ii)~
${\mapt{\Gamma}};{\mapt{\Delta''}}\proves_2 {\map{P'}}{\wb_2}
{\mapt{\Delta'}}\proves_2 {Q}$.

		\end{enumerate}
		
		\item \emph{Full Abstraction:} 
		\wbbarg{\Gamma}{}{\Delta_1}{P}{\Delta_2}{Q}{1}
		if and only if \\
		\wbbarg{\mapt{\Gamma}}{}{\mapt{\Delta_1}}{\map{P}}{\mapt{\Delta_2}}{\map{Q}}{2}.
		
	\end{enumerate}
\end{definition}

\smallskip 

\noi Together with type preservation (\defref{def:tp}), type soundness is a distinguishing criterion in our notion of encoding: it enables us to focus on encodings which retain the communication structures denoted by session types.
%The other semantic
%criteria build upon analogous definitions in the untyped setting, as we explain now. 
Operational correspondence, standardly divided into completeness and soundness, is based
%in the formulation given i
on~\cite{DBLP:journals/iandc/Gorla10,DBLP:conf/icalp/LanesePSS10};
it relies on 
%the typed LTS of \defref{def:rlts}, 
%labelled transitions rather than on 
$\tau$-labeled transitions (reductions).
Completeness ensures that an action of the source process is mimicked
by an action of its associated encoding; soundness is its converse.
%Soundness ensures that the source process is mimicked 
%by its associated encoding; completeness is its converse.
%Completeness and soundness rely on 
%the typed LTS of \defref{def:rlts}, 
%rather than on reductions;
%Labels are considered up to  mapping $\mapa{\cdot}$, which offers flexibility when comparing different calculi. We require that $\mapa{\cdot}$ preserves  subjects, in accordance with the criteria in~\cite{DBLP:conf/icalp/LanesePSS10}.
\jpc{Above, operational correspondence is stated in generic terms.}
It is worth stressing that 
the operational correspondence statements 
%given in \secref{sec:positive} 
for our encodings 
 are tailored to the specifics of each encoding, and so they
 are actually stronger than the criteria given above
 \jpc{(see~\cite{KouzapasPY15} for details).}
Finally, following~\cite{SangiorgiD:expmpa,DBLP:conf/lics/PalamidessiSVV06,Yoshida96},
we consider full abstraction as an encodability criterion: this results into 
stronger encodability results. 
%The completeness direction of full abstraction is dropped when we prove the negative result. 
%From the criteria in \defref{def:sep} and~\ref{def:ep}
%we have the following derived criterion: 

%\smallskip 
%
%\begin{proposition}[Barb Preservation]\label{p:barbpres}
%Let
%	$\enco{\map{\cdot}, \mapt{\cdot}, \mapa{\cdot}}: \tyl{L}_1 \to \tyl{L}_2$
%	be a typed encoding.
%%	\begin{enumerate}[1.]
%%	\item Suppose the typed encoding is name invariant. Then $u \in \fn{P}$ implies $u \in \fn{\map{P}}$.
%%	\item 
%	Suppose the encoding is both
%% name invariant (cf.~\defref{def:sep}),
% operationally complete (cf.~\defref{def:ep}-3(a)) 
% and subject preserving (cf.~\defref{def:ep}-2).
% Then, it is also \emph{barb preserving}, i.e., 
%$\Gamma; \Delta \proves_1 P \barb{n}$
%implies
%$\mapt{\Gamma}; \mapt{\Delta} \proves_2 \map{P} \Barb{n}$.
%%\end{enumerate}
%\end{proposition}

%\smallskip 

%\begin{proof}%[Sketch]
%%Part (1) follows by contradiction.
%%Consider two distinct names $v$ and $u$.
%%Suppose $u \in \fn{P}$. Clearly:
%%$$P\subst{v}{u} \neq P$$
%%For the sake of contradiction, now suppose $u \not\in \fn{\map{P}}$.
%%Then, $\map{P}\subst{v}{u} = \map{P}$ (the substitution has no effect). 
%%Name invariance  ensures that
%%$\map{P}\subst{v}{u} = \map{P\subst{v}{u}}$. 
%%By the inequality
%%on source processes given above, one has $\map{P}\subst{v}{u} \neq \map{P\subst{v}{u}}$,
%%a contradiction.
%%
%%Part (2) 
%The proof
%follows from the definition of barbs, operational completeness, and subject preservation.
%\qed
%\end{proof}

We may now define \emph{precise} and \emph{minimal} typed criteria: 

\smallskip 

\begin{definition}[Typed Encodings: Precise and Minimal]\rm\label{def:goodenc}
We say that 
	the typed encoding 
	$\enco{\map{\cdot}, \mapt{\cdot} %, \mapa{\cdot}
	}: \tyl{L}_1 \to \tyl{L}_2$ is 
	\begin{enumerate}[$\bullet$]
	\item \emph{precise}, if it is syntax, type, and semantic preserving (cf.~\defsref{def:sep}, \ref{def:tp}, and~\ref{def:ep}).
	\item \emph{minimal}, if it is syntax preserving 
	(cf.~\defref{def:sep}),
	barb preserving (cf.~\defref{def:ep}-2), 
	and operationally complete (cf.~\defref{def:ep}-3(a)).
	\end{enumerate}
\end{definition}

\smallskip 

\noi %As explained earlier, o
Our encodability results %presented next 
rely on precise encodings; 
our non encodability result %, presented in \secref{sec:negative}, 
uses minimal encodings.
Further we have:

\smallskip 

\begin{proposition}[Composability of Precise Encodings]\rm
	\label{pro:composition}
	Let %encodings 
	$\enco{\map{\cdot}^{1}, \mapt{\cdot}^{1}%, \mapa{\cdot}^{1}
	}: \tyl{L}_1 \to \tyl{L}_2$
	and 
	$\enco{\map{\cdot}^{2}, \mapt{\cdot}^{2}%, \mapa{\cdot}^{2}
	}: \tyl{L}_2 \to \tyl{L}_3$
	be two precise typed encodings.
	Then their composition, denoted 
	$\enco{\map{\cdot}^{2} \circ \map{\cdot}^{1}, \mapt{\cdot}^{2} \circ \mapt{\cdot}^{1} %, \mapa{\cdot}^{2}\circ \mapa{\cdot}^{1}
	}: \tyl{L}_1 \to \tyl{L}_3$
	is precise. 
\end{proposition}

\smallskip

\noi {\bf\em Concrete Typed Calculi:}
In \secref{sec:positive} %and \secref{sec:negative}, 
we consider the following concrete instances of typed calculi (cf.~\defref{d:tcalculus}): \\
%
%\begin{enumerate}[-]
%	\item
	$\tyl{L}_{\HOp}=\calc{\HOp}{{\cal{T}}_1}{\hby{\tau}}{\hwb}{\proves}$,
%	\item
	$\tyl{L}_{\HO}=\calc{\HO}{{\cal{T}}_2}{\hby{\tau}}{\hwb}{\proves}$, and
%	\item
	$\tyl{L}_{\sessp}=\calc{\sessp}{{\cal{T}}_3}{\hby{\tau}}{\fwb}{\proves}$ 
%\end{enumerate}
%
where: 
${\cal{T}}_1$, ${\cal{T}}_2$, 
and ${\cal{T}}_3$
are sets of types of $\HOp$, $\HO$, and $\sessp$, resp.;
LTSs are as in \defref{def:rlts}, 
the typing $\proves$ is defined in 
\figref{fig:typerulesmy};  
$\hwb$ is as in \defref{d:hbw}; 
$\fwb$ is as in \defref{d:fwb}.


\dk{
\subsection{Equivalence Relations}

In~\cite{characteristic_bis} we have characterised
reduction-closed, barbed congruence for the \HOp
with a tractable bisimularity relation called
{\em characteristic bisimularity}. The definition
of the characteristic bisimilarity uses a
labelled transition system informed by session
types. The main intuition for the labelled transition
semantics is the fact that we observe a transition
on a typed process only if the type environment of
the process allows it:
%
\[
	\tree {
		P \hby{\ell} P' \qquad (\Gamma, \Delta) \hby{\ell} (\Gamma, \Delta')
	}{
		\Gamma; \es; \Delta \proves P \hby{\ell} \Delta' \proves P' \hastype \Delta'
	}
\]
%
\noi An example of how types allow transitions can be seen in the rule:
%
\[
	\tree{
		\dual{s} \notin \dom{\Delta} \qquad \Gamma; \Lambda'; \Delta' \proves V \hastype U
		\qquad
		V = m \vee  V \scong \omapchar{U} \vee V \scong \abs{{x}}{\binp{t}{y} (\appl{y}{{x}})}
					\textrm{ with } t \textrm{ fresh} 
	}{
		(\Gamma; \Lambda; \Delta \cat s: \btinp{U} S) \hby{\bactinp{s}{V}} (\Gamma; \Lambda\cat\Lambda'; \Delta\cat\Delta' \cat s: S)
	}
\]

The above rule states that an session channel environment can input a value
if the channel is typed with an input prefix and the input value is either
a name or a characteristic value or a trigger value. A characteristic value
(cf.~Def.~11~in~\cite{characteristic_bis})
is the {\em simplest} process that can inhabit a type, in this case the
type carried by the session input prefix. A trigger value is the abstraction
$\abs{{x}}{\binp{t}{y} (\appl{y}{{x}})}$. The above rule is used to restrict
the input actions that can be observed from a session input prefix.
For full details of the labelled transition system see~\cite{characteristic_bis}.

We define two trigger processes:
%
\begin{eqnarray*}
	\htrigger{t}{V_1}  & \defeq &  \hotrigger{t}{V} \label{eqb:0} \\
	\ftrigger{t}{V}{U} & \defeq &  \fotrigger{t}{x}{s}{\btinp{U} \tinact}{V} 	\label{eqb:4}
\end{eqnarray*}
%

A trigger process is defined as a process input prefixed on
a fresh name. The first trigger process applies a value
on the receiving (characteristic) process.
The second trigger process applies a value
on the (hard coded) characteristic process:

We now define Higher-Order Bisimilarity and Characteristic Bisimilarity:

\begin{definition}[Higher-Order Bisimilarity]\rm
	\label{d:hbw}
	A typed relation $\Re$ is a {\em  HO bisimulation} if 
	for all $\horel{\Gamma}{\Delta_1}{P_1}{\ \Re \ }{\Delta_2}{Q_1}$ 
	\begin{enumerate}[1)]
		\item 
				Whenever 
				$\horel{\Gamma}{\Delta_1}{P_1}{\hby{\news{\tilde{m_1}} \bactout{n}{V_1}}}{\Delta_1'}{P_2}$, there exist 
				$Q_2$, $V_2$, $\Delta'_2$ such that 
				$\horel{\Gamma}{\Delta_2}{Q_1}{\Hby{\news{\tilde{m_2}} \bactout{n}{V_2}}}{\Delta_2'}{Q_2}$ and, for fresh $t$, 
%
				\[
					\begin{array}{lrlll}
						\Gamma; \Delta''_1  \proves  {\newsp{\tilde{m_1}}{P_2 \Par \htrigger{t}{V_1}}}
						\ \Re \
						\Delta''_2 \proves {\newsp{\tilde{m_2}}{Q_2 \Par \htrigger{t}{V_2}}}
					\end{array}
				\]

		\item	
				For all $\horel{\Gamma}{\Delta_1}{P_1}{\hby{\ell}}{\Delta_1'}{P_2}$ such that 
				$\ell$ is not an output, 
				 there exist $Q_2$, $\Delta'_2$ such that 
				$\horel{\Gamma}{\Delta_2}{Q_1}{\Hby{\hat{\ell}}}{\Delta_2'}{Q_2}$
				and
				$\horel{\Gamma}{\Delta_1'}{P_2}{\ \Re \ }{\Delta_2'}{Q_2}$; and 

		\item	The symmetric cases of 1 and 2.                
	\end{enumerate}
	The largest such bisimulation
	is called \emph{higher-order bisimilarity} \jpc{and} denoted by $\hwb$.
\end{definition}
 

\noi 
We define characteristic bisimilarity
(cf.~Def.~14~in~\cite{characteristic_bis}):
% is given using characteristic trigger processes. 


\begin{definition}[Characteristic Bisimilarity]\rm
	\label{d:fwb}
	Characteristic bisimilarity, denoted by $\fwb$, is defined \jpc{by} replacing 
	Clause 1) in \defref{d:hbw} with the following clause:\\[1mm]
	Whenever 
	$\horel{\Gamma}{\Delta_1}{P_1}{\hby{\news{\tilde{m_1}} \bactout{n}{\dk{V_1: U}}}}{\Delta_1'}{P_2}$ %with $\Gamma; \es; \Delta \proves V_1 \hastype U$,  
	then there exist 
	$Q_2$, $V_2$, $\Delta'_2$ such that 
	$\horel{\Gamma}{\Delta_2}{Q_1}{\Hby{\news{\tilde{m_2}}\bactout{n}{\dk{V_2: U}}}}{\Delta_2'}{Q_2}$ %with $\Gamma; \es; \Delta' \proves V_2 \hastype U$,  
	and, for fresh $t$, \\[1mm]
	$\begin{array}{lrlll}
	\!\!\Gamma; \Delta''_1  \proves  {\newsp{\tilde{m_1}}{P_2 \Par 
	\ftrigger{t}{V_1}{U_1}}}
	\ \!\!\Re\!\!
	\ \Delta''_2 \proves {\newsp{\tilde{m_2}}{Q_2 \Par \ftrigger{t}{V_2}{U_2}}}
\end{array}
$
\end{definition}

\noi In Theorem~16~in~\cite{characteristic_bis}
we showed that
$\cong$, $\wbc$ and $\fwb$ coincide.


%\begin{definition}[Characteristic Bisimilarity]\rm
%	\label{d:fwb}
%
%	A typed relation $\Re$ is a {\em characteristic bisimulation} if 
%	for all $\horel{\Gamma}{\Delta_1}{P_1}{\ \Re \ }{\Delta_2}{Q_1}$ 
%	if whenever:
%	\begin{enumerate}[1)]
%		\item	$\horel{\Gamma}{\Delta_1}{P_1}{\hby{\news{\tilde{m_1}} \bactout{n}{V_1: U}}}{\Delta_1'}{P_2}$
%				then there exist  $Q_2$, $V_2$, $\Delta'_2$ such that 
%				$\horel{\Gamma}{\Delta_2}{Q_1}{\Hby{\news{\tilde{m_2}}\bactout{n}{V_2: U}}}{\Delta_2'}{Q_2}$
%				and, for fresh $t$,
%				$
%				\begin{array}{lrlll}
%					\Gamma; \Delta''_1 \proves {\newsp{\tilde{m_1}}{P_2 \Par  \ftrigger{t}{V_1}{U_1}}}
%					\ \Re\ \Delta''_2 \proves {\newsp{\tilde{m_2}}{Q_2 \Par \ftrigger{t}{V_2}{U_2}}}
%				\end{array}
%				$
%
%			\item	For all $\horel{\Gamma}{\Delta_1}{P_1}{\hby{\ell}}{\Delta_1'}{P_2}$ such that 
%					$\ell$ is not an output, there exist $Q_2$, $\Delta'_2$ such that 
%					$\horel{\Gamma}{\Delta_2}{Q_1}{\Hby{\hat{\ell}}}{\Delta_2'}{Q_2}$
%					and $\horel{\Gamma}{\Delta_1'}{P_2}{\ \Re \ }{\Delta_2'}{Q_2}$; and 
%
%			\item	The symmetric cases of 1 and 2.
%	\end{enumerate}
%	The largest such bisimulation is called \emph{characteristic bisimilarity} and denoted by $\fwb$.
%\end{definition}


\subsubsection{Up-to techniques}

\noi Processes that do not use shared names are inherently deterministic. 
The following determinacy property will be useful 
for proving our expressiveness results (\secref{sec:positive}).
We require an auxiliary definition.

\begin{definition}[Deterministic Transition]\myrm
\label{def:dettrans}
	Let  $\Gamma; \es; \Delta \proves P \hastype \Proc$ be a balanced \HOp process. 
	Transition $\horel{\Gamma}{\Delta}{P}{\hby{\tau}}{\Delta'}{P'}$ is called:
	\begin{enumerate}[$-$]
		\item	{\em Session transition}
				whenever the untyped transition $P \by{\tau} P'$ 
				is derived using  rule~$\ltsrule{Tau}$ 
				(where $\subj{\ell_1}$ and $\subj{\ell_2}$ in the premise are dual endpoints), 
				possibly followed by uses of
				$\ltsrule{Alpha}$, $\ltsrule{Res}$, $\ltsrule{Rec}$, or $\ltsrule{Par${}_L$}/
				\ltsrule{Par${}_R$}$.
		
		\item	{\em $\beta$-transition}
				whenever the untyped transition $P \by{\tau} P'$
				is derived using rule $\ltsrule{App}$,
				possibly followed by uses of  $\ltsrule{Alpha}$, $\ltsrule{Res}$, $\ltsrule{Rec}$, or $\ltsrule{Par${}_L$}/
				\ltsrule{Par${}_R$}$.
	\end{enumerate}
%
	We write
	$\horel{\Gamma}{\Delta}{P}{\hby{\stau}}{\Delta'}{P'}$
	and 
	$\horel{\Gamma}{\Delta}{P}{\hby{\btau}}{\Delta'}{P'}$
	to denote session and $\beta$-transitions, resp. Also, 
	 $\horel{\Gamma}{\Delta}{P}{\hby{\dtau}}{\Delta'}{P'}$ denotes
	either a session transition or a $\beta$-transition.
\end{definition}
%
%A transition $\horel{\Gamma}{\Delta}{P}{\hby{\tau}}{\Delta'}{P'}$ is said
%{\em deterministic} if it is derived using~$\ltsrule{App}$ or~$\ltsrule{Tau}$ 
%(where $\subj{\ell_1}$ and $\subj{\ell_2}$ in the premise  are dual endpoints), 
%possibly followed by uses of  $\ltsrule{Alpha}$, $\ltsrule{Res}$, $\ltsrule{Rec}$, or $\ltsrule{Par${}_L$}/\ltsrule{Par${}_R$}$.

\begin{lemma}[$\tau$-Inertness]\rm
	\label{lem:tau_inert}
	\begin{enumerate}[1)]
		\item
				Let $\horel{\Gamma}{\Delta}{P}{\hby{\dtau}}{\Delta'}{P'}$ be a deterministic transition,
				with balanced $\Delta$. Then 
				$\Gamma; \Delta \proves P \cong \Delta'\proves P'$ 
				with $\Delta \red^\ast \Delta'$ balanced.

		\item 
				Let $P$ be an $\HOp^{-\mathsf{sh}}$ process. 
				Assume $\Gamma; \emptyset; \Delta \proves P \hastype \Proc$. Then 
				$P \red^\ast P'$ implies $\Gamma; \Delta \proves 
				P \cong \Delta'\proves P'$ with $\Delta \red^\ast \Delta'$. 
	\end{enumerate}
\end{lemma}


\begin{proof}
	The proof is relied on the fact that processes of the
	form $\Gamma; \es; \Delta \proves_s \bout{s}{V} P_1 \Par \binp{k}{x} P_2$
	cannot have any typed transition observables and the fact
	that bisimulation is a congruence.
	See details in Appendix~\ref{app:sub_tau_inert}.
	The proof for Part 2 is straightforward from Part 1.
	\qed
\end{proof}
}

%\noi Precise encodings offer more detailed criteria and used for positive 
%encodability results (\secref{sec:positive}).
%In contrast, minimal encodings contains only 
%some of the criteria of precise encodings:    
%this reduced notion will be used 
%for the negative result in \secref{sec:negative}.
