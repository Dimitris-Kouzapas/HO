\begin{figure}[t]
\[
	\begin{array}{c}
		\trule{Sess}~~\Gamma; \emptyset; \set{u:S} \proves u \hastype S 
		\qquad
		\trule{Sh}~~\Gamma \cat u : U; \emptyset; \emptyset \proves u \hastype U
		\qquad
		\trule{LVar}~~\Gamma; \set{x: \lhot{C}}; \emptyset \proves x \hastype \lhot{C}
		\\[4mm]

		\trule{Prom}~~\tree{
			\Gamma; \emptyset; \emptyset \proves V \hastype 
                         \lhot{C}
		}{
			\Gamma; \emptyset; \emptyset \proves V \hastype 
                         \shot{C}
		} 
		\qquad
		\trule{EProm}~~\tree{
		\Gamma; \Lambda \cat x : \lhot{C}; \Delta \proves P \hastype \Proc
		}{
			\Gamma \cat x:\shot{C}; \Lambda; \Delta \proves P \hastype \Proc
		}
		\\[4mm]

		\trule{Abs}~~\tree{
			\Gamma; \Lambda; \Delta_1 \proves P \hastype \Proc
			\quad
			\Gamma; \es; \Delta_2 \proves x \hastype C
		}{
			\Gamma\backslash x; \Lambda; \Delta_1 \backslash \Delta_2 \proves \abs{{x}}{P} \hastype \lhot{{C}}
		}
		\\[4mm]

		\trule{App}~~\tree{
			\begin{array}{c}
				U = \lhot{C} \lor \shot{C}
				\quad
				\Gamma; \Lambda; \Delta_1 \proves V \hastype U
				\quad
				\Gamma; \es; \Delta_2 \proves u \hastype C
			\end{array}
		}{
			\Gamma; \Lambda; \Delta_1 \cat \Delta_2 \proves \appl{V}{u} \hastype \Proc
		} 
		\\[4mm]

		\trule{Send}~~\tree{
			\Gamma; \Lambda_1; \Delta_1 \proves P \hastype \Proc
			\quad
			\Gamma; \Lambda_2; \Delta_2 \proves V \hastype U
			\quad
			u:S \in \Delta_1 \cat \Delta_2
		}{
			\Gamma; \Lambda_1 \cat \Lambda_2; ((\Delta_1 \cat \Delta_2) \setminus u:S) \cat u:\btout{U} S \proves \bout{u}{V} P \hastype \Proc
		}
		\\[4mm]

		\trule{Rcv}~~\tree{
			\Gamma; \Lambda_1; \Delta_1 \cat u: S \proves P \hastype \Proc
			\quad
			\Gamma; \Lambda_2; \Delta_2 \proves {x} \hastype {U}
		}{
			\Gamma \backslash x; \Lambda_1\cat \Lambda_2; \Delta_1\backslash \Delta_2 \cat u: \btinp{U} S \vdash \binp{u}{{x}} P \hastype \Proc
		}
		\\[4mm]

		\trule{Req}~~\tree{
			\begin{array}{c}
				\Gamma; \es; \es \proves u \hastype U_1
				\quad
				\Gamma; \Lambda; \Delta_1 \proves P \hastype \Proc
				\quad
				\Gamma; \es; \Delta_2 \proves V \hastype U_2
				\\
				(U_1 = \chtype{S} 
                                \land %\Leftrightarrow 
                                U_2 = S)
				\lor
				 (U_1 = \chtype{L} 
                                \land %\Leftrightarrow 
                                %\Leftrightarrow 
                                 U_2 = L)
			\end{array}
		}{
			\Gamma; \Lambda; \Delta_1 \cat \Delta_2 \proves \bout{u}{V} P \hastype \Proc
		}
		\\[4mm]

		\trule{Acc}~~\tree{
			\begin{array}{c}
				\Gamma; \emptyset; \emptyset \proves u \hastype U_1 
				\quad
				\Gamma; \Lambda_1; \Delta_1 \proves P \hastype \Proc
				\quad
				\Gamma; \Lambda_2; \Delta_2 \proves x \hastype U_2\\
				(U_1 = \chtype{S} 
                                \land %\Leftrightarrow 
                                U_2 = S)
				\lor
				 (U_1 = \chtype{L} 
                                \land %\Leftrightarrow 
                                %\Leftrightarrow 
                                 U_2 = L)
	               \end{array}
		}{
			\Gamma\backslash x; \Lambda_1 \backslash \Lambda_2; \Delta_1 \backslash \Delta_2 \proves \binp{u}{x} P \hastype \Proc
		}
		\\[4mm]

		\trule{Bra}~~\tree{
			 \forall i \in I \quad \Gamma; \Lambda; \Delta \cat u:S_i \proves P_i \hastype \Proc
		}{
			\Gamma; \Lambda; \Delta \cat u: \btbra{l_i:S_i}_{i \in I} \proves \bbra{u}{l_i:P_i}_{i \in I}\hastype \Proc
		}
		\qquad
	 	\trule{Sel}~~\tree{
			\Gamma; \Lambda; \Delta \cat u: S_j  \proves P \hastype \Proc \quad j \in I

		}{
			\Gamma; \Lambda; \Delta \cat u:\btsel{l_i:S_i}_{i \in I} \proves \bsel{u}{l_j} P \hastype \Proc
		}
		\\[4mm]

		\trule{ResS}~~\tree{
			\Gamma; \Lambda; \Delta \cat s:S_1 \cat \dual{s}: S_2 \proves P \hastype \Proc \quad S_1 \dualof S_2
		}{
			\Gamma; \Lambda; \Delta \proves \news{s} P \hastype \Proc
		}
		\qquad
		\trule{Res}~~\tree{
			\Gamma\cat a:\chtype{S} ; \Lambda; \Delta \proves P \hastype \Proc
		}{
			\Gamma; \Lambda; \Delta \proves \news{a} P \hastype \Proc
		}
		\\[4mm]
 
		\trule{Par}~~\tree{
			\Gamma; \Lambda_{i}; \Delta_{i} \proves P_{i} \hastype \Proc \quad i=1,2
		}{
			\Gamma; \Lambda_{1} \cat \Lambda_2; \Delta_{1} \cat \Delta_2 \proves P_1 \Par P_2 \hastype \Proc
		}
		\qquad
		\trule{End}~~\tree{
			\Gamma; \Lambda; \Delta  \proves P \hastype T \quad u \not\in \dom{\Gamma, \Lambda,\Delta}
		}{
			\Gamma; \Lambda; \Delta \cat u: \tinact  \proves P \hastype \Proc
		}
		\\[4mm]

	 	\trule{Rec}~~\tree{
			\Gamma \cat \rvar{X}: \Delta; \emptyset; \Delta  \proves P \hastype \Proc
		}{
			\Gamma ; \emptyset; \Delta  \proves \recp{X}{P} \hastype \Proc
		}
		\qquad
		\trule{RVar}~~\Gamma \cat \rvar{X}: \Delta; \emptyset; \Delta  \proves \rvar{X} \hastype \Proc
		\qquad
		\trule{Nil}~~\Gamma; \emptyset; \emptyset \proves \inact \hastype \Proc
	\end{array}
\]
\caption{Typing Rules for $\HOp$.\label{fig:typerulesmy}}
%\Hline
\end{figure}
%\myparagraph{Typing System of \HOp}
The typing system is defined in \figref{fig:typerulesmy}.
%Types for session names/variables $u$ and
%directly derived from the linear part of the typing
%environment, i.e.~type maps $\Delta$ and $\Lambda$.
Rules $\trule{Sess, Sh, LVar}$ are name and variable introduction rules. 
The type $\shot{C}$ %for shared higher order values $V$
is derived using rule $\trule{Prom}$, where we require
a value with linear type to be typed without a linear
environment to be typed as a shared type.
Rule $\trule{EProm}$ allows to freely use a linear
type variable as shared.

Abstraction values are typed with rule $\trule{Abs}$.
%The key type for an abstraction is the type for
%the bound variables of the abstraction, i.e.~for
%bound variable type $C$ the abstraction
%has type $\lhot{C}$.
The dual of abstraction typing is application typing
governed by rule $\trule{App}$, where we expect
the type $C$ of an application name $u$ 
to match the type $\lhot{C}$ or $\shot{C}$
of the application variable $x$.

%A process prefixed with a session send operator $\bout{k}{V} P$
%is typed using rule $\trule{Send}$.
In $\trule{Send}$, 
the type $U$ of a send value $V$ should appear as a prefix
on the session type $\btout{U} S$ of $u$.
$\trule{Rcv}$ is its dual.  
%defined the typing for the 
%reception of values $\binp{u}{V} P$.
%the type $U$ of a receive value should 
%appear as a prefix on the session type $\btinp{U} S$ of $u$.
We use a similar approach with session prefixes
to type interaction between shared names as defined 
in rules $\trule{Req}$ and $\trule{Acc}$,
where the type of the sent/received object 
($S$ and $L$, respectively) should
match the type of the sent/received subject
($\chtype{S}$ and $\chtype{L}$, respectively).
Rules for selection and branching, 
$\trule{Sel}$ and $\trule{Bra}$, are standard. 
%Both
%rules prefix the session type with the selection
%type $\btsel{l_i: S_i}_{i \in I}$ and
%$\btbra{l_i:S_i}_{i \in I}$.

A
shared name creation $a$ creates and restricts
$a$ in environment $\Gamma$ as defined in 
rule \trule{Res}. 
Creation of a session name $s$
creates and restricts two endpoints with dual types 
%and restricts
%them by removing them from the session environment
%$\Delta$ as defined 
in rule \trule{ResS}. 
Rule \trule{Par}, 
combines the environments
$\Lambda$ and $\Delta$ of
the parallel components of a parallel process.
%to create a type for the entire parallel process.
The disjointness of environments $\Lambda$ and $\Delta$
is implied. Rule \trule{End} adds 
the names with type $\tinact$ in $\Delta$.  
The recursion requires that the body process 
matches the type of the recursive
variable as in rule \trule{Rec}.
The recursive variable is typed
directly from the shared environment $\Gamma$ as
in rule \trule{RVar}.
The inactive process $\inact$ is typed with no
linear environments as in \trule{Nil}. 




