% !TEX root = main.tex




Here we extend \HOp in two directions: %to define two more higher-order
%process calculi:
(i)~\HOpp is the extension of \HOp with higher-order applications/abstractions;
(ii)~\PHOp is the extension of \HOp
with polyadicity.
In both cases, we detail the
required modifications in the syntax and types.
The two extensions may be combined into \PHOpp: the polyadic extension of \HOpp.


\subsection{\HOp with Higher-Order Abstractions ($\HOpp$) and 
with Polyadicity (\PHOp)
}
%\label{subsec:hop}
We first introduce \HOpp, the  extension of \HOp with higher-order abstractions and applications.
This is the calculus that we studied in~\cite{characteristic_bis}. The syntax of \HOpp is obtained 
from  \figref{fig:syntax} by extending
$\appl{V}{u}$ to $\appl{V}{W}$, where  $W$ is a higher-order value. 
As for the reduction semantics, we keep the rules in \figref{fig:reduction}, except for 
 $\orule{App}$ which is replaced by 
\[
	\appl{(\abs{x}{P})}{V} \red P \subst{V}{x}
\]
The syntax of types is modified as follows: %changes to include: 
$$
		L \bnfis \shot{U} \bnfbar \lhot{U}
$$
These types can be easily accommodated in the type system in \figref{fig:typerulesmy}, 
we replace $C$ by $U$ in \trule{Abs} and $C$ by $U'$ in \trule{App}. Subject
reduction~(\thmref{t:sr}) holds for \HOpp (cf.~\cite{characteristic_bis})

%\subsection{\HOp with Polyadic Communication: \PHOp}
%
%\noi Embeddings of polyadic name passing into monadic name passing are
%well-studied. % in the literature. 
%Using a linear typing, precise
%encodings (including full abstraction) can be obtained~\cite{Yoshida96}.
The calculus  
$\PHOp$ 
extends $\HOp$ 
with polyadic name passing $\tilde{n}$ and $\abs{\tilde{x}}{Q}$ in the syntax 
of value $V$. 
The operational semantics is kept unchanged, with the expected use of the simultaneous substitution $\subst{\tilde{V}}{\tilde{x}}$.
The type syntax is extended to: 
%
\begin{center}
	\begin{tabular}{c}
	$	L \bnfis \shot{\tilde{C}} \bnfbar \lhot{\tilde{C}}
		\quad\quad
		S \bnfis  \btout{\tilde{U}} S \bnfbar \btinp{\tilde{U}} S \bnfbar \cdots$
	\end{tabular}
\end{center}
%
As in \cite{tlca07,MostrousY15},
the type system for \PHOp 
disallows a shared name that directly carries polyadic
shared names.

By combining \HOpp and \PHOp into a single calculus we obtain \PHOpp:
the extension of \HOp allows \emph{both} higher-order
abstractions/aplications and polyadicity.
