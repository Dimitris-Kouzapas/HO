% !TEX root = main.tex




In this section we extend \HOp to define two more higher-order
process calculi:
(i)~\HOpp is the extension of \HOp with higher-order applications/abstractions
and 
(ii)~\PHOp is the extension of \HOp
with polyadicity.
In both cases, we detail the
required modifications in the syntax and types.
The two extensions allow us to assume the \PHOpp
which is the polyadic extension of \HOpp.


\subsection{\HOp with Higher-Order Abstractions: The  $\HOpp$}
%\label{subsec:hop}
\noi 
The calculus \HOpp 
extends \HOp with higher-order abstractions and applications.
\HOpp is the calculus defined in~\cite{characteristic_bis}.

\myparagraph{Syntax, Operational Semantics and Types.}
\noi First, the syntax of \figref{fig:syntax} extends 
$\appl{V}{u}$ to $\appl{V}{W}$, including higher-order value $W$. 
Rule 
\[
	\appl{(\abs{x}{P})}{V} \red P \subst{V}{x}
\]
replaces rule $\orule{App}$ in \figref{fig:reduction}.
The syntax of types is modified as follows: %changes to include: 
%
\begin{center}
	\begin{tabular}{c}
		$L \bnfis \shot{U} \bnfbar \lhot{U}$
	\end{tabular}
\end{center}
These types can be easily accommodated in the type system in \figref{fig:typerulesmy}, 
we replace $C$ by $U$ in \trule{Abs} and $C$ by $U'$ in \trule{App}. Subject
reduction~(\thmref{t:sr}) holds for \HOpp (cf.~\cite{characteristic_bis})

\subsection{\HOp with Polyadic Communication: \PHOp}

\noi Embeddings of polyadic name passing into monadic name passing are
well-studied. % in the literature. 
Using a linear typing, precise
encodings (including full abstraction) can be obtained~\cite{Yoshida96}.
The syntax of 
$\HOp$ is extended with
polyadic name passing $\tilde{n}$ and $\abs{\tilde{x}}{Q}$ in the syntax 
of value $V$. The type syntax is extended to: 
%
\begin{center}
	\begin{tabular}{c}
	$	L \bnfis \shot{\tilde{C}} \bnfbar \lhot{\tilde{C}}
		\quad\quad
		S \bnfis  \btout{\tilde{U}} S \bnfbar \btinp{\tilde{U}} S \bnfbar \cdots$
	\end{tabular}
\end{center}
%
The type system disallows a shared name that directly carries polyadic
shared names as in \cite{tlca07,MostrousY15}.

The combined syntax, semantics and type syntax for \HOpp and \PHOp
define the \PHOpp which is the calculus that allows higher-order
abstractions and aplications, and polyadicity.
