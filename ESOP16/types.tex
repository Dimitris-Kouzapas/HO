% !TEX root = main.tex
\section{Session Types for \HOp}
\label{sec:types}

We define a session type system for \HOp and state
\emph{type soundness} (\thmref{t:sr}), 
its main property
Our system distills the key features of~\cite{tlca07,MostrousY15} and so it is simpler.


%The system almost identical with the system developed in~\cite{characteristic_bis}
%and we describe it in brief.
%Our system is simpler than that in~\cite{tlca07,MostrousY15}, thus distilling the key
%features of higher-order sessions. %communications. %in a session-typed setting.

%\smallskip 

%\subsection{Types}
%\label{subsec:types}
\jparagraph{Types.}
The syntax of types of \HOp follows. We write $\Proc$ to denote the process type.
\[
	\begin{array}{rcl}
%		\text{(value)} &
		U & \bnfis &	\nonhosyntax{C} \bnfbar L
%		\\[1mm]  % \bnfbar \Proc
		\qquad \qquad
%		\text{(name)} 
		C  \bnfis		S \bnfbar \chtype{S} \bnfbar \chtype{L}
%		\\[1mm]
		\qquad \qquad
%		\text{(abstr)}
		L \bnfis		\shot{C} \bnfbar \lhot{C}
		\\[1mm]

%		\text{(session)} 
		S & \bnfis &	\btout{U} S \bnfbar \btinp{U} S \bnfbar \btsel{l_i:S_i}_{i \in I}
%		\\ 
%						& \bnfbar & 
						\bnfbar \btbra{l_i:S_i}_{i \in I} \bnfbar  \trec{t}{S} \bnfbar \vart{t}  \bnfbar \tinact
	\end{array}
\]
Value type $U$ includes
  first-order types $C$ and  higher-order
types $L$.
%Note that we dissallow type $\chtype{U}$, thus
%in the type discipline shared names cannot carry shared names.
%In name types, $\chtype{U}$ is shared name types 
%which are sent via shared names. 
Types $\shot{C}$ and $\lhot{C}$ denote
{\em shared} and {\em linear} higher-order 
%\jpc{functional}
types, respectively.
Session types, denoted by $S$, follow the standard binary session type syntax~\cite{honda.vasconcelos.kubo:language-primitives}, with
the extension that carried types $U$ may be higher-order.
Shared channel types are denoted $\chtype{S}$ and $\chtype{L}$.
%,
%used to type abstraction values.
%$\lhot{C}$ \cite{tlca07,mostrous_phd} ensures values which contain free 
%session names used once. 
 %We write $S$ to denote %binary 
%session types.  {\em Output type}
%$\btout{U} S$ %is assigned to a name that 
%first sends a value of
%type $U$ and then follows the type described by $S$.  Dually,
%$\btinp{U} S$ denotes an {\em input type}. The {\em branching type}
%$\btbra{l_i:S_i}_{i \in I}$ and the {\em selection type}
%$\btsel{l_i:S_i}_{i \in I}$ define the labelled choice. 
%We assume the {\em recursive type} $\trec{t}{S}$ is guarded,
%i.e.,  $\trec{t}{\vart{t}}$ is not allowed. 
%%We stress that carried type $U$ in $\btout{U} S$ and
%%$\btinp{U} S$ can contain free type variables, which is crucial
%%to encode $\HOp$ into $\HO$.
%Type $\tinact$ is the termination type. 
Types of \HO exclude $\nonhosyntax{C}$ from 
value types of \HOp; the types of \sessp exclude $L$. 
From each $\CAL \in \{\HOp, \HO, \pi \}$, $\CAL^{-\mathsf{sh}}$ 
excludes shared name types ($\chtype{S}$ and $\chtype{L}$), 
from name type $C$.

We use the co-inductive definition of \emph{duality} of \cite{TGC14}.
We write $S_1 \dualof S_2$ if $S_1$ is the dual of $S_2$.   
Intuitively, the duality of types is obtained by 
dualising $!$ by $?$, $?$ by $!$, $\oplus$ by $\&$ and $\&$ by $\oplus$,  
incorporating the fixed point construction 
(see \defref{def:dual} in the Appendix). 

%\smallskip 

%\subsection{Typing System of \HOp}
%\label{subsec:typing}
\jparagraph{Typing Environments and Judgements}
We consider three kinds of environments, denoted $\Gamma$, $\Lambda$, and $\Delta$:
\[
	\begin{array}{l}
		\Lambda \bnfis  \emptyset \bnfbar \Lambda \cat \AT{x}{\lhot{C}}
		\qquad\qquad \Delta  \ \bnfis  \ \emptyset \bnfbar \Delta \cat \AT{u}{S}
		\\
		\Gamma  \bnfis  \emptyset \bnfbar \Gamma \cat \varp{x}: \shot{C} \bnfbar \Gamma \cat u: \chtype{S} \bnfbar \Gamma \cat u: \chtype{L} 
		\bnfbar \Gamma \cat \rvar{X}: \Delta
	\end{array}
\]
%Environment 
$\Gamma$ maps variables and shared names to value types, and recursive 
variables to session environments;  
it admits weakening, contraction, and exchange principles.
$\Lambda$ is a mapping from variables to 
%the
linear
%functional 
higher-order
types; and $\Delta$ is a mapping from 
session names to session types. 
Both $\Lambda$ and $\Delta$
%behave linearly: they 
are
only subject to exchange.  
We require that the domains of $\Gamma,
\Lambda$ and $\Delta$ are pairwise distinct. 
$\Delta_1\cdot \Delta_2$ denotes the disjoint union of $\Delta_1$ and $\Delta_2$.  
We are interested in \emph{balanced} session environments: 
%that contain dual endpoints typed with dual types.
%The following definition ensures two session endpoints 
%are dual each other. 

%\smallskip

\begin{definition}[Balanced]\label{d:wtenv}%\rm
	%Let $\Delta$ be a session environment.
	We say that a session environment $\Delta$ is {\em balanced} if whenever
	$s: S_1, \dual{s}: S_2 \in \Delta$ then $S_1 \dualof S_2$.
\end{definition}

Given the above intuitions for environments, 
the typing judgements for values $V$ and processes $P$ are self-explanatory:
%
\begin{center}
	\begin{tabular}{c}
		$\Gamma; \Lambda; \Delta \proves V \hastype U \qquad \qquad \qquad \qquad \Gamma; \Lambda; \Delta \proves P \hastype \Proc$
	\end{tabular}
\end{center}
%%
%\noi The first judgement states that under environments $\Gamma; \Lambda; \Delta$ value $V$
%has type $U$, whereas the second judgement states that under
%environments $\Gamma; \Lambda; \Delta$ process $P$ has the process type~$\Proc$. %
 
%\smallskip

% !TEX root = ../journal16kpy.tex
\begin{figure}[t]
\[
	\begin{array}{c}
		\inferrule[(Prom)]{
			\Gamma; \emptyset; \emptyset \proves V \hastype 
                         \lhot{C}
		}{
			\Gamma; \emptyset; \emptyset \proves V \hastype 
                         \shot{C}
		} 
		\quad
		\inferrule[(EProm)]{
		\Gamma; \Lambda \cat x : \lhot{C}; \Delta \proves P \hastype \Proc
		}{
			\Gamma \cat x:\shot{C}; \Lambda; \Delta \proves P \hastype \Proc
		}
		\quad
		\inferrule[(Abs)]{
			\Gamma; \Lambda; \Delta_1 \proves P \hastype \Proc
			\quad
			\Gamma; \es; \Delta_2 \proves x \hastype C
		}{
			\Gamma\backslash x; \Lambda; \Delta_1 \backslash \Delta_2 \proves \abs{{x}}{P} \hastype \lhot{{C}}
		}
		\\[1mm] \\
		\inferrule[(App)]{
			\begin{array}{c}
				U = \lhot{C} \lor \shot{C}
				\\
				\Gamma; \Lambda; \Delta_1 \proves V \hastype U \quad
				\Gamma; \es; \Delta_2 \proves u \hastype C
			\end{array}
		}{
			\Gamma; \Lambda; \Delta_1 \cat \Delta_2 \proves \appl{V}{u} \hastype \Proc
		} 
		~~
		\inferrule[(Send)]{
					\begin{array}{c}
					u:S \in \Delta_1 \cat \Delta_2 \\
			\Gamma; \Lambda_1; \Delta_1 \proves P \hastype \Proc
			\quad
			\Gamma; \Lambda_2; \Delta_2 \proves V \hastype U
			\end{array}
		}{
			\Gamma; \Lambda_1 \cat \Lambda_2; ((\Delta_1 \cat \Delta_2) \setminus u:S) \cat u:\btout{U} S \proves \bout{u}{V} P \hastype \Proc
		}
		\\[2mm] \\
		\inferrule*[left=(Rcv)]{
		\begin{array}{c}
			\Gamma; \Lambda_1; \Delta_1 \cat u: S \proves P \hastype \Proc
			\quad
			\Gamma; \Lambda_2; \Delta_2 \proves {x} \hastype {U}
			\end{array}
		}{
			\Gamma \backslash x; \Lambda_1\cat \Lambda_2; \Delta_1\backslash \Delta_2 \cat u: \btinp{U} S \vdash \binp{u}{{x}} P \hastype \Proc
		}
		\\[2mm] \\
		\inferrule[(Req)]{
			\begin{array}{c}
				\Gamma; \es; \es \proves u \hastype U_1
				\quad
				\Gamma; \Lambda; \Delta_1 \proves P \hastype \Proc
				\\
				\Gamma; \es; \Delta_2 \proves V \hastype U_2
				\\
				(U_1 = \chtype{S} 
                                \land %\Leftrightarrow 
                                U_2 = S)
				\lor
				 (U_1 = \chtype{L} 
                                \land %\Leftrightarrow 
                                %\Leftrightarrow 
                                 U_2 = L)
			\end{array}
		}{
			\Gamma; \Lambda; \Delta_1 \cat \Delta_2 \proves \bout{u}{V} P \hastype \Proc
		}
		~~
		\inferrule[(Acc)]{
			\begin{array}{c}
				\Gamma; \emptyset; \emptyset \proves u \hastype U_1 
				\quad
				\Gamma; \Lambda_1; \Delta_1 \proves P \hastype \Proc
				\\
				\Gamma; \Lambda_2; \Delta_2 \proves x \hastype U_2\\
				(U_1 = \chtype{S} 
                                \land %\Leftrightarrow 
                                U_2 = S)
				\lor
				 (U_1 = \chtype{L} 
                                \land %\Leftrightarrow 
                                %\Leftrightarrow 
                                 U_2 = L)
	               \end{array}
		}{
			\Gamma\backslash x; \Lambda_1 \backslash \Lambda_2; \Delta_1 \backslash \Delta_2 \proves \binp{u}{x} P \hastype \Proc
		}	
		\end{array}
\]
%\vspace{-3mm}
\caption{Selected Typing Rules for $\HOp$.
See \appref{app:types} for a full account.
\label{fig:typerulesmys}}
%\Hline
%\vspace{-1mm}
\end{figure}
%\myparagraph{Typing System of \HOp}




\jparagraph{Typing Rules} 
Selected rules for the typing system are given in \figref{fig:typerulesmys}; 
see \appref{app:typrules} for a full account.
%Types for session names/variables $u$ and
%directly derived from the linear part of the typing
%environment, i.e.~type maps $\Delta$ and $\Lambda$.
%Rules $\trule{Sess, Sh, LVar}$ are name and variable introduction rules. 
The shared type $\shot{C}$ %for shared higher order values $V$
is derived using rule $\trule{Prom}$ only  
if the value has a linear type with an empty linear
environment.
Rule~$\trule{EProm}$ allows us to freely use a linear
type variable as shared.
%
Abstraction values are typed with rule~$\trule{Abs}$.
%The key type for an abstraction is the type for
%the bound variables of the abstraction, i.e.~for
%bound variable type $C$ the abstraction
%has type $\lhot{C}$.
Application typing
is governed by rule $\trule{App}$: we expect
the type $C$ of an application name $u$ 
to match the type $\lhot{C}$ or $\shot{C}$
of the application variable $x$.
%
%A process prefixed with a session send operator $\bout{k}{V} P$
%is typed using rule $\trule{Send}$.
In rule $\trule{Send}$, 
the type $U$ of a send value $V$ should appear as a prefix
on the session type $\btout{U} S$ of $u$.
Rule $\trule{Rcv}$ is its dual.  
%defined the typing for the 
%reception of values $\binp{u}{V} P$.
%the type $U$ of a receive value should 
%appear as a prefix on the session type $\btinp{U} S$ of $u$.
We use a similar approach with session prefixes
to type interaction between shared names as defined 
in rules $\trule{Req}$ and $\trule{Acc}$,
where the type of the sent/received object 
($S$ and $L$, respectively) should
match the type of the sent/received subject
($\chtype{S}$ and $\chtype{L}$, respectively).


\begin{definition}[Reduction of Session Environments]%\rm
	\label{def:ses_red}
	We define the relation $\red$ on session environments as:
	\begin{eqnarray*}
			\Delta \cat s: \btout{U} S_1 \cat \dual{s}: \btinp{U} S_2 & \red &
			\Delta \cat s: S_1 \cat \dual{s}: S_2\\%[1mm]
			\Delta \cat s: \btsel{l_i: S_i}_{i \in I} \cat \dual{s}: \btbra{l_i: S_i'}_{i \in I} &\red& 
			 \Delta \cat s: S_k \cat \dual{s}: S_k' \ (k \in I)
		\end{eqnarray*}
	%\end{center}
%\begin{tabular}{rcl}
%	\setlength{\tabcolsep}{0pt}
%	$\Delta \cat s: \btout{U} S_1 \cat \dual{s}: \btinp{U} S_2$ & $\red$ & 
%	$\Delta \cat s: S_1 \cat \dual{s}: S_2$\\[1mm]
%	$\Delta \cat s: \btsel{l_i: S_i}_{i \in I} \cat \dual{s}: \btbra{l_i: S_i'}_{i \in I}$ & $\red$ & $\Delta \cat s: S_k \cat \dual{s}: S_k' \ (k \in I)$
%\end{tabular}
%\[
%\begin{array}{rcl}
%\Delta \cat s: \btout{U} S_1 \cat \dual{s}: \btinp{U} S_2 & \red & 
%\Delta \cat s: S_1 \cat \dual{s}: S_2\\[1mm]
%\Delta \cat s: \btsel{l_i: S_i}_{i \in I} \cat \dual{s}: \btbra{l_i: S_i'}_{i \in I} & \red & \Delta \cat s: S_k \cat \dual{s}: S_k' \ (k \in I)
%\end{array}
%\]
\end{definition}

%\smallskip

%The following result %Theorem 7.3 in M\&Y
\noi We state the type soundness result for \HOp; it implies 
the type soundness of the sub-calculi \HO, \sessp, and $\CAL^{-\mathsf{sh}}$. 

%\smallskip

\begin{theorem}[Type Soundness]\label{t:sr}\rm
%	\begin{enumerate}[1.]
%		\item	(Subject Congruence) Suppose $\Gamma; \es; \Delta \proves P \hastype \Proc$.
%			Then $P \scong P'$ implies $\Gamma; \es; \Delta \proves P' \hastype \Proc$.
%
%		\item
%			(Subject Reduction)
			Suppose $\Gamma; \es; \Delta \proves P \hastype \Proc$
			with
			$\Delta$ balanced. 
			Then $P \red P'$ implies $\Gamma; \es; \Delta'  \proves P' \hastype \Proc$
			and $\Delta = \Delta'$ or $\Delta \red \Delta'$
			with $\Delta'$ balanced. 
%	\end{enumerate}
\end{theorem}
