% !TEX root = main.tex
\noi %Here we define 
This section defines a session typing system for \HOp and states its main properties. 
Our system distills the key features of~\cite{tlca07,MostrousY15},
thus it is simpler.
%Our system is simpler than that in~\cite{tlca07,MostrousY15}, thus distilling the key
%features of higher-order sessions. %communications. %in a session-typed setting.

\smallskip 

%\subsection{Types}
%\label{subsec:types}
\myparagraph{Types.}
The syntax of types of \HOp is given below: 
\[
	\begin{array}{lrl}
		\text{(value)}	& U \bnfis &	\nonhosyntax{C} \bnfbar L\\[1mm]  % \bnfbar \Proc
		\text{(name)}   & C  \bnfis &	S \bnfbar \chtype{S} \bnfbar \chtype{L}
		\\[1mm]

		\text{(abstr)}	& L \bnfis &	\shot{C} \bnfbar \lhot{C}
		\\[1mm]

		\text{(session)} &  S \bnfis &	\btout{U} S \bnfbar \btinp{U} S \bnfbar \btsel{l_i:S_i}_{i \in I}
		\\ 
						& \bnfbar & \btbra{l_i:S_i}_{i \in I}
						\bnfbar  \trec{t}{S} \bnfbar \vart{t}  \bnfbar \tinact
	\end{array}
\]
\dk{Value type $U$ includes
the first-order types $C$ and the higher-order
types $L$. Session types are denoted with $S$ and
shared types with $\chtype{S}$ and $\chtype{L}$.
%Note that we dissallow type $\chtype{U}$, thus
%in the type discipline shared names cannot carry shared names.
}
%In name types, $\chtype{U}$ is shared name types 
%which are sent via shared names. 
Types $\shot{C}$ and $\lhot{C}$ denote
{\em shared} and {\em linear} higher-order 
%\jpc{functional}
types, respectively.
%,
%used to type abstraction values.
%$\lhot{C}$ \cite{tlca07,mostrous_phd} ensures values which contain free 
%session names used once. 
 
We write $S$ to denote %binary 
session types.  {\em Output type}
$\btout{U} S$ %is assigned to a name that 
first sends a value of
type $U$ and then follows the type described by $S$.  Dually,
$\btinp{U} S$ denotes an {\em input type}. The {\em branching type}
$\btbra{l_i:S_i}_{i \in I}$ and the {\em selection type}
$\btsel{l_i:S_i}_{i \in I}$ define the labelled choice. 
We assume the {\em recursive type} $\trec{t}{S}$ is guarded,
i.e.,  $\trec{t}{\vart{t}}$ is not allowed. 
%We stress that carried type $U$ in $\btout{U} S$ and
%$\btinp{U} S$ can contain free type variables, which is crucial
%to encode $\HOp$ into $\HO$.
Type $\tinact$ is the termination type. 

Types of \HO exclude $\nonhosyntax{C}$ from 
value types of \HOp; the types of \sessp exclude $L$. 
From each $\CAL \in \{\HOp, \HO, \pi \}$, $\CAL^{-\mathsf{sh}}$ 
excludes shared name types ($\chtype{S}$ and $\chtype{L}$), 
from name type $C$.

Following \cite{TGC14}, we co-inductively define the notion of duality.
The definition of duality is based on type equivalence.

\begin{definition}[Type Equivalence]
\label{def:iso}
Let $\mathsf{ST}$ a set of closed session types. 
Two types $S$ and $S'$ are said to be {\em isomorphic} if a pair $(S,S')$ is 
in the largest fixed point of the monotone function
$F:\mathcal{P}(\mathsf{ST}\times \mathsf{ST}) \to 
\mathcal{P}(\mathsf{ST}\times \mathsf{ST})$ defined by:

\hspace{-0.5cm}\begin{tabular}{rcl}
$F(\Re)$ &$\!\!=\!\!$&	$\set{(\tinact, \tinact)}$\\
         &$\!\!\cup\!\!$&	$\set{(\btout{U_1} S_1, \btout{U_2} S_2)
\bnfbar (S_1, S_2),(U_1, U_2)\in \Re}$\\ 
       &$\!\!\cup\!\!$&	$\set{(\btinp{U_1} S_1, \btinp{U_2} S_2)
\bnfbar(S_1, S_2),(U_1, U_2)\in \Re}$\\ 
	&$\!\!\cup\!\!$&	$\set{(\btbra{l_i: S_i}_{i \in I} \,,\, \btbra{l_i: S_i'}_{i \in I}) \bnfbar \forall i\in I. (S_i, S_i')\in \Re}$\\
	&$\!\!\cup\!\!$&	$\set{(\btsel{l_i: S_i}_{i \in I}\,,\, \btsel{l_i: S_i'}_{i \in I}) \bnfbar \forall i\in I. (S_i, S_i')\in \Re}$\\
	&$\!\!\cup\!\!$&	$\set{(\trec{t}{S}, S')
\bnfbar (S\subst{\trec{t}{S}}{\vart{t}},S')\in \Re}$\\
	&$\!\!\cup\!\!$&	$\set{(S,\trec{t}{S'})
\bnfbar (S,S'\subst{\trec{t}{S'}}{\vart{t}})\in \Re}$
\end{tabular}
	
\noindent
Standard arguments ensure that $F$ is monotone, thus the greatest fixed point
of $F$ exists. We write $S_1 \sim S_2$ if  $(S_1,S_2)\in \Re$. 
\end{definition}

\smallskip 

We now define duality:

\begin{definition}[Duality]
	\label{def:dual}
	Let $\mathsf{ST}$ a set of closed session types. 
	Two types $S$ and $S'$ are said to be {\em dual} if a pair $(S,S')$ is 
	in the largest fixed point of the monotone function
	$F:\mathcal{P}(\mathsf{ST}\times \mathsf{ST}) \to 
	\mathcal{P}(\mathsf{ST}\times \mathsf{ST})$ defined by:
	\\[1mm]
%
	\begin{tabular}{rcl}
		$F(\Re)$	&$\!\!=\!\!$&	$\set{(\tinact, \tinact)}$
		\\
				&$\!\!\cup\!\!$&	$\set{(\btout{U_1} S_1, \btinp{U_2} S_2)
							\bnfbar(S_1, S_2)\in \Re, \  U_1 \sim U_2 }$
		\\ 
				&$\!\!\cup\!\!$&	$\set{(\btinp{U_1} S_1, \btout{U_2} S_2)
							\bnfbar(S_1, S_2)\in \Re, \ U_1 \sim U_2}$
		\\ 
				&$\!\!\cup\!\!$&	$\set{(\btsel{l_i: S_i}_{i \in I} \,,\, \btbra{l_i: S_i'}_{i \in I}) \bnfbar \forall i\in I. (S_i, S_i')\in \Re}$
		\\
				&$\!\!\cup\!\!$&	$\set{(\btbra{l_i: S_i}_{i \in I}\,,\, \btsel{l_i: S_i'}_{i \in I}) \bnfbar \forall i\in I. (S_i, S_i')\in \Re}$
		\\
				&$\!\!\cup\!\!$&	$\set{(\trec{t}{S}, S') \bnfbar (S\subst{\trec{t}{S}}{\vart{t}},S')\in \Re}$
		\\
				&$\!\!\cup\!\!$&	$\set{(S,\trec{t}{S'}) \bnfbar (S,S'\subst{\trec{t}{S'}}{\vart{t}})\in \Re}$
		\\[1mm]
	\end{tabular}
	\noindent
	%where $U_1 \sim U_2$ means $U_1$ is type equivalent to $U_2$ \cite{yoshida.vasconcelos:language-primitives}.
	Standard arguments ensure that $F$ is monotone, thus the greatest fixed point
	of $F$ exists. We write $S_1 \dualof S_2$ if  $(S_1,S_2)\in \Re$. 
\end{definition}

%we write $S_1 \dualof S_2$ if 
%$S_1$ is the dual of $S_2$.   
%Intuitively, 
%the duality of types is obtained by 
%dualising $!$ by $?$, $?$ by $!$, $\oplus$ by $\&$ and $\&$ by $\oplus$,  
%incorporating the fixed point construction 
%(see \defref{def:dual} in the Appendix). 

\smallskip 

%\subsection{Typing System of \HOp}
%\label{subsec:typing}
\myparagraph{Typing Judgements of \HOp.}
\noi We define typing judgements for values $V$ and processes $P$:
%
\begin{center}
	\begin{tabular}{c}
		$\Gamma; \Lambda; \Delta \proves V \hastype U \qquad \qquad \qquad \Gamma; \Lambda; \Delta \proves P \hastype \Proc$
	\end{tabular}
\end{center}
%
\noi The first judgement states that under environments $\Gamma; \Lambda; \Delta$ value $V$
has type $U$, whereas the second judgement states that under
environments $\Gamma; \Lambda; \Delta$ process $P$ has the process type~$\Proc$. The environments are defined below:
%
\[
	\begin{array}{l}
		\Lambda \bnfis  \emptyset \bnfbar \Lambda \cat \AT{x}{\lhot{C}}
		\quad\quad \Delta  \ \bnfis  \ \emptyset \bnfbar \Delta \cat \AT{u}{S}
		\\
		\Gamma  \bnfis  \emptyset \bnfbar \Gamma \cat \varp{x}: \shot{C} \bnfbar \Gamma \cat u: \chtype{S} \bnfbar \Gamma \cat u: \chtype{L} 
		\bnfbar \Gamma \cat \rvar{X}: \Delta
	\end{array}
\]
%
\noi 
$\Gamma$ maps variables and shared names to value types, and recursive 
variables to session environments;  
it admits weakening, contraction, and exchange principles.
$\Lambda$ is a mapping from variables to 
%the
linear
%functional 
higher-order
types; and $\Delta$ is a mapping from 
session names to session types. 
Both $\Lambda$ and $\Delta$
%behave linearly: they 
are
only subject to exchange.  
We require that the domains of $\Gamma,
\Lambda$ and $\Delta$ are pairwise distinct. 
$\Delta_1\cdot \Delta_2$ means 
a disjoint union of $\Delta_1$ and $\Delta_2$.  
We are interested in \emph{balanced} session environments: 
%that contain dual endpoints typed with dual types.
%The following definition ensures two session endpoints 
%are dual each other. 

\smallskip

\begin{definition}[Balanced]\label{d:wtenv}%\rm
	%Let $\Delta$ be a session environment.
	We say that a session environment $\Delta$ is {\em balanced} if whenever
	$s: S_1, \dual{s}: S_2 \in \Delta$ then $S_1 \dualof S_2$.
\end{definition}

\smallskip

\myparagraph{The Typing System of \HOp.}
We are ready to present the typing system for the \HOp,
which is similar to~\cite{tlca07,MostrousY15}.

% !TEX root = ../journal16kpy.tex


\begin{figure}[t]
\[
	\begin{array}{c}
		\trule{Sess}~~\Gamma; \emptyset; \set{u:S} \proves u \hastype S 
		\qquad
		\trule{Sh}~~\Gamma \cat u : U; \emptyset; \emptyset \proves u \hastype U
		\qquad
		\trule{LVar}~~\Gamma; \set{x: \lhot{C}}; \emptyset \proves x \hastype \lhot{C}
		\\[4mm]

		\trule{Prom}~~\tree{
			\Gamma; \emptyset; \emptyset \proves V \hastype 
                         \lhot{C}
		}{
			\Gamma; \emptyset; \emptyset \proves V \hastype 
                         \shot{C}
		} 
		\qquad
		\trule{EProm}~~\tree{
		\Gamma; \Lambda \cat x : \lhot{C}; \Delta \proves P \hastype \Proc
		}{
			\Gamma \cat x:\shot{C}; \Lambda; \Delta \proves P \hastype \Proc
		}
		\\[4mm]

		\trule{Abs}~~\tree{
			\Gamma; \Lambda; \Delta_1 \proves P \hastype \Proc
			\quad
			\Gamma; \es; \Delta_2 \proves x \hastype C
		}{
			\Gamma\backslash x; \Lambda; \Delta_1 \backslash \Delta_2 \proves \abs{{x}}{P} \hastype \lhot{{C}}
		}
		\\[4mm]

		\trule{App}~~\tree{
			\begin{array}{c}
				U = \lhot{C} \lor \shot{C}
				\quad
				\Gamma; \Lambda; \Delta_1 \proves V \hastype U
				\quad
				\Gamma; \es; \Delta_2 \proves u \hastype C
			\end{array}
		}{
			\Gamma; \Lambda; \Delta_1 \cat \Delta_2 \proves \appl{V}{u} \hastype \Proc
		} 
		\\[4mm]

		\trule{Send}~~\tree{
			\Gamma; \Lambda_1; \Delta_1 \proves P \hastype \Proc
			\quad
			\Gamma; \Lambda_2; \Delta_2 \proves V \hastype U
			\quad
			u:S \in \Delta_1 \cat \Delta_2
		}{
			\Gamma; \Lambda_1 \cat \Lambda_2; ((\Delta_1 \cat \Delta_2) \setminus u:S) \cat u:\btout{U} S \proves \bout{u}{V} P \hastype \Proc
		}
		\\[4mm]

		\trule{Rcv}~~\tree{
			\Gamma; \Lambda_1; \Delta_1 \cat u: S \proves P \hastype \Proc
			\quad
			\Gamma; \Lambda_2; \Delta_2 \proves {x} \hastype {U}
		}{
			\Gamma \backslash x; \Lambda_1\cat \Lambda_2; \Delta_1\backslash \Delta_2 \cat u: \btinp{U} S \vdash \binp{u}{{x}} P \hastype \Proc
		}
		\\[4mm]

		\trule{Req}~~\tree{
			\begin{array}{c}
				\Gamma; \es; \es \proves u \hastype U_1
				\quad
				\Gamma; \Lambda; \Delta_1 \proves P \hastype \Proc
				\quad
				\Gamma; \es; \Delta_2 \proves V \hastype U_2
				\\
				(U_1 = \chtype{S} 
                                \land %\Leftrightarrow 
                                U_2 = S)
				\lor
				 (U_1 = \chtype{L} 
                                \land %\Leftrightarrow 
                                %\Leftrightarrow 
                                 U_2 = L)
			\end{array}
		}{
			\Gamma; \Lambda; \Delta_1 \cat \Delta_2 \proves \bout{u}{V} P \hastype \Proc
		}
		\\[4mm]

		\trule{Acc}~~\tree{
			\begin{array}{c}
				\Gamma; \emptyset; \emptyset \proves u \hastype U_1 
				\quad
				\Gamma; \Lambda_1; \Delta_1 \proves P \hastype \Proc
				\quad
				\Gamma; \Lambda_2; \Delta_2 \proves x \hastype U_2\\
				(U_1 = \chtype{S} 
                                \land %\Leftrightarrow 
                                U_2 = S)
				\lor
				 (U_1 = \chtype{L} 
                                \land %\Leftrightarrow 
                                %\Leftrightarrow 
                                 U_2 = L)
	               \end{array}
		}{
			\Gamma\backslash x; \Lambda_1 \backslash \Lambda_2; \Delta_1 \backslash \Delta_2 \proves \binp{u}{x} P \hastype \Proc
		}
		\\[4mm]

		\trule{Bra}~~\tree{
			 \forall i \in I \quad \Gamma; \Lambda; \Delta \cat u:S_i \proves P_i \hastype \Proc
		}{
			\Gamma; \Lambda; \Delta \cat u: \btbra{l_i:S_i}_{i \in I} \proves \bbra{u}{l_i:P_i}_{i \in I}\hastype \Proc
		}
		\qquad
	 	\trule{Sel}~~\tree{
			\Gamma; \Lambda; \Delta \cat u: S_j  \proves P \hastype \Proc \quad j \in I

		}{
			\Gamma; \Lambda; \Delta \cat u:\btsel{l_i:S_i}_{i \in I} \proves \bsel{u}{l_j} P \hastype \Proc
		}
		\\[4mm]

		\trule{ResS}~~\tree{
			\Gamma; \Lambda; \Delta \cat s:S_1 \cat \dual{s}: S_2 \proves P \hastype \Proc \quad S_1 \dualof S_2
		}{
			\Gamma; \Lambda; \Delta \proves \news{s} P \hastype \Proc
		}
		\qquad
		\trule{Res}~~\tree{
			\Gamma\cat a:\chtype{S} ; \Lambda; \Delta \proves P \hastype \Proc
		}{
			\Gamma; \Lambda; \Delta \proves \news{a} P \hastype \Proc
		}
		\\[4mm]
 
		\trule{Par}~~\tree{
			\Gamma; \Lambda_{i}; \Delta_{i} \proves P_{i} \hastype \Proc \quad i=1,2
		}{
			\Gamma; \Lambda_{1} \cat \Lambda_2; \Delta_{1} \cat \Delta_2 \proves P_1 \Par P_2 \hastype \Proc
		}
		\qquad
		\trule{End}~~\tree{
			\Gamma; \Lambda; \Delta  \proves P \hastype T \quad u \not\in \dom{\Gamma, \Lambda,\Delta}
		}{
			\Gamma; \Lambda; \Delta \cat u: \tinact  \proves P \hastype \Proc
		}
		\\[4mm]

	 	\trule{Rec}~~\tree{
			\Gamma \cat \rvar{X}: \Delta; \emptyset; \Delta  \proves P \hastype \Proc
		}{
			\Gamma ; \emptyset; \Delta  \proves \recp{X}{P} \hastype \Proc
		}
		\qquad
		\trule{RVar}~~\Gamma \cat \rvar{X}: \Delta; \emptyset; \Delta  \proves \rvar{X} \hastype \Proc
		\qquad
		\trule{Nil}~~\Gamma; \emptyset; \emptyset \proves \inact \hastype \Proc
	\end{array}
\]
\caption{Complete Typing Rules for $\HOp$.\label{fig:typerulesmy}}
%\Hline
\end{figure}
%\myparagraph{Typing System of \HOp}





%Since the typing system is similar to~\cite{tlca07,MostrousY15}, 
%we fully describe it  in \appref{app:types}.  The %rest of the 
%paper can
%be read without knowing the details of the typing system. 
%\jpc{Type soundness relies on the following auxiliary notion}.
%%We list the key properties.

\smallskip

\begin{definition}[Reduction of Session Environment]%\rm
\label{def:ses_red}
We define the relation $\red$ on session environments as:
\\[-2mm]
%
\begin{center}
\begin{tabular}{l}
	$\Delta \cat s: \btout{U} S_1 \cat \dual{s}: \btinp{U} S_2 \red
	\Delta \cat s: S_1 \cat \dual{s}: S_2$\\[1mm]
	$\Delta \cat s: \btsel{l_i: S_i}_{i \in I} \cat \dual{s}: \btbra{l_i: S_i'}_{i \in I} \red \Delta \cat s: S_k \cat \dual{s}: S_k' \ (k \in I)$
\end{tabular}
\end{center}
%\begin{tabular}{rcl}
%	\setlength{\tabcolsep}{0pt}
%	$\Delta \cat s: \btout{U} S_1 \cat \dual{s}: \btinp{U} S_2$ & $\red$ & 
%	$\Delta \cat s: S_1 \cat \dual{s}: S_2$\\[1mm]
%	$\Delta \cat s: \btsel{l_i: S_i}_{i \in I} \cat \dual{s}: \btbra{l_i: S_i'}_{i \in I}$ & $\red$ & $\Delta \cat s: S_k \cat \dual{s}: S_k' \ (k \in I)$
%\end{tabular}
%\[
%\begin{array}{rcl}
%\Delta \cat s: \btout{U} S_1 \cat \dual{s}: \btinp{U} S_2 & \red & 
%\Delta \cat s: S_1 \cat \dual{s}: S_2\\[1mm]
%\Delta \cat s: \btsel{l_i: S_i}_{i \in I} \cat \dual{s}: \btbra{l_i: S_i'}_{i \in I} & \red & \Delta \cat s: S_k \cat \dual{s}: S_k' \ (k \in I)
%\end{array}
%\]
\end{definition}

\smallskip

%The following result %Theorem 7.3 in M\&Y
\noi We state the type soundness results of \HOp; it implies 
the type soundness of the sub-calculi \HO, \sessp, and $\CAL^{-\mathsf{sh}}$. 

\smallskip

\begin{theorem}[Type Soundness]\label{t:sr}%\rm
%	\begin{enumerate}[1.]
%		\item	(Subject Congruence) Suppose $\Gamma; \es; \Delta \proves P \hastype \Proc$.
%			Then $P \scong P'$ implies $\Gamma; \es; \Delta \proves P' \hastype \Proc$.
%
%		\item
%			(Subject Reduction)
			Suppose $\Gamma; \es; \Delta \proves P \hastype \Proc$
			with
			$\Delta$ balanced. 
			Then $P \red P'$ implies $\Gamma; \es; \Delta'  \proves P' \hastype \Proc$
			and $\Delta = \Delta'$ or $\Delta \red \Delta'$
			with $\Delta'$ balanced. 
%	\end{enumerate}
\end{theorem}
