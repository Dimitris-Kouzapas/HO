% !TEX root = main.tex
\noi This section studies %two extensions of \HOp: 
(i)~the encoding of \HOpp into \HOp
%with higher-order applications/abstractions 
%and 
(ii)~the encoding of \PHOp
,i.e.~ polyadic \HOp, into \HOp.
We also show that using encoding composition we can encode
\PHOpp into \HO and \sessp.
%with polyadicity.
%In both cases, we detail required modifications in the syntax
%and types, and
%describe further encodability results.
%These extensions are denoted \HOpp and \PHOp, respectively. 
 
\subsection{Encoding  $\HOpp$ into $\HOp$}
\label{subsec:hop}
\noi 
The calculus \HOpp 
extends \HOp with higher-order abstractions and applications.

%\myparagraph{Syntax, Operational Semantics and Types.}
%\noi First, the syntax of \figref{fig:syntax} extends 
%$\appl{V}{u}$ to 
% $\appl{V}{W}$, including higher-order value $W$. 
%The rule $\appl{(\abs{x}{P})}{V} \red P \subst{V}{x}$
%replaces
%rule $\orule{App}$ in \figref{fig:reduction}.
%The syntax of types is modified as follows: %changes to include: 
%\begin{center}
%\begin{tabular}{c}
%$L \bnfis \shot{U} \bnfbar \lhot{U}$
%\end{tabular}
%\end{center}
%These types can be easily accommodated in the type system:
% in \figref{fig:typerulesmy}, 
%we replace $C$ by $U$ in \trule{Abs} and $C$ by $U'$ in \trule{App}.
%\smallskip 
%
%\myparagraph{Behavioural Semantics.}
%Labels remain the same. Rule $\ltsrule{App}$ in the untyped LTS
%(\figref{fig:untyped_LTS}) 
%is replaced with rule $\appl{(\abs{x}{P})}{V} \by{\tau} P \subst{V}{x}$.
%\defref{def:char} (characteristic processes) is extended with  
%${\mapchar{\shot{U}}{x}} \defeq\! \mapchar{\lhot{U}}{x} \defeq\! {\appl{x}{\omapchar{U}}}$ and 
%${\omapchar{\shot{U}}} \defeq {\omapchar{\lhot{U}}} \!\!\defeq\!\! \abs{x}{\mapchar{U}{x}}$. 
%We can then use the same definitions for $\cong$, $\wbc$, $\hwb$ and $\fwb$. 
%\smallskip 

\myparagraph{Encoding \HOpp into \HOp.} 
Let
	$\tyl{L}_{\HOpp}=\calc{\HOpp}{{\cal{T}}_4}{\hby{\ell}}{\wb_H}{\proves}$
where 
	${\cal{T}}_4$ is a set of types of $\HOpp$;  
the typing $\proves$ is defined in 
\figref{fig:typerulesmy} with extended rules \trule{Abs} and \trule{App}. 
We define 
the typed encoding
$\enco{\map{\cdot}^{3}, \mapt{\cdot}^{3}}: \HOpp \to \HOp$
in \figref{f:enc:hopip_to_hopi}.
By \propref{pro:composition}, 
we derive the following theorem. 

%\smallskip 

\begin{theorem}[Encoding \HOpp into~\HOp]
	\label{f:enc:hopiptohopi}
	The encoding from $\tyl{L}_{\HOpp}$ into $\tyl{L}_{\HOp}$ (cf. \figref{f:enc:hopip_to_hopi})
	is precise. Hence, the encodings 
	from $\tyl{L}_{\HOpp}$ to $\tyl{L}_{\HO}$ 
	and $\tyl{L}_{\sessp}$ 
	are also precise. 
\end{theorem}
%\smallskip 
\begin{figure}[t]
$
{%\small
\begin{array}{c}
	\multicolumn{1}{l}{\noindent{\bf Types:}}
	\\
	\tmap{\shot{L}}{3} \defeq \shot{\btinp{\tmap{L}{3}} \tinact}
	\qquad
	\tmap{\btout{\shot{L}} S}{3} \defeq \btout{\tmap{\shot{L}}{3}} \tmap{S}{3}
	\\
	\tmap{\btinp{\shot{L}} S}{3} \defeq \btinp{\tmap{\shot{L}}{3}} \tmap{S}{3}
	\\
%\hline
%\noindent{\bf Labels:} \  
%		\mapa{\news{\tilde{m}} \bactout{n}{\abs{\AT{x}{L}}{P}}}^{3} &\!\!\!\!\defeq\!\!\!\!& \news{\tilde{m}} \bactout{n}{\abs{z}{\binp{z}{x} \pmap{P}{3}}}
%		\\
%		\mapa{\bactinp{n}{\abs{\AT{x}{L}}{P}}}^{3} &\!\!\!\!\defeq\!\!\!\!& \bactinp{n}{\abs{z}{\binp{z}{x} \pmap{P}{3}}}
%\\
%\hline
	\multicolumn{1}{l}{\noindent{\bf Terms:}}
	\\
	\auxmap{x}{3} \defeq x
	\qquad
	\auxmap{\abs{x: L}{P}}{3} \defeq \abs{z}{\binp{z}{x} \pmap{P}{3}}
	\\
	\pmap{\appl{(x:L)}{V}}{3} \defeq \newsp{s}{\appl{x}{s} \Par \bout{\dual{s}}{\auxmap{V}{3}} \inact}
%	\\
%	 \defeq \newsp{s}{\appl{x}{s} \Par \bout{\dual{s}}{\abs{y} \pmap{P}{3}} \inact}
	\qquad
	\pmap{\bout{u}{\abs{x: L}{Q}} P}{3} \defeq \bout{u}{\auxmap{\abs{x}{Q}}{3} } \pmap{P}{3}
	\\
%	\pmap{\bout{u}{\abs{x: C}{Q}} P}{3} &\!\!\!\!\defeq\!\!\!\!& \bout{u}{\abs{x}{\pmap{Q}{3}}} \pmap{P}{3}
	\pmap{\appl{(\abs{x: L}{P})}{V}}{3} \defeq \newsp{s}{\binp{s}{x} \pmap{P}{3} \Par  \bout{\dual{s}}{\auxmap{V}{3}} \inact}
%	[[(?x P_1) (?x P_2)]] = (? s)((?x [[P]])s | \bout{s}{?x P_2} 0 )
\end{array}
}
$

$\tmap{\lhot{L}}{3}$ is defined as $\tmap{\shot{L}}{3}$
by replacing $\shot{L}$ with~$\lhot{L}$.
Label and term mappings for $\abs{x:C}{P}$ are
%defined 
as in \figref{f:enc:hopi_to_ho}, replacing 
%The mapping of types for $\lhot{L}$ is defined by replacing 
%$\shot{L}$ by $\lhot{L}$. 
%The case of $\abs{x:C}{P}$ in the label and term mappings 
%are %defined 
%as in \figref{f:enc:hopi_to_ho}, replacing 
$\tmap{\cdot}{1}$,
$\mapa{\cdot}^{1}$, and 
$\pmap{\cdot}{1}_f$, by  
$\tmap{\cdot}{3}$,
$\mapa{\cdot}^{3}$, and 
$\pmap{\cdot}{3}$.
The other mappings for processes, types and labels are  homomorphic.

\caption{\label{f:enc:hopip_to_hopi} Encoding of \HOpp into \HOp.}
%\Hlinefig
\end{figure} 


\subsection{Encoding Polyadic $\HOp$ into $\HOp$}
\label{subsec:pho}
%\noi Embeddings of polyadic name passing into monadic name passing are
%well-studied. % in the literature. 
%Using a linear typing, precise
%encodings (including full abstraction) can be obtained~\cite{Yoshida96}.
Here we summarise how $\PHOp$ can be encoded into $\HOp$. 
%The syntax of 
%$\HOp$ is extended %from \HOp by including 
%with
%polyadic name passing $\tilde{n}$ and $\abs{\tilde{x}}{Q}$ in the syntax 
%of value $V$. The type syntax is extended to: 
%
%\begin{center}
%\begin{tabular}{c}
%$
%L ::= \shot{\tilde{C}} \ | \ \lhot{\tilde{C}}
%\quad\quad S \ ::= \  \btout{\tilde{U}} S \bnfbar \btinp{\tilde{U}} S \bnfbar \cdots 
%$
%\end{tabular}
%\end{center}
%%
%The type system disallows a shared name that directly carries polyadic
%shared names as in \cite{tlca07,MostrousY15}.
%i.e. $\chtype{\tilde{\chtype{S}}}$ 
%and $\chtype{\tilde{\chtype{L}}}$ 
%are disallowed.
%Other definitions are straightforwardly extended. 
%\jpc{We slightly modify \defref{def:tenc} to capture that a 
%label $\ell$ may be mapped into a sequence of labels~$\tilde{\ell}$.}
%We extend the mapping for labels 
%($\mapa{\cdot}: \ell \to \tilde{\ell}$ in  
%\defref{def:tenc}) to capture 
%a sequence of labels  and 
%Also, \defref{def:ep} is kept unchanged, 
%assuming that if 
%$P \hby{\ell} P'$ and $\mapa{\ell} = \ell_1, \ell_2,  \cdots, \ell_m$ then
%$\map{P} \Hby{\mapa{\ell}} \map{P'}$
%should be understood as
%$\map{P} \Hby{\ell_1} P_1 \Hby{\ell_2} P_2 \cdots \Hby{\ell_m} P_m =  \map{P'}$,
%for some
%$P_1, P_2, \ldots, P_m$.

Let
	$\tyl{L}_{\PHOp}=\calc{\PHOp}{{\cal{T}}_5}{\hby{\ell}}{\wb_H}{\proves}$
where 
	${\cal{T}}_5$ is a set of types of $\HOpp$;  
the typing $\proves$ is defined in
\figref{fig:typerulesmy} with polyadic types. 
We define the typed encoding
	$\enco{\map{\cdot}^{4}, \mapt{\cdot}^{4}}: \PHOp \to \HOp$ 
in \figref{f:enc:poltomon}.
We give the dyadic case (tuples of length 2), for simplicity;
the general case is as expected.
Then we have:

\begin{theorem}[Encoding of \pHOp into \HOp]
	\label{f:enc:phopiptohopi}
	The encoding from
		$\tyl{L}_{\PHOp}$ into $\tyl{L}_{\HOp}$ (cf. \figref{f:enc:poltomon})
	is precise. 
	Hence, the encodings 
	from
	$\tyl{L}_{\PHOp}$ to $\tyl{L}_{\HO}$ 
	and $\tyl{L}_{\sessp}$ 
	are also precise. 
\end{theorem}

By combining Thms.~\ref{f:enc:hopiptohopi} and~\ref{f:enc:phopiptohopi},
we can easily extend the preciseness to 
$\PHOpp$, the super-calculus of $\HOpp$ and $\PHOp$.
% (denoted by   in Fig.~\ref{fig:express}) 


% !TEX root = ../journal16kpy.tex

\begin{figure}[t]
{\bf Terms:} 
\begin{align*}
	 \map{\bout{u}{u_1, u_2} P}^{4} &\defeq \bout{u}{u_1} \bout{u}{u_2} \map{P}^{4}
	\\
	 \map{\bbout{u}{\abs{x_1, x_2} Q} P}^{4} &\defeq \bbout{u}{\abs{z}\binp{z}{x_1} \binp{z}{x_2} \map{Q}^{4}} \map{P}^{4} %\qquad \text{($z$ fresh in $Q$)}
	\\
 \map{\appl{x}{(u_1, u_2)}}^{4} &\defeq \newsp{s}{\appl{x}{s} \Par \bout{\dual{s}}{u_1}   \bout{\dual{s}}{u_2} \inact}
	\\
	\map{\appl{(\abs{x_1,x_2}{P})}{(u_1, u_2)}}^{4} &\defeq
	\newsp{s}{\binp{s}{x_1}  \binp{s}{x_2} \pmap{P}{4} \Par \bout{\dual{s}}{u_1}  \bout{\dual{s}}{u_2} \inact} 
\end{align*}
{\bf Types:}
\begin{align*}
		\tmap{\btout{S_1, S_2}S}{4} &\defeq \btout{\tmap{S_1}{4}}  \btout{\tmap{S_2}{4}}\tmap{S}{4}
	\\
	 \tmap{\bbtout{L} S}{4} & \defeq  \bbtout{\mapt{L}^{4}}\mapt{S}^{4}
	\\
	  \tmap{\shot{(C_2,  C_2)}}{4} &\defeq \shot{\big(\btinp{\tmap{C_1}{4}} \btinp{\tmap{C_2}{4}}\tinact\big)}
	\\
	  \tmap{\lhot{(C_1,  C_2)}}{4} &\defeq \lhot{\big(\btinp{\tmap{C_1}{4}}  \btinp{\tmap{C_2}{4}}\tinact\big)}
\end{align*}
%We give the dyadic case; the general polyadic case is as expected.
The input cases are defined as the output cases by replacing $!$ by $?$. 
%Also, $\mapa{\tau}^4 =\tau, \tau$.
Elided mappings for  processes and types are 
homomorphic.
%\vspace{-2mm}
\caption{\label{f:enc:poltomon}Encoding of \PHOp (dyadic case) into \HOp. }
%\Hlinefig
%\vspace{-1mm} 
\end{figure}


