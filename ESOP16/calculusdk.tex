% !TEX root = main.tex
\section{Higher-Order Session $\pi$-Calculi}
\label{sec:calculus}

We introduce 
the \emph{higher-order session $\pi$-calculus} (\HOp).
We define 
syntax, operational semantics, and 
its sub-calculi (\sessp and \HO).
A type system and behavioural equivalences are introduced in 
\secref{sec:types} and \secref{sec:bt}. 
Extensions of \HOp (\HOpp and \PHOp) are discussed in \secref{sec:extension}.


%We also introduce two subcalculi of \HOp. In particular, we define the 
%core higher-order session
%calculus (\HO), which 
%%. The \HO calculus is  minimal: it 
%includes constructs for shared name synchronisation and 
%%constructs for session establish\-ment/communication and 
%(monadic) name-abstraction, but lacks name-passing and recursion.

%Although minimal, in \secref{s:expr}
%the abstraction-passing capabilities of \HOp will prove 
%expressive enough to capture key features of session communication, 
%such as delegation and recursion.

\subsection{\HOp: Syntax, Operational Semantics, Subcalculi}
\label{subsec:syntax}

\jparagraph{Syntax}
The syntax of \HOp is defined in \figref{fig:syntax}.
\HOp it is a subcalculus of the language studied 
in~\cite{tlca07}. It is also a variant of the language that we investigated in~\cite{characteristic_bis}, 
where higher-order value applications were considered. 


% !TEX root = ../journal16kpy.tex
	\begin{figure}[t]
		\begin{align*}
			n  & \bnfis a,b \bnfbar s, \dual{s} 
			\\
			u,w  & \bnfis n \bnfbar x,y,z 
			\\
			V,W  & \bnfis \nonhosyntax{u} \bnfbar \nonpisyntax{\abs{x}{P}}
			\\[1mm]
			P,Q
			 & \bnfis 
			\bout{u}{V}{P}  \sbnfbar  \binp{u}{x}{P} \sbnfbar
			\bsel{u}{l} P \sbnfbar \bbra{u}{l_i:P_i}_{i \in I} \sbnfbar \nonpisyntax{\appl{V}{u}}
			\\[1mm]
			 & \sbnfbar P\Par Q \sbnfbar \news{n} P 
			\sbnfbar \inact \sbnfbar \nonhosyntax{\rvar{X} \sbnfbar \recp{X}{P}}
%			\\%[1mm]
%			 \bnfbar &
%			\nonhosyntax{\rvar{X} \bnfbar \recp{X}{P}}
		\end{align*}
	\caption{\newjb{Syntax of \HOp. While \HO lacks \nonhosyntax{\text{shaded}} constructs, \sessp lacks \nonpisyntax{\text{boxed}} constructs.}}
	\label{fig:syntax}
%	\vspace{-1mm}
%	\Hlinefig
\end{figure}



%\myparagraph{Values}
\emph{Names} $a,b,c, \dots$ (resp.~$s, \dual{s}, \dots$) 
range over shared (resp. session) names. 
Names $m, n, t, \dots$ are session or shared names.
Dual endpoints are $\dual{n}$ with
$\dual{\dual{s}} = s$ and $\dual{a} = a$.
%We define the dual operation over names $n$ as $\dual{n}$ with
%$\dual{\dual{s}} = s$ and $\dual{a} = a$.
%Intuitively, names $s$ and $\dual{s}$ are dual (two) \emph{endpoints} while 
%shared names represent shared (non-deterministic) points. 
Variables are denoted with $x, y, z, \dots$, 
and recursive variables are denoted with $\varp{X}, \varp{Y} \dots$.
An abstraction %(or higher-order value) 
$\abs{x}{P}$ is a process $P$ with name parameter $x$.
%Symbols $u, v, \dots$ range over identifiers; and  $V, W, \dots$ to denote values. 
\emph{Values} $V,W$ include 
identifiers $u, v, \ldots$ %(first-order values) 
and 
abstractions $\abs{x}{P}$ (first- and higher-order values, resp.). 

%\myparagraph{Terms} 

Terms
include $\pi$-calculus prefixes for sending and receiving values $V$.
%Process $\bout{u}{V} P$ denotes the output of value $V$
%over name $u$, with continuation $P$;
%process $\binp{u}{x} P$ denotes the input prefix on name $u$ of a value
%that 
%will substitute variable $x$ in continuation $P$. 
Recursion   $\recp{X}{P}$ binds the recursive variable $\varp{X}$ in process $P$.
Process 
%ny
%$\appl{x}{u}$ 
$\appl{V}{u}$ 
is the application
which substitutes name $u$ on the abstraction~$V$. 
Typing  ensures that $V$ is not a name.
Processes $\bsel{u}{l} P$ and $\bbra{u}{l_i: P_i}_{i \in I}$ are the
standard session processes for selecting and branching.
%Prefix $\bsel{u}{l} P$ selects label $l$ on name $u$ and then behaves as $P$.
%Given $i \in I$ 
%Process $\bbra{u}{l_i: P_i}_{i \in I}$ offers a choice on labels $l_i$ with
%continuation $P_i$, given that $i \in I$.
%Others are standard c
Constructs for 
inaction $\inact$,  parallel composition $P_1 \Par P_2$, and 
name restriction $\news{n} P$ are standard.
Session name restriction $\news{s} P$ simultaneously binds endpoints $s$ and $\dual{s}$ in $P$.
%A well-formed process relies on assumptions for
%guarded recursive processes.
Functions $\fv{P}$ and $\fn{P}$ denote the sets of free 
%\jpc{recursion}
variables and names.
We assume $V$ in $\bout{u}{V}{P}$ does not include free recursive 
variables $\rvar{X}$.
If $\fv{P} = \emptyset$, we call $P$ {\em closed}.
%; and closed $P$ without 
%free session names a {\em program}. 




%\subsection{Operational Semantics}
%\label{subsec:semantics}


\jparagraph{Operational Semantics}
The \emph{operational semantics} of \HOp is defined in terms of a reduction relation, 
denoted $\red$ and 
given in 
 \figref{fig:reduction} (top).
 We briefly explain the rules. 
Rule $\orule{App}$ defines  name application.
Rule $\orule{Pass}$ defines a shared interaction at $n$ 
(with $\dual{n}=n$) or a session interaction.
Rule $\orule{Sel}$ is the standard rule for labelled choice/selection.%:
%given an index set $I$, 
%a process selects label $l_j$ on name $n$ over a set of
%labels $\set{l_i}_{i \in I}$ offered by a branching 
%on the dual endpoint $\dual{n}$;
Other rules are standard $\pi$-calculus rules.
Reduction is closed under \emph{structural congruence} as defined in \figref{fig:reduction} (bottom). 
We assume the expected extension of $\scong$ to values $V$.
We write $\red^\ast$ for a multi-step reduction.

\begin{figure}[!t]
\[
	\begin{array}{rcllcrcll}
%	\begin{array}{c}
		\appl{(\abs{x}{P})}{u}  & \red & P \subst{u}{x} 
		& \orule{App}
%		\\[1mm]
		&&
		\bout{n}{V} P \Par \binp{\dual{n}}{x} Q & \red & P \Par Q \subst{V}{x} 
		& \orule{Pass}
		\\[1mm]

		\bsel{n}{l_j} Q \Par \bbra{\dual{n}}{l_i : P_i}_{i \in I} & \red & Q \Par P_j ~~(j \in I)~~ 
		& \orule{Sel}
%		\\[1mm]
		&&
		P \red P'\Rightarrow  \news{n} P   & \red  &  \news{n} P' 
		& \orule{Res}
		\\[1mm]

		P \red P' & \Rightarrow  &  P \Par Q  \red   P' \Par Q  
		& \orule{Par}
%		\\[1mm]
		&&
		P \scong Q \red Q' \scong P' & \Rightarrow & P  \red  P'
		& \orule{Cong}
	\end{array}
\]
{\small
\[
	\begin{array}{c}
		P \Par \inact \scong P
		\quad
		P_1 \Par P_2 \scong P_2 \Par P_1
		\quad
		P_1 \Par (P_2 \Par P_3) \scong (P_1 \Par P_2) \Par P_3
%		\\[1mm]
		\quad
		\news{n} \inact \scong \inact
%		\quad 
		\\[1mm]
		P \Par \news{n} Q \scong \news{n}(P \Par Q)
		\ (n \notin \fn{P})
		\quad 
		\recp{X}{P} \scong P\subst{\recp{X}{P}}{\rvar{X}}
%		\\[1mm]
		\quad
		P \scong Q \textrm{ if } P \scong_\alpha Q
%		\qquad
%		\dk{V \scong W \textrm{ if } V \scong_\alpha W
%\quad \abs{x}{P} \scong \abs{x}{Q} \textrm{ if } P \scong Q}
	\end{array}
\]
}
%\vspace{-3mm}
\caption{Operational Semantics of $\HOp$. 
%\vspace{-1mm}
\label{fig:reduction}}
%\Hlinefig
\end{figure}



\jparagraph{Subcalculi}
%\label{subsec:subcalculi}
\noi As motivated in the introduction, 
we define two \emph{subcalculi} of \HOp. 
%We now define several sub-calculi of \HOp. 
\begin{enumerate}[$\bullet$]
	\item	The  
		{\em core higher-order session calculus} (denoted \HO),
		 lacks recursion and name passing; its 
		formal syntax is obtained from \figref{fig:syntax} by excluding 
		constructs in \nonhosyntax{\text{grey}}.

	\item	The   
		 {\em session $\pi$-calculus} 
		(denoted $\sessp$), which 
		lacks  
		higher-order constructs
		(i.e., abstraction passing and application), but includes recursion.

%	\item	The third sub-calculus, denoted \haskp, represents cloud Haskell:
%		\[
%			\begin{array}{rclllll}
%				V,W	& ::= &		u \bnfbar  \abs{x}{P}
%				\\
%				P,Q	& ::= &		\bout{u}{m}{P}  \bnfbar  \binp{u}{x}{P} \bnfbar
%							\bsel{u}{l} P \bnfbar \bbra{u}{l_i:P_i}_{i \in I}
%				\\[1mm]
%					& \bnfbar &	\appl{V}{V} \bnfbar P\Par Q \bnfbar \news{n} P \bnfbar \inact
%		\end{array}
%		\]
\end{enumerate}
%
Let $\CAL \in \{\HOp, \HO, \sessp\}$. We write 
$\CAL^{-\mathsf{sh}}$ for $\CAL$ without shared names
(we delete $a,b$ from $n$). 
We shall demonstrate in \secref{sec:positive} that 
$\HOp$, $\HO$, and $\sessp$ have the same expressivity.
